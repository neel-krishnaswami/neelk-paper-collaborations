\documentclass{article}
\usepackage{amsmath}
\usepackage{amssymb}
\usepackage{mathpartir}

\newcommand{\powerset}[1]{\mathcal{P}(#1)}
\newcommand{\setof}[1]{\left\{{#1}\right\}}
\newcommand{\comprehend}[2]{\setof{{#1}\;|\;{#2}}}
\newcommand{\interp}[1]{[\![{#1}]\!]}
\renewcommand{\implies}{\Rightarrow}
\newcommand{\wand}{-\!\!\!*}
\newcommand{\smallprop}[1]{\mathsf{S}_s({#1})}
\newcommand{\pointsto}{\mapsto}
\title{Deforming Separation Logic}
\author{Neel Krishnaswami}

\begin{document}
\maketitle

\section{Grothendieck Topological Monoids}

What follows is basically a copy of what Pym, O'Hearn and Yang did in their ``Possible Worlds
and Resources: The Semantics of BI'' paper. They basically combine topological models of 
logic with monoidal resource semantics by viewing the monoid as a category and equipping it 
with a Grothendieck topology. To do this, they equip the monoid with preorder structure
(so that it can be viewed as a skeletal category) and then give that category a Grothendieck
topology. 

The idea behind a Grothendieck topology is to take the notion of open
cover from topology and formally apply it to settings where we don't
necessarily have all the union/intersection operations available in a
real topology.

So, given a preordered monoid $M = (M, \circ, e, \sqsubseteq)$, a
Grothendieck topology $J$ is an assignment $M \to
\powerset{\powerset{M}}$. That is, for each element $m$ of $M$, $J$ gives us 
a set of formal open covers for $m$. So if $S \in J(m)$, the idea is that
the elements of $S$ ``cover'' $m$. Naturally, not every function $J$ is
a Grothendieck topology -- there are some axioms that need to be satisfied. 
(Caveat: we're already cheating, since a real Grothendieck topology assigns
a family of sets of incoming morphisms to each object. But since a preorder
is a degenerate category it's ok to just assign a family of sets of objects.)

I'm just going to take the properties given in POY paper. This is also a bit
confusing, since presheaves are functors $\mathbb{C}^{op} \to \mathrm{Set}$, and
I think they flipped the directionality of all the arrows in their axioms, relative 
to the standard ones for Grothendieck topologies. It should all be equivalent,
but it's also a bit hairy. For now I'll just follow their development, but it
might make sense to restate them more conventionally. 

\begin{enumerate}
\item Sieve: For all $m \in M, S \in J(m)$, we have that for all $m' \in S, m \sqsubseteq m'$. (That is,
``$m \sqsubseteq S$''.) 

\item Maximality: For any $m, m' \in M$, if $m \sqsubseteq m'$ and $m' \sqsubseteq m$, then $\setof{m} \in J(m')$. 

\item Stability: For any $m, n \in M$ such that $m \sqsubseteq n$, and for all $S \in J(m)$, there
  exists $S' \in J(n)$ such that for any $n' \in S'$, there exists $m' \in S$ such that $m' \sqsubseteq n'$. (That is
  ``$S \sqsubseteq S'$''.)

\item Transitivity: For any $m \in M$ and $S \in J(m)$, we have 
\begin{mathpar}
  \left(\bigcup\limits_{m' \in S} \bigcup\limits_{S' \in J(m')} S'\right)$$\in J(m)
\end{mathpar}

\item Continuity: For any $m,n \in M$ and $S \in J(m)$, we have that 
\begin{mathpar}
\comprehend{ m' \circ n}{m' \in S}$$\in J(m \circ n)
\end{mathpar}
(So ``$S \circ n \in J(m \circ n)$'')
\end{enumerate}

\begin{figure}
\begin{center}

\begin{mathpar}
  \begin{array}{lcl}
    m \models p 
    & \iff &  m \in \interp{p} \\

    m \models \top 
    & \iff & \mbox{always} \\

    m \models \phi \land \psi 
    & \iff & m \models \phi \mbox{ and } m \models \psi \\

    m \models \phi \implies \psi 
    & \iff & \forall n \sqsubseteq m.\; \mbox{if } n \models \phi \mbox{ then } n \models \psi \\

    m \models \bot 
    & \iff & \emptyset \in J(m) \\

    m \models \phi \vee \psi 
    & \iff & \exists S \in J(m).\; \forall m' \in S.\; m' \models \phi \mbox{ or } m' \models \psi \\

    m \models I 
    & \iff & \exists S \in J(m).\; \forall m' \in S.\; m' \sqsubseteq e \\

    m \models \phi \star \psi 
    & \iff & \exists S \in J(m).\; \forall m' \in S.\; \exists n_\phi, n_\psi.\\
    &      & \qquad m' \sqsubseteq n_\phi \circ n_\psi \mbox{ and } n_\phi \models \phi \mbox{ and } n_\psi \models \psi \\

    m \models \phi \wand \psi
    & \iff & \forall n \in M.\; \mbox{if } n \models \phi \mbox{ then } n \circ m \models \psi \\
  \end{array}
\end{mathpar}
\end{center}
\label{sheaf-semantics}
\caption{Sheaf Semantics}
\end{figure}

So, if we have a preordered monoid with a Grothendieck topology, we
can give a forcing relation for separation logic, which is described
in Figure~\ref{sheaf-semantics}. Observe that we use the covering
relation in the cases of disjunction and the separating conjunction. 

Now, this semantics satisfies two properties (assuming the atomic formulas do): 

\begin{enumerate}
\item Kripke Monotonicity:  For any $n, m \in M$ such that $n \sqsubseteq m$, if $m \models \phi$ then $n \models \phi$

\item Sheaf Condition: For any $m \in M$ and $S \in J(m)$, if for all $m' \in S$ we have $m' \models \phi$, then $m \models \phi$. 
\end{enumerate}

\section{From Partial to Total Monoids}

One thing that we've done so far is to assume that the monoids in
question are total. One nice thing about the Grothendieck semantics is
that it gives a simple way of completing any partial monoid into a
total one, without changing the resulting separation logic continues
working the way we expect.

So if $(M', \circ', e')$ is a partial commutative monoid, we can define a
partially ordered total commutative monoid as follows: 

\begin{enumerate}
\item Let $M = M' \uplus \setof{0}$. (That is, adjoin a new element to $M'$, disjoint from
  the existing elements.)

\item Define $m_1 \circ m_2$ as follows:
  \begin{mathpar}
    \begin{array}{lcll}
      m'_1 \circ m'_2 & = & m'_1 \circ' m'_2 & \mbox{ when defined } \\
      m'_1 \circ m'_2 & = & \bot            & \mbox{ when undefined } \\
      m \circ \bot & = & \bot & \\
      \bot \circ m & = & \bot & \\
    \end{array}
  \end{mathpar}

\item Let $e = e'$. 

\item Equip $M$ with the flat order where $\bot \sqsubseteq m$, and otherwise $m'_1 \sqsubseteq m'_2$
  if and only if $m'_1 = m'_2$. 
\end{enumerate}

Now, we can equip this preordered monoid with a Grothendieck topology as follows:

\begin{mathpar}
  \begin{array}{lcll}
    J(0)  & = & \setof{\setof{0}, \emptyset} & \\
    J(m)  & = & \setof{\setof{m}} & \mbox{when } m \not= 0 \\
  \end{array}
\end{mathpar}

POY show that propositional entailment coincides between this semantics for the partial
monoid of heaps, and the standard semantics of separation logic. 

\section{Toward Finite Memory}

The reason I'm interested in this model of BI is that it seems to give a systematic way 
to talk about enriching separation logic along several dimensions. For example, we can
vary the monoid we use, vary the preorder to require respecting some invariants, and
perhaps most importantly we can use relations between monoids to explain the new modal 
operators that creep into every separation logic paper. 

As a simple example, let's consider how separation logic for finite memory might work. 

So, as our first step, let's define a family of partial commutative
monoids $H'_n$. Each $H'_n$ is the restriction of the usual heap
monoid to heaps of size at most $n$, with the partial commutative
monoid operation being undefined if the join of two heaps is bigger
than $n$, in addition to being undefined if the two heaps overlap. 
Observe the existence of functions $s : H'_n \to H'_m$, the
inclusion function from $H'_n$ to $H'_m$, which exist when $n < m$.

Now, note that as in the previous section, we can add formal zero 
elements to get a total commutative monoid, as well as using it to 
get a Grothendieck topology for each of these monoids. 

This gives us a family of separation logics, one for each memory
size. The question is, can we integrate these logics into one big
one? Furthermore, can we do it \emph{without} building a crazy super
model?

So, for simplicity let's consider $H_n$ and $H_m$ for fixed $n < m$,
and let $s$ be some chosen embedding. Now, in $H_m$, we can add a
modal operator $\smallprop{\phi}$, whose intuitive meaning is ``I am
modelled by those smaller heaps that are in the image of $H_n$''. 

\begin{mathpar}
  \begin{array}{lcl}
    m \models \smallprop{\phi} & \iff & \forall n.\; \mbox{if } s^{-1}(m) \in J(n) \mbox{ then } n \models \phi\\
  \end{array}
\end{mathpar}

I think (i.e., I haven't yet completely checked) that this satisfies
the Kripke monotonicity and sheaf conditions, and so this is a fine 
connective of the big logic. 

The nice thing is that this lets us state a lot of properties about sizeful
computations in an ``internal'' way. We can show that $\smallprop{\bot} \implies \bot$, and
then we can reason under a modal operator using size-specific logical principles. For example, 
if $n = 1$, then we know that $x \pointsto 3 \star y \pointsto 7$ is false, since we only
have one word of memory. So this means that $\smallprop{x \pointsto 3 \star y \pointsto 7}
\implies \smallprop{\bot} \implies \bot$. 



\end{document}
