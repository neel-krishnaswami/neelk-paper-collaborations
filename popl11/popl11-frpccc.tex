%-----------------------------------------------------------------------------
%
%               Template for sigplanconf LaTeX Class
%
% Name:         sigplanconf-template.tex
%
% Purpose:      A template for sigplanconf.cls, which is a LaTeX 2e class
%               file for SIGPLAN conference proceedings.
%
% Author:       Paul C. Anagnostopoulos
%               Windfall Software
%               978 371-2316
%               paul@windfall.com
%
% Created:      15 February 2005
%
%-----------------------------------------------------------------------------


\documentclass[nocopyrightspace,preprint]{sigplanconf}

% The following \documentclass options may be useful:
%
% 10pt          To set in 10-point type instead of 9-point.
% 11pt          To set in 11-point type instead of 9-point.
% authoryear    To obtain author/year citation style instead of numeric.

\usepackage{amsmath}
\usepackage{amssymb}
\usepackage{mathpartir}
\usepackage{verbatim}

\newcommand{\ultrametric}{\mathbb{U}}
\newcommand{\R}{\mathbb{R}}
\newcommand{\N}{\mathbb{N}}
\newcommand{\setof}[1]{\left\{{#1}\right\}}
\newcommand{\comprehend}[2]{\setof{{#1}\;|\;{#2}}}
\newcommand{\betterstate}[3]{{#2}\, {\sqsupseteq}^{#1} {#3}}
\newcommand{\futurestate}[4]{{#3} \gg^{#1}_{#2} {#4}}
\newcommand{\worsestate}[5]{{#4} {\sqsubseteq}^{(#1,#2)}_{#3} {#5}}
% \newcommand{\satisfies}[4]{({#1}, {#2}, {#3}) \;\mathsf{sat}\; {#4}}
\newcommand{\dom}[1]{\mathrm{dom}({#1})}

\newcommand{\ready}[3]{\mathsf{ready}({#1}, {#2}, {#3})}
\newcommand{\unready}[2]{\mathsf{unready}({#1}, {#2})}
\newcommand{\cells}[1]{\mathrm{cells}({#1})}
\newcommand{\refs}[1]{\mathrm{refs}({#1})}
\newcommand{\hasref}[3]{\mathsf{hasref}({#1}, {#2}, {#3})}
\renewcommand{\implies}{\Rightarrow}

\newcommand{\newref}[1]{\term{new}({#1})}

\newcommand{\stream}[1]{\mathbf{#1}}
\newcommand{\term}[1]{\ensuremath{\mathtt{{#1}}}}
\newcommand{\spec}[4]{\setof{{#1}}{#2}\setof{{#3}.\;{#4}}}

\newcommand{\streams}{\mathit{streams}}
\newcommand{\thunks}{\mathit{thunks}}
\newcommand{\lags}{\mathit{lags}}
\newcommand{\locs}{\mathit{locs}}

\newcommand{\discrete}[1]{D({#1})}
\newcommand{\To}{\Rightarrow}
\newcommand{\shrink}{\rightsquigarrow}
\newcommand{\interp}[1]{[\![{#1}]\!]}
\newcommand{\unitval}{\left<\right>}

\newcommand{\floor}[1]{\left\lfloor{#1}\right\rfloor}

\newcommand{\interps}[1]{(\!|{#1}|\!)_s}
\newcommand{\interpu}[1]{(\!|{#1}|\!)_u}

\newcommand{\Head}{\mathit{Head}}
\newcommand{\Tail}{\mathit{Tail}}
\newcommand{\Code}{\mathit{Code}}
\newcommand{\Build}{\mathit{Build}}
\newcommand{\Local}{\mathit{Local}}
\newcommand{\Ref}{\mathit{Ref}}
\newcommand{\Ready}{\mathit{Ready}}
\newcommand{\Opt}{\mathit{Opt}}
\newcommand{\Stream}{\mathit{Stream}}
\newcommand{\NewStream}{\mathit{NewStream}}
\newcommand{\Mem}[1]{\mathit{Mem}(#1)}
\newcommand{\Seq}{\mathit{Seq}}
\newcommand{\Update}{\mathit{Update}}
\newcommand{\StableRef}{\mathit{StableRefs}}
\newcommand{\StableImp}{\mathit{StableImps}}
\newcommand{\Stable}{\mathit{Stable}}
\newcommand{\realize}[5]{\mathit{Realize}_{#1}^{#2 \to #3}({#4}, {#5})}
\newcommand{\ultrahom}[4]{\realize{\ultrametric}{#1}{#2}{#3}{\term{#4}}}
\newcommand{\cokleislihom}[4]{\realize{\ultrametric^S}{#1}{#2}{#3}{\term{#4}}}

\newcommand{\Start}[1]{\mathsf{start}(#1)}

\newcommand{\unittype}{\mathsf{unit}}
\newcommand{\celltype}[1]{\mathsf{cell}\;{#1}}
\newcommand{\opttype}[1]{\mathsf{option}\;{#1}}
\newcommand{\reftype}[1]{\mathsf{ref}\;{#1}}
\newcommand{\monad}[1]{\bigcirc{#1}}
\newcommand{\suspend}[1]{[{#1}]}
\newcommand{\clock}{\mathsf{clock}}
\newcommand{\comp}[1]{\mathsf{code}\;{#1}}
\newcommand{\thunk}[1]{\mathsf{thunk}\;{#1}}
\newcommand{\streamtype}[1]{\mathsf{stream}\;{#1}}
\newcommand{\contracttype}{\mathsf{bool}}
\newcommand{\lolli}{\multimap}
\newcommand{\lollishrink}{-\!\!\!\,\bullet}
\newcommand{\valtype}[1]{\mathsf{val}\;{#1}}
\newcommand{\None}{\mathsf{None}}
\newcommand{\Some}[1]{\mathsf{Some}({#1})}
\newcommand{\stateok}[2]{\mathsf{updateok}({#1}, {#2})}

\newcommand{\counit}{\epsilon}
\newcommand{\tails}{\delta}

\newcommand{\powerset}[1]{\mathcal{P}(#1)}
\newcommand{\cellminus}[2]{\mathsf{cell}^{-}({#1}, {#2})}
\newcommand{\cellplus}[4]{\mathsf{cell}^{+}({#1}, {#2}, {#3}, {#4})}

\newenvironment{proof}[1][(Sketch)]{\noindent \textsc{Proof {#1}} }{}

\newcommand{\judgeu}[3]{{#1} \vdash {#2} : {#3}}
\newcommand{\judgek}[4][\Gamma]{{#1};{#2} \vdash {#3} : {#4}}
\newcommand{\judgec}[4][\Gamma]{{#1};{#2} \vdash {#3} : {#4}}
\newcommand{\judgekc}[4][\Delta]{\Gamma;{#1};{#2} \vdash {#3} : {#4}}


\newcommand{\letval}[3]{\mathsf{letval}\;{#1} = {#2}\;\mathsf{in}\;{#3}}
% Contractive commands
\newcommand{\const}[1]{\left<{#1}\right>}
\newcommand{\pair}[2]{({#1}, {#2})}
\newcommand{\fst}[1]{\pi_1{#1}}
\newcommand{\snd}[1]{\pi_2{#1}}
\newcommand{\unit}{()}
\newcommand{\letc}[3]{\mathsf{letc}\;{#1} = {#2}\;\mathsf{in}\;{#3}}
\newcommand{\fun}[2]{\lambda {#1}.\;{#2}}
\newcommand{\sfun}[2]{\hat{\lambda} {#1}.\;{#2}}

\newcommand{\eval}{\mathit{eval}}
\newcommand{\swap}{\mathit{swap}}
\newcommand{\sweak}{\mathit{sweak}}
\newcommand{\seval}{\mathit{seval}}
\newcommand{\scomposer}{\mathit{scomposer}}
\newcommand{\scomposel}{\mathit{scomposel}}
\newcommand{\spair}{\mathit{spair}}
\newcommand{\scurry}{\mathit{scurry}}

\newcommand{\letv}[3]{\term{letv}\;{#1} = {#2}\;\term{in}\;{#3}}
\newcommand{\fix}[2]{\term{fix}\;{#1}.\;{#2}}

\newcommand{\Delays}{\mathbb{D}}
\newcommand{\U}{\mathsf{u}}
\newcommand{\D}{\mathsf{d}}

%
\newcommand{\bnfalt}{\;\;|\;\;}
\newcommand{\pointsto}{\mapsto}
\newcommand{\emp}{\mathsf{emp}}
\newcommand{\specimp}{\Longrightarrow}
\newcommand{\specand}{\;\mathrm{and}\;}
\newcommand{\specor}{\;\mathrm{or}\;}


\newcommand{\localref}[2]{\mathsf{ref}\term{(#1, #2)}}

%
\newcommand{\fixme}[1]{\texttt{FIXME: {#1}}}
\newcommand{\head}[1]{\mathit{head}(#1)}
\newcommand{\tail}[2][]{\mathit{tail}^{#1}(#2)}
\newcommand{\ramify}[1]{\mathsf{U}(\mathtt{clock}, {#1})}

\newcommand{\einvariant}[3]{{#2} \stackrel{#1}{=} {#3}}
\newcommand{\satisfy}[2]{{#1}\;\mathrm{sat}\;{#2}}
\newcommand{\satisfyext}[2]{\mathrm{extsat}({#1}, {#2})}
\newcommand{\complete}[1]{\mathrm{complete}(#1)}

\newcommand{\assert}{\mathsf{prop}}

\newcounter{lineno}
\newenvironment{tabbingspec}{\setcounter{lineno}{0}\begin{tabbing}\refstepcounter{lineno}{\small \arabic{lineno}}\;\;\;\=}{\end{tabbing}}
\newcommand{\Newline}[1][0em]{\refstepcounter{lineno}\\[#1]{\small \arabic{lineno}}\>}


\newtheorem{prop}{Proposition}
\newtheorem{lemma}{Lemma}
\newtheorem{definition}{Definition}
\newtheorem{theorem}{Theorem}



\begin{document}

\conferenceinfo{POPL '11}{date, City.} 
\copyrightyear{2011} 
\copyrightdata{[to be supplied]} 

\titlebanner{unpublished -- do not circulate}        % These are ignored unless
\preprintfooter{An Ultrametric Model of Reactive Programming}   % 'preprint' option specified.

\title{An Ultrametric Model of Reactive Programming}
%\subtitle{Subtitle Text, if any}

\authorinfo{Neelakantan R. Krishnaswami}
           {Microsoft Research}
           {neelk@microsoft.com}
\authorinfo{Nick Benton}
           {Microsoft Research}
           {nick@microsoft.com}

\maketitle

\begin{abstract}
We describe a denotational model of higher-order functional reactive
programming using ultrametric spaces, which provide a natural
Cartesian closed generalization of causal stream functions. We define
a domain-specific language corresponding to the model. We then show
how reactive programs written in this language may be implemented
efficiently using an imperatively updated dataflow graph and give a
higher-order separation logic proof that this low-level implementation
is correct with respect to the high-level semantics.
\end{abstract}

%% \category{CR-number}{subcategory}{third-level}

%% \terms
%% term1, term2

%% \keywords
%% keyword1, keyword2

\section{Introduction}
%% In many application domains, the designers and implementers of
%% DSLs and frameworks have to strike a difficult balance between
%% providing highly expressive, uniform abstractions of the sort
%% provided by general-purpose languages, and restricting users to a
%% domain-specific programming model that can be implemented
%% particularly efficiently, or with extra guarantees, or that is
%% amenable to specific analyses or transformations. Relational
%% databases and parser generators provide obvious examples: naive
%% models of tables and queries, and rich libraries of parser
%% combinators, are easy to build in a modern general-purpose
%% language and naturally integrate well with the rest of the
%% language. But the performance (and other properties) of such
%% implementations precludes their use in many real
%% applications. Instead, programs are typically factored into
%% components written in a general purpose language and components
%% that are expressed in a restricted language, such as SQL or some
%% form of BNF, that is implemented quite differently. Much work
%% then goes into addressing the `impedence mismatch' between the
%% different models and into gradually enriching the domain-specific
%% model with more of the features of general-purpose
%% languages. Within general-purpose languages, well-optimized and
%% staged libraries with APIs that are carefully restricted to a
%% well-understood domain-specific model can perform orders of
%% magnitude better than straightforward implementations
%% \cite{vieraetal:hdyrm,henglein:optrelational}.

There is a broad spectrum of models for reactive programming.
Functional reactive programming (FRP), as introduced by Elliott and
Hudak \cite{fran}, is highly expressive and generally shallowly
embedded in powerful general-purpose languages.  At the other end,
synchronous languages such as Esterel \cite{esterel}, Lustre
\cite{lustre} and Lucid Synchrone \cite{synchrone} provide a
restricted, domain-specific model of computation supporting
specialized compilation strategies and analysis
techniques. Synchronous languages have been extremely successful in
application areas such as hardware synthesis and embedded control
software, and provide strong guarantees about bounded usage of space
and time. FRP was initially aimed at dynamic interactive applications
running is less resource-constrained environments, such as desktop
GUIs, games and web applications. Even in such environments, however,
naive versions of FRP are too unconstrained to be implemented
efficiently (the implementations are far from naive, but it is still
all too easy to introduce significant space and time leaks), but also
too unconstrained from the point of view of the programmer, allowing
unimplementable programs (e.g. ones that violate causality) to be
written. More recent variants of FRP \cite{liu:cca,sculthorpe:safefrp}
restrict the model to rule out non-causal functions and ill-formed
feedback.

In practice, of course, interactive GUIs and the like are usually
implemented in general-purpose languages in a very imperative style. A
program implements dynamic behavior by modifying state, and accepting
callbacks to modify its own state. These programs exhibit complex uses
of aliasing, tricky control flow through callback functions living in
the heap, and in general are very difficult to reason about. Part of
the difficulty is the inherent complexity of verifying programs using
such powerful features, but an even more fundamental problem is that
it is not immediately clear even what the semantics of such programs
should be --- and even the most powerful verification techniques are
useless without a specification for a program to meet.

The starting point for the work described here is the
\emph{synchronous dataflow} paradigm of, for example, Lustre
\cite{lustre} and Lucid Synchrone \cite{synchrone}. We wish to be able
to write complex dynamic reactive applications in a high-level
declarative style, without abandoning the efficient stateful execution
model that those languages provide, at least for the first-order parts
of our programs. To this end, we first present a new semantic model
for reactive programs in terms of ultrametric spaces, which
generalizes previous models based on causal stream functions. Our
model supports full Cartesian closed structure, thus giving a
natural mathematical notion of higher-order reactive programs.

%% We give a new semantic model of functional reactive programming
%%   based on ultrametric spaces, which builds upon prior work in two
%%   ways. First, it generalizes existing models by supporting full
%%   cartesian closed structure, rather than more restrictive structures
%%   such as arrows, while still respecting important concepts such as
%%   the causality of stream functions (and in fact extending them
%%   naturally to higher-order).

  The use of metric spaces means that we can use
  Banach's contraction map theorem to interpret feedback. Unlike earlier semantics based on domain models of streams, we
  can thus restrict our semantics to \emph{total, well-founded}
  stream programs. Furthermore, by using an abstract notion of
  contractiveness instead of an explicit notion of guardedness, our
  semantics lifts easily to model higher-type streams (e.g., streams
  of streams) and recursion at higher type. 

Next, we give a domain specific language for writing reactive
  programs. Since streams distribute over products and form a comonad,
  the co-Kleisli category of streams is also Cartesian closed, thus
  giving us \emph{two} natural notions of function for reactive
  programs.  Prior work~\cite{coiterative, essence-dataflow} has
  focused primarily on the co-Kleisli category, but the interpretation of fixed
  points is significantly more natural in the base category of ultrametric spaces. By adapting the adjoint calculus presentation of intuitionistic linear logic \cite{benton:lnl,benton-wadler}, our language allows one to work in the two categories simultaneously.
The idea is to decompose the stream comonad into a pair of
  adjoint functors, which in the term calculus become modal operators
  connecting the two lambda calculi. We can then interpret fixed
  points in the category of ultrametrics (thereby retaining a simple
  equational theory for them) while still enabling programming
  implicitly with streams, as is common in dataflow languages. Furthermore, we extend the adjoint
  calculus with additional judgements to track contractiveness, so
  that we can use typechecking to ensure that clients can only take
  fixed points of well-defined, strictly-contractive functions.

In the second part of the paper, we give a reasonably efficient
implementation of our language in terms of an imperative dataflow
graph and prove the correctness of the implementation with respect to
the semantics. The correctness proof uses a rather non-trivial Kripke
logical relation, built using ideas from separation logic,
rely-guarantee reasoning and step-indexed models, but ensures that
clients can reason about programs as well-behaved mathematical
objects, satisfying the full range of $\beta$, $\eta$ and fixpoint
equations, with all the complexities of the higher-order imperative
implementation hidden behind an abstraction barrier.

  %% Furthermore, we believe our correctness proof for our library is of
  %% independent interest. It integrates techniques from verification ---
  %% such as separation logic and rely-guarantee -- with techniques from
  %% semantics, such as step-indexed models and logical relations. The
  %% upshot is that clients of the library can reason as if it were
  %% purely a mathematical object, even though it is implemented in terms
  %% of complex imperative higher-order code. The full suite of equations
  %% --- $\beta$, $\eta$ and fixed-point equations --- are all sound
  %% reasoning principles from the perspective of the client, even though
  %% the implementation is pervasively imperative. 

%% Finally, GUI toolkits often expose an essentially imperative
%%   interface to certain resources (such as the display). We address the
%%   question of how to smoothly integrate these kinds of effectful
%%   operations into our model. 

\section{Reactive Programs and Stream Transformers}

Reactive programs are usually interpreted as \emph{stream
  transformers}. A time-varying value of type $X$ can
be viewed as a stream of $X$s, and so a program that takes a
time-varying $X$ and produces a time-varying $Y$ is then a function
that takes a stream of $X$s and produces a stream of $Y$s. 
%% Then,
%% interactive programs can be specified in terms of the usual suite
%% of function operations on streams. 

%\subsection{Causality}

However, the full function space on streams is too generous: many
functions on streams do not have sensible interpretations as reactive
processes.For example, a stock trading program receives a stream of
prices and emits a stream of orders, but the type
\term{Price^\omega\to Order^\omega} includes functions that produce
orders today that are a function of the price tomorrow; such functions
are (much to our regret) unrealizable.

%% consider the following simple
%% example:

%% \begin{tabbingspec}
%% \term{profit : Price^\omega \to Order^\omega} \Newline
%% \term{profit\; prices = } \Newline
%% \;\;\= \term{let\; prices\_tomorrow = tail(prices) \;in} \Newline
%%     \> \term{if\; head(prices) < head(prices\_tomorrow)\;then} \Newline
%%     \> \qquad\= \term{Buy :: profit(tail(prices))} \Newline
%%     \> \term{else} \Newline
%%     \> \> \term{Sell :: profit(tail(prices))} 
%% \end{tabbingspec}

%% Here, the \term{profit} function receives a stream denoting a stream
%% of daily stock prices. If all the usual stream functions were
%% available, it could take the tail of this stream, and then compare the
%% head of today and tomorrow's stream to determine whether it should buy
%% or sell. Since compiler writers cannot yet generate code which travels
%% in time, we need some kind of semantic condition to explain which
%% stream functions should be ruled out, and which should be ruled in.

The semantic condition that expresses which functions do correspond to implementable processes is \emph{causality}: 
the output at time $n$ should depend only on the first $n$ inputs. We 
formalize this as follows, writing $\floor{xs}_n$ for the 
$n$-element prefix of the stream $xs$:
\begin{definition}{(Causality)}
A stream function $f : X^\omega \to Y^\omega$ is said to be \emph{causal} when,
for all for all $n$ and streams $xs$ and $xs'$, if $\floor{xs}_n = \floor{xs'}_n$
then $\floor{f(xs)}_n = \floor{f(xs')}_n$.   
\end{definition}
This definition rules out, for example, the \term{tail} function, for which the first
$n$ outputs depend upon the first $n+1$ inputs. 
%% For example, the
%% streams $1, 1, 1, \ldots$ and $1, 2, 3, \ldots$ agree on their first
%% output, but their tails disagree right away --- and so \term{tail} is
%% not causal function.

Whilst causality is an intuitive and appealing definition for
streams of basic types (such as integers), it is not
immediately clear how to generalize it. What
might causality mean over a stream of \emph{streams}, or even a
stream of stream functions? 

% \subsection{Guardedness and Fixed Points}

We also want to define streams by feedback or recursion, as in this definition of the increasing sequence of naturals:
\begin{displaymath}
\term{nats = fix(\lambda xs.\; 0 :: map\;succ\;xs)}  
\end{displaymath}
An operational way of thinking about when such fixed points are well-defined is to
observe that the function \term{\lambda xs.\; 0 :: map\;succ\;xs} 
can produce its first output without looking at its input. So we can 
imagine implementing the fixed point by feeding the output at time $n$ back in as the input at time $n+1$, exploiting the fact that at 
time 0 the input value does not matter. This leads us to define:

\begin{definition}{(Guardedness)}
A function $f : X^\omega \to X^\omega$ is said to be \emph{guarded}
when there exists a $k > 0$ such that for all for all $n$ and streams
$xs$ and $xs'$, if $\floor{xs}_n = \floor{xs'}_n$ then
$\floor{f(xs)}_{n+k} = \floor{f(xs')}_{n+k}$.
\end{definition}

\begin{prop}{(Fixed Points of Guarded Functions)}
Every guarded endofunction $f : X^\omega \to X^\omega$ (where $X$ is a
nonempty set) has a unique fixed point.
\end{prop}

As with causality, guardedness is an intuitive and natural property,
but generalizations to higher types seem both useful and unobvious. For
example, we may want to write a recursive \emph{function}:
\begin{displaymath}
\term{fib = fix(\lambda f\; \lambda (j,k).\; j :: f(k,j+k))}
\end{displaymath}
So the natural questions to ask are: what does guardedness mean at
higher type, and how can we interpret fixed points at higher type?
We will answer these questions by moving to metric spaces.

\section{An Ultrametric Model of Reactive Programs}

%\subsection{Ultrametric Spaces in a Nutshell}

A complete 1-bounded \emph{ultrametric space} is a pair $(A, d_A)$,
where $A$ is a set and $d_A \in A \times A \to [0,1]$ is a distance
function, satisfying the following axioms:
\begin{itemize}
\item $d_A(x, y) = 0$ if and only if $x = y$
\item $d_A(x, x') = d_A(x', x)$
\item $d_A(x, x') \leq \max(d_A(x, y), d_A(y, x'))$
\item Every Cauchy sequence in $A$ has a limit
\end{itemize}
A sequence $\langle x_i\rangle$ is Cauchy if for any $\epsilon\in[0,1]$, there is an $n$ such that for all $i>n$, $j>n$, $d(x_i,x_j)\leq \epsilon$. A limit is an $x$ such that for all $\epsilon$, there is an $n$ such that for all $i>n$, $d(x,x_i)\leq \epsilon$. 
Ultrametric spaces satisfy a stronger version of the triangle inequality than
ordinary metric spaces, which only ask that
$d(x,x')$ be less than or equal to $d_A(x, y) + d_A(y, x')$,
rather than $\max(d_A(x, y), d_A(y, x'))$.
We often just write $A$ rather than $(A, d_A)$. 

A map $f : A \to B$ between ultrametric spaces is \emph{nonexpansive} when 
it is non-distance-increasing:
\begin{displaymath}
  \forall x\, x', d_B(f\;x, f\;x') \leq d_A(x, x')
\end{displaymath}

A map $f : A \to B$ between ultrametric spaces is said to be
\emph{strictly contractive} when it shrinks the distance between 
any two points by a nonzero factor:
\begin{displaymath}
  \exists q \in [0,1),\;\forall x\, x',\; d_B(f\;x, f\;x') \leq q \cdot d_A(x, x')
\end{displaymath}

Complete 1-bounded ultrametric spaces and nonexpansive maps form a 
Cartesian closed category. The product is given by
is given by the Cartesian product of the underlying sets, equipped with the pointwise sup-metric:
\begin{displaymath}
  d_{A \times B}((a,b), (a',b')) = \max \setof{d_A(a,a'), d_B(b,b')}
\end{displaymath}
Exponentials give the set of nonexpansive maps a sup-metric over all inputs:
\begin{displaymath}
  d_{A \to B}(f, f') = \sup \comprehend{d_B(f\;a,f'\;a)}{a \in A}
\end{displaymath}
%We take $0$ as the formal maximum of the empty set.

Any set $X$ can be made into an ultrametric space $D(X)$ by
equipping it with the discrete metric that defines $d(x,x')$ to be $0$ if $x=x'$ and $1$ otherwise.

%% \begin{displaymath}
%% d(x,x') = \left\{\begin{array}{ll}
%%                   0 & \mbox{if } x = x' \\
%%                   1 & \mbox{if } x \not= x' \\
%%                 \end{array}
%%           \right.
%% \end{displaymath}

For an ultrametric space $A$, the ultrametric space of
streams on $A$ is defined by equipping the set $A^\omega$ with 
the \emph{causal metric of streams}:
\begin{displaymath}
  d_{A^\omega}(as, as') = \sup \comprehend{2^{-n}\cdot d_A(as_n, as'_n)}{n \in \N}
\end{displaymath}

\noindent Furthemore, this is functorial: for any map $f : A \to B$, 
we define $f^\omega : A^\omega \to B^\omega$ by mapping $f$ over the input stream, which is easily seen to preserve identity and composition.

The interpretation of the stream metric is easiest to understand in
the case of streams of discrete elements. In this case, the metric
says that two streams are closer, the later the time at which they
first disagree. So two streams which have differing values at time $0$
are at a distance of $1$, whereas two streams which never disagree
will have a distance of $0$ (and hence will be equal streams).

\begin{prop}{(Banach's Fixed Point Theorem)}
For any nonempty, complete metric space $A$ and strictly contractive
endofunction $f : A \to A$, there exists a unique fixed point of $f$.
\end{prop}

\subsection{From Ultrametrics to Functional Reactive Programs}

For streams of base type, the properties of maps
in the category of ultrametric spaces correspond exactly to the
properties of first-order reactive programs discussed in the
previous section.

\begin{theorem}{(Causality is Nonexpansiveness)}
Suppose $X$ and $Y$ are sets. Then a function $f : X^\omega \to
Y^\omega$ is causal if and only if it is a nonexpansive function under
the causal metric of streams of elements of the discrete spaces $D(X)$
and $D(Y)$.
\end{theorem}

\begin{theorem}{(Guardedness is Contractiveness)}
Suppose $X$ and $Y$ are sets. Then a function $f : X^\omega \to
Y^\omega$ is guarded if and only if it is a strictly contractive
function under the causal metric of streams of elements of the
discrete spaces $D(X)$ and $D(Y)$.
\end{theorem}

The proof of these two theorems is nothing more than the unwinding of
a few definitions. However, the consequences of are quite dramatic! By
interpreting our programs in the category of ultrametric spaces:
\begin{enumerate}
\item We can interpret tuples and functions (with the full $\beta$ and
  $\eta$ rules) thanks to the Cartesian closure of this category.
\item Since streams are functorial, we can interpret streams of
  streams.
\item Furthermore, contractiveness gives an analogue of guardedness
  that makes sense at all types, and likewise Banach's fixed point
  theorem gives an interpretation of fixed points that makes sense at
  all types.
\end{enumerate}

\noindent In an abstract sense, this semantics fulfill the original
promise of functional reactive programming in a ``no-compromise'' way:
one can freely and naturally write higher-order programs with stream
values, and the properties of ultrametric spaces ensure that all
functions are causal and all recursions well-founded.

\subsection{The Co-Kleisli Category of Streams}
Synchronous dataflow languages like Lucid Synchrone~\cite{synchrone},
have a programming model that differs somewhat from that which the
previous section suggests.  In these languages time is
\emph{implicit}, and streams are only rarely manipulated directly. A
definition like $\term{sum = \fun{(x,y)}{x + y}}$ is implicitly
\emph{lifted} to operate pointwise over streams.  One reason for this
choice is brevity, but a more important one is operational. Arbitrary
stream functions, even causal ones, can be hard or impossible to
implement without \emph{space leaks}. We aim to implement reactive
programs using some state such that the current inputs and state
determine the current outputs and next state. For hardware compilation
or hard real-time programming, one needs that state to be bounded, and
even for less constrained applications, it is unacceptable for the
state to grow unboundedly. By keeping time implicit, restricting
oneself to stream functions that are the the result of the compiler's
automatic lifting, one can make it harder (or impossible) to write
programs that leak memory by retaining arbitrary amounts of history.

%% that is updated on each time step on the basis of the previous state and the current inputs Consider a causal stream function like this:
%% %
%% \begin{displaymath}
%%   \term{skew = \fun{xs}{(xs, 0 :: xs, 2 :: 3 :: xs)}}
%% \end{displaymath}
%% % 
%% This program accepts a stream \term{xs} as an input, and then
%% returns three streams --- the original stream \term{xs}; a stream
%% which first produces 0 and then the values of \term{xs}; and finally a
%% stream which yields 2, 3, and then the values of \term{xs}.

%% This means that if we implement this stream program as an imperative
%% program which modifies some state at each time step to dynamically
%% emit the values of the streams it computes, then at time $n+2$, this
%% program will also need to know the value of \term{xs} at times $n+1$
%% and $n$, in order to correctly emit the values of the second and third
%% components of its result. If programs which do unbounded buffering can
%% be written, then the danger of \emph{space leaks} arises, since the
%% runtime system may need to retain potentially arbitrary amounts of
%% history to compute values. By making time implicit and having the
%% compiler lift values to streams, it becomes difficult to write
%% programs which leak memory under this implementation strategy.

Fortunately, we can capture the essence of this class of restriction
in our mathematical semantics by working in the co-Kleisli category of
the stream comonad on the category of ultrametric spaces.

Recall that a \emph{comonad} on a category $\ultrametric$
is a functor $S : \ultrametric \to \ultrametric$, equipped with two
natural transformations $\counit_A : S(A) \to A$ (the counit) and
$\tails_A : S(A) \to S(S(A))$ (the comultiplication) satisfying the
equations $\tails_A; S(\tails_A) = \tails_A; \tails_{S(A)}$ and
$\tails_A; S(\counit_A) = id = \tails_A; \counit_{T(A)}$. 
%(Here we
%use the semicolon to indicate composition in diagrammatic order.)  
In
the case of streams, the counit $\counit$ is the head function on
streams, and the comultiplication takes a stream and returns the a
stream containing the successive tails of the input stream.\footnote{The ``Kleisli triple'' formulation for monads is perhaps
  more familiar to functional programmers, which is given in terms of
  an extension operator sending maps $f : A \to T(B)$ to $\term{bind}(f)  : T(A) \to
  T(B)$.  Dually, the \emph{coextension} for comonads sends maps $f :
  S(A) \to B$ to $f^\dagger : S(A) \to S(B)$, and can be defined as
  $f^\dagger = \delta; S(f)$.}
$S(\cdot)$ is a \emph{strong} functor: $S(A \times B) \simeq S(A) \times
S(B)$.
%, meaning that a stream of pairs is isomorphic to a pair of
%streams. 
We write \term{unzip} and \term{zip} for the components
of this isomorphism.
%, corresponding to the similarly-named functions in
%functional programs.

\newcommand{\us}{$\ultrametric^S$} 
The co-Kleisli category
$\ultrametric^S$ of a comonad $S : \ultrametric \to \ultrametric$ is
the category of free $S$-coalgebras, which may be presented as having
the same objects $\ultrametric$, but taking maps from $A$ to $B$ to be
in $\ultrametric^S$ to be maps $f : S(A) \to B$ in
$\ultrametric$. Amazingly, \us\ is also cartesian closed; the
identity, composition, projection, pairing, currying and evaluation
maps are defined in Figure~\ref{cokleisli-defs}.

We think of a map $e : A \to B$ in \us\ as the interpretation of a
synchronous dataflow program with a free variable of type $A$ in the
style of Lucid Synchrone - i.e. something that is really implicitly
lifted to work on streams.
%
%% , as is
%% usual for categorical semantics. The fact we \emph{implicitly} lift
%% variables to streams means we need to treat the free variable as a
%% stream value. So we want a morphism $e : A \to B$ to be represented
%% by a function $e : S(A) \to B$ in the underlying category $\ultrametric$. 
%% Intuitively, the meaning of the term can be read as saying ``at each
%% instant, $e$ will have a value of type $B$ depending on the values
%% that its free variables will take on''.
%
The dynamic behavior of synchronous terms is understood
via the coextension; $e^\dagger$ is a map $e : S(A) \to S(B)$ in
$\ultrametric$, and is the function we get by feeding
$e$ the successive tails of the original input stream. So this says
that if the input $A$ takes on the values $as = [a_0, a_1, a_2,
  \ldots]$, then the output results will be $[e(as), e(\tail{as}),
  e(\tail{\tail{as}}), \ldots]$.

Although \us\ is Cartesian closed, it does not have coproducts.
Furthermore the instantaneous
interpretation of terms makes it hard to support 
operations acting on streams of streams. The key difficulty is that
there is no map in this category which takes a stream of streams, and
whose coextension yields the head of the stream of streams.
%
%% More concretely, $\ultrametric^S$ does not have a map which
%% corresponds to taking an input like $[[0,1,2,\ldots], [0,3,6,\ldots],
%%   [0,5,10,\ldots], \ldots]$ and returns 0 on the first time step, 1 on
%% the second, 2 on the third, and so on. The reason is that this
%% sequence must be generated by the coextension of a map of type $S(\N) \to
%% \N$ (in $\ultrametric^S$), and so at time 0 it receives the argument
%% $[[0,1,2,\ldots], [0,3,6,\ldots], [0,5,10,\ldots], \ldots]$, and at
%% time 1 it receives the argument $[0,3,6,\ldots], [0,5,10,\ldots],
%% \ldots]$ and so on. So at time 1, we no longer remember what the
%% argument at the time 0 was, and so cannot return the right answer.
%
This, in turn, makes defining fixed point operators very difficult: a
variable of type 
\footnote{We use $A \lolli A$ to denote
  the exponential in $\ultrametric^S$.} $A \lolli A$
denotes a stream of functions. If all of these
functions are contractive, and we take fixed point of coextensions
pointwise, then we get a stream of streams --- at which point we
discover that we can never look at the \emph{second} element of any of
the results. This amnesia is the very reason that space leaks become
harder to program: this is not an accidental difficulty!

\begin{figure}
{\small
\begin{mathpar}
  \begin{array}{lcl}
    \mathsf{id}   & = & \counit \\
    \mathsf{f; g}  & = & \tails; S(f); g \\[1em]

    \mathsf{fst} & = & \mathsf{unzip}; \fst{}; \counit \\
    \mathsf{snd} & = & \mathsf{unzip}; \snd{}; \counit \\
    \mathsf{pair}(f,g) & = & (f,g) \\[1em]

    \mathsf{curry}(f) & = & \lambda(\mathsf{zip}; f) \\
    \mathsf{eval}     & = & \mathsf{unzip}; (\counit \times \mathit{id}); \mathit{eval} \\
 \end{array}
\end{mathpar}
}
\caption{Definition of operations in the co-Kleisli category}
\label{cokleisli-defs}
\end{figure}

\subsection{Adjoint Logic}

At this point, we have two views of reactive programming. One, which
we might call ``FRP-style'', has a very simple semantics in terms of
ultrametrics and stream values, and naturally supports fixed points at
any type. However, it seems difficult to implement efficiently.  On
the other hand, we can take a ``synchronous dataflow'' view to support
more efficient implementation techniques, at the price of making it
difficult to give good semantics to feedback. So it is natural to ask
if there is some way of combining the strengths of the two approaches.
We meet this goal by adapting 
adjoint-style models of intuitionistic linear logic \cite{benton:lnl,benton-wadler}. Originally developed
to give a model of linear logic, these turn out to be abstract enough
to apply naturally to the setting of dataflow programming as
well. 
\begin{definition}{} 
An \emph{adjoint model} is specified by:
\begin{enumerate}
\item A cartesian closed category $(\ultrametric, 1, \times, \To)$. 
\item A symmetric monoidal closed category $(\ultrametric^S, I, \otimes, \lolli)$.
\item A symmetric monoidal adjunction $((-)^\omega, V, \eta, \varepsilon, m, n)$ from $\ultrametric$ to $\ultrametric^S$. Here, $\eta$ and $\varepsilon$ witness the adjunction, and $m$ and $n$ are the natural transformations
showing that the monoidal structure is preserved.
\end{enumerate}
\end{definition}
In our setting, we take the CCC to be the category of ultrametric
spaces, and take the monoidal closed structure in the definition to be
the cartesian closed structure of the co-Kleisli category (since 
Cartesian products are a special case of monoidal products). We
decompose the stream comonad into the usual free and forgetful functors that go between the category of 
ultrametric spaces and the co-Kleisli category of streams over
it. Specifically, the functors $(-)^\omega : \ultrametric^S \to \ultrametric$
and $V(-) : \ultrametric \to \ultrametric^S$ are defined as:
{\small
\begin{mathpar}
  \begin{array}{lcl} 
    V(X)           & = & X \\
    V(f : X \to Y) & = & \epsilon_X; f \\[0.5em]

    A^\omega             & = & S(X) \\
    (f : A \to B)^\omega & = & \delta_{A}; S(f) = f^\dagger\\
  \end{array}
\end{mathpar}
}
Intuitively $V(f)$ takes a function $f$ from the general FRP world and
embeds it into the synchronous dataflow world by having it act at each
instant on the head of the stream. The action of the other half of the
adjoint $(g)^\omega$ takes a synchronous dataflow function, and turns
it into a general function on streams, via the coextension operation. 
Verifying that $V(f)^\omega = S(f)$ is immediate from the definitions. 

Furthemore, we know that $V(X \times Y) \simeq V(X) \otimes V(Y)$, and
also $X \To B^\omega \simeq (V(X) \lolli B)^\omega$. This latter
isomorphism is actually the key property which lets us switch our view
of a program between the synchronous and the functional reactive
views.

For example, under the adjoint view, it becomes easy to resolve the
puzzle at the end of the last subsection. We give the feedback
operator the type $(A \lollishrink A)^\omega \To A^\omega$, with
the semantics that it returns a stream corresponding to the fixed
point of the \emph{first} function in the input stream. Then, a
synchronous program can start looking at the elements of the
result whenever it wants to.

\section{A Domain Specific Language}

We now give a small domain-specific language
corresponding to the semantics of the previous section. As in adjoint
logic, we have two sorts of types, writing $X,Y,Z$ for the types of
the lambda calculus interpreted in the base category of ultrametric
spaces, and $A,B,C$ for the types interpreted in the co-Kleisli
category. In addition to the standard function, product and adjoint
types, we also introduce types corresponding to \emph{contractive
  functions}, $X \shrink Y$ and $A \lollishrink B$.

\begin{figure}
{\small
\begin{mathpar}
\inferrule*[]
          {x:X \in \Gamma}
          {\judgeu{\Gamma}{x}{X}}
\and
\inferrule*[]
          {\judgeu{\Gamma}{e}{X \To Y} \\ \judgeu{\Gamma}{e'}{X}}
          {\judgeu{\Gamma}{e\;e'}{Y}}
\and
\inferrule*[]
          {\judgeu{\Gamma,x:X}{e}{Y}}
          {\judgeu{\Gamma}{\fun{x}{e}}{X \To Y}}
\and
\inferrule*[]
          { }
          {\judgeu{\Gamma}{\unitval}{1}}
\and
\inferrule*[] 
          {\judgeu{\Gamma}{e}{X} \\ \judgeu{\Gamma}{e'}{Y}}
          {\judgeu{\Gamma}{\pair{e}{e'}}{X \times Y}}
\and
\inferrule*[]
          {\judgeu{\Gamma}{e}{X_1 \times X_2}}
          {\judgeu{\Gamma}{\pi_i\;e}{X_i}}
\and
\inferrule*[]
          {x:A \in \Gamma}
          {\judgek{\Delta}{x}{A}}
\and
\inferrule*[]
          {\judgek{\Delta}{t}{A \lolli B} \\ \judgek{\Delta}{t'}{A}}
          {\judgek{\Delta}{t\;t'}{B}}
\and
\inferrule*[]
          {\judgek{\Delta,x:A}{t}{B}}
          {\judgek{\Delta}{\fun{x}{t}}{A \lolli B}}
\and
\inferrule*[]
          { }
          {\judgek{\Delta}{\unitval}{I}}
\and
\inferrule*[] 
          {\judgek{\Delta}{t}{A} \\ \judgek{\Delta}{t'}{B}}
          {\judgek{\Delta}{\pair{t}{t'}}{A \otimes B}}
\and
\inferrule*[]
          {\judgek{\Delta}{t}{A_1 \otimes A_2}}
          {\judgek{\Delta}{\pi_i\;t}{A_i}}
\\
\inferrule*[]
          {\judgek{\cdot}{t}{A}}
          {\judgeu{\Gamma}{t^\omega}{A^\omega}}
\and
\inferrule*[]
          {\judgeu{\Gamma}{e}{A^\omega}}
          {\judgek{\Delta}{\Start{e}}{A}}
\and
\inferrule*[]
          {\judgeu{\Gamma}{e}{X}}
          {\judgek{\Delta}{\valtype{e}}{\valtype{X}}}
\and
\inferrule*[]
         {\judgek{\Delta}{t}{\valtype{X}} \\ \judgek[\Gamma,x:X]{\Delta}{t'}{A}}
         {\judgek{\Delta}{\letval{x}{t}{t'}}{A}}
\end{mathpar}
}
\caption{Syntax for Adjoint Logic}
\label{adjoint-syntax}
\end{figure}


\begin{figure}
{\small
\begin{mathpar}
\inferrule*
          {\judgeu{\Gamma}{e}{X}}
          {\judgec{\hat{\Gamma}}{\const{e}}{X}}
\and
\inferrule*
          {\judgec{\hat{\Delta}}{c}{X \shrink Y} \\
           \judgeu{\Gamma, \hat{\Gamma}}{e}{X}}
          {\judgec{\hat{\Gamma}}{c\;e}{Y}}
\and
\inferrule*
          {\judgec{\hat{\Gamma}}{c}{X} \\ 
           \judgec{\hat{\Gamma}}{c'}{Y}}
          {\judgec{\hat{\Gamma}}{\pair{c}{c'}}{X \times Y}}
\and
\inferrule*
          {\judgec{\hat{\Gamma}}{c}{X_1 \times X_2}}
          {\judgec{\hat{\Gamma}}{\pi_i\;c}{X_i}}
\and
\inferrule*
          { }
          {\judgec{\hat{\Gamma}}{\unit}{1}}
\and
\inferrule*
          {\judgec{\hat{\Gamma}}{c}{X} \\ 
           \judgec[\Gamma, x:X]{\hat{\Gamma}}{c'}{Y}}
          {\judgec{\hat{\Gamma}}{\letc{x}{c}{c'}}{Y}}
\and
\inferrule*
          {\judgec[\Gamma, x:X]{\hat{\Gamma}}{c}{b}}
          {\judgec{\hat{\Gamma}}{\fun{x:X}{c}}{X \To Y}}
\and
\inferrule*
          {\judgec{\hat{\Gamma}}{c}{X \To Y} \\ 
           \judgec{\hat{\Gamma}}{c'}{X}} 
          {\judgec{\hat{\Gamma}}{c\;c'}{Y}}
\and
\inferrule*
          {\judgec{\hat{\Gamma}, x:X}{c}{Y}}
          {\judgec{\hat{\Gamma}}{\sfun{x:X}{c}}{X \shrink Y}}

\\

\inferrule*
          {\judgek{\Delta}{t}{A}}
          {\judgekc{\hat{\Delta}}{\const{t}}{A}}
\and
\inferrule*
          {\judgekc{\hat{\Delta}}{d}{A \lollishrink B} \\
           \judgek{\Delta, \hat{\Delta}}{t}{A}}
          {\judgekc{\hat{\Delta}}{d\;t}{B}}
\and
\inferrule*
          {\judgekc{\hat{\Delta}}{d}{A} \\ 
           \judgekc{\hat{\Delta}}{d'}{B}}
          {\judgekc{\hat{\Delta}}{\pair{d}{d'}}{A \otimes B}}
\and
\inferrule*
          {\judgekc{\hat{\Delta}}{d}{A_1 \otimes A_2}}
          {\judgekc{\hat{\Delta}}{\pi_i\;d}{A_i}}
\and
\inferrule*
          { }
          {\judgekc{\hat{\Delta}}{\unit}{I}}
\and
\inferrule*
          {\judgekc{\hat{\Delta}}{d}{A} \\ 
           \judgekc[\Delta, x:A]{\hat{\Delta}}{d'}{B}}
          {\judgekc{\hat{\Delta}}{\letc{x}{d}{d'}}{B}}
\and
\inferrule*
          {\judgekc[\Delta, x:A]{\hat{\Delta}}{d}{b}}
          {\judgekc{\hat{\Delta}}{\fun{x:A}{d}}{A \lolli B}}
\and
\inferrule*
          {\judgekc{\hat{\Delta}, x:A}{d}{B}}
          {\judgekc{\hat{\Delta}}{\sfun{x:A}{d}}{A \lollishrink B}}

\end{mathpar}
}
\caption{Syntax for Contractive Terms}
\label{contractive-syntax}      
\end{figure}

There are two judgement forms, $\judgeu{\Gamma}{e}{Y}$ and
$\judgek{\Delta}{t}{B}$, where $\Gamma = x_1:X_1,\label,x_n:X_N$ and
$\Delta = y_1:A_a, \ldots, y_m:A_n$.  We then interpret the first
judgement as a map in $\ultrametric$, of type $X_1 \times \ldots
\times X_n \to Y$, and the second judgement as a map in
$\ultrametric^S$, of type $V(X_1 \times \ldots \times X_n \to Y)
\otimes A_1 \otimes \ldots \otimes A_m \to B$. Here, $e$ ranges over
``FRP programs'' interpreted in the base category of ultrametric
spaces, and $t$ ranges over ``synchronous dataflow programs''
interpreted in the co-Kleisli category. Note the asymmetry in 
the judgments -- the ``FRP'' context appears in the ``dataflow''
judgement, but not vice-versa. Intuitively, this is because the FRP
world is pure, and can be used at any time, whereas the synchronous
world is implicitly clocked as a kind of ambient effect. 

We give typing rules for these programs in
Figure~\ref{adjoint-syntax}.  Most of the rules are routine, with all
of the interest lying in the rules that permit passing between the two
worlds, listed as the last four rules of
Figure~\ref{adjoint-syntax}. The $\omega$-introduction rule takes a
dataflow program with no free variables, and treats it as a stream in
$\ultrametric$. The corresponding elimination rule says that in a
dataflow program, we can take a term $e : A^\omega$, and start
enumerating its values with $\Start{e}$. Conversely, the final rule
says how synchronous programs can decompose embedded FRP values
and bind them for use in subsequent synchronous computations.

In addition to these two judgements, we also need a pair of judgements
permitting programmers to define contractive functions, which we
describe in Figure~\ref{contractive-syntax}. The first is
$\judgec{\hat{\Gamma}}{c}{X}$, which asserts that $c$ is a term which
is strictly contractive in the variables in $\hat{\Gamma}$.  (It can
be merely nonexpansive in the variables in $\Gamma$.) Likewise, we
have a judgement $\judgekc{\hat{\Delta}}{d}{A}$, with $d$ 
strictly contractive in the variables in $\hat{\Delta}$ and
nonexpansive in the others. The interpretation of
$\judgec{\hat{\Gamma}}{c}{X}$ is a morphism with type $\Gamma \to
(\hat{\Gamma} \shrink X)$, and the interpretation of
$\judgekc{\hat{\Delta}}{d}{X}$ is a morphism of type $V(\Gamma)
\otimes \Delta \to (\hat{\Delta} \lollishrink A)$.  

At first glance, these judgements look similar to the usual rules for
introducing and eliminating functions, pairs, and so on. Upon second
glance, they look extremely peculiar! We introduce a second (or third)
context of variables, into which we move the binders for contractive
functions. However, we do so \emph{without} giving a corresponding
variable rule that permits directly using these hypothesis. Instead,
there are some rules (such as the application rule for contractive
functions) that permit moving the variables from the new context into
the old. From a semantic point of view, this is only to be expected ---
the interpretation of the variable rule in categorical proof theory is
an identity morphism, and identities \emph{are not contractive}. So we
cannot expect to have a normal variable rule for the hypotheses
corresponding to arguments of contractive functions. Operationally,
this embodies our need to ensure that the variables are only used as
arguments to contractive functions (such as \term{cons}), which will
ensure that the terms they appear in are always guarded.

To define the semantics, we need some maps in the category of
ultrametrics which witness the fact that contractivenss is preserved
by almost everything. The implementations of these functions are the
evident ones given the types, since contractive functions are just a
subset of the nonexpansive ones. (Also, we elide the essentially
identical semantics of contractive terms in the co-Kleisli category.)

\begin{figure*}
{\small
\begin{mathpar}
  \begin{array}{ll}
  \begin{array}{lcl}
    \interp{\judgeu{\Gamma}{x_i}{X_i}} & = & \pi_i \\
    \interp{\judgeu{\Gamma}{e\;e'}{Y}} & = & 
       (\interp{\judgeu{\Gamma}{e}{X \To Y}}, \interp{\judgeu{\Gamma}{e'}{X}}); \mathit{eval} \\
    \interp{\judgeu{\Gamma}{\fun{x}{e}}{X \To Y}} & = & 
      \lambda(\interp{\judgeu{\Gamma,x:X}{e}{Y}}) \\
    \interp{\judgeu{\Gamma}{\unit}{1}} & = & 1_{\Gamma} \\
    \interp{\judgeu{\Gamma}{\pair{e}{e'}}{A \times B}} & = & 
      (\interp{\judgeu{\Gamma}{e}{A}}, \interp{\judgeu{\Gamma}{e'}{B}}) \\
    \interp{\judgeu{\Gamma}{\pi_i(e)}{A_i}} & = & 
      \interp{\judgeu{\Gamma}{e}{A_1 \times A_2}}; \pi_i \\
    \interp{\judgeu{\Gamma}{t^\omega}{A^\omega}} & = &  \eta_{\Gamma}; \interp{\judgek{\cdot}{t}{A}}^\omega 
  \end{array}
    & 
  \begin{array}{lcl}
    \interp{\judgek{\Delta}{x_i}{A_i}} & = & \pi_2; \pi_i \\
    \interp{\judgek{\Delta}{t\;t'}{B}} & = & 
       (\interp{\judgek{\Delta}{e}{A \lolli B}}, \\
       & & \;\interp{\judgek{\Delta}{e'}{A}}); \mathit{eval} \\
    \interp{\judgek{\Delta}{\fun{x}{t}}{A \lolli B}} & = & 
      \lambda(\interp{\judgek{\Delta,x:A}{t}{B}}) \\
    \interp{\judgek{\Delta}{\unit}{I}} & = & 1_{\Delta} \\
    \interp{\judgek{\Delta}{\pair{t}{t'}}{A \otimes B}} & = & 
      (\interp{\judgek{\Delta}{t}{A}}, \interp{\judgek{\Delta}{t'}{B}}) \\
    \interp{\judgek{\Delta}{\pi_i(t)}{A_i}} & = & 
      \interp{\judgek{\Delta}{t}{A_1 \otimes A_2}}; \pi_i  \\
    \interp{\judgek{\Delta}{\valtype{t}}{\valtype{X}}} & = & 
      \pi_1; V(\interp{\judgeu{\Gamma}{t}{X}}) \\
    \interp{\judgek{\Delta}{\Start{t}}{A}} & = & 
      \pi_1; V(\interp{\judgeu{\Gamma}{t}{A^\omega}}); \varepsilon_A \\
    \interp{\judgek{\Delta}{\letval{x}{t}{t'}}{A}} & = & 
       ((\pi_1, \interp{\judgek{\Delta}{t}{\valtype{X}}}); n, \pi_2); \\ 
                                        & & \interp{\judgek[\Gamma,x:X]{\Delta}{t'}{A}}
  \end{array}
\end{array}
\end{mathpar}
}
\caption{Semantics of Adjoint Language}
\label{adjoint-semantics}
\end{figure*}
%%

%%
\begin{figure*}
{\small
\begin{mathpar}
\begin{array}{ll}
  \begin{array}{l}
    \sweak  : X \to (Y \shrink Z) \\
    \spair  : (X \shrink Y) \times (X \shrink Z) \to (X \shrink Y \times Z) \\
    \scurry : (X \times Y \shrink Z) \to (X \shrink Y \shrink Z) \\
    \seval  : (X \shrink Y \shrink Z) \times (X \To Y) \to (X \shrink Z) \\
    \swap   : (X \To Y \shrink Z) \simeq (Y \shrink X \To Z) \\
    \scomposer(f : Y \to Z)  :  (X \shrink Y) \to (X \shrink Z) 
  \end{array}
&
 \begin{array}{lcl}
  \interp{\judgec{\hat{\Gamma}}{\const{e}}{X}} & = & \interp{\judgeu{\Gamma}{e}{X}}; \sweak \\
  \interp{\judgec{\hat{\Gamma}}{c\;e}{Y}} & = & 
     (\interp{\judgec{\hat{\Gamma}}{c}{X \shrink Y}}, \lambda(\interp{\judgeu{\Gamma,\hat{\Gamma}}{e}{X}})); \seval \\ 
  \interp{\judgec{\hat{\Gamma}}{\pi_i(c)}{X_i}} & = & 
    \interp{\judgec{\hat{\Gamma}}{c}{X_1 \times X_2}}; \scomposer(\pi_i)
\\
  \interp{\judgec{\hat{\Gamma}}{\unit}{1}} & = & 
     1_\Gamma; \sweak_{\hat{\Gamma}}
\\
  \interp{\judgec{\hat{\Gamma}}{\letc{x}{c}{c'}}{Y}} & = & 
    (\swap(\lambda(\interp{\judgec[\Gamma,x:X]{\hat{\Gamma}}{c'}{Y}})), \interp{\judgec{\hat{\Gamma}}{c}{X}}); \\
    & & \spair; \scomposer(\eval)
\\
  \interp{\judgec{\hat{\Gamma}}{\fun{x:X}{c}}{X \To Y}} & = & 
    \swap(\lambda(\interp{\judgec[\Gamma,x:X]{\hat{\Gamma}}{c}{Y}})) 
\\
  \interp{\judgec{\hat{\Gamma}}{\fun{x:X}{c}}{X \shrink Y}} & = & 
    \interp{\judgec{\hat{\Gamma}, x:X}{c}{Y}}; \scurry 
\\
  \interp{\judgec{\hat{\Gamma}}{\pair{c}{c'}}{X \times Y}} & = & 
      (\interp{\judgec{\hat{\Gamma}}{c}{X}},
       \interp{\judgec{\hat{\Gamma}}{c'}{Y}}); \spair
 \end{array}
\end{array}
\end{mathpar}
}
\caption{Semantics of Contractive Terms}
\end{figure*}

\section{Implementation Language and Dataflow Library}

\noindent\textbf{Implementation Language}. The programming language
in which we implement our domain-specific language is a 
polymorphic lambda calculus with monadically typed 
side-effects.  The types are
the unit type $1$, the function space $\tau \to \sigma$, sums $\tau +
\sigma$, products $\tau \star \sigma$, inductive types like the
natural number type $\N$, the general reference type $\reftype{\tau}$,
as well as (higher-kinded but still predicative) universal and
existential types $\forall \alpha:\kappa.\;\tau$ and $\exists
\alpha:\kappa.\;\tau$. In addition, we have the monadic type
$\monad{\tau}$ for side-effecting
computations producing values of type $\tau$. The side effects we
consider are heap effects (such as reading, writing, or allocating
references) and nontermination.
%% We maintain such a strong distinction between pure and impure code for
%% two reasons. First, it gives us the full suite of equational reasoning
%% principles for pure code, which simplifies reasoning even about
%% imperative code, since it allows us to freely restructure our program
%% to match our invariants. Second, it lets us use a rich set of program
%% terms (including things like function calls and arithmetic) in
%% assertions about the program.
The syntax, typing, and semantics of the implementation
language are all standard, and we omit them for reasons of space. The construction of fixed points in the implementation language is restricted to \emph{pointed} types, which are the monadic types, products of pointed types and function types whose codomain is pointed. The other monadic primitives are 
$\newref{e}$, $!e$, and $e := e'$,
which allocate, read and write references (inhabiting type
$\reftype{\tau}$), respectively. 

%% The
%% exception is our syntax for monadic computations: suspended
%% computations $\suspend{c}$ inhabit the monadic type $\monad{A}$.
%% These computations are not immediately evaluated, which allows us to
%% embed them into the pure part of the programming
%% language. Furthermore, we can take fixed points $\fix{x:D}{e}$ of
%% terms, to give us a general recursion. Because we wish to permit
%% nonterminating programs only at monadic types, we must restrict
%% $\term{fix}$ to a limited family of types so that we do not
%% contaminate our language with infinite loops at every type, including
%% monadic types, functions returning a monadic type, and products of
%% same. (The allowed types are those whose interpretations are pointed
%% CPOs in domain theory.)  We will write recursive functions as
%% syntactic sugar for $\term{fix}$.

%% The computations themselves include all expressions $e$, as
%% computations that coincidentally have no side-effects. Furthermore, we
%% have sequential composition $\letv{x}{e}{c}$. Intuitively, the
%% behavior of this command is as follows. We evaluate $e$ until we get
%% some $\suspend{c'}$, and then evaluate $c'$, modifying the heap and
%% binding the return value to $x$. Then, in this augmented environment,
%% we run $c$. The fact that monadic commands have return values explains
%% why our sequential composition is also a binding construct. Finally,
%% we have primitive computations $\newref{e}$, $!e$, and $e := e'$,
%% which let us allocate, read and write references (inhabiting type
%% $\reftype{A}$), respectively. 

\noindent \textbf{Program Logic.} We reason about programs in the implementation language in the program
logic whose syntax is shown in Figure~\ref{assert-syntax}. The Hoare triple
$\spec{p}{c}{a:\tau}{q}$ is used to specify computations, and is satisfied when
running the computation $c$ in any heap satisfying the predicate $p$ either diverges or yields a heap satisfying $q$; note that the value returned by terminating executions of $c$ is bound (by $a:\tau$) in the postcondition.
These atomic specifications can then be combined with
the usual logical connectives of intuitionistic logic including
conjunction, disjunction and implications, as well as quantifiers
ranging over the sorts in $\omega$. This permits us to give abstract
specifications to modules using existential quantifiers to hide
program implementations and predicates.

The assertions in the pre- and post-conditions are drawn
from higher-order separation logic~\cite{hosl}. In addition to the
usual connectives of Hoare logic, we add \emph{spatial}
connectives to talk about shared state. The separating conjunction $p *
q$ is satisfied by states can be split into two \emph{disjoint} parts,
one of which satisfies $p$, and the other of which satisfies
$q$. The disjointness property makes the
noninterference of $p$ and $q$ implicit, simplifying specifications
greatly. 
$\emp$, which is true only of the empty heap, is the unit of $*$. The points-to relation $e
\pointsto e'$, holds of the one-element heap in which the value
of the reference $e$ has contents equal to the value of $e'$.

The universal and existential quantifiers $\forall x:\omega.\;p$ and
$\exists x:\omega.\;p$ are higher-order quantifiers ranging over all
sorts $\omega$. The sorts include the language types $A$, kinds
$\kappa$, the sort of propositions $\assert$, and function spaces over
sorts $\omega \To \omega'$.  For the function space, we
include lambda-abstraction and application. Because our assertion
language contains within it the classical higher-order logic of sets,
we will freely make use of features like subsets, indexed sums, and
indexed products, exploiting their definability.


\begin{figure}
{\small
\begin{displaymath}
\begin{array}{llcl}
\mbox{Assertion Sorts} & 
\omega & ::= & \tau \bnfalt \kappa \bnfalt \omega \To \omega \bnfalt \assert 
\\[0.5em]
\mbox{Assertion} & 
p & ::= & e \bnfalt \tau \bnfalt x \bnfalt \fun{x:\omega}{p} \bnfalt p\;q \\
\mbox{Constructors}
& &  |  & \top \bnfalt p \land q \bnfalt p \implies q 
          \bnfalt \bot \bnfalt p \vee q \\
& &  |  &  \emp \bnfalt p * q  \bnfalt e \pointsto e' \\
& &  |  & \forall x:\omega.\; p \bnfalt \exists x:\omega.\; p \bnfalt S
          
\\[0.5em]
\mbox{Specifications} &
S & ::= & \spec{p}{c}{a:A}{q}  \bnfalt \setof{p} \\
& &  |  & S \specand S' \bnfalt S \specimp S' \bnfalt S \specor S' \\
& &  |  & \forall x:\omega.\; S \bnfalt \exists x:\omega.\;S 
\\
\end{array}
\end{displaymath}
}
\caption{Specification Language}
\label{assert-syntax}
\end{figure}

\textbf{Dataflow Library.} We implement our DSL on top of an
imperative dataflow network, which is rather like
a generalized spreadsheet. There is a collection
of cells, each of which contains some code whose evaluation may read
other cells. When a cell is read, the expression within the cell is
evaluated, recursively triggering the evaluation of other cells as
they are read by the program expression. Furthermore, each cell
memoizes its expression, so that repeated reads of the same cell will
not trigger re-evaluation.  
%Whenever a cell is modified, it tells
%every cell that has read it that its value may have changed, and that
%any memoization that they have done is no longer valid.

We will compile a synchronous dataflow program into a dataflow graph,
which is run inside an event loop. The event loop updates a clock cell
to notify the cells in the graph that they may need to recompute
themselves, and then it reads the cells it is interested in, doing
(hopefully) the minimal amount of computation needed at each time
step.

We give the interface to a dataflow library in
Figure~\ref{notification-implementation}. We have given a correctness
proof of this library in prior work~\cite{krishnaswamietal:ramified},
but will describe the specification here in detail, since we use it as
a component of the present work.

The interface features two abstract data types, $\mathsf{cell}$
and $\mathsf{code}$. We actually expose the implementations in the
figure, to better discuss them. $\mathsf{cell}\;\alpha$ is the type of
cells that compute a value of type $\alpha$. It is implemented as a
record with four fields.  The $\mathit{code}$ field contains an
expression that will compute both a value of type $\alpha$ and
a set of cells that were read in the process.  The ($\mathit{value}$)
field is used for memoizing the computed value.  ($\mathit{reads}$)
tracks which cells this one has read, whilst ($\mathit{obs}$) records
which cells are observing this one.  When the cell is read it returns its memoized value if it has one; otherwise it runs its stored code
to compute a value, updates its memo field, transitively
invalidates its observers and registers itself as an observer of those cells it
read during evaluation.

The $\mathsf{code}\;\alpha$ type
is a user-defined monadic type, as is commonly defined in Haskell. It
has the responsibility of both computing a value of type $\alpha$, as
well as returning a set of all the cells that it read in the process
of computing its return value. 
 The \term{bind} and \term{return}
operations are the unit and extension operations of the user-level
monad, and the operations it supports are given in lines 12-16. There
is an operation to \term{read} a cell, an operation \term{cell} to
create a cell, and \term{getref}, \term{setref}, and \term{newref}
operations to create local state in the dataflow network. On line 17,
there is also an operation to update a cell, but it does not live
within the $\mathsf{code}$ monad.

As each cell tracks both who it reads and who
observes it, it may seem that there is a global invariant on the whole
dataflow graph, which would make local reasoning difficult. This is
true, but it is possible to work around this difficulty. The key idea
is to introduce an \emph{domain-specific separation logic} tuned to
the needs of proving dataflow programs correct. That is, we introduce
a predicate $H(\theta)$ describing the global invariant of the whole
dataflow graph, but index it by spatial formulas which describe the
dataflow graph in a \emph{local} way. Then, we can use these formulas
to specify the operations of the library. 

We give the syntax of formulas in Figure~\ref{library-specification},
above the specifications of the library operations. $I$ and $\phi
\otimes \psi$ correspond to the $\emp$ and separating conjunction of
separation logic, denoting empty graphs and two disjoint collections
of cells. However, in addition to the $\localref{r}{v}$ predicate
(which corresponds to points-to in separation logic), we include a
pair of predicates describing cells. The predicate
$\cellminus{\term{c}}{e}$ means that \term{c} is a cell in the
dataflow graph containing code $e$, and that it is not ready --- i.e.,
it needs to be evaluated before producing a value.  The predicate
$\cellplus{\term{c}}{e}{\term{v}}{rs}$ almost means the opposite: it
means that \term{c} is ready (i.e., has a memoized value),
\emph{conditional} on all its dependencies in $rs$ being ready
themselves.  Since establishing this can require us to follow paths in
the heap, we introduce two inductively-defined relations
$\unready{\theta}{\term{c}}$, and
$\ready{\theta}{\term{c}}{\term{v}}$. These are defined in the obvious
way on the syntax of formulas $\theta$, and establish respectively
that the cell \term{c} is unready --- either it or one of its
ancestors are a negative cell --- or that \term{c} and all of its
ancestors are positive cells.

Now, we can explain the specifications of the code expressions in
Figure~\ref{library-specification}. First, all of these specifications
are parameterized by an extra quantifier $\forall \psi.\ldots$, which
lets us manually build in a kind of frame rule into this specification
--- any formula we can derive will also be quantified, and hence work
in larger dataflow graphs. However, one oddity of these rules is that
the framed formula $\psi$ is asymmetric; in the postcondition, we
frame on a formula like $\Re(u, \psi)$. This is a ``ramification
operator'', whose purpose is to look at the dependencies of cells in
$\psi$ and ensure that they are not falsely marked as ready due to 
other updates. 

On line 1, \term{return} leaves its frame untouched, and returns its
value \term{v} without reading any cells. On lines 2-5, we have a
specification for \term{bind}. It looks complicated, but is actually
very straightforward --- the specification of \term{bind\;e\;f} is
that it takes its state from $\theta$ to $\theta''$, assuming that $e$
takes $\theta$ to $\theta'$, and \term{f} (with the return value of
\term{e}) takes $\theta'$ to $\theta''$. The operations \term{newref},
\term{getref}, \term{setref} have the obvious actions on local
references, and \term{cell} simply allocates a new cell, leaving it
in an unready state. On lines 10-11, we have a specification for 
\term{read}, in the case that its argument is ready. Finally, on
lines 12-16, we have the rule that explains what happens when the
cell is unready --- we need to know what the code in the cell does,
and we also need to know that this code does not modify the current
cell itself. If so, then the heap is updated to reflect \emph{both}
the action of the code, and the effect of setting the cell to a 
positive state. 

\begin{figure}
{\small
\begin{tabbingspec}
\label{notify:codetype}
\!$\mathsf{code} :\; \star \to \star$ \Newline
$\mathsf{code}\;\alpha = \monad{(\alpha \times \mathsf{set(ecell)})}$ 
\Newline[0.5em]\label{notify:celltype}
\!$\celltype{} :\; \star \to \star$ \Newline
$\celltype{\alpha} = \{$\=$code: \reftype{\comp{\alpha}};$ \Newline
                   \>$value: \reftype{\opttype{\alpha}};$ \Newline
                   \>$reads: \reftype{\mathsf{set(ecell)})};$ \Newline
                   \>$obs:   \reftype{\mathsf{set(ecell)})};$ \Newline
                   \>$unique:    \N\}$ 

\Newline[0.5em]\label{notify:ecell}
\!$\mathsf{ecell} = \exists \alpha:\star.\; \celltype{\alpha}$ 

\Newline[0.5em]\label{notify:return}
$\term{return} : \forall \alpha:\star.\; \alpha \to \comp{\alpha}$ 

\Newline\label{notify:bind}
$\term{bind} : \forall \alpha,\beta:\star.\; \comp{\alpha} \to (\alpha \to \comp{\beta}) \to \comp{\beta}$ 

\Newline\label{notify:read}
$\term{read} : \forall \alpha:\star.\; \celltype{\alpha} \to \comp{\alpha}$ 

\Newline\label{notify:newcell}
$\term{cell} : \forall \alpha:\star.\; \comp{\alpha} \to \comp{\celltype{\alpha}}$

\Newline\label{notify:newref}
$\term{newref} : \forall \alpha : \alpha \to \comp{\reftype{\alpha}}$ 


\Newline\label{notify:getref}
$\term{getref} : \forall \alpha : \reftype{\alpha} \to \comp{\alpha}$ 

\Newline\label{notify:setref}
$\term{setref} : \forall \alpha : \reftype{\alpha} \to \alpha \to \comp{\unittype}$ 

\Newline[0.5em]\label{notify:update}
$\term{update} : \forall \alpha:\star.\; \comp{\alpha} \to \celltype{\alpha} \to \monad{\unittype}$
\end{tabbingspec}
}
\caption{Implementation of Notification Networks}
\label{notification-implementation}
\end{figure}


\begin{figure}
{\small

\begin{displaymath}
  \begin{array}{lcl}
    \phi,\psi,\theta \!\!\!& ::= & \!\!\!\!I \bnfalt \phi \otimes \psi \bnfalt \cellplus{\term{c}}{e}{v}{rs} \bnfalt \cellminus{\term{c}}{e} \bnfalt  \localref{r}{v} 
  \end{array}
\end{displaymath}

\begin{tabbingspec}
$\forall \psi.\;\spec{H(\psi)}{\term{return(v)}}{(\term{a},\emptyset)}{H(\psi) \land \term{a = v}}$ 
\Newline[0.5em]

$\forall \psi.\;\spec{H(\theta \otimes \psi)}{\term{e}}{(\term{a}, r)}{H(\theta' \otimes \Re(u, \psi)) \land \term{a = v} \land r = r_1} \specand$ \Newline
$\forall \psi.\;\spec{H(\theta' \otimes \psi)}{\term{f\;v}}{(\term{a}, r)}{H(\theta'' \otimes \Re(u', \psi)) \land \term{a = v'} \land r = r_2}$ \Newline
$\specimp \forall \psi.\;\setof{H(\theta \otimes \psi)}{\term{bind\;e\;f}}\{(\term{a}, r).\;$\=$H(\theta'' \otimes \Re(u \cup u', \psi))$ \Newline
\> $\land\; \term{a = v'} \land r = r_1 \cup r_2\}$
\Newline[0.5em]

$\forall \psi.\spec{H(\psi)}{\term{newref(v)}}{(\term{a}, \emptyset)}{H(\psi\otimes \mathsf{ref}\term{(\term{a},v)})}$ \Newline[0.5em]

$\forall \psi.\spec{H(\mathsf{ref}\term{(r,v)} \otimes \psi)}{\term{getref(r)}}{(\term{a}, \emptyset)}{H(\mathsf{ref}\term{(r,v)} \otimes \psi) \land \term{a = v}}$ \Newline[0.5em]

$\forall \psi.\spec{H(\mathsf{ref}\term{(r,-)} \otimes \psi)}{\term{setref(r,v)}}{(\term{a}, \emptyset)}{H(\mathsf{ref}\term{(r,v)} \otimes \psi)}$ \Newline[0.5em]

$\forall \psi.\spec{H(\psi)}{\term{cell(code)}}{(\term{a},\emptyset)}{H(\cellminus{\term{a}}{\term{code}} \otimes \psi)}$ \Newline[0.5em]

$\ready{\theta}{\term{c}}{\term{v}} \specimp \forall \psi.\setof{H(\theta \otimes \psi)}{\term{read(c)}}\{$\=$(\term{a},r).\;H(\theta \otimes \psi)$ \Newline
      \>$\land\;r = \setof{\term{c}} \land \term{a = v}\}$ 
\Newline[0.5em]
$\unready{\theta}{\term{c}} \specand \unready{\theta'}{\term{c}} \specand
 \mathsf{code}(\theta,\term{c},\term{code}) \specand $ \Newline
$\forall \psi.\spec{H(\theta \otimes \psi)}{\term{code}}{(\term{a},r)}{H(\theta' \otimes \Re(u, \psi)) \land \term{a = v} \land r = rs}$ \Newline
$\specimp$ $\forall \psi.$\=$\setof{H(\theta \otimes \psi)}$ \Newline
\> \term{read(c)} \Newline
\> $\{(\term{a},\setof{\term{c}}).\;$\=$H([\Re(\setof{c}, \theta')|\cellplus{\term{c}}{\term{code}}{\term{v}}{rs}]) \land \term{a = v}\}$
\end{tabbingspec}
}  
\caption{Library Specification}
\label{library-specification}  
\end{figure}

\section{The Implementation}

We have two cartesian closed categories in play, each of which is
represented a bit differently. Below, we give the interpretation 
of the types into our functional language. 
{\small
\begin{mathpar}
  \begin{array}{lcl}
    \interpu{1}           & = & \unittype \\
    \interpu{X \times Y}  & = & \interpu{X} \star \interpu{Y} \\
    \interpu{X \to Y}     & = & \interpu{X} \to \interpu{Y} \\
    \interpu{X \shrink Y} & = & \interpu{X} \to \interpu{Y} \\
    \interpu{A^\omega}     & = & \comp{\streamtype{\interps{A}}}
    \\[1em]
    \interps{I}           & = & \unittype \\
    \interps{A \otimes B} & = & \interps{A} \star \interps{B} \\
    \interps{A \lolli B}  & = & \streamtype{\interps{A}} \to \comp{\streamtype{\interps{B}}} \\
    \interps{A \lollishrink\, B}  & = & \streamtype{\interps{A}} \to \comp{\streamtype{\interps{B}}} \\
    \interps{\valtype{X}} & = & \interpu{X} 
    \\[1em]
    \streamtype{\tau}     & = & \celltype{\opttype{\tau}}
  \end{array}
\end{mathpar}
} 
Most of the types of ultrametric world can be represented using ML
types. One addition is that we have a type of strictly contractive
functions $A \shrink B$. This is represented in exactly the same way
as the ordinary ML function space, but must be kept isolated in order
to support the extra operation of fixed points.

The interesting case is at $A^\omega$, where we give the
representation of streams of values of type $A$.  This clause is
interpreted as $\comp{\interps{A}}$, which can be understood as
follows. In the category of ultrametric spaces, we want to understand
streams as pure values. However, we want to represent streams via
imperative data structures which change over time. This conflict is
resolved by representing a stream as a thunk, which constructs and
initalizes a stream data structure each time it is evaluated. So each
time we evaluate a value of type $A^\omega$, we construct a fresh
stream, ready to begin counting regardless of when it was created.

In contrast, the interpretation of the co-Kleisli category is one in
which time plays a very significant part. We implement it via a
dataflow graph which we update to get to the next time step. So the
implementation of functions is more complex. In our implementation,
the argument to a function $f : A \lolli B$ comes as a stream, and the
return value is term of type $\comp{\streamtype{B}}$. The code
constructor permits the implementation to read and extend the dataflow
graph, doing some initialization to return a stream cell. Surpisingly,
this result is a stream, and not a point value, the way that the
intepretation of morphisms in the co-Kleisli category might initially
suggest. In fact, our implementations actually realize the coextensions
of the morphisms of the co-Kleisli category, and the logical relation
needs to be adjusted in this clause to make it fit. 

One further observation is that we do not realize streams of type $\tau$
with a $\celltype{\tau}$. Instead, we use a $\celltype{\opttype{\tau}}$. 
The reason for this decision is that we want to implement recursion via
feedback, and so we need to be able to have cells which are unitialized
on their first time step. 

Selected parts of this implementation can be seen in
Figure~\ref{ultrametric-implementation}. It should be clear that for
products and exponentials, we are simply directly implementing it in
terms of the corresponding products and exponentials of our
programming language. Even the ultrametric fixed point \term{fix} :
$(X \shrink X) \to X$ simply calls out to the underlying fixed point
operator of the programming language. 

However, the implementations of \term{cons} and \term{fix'} are two of
the most complex functions in the whole library. The \term{cons}
function takes a value \term{x}, and then returns a function (in the
co-Kleisli category) which will append \term{x} to the front of any
input stream it receives. The complexity arises from the fact that
streams may or may not have a value on the first time step. If they
are unitialized, then \term{cons} should simply replace the first
$\None$ element with \term{x}, and if it is initialized, it should
buffer its input for the next time step. The trick is to use a local
reference cell as a one-place buffer. 

The $\term{fix'}$ operator implements recursion via feedback. It takes
a stream of contractive dataflow functions, and returns a stream
realizing the fixed point of the first element of this stream. To do
this, we allocate a reference \term{r} (initially set to $\None$), which
the cell \term{input} reads to produce its values. Then, we call \term{f}
with the input, and then construct a cell \term{output} which takes the
return values and writes them to \term{r} in order to prepare \term{input}
for the next time step. 

The implementation of (part of) the co-Kleisli category, given in
Figure~\ref{cokleisli-implementation} is very straightforward. Each
operation simply takes in a stream cell, and builds a cell to return
its return value. This is because all the difficulties have been
encapsulated into a single function: the \term{zip} operation (defined
in Figure~\ref{cokleisli-util}). Given two stream cells, it returns a
cell which pairs the successive elements of its two inputs. This is a
tricky function to verify, since the function must work with lagged
inputs, and we may receive a pair of inputs in which one component is
lagged and the other not. However, our mathematical specification does
not mention delays in it at all. So even if one input yields $(a_0,
a_1, a_2, \ldots)$ and the other yields $(\None, b_0, b_1, b_2,
\ldots)$, the programmer is still entitled to assume that $a_0$ is
paired with $b_0$, and that $a_1$ is paired with $b_1$, and so on.

To implement this, \term{zip} tests the two inputs, and introduces an
artificial delay, if one cell is delayed and the other is not.
Otherwise, it simply returns a cell which performs the pairing. Since
implementing a delay uses auxilliary state, we need to register the
cell --- but we only \term{register} the cell in the case it needs the
state. This reduces the number of cells that will get forced at the
end of each trip through the event loop, and so lets the dataflow
graph remain lazier.

Finally, in Figure~\ref{adjoint-implementation}, we give the
implementation of the functorial actions of the two adjoint functors,
which are straightforward. 

For space reasons, we have suppressed most of the definitions of the
combinators of the contractive operations. However, these are all 
the same as the implementation of the ordinary functions --- we simply
use them in restricted contexts.

\begin{figure}
{\small
\begin{tabbing}
\term{id = \fun{x}{x}} 
\\
\term{compose\;f\;g = \fun{x}{g(f(x))}}
\\
\term{one = \fun{x}{\unitval}}
\\
\term{fst = \fun{(x,y)}{x}}
\\
\term{snd = \fun{(x,y)}{y}}
\\
\term{pair\;f\;g = \fun{x}{(f(x),g(x))}}
\\
\term{eval = \fun{(f,x)}{f(x)}}
\\
\term{curry\;f = \fun{x}{\fun{y}{f(x,y)}}}
\\[0.5em]
\term{fix} : $(X \shrink X) \to X$ \\
\term{fix\; f = \fix{x:\interpu{X}}{f}}
\\[0.5em]

\term{cons} : $X \to (\valtype{X} \lollishrink \;\valtype{X})^\omega$ \\
\term{cons = \lambda x.\; cell(return(\Some{f}))} \\
with \term{f = \lambda ys.\;do} \= \term{r \leftarrow newref(\Some{x});} \\
\>     \term{zs \leftarrow cell(}\= 
      \term{do} \=\term{old \leftarrow getref(r);} \\
\> \> \> \term{new \leftarrow read(ys);} \\
\> \> \> \term{case\;old\;of} \\
\> \> \> \;\;\= \term{\None \to return(new)} \\
\> \> \> \>     \term{\Some{\_} \to do} \=\term{setref(r, new);} \\
\> \> \> \>  \>                           \term{return(old));} \\                 
\> \term{register(zs);} \\
\> \term{return(zs)}\\[0.5em]

% \term{head\;thunk =} \\
% \;\;\term{do\;}
%   \=\term{xss \leftarrow thunk;} \\
%   \>\term{xs' \leftarrow read(xss);} \\
%   \>\term{case\;xs'\;of} \\
%   \>\;\;\=\term{\Some{xs} \to return(xs)}\\
%   \>    \>\term{\None \to ERROR} (invariant ensures this case cannot happen) \\[0.5em]

\term{fix'} : $(A \lollishrink A)^\omega \to A^\omega$ \\
\term{fix' = \lambda thunk.\;do}
  \=\term{fs \leftarrow read(thunk);} \\
  \>\term{f' \leftarrow read(fs);} \\
  \>\term{case\;f'\;of}\\
  \>\;\;\=\term{\None \to ERROR} (provably never occurs)\\
  \>    \>\term{Some(f) \to} \\
  \>    \>\;\;\term{do} \=\term{r \leftarrow newref(\None);}\\
  \>    \>  \>\term{input \leftarrow cell(do\;}\=\term{\_ \leftarrow read(clock);}\\
  \>    \>  \>                                 \>\term{v \leftarrow getref(r);}\\
  \>    \>  \>                                 \>\term{return(v));}\\
  \>    \>  \>\term{pre \leftarrow f(input);}\\
  \>    \>  \>\term{out \leftarrow cell(do\;}\=\term{\_ \leftarrow read(clock);} \\
  \>    \>  \>                               \>\term{\_ \leftarrow read(input);} \\
  \>    \>  \>                               \>\term{v \leftarrow read(pre);}\\
  \>    \>  \>                               \>\term{setref(r, v);}\\
  \>    \>  \>                               \>\term{return(v));}\\
  \>    \>  \>\term{register(out);}\\
  \>    \>  \>\term{return(out)}
\end{tabbing}
}
\caption{The Implementation of the Ultrametric Category}
\label{ultrametric-implementation}
\end{figure}


\begin{figure}
{\small
\begin{tabbing}
\term{id = \lambda xs.\;cell(read\;xs)} 
\\[0.5em]

\term{compose\;f\;g =\lambda as.\; do\;}
 \=\term{bs \leftarrow f(as);} \\
 \>\term{cs \leftarrow g(bs);} \\
 \>\term{return(cs)} 
\\[0.5em]

\term{one\;xs = cell(return(\Some{\unitval}))}
\\[0.5em]

\term{pair\;f\;g = \lambda as.\;do\;}
  \=\term{bs \leftarrow f(as);} \\
  \>\term{cs \leftarrow g(as);} \\
  \>\term{zip(bs,cs)}
\\[0.5em]

\term{fst = \lambda abs.\;cell(do}
  \= \term{ab' \leftarrow read(abs);} \\
  \>\term{case \; ab'\; of} \\
  \>\term{\;\None \to return(\None)} \\
  \>\term{\;\Some{a,b} \to return(\Some{a}))} 
\\[0.5em]


\term{snd = \lambda abs.\;cell(do}
  \= \term{ab' \leftarrow read(abs);} \\
  \>\term{case \; ab'\; of} \\
  \>\term{\;\None \to return(\None)} \\
  \>\term{\;\Some{a,b} \to return(\Some{b}))} 
\\[0.5em]

\term{eval = \lambda fas.\;do}\;
              \=\term{fs \leftarrow \;fst(fas);}\\
              \>\term{as \leftarrow \;snd(fas);}\\
              \>\term{cell(do\;}\=\term{f' \leftarrow read(fs)}\\
              \>                  \>\term{case\;f'\;of}\\
              \>                  \>\term{\;\None \to return\;\None} \\
              \>                  \>\term{\;\Some{f} \to do\;}\=\term{bs \leftarrow f(as);} \\
              \>                  \>                            \>$\term{read(bs)})$ 
\\[0.5em]

\term{curry\;f =\lambda as.\;cell(\mathsf{Some}(\lambda bs.\;do\;}
  \=\term{abs \leftarrow zip(as,bs); f(abs)))}
\end{tabbing}
}
\caption{The Implementation of the co-Kleisli Category}
\label{cokleisli-implementation}
\end{figure}

\begin{figure}
{\small
\begin{tabbing}
\term{zip(as, bs) =} \\
\;\;\= \term{do\;}\=\term{a' \leftarrow read(as);} \\
\>              \>\term{b' \leftarrow read(bs);} \\
\>              \>\term{case \;(a',b')\;of}\\ 
\>           \>\;\;\=\term{(\None, \None)} \\
\>           \>    \>\term{(\Some{\_}, \Some{\_}) \to} \\
\>           \>    \> \qquad \term{cell(do\;}\=\term{a' \leftarrow read(as);} \\
\>           \>    \>                      \>\term{b' \leftarrow read(bs);} \\
\>           \>    \>                      \>\term{case \;(a',b')\; of} \\
\>           \>    \>                      \>\;\;\=\term{(\Some{a}, \Some{b}) \to return(\Some{(a,b)})} \\
\>           \>    \>                      \>\;\;\=\term{(\_, \_) \to return(\None))} \\
\>           \>    \>\term{(\None, \Some{\_}) \to}\\
\>           \>    \> \qquad\term{do\;}\=
                                      \term{r \leftarrow newref(\None);} \\
\>           \>    \>               \>\term{abs \leftarrow cell(do\;}\=\term{a \leftarrow read(as);} \\
\>           \>    \>               \>                               \>\term{new \leftarrow read(bs);} \\
\>           \>    \>               \>                               \>\term{old \leftarrow getref(r);}\\
\>           \>    \>               \>                               \>\term{setref(r, new);}\\
\>           \>    \>               \>                               \>\term{return(\Some{a,old}));}\\
\>           \>    \>               \>\term{register(abs);} \\
\>           \>    \>               \>\term{return(abs)} \\
\>           \>    \>\term{(\Some{\_}, \None) \to}\\
\>           \>    \> \qquad\term{do\;}\=
                                      \term{r \leftarrow newref(\None);} \\
\>           \>    \>               \>\term{abs \leftarrow cell(do\;}\=\term{b \leftarrow read(bs);} \\
\>           \>    \>               \>                               \>\term{new \leftarrow read(as);} \\
\>           \>    \>               \>                               \>\term{old \leftarrow getref(r);}\\
\>           \>    \>               \>                               \>\term{setref(r, new);}\\
\>           \>    \>               \>                               \>\term{return(\Some{old,b}));}\\
\>           \>    \>               \>\term{register(abs);} \\
\>           \>    \>               \>\term{return(abs)} 
\\[0.5em]

\term{register(xs) = do\;}\=\term{dummy \leftarrow read(xs);}\\
                          \>\term{lst \leftarrow getref(i);} \\
                          \>\term{setref(i, pack(xs) :: lst)} 
\end{tabbing}
}
\caption{Utility Functions}
\label{cokleisli-util}
\end{figure}




\begin{figure}
{\small
\begin{tabbing}
$(-)^\omega : U^S(A,B) \to U(A^\omega, B^\omega)$ \\
\term{omega\;f = \lambda athunk.\;do} \=\term{as \leftarrow athunk; f(as)} 
\\[0.5em]

$\mathit{Value} : U(X,Y) \to U^S(\valtype{X}, \valtype{Y})$ \\[0.2em]
\term{value\;f = \lambda xs.\;cell(do} \=\term{x' \leftarrow read(xs);}\\
                                       \>\term{case\;x'\;of}\\
                                       \>\;\;\=\term{\None \to return(\None)} \\
                                       \>\>    \term{\Some{x} \to return(\Some{f(x)}))}

  
\end{tabbing}
}
\caption{Implementing the Adjunction}
\label{adjoint-implementation}  
\end{figure}

\section{The Specification}

In Figures \ref{logical-relation} and \ref{satisfaction-relation}, we
give three mutually-dependent relations, one for ultrametric values,
one for co-Kleisli values, and one for the memory state of the
dataflow graph. 

The relation for ultrametric values, $U^d_A(v, \term{v})$, is given in
lines 1-6 of Figure~\ref{logical-relation}. It relates elements $v$ of
a metric space $A$ to a concrete program term \term{v}. Intuitively,
it can be read as saying ``the semantic value $v$ is approximated by
the computational value \term{v} to at least distance $d$''. The
clauses for unit and products are straightforward, and the cases for
the two function spaces are only slightly more complicated --- we
quantify over all greater distances (the future worlds in the Kripke
ordering) before asserting that applying related values to related
functions returns related results. However, even though the clauses
for the two function spaces look identical, they have different
semantics due to the fact that the mathematical inhabitants of $X
\shrink Y$ are contractive, which means that we know the fixed points
of $X \shrink X$ necessarily exist.

On lines 5-6, we give the relation for streams $A^\omega$. The
relation $\Build^{d'}_A(vs, code, \mu, \emptyset, \U)$ says that in
all well-formed memories, running \term{code} will build a fresh piece
of dataflow graph to represent the stream value $vs$, with no static
dependencies on any existing piece of the graph (the $\emptyset$) and
without any delays in the output (the $\U$). So no matter when we run
this code, we build streams which will all enumerate the same sequence
of mathematical values. Looking inside the definition of $\Build$,
we see that it takes a memory, and returns a memory in the extension
ordering, subject to several additional constraints --- the $\Update$
and $\Stable$ predicates. These ensure that the number of dangling
references and stateful cells unregistered with the update list 
do not increase, and that the return value is a genuine stream. 

Next, the relation $V^d_A(v, \term{v}, \mu, \sigma)$, defined on lines
7-17 of Figure~\ref{logical-relation}, relates values in the
co-Kleisli category of streams to the program terms \term{v}. The
relation $V^d_A(v, \term{v}, \mu, \sigma)$ has the intuitive reading
``the semantic value $v$ is approximated by computational value
\term{v} and memory state $\mu$ to at least distance $d$. Furthermore,
the use of \term{v} may involve evaluating the cells in $\sigma$''.

On line 7, the other side of the adjunction, $\valtype{X}$, is
interpreted by deferring to the $U$-relation.  Units and pairs (lines
8-9) continue to be interpreted simply, as before.  As a convenience
prior defining functions, on line 10 we add a clause in this relation
for streams, which are simply cells in the dataflow graph which will
produce the correct stream of values, and whose static dependencies
are less than what the relation asks for.

Then, on lines 11-13, we give the interpretation of the function
space.  This interpretation first clears its breath by quantifying
over extensions to the distance, the memory, and the dependencies, and
then says that if we have a dataflow cell \term{v} realizing some
stream $vs$, we will $\Build$ an output stream cell realizing
$f^\dagger\;vs$. Furthermore, the static dependencies of the result
are no worse than the inputs'. Finally, it says that the result will
be no more delayed than the input is (i.e., if the input is not
lagged, then the output won't be, but if the input is lagged, then the
output might or might not be). The definition of the contractive
function space $A \lollishrink B$ on lines 15-17 is similar, except that it
promises that its output will not be lagged, full stop. (This is what
justifies the implementation of the fixed point using a lagged input.)

A memory state is a pair $(\theta, R)$, where $\theta$ is a formula
describing a dataflow network, and $R$ is a \emph{rely} describing the
stream of semantic values each cell in the network is expected to
produce. To relate these two, we have a satisfaction relation
$\satisfy{(\theta, R)}{d}$, which can be read as saying, roughly, that
$\theta$ implements the streams in $R$ to distance $d$. This
satisfaction relation is the most complicated part of the definition,
since we have to account for all of the issues discussed in the
previous section. Our satisfaction relation is given in
Figure~\ref{satisfaction-relation}.
We specify a rely $R$ as 9-tuple:



% \subsection{Waffle}
% 
% \begin{enumerate}
% 
% \item Next, multiple streams may share the dataflow graph.
% 
%   If we have two cells $a$ and $b$, both of whose code reads a cell
%   $c$, then on a timestep in which we read both $a$ and $b$, one of
%   them will $c$ in a state when it has already been updated by the
%   other one. Therefore we need to specify what makes this interference
%   safe --- we need to say what it means for a subgraph to ``already
%   have been updated''. To do this, we must specify the values a term's
%   dependencies will take on as they are evaluated, so that our
%   specification can say that we are happy with a dataflow cell as long
%   as it produces the appropriate value. We call this the \emph{rely},
%   in analogy to rely-guarantee. Then, we can say that a state is
%   ``more defined'' than another one when more of its cells are ready,
%   and give a Kripke relation which asserts that an expression good in
%   one state is also good in all ``more defined'' states.
% 
%   Since we relate mathematical values to program values, and this
%   relation includes the memory, we are lead to the further
%   complication that stream cells in the dataflow graph may have types
%   which are larger than the type of the current clause of the logical
%   relation. Therefore, we cannot define our logical relation by
%   induction on the type structure --- we need to borrow some ideas
%   from step-indexed models, and define the rely mutually-recursively
%   with the logical relation.
% 
% \item We also build our dataflow graph incrementally and dynamically,
%   adding new cells even as current cells are evaluted. So we need to
%   know that operations which extend the graph will not break existing
%   code. So the Kripke relation on memories also needs to support a
%   notion of the dataflow graph getting bigger, in addition to getting
%   more defined.
% 
% \item Since we are implementing streams with stateful dataflow cells,
%   we need to ensure that there are no cyclic dependencies in the flow
%   graph. Therefore we need to ensure that a cell will not write to
%   anything which depends upon it, to prevent event storms.
% 
% \item Our dataflow graph is stateful: operations like accumulators and
%   fixed points are implemented with imperative references, which give
%   a form of state whose access does not create new dependencies.
% 
%   Furthermore, these references may need to updated by a collaboration
%   between dataflow cells. For example, in the code for \term{fix}, we
%   want to feed outputs at time $n$ as inputs at time $n+1$. We do this
%   with a reference cell which is read by the input, and written to by
%   the output cell. So the getter and the setter of the cell are
%   different, and the output should not write to the reference before
%   it knows the input has read it.
% 
% \item The next problem connected with state in the dataflow graphs is
%   that in order to correctly realize a stream, a cell with auxilliary
%   state must have been evaluated on every tick. For example, consider
%   a cell $c$ which realizes the stream 1, 2, 3, \ldots by maintaining
%   a counter which it updates. If the cell is not evaluated at a
%   particular time step, then the values it produces will fall behind
%   the global clock.
% 
%   However, not all cells need to be evaluated every time step. For
%   example, suppose we have a cell $c'$ which produces $2, 4, 6,
%   \ldots$ by reading $c$ and returning twice its value. In this case,
%   the computation at $c'$ is stateless, and so it does not need to be
%   evaluated except on the ticks when its value is actually needed.
% 
%   So our program should maintain a list of the state-modifying cells in
%   the dataflow graph, so that the event loop can evaluate the ones
%   that have yet to be read before it advances time by a step. Furthermore, 
%   every stateful cell also needs to register itself with the global clock
%   cell, so that when the clock is ticked, they will invalidate themselves
%   properly. 
% 
% \item Finally, when computing the fixed point of a stream function
%   $f$, we proceed by giving $f$ a dummy input on the first time step,
%   and giving the output at time $n$ as the input at time $n+1$. This
%   means that each stream object in our rely may also contain a delay
%   -- that is, it may lack a value (ie, be $\None$) at this time step,
%   and only contain values beginning at the next time step.
% \end{enumerate}

\begin{enumerate}
\item A finite set of stream cells and metric types $C$. 
\item An assignment of streams values $V_S : C \to \mathit{Value}^\omega$ 
  of the appropriate type to each cell.\footnote{We should write this 
  as a dependent product, with an element of $C$ having type 
  $\Sigma A:\mathrm{type}.\;\celltype{\opttype{\interps{A}}}$, and $V_S$ having
  the type $V_S : \Pi (A, \_) \in C.\; A^\omega$. However, we will suppress these 
  dependencies to reduce notational clutter.}
\item A ``delay flag'' $D : C \to \Delays$ which says for each cell whether 
  its output is delayed $\D$ or whether it is undelayed $\U$. 
\item A set of reference cells $L$. 
\item An assignment of a stream of values $V^L : L \to \mathit{(1 +
  \mathit{Value})^\omega}$ for each local reference. Note in
  particular that our local state is given as a stream of
  \emph{options}: this is because we might want to use some state for
  ``only a little while''. (The \term{cons} function does this when it
  is applied to a lagged cell. It will use the value in its reference cell
  on the first time step, and then never store a value in it again.)
\item A ``getter'' for each reference cell $G : L \rightharpoonup
  C$. This is the cell which will read that reference cell. This is a
  \emph{partial} function, so  there can be reference
  cells which are not yet going to be read by anyone. This lets us
  build up the dataflow graph incrementally, while still remaining
  within the rely.
\item A ``setter'' for each reference cell $S : L \rightharpoonup
  C$. This is the cell which has responsibility for writing the next
  value of the reference cell. Like the getter $G$, the getter $S$ is
  also partial. However, we require that its domain be a superset of $G$'s ---
  that is, we will always define getters before setters. 

\item A function $\Delta : C \to \powerset{C}$, giving the ``static
  dependencies'' of each cell. The idea is that for each cell
  \term{c}, then $\term{c'} \in \Delta(\term{c})$ tells us that
  \term{c'} is a cell in the \emph{current} heap which the evaluation
  of \term{c} may update. (So reading \term{c} may create new cells it
  depends on, but since they are not in the current heap they do not
  appear in $\Delta(\term{c})$.)  Viewed as a relation, the reflexive
  transitive closure of $\Delta$ must be a partial order, to ensure
  that there will be no nontrivial cycles in the dependency graph. We
  will write $\Delta^{*}(\term{c})$ to denote the cells reachable via
  the reflexive transitive closure of $\Delta$, and $\Delta^{+}(\term{c})$
  for the transitive closure.  

\item A function $\Delta^L : L \to \powerset{C}$, which describes the 
  static dependencies of the contents of each reference cell. The reason
  we need to track the dependencies of reference contents is that 
  cells may read references and use their contents, and so we need to 
  know what the dependencies for each value may be. 
\end{enumerate}

When we need to deal with multiple relies, we will name the
appropriate component using the name subscripted with the rely. So if
$R$ is a rely, then we will write $C_R$ for its cells, $V_R$ for the
values of the cells, and so on.

We equip relies with a partial order as follows. We say that $R
\sqsubseteq R'$, when $C_R \subseteq C_{R'}$ and $L_R \subseteq L_{R'}$, and
furthermore each function is extended pointwise. That is, if $c \in
\dom{R}$, then $V_{R'}(c) = V_{R}(c)$, and similarly for $D_S$, $L$,
$V_L$, $G$, $S$, and $\Delta$. (Note in particular that the static
dependencies for a given cell do not grow --- the extension order for
$\Delta$ is more stringent than simply extension of the partial order.)

Furthermore, for any rely, we can also define its tail
$\tail{R}$. First, the footprints of the heap are unchanged
$C_{\tail{R}} = C_R$ and $L_{\tail{R}}$. Second, the reference values
all go to their tails $V^L_{\tail{R}} =
\fun{\term{r}}{\tail{V^L_R(\term{r})}}$.  Third, the stream values for
cells go to their tails if they are not delayed, and remain unchanged
if they are. $V_{\tail{R}} = \fun{\term{c}}{\mathsf{if}\;D_R(\term{c})
  =
  \U\;\mathsf{then}\;\tail{V_R(\term{c})}\;\mathsf{else}\;V_R(\term{c})}$.
Fourth, the delay flag becomes $\U$ for all the cells, $D_{\tail{R}} = \fun{\term{c}}{\U}$. 
Finally, the other components --- $\Delta, \Delta^L, S, G$ --- remain unchanged. 

Given this, we can explain the satisfaction relation in
Figure~\ref{satisfaction-relation}. We say when a graph $\phi$
\emph{satisfies} a rely $R$ to distance $d$, (written
``$\satisfy{(\phi, R)}{d}$'') when:
\begin{itemize}
  \item The cells of the graph are $C_R$ plus the clock, and the references
    of the graph are equal to $L_R$ plus the imperative list \term{i}. This is lines 2-3. 
  \item Each cell in the graph is either ready or unready. (Line 4)
  \item Each stateful ready cell depends upon the \term{clock}. (Lines 5-6)
  \item The update list \term{i} has some of the state-handling cells. (Line 7)
  \item References realizes the head of their
    stream if their setter is unready, and the head of their tail
    stream otherwise. (Line 9)
  \item Each cell \term{c} in the graph realizes a stream of values corresponding
    to $V_R(\term{c})$, out to distance $d$. (Line 8)
\end{itemize}

Notice that the satisfaction relation \emph{does not require} the
contents of references without setters to realize the expected values
for those references. Similarly, the clause of the satisfaction
relation for the update list \term{i} is imprecise --- it only
requires a \emph{subset} of the state-accessing cells. This is a
deliberate design decision: the reason we make this choice is to let
incomplete networks be extensions of complete ones. This lets us use
our logical relation to say something the behavior of programs which
run in incomplete dataflow networks. This enables us to write programs
which evaluate some cells while building another part of a dataflow
network.

The way that we reconcile this with our desire to say that closed
programs always build complete, closed networks, is to require that
every cell and function in the relation must always ``make things
better''. That is, we never admit \emph{values} into our relation which
increase the number of references without setters, or which increase
the number of stateful references which do not appear in the update
list \term{i}. Then, since we start a program in a complete dataflow
graph, we can only procede to complete graphs. 

Having described the satisfaction relation and the extension ordering
for relies, we can now describe the extension ordering for memories
(i.e., pairs of a state formula and a rely). This is given in
Figure~\ref{memory-orderings}. On line 1, we introduce the set
$\Mem{d}$, which are just pairs of formulas and relies in the
satisfaction relation. Then, on lines 2-7, we describe what the
ordering for valid memories is. First, the relies must lie in the rely
extension ordering, and then the formulas must satisfy a number of
extra conditions. First, anything ready in the smaller state must
remain ready in the larger one. Second, all of the code in the cells
must be the same in the smaller memory and its extension.  Finally,
any reference which does not have a defined writer must be unchanged
(be the same physical program value) in the smaller and larger states.

Then, on lines 8-10, we describe the \emph{temporal} ordering of
states.  The idea is that the Kripke ordering seen so far describes
how a dataflow graph can change \emph{during} a time step, and the
temporal ordering describes how it changes \emph{upon} a tick of the
clock. The idea is that to advance time one step, we update the clock
to make it invalid. This propagates a wave of invalidations throughout
the dataflow network, leaving it ready to compute the values of the
next time step. Then, two memories are in the relation
$\futurestate{d}{n}{\mu'}{\mu}$ when $\mu'$ is a state that could lie
$n$ steps in the future of $\mu$. What this means is that if $n = 0$,
then $\mu'$ and $\mu$ merely need to be in the Kripke order (line 9).
However, if $n = m + 1$, then there needs to be a state $\mu_0$ which
(a) larger is in the Kripke order with respect to $\mu$, and whose
tail is $n$ steps away in time from $\mu'$. Here, $\tail{\theta, R}$
is $(\Re(\term{clock}, \theta), \tail{R})$, in accordance with the idea
that the event loop updates the clock to propagate this notification
out to the rest of the flow graph. Finally, one additional condition
we impose on this order is that we only tick on \emph{complete}
memories (i.e., ones in which all the references have getters and
setters). 

Now, we can finally describe the $\Stream^d_A(\term{v}, (\theta, R))$
clause on line 8 of Figure~\ref{satisfaction-relation}. It says that
for any $n$ less than the log of $1/(2 \times d)$, reading the stream
cell should return the $n$-th value of the stream. (Or, if the stream
is delayed, it should return $\None$ on the first timestep, and the
$n-1$st value of the stream at subsequent times.) It does this by
appeal to the $\Head^d_A(\term{v}, \mu)$ predicate, defined on lines
12-18. This predicate says that if we have a heap implementing the
dataflow network in $\theta$, then reading it should return the
appropriate value --- either the head of the stream, or $\None$,
depending on whether the stream is lagged or not. Regardless, the
updated cells in the network should either be in $\Delta(\term{v})$ or
new cells (this is the meaning of the $\Update$ predicate, defined on
27-29 of Figure~\ref{logical-relation}). Furthermore, reading \term{v}
should change neither the set of references lacking getter or setters,
nor the set of cells touching mutable state but not appearing in
\term{i} (this is the $\Stable$ predicate, defined on lines 30-38 of 
Figure~\ref{logical-relation}). 



\subsection{Correctness Proof}

First, we establish some basic properties of our logical relation. 

\begin{prop}{(Kripke Monotonicity)}
If $d' \geq d$, we have that
\begin{enumerate}
\item If $U^d_X(v, \term{v})$ then $U^{d'}_X(v, \term{v})$. 
\item If $V^d_A(v, \term{v}, \mu, \sigma)$ then $V^{d'}_A(v, \term{v}, \mu, \sigma)$. 
\item If $\satisfy{\mu}{d}$ then $\satisfy{\mu}{d'}$. 
\item If $V^d_A(v, \term{v}, \mu, \sigma)$ and $\betterstate{d}{\mu'}{\mu}$, then $V^d_A(v, \term{v}, \mu',\sigma)$.
\item If $V^d_A(v, \term{v}, \mu, \sigma)$ and $\sigma' \supseteq \sigma$, then $V^d_A(v, \term{v}, \mu,\sigma')$.
\end{enumerate}
\end{prop}

\begin{lemma}{(Approximation Lemma)}
\begin{enumerate}
  \item If $\forall d' > d.\; U^{d'}_A(v, \term{v})$ then $U^d_A(v, \term{v})$. 
  \item If $\forall d' > d.\; V^{d'}_A(v, \term{v}, \mu, \sigma)$ then $V^d_A(v, \term{v}, \mu, \sigma)$. 
  \item If $\forall d' > d.\; \satisfy{\mu}{d'}$ then $\satisfy{\mu}{d}$. 
\end{enumerate}
\end{lemma}

\begin{lemma}{(Induction Lemma)}
\begin{enumerate}
  \item If $\forall d' > 2\cdot d.\; U^{d'}_A(v, \term{v}) \implies U^{d'/2}_A(v, \term{v})$ then $U^d_A(v, \term{v})$. 
  \item If $\forall d' > 2\cdot d.\; V^{d'}_A(v, \term{v}, \mu) \implies V^{d'}_A(v, \term{v}, \mu)$ then $V^d_A(v, \term{v}, \mu)$. 
  \item If $\forall d' > 2\cdot d.\; \satisfy{\mu}{d'} \implies \satisfy{\mu}{d'/2}$ then $\satisfy{\mu}{d}$. 
\end{enumerate}
\end{lemma}

However, these properties are a little clunky to use directly. The
reason is that we often need to extend the memory in our proofs, and
that requires us to re-establish the satisfaction relation for each
cell in the heap with the extended memory. We can encapsulate this
pattern of reasoning with the following two lemmas. (We use the
notation $[f|x:v]$ to denote the function which extends $f$'s domain
by $x$ and gives it the value $v$ at that point.)

\begin{lemma}{(Reference Allocation)}
Suppose that $\satisfy{(\theta, R)}{d}$, and that $\theta' =
\theta \otimes \mathsf{ref}(\term{r,v})$. Let $R'$ be the same
as $R$, except that $L_{R'} = L_R \cup \setof{\term{r}}$, $V_{R'} = [V_R|\term{r}:vs]$ for some $vs$,
and $\Delta^L_{R'} = [\Delta^L_R|\term{r}:\kappa]$ for some $\kappa \subseteq C_R$.
Then we have that $\satisfy{(\theta', R')}{d}$. 
\end{lemma}

\begin{lemma}{(Cell Allocation)} \\
Suppose that $\satisfy{(\theta, R)}{d}$ and
$\satisfyext{(\theta,R)}{d}$ and let $\theta' = \theta \otimes
\cellminus{\term{c}}{\term{code}}$. Further suppose we have $R'
\sqsupseteq R$ such that $C_{R'} = C_R \cup \setof{\term{c}}$, $L_{R'}
= L_R$, $G_{R'} = [G_R | \term{r_i : c}]$ for some set of references
indexed by $I$, and $S_{R'} = [G_R | \term{r_j : c}]$ for some set of
references indexed by $J$.

Then if we can show that for all $d' > 2 \cdot d$,
$\satisfy{(\theta',R')}{d'}$ implies $\Stream^{d'/2}_A(\term{c},
(\theta', R'))$, we can conclude that $\satisfy{(\theta',R')}{d}$. 
\end{lemma}

These two lemmas are enough to prove the correctness of every
operation in our library except for allocating the inpute cell in the
definition of the fixed point operation. The reason for this is that
when we write back values into the reference cell, the static
dependencies of the values may include the input.  However, this is a
harmless dependency, since the input cell just dereferences a pointer
--- it doesn't read any other cells or do any other computation. So we
can prove a custom lemma just for this case.

\begin{lemma}{(Feedback Input Cell)}
Suppose that $\satisfy{(\theta, R)}{d}$. Then suppose that $\theta' =
\theta \otimes \mathsf{ref}(\term{r,\None}) \otimes \cellminus{\term{c}}{\term{code}}$, where
\term{code} is $\term{do \_ \leftarrow read(clock); !r}$. Furthermore, suppose we
have $R'$ such that 
 $C_{R'} = C_R \cup \setof{\term{c}}$,
 $V_{R'} = [V_R|\term{c}:vs]$,
 $D_{R'} = [D_R|\term{c}:\D]$,
 $\Delta_{R'} = [\Delta|\term{c}:C_{R'}]$ ,
 $L_{R'} = L_R \cup \setof{\term{r}}$,
 $V_{R'} = [V_R|\term{r}:(\None :: (\mathsf{map\;Some}\;vs)]$ ,
 $\Delta^L_{R'} = [\Delta^L_R|\term{r}:C_{R'}]$ ,
 $G_{R'} = [G_R|\term{r}:\term{c}]$, and
 $S_{R'} = S_R$.

Then we have that $\satisfy{(\theta', R')}{d}$. 
\end{lemma}
\noindent Now we have enough machinery to prove the correctness of the library.
Let $\mu_0$ be the least memory, with an empty sets of cells and local
references (except for the clock and the mutable cell list \term{i}). Then,
we can show that
\begin{theorem}
Define $\realize{\ultrametric}{X}{Y}{f}{\term{f}}$ to be $U^0_{X \To
  Y}(f, \term{f})$, and define
$\realize{\ultrametric^S}{A}{B}{g}{\term{g}}$ to be $V^0_{A \lolli
  B}(g, \term{g}, \mu_0, \emptyset)$. 
Then it follows that:
\begin{itemize}
\item If $\judgeu{\Gamma}{e}{X}$, then $\realize{\ultrametric}{\Gamma}{X}{\interp{\judgeu{\Gamma}{e}{X}}}{\interpu{\judgeu{\Gamma}{e}{X}}}$
\item If $\judgek{\Delta}{t}{A}$, then \\
$\realize{\ultrametric^S}{V(\Gamma) \otimes \Delta}{A}{\interp{\judgek{\Delta}{t}{A}}}{\interpu{\judgek{\Delta}{t}{A}}}$
\end{itemize}
\end{theorem}

We use the banana brackets on typing derivations to mean replacing the
categorical combinators of Figure~\ref{adjoint-semantics} with our
implementation combinators.
\begin{figure}
{\small
\begin{tabbingspec}
$U^d_1(\unitval, \unitval) = $ true 
\Newline[0.5em]

$U^d_{X \times Y}((x,y), (\term{x}, \term{y})) = U^d_X(x, \term{x}) \land U^d_Y(y, \term{y})$ 
\Newline[0.5em]

$U^d_{X \To Y}(f, \term{f}) = \forall d' > 2\cdot d, v, \term{v}.\; 
    U^{d'}_X(v, \term{v}) \implies U^{d'}_Y(f\;v, \term{f\;v})$ 
\Newline[0.5em]

$U^d_{X \shrink Y}(f, \term{f}) = \forall d' > 2\cdot d, v, \term{v}.\; 
    U^{d'}_X(v, \term{v}) \implies U^{d'}_Y(f\;v, \term{f\;v})$ 
\Newline[0.5em]

$U^d_{A^\omega}(vs, \term{code}) = $ \Newline
\;\;\= $\forall d' > 2\cdot d, \mu \in \Mem{d'}.\;\Build^{d'}_A(vs, \term{code}, \mu, \emptyset, \U)$ 
\Newline[1em]

$V^d_{\valtype{X}}(v, \term{v}, \mu, \sigma) = U^d_{X}(v, \term{v})$ 
\Newline[0.5em]

$V^d_I(\unitval, \unitval, \mu, \sigma) = $ true
\Newline[0.5em]

$V^d_{A \otimes B}((a,b), \term{(a,b)}, \mu, \sigma) = V^d_A(a, \term{a}, \mu, \sigma) \land V^d_B(b, \term{b}, \mu, \sigma)$
\Newline[0.5em]

$V^d_{S(A)}(vs, \term{v}, (\theta,R), \sigma) = V_R(\term{v}) = vs \land \Delta_R(\term{v}) \subseteq \sigma$ 
\Newline[0.5em]


$V^d_{A \lolli B}(f, \term{f}, \mu, \sigma) = $ \Newline
\> $\forall d' > 2\cdot d, \betterstate{d'}{\mu'}{\mu}, \sigma' \supseteq \sigma, v, \term{v}$ \Newline
\> \;\;\=$V^{d'}_{S(A)}(vs, \term{v}, \mu', \sigma') \implies $ \Newline
\>     \> \;\;$\exists L \sqsupseteq D_R(\term{v}).\; \Build^{d'}_{B}(f^\dagger\;vs, \term{f\;v}, \theta', \sigma' \cup \setof{\term{v}}, L)$
\Newline[0.5em]

$V^d_{A \lollishrink B}(f, \term{f}, \mu, \sigma) = $ \Newline
\> $\forall d' > 2\cdot d, \betterstate{d'}{\mu'}{\mu}, \sigma' \supseteq \sigma, v, \term{v}.$ \Newline
\> \> $V^{d'}_{S(A)}(vs, \term{v}, \mu', \sigma') \implies 
      \Build^{d'}_{B}(f^\dagger\;vs, \term{f\;v}, \mu', \sigma'  \cup \setof{\term{v}}, \U)$
\end{tabbingspec}
}
\caption{The Logical Relation}
\label{logical-relation}  
\end{figure}


\begin{figure}
{\small
\begin{tabbingspec}
$\Build^d_A(vs, \term{code}, (\theta,R) \;\mathrm{as}\;\mu, \sigma, L) = $ \Newline
\;\;\= $\{H(\theta) \land \satisfyext{(\theta,R)}{d} \}$ \Newline
\> $\term{code}$ \Newline
\> $\{(\term{a},\_).\;\exists$\=$\betterstate{d}{(\theta,R')\;\mathrm{as}\;\mu'}{\mu}, u.\; H(\theta') \land \term{a} \in C_{R'} - C_R \;\land$ \Newline
\> \;\;\= $\NewStream^d_A(vs, \term{a}, \mu, \mu', \sigma, L) \;\land$ \Newline
\> \> $\Update(u, \mu, \mu', \term{a}) \;\land$ \Newline
\> \> $\Stable(\mu, \mu')\}$
\Newline[0.5em]

$\NewStream^d_A(vs, \term{a}, (\theta,R), (\theta', R'), \sigma, L) = $\Newline
\> $D_{R'}(\term{a}) = L \land \Delta^{*}_{R'}(\term{a}) \cap C_R \subseteq \sigma \land 
    V^d_{S(A)}(vs, \term{a}, (\theta',R'), \Delta_{R'}(\term{a}))$
\Newline[0.5em]

$\Update(u, (\theta, R), (\theta', R'), \term{a}) = $ \Newline
\> $\forall \term{c}.\;\ready{\theta'}{c}{-} \land \unready{\theta}{c} \iff c \in u \;\land$ \Newline
\> $\forall \term{c} \in u.\; \term{c} \in \Delta^{*}_R(\term{a}) \vee \term{c} \in (C_{R'} - C_R)$ 
\Newline[0.5em]

$\Stable(\mu, \mu') = \StableRef(\mu, \mu') \land \StableImp(\mu, \mu')$ \Newline[1em]

$\StableRef((\theta,R), (\theta'R')) = $ \Newline
\> $\forall \term{r} \in L_{R'}.\; S_{R'}(\term{r}) \mbox{ undef} \iff \term{r} \in L_R \land  S_{R}(\term{r}) \mbox{ undef}$ and \Newline
\> $\forall \term{r} \in L_{R'}.\; G_{R'}(\term{r}) \mbox{ undef} \iff \term{r} \in L_R \land  G_{R}(\term{r}) \mbox{ undef}$ and \Newline
\> $\forall \term{r} \in L_{R} - \dom{G_R}.\;\exists \term{v}.\;\hasref{\theta}{\term{r}}{\term{v}} \land \hasref{\theta'}{\term{r}}{\term{v}}$
\Newline[0.5em]

$\StableImp((\theta, R), (\theta',R'))\} = $ \Newline
\> $\forall I,I'.\;\hasref{\theta}{\term{i}}{I} \land \hasref{\theta}{\term{i}}{I'} \implies $ \Newline
\>\;\;\= $[I' - S^{-1}_{R'}(L_{R'}) - G^{-1}_{R}(L_{R})] = [I - S^{-1}_{R}(L_{R}) - G^{-1}_{R}(L_{R})]$ \Newline
\> \> and $\forall \term{c} \in I'.\; \unready{\theta'}{\term{c}} \implies \term{c} \in I \land \unready{\theta}{c}$
\end{tabbingspec}
}
\caption{Auxilliary Relations}
\label{auxilliary-relations}  
\end{figure}

\begin{figure}
{\small
\begin{tabbingspec}
$\satisfy{(\theta,R)}{d} \triangleq$ \Newline
\;\;\=$\cells{\theta} = C_R \uplus \setof{\term{clock}}$ and \Newline
    \>$\refs{\theta} = L_R \uplus \setof{\term{i}}$ and \Newline
    \>$\forall \term{c}:A \in C_R.\;$\=$\unready{\theta}{\term{c}} \vee \exists \term{v}.\; $\=$\ready{\theta}{\term{c}}{\term{v}}$ and \Newline
    \>$\forall \term{c} \in G_R(L_R) \cup S_R(L_R).$ \Newline
    \> \> $\ready{\theta}{\term{c}}{-} \implies \term{clock} \in \mathsf{deps}(\theta, \term{c})$ and \Newline
    \>$\exists I.\;\hasref{\theta}{\term{i}}{I} \land I \subseteq (G_R(L_R) \cup S_R(L_R))$ and \Newline
    \>$\forall \term{c}:A \in C_R.\;\Stream^d_A(\term{c}, (\theta, R))$ and \Newline
    \>$\forall \term{r}:A \in L_R.\; \Local^d_A(\term{r}, (\theta, R))$ 
\Newline[0.5em]

$\Stream^d_A(\term{v}, \mu) = $ \Newline
\> $\forall d' > 2\cdot d, n \leq \log(1/d'), \futurestate{d'}{n}{\mu'}{\mu}.\; \Head^{d'\cdot 2^n}_A(\term{v}, \mu')$ \Newline[0.5em]

$\Head^d_A(\term{v}, (\theta,R)\;\mathrm{as}\;\mu) = $\Newline
\> $\{H(\theta) \land \satisfyext{\mu}{d}\} $\Newline
\> \term{read\;v} \Newline
\> $\{(\term{a},\_).\;\exists$\=$\betterstate{d}{(\theta', R')\;\mathrm{as}\;\mu'}{\mu}, u.\; H(\theta') \;\land$ \Newline
\> \> $\Ready^d_A(\term{v}, \term{a}, \mu') \;\land$ \Newline
\> \> $\Update(u, \mu, \mu', \term{v}) \;\land$ \Newline 
\> \> $\Stable(\mu, \mu') \}$ 
\Newline[0.5em]

$\Local^d_A(\term{r}, (\theta, R)) = $ \Newline
\> $\forall d' > 2\cdot d.\; S_R(\term{r})$ defined $\implies$ \Newline
\> \;\;\= $\unready{\theta}{S_R(\term{r})} \implies$ \Newline
\> \>\;\;\= $\Ref^{d'}_A(\head{V^R_L(r)}, \term{r}, \mu, \Delta^+_R(S_R(\term{r})))$ and \Newline
\> \> $\ready{\theta}{S_R(\term{r})}{-} \implies $\Newline
\> \> \> $\Ref^{d'}_A(\head{\tail{V^R_L(r)}}, \term{r}, \mu, \Delta^+_R(S_R(\term{r})))$ \Newline
\> \> \> $\;\land\; \exists \term{v}.\;\ready{\theta}{G_R(\term{r})}{\term{v}})$
\Newline[0.5em]


$\Ref^d_A(v, \term{r}, (\theta,R), \sigma) = $ \Newline
\> $\exists \term{v}.\;\hasref{\theta}{\term{r}}{\term{v}} \land \Opt^{d'}_A(v, \term{v}, (\theta,R), \sigma)$ 
\Newline[0.5em]

$\Ready^d_A(\term{c}, \term{v}, (\theta,R)) = $ \Newline
\> $\Opt^{d}_A(\mbox{if } D_R(\term{v}) \mbox{ then } \None \mbox{ else } \Some{V_R(\term{v})}, 
              \term{a}, (\theta,R), \Delta^+_R(\term{c}))$ 
\Newline[0.5em]


$\Opt^d_A(v, \term{v}, \mu, \sigma) = $ \Newline
\> $(v = \None \land \term{v} = \None) \;\vee$ \Newline
\> $(\exists v', \term{v'}.\; v = \Some{v'} \land \term{v} = \Some{\term{v'}} \land V^d_A(v', \term{v'}, \mu, \sigma))$ 
\Newline[0.5em]

$\satisfyext{(\theta,R)}{d} = $ \Newline
\> $\forall \term{r}:A \in L_R.\; S_R(\term{r}) \mbox{ undef} \implies \Ref^d_A(\head{V^L_R(\term{r})}, \term{r}, (\theta,R))$
\end{tabbingspec}
}
\caption{The Satisfaction Relation}
\label{satisfaction-relation}
\end{figure}

\begin{figure}
{\small
\begin{tabbingspec}
$\Mem{d} = \comprehend{(\theta,R)}{\satisfy{(\theta,R)}{d}}$ 
\Newline[0.5em]

$(\sqsupseteq^d) \subseteq \Mem{d} \times \Mem{d}$ \Newline[0.2em]

$\betterstate{d}{(\theta',R')}{(\theta,R)}$ iff \Newline
\;\;\= $R' \sqsupseteq R$ and \Newline
\> $\forall \term{c}:A \in C_R, \term{v}.\; \ready{\theta}{\term{c}}{\term{v}} \implies \ready{\theta'}{\term{c}}{\term{v}}$ and \Newline
\> $\forall \term{c}:A \in C_R, \term{code}.\; \mathsf{code}(\theta, \term{c}, \term{code}) \implies \mathsf{code}(\theta', \term{c}, \term{code})$ and \Newline
\> $\forall \term{r}\in E_R, \term{v}.\; \mathsf{ref}(\theta, \term{r}, \term{v}) \iff 
                                         \mathsf{ref}(\theta', \term{r}, \term{v}) $
\Newline[0.5em]

$\futurestate{d}{0}{\mu'}{\mu} \;\;\;\,= \betterstate{d}{\mu'}{\mu}$ \Newline
$\futurestate{d}{n+1}{\mu'}{\mu} = $ \Newline
\> $\exists \mu_0 \in \Mem{d}.\; \betterstate{d}{\mu_0}{\mu} \land \complete{\mu_0} \land \futurestate{2\cdot d}{n}{\mu'}{\tail{\mu_0}}$ 
\Newline[0.5em]

$\complete{\theta,R} = $ \Newline
\> $\forall \term{r} \in L_R. \exists \term{c} \in C_R.\; S_R(\term{r}) = \term{c} \land \exists \term{v}.\;\ready{\theta}{\term{c}}{\term{v}}$ 
\end{tabbingspec}
}
\caption{Orderings on Memories}
\label{memory-orderings}
\end{figure}

\section{Discussion}
We have used ultrametric spaces to reinterpret the stream transformers
familiar from the semantics of synchronous dataflow. In the special
case of functions from streams to streams, causality and
nonexpansiveness precisely coincide, but complete ultrametric spaces
are Cartesian closed, supporting function spaces at all orders, and a
general notion of contractiveness for defining well-founded fixed
points.  Furthermore, to support efficient implementation, we also
need to make use of the co-Kleisli category over the stream comonad,
which we connect to the base category via an adjunction.

%% There are four main strands of related work: synchronous dataflow
%% languages, purely functional reactive programming systems, imperative
%% FRP systems, and metric methods in denotational semantics.

%% The family of synchronous dataflow languages (such as
%% Esterel~\cite{esterel}, Lustre~\cite{lustre}, and Lucid
%% Synchrone~\cite{synchrone}) are languages based on a model of
%% synchronous time. The discrete ultrametric semantics we use is related
%% to these systems, most especially to Lucid Synchrone (which supports
%% higher-order functions).

Pouzet and Caspi~\cite{coiterative} extended synchronous dataflow
programming to higher order with their co-iterative semantics. They
illustrated how that this generated a Cartesian closed category (of
size-preserving functions), which they used to interpret
functions. Uustalu and Vene~\cite{essence-dataflow} subsequently
observed that size-preserving functions could be understood more
abstractly as the co-Kleisli category of streams. However, in both of
these works, feedback was handled in a somewhat \emph{ad hoc} fashion.

The proper treatment of feedback is delicate, and disentangling the
two main pieces of it kept us busy for a long time. First, we use
ultrametrics to give a semantic criterion for causality, which permits
us to avoid explicitly looking at the syntax of a program to identify
dependencies. Second, we needed to make explicit use of the adjunction
between the base category of ultrametric spaces and the co-Kleisli
category of streams, in order to explain the semantics of feedback.
By using two categories, we do not need to compromise on our reasoning
principles in any way, which is quite remarkable given the low-level,
imperative nature of our implementation.

Functional reactive programming was introduced by Elliott and
Hudak~\cite{fran}, and was given a semantics in terms of event streams
and unrestricted functions over them. In this and subsequent
work~\cite{courtney-thesis}, the semantics of fixed points were given
denotationally. This gives semantics to all FRP expressions, including
non-well-founded programs (which will go into infinite loops). 

A notable feature of FRP is a treatment of continuous
time. We believe that our proof framework should extend to proving an
sampling theorem as in Wan and Hudak~\cite{frp-first-principles}.  On
the semantic side, continuous streams can be modelled as functions
$\mathbb{R} \to A$, and the causal ultrametric extends naturally to
this case. On the implementation side, the \term{clock} can supply
time deltas (in contrast to its current delivery of pure ticks).

Due to the problem of space leaks, arrowized FRP~\cite{arrowized-frp}
was introduced in order to restrict the set of definable stream
transformers to the the causal ones. The restriction to arrows is
roughly equivalent to first-order functional programming, though
Nilsson~\emph{et al.}  introduced additional combinators to recover
higher-order and dynamic behavior. Our semantics gives a way of
eliminating these restrictions and admitting higher-order and dynamic
behavior in a very uniform way.

% reinstate Nivat sometime

Metric methods were introduced into semantics in the late
1970s and early 1980s, to simplify the
denotational semantics of
concurrency~\cite{concurrency-semantics}. The applications to stream
programming were recognized early, but not followed up on: in a
surprisingly little-cited 1985 paper~\cite{metric-dataflow}, de Bakker
and Kok proposed an ultrametric semantics for a language of
first-order stream programs over integers and wrote
``We think there are no problems when we allow functions of higher
order[\ldots]''.  This is a conjecture which we have attended to, a
full quarter-century later: we can confirm that it turns out to be
true, but only if we make it true twice over!

More recently, Birkedal and his coworkers~\cite{birkedal-ultrametrics}
have used ultrametric models to give logics and semantics for
sequential programs involving advanced features such as higher-order
state and polymorphism, and have shown connections between these
ideas and the more operationally-flavoured technique of step-indexed
models~\cite{appel-mcallester}. Our `custom' logical relation is very much in the spirit of the recursively-defined predicates and relations used in these metric models, and we additionally exploit metric structure to define recursive \emph{values}.

A fascinating and suggestive paper by Escardo~\cite{escardo-metric},
gives a metric model to PCF extended with timeout operators. Since
cancels and interrupt operations pervade interactive programs, this
suggests we should investigate whether they can be supported without
harming the reasoning principles of the language.

Finally, we have actually implemented the API of this paper as a
library in Objective Caml. After implementing the combinatory
interface as a module interface, we implemented syntax extensions
using CamlP4, which compile terms of the adjoint calculus into calls
to our Ocaml library. In addition, we have implemented bindings to the
GTK GUI toolkit, and are currently investigating proof principles
for reasoning about ``retained mode'' architectures (of
which GUI toolkits and the HTML DOM are two examples). 

% \acks

% Acknowledgments, if needed.

% We recommend abbrvnat bibliography style.

\bibliographystyle{abbrvnat}

% The bibliography should be embedded for final submission.

\bibliography{popl11-frpccc}


\end{document}
