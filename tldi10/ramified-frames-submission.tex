%-----------------------------------------------------------------------------
%
%               Template for LaTeX Class/Style File
%
% Name:         sigplanconf-template.tex
% Purpose:      A template for sigplanconf.cls, which is a LaTeX 2e class
%               file for SIGPLAN conference proceedings.
%
% Author:       Paul C. Anagnostopoulos
%               Windfall Software
%               978 371-2316
%               paul@windfall.com
%
% Created:      15 February 2005
%
%-----------------------------------------------------------------------------


\documentclass[natbib]{sigplanconf}

\usepackage{amsmath}
\usepackage{amssymb}
\usepackage{mathpartir}
\newcommand{\ctext}[1]{\mathsf{#1}}

\newcommand{\normalize}[1]{\ctext{nf}({#1})}

\newcommand{\unittype}{\mathbf{1}}
\newcommand{\reftype}[1]{\ctext{ref}\;#1}
\newcommand{\monad}[1]{\bigcirc{#1}}
\newcommand{\cont}[1]{\ctext{cont}\;{#1}}
\newcommand{\opttype}[1]{\ctext{option }{#1}}
\newcommand{\seqsort}[1]{\ctext{seq}\;{#1}}
\newcommand{\listtype}[1]{\ctext{list}\;{#1}}

\newcommand{\pair}[2]{\left<{#1},{#2}\right>}
\newcommand{\fst}[1]{\ctext{fst}\; #1}
\newcommand{\snd}[1]{\ctext{snd}\; #1}

\newcommand{\inj}[1]{\iota_{#1}}
\newcommand{\inl}[1]{\ctext{inl }#1}
\newcommand{\inr}[1]{\ctext{inr }#1}
\newcommand{\Case}[5]{\ctext{case}(#1,\; {#2}.\; #3,\; {#4}.\; #5)}
\newcommand{\listcase}[5]{\ctext{case}(#1,\;\ctext{Nil} \to {#2},\;
                                       \ctext{Cons}({#3},{#4}) \to {#5})}
\newcommand{\optcase}[4]{\ctext{case}(#1,\;
                                      \ctext{None} \to {#2},\;
                                      \ctext{Some}\;{#3} \to {#4})}
\newcommand{\z}{\ctext{z}}
\newcommand{\s}[1]{\ctext{s}(#1)}
\newcommand{\iter}[4]{\ctext{iter}(#1, {#2}, {#3}.\; {#4})}
\newcommand{\iterseq}[4]{\ctext{iter}_{\mathsf{seq}}(#1, {#2}, #3.\; {#4})}
\newcommand{\comp}[1]{[#1]}
\newcommand{\fun}[3]{\lambda #1:#2.\;#3}
\newcommand{\Fun}[3]{\Lambda #1:#2.\;#3}
\newcommand{\unit}{\left<\right>}
\newcommand{\pack}[2]{\ctext{pack}({#1}, {#2})}
\newcommand{\unpack}[4]{\ctext{unpack}({#1}, {#2}) = {#3} \;\ctext{in}\;{#4}}
\newcommand{\alt}{\;|\;}
\newcommand{\letv}[3]{\ctext{letv}\;#1 = #2\;\ctext{in}\;#3}
\newcommand{\newref}[2]{\ctext{new}_{#1}(#2)}
\newcommand{\run}[1]{\ctext{run}\;{#1}}
\newcommand{\ok}{\ctext{ ok}}
\newcommand{\FV}[1]{\mathrm{FV}({#1})}

\newcommand{\fix}[2]{\ctext{fix}\;{#1}.\;{#2}}

\newcommand{\statecfg}[2]{\left<{#1};\;{#2}\right>}
\newcommand{\eval}[4]{\left<{#1};\;{#2}\right> \leadsto \left<{#3};\;{#4}\right>}
\newcommand{\evalabort}[2]{\left<{#1};\;{#2}\right> \leadsto \mathbf{abort}}

\newcommand{\domain}[1]{\mbox{dom}({#1})}
\newcommand{\upset}[1]{\mathcal{P}^{\uparrow}({#1})}

\newcommand{\pointsto}{\mapsto}
\newcommand{\disj}{\vee}
\renewcommand{\implies}{\supset}
\newcommand{\wand}{-\!\!*\,}
\newcommand{\emp}{\mathsf{emp}}
\newcommand{\validprop}[1]{{#1}\;\ctext{valid}}

\newcommand{\todo}[1]{\texttt{[TODO: {#1}]}}

\newcommand{\setof}[1]{\{{#1}\}}

\newcommand{\To}{\Rightarrow}
\newcommand{\From}{\Leftarrow}

\newcommand{\N}{\mathbb{N}}

\newcommand{\assert}{\ctext{prop}}

\newcommand{\bigstep}{\Downarrow}

\newcommand{\bigeval}[4]{\left<{#1};\;{#2}\right> \bigstep \left<{#3};\;{#4}\right>}
\newcommand{\bigevalabort}[2]{\left<{#1};\;{#2}\right> \bigstep \mathbf{abort}}

\newcommand{\spec}[4]{\{{#1}\}{#2}\{{#3}.\;{#4}\}}
\newcommand{\specX}[3]{\{{#1}\}{#2}\{{#3}\}}
\newcommand{\mspec}[4]{\langle{#1}\rangle{#2}\langle{#3}.\;{#4}\rangle}
\newcommand{\bnfalt}{\;\;|\;\;}

\newcommand{\specor}{\;||\;}
\newcommand{\specand}{\;\&\;}
\newcommand{\specimp}{\Rightarrow\!\!\!>}
% \newcommand{\specimp}{\Longrightarrow}
% \newcommand{\specor}{\ctext{ or }}
% \newcommand{\specand}{\ctext{ and }}
% \newcommand{\specimp}{\ctext{ implies }}
\newcommand{\spectype}{\ctext{spec}}
\newcommand{\valid}{\ctext{ valid}}

\newcommand{\interp}[1]{[\![{#1}]\!]}
\newcommand{\interpE}[1]{\interp{#1}^e}
\newcommand{\interpC}[1]{\interp{#1}^c}

\newcommand{\interpF}[1]{[\![{#1}]\!]_f}
\newcommand{\interpmono}[1]{\interp{#1}^{\mathrm{m}}}

\newcommand{\entails}{\models}

\newcommand{\judgeE}[4][\Theta]{{#1};\;{#2} \vdash {#3} : {#4}}
\newcommand{\judgeC}[4][\Theta]{{#1};\;{#2} \vdash {#3} \div {#4}}
\newcommand{\judgeEq}[5][\Theta]{{#1};\;{#2} \vdash {#3} \equiv {#4} : {#5}}
\newcommand{\judgeEqC}[5][\Theta]{{#1};{#2} \vdash {#3} \equiv {#4} \div {#5}}




%% semantic operations

\newcommand{\worldleq}{\preceq}
\newcommand{\worldgeq}{\succeq}

\newcommand{\semfun}[2]{\lambda #1.\;#2}
\newcommand{\sempair}[2]{\left({#1}, {#2}\right)}
\newcommand{\powerset}[1]{\mathcal{P}(#1)}
\newcommand{\powersetfin}[1]{\mathcal{P}^{\mathrm{fin}}(#1)}

\newcommand{\paircat}[2]{\left<{#1};{#2}\right>}
\newcommand{\sumcat}[2]{\left[{#1};{#2}\right]}
\newcommand{\abscat}[1]{\lambda({#1})}


\newcommand{\judgeP}[3]{{#1} \vdash {#2} : {#3}}
\newcommand{\judgeS}[2][\Delta]{{#1} \vartriangleright {#2} : \spectype}

\newcommand{\judgeSCtx}[2]{{#1} \vartriangleright {#2} : \ctext{context}}

\newcommand{\judgeEqP}[4]{{#1} \vdash {#2} \equiv {#3} : {#4}}


\newcommand{\judgeWK}[3][\Theta]{{#1} \vdash {#2} : {#3}}
\newcommand{\judgeKeq}[4][\Theta]{{#1} \vdash {#2} \equiv {#3} : {#4}}

\newcommand{\entailsP}[3]{{#1} \vartriangleright {#2} \vdash {#3}}
\newcommand{\entailsS}[3]{{#1}; {#2} \vdash {#3} \ok}

\newcommand{\chartp}{\ctext{char}}
\newcommand{\fonttp}{\ctext{font}}

\newcommand{\LOC}{loc}
\newcommand{\MONO}{\mathbf{mono}}
\newcommand{\HEAP}{heap}
\newcommand{\PROP}{Prop}
\newcommand{\TRUE}{True}
\newcommand{\HPROP}{Prop}
\newcommand{\TYPE}{Type}

\newtheorem{prop}{Proposition}
\newtheorem{lemma}{Lemma}

\newcommand{\comprehend}[2]{\setof{{#1}\;|\;{#2}}}


% \newenvironment{proof}{\begin{comment}}{\end{comment}}

\newenvironment{proof}{\noindent\textbf{Proof.}}{\noindent\ensuremath{\Box}}

\newcounter{prooflinenum}
\newenvironment{tabbedproof}
   {\setcounter{prooflinenum}{0}
    \begin{tabbing}\;\;\=\;\;\;\;\;\=\;\;\;\;\=\;\;\;\;\=\;\;\;\;\=\;\;\;\;\=\;\;\;\;\=\;\;\;\;\=\;\;\;\;\=\\[-2em]}
   {\end{tabbing}}

\newcommand{\oo}{\addtocounter{prooflinenum}{1}\arabic{prooflinenum}\>\>}
\newcommand{\ooo}{\addtocounter{prooflinenum}{1}\arabic{prooflinenum}\>\>\>}
\newcommand{\oooo}{\addtocounter{prooflinenum}{1}\arabic{prooflinenum}\>\>\>\>}
\newcommand{\ooooo}{\addtocounter{prooflinenum}{1}\arabic{prooflinenum}\>\>\>\>\>}
\newcommand{\oooooo}{\addtocounter{prooflinenum}{1}\arabic{prooflinenum}\>\>\>\>\>\>}
\newcommand{\ooooooo}{\addtocounter{prooflinenum}{1}\arabic{prooflinenum}\>\>\>\>\>\>\>}
\newcommand{\oooooooo}{\addtocounter{prooflinenum}{1}\arabic{prooflinenum}\>\>\>\>\>\>\>\>}
\newcommand{\ooooooooo}{\addtocounter{prooflinenum}{1}\arabic{prooflinenum}\>\>\>\>\>\>\>\>\>}

\newcommand{\ox}{\>\>}
\newcommand{\oox}{\>\>\>}
\newcommand{\ooox}{\>\>\>\>}
\newcommand{\oooox}{\>\>\>\>\>}
\newcommand{\ooooox}{\>\>\>\>\>\>}
\newcommand{\oooooox}{\>\>\>\>\>\>\>}
\newcommand{\ooooooox}{\>\>\>\>\>\>\>\>}
\newcommand{\oooooooox}{\>\>\>\>\>\>\>\>\>}


\newenvironment{eqnproof}[1][]{${#1}$\begin{displaymath}\begin{array}{lcll}}
                         {\end{array}\end{displaymath}}

\newcommand{\eline}[3][]{{#1} & = & {#2} & \mbox{{#3}} \\}

\newcommand{\elines}[3][]{{#1} & = & \begin{array}{l} #2 \end{array} & \mbox{{#3}} \\}

\newcommand{\eclaim}[3][]{{#1} &  & {#2} & \mbox{{#3}} \\}
\newcommand{\efact}[2]{{#1} & & & \mbox{#2} \\}

\newcommand{\basicspec}[4]{[{#1}]\;{#2}\;[{#3}.\;{#4}]}

% Macros for type checking assertions

\newcommand{\pfun}[3]{\hat{\lambda} #1:#2.\;#3}
\newcommand{\restrictkind}[1]{({#1})\Downarrow_K}
\newcommand{\restricttype}[1]{({#1})\Downarrow_T}

\newcommand{\restricttyenv}[2]{{#2}\Downarrow^{#1}_K}
\newcommand{\restrictvals}[2]{{#2}\Downarrow^{#1}_T}
\newcommand{\judgeACtx}[1]{\vartriangleright {#1} \;\ctext{ok}}
\newcommand{\judgeA}[3][\Delta]{{#1} \vartriangleright {#2} : {#3}}
\newcommand{\judgeSort}[2][\Delta]{\judgeA[{#1}]{{#2}}{\ctext{sort}}}
\newcommand{\judgeSortEq}[3][\Delta]{{#1} \vartriangleright {#2} \equiv {#3} : \ctext{sort}}

\newcommand{\ms}[1]{\mathcal{#1}}
\newcommand{\Frame}[2]{{#1} \otimes {#2}}


\begin{document}

\conferenceinfo{TLDI '10}{January 23, 2010, Madrid, Spain.} 
\copyrightyear{2010} 
\copyrightdata{978-1-60558-891-9/10/01} 

\titlebanner{preprint}        % These are ignored unless
\preprintfooter{short description of paper}   % 'preprint' option specified.

\title{Verifying Event-Driven Programs using Ramified Frame Properties}
% \subtitle{Subtitle Text, if any}

\authorinfo{Neelakantan R. Krishnaswami}
           {Microsoft Research}
           {neelk@microsoft.com}
\authorinfo{Lars Birkedal}
           {IT University of Copenhagen}
           {birkedal@itu.dk}
\authorinfo{Jonathan Aldrich}
           {Carnegie Mellon University}
           {jonathan.aldrich@cs.cmu.edu}
% \authorinfo{John C. Reynolds}
%            {Carnegie Mellon University}
%            {jcr@cs.cmu.edu}
\maketitle

\begin{abstract}
Interactive programs, such as GUIs or spreadsheets, often maintain
dependency information over dynamically-created networks of objects.
That is, each imperative object tracks not only the objects its own
invariant depends on, but also all of the objects which depend upon
it, in order to notify them when it changes.

These bidirectional linkages pose a serious challenge to verification,
because their correctness relies upon a global invariant over the
object graph. %  --- all inbound and outbound links must match. 
% Consequently, simple local reasoning principles such as the
% frame rule of separation logic 
% (which permit ignoring any heap outside
% the footprint of a command) 
% no longer suffice for modular
% verification, which makes it challenging to verify different parts of a
% dependency network separately and combine their correctness proofs.

We show how to \emph{modularly} verify programs written
using dynamically-generated bidirectional dependency information. The
critical idea is to distinguish between the footprint of a command,
and the state whose invariants depends upon the footprint. To do so,
we define an application-specific semantics of updates, and introduce
the concept of a \emph{ramification operator} to explain how local
changes can alter our knowledge of the rest of the heap.
We illustrate the applicability of this style of proof with a case
study from functional reactive programming, and formally justify
reasoning about an extremely imperative implementation as if it were
pure.
\end{abstract}

\category{F.3.1}{Logics and Meanings of Programs}{Specifying and Verifying and Reasoning about Programs}

\terms
languages, verification

\keywords
separation logic, frame rule, ramification problem, dataflow, 
functional reactive programming, subject-observer

\section{Introduction}

In many interactive programs, there are mutable data structures which
change over time, and which must maintain some relationships with one
another. For example, in a spreadsheet, each cell contains a formula,
which may refer to other cells, and whenever the user changes a cell,
all of the cells which transitively depend upon it must be
updated. Since spreadsheets can get very large, this should ideally be
done in a lazy way, so that only the cells visible on the screen (and
the cells necessary to compute them) are themselves recomputed.

Typically, these dependencies are written using the
\emph{subject-observer} pattern. A mutable data structure (the
subject) maintains a list of all of the data structures whose
invariants depend upon it (the observers). Whenever it changes, it
calls a function on each observer to update it in response
to the change. (And in turn, the observers of the subject may be
subjects of still other observers.)

While natural, these programs are very challenging to verify in a
modular way, even when using a resource-sensitive logic adapted to
reasoning about aliased mutable data, such as separation logic. The
reason is that there are two directions of dependency, both of which
matter for program proof. First, our program invariant must have
ownership over the subject's data (its \emph{footprint}) in order to
prove the correctness of code modifying the subject. This direction
of ownership is natural to verify with separation logic. 

However, we must explicitly maintain the \emph{other} direction of
dependency as well --- we track everything which depends upon the
subject, and modify them appropriately whenever the subject changes.
Hence, the natural program invariant now becomes a global property:
we need to know the full dependency graph covering all subjects and
observers to express that the reads and is-read-by relations
are relational transposes of one another. The global nature of this
invariant means that a naive correctness proof will not respect the
modular structure of the program --- if we modify the dependency graph
in any way, we now have to re-verify the entire program!

But, the intention of the subject-observer pattern is
precisely to allow the program to remain oblivious to the exact number
and nature of the observers, so that the programmer may add new
observers without disturbing the behavior of the rest of the program.
Our goal, then, is to find a way of taking this piece of practical
software engineering wisdom, and casting it into formal terms amenable
to proof.

Concretely, our contributions are as follows: 

\begin{itemize}
  \item We define a library with a monadic API for writing
    demand-driven computations with dynamic dependencies and local
    state. This library is implemented using higher-order functions
    dynamically creating networks of imperative callbacks.

    We then give an ``abstract semantics'' for this
    library, structured as a set of separation logic lemmas about our
    dataflow library. These lemmas permit \emph{modular} correctness
    proofs about programs using this API, even in the face of the fact
    that the program invariants must be defined globally upon the
    whole callback network.

    The key idea is to distinguish between the direct footprint of a
    command, and the program state which depends upon that
    footprint. The lemmas are then phrased so that they refer only to
    the direct footprint of each command in the API. In addition, we
    structure our lemmas to justify an unusual frame property for our
    abstract semantics, which we can use to verify different parts of
    an imperative dataflow network separately.

    Unlike typical frame properties, the frame in our frame rule is
    not the same in the pre- and the post-states. Instead, the two
    sides of the frame are related by a \emph{ramification operator}
    (so named in analogy to the ``ramification problem'' in AI), which
    explains how local changes can alter our knowledge of the rest of
    the heap.

  \item To illustrate the utility of this proof technique, we verify
    an imperative implementation of combinators implementing stream
    transducers in the style of functional reactive programming. 

    Ultimately, clients can reason about the behavior of the imperative
    implementation as if it were purely functional, even though it
    is implemented using local state and imperative callback
    procedures.
\end{itemize}
%
For space reasons, many of the detailed proofs have been omitted from
this extended abstract, but we emphasize that full proofs have been
carried out; in particular, each triple in the library specification
has been proven using the specification logic in
Section~\ref{sec:logic}.

\section{Programming Language and Logic}
\label{sec:logic}
The formal system we present has three layers. First, we have a core
programming language we call Idealized ML. It is a
predicatively-polymorphic functional language which isolates all side
effects inside a monadic type~\cite{pfenning-davies}. Our notion of
side effects includes nontermination in addition to the allocation,
access, and modification of general references (including pointers to
closures).  Then, we give an assertion language based on higher-order
separation logic~\cite{hosl} to describe the state of a
heap. Separation logic allows us to give a clean treatment of issues
related to specifying and controlling aliasing, and higher-order
predicates allow us to abstract over the heap, hiding the exact layout
of a module's heap data structures and thereby enforcing
encapsulation.  Finally, we have a specification logic to describe the
effects of programs, which is a first-order logic whose atomic
propositions are Hoare triples $\spec{p}{c}{a:A}{q}$, which assert
that if the heap is in a state described by the assertion $p$, then
executing the command $c$ will result in a postcondition state $q$
(with the variable $a$ referring to the return value of the command).


\textbf{Programming Language.} The core programming language we have
formalized is an extension of the polymorphic lambda calculus with a
monadic type constructor to represent side-effecting computations.
The types of our language are the unit type $1$, the function space $A
\to B$, inductive types like the natural number type $\N$, the reference type
$\reftype{A}$, as well as universal and existential types $\forall
\alpha:\kappa.\;A$ and $\exists \alpha:\kappa.\;A$.\footnote{These
  quantifiers are actually all restricted to \emph{predicative}
  quantification (i.e., they can only be instantiated with terms
  lacking any quantifiers themselves) in order to keep the
  denotational semantics simple, though recent
  work~\cite{birkedal-impred} has studied how to combine store with
  impredicative polymorphism.}

In addition, we have the monadic type $\monad{A}$, which is the type
of suspended side-effecting computations producing values of type
$A$. Side effects include both heap effects (such as reading, writing,
or allocating a reference) and nontermination.

We maintain such a strong distinction between pure and impure code for
two reasons. First, it allows us to use strong equational reasoning
principles for our language: we can validate the full $\beta$ and
$\eta$ rules of the lambda calculus for terms of non-monadic types,
such as functions, sums, and products. These rules simplify reasoning
even about imperative programs, because we can relatively freely
restructure the program source to follow the logical structure of a
proof. Second, when program expressions appear in assertions --- that
is, the pre- and post-conditions of Hoare triples --- they must be
pure. However, allowing a rich set of program expressions like
function calls or arithmetic in assertions makes it much easier to
write specifications. So we restrict which types can contain
side-effects, and thereby satisfy both requirements.

The pure terms of the language are typed with a typing judgment
$\judgeE{\Gamma}{e}{A}$, which can be read as ``In the type context
$\Theta$ and the variable context $\Gamma$, the pure expression $e$
has type $A$.'' Computations are typed with the judgment
$\judgeC{\Gamma}{c}{A}$, which can be read as ``In the type context
$\Theta$ and the variable context $\Gamma$, the computation $c$ is
well-typed at type $A$ .'' The rules for both of these judgments are
standard and omitted.

We have $\unit$ as the inhabitant of $1$, natural numbers $\z$ and
$\s{e}$, and functions $\fun{x}{A}{e}$. We also have the corresponding
eliminations for each type, including projections for products and
case statements for sum types. For the natural numbers, we add a
primitive iteration construct $\iter{e}{e_z}{x}{e_s}$. If $e = \z$,
this computes $e_z$ , and if $e = \s{e'}$, it computes
$e_s[(\iter{e'}{e_z}{x}{e_s})/x]$. Bounded iteration allows us
to implement (for example) arithmetic operations as pure expressions.
We will also freely make use of other polynomial data types (such as
sum types, lists, option types, and trees) as needed. 

Suspended computations $\comp{c}$ inhabit the monadic type
$\monad{A}$.  These computations are not immediately evaluated, which
allows us to embed them into the pure part of the programming
language. Furthermore, we can take fixed points $\fix{x:D}{e}$ of
terms, to give us a general recursion. Because we wish to permit
nonterminating programs only at monadic types, we must restrict
$\ctext{fix}$ to a limited family of types $D$ (given in
Figure~\ref{lang-syntax}), so that we do not contaminate our language
with infinite loops at every type.\footnote{The allowed types are those whose
interpretations are pointed CPOs in domain theory.}  We will
write recursive functions as syntactic sugar for $\ctext{fix}$.

The computations themselves include all expressions $e$, as
computations that coincidentally have no side-effects. Furthermore, we
have sequential composition $\letv{x}{e}{c}$. Intuitively, the
behavior of this command is as follows. We evaluate $e$ until we get
some $\comp{c'}$, and then evaluate $c'$, modifying the heap and
binding the return value to $x$. Then, in this augmented environment,
we run $c$. The fact that monadic commands have return values explains
why our sequential composition is also a binding construct. Finally,
we have primitive computations $\newref{A}{e}$, $!e$, and $e := e'$,
which let us allocate, read and write references (inhabiting type
$\reftype{A}$), respectively. To save space, we will also write
$\ctext{run}\;e$, when $e$ is a term of monadic type, as an
abbreviation for $\letv{x}{e}{x}$. Consider the following example,
which creates and swaps the contents of two references:
\begin{tabbing}
1 \qquad \= $\letv{r}{\comp{\newref{\N}{5}}}{}$ \\
2 \> $\letv{s}{\comp{\newref{\N}{14}}}{}$ \\
3 \> $\letv{x}{\comp{!r}}{}$ \\
4 \> $\letv{y}{\comp{!s}}{}$ \\
5 \> $\letv{\_}{\comp{r := y}}{}$ \\
6 \> $\letv{\_}{\comp{s := x}}{}$ \\
7 \> $x + y$ \\
\end{tabbing}
On lines 1 and 2, we allocate two references to natural numbers, $r$
and $s$, initializing them with the contents $5$ and $14$. Then, on
lines 3 and 4, we dereference $r$ and $s$, binding their contents to
the variables $x$ and $y$, respectively. Then, in lines 5 and 6, we
swap the contents of the two references, assigning $y$ to $r$ and $x$
to $s$, so that $r$ now points to $14$ and $s$ now points to $5$. 
Finally, on line 7, we return $x+y$, the sum of the two contents.

The primitive commands are all composed using the $\letv{x}{e}{c}$
construct. Because this form expects the bound term to be an
expression of monadic type, each of the primitive commands
(allocation, dereference, assignment) are wrapped in a suspension
$\comp{c}$ before being given to the let-binder. 

\begin{figure}
{\small
\begin{displaymath}
  \begin{array}{lcll}
    \mbox{Kinds} & 
      \kappa & ::= & \star \bnfalt \kappa \to \kappa 
    \\[0.5em]
     \mbox{Monotypes} & 
      \tau & ::= & 
         \unittype \bnfalt 
         \tau \times \tau \bnfalt 
         \tau \to \tau \\
     &&& \N \bnfalt 
         \reftype{A} \bnfalt
         \monad{\tau} \bnfalt \\
     &&& \alpha \bnfalt
         \tau\;\tau \bnfalt 
         \fun{\alpha}{\kappa}{\tau} 
    \\[0.5em]
    \mbox{Polytypes} & 
      A & ::= & 
         \unittype \bnfalt 
         A \times B \bnfalt 
         A \to B \\
    &&&  \N \bnfalt 
         \reftype{A} \bnfalt
         \monad{A} \bnfalt \\
    &&&  \alpha \bnfalt
         \tau\;\tau \bnfalt \\
    &&&  \forall \alpha:\kappa.\; A \bnfalt 
         \exists \alpha:\kappa.\; A \\[0.5em]
    \mbox{Type Contexts} & 
      \Theta & ::= & \cdot \bnfalt \Theta, \alpha:\kappa \\
  \end{array}
\end{displaymath}
}
\caption{Language Types}
\label{type-syntax}
\end{figure}


\begin{figure}
{\small
\begin{displaymath}
  \begin{array}{llcl}
    \mbox{Pure expressions} & 
     e & ::= & 
         \unit \bnfalt
         \pair{e}{e'} \bnfalt
         \fst{e} \bnfalt
         \snd{e} 
\\
     &&|& x \bnfalt \fun{x}{A}{e} \bnfalt e\;e' 
\\ 
     &&|& \z \bnfalt 
          \s{e} \bnfalt 
          \iter{e}{e_0}{x}{e_1}
\\ 
     &&|& \Fun{\alpha}{\kappa}{e} \bnfalt e\;\tau 
\\ 
     &&|& \pack{\tau}{e} \bnfalt \unpack{\alpha}{x}{e}{e'} 
\\
     &&|& \comp{c} \bnfalt \fix{x:D}{e}
\\[0.5em]
  \mbox{Computations} & 
    c & ::= & e \bnfalt \letv{x}{e}{c} \\
   &  &  |  & \newref{A}{e} \bnfalt !e \bnfalt e := e'
\\[0.5em]
  \mbox{Contexts} & 
    \Gamma & ::= & \cdot \bnfalt \Gamma, x:A 
\\[0.5em]
  \mbox{Pointed Types} & 
     D & ::= & \unittype \bnfalt \monad{A} \bnfalt D \times D \bnfalt A \to D  \\
    &  &  |  & \forall \alpha:\kappa.\; D 
\\[0.5em] 
  \end{array}
\end{displaymath}
}
\caption{Syntax of the Programming Language}
\label{lang-syntax}
\end{figure}


\textbf{Assertion Language.} The sorts and syntax of the assertion
language are given in Figure~\ref{assert-syntax}. The assertion
language is a version of separation logic~\cite{sep-logic}, extended
to higher order.

In ordinary Hoare logic, a predicate describes a set of program states
(in our case, heaps), and a conjunction like $p \land q$ means that a
heap in $p \land q$ is in the set described by $p$ and the set described
by $q$. While this is a natural approach, aliasing can become quite
difficult to treat --- if $x$ and $y$ are pointer variables, we need
to explicitly state whether they alias or not. So as the number of
variables in a program grows, the number of aliasing conditions grows
quadratically.

With separation logic, we add the \emph{spatial} connectives to
address this difficulty. A separating conjunction $p * q$ means that
the state can be broken into two \emph{disjoint} parts, one of which
is in the set described by $p$, and the other of which is in the
set described by $q$. The disjointness property makes the
noninterference of $p$ and $q$ implicit, letting us avoid the unwanted
quadratic growth in the size of our assertions. In addition to the
separating conjunction, we have its unit $\emp$, which is true of the
empty heap, and the points-to relation $e \pointsto e'$, which holds
of the one-element heap in which the value of the reference $e$ has 
contents equal to the value of $e'$. 

The universal and existential quantifiers $\forall x:\omega.\;p$ and
$\exists x:\omega.\;p$ are higher-order quantifiers ranging over all
sorts $\omega$. The sorts include the language types $A$, kinds $\kappa$, the sort of
propositions $\assert$, and function spaces over sorts $\omega \To
\omega'$. Constructors for terms of all these sorts are
given in Figure~\ref{assert-syntax}. For the function space, we
include lambda-abstraction and application. Because our assertion
language contains within it the classical higher-order logic of sets,
we will freely make use of features like subsets, indexed sums, and
indexed products, exploiting their definability.

Finally, we include the atomic formulas $\validprop{S}$, which are
\emph{assertions} that a \emph{specification} $S$ holds. This facility
is useful when we write assertions about pointers to code --- for
example, the assertion $r \pointsto e$ $\land$
$(\validprop{\spec{p}{\runcmd{e}}{a:A}{q}})$ says that the reference $r$ points
to a monadic expression $e$, whose behavior is described by the Hoare
triple $\spec{p}{\runcmd{e}}{a:A}{q}$.


\begin{figure}
{\small
\begin{displaymath}
\begin{array}{llcl}
\mbox{Assertion Sorts} & 
\omega & ::= & A \bnfalt \kappa \bnfalt \omega \To \omega \bnfalt \assert 
\\[0.5em]
\mbox{Assertion} & 
p & ::= & e \bnfalt A \bnfalt x \bnfalt \fun{x}{\omega}{p} \bnfalt p\;q \\
\mbox{Constructors}
& &  |  & \top \bnfalt p \land q \bnfalt p \implies q 
          \bnfalt \bot \bnfalt p \vee q \\
& &  |  &  \emp \bnfalt p * q  \bnfalt e \pointsto e' \\
& &  |  & \forall x:\omega.\; p \bnfalt \exists x:\omega.\; p \bnfalt
          \validprop{S} 
\\[0.5em]
\mbox{Specifications} &
S & ::= & \spec{p}{c}{a:A}{q}  \bnfalt \setof{p} \\
& &  |  & S \specand S' \bnfalt S \specimp S' \bnfalt S \specor S' \\
& &  |  & \forall x:\omega.\; S \bnfalt \exists x:\omega.\;S 
\\
\end{array}
\end{displaymath}
}
\caption{Syntax of Assertions and Specifications}
\label{assert-syntax}
\end{figure}

\textbf{Specification Language.} Given programs and assertions about
the heap, we need specifications to relate the two. We begin with the
Hoare triple $\spec{p}{c}{a:A}{q}$. This specification represents the
claim that if we run the computation $c$ in any heap the predicate $p$
describes, then if $c$ terminates, it will end in a heap described by
the predicate $q$. Since monadic computations can return a value in
addition to having side-effects, we add the binder $a:A$ to the third
clause of the triple to let us name and use the return value in the
postcondition. 

We then treat Hoare triples as one of the atomic proposition forms of
a first-order intuitionistic logic (see
Figure~\ref{assert-syntax}). The other form of atomic proposition are
the specifications $\setof{p}$, which are \emph{specifications} saying that
an \emph{assertion} $p$ is true. These formulas are useful for
expressing aliasing relations between defined predicates, without
necessarily revealing the implementations. In addition, we can form
specifications with conjunction, disjunction, implication, and
universal and existential quantification over the sorts of the
assertion language. 

With a full logic of triples at our disposal, we can express program
modules as formulas of the specification logic. We can expose a module
to a client as a collection of existentially quantified functions and
variables, and provide the client with Hoare triples describing the
behavior of those functions. Furthermore, modules can existentially
quantify over predicates to grant client programs access to module
state without revealing the actual implementation. A client program
that uses an existentially quantified specification cannot depend on
the concrete implementation of this module, since the existential
quantifier hides that from it.

The language and specification logic have been given a denotational
semantics in the first author's forthcoming PhD
thesis~\cite{tech-report}. We do not give the semantics here both for
space reasons, and because it is not central to the contributions of
this paper. 

\section{Demand-Driven Notification Networks}

A simple intuition for a ``demand-driven notification network'' is to
think of it as a generalized spreadsheet. We have a collection of
cells, each of which contains a program expression whose evaluation may
read other cells. When a cell is read, the expression within the
cell is evaluated, recursively triggering the evaluation of other
cells as they are read by the program expression. Furthermore, each
cell memoizes its expression, so that repeated reads of the same cell
will not trigger re-evaluation.

In addition, we can modify the code expression within a cell.  Any
time a cell is updated, its memoized value is cancelled, so that any
future reads of that cell will force the evaluation of its new
code. Furthermore, every cell also maintains a set tracking every
other cell which has read it, so that when its code is updated, it can
notify all of its readers --- i.e., every other cell whose value may
depend upon it -- to invalidate their own memoized values. 

We will call the entire collection of cells a ``notification
network'', because we have a graph (i.e., network) structure of cells,
which maintains dependency information between themselves, and
whenever a change is made to a cell, it notifies everything that
depends on it of the change.

In this section, we will describe an implementation of a notification
network library, informally explaining its design and how it
works. Then, in the next section, we will see how to take the informal
explanation and turn it into a precise specification suitable for
verification.

\subsection{Implementing Notification Networks}

\begin{figure}
{\small
\begin{specification}
\nextlinelabel{notify:codetype}
$\codetype :\; \star \to \star$ \nextline
$\codetype{\alpha} = \monad{(\alpha \times \cellset)}$ 
\nextlinelabel[0.5em]{notify:celltype}
$\celltype :\; \star \to \star$ \nextline
$\celltype{\alpha} = \{$\=$code: \reftype{\codetype{\alpha}};$ \nextline
                   \>$value: \reftype{\opttype{\alpha}};$ \nextline
                   \>$reads: \reftype{\cellset};$ \nextline
                   \>$obs:   \reftype{\cellset};$ \nextline
                   \>$unique:    \N\}$ 

\nextlinelabel[0.5em]{notify:ecell}
$ecell = \exists \alpha:\star.\; \celltype{\alpha}$ 

\nextlinelabel[0.5em]{notify:return}
$\return : \forall \alpha:\star.\; \alpha \to \codetype{\alpha}$ \nextline
$\return\;\alpha\;x = \comp{\pair{x}{\ctext{emptyset}}}$ 

\nextlinelabel[0.5em]{notify:bind}
$\ctext{bind} : \forall \alpha,\beta:\star.\; \codetype{\alpha} \to (\alpha \to \codetype{\beta}) \to \codetype{\beta}$ \nextline
$\ctext{bind}\;\alpha\;\beta\;e\;f = [$\=$\letv{(v,r_1)}{e}{}$ \nextline
                        \>$\letv{(v',r_2)}{f\;v}{}$ \nextline
                        \>$\;\pair{v'}{\ctext{union}\;r_1\;r_2}]$

\nextlinelabel[0.5em]{notify:read}
$\ctext{read} : \forall \alpha:\star.\; \celltype{\alpha} \to \codetype{\alpha}$ \nextline
$\ctext{read}\;\alpha\;a = [$\=$\letv{o}{\comp{!a.value}}{}$ \nextline
                     \>$\ctext{run}\;\ctext{case}(o,$ \nextline
                     \>\qquad\= $\ctext{Some}\;v \to \comp{\pair{v}{\ctext{singleton}\;a}},$ \nextline
                     \>      \> $\ctext{None}$\=$ \to$ \nextline
                     \>      \>               \>$[$\=$\letv{exp}{\comp{!a.code}}{}$ \nextline
                     \>      \>                     \>$\letv{(v,r)}{exp}{}$ \nextline
                     \>      \>                     \>$\letv{\_}{\comp{a.value := \Some(v)}}{}$\nextline
                     \>      \>                     \>$\letv{\_}{\comp{a.reads := r}}{}$ \nextline
                     \>      \>                     \>$\letv{\_}{\ctext{iterset}\;(\ctext{add\_observer}\;\pack{\alpha}{a})\;r}{}$ \nextline
                     \>      \>                     \>$\;\pair{v}{\ctext{singleton}\;a}])$ 

\nextlinelabel[0.5em]{notify:getref}
$\getref : \forall \alpha : \reftype{\alpha} \to \codetype{\alpha}$ \nextline
$\getref \alpha\;r = \comp{\letv{v}{\comp{!r}}{\pair{v}{\ctext{emptyset}}}}$ 

\nextlinelabel[0.5em]{notify:setref}
$\setref : \forall \alpha : \reftype{\alpha} \to \alpha \to \codetype{\unittype}$ \nextline
$\setref \alpha\;r\;v = \comp{\letv{\_}{\comp{r := v}}{\pair{\unit}{\ctext{emptyset}}}}$ 

\nextlinelabel[0.5em]{notify:newcell}
$\newcell : \forall \alpha:\star.\; \codetype{\alpha} \to \monad{\celltype{\alpha}}$ \nextline
$\newcell\;\alpha\;code = [$\=$\letv{unique}{!counter}{}$ \nextline
                       \>$\letv{\_}{\comp{counter := unique + 1}}{}$ \nextline
                                   \>$\letv{code}{\newref{\codetype{\alpha}}{code}}{}$ \nextline
                                   \>$\letv{value}{\newref{\opttype{\alpha}}{\ctext{None}}}{}$ \nextline
                                   \>$\letv{reads}{\newref{\cellset}{\ctext{emptyset}}}{}$ \nextline
                                   \>$\letv{obs}{\newref{\cellset}{\ctext{emptyset}}}{}$ \nextline
                                   \>$\; (code, value, reads, obs, unique)]$ 

\nextlinelabel[0.5em]{notify:update}
$\updatecell : \forall \alpha:\star.\; \codetype{\alpha} \to \celltype{\alpha} \to \monad{\unittype}$\nextline
$\updatecell\;\alpha\;exp\;a = 
     [$\=$\letv{\_}{\ctext{mark\_unready}\;\pack{\alpha}{a}}{}$ \nextline
       \>$a.code := exp]$ 

\nextlinelabel[0.5em]{notify:mark-unready}
$\ctext{mark\_unready} : ecell \to \monad{\unittype}$ \nextline
$\ctext{mark\_unready}\;cell =\; $\=$\ctext{unpack}(\alpha, a) = cell\; \ctext{in}$\nextline
\>  $[$\=$\letv{os}{\comp{!a.obs}}{}$ \nextline
\>     \>$\letv{rs}{\comp{!a.reads}}{}$ \nextline
\>     \>$\letv{\_}{\ctext{iterset}\;\ctext{mark\_unready}\;os}{}$ \nextline
\>     \>$\letv{\_}{\ctext{iterset}\;(\ctext{remove\_obs}\;cell)\;rs}{}$ \nextline
\>     \>$\letv{\_}{\comp{a.value := \ctext{None}}}{}$ \nextline
\>     \>$\letv{\_}{\comp{a.reads := \ctext{emptyset}}}{}$ \nextline
\>     \>$a.obs   := \ctext{emptyset}]$ \nextline[0.5em]

$\ctext{add\_observer} : ecell  \to ecell \to \monad{1}$\nextline
$\ctext{add\_observer}\;a\;\pack{\beta}{b} = [$\=
 $\letv{os}{\comp{!b.obs}}{}$ \nextline
\> $b.obs := \addset\;os\;a]$ \nextline[0.5em]

$\ctext{remove\_obs} : ecell \to ecell \to \monad{1}$\nextline
$\ctext{remove\_obs}\; a\;\pack{\beta}{b} = [$\=
  $\letv{os}{\comp{!b.obs}}{}$ \nextline
\>$b.obs := \ctext{removeset}\;os\;a]$ 
\end{specification}
}
\caption{Implementation of Notification Networks}
\label{notification-implementation}
\end{figure}

Our API for creating notification networks is given in
Figure~\ref{notification-implementation}. First, we'll describe
the interface, and then discuss its implementation. 

The interface exposes two basic abstract data types, $\ctext{cell}$
and $\ctext{code}$.

The type $\celltype{\alpha}$ (defined on line~\ref{notify:celltype}) is the
type of dynamic data values. A cell contains a reference to a piece of
code, a possible memoized value, plus enough information to correctly
invalidate its memoized value when the cell's dependencies change. We
can create a new cell by calling $\newcell \alpha\; e$, which returns a brand
new cell with the code expression $e$ inside it. We can also modify a
cell with the command $\updatecell\; \alpha\;cell\;e$, which modifies the
cell $cell$ by installing the new expression $e$ in it.

The type $\codetype{\alpha}$ (defined on line~\ref{notify:codetype}) is a
monadic type, representing the type of computations that can read
cells. It supports the usual operations $\return \alpha\;e$ and $\bind \alpha\;\beta\;e\;
(\semfun{x}{e'})$, which embed a pure value into the $\ctext{code}$
type and implement sequential composition, respectively. In addition,
the primitive operations on this monad include reading a cell with the
$\readcell \alpha\;cell$ function call, and reading and modifying local state
with the $\getref \alpha\;r$ and $\setref \alpha\;r\;v$ operations.

This monadic type is not the state monad of the programming language;
it is a user-level monadic type we implement as a library (as is
commonly done in Haskell, for example), in order to support the
transparent propagation and maintenance of dependency information.
For example, assuming that $a$ and $b$ are variables of type
$\celltype{\N}$, then the following code expression will read the two
cells and return their sum (suppressing obvious type arguments to
functions):
\begin{specification}
\nextline  $\bind \;(\readcell a) \;(\lambda x:\N.\;$ 
\nextline  $\bind \;(\readcell b) \;(\lambda y:\N.\;$ 
\nextline  $\;\;\return(x+y)))$ 
\end{specification}
This expression does not explicitly mention any dependency
information; it is up to the primitive operations of our library to
generate it, and up to the $\return$ and $\bind$ operations to propagate it
appropriately. In this way, we can (1) avoid the error-prone business
of explicitly managing the dependencies, and (2) we can use typing to
forbid invoking arbitrary stateful operations that might ruin  the
invariants of the library. The only state manipulations we perform are 
the ones under our control. 
 
The actual implementation is also given in
Figure~\ref{notification-implementation}. The abstract type of code is
implemented using the underlying monad of imperative commands, so that
$\codetype{\tau}$ is implemented with the type $\monad{(\tau \times
  \cellset)}$.  The intuition is that when we evaluate a term we are
allowed to read some cells along the way, and so we must return a set of
all the cells that we read in order to do proper dependency
management. So, $\cellset$ is a type representing sets of
cells\footnote{Properly, these are sets of cells packed inside an
  existential type -- i.e., terms of type $\exists \alpha:\star.\;
  \celltype{\alpha}$}. (The precise specification of $\cellset$ is
given in Appendix A, since describing it is a distraction from the
main development.)

Cells, defined on line~\ref{notify:celltype}, are represented with a
5-tuple. (We take the liberty of using record syntax for this tuple.)
We have a field $code$, which is a reference pointing to the code
expression, as well as a field $value$ which is a pointer to an
optional value. The value field's contents will be $\ctext{None}$ if
the cell is in an unready, un-memoized state, and will be
$\ctext{Some}\;v$ if the cell's code has already been evaluated to a
value $v$. In addition there are two fields representing the
dependencies. If the code expression has been evaluated and a memoized
value generated, then the $reads$ field will point to the set of cells
that the computation directly read while computing its
value. Otherwise it will point to the empty set. Conversely, the field
$obs$ contains the cell's observers --- the set of cells that have
read the current cell as part of their own computations. Obviously,
this is only non-empty when the cell has been evaluated. Finally, each
cell also has a numeric field $unique$, which is a unique numeric
identifier for each cell created by the dependency management library
in Figure~\ref{notification-implementation}. It allows us to compare
cells (even of different type) for equality, which we need to
implement the $\cellset$ type.

The $\return$ operation (defined on line~\ref{notify:return}) for the
library simply returns its argument value and the empty set, since it
does not read any cells. Likewise, $\bind \alpha\;\beta\;e\;f$ (defined on
line~\ref{notify:bind}) will evaluate the argument $e$ and pass the
returned value to the function $f$. It will then return the function's
return value, together with the union of the two read sets. 

There are two functions $\getref \alpha\;r$ (line~\ref{notify:getref})
and $\setref \alpha\;r\;v$ (line~\ref{notify:setref}) whose
specifications say that they simply read and update their argument
reference. These two functions allow us to use local state within a
notification network, which we will need to implement things like
accumulators when implementing reactive programs. They both return
empty read sets, since neither of them read any cells.

Interesting things first happen with the $\readcell \alpha\;e$ operation,
defined on line~\ref{notify:read}. This function will first check
to see if the cell has a memoized value ready. If it does, we return
that immediately. Otherwise, we evaluate the cell's code, and update
the current cell's value and read set. In addition, each cell that was
read in the evaluation of the code (i.e., the set returned as the
second component of the monadic type's return value) also has its
observer set updated with the newly-ready current cell. Now, if any of
the dependencies change, they will be able to invalidate the current
cell, which observes them. Note that the dependencies between cells
are all dynamic --- we cannot examine the inside of a code expression
to find its ``free cells'', and so we rely upon the invariant that a
code expression will return every cell it read, in addition to its
return value.

Further interesting things happen with the $\newcell \alpha\; e$
operation, defined on line~\ref{notify:newcell}.  It creates a new
cell value, initializing the code field with the argument $e$, and
generating a unique id by dereferencing and incrementing the variable
$counter$. The $counter$ variable occurs freely in this definition,
because it is a piece of state global to this module, which must be
initialized by whatever initialization routine first constructs the
whole module as an existential package.  Since $counter$ is otherwise
private, we can generate unique identifiers by incrementing it as we
create new cells.

Finally, the $\updatecell\;cell\;e$ operation (line~\ref{notify:update}) updates a cell
$cell$ with a new code expression $e$. (As an aside, it's worth
noting that this is a genuine, unavoidable, use of higher-order store:
we make use of pointers to code, including the ability to dynamically
modify them.) Once we modify a cell, any memoized value it has is no
longer necessarily correct.

Therefore, we have to drop the memoized value of the cell, and any
cell that transitively observes the cell. The $\ctext{mark\_unready}$
function (line~\ref{notify:mark-unready}) does this. Given a cell, it
takes all of the observers of the current cell and recursively makes
all of them unready. Then it removes the current cell from the
observer sets of all the cells it reads, and then it nulls out the
current cell's memoized value, as well as setting its read and
observer sets to empty. Notice that there is no explicit base case to
the recursive call; if there are any cycles in the dependency graph,
invalidation could go into an infinite loop.

So far, we have described the implementation invariants incrementally.
Before proceeding to describe them formally, we will state them again
informally, all in one place:

\begin{itemize}
  \item Every cell must have a unique numeric identifier
  \item Every cell is either ready, or unready. 
  \item Every ready cell has a memoized value, and maintains 
    two sets, one containing every cell that it reads, and the
    other containing every cell it is observed by. 
  \item Every unready cell has no memoized value, and has 
    both an empty read set and an empty observer set. 
  \item The overall dependency graph among the valid cells must form a
    directed acyclic graph. 
  \item The reads and the observers must be the same, only 
    pointing in opposite directions.
\end{itemize}

Formalizing these constraints is relatively straightforward, but we
have the problem that these constraints are global in nature:
we cannot be sure that the dependency graph is acyclic without having
it all available to examine, and likewise we cannot in general know
that a cell is in the read set of everything in its observed set
without knowing the whole graph. Handling this difficulty is one of
the primary contributions of this work. 

\section{The Abstract Semantics of Notifications}

We will formalize the informal invariants of the previous section in
three stages. In the first stage, we will describe how to accurately
formalize the global invariant of the cell graph, albeit in a
non-modular way. In the second stage, we will recover a basic modular
reasoning principle through an interesting use of polymorphism, which
will suffice to let us reason modularly about adding cells to and
modifying the cells in the cell graph.  However, this will not be
strong enough to reason modularly about \emph{evaluating} cells in the
network, and so in the third stage we will introduce a generalized
frame rule, which we will call a ``ramified frame rule'' after a
similar concept in AI.

\subsection{The Structure of the Global Invariant}

The key to getting around our difficulties lies in the difference
between the implementation of $\updatecell$ and of $\ctext{read}$.
The $\updatecell$ function calls $\ctext{mark\_unready}$, which 
recursively follows the observers. The $\ctext{read}$ function, on the 
other hand, proceeds in the opposite direction --- it evaluates code expressions,
recursively descending into the footprint of its command. The opposite
direction these two functions look is why we end up needing a global
invariant: we need to know that these two directions are in harmony
with one another.

Now, note that we have given the type of $\ctext{mark\_unready}$ the
monadic type $\monad{\unittype}$. This precludes it from being called
from within a $\codetype{\tau}$, because the user-level monadic type 
discipline of $\codetype{\tau}$ will only let us use compute with pure 
expressions and other $\codetype{\sigma}$ terms. 
Therefore, when we evaluate a code expression, we will never actually
follow the observer fields -- we will only add entries to them
whenever we evaluate a cell and change it from unready to ready.  As a
result, an abstract description of the heap which \emph{does not
  explicitly mention the observer sets} will prove sufficient for
reasoning about the behavior of $\codetype{\tau}$ expressions. 

With this plan, we introduce \emph{abstract heap formulas}, which are
syntactic descriptions of the state of part of the cell heap. These
syntactic expressions are given by the following grammar:
\begin{displaymath}
  \begin{array}{lcl}
    \phi,\psi & ::= & I \bnfalt \phi \otimes \psi \bnfalt \cellpos{a}{e}{v}{r} \bnfalt \cellneg{a}{e}\\
         &  |  & \localref{r}{v} 
  \end{array}
\end{displaymath}

Informally, a formula $I$ represents an empty abstract heap, and a
formula $\phi \otimes \psi$ represents an abstract heap that can be
broken into two disjoint parts $\phi$ and $\psi$. We will only
consider formulas modulo the associativity and commutativity of
$\otimes$, and take $I$ to be the unit of this binary operator.

The atomic form $\localref{r}{v}$ says that $r$ is a piece of local
state owned by the network, currently with value $v$. There are two
atomic forms representing cells. $\cellneg{a}{e}$ says that $a$ is a
cell with code $e$, which is unready to deliver a value --- it needs
to be re-evaluated before it can yield a value. $\cellpos{a}{e}{v}{r}$
says that $a$ is a cell with code $e$. Furthermore, it is ready to
deliver the value $v$, but only if all the cells in its read set $r$
are themselves ready. Otherwise, if anything in $a$'s read set $r$ is 
unready, then $a$ is unready itself. (Because we will sometimes want to
write $\celleither{a}{e}$ when we do not care whether $a$ is ready or
not, the $\cellneg{a}{e}$ formula has two dummy argument positions.)

First, notice the must/may flavor of this reading. The formula
$\cellneg{a}{e}$ says that $a$ \emph{must} be unready.  The formula
$\cellpos{a}{e}{v}{r}$ says that $a$ \emph{may} be ready, conditional
on the readiness of the elements of its read set $r$. Second, notice
that the backwards dependencies are entirely missing from these
formulas. We have simply left out the other half of the dependency
graph from this description. Forgetting this information will let us
begin to regain for local reasoning, as we will see in the statements
of Propositions 1 and 2. 

We have emphasized that the straightforward invariant is not obviously
modular. To elaborate upon this point, we will need to look at the
formal statement of the heap invariant, to see how exactly modularity
fails. We introduce the predicate $G(\phi)$. This predicate describes
the entire heap of cells allocated by our library and ensures they
satisfy the conditions described at the end of the previous
section. It also enforces the additional constraint that the cell heap
agree with $\phi$.\footnote{This is why we insisted that the abstract
  heap formulas are syntactic objects --- this permits us to define
  predicates on them by induction over the structure of the formula.}
\begin{tabbing}
    $G(\phi) \triangleq \exists H \in CellHeap.\; Inv(H, \phi)$ \\ [0.5em]

$Inv(H, \phi) \triangleq $ \\
\;\;\= $$\=$ R_H^\dagger = O_H \land R_H^+ \mbox{ strict partial order }$ \\
    \> \> $\land\; R_H \subseteq V_H \times V_H \land \mathit{uniqueids}(H)$ \\
    \> \> $\land\; \satisfies{H}{\phi} \land heap(H) * \localstate(\phi)$ \\
\end{tabbing}
The auxiliary definitions we used in this definition are all given in
Figure~\ref{heap-invariant}. 

\begin{figure}
{\small
\begin{specification}
\nextlinelabel{invariant:cell-heap}
$CellHeap =$ \nextline
\;\;$\Sigma$\=$ D \in \powersetfin{ecell}.$ \nextline
               \>$(\Pi (\pack{\alpha}{\_}) \in D. 
  ($\=${\codetype{\alpha}} \;\times \opttype{\alpha} \;\times$ \nextline
\>  \>${\powersetfin{ecell}} \;\times \powersetfin{ecell} \;\times \N)$ \nextline[0.5em]
       

$code = \pi_1$ \qquad\qquad\= $obs = \pi_4$ \nextline
$value = \pi_2$            \> $unique = \pi_5$ \nextline
$reads = \pi_3$ 

\nextlinelabel[0.5em]{invariant:value-rel}
$V_{(D, h)} = \comprehend{c \in D}{\exists v.\; value(h(c)) = \ctext{Some}(v)}$ 
\nextlinelabel{invariant:reads-rel}
$R_{(D, h)} = \comprehend{(c,c') \in D \times D}{ c' \in reads(h(c)) }$ 
\nextlinelabel{invariant:obs-rel}
$O_{(D, h)} = \comprehend{(c,c') \in D \times D}{ c' \in obs(h(c)) }$ 

\nextlinelabel[0.5em]{invariant:uniqueids}
$\mathit{uniqueids}(D,h) = \exists$\=$i : Fin(|D|) \to D.$\nextline
                       \>$i \circ (unique \circ h) = id \; \land$ \nextline
                       \>$(unique \circ h) \circ i = id$ 

\nextlinelabel[0.5em]{invariant:satisfies}
$\satisfies{(D,h)}{\phi} = sat((D,h), D, \phi)$ 

\nextlinelabel[0.5em]{invariant:sat}
$\sat((D,h), D', \localref{r}{v}) = \top$ \nextlinelabel{invariant:sat-I}
$\sat((D,h), D', I) = \top$ \nextlinelabel{invariant:sat-otimes}
$\sat((D,h), D', \phi \otimes \psi) = \exists D_1, D_2.\;$\=$D = D_1 \uplus D_2$ \nextline
                                                    \>$\land\; \sat((D,h), D_1, \phi)$ \nextline
                                                    \>$\land\; \sat((D,h), D_2, \psi)$ 
\nextlinelabel{invariant:sat-cellneg}
$sat((D,h), D', \cellneg{a}{e}) = $ \nextline
\qquad $a \in D \land code(h(a)) = e \land a \not \in V_H$ 
\nextlinelabel{invariant:sat-cellpos}
$sat((D,h), D', \cellpos{a}{e}{v}{r}) = $ \nextline
\qquad\= $a \in D \land code(h(a)) = e \;\land$ \nextline
      \>     (\= if $r \cap V_H = r$ \nextline
      \>     \>then $value(h(a)) = \ctext{Some}\;v \land reads(h(a)) = r$ \nextline
      \>     \>else $a \not\in V_H)$

\nextlinelabel[0.5em]{invariant:heap}
$heap(D,h) = $ \nextline
\;\;$counter \pointsto |D| \;* $ \nextline
\;\;$\forall^* c \in D.\;$\=$\exists v_r, v_o : \cellset.\;$ \nextline
                         \>$c.code \pointsto code(h(c))   \;* $ \nextline
                         \>$c.value \pointsto value(h(c)) \;* $ \nextline
                         \>$c.reads \pointsto v_r \;* $ \nextline
                         \>$c.obs   \pointsto v_o \;* $ \nextline
                         \>$c.unique  = unique(h(c)) \;\land$ \nextline
                         \>$set(D, v_r, reads(h(c))) \;\land$ \nextline
                         \>$set(D, v_o, obs(h(c)))$
\nextlinelabel[0.5em]{invariant:localstate}
$\localstate(\celleither{a}{e})$\=$= \emp$ \nextline
$\localstate(I)$                \>$= \emp$ \nextline
$\localstate(\phi \otimes \psi)$\>$= \localstate(\phi) * \localstate(\psi)$ \nextline
$\localstate(\localref{r}{v})$  \>$= r \pointsto v$ 
\end{specification}
}
\caption{Definitions for Heap Invariant}
\label{heap-invariant}
\end{figure}

We first assert the existence of a cell heap $H$ drawn from the set
$CellHeap$.  An element of $CellHeap$, defined on
line~\ref{invariant:cell-heap} of Figure~\ref{heap-invariant}, is a
collection of cells, paired with a function mapping each cell in that
collection to a code expression, a possible value, a read set, an
observed set, and an identifier. We will use $H$ as a variable ranging
over cell heaps, and will use the pair pattern $(D,h)$ to range over
cell heaps when we need to use the individual components of the pair.

In the first two lines of $Inv(H, \phi)$, we assert all of the global
conditions in terms of the mathematical cell heap $H$. First, we
assert that the relational transpose $(\cdot)^\dagger$ of the reads
relation $R_H$ is the observes relation $O_H$. These two relations
(defined in lines \ref{invariant:reads-rel} and
\ref{invariant:obs-rel}) are computed from the cell heap. $R_H$
consists of those pairs of cells in $H$, such that the first component
reads the second component. Likewise, $O_H$ consists of those pairs of
cells in $H$ such that the first component is observed by the second
component.  Requiring that $R_H = O^\dagger_H$ enforces the condition
that the reads and observe relations be the same, only pointing in
opposite directions (i.e., if $a$ reads $b$, then $b$ is observed by
$a$).

Then, we require that the transitive (but not reflexive) closure of
the reads relation, $R^+_H$ form a strict partial order. Strictness
enforces the condition that there be no cycles in the dependence graph
(because otherwise there could be elements $a$, such that $(a, a) \in
R^+_H$).  Next, we require that the reads relation $R_H$ is a subset
of the Cartesian product $V_H \times V_H$ of the set $V_H$ of cells
carrying values (defined on line~\ref{invariant:value-rel}). This
ensures that (1) there are no dependencies on unready cells, and (2)
all unready cells have empty read and observe sets.

Finally, we ask that all of the cells in $H$ have unique identifiers
--- $\mathit{uniqueids}(D,h)$ (defined on line \ref{invariant:uniqueids}) 
asserts that there is a bijective map
between $Fin(|D|)$ (the finite set consisting of the natural numbers
from 0 to the size of the cell heap) and the cells in $D$, and that 
each cell carry its uniquely identifying number in its $unique$ field.

In the third line of the definition of $Inv(H, \phi)$, we begin by
requiring that the cell heap $H$ satisfy the abstract heap formula
$\phi$, which formalizes the informal reading of the abstract heap
formulas given earlier. The definition of the satisfaction relation is
given on line~\ref{invariant:satisfies}. The satisfaction relation
$\satisfies{H}{\phi}$ asserts that $\phi$ describes the heap $H$. 

This relation closely follows the standard pattern of separation
logic, with one exception: we need to remember the whole heap in order
to check whether or not the read sets of positive cell formulas like
$\cellpos{a}{e}{v}{rs}$ are ready. To model this, we use an auxilliary
relation $\sat((D,h), D', \phi)$, in which $(D,h)$ is the whole heap, and 
$D'$ is the fragment of the heap within which $\phi$ must lie. The case for
the unit $I$, on line~\ref{invariant:sat-I}, is satisfied by any cell
heap, and the tensor $\phi \otimes \psi$ (on line~\ref{invariant:sat-otimes}) 
is satisfied if we can break $D'$ into two disjoint pieces, one of which
satisfying $\phi$ and the other satisfying $\psi$. In this sense, we are
building a domain-specific separation logic on top of separation logic. 
The clause for $\cellneg{a}{e}$, on line~\ref{invariant:sat-cellneg}, says
that (1) the cell $a$ must be within $D'$, (2) its code must be $e$, and
(3) it must be unready (i.e., have no value). The clause for $\cellpos{a}{e}{v}{r}$
(on line~\ref{invariant:sat-cellpos}) is a little more complex. It also says that 
$a$ must be in $D'$ and that $a$'s code must be $e$. In addition, it says that 
if all the cells in $r$ have values, then $a$'s value must be $v$ and otherwise 
$a$ must not have a value. The clause for $\localref{r}{v}$ is simply
the true assertion --- since this is a piece of local state that does not 
participate in dependency tracking, we leave it out of this 
invariant and use an ordinary separation logic formula to track it. 

The second-to-last clause of $Inv(H, \phi)$ is the predicate $heap(H)$. This
predicate, defined on line~\ref{invariant:heap}, finally connects the
cell heap, which is a purely mathematical object, to the actual
low-level heap the implementation uses. We ask that the global counter
reference $counter$ point to an integer field equal to the size of the
cell heap, and then use the iterated separating conjunction
$\forall^*$ to require that for each cell in the cell heap, we have
pointers to the appropriate code, value, read, observer, and unique
identifier fields. The $unique$ identifier field contains the natural
number uniquely identifying the cell. The read and observer fields
point to values of type $\cellset$, with the predicate $set(D, v_r,
reads(h(c)))$ and $set(D, v_o, obs(h(c)))$ asserting that the program
value $v_r$ and $v_o$ respectively representing the read and observed
sets of the cell. (This predicate is explained in the appendix, as part
of the specification of sets of cells.)

The last clause in $Inv(H,\phi)$ is $\localstate(\phi)$, which 
finds each local reference formula in $\phi$ and asserts that it is in
the physical heap.

The global character of this invariant should be evident; we describe
\emph{all} of the cells in the heap at once in order to state our
invariants. So it is not immediately clear that we have made much
progress towards a modular proof technique. However, we are actually
very close: with just two more ideas, we will be able to give a
solution to this problem.

\subsection{Frame Properties via Polymorphism}

As we mentioned earlier, our abstract heap formulas essentially give
us a small domain-specific separation logic. This means that in order
to reason locally over cell heaps, we need to find an
application-specific version of the frame rule for our library. 

To do this, we will adapt some ideas proposed by
Birkedal et.\ al.~\cite{birkedal-torpsmith-yang-lics05}.
They suggested interpreting the frame rule of separation logic as a form of
quantification --- instead of having a separate frame rule that allows
adding a frame to any triple, they proposed that all of the atomic
rules of the program logic be replaced with rules possessing an extra
quantifier ranging over ``the rest of the heap'':

\begin{mathpar}
  \inferrule*[right=Example]
          { }
          { \forall R.\; \setof{ (e \pointsto v) * R } \;e := v'\; \setof{ (e \pointsto v') * R}}
\end{mathpar}

This quantifier is propagated through the proof, and any use of the
frame rule can be interpreted as instantiating the universal
quantifier appropriately. The reason this idea is fruitful for us is
that it will allow us to give a frame rule, even though the underlying
semantics of our library does not actually satisfy any analogues of
the traditional safety monotonicity and frame lemmas. For example, the
$\updatecell$ operation certainly does not act locally -- it
recursively traverses the observers set, possibly mutating a very
large part of the cell graph.

Nonetheless, we can prove the soundness of the following triple
specifying $\updatecell$.

\begin{prop}{(Update Rule)}
For all appropriately-typed cells $o$ and code expressions $e$ and
$e'$, the following triple is derivable in our specification logic:

\begin{tabbing}
$\forall \psi:\formula.\; $\=$\setof{G(\celleither{o}{e'} \otimes \psi)}$ \\
                           \>$\runcmd {\updatecell\;o\;e}$ \\
                           \>$\setof{a:1.\; G(\cellneg{o}{e} \otimes \psi)}$
\end{tabbing}
\end{prop}

\begin{proof}
The key to this proof is the conditional interpretation of the
$\cellpos{c}{e}{v}{r}$ formula. When the $\updatecell\;o\;e$
command executes, it recursively finds every cell which depends 
on $o$, and modifies it to be unready.

Now consider any positive cell formula in $\psi$ which depends on $o$,
directly or indirectly. The satisfaction relation for $\phi$ asserts
that in order for a positive cell formula to represent a ready cell, 
everything in its read set also has to be ready. So when $o$'s formula switches
to the unready state, we now require that every positive formula depending 
on $o$ represents an unready cell --- which is exactly the effect of
executing $\updatecell$.  As a result, we can leave the entire frame
$\psi$ untouched, even though the physical heap it represents may have
been (quite drastically) modified, and many cells may have gone from
a ready to an unready state. 
\end{proof}

We can prove the soundness of a similar specification for $\newcell$ as
well:

\begin{prop}{(New Cell Rule)}
For all code expressions $e$, the following specification is derivable in
our specification logic: 
\begin{tabbing}
$\forall \psi:\formula.\; $\=$\setof{G(\psi)}$ \\
                           \>$\run {\newcell\;e}$ \\
                           \>$\setof{a:\celltype{\tau}.\; G(\cellneg{a}{e} \otimes \psi)}$
\end{tabbing}
\end{prop}

\begin{proof}
This is much easier than $\updatecell$: after $\newcell$ allocates a new
numeric id for the new cell, we can extend the cell heap with the new 
cell and show that it continues to satisfy the invariant. 
\end{proof}

As we can see, the conditional interpretation of $\cellpos{a}{e}{v}{r}$
gives us quite a strong modular reasoning property for $\updatecell$ and
$newcell$ -- we can simply pretend that we are locally changing or
creating a cell, and leave the frame unchanged. This property lets us write 
programs whose components independently modify the cell heap, without
having to know what cells might be updated by the change. 

\begin{figure}
{\small
\begin{mathpar}
  \inferrule*[right=Ready]
            {\forall a' \in r.\; \exists v'.\; \ready{\phi}{a'}{v'}}
            {\ready{\phi \otimes \cellpos{a}{e}{v}{r}}{a}{v}}
  \\
  \\
  \inferrule*[right=UnreadyPos]
            {\exists a' \in r.\; \unready{\phi}{a'}}
            {\unready{\phi \otimes \cellpos{a}{e}{v}{r}}
                     {a}}
  \and
  \inferrule*[right=UnreadyNeg]
            { }
            {\unready{\phi \otimes \cellneg{a}{e}}{a}}
\end{mathpar}
}
\caption{Ready and Unready Judgments}
\label{readiness}
\end{figure}

\begin{figure}
{\small
  \begin{displaymath}
    \begin{array}{lcl}
      closed(I, s) & = & \top \\
      closed(\phi \otimes \psi, s) & = & closed(\phi, s) \land closed(\psi, s) \\ 
      closed(\localref{r}{v}, s) & = & \top \\
      closed(\cellneg{a}{e}, s) & = & \top \\
      closed(\cellpos{a}{e}{v}{r}, s) & = & r \subseteq s \\
    \end{array}
  \end{displaymath}
}
\caption{Closedness predicate}
\label{closedness}  
\end{figure}

\begin{figure}
{\small
  \begin{displaymath}
    \begin{array}{lcl}
      R(s, I)                 & = & I \\
      R(s, \phi \otimes \psi) & = & R(s, \phi) \otimes R(s, \psi) \\
      R(s, \localref{r}{v})   & = & \localref{r}{v} \\
      R(s, \cellneg{a}{e})    & = & \cellneg{a}{e} \\
      R(s, \cellpos{a}{e}{v}{r}) & = & \left\{\begin{array}{ll}
                                                \cellpos{a}{e}{v}{r} 
                                              & \mbox{if } s \cap r = \emptyset \\
                                                \cellneg{a}{e}
                                              & \mbox{otherwise}
                                              \end{array}
                                       \right.
    \end{array}
  \end{displaymath}
}
\caption{Definition of the Ramification Operator $R$}
\label{ramify-def}
\end{figure}

\subsection{Ramified Frame Properties}

While this strategy is sufficient for $\newcell$ and
$\updatecell$, it is not adequate for defining a frame property
for $\codetype{\tau}$ expressions. 

As an example, suppose that we want to evaluate the code expression $\readcell \tau\;
a$, in a cell heap described by $\cellneg{a}{\return 5}$.  Clearly,
this is a sufficient footprint, and we expect to get the return
value 5, and see the cell formula change to $\cellpos{a}{\return
  5}{5}{\emptyset}$.  However, the fact that we are now changing cells
from negative to positive means that the conditional character of
readiness, which worked in our favor with $\updatecell$ and
$\newcell\!\!$, now works against us.

In particular, suppose that we run this command with a framed abstract heap
formula $\psi = \cellpos{b}{\readcell a}{17}{\setof{a}}$. Now, the
whole starting heap will be described by the formula:
\begin{displaymath}
\cellneg{a}{\return 5} \otimes \cellpos{b}{\readcell a}{17}{\setof{a}}  
\end{displaymath}
In any heap satisfying this formula, $b$ will be unready, because it depends 
on an unready cell. But when we execute $\readcell a$, simply copying $\psi$ 
into the post-state will give us the cell formula:
\begin{displaymath}
\cellpos{a}{\return 5}{5}{\emptyset} \otimes \cellpos{b}{\readcell a}{17}{\setof{a}}
\end{displaymath}
That is, our satisfaction relation now expects $b$ to be ready and have the 
value 17, even though $\readcell a$ never touches $b$ at all!

Clearly, we cannot expect to be able to simply copy the same frame
formula into the pre- and the post-condition states in the
specification of commands like $\readcell a$.

To deal with this problem, we look back to the original paper
introducing the frame problem~\cite{mccarthy}. They described the
frame problem as the problem of how to specify what parts of a state
were \emph{unaffected} by the action of a command, which inspired the
name of the frame rule in separation logic. In that paper, he also
described the \emph{qualification problem}. He observed that many
commands (such as a flipping a light switch turning on a light bulb)
have numerous implicit preconditions (such as there being a bulb in
the light socket), and dubbed the problem of identifying these
implicit preconditions the qualification problem.

Some years later, \citet{finger} observed that the qualification
problem has a dual: actions can have indirect effects that are not
explicitly stated in their specification (e.g., turning on the light can
startle the cat). He called the problem of deducing these implicit
consequences the ``ramification problem'' --- is there a simple way to
represent all of the indirect consequences of an action?

We can understand our difficulty as an instance of the ramification
problem. When we evaluate a code expression, we may read some unready
cells and send them from an unready state in the precondition to a ready
state in the postcondition. However, we may have had some cell formulas in
our frame which claimed their corresponding cells were unready purely
because one of the cells in our footprint was unready. Therefore, when
we update the footprint, we must modify the frame formula to account
for the ramifications of our update in the footprint. So even though
the actual physical storage representing the frame does not change at all, 
we need to modify our abstract formula to reflect our updated state
of knowledge. 

In our case, \emph{all} of the effects on the frame will arise from
the cell formulas we change from unready to ready. Thus, given the set of cells
which became ready, we can repair the framing formula by taking each
positive cell formula, and setting it to a negative state if its read
set includes anything that went from unready to ready. We define the
ramification operator $R(s, \psi)$ in Figure~\ref{ramify-def}.  It is
a simple structural induction over a framing formula, whose only
action is to replace the positive cell formulas in $\psi$ whose read
sets intersect with $s$ with a corresponding negative cell formula.
The ramification operator has a number of useful properties, which are
most easily expressed after we have introduced a few auxiliary
judgments and predicates.

In Figure~\ref{readiness} we define the two judgments $\unready{\phi}{o}$
and $\ready{\phi}{o}{v}$, which establish whether a cell is ready or
unready, from the syntactic structure of $\phi$. $\ready{\phi}{o}{v}$ is intended
to mean that the cell $o$ is ready and will return value $v$, in any heap 
described by $\phi$. Correspondingly, $\unready{\phi}{o}$ means that $o$ is not 
ready in any heap described by $\phi$. Since these are purely syntactic judgments, 
we need to show that they are consistent with heaps described by $\phi$. 

\begin{prop}{(Soundness of $\ready{\phi}{o}{v}$ and $\unready{\phi}{o}$)}
For all $\phi, o,$ and $H$ such that $H = (D,h)$, the following assertions
are tautologies in separation logic. 

\begin{itemize}
\item $(Inv(H, \phi) \land \ready{\phi}{o}{v}) \implies value(h(o)) = \ctext{Some}\;v$
\item $(Inv(H, \phi) \land \unready{\phi}{o}) \implies o \not\in V_H$
\end{itemize}
\end{prop}

Next, in Figure~\ref{closedness}, we define the $closed(\phi, s)$
predicate, which asserts that every cell formula in $\phi$ reads at
most the cells in $s$. Now, we can summarize the interactions between
the ramification operator $R$ and abstract heap formulas as follows:

\begin{prop}{(Interaction Properties)}
Given sets of cells $s$ and $u$, cell $o$, value $v$, and formula $\phi$, we have
that:
\begin{itemize}
\item $R(s, R(u, \phi)) = R(s \cup u, \phi)$
\item If $\unready{\phi}{o}$, then $\unready{R(u, \phi)}{o}$ 
\item If $\ready{R(u, \phi)}{o}{v}$, then $\ready{\phi}{o}{v}$ 
\item If $closed(\phi, s)$, then $R(u, \phi) = R(u \cap s, \phi)$ 
\end{itemize}
\end{prop}

All of these facts can be proved with simple inductive arguments, since
they are all syntactic facts. 

The first property means that if we evaluate two expressions, we can
simply combine their ramification effects without having to worry
about the order that they were evaluated in. The second and third let
us know that a ramification cannot make us forget a cell is unready,
nor can it make anything ready that was not ready before. The last
property permits us to constrain the effect of a ramification --- if we
know that two parts of the abstract heap formula do not read each
other at all, we can deduce that ramifications from one will not
affect the other.

Now we can define the abstract semantics of the code monad. We
introduce the ``judgment'' $\astep{\phi}{e}{\phi'}{v}{r}{u}$, which is
read as ``from an initial state $\phi$, evaluating the
$\codetype{\tau}$ expression $e$ will result in a modified state
$\phi'$ and a return value $v$ of type $\tau$. The expression $e$ will
have directly read the cells in $r$, and will have evaluated the cells
in $u$, sending them from an unready to a ready state.''  The formal
definition of the meaning is given in Figure~\ref{abstract-semantics},
where our ``judgment'' is revealed to be a notational abbreviation for
a formula in our logic of specifications.


\begin{figure}
{\small
\begin{tabbing}
$\astep{\phi}{e}{\phi'}{v}{r}{u} \triangleq$ \\[0.5em]
\;\;\= $\forall \psi.\;$\=$\setof{G(\phi \otimes \boxed{\psi})}$ \\
    \>                  \>${\runcmd e}$ \\
    \>                  \>$\setof{a:\tau.\;
                             G(\phi' \otimes \boxed{R(u, \psi)})
                             \land \exists z.\; a = (v, z) \land set(u, z, u)}$ \\
    \>                  \>$\!\!\specand$ \\
    \>                  \>$\{\forall o.\;$\=$(\unready{\phi \otimes \psi}{o} \land o \in \domain{\psi})$ \\
    \>                  \> \> $\implies \unready{\phi' \otimes R(u, \psi)}{o}\}$ \\
\>$\!\!\specand$ \\ 
\>$\{\forall c \in r \cup u.\; \exists v.\; \ready{\phi'}{c}{v} \;\land \forall c \in u.\; \unready{\phi}{c}\}$ 
\end{tabbing}
}
\caption{Definition of the Abstract Semantics}
\label{abstract-semantics}
\end{figure}

\begin{figure}
{\small
\begin{mathpar}
  \inferrule*[right=AUnit]
            { }
            {\astep{I}{\return \alpha\;v}{I}{v}{\emptyset}{\emptyset}}
\and
  \inferrule*[right=ABind]
             {\astep{\phi}{e}{\phi'}{v'}{r_1}{u_1} \\
              \astep{\phi'}{f\;v'}{\phi''}{v''}{r_2}{u_2}}
             {\astep{\phi}{\bind \alpha\;\beta\;e\; f}{\phi''}{v''}{r_1 \cup r_2}{u_1 \cup u_2}}
\\
\\
  \inferrule*[right=AReady]
            {\ready{\phi}{a}{v}}
            {\astep{\phi}{\readcell \alpha\;a}{\phi}{v}{\setof{a}}{\emptyset}}
\and
  \inferrule*[right=AUnready]
            {\unready{\celleither{a}{e} \otimes \phi}{a} \;\;
             \astep{\phi}{e}{\phi'}{v}{r}{u}}
            {\aconfig{\celleither{a}{e} \otimes \phi}
                     {\readcell \alpha\;a} \Downarrow \\
             \aconfig{\cellpos{a}{e}{v}{r} \otimes R(\setof{a},\phi')}
                     {v}
             \aeffect{\setof{a}}
                     {u \cup \setof{a}}}
\\
\\
  \inferrule*[right=AGetRef]
            { } 
            {\astep{\localref{r}{v}}{\getref r}
                   {\localref{r}{v}}{v}{\emptyset}{\emptyset}}
\and
  \inferrule*[right=ASetRef]
            { } 
            {\astep{\localref{r}{v}}{\setref r\;v'}
                   {\localref{r}{v'}}{\unit}{\emptyset}{\emptyset}}
\\
\\
  \inferrule*[right=AbstractFrame]
            {\astep{\phi}{e}{\phi'}{v}{r}{u}}
            {\astep{\phi \otimes \psi}{e}{\phi' \otimes R(u, \psi)}{v}{r}{u}}
\end{mathpar}
}
\caption{Abstract Semantics of Notifications}
\label{abs-semantics}
\end{figure}



The step relation $\astep{\phi}{e}{\phi'}{v}{r}{u}$ really means three
things. First, it means that if we run $e$ in a heap $G(\phi \otimes
\psi)$, then we will end in a heap $G(\phi' \otimes R(u, \psi))$, and
that the return value will be a pair consisting of the value $v$ and
the read set $u$ (that is, the return value is a pair $(v,z)$, and the
second component of the return value $z$ is a $\cellset$ which
represents the set $u$, as indicated by the predicate $set(u, z, u)$).
Note the use of quantification over $\psi$ to describe the frame, and
that furthermore we need to use the ramification operator to describe
the change in the frame in the postcondition. 

Second, we assert that anything which was unready in the frame,
remains unready in the frame in the postcondition, which is a way of
saying that we never read anything outside the footprint $\phi$.
Third, we assert that everything that was read or evaluated (the cells
in $r$ or $u$) is really syntactically ready in the postcondition, and
that everything we claim we evaluated (i.e., the cells in $u$) was
really syntactically unready in the precondition.

The reason we have to maintain these conditions is that the readiness
judgments are syntactic derivations which do not know anything about
the effect of the execution of the command $e$. So we need to
explicitly construct syntactic facts reflecting any changes in the
semantic heap if we wish to use them in further reasoning about the
syntactic description. In particular, we need these facts to prove the
rules given in Figure~\ref{abs-semantics}, where we finally show that
terms of type $\codetype{\alpha}$ behave as we claimed in our informal
description. We give a number of implications over specifications in
inference rule format, mimicking the structure of a big-step
semantics. Again, we emphasize that these rules are really just 
implications in specification logic --- the inference rule format
is just a suggestive notation for the actual specifications we use.

In rule \textsc{AUnit}, we give a specification for the $\return$
command, which simply returns its argument and neither reads nor
updates any cells or state. The \textsc{ABind} rule explains how
sequential composition works --- as expected, we evaluate the first
monadic argument, and pass the result to the functional argument, and
evaluate that. The read and update sets are simply the union of the
two executions. Reading a cell comes in two variants, \textsc{AReady}
and \textsc{AUnready}. If a cell is ready, we simply return its
memoized value without any further computation. If a cell is unready,
we need to evaluate its code body, and then update the cell with its
new value. Note that we have to apply the ramification operator in
\textsc{AUnready}, because the cell we are reading goes from unready to
ready itself. We can also read (\textsc{AGetRef}) and write
(\textsc{ASetRef}) local state, which do not have any effect on the
cells.

Finally, we have the \textsc{AbstractFrame} rule, which allows us to
extend the abstract heap formulas in the style of the frame rule of 
separation logic --- the signal difference being that we have to apply 
the ramification operator $R$ to the frame in the post-state. 

\begin{prop}{(Soundness of Abstract Semantics)}
All of the rules of the abstract semantics in Figure~\ref{abs-semantics} are
derivable in our specification logic. 
\end{prop}

We can now reason about the behavior of our imperative notification
network library in terms of its action on the abstract
heap. Quantification and ramification give us a domain-specific frame
property, which allows us to modularly prove the correctness of
programs that construct and use this dependency-tracking library.

\section{Verifying an Imperative Implementation of Functional Reactive Programming}

In this section, we will see how to verify an imperative
implementation of a simple synchronous functional reactive programming
system. We will begin by giving the purely functional/mathematical
semantics of stream transducers. This semantics is easy to reason
about, but too inefficient to consider as an implementation. Then,
we'll give an imperative implementation of the reactive primitives,
which is intended to be driven by an event loop. Finally, we will 
prove the correctness of a realization relation between the
imperative and functional implementations, which will let us reason
about the imperative implementation as if it were functional. 

\subsection{Specifying Functional Reactive Programs}

\emph{Functional Reactive Programming}~\cite{frp} is a style of
writing interactive programs based on the idea of \emph{stream
  transducers}.  The idea is to model a time-varying input signal of
type $A$ as an infinite stream of $A$'s, and to model an interactive
system as a function that takes a stream of inputs $\stream{A}$ and
yields a stream of outputs $\stream{B}$. Note that a stream can be
viewed either as an infinite sequence of values, or isomorphically as
a function from natural numbers to values (i.e., a function from times
to values). In our discussion, we'll switch freely between these two
views, using the most convenient viewpoint.\footnote{Given an infinite stream $vs$, we will use use $take\;n\;vs$ to denote
the finite list consisting of the first $n$ elements of the stream
$vs$. Correspondingly, $drop\;n\;vs$ is the infinite stream with $vs$
with its first $n$ elements cut off. With a function $f$, $map\;f\;vs$
maps $f$ over the elements of $vs$, and given another infinite stream
$us$, the call $zip\;us\;vs$ returns the infinite stream of pairs of
elements of $us$ and $vs$. If $v$ is an element, $v :: vs$ will 
denote consing $v$ to the front of $vs$, and if $xs$ is a finite list, then
$xs \cdot vs$ will denote appending the finite sequence $xs$ to the
front of $vs$. Finally, we will write $vs_n$ to denote the $n$-th element
of the stream $vs$, and $\mathit{last}\;xs$ to denote the last element of
a non-empty finite list.}

However, not all functions $\stream{A} \to \stream{B}$ are legitimate
stream transducers: we need to restrict our attention to \emph{causal}
functions. A transducer is causal if we can compute the first
$n$ elements of the output having read at most $n$ elements of
the input. 

\begin{tabbing}
$causal(f : \stream{A} \to \stream{B}) \equiv$ \\
\;\;\= $\exists \hat{f} : \listtype{A} \to \listtype{B}.\;\forall as:\stream{A}, n:\N.$ \\
    \> \;\;$take\;n\;(f\;as) = \hat{f}\;(take\;n\;as)$ 
\end{tabbing}

Given a causal transducer $p$, we will write $\hat{p}$ to indicate the
corresponding list function which computes its finite
approximations. In Figure~\ref{transducer-semantics}, we define a
family of combinators acting on causal transducers.

\begin{figure}
{\small
\begin{specification}
\nextlinelabel{transducer:spec:ST}
$\ST{A}{B} = \comprehend{f \in \stream{A} \to \stream{B}}{causal(f)}$

\nextlinelabel[0.5em]{transducer:spec:lift}
$\lifttrans : (A \to B) \to \ST{A}{B}$ \nextline
$\lifttrans \;f\;as = map\;f\;as$ 

\nextlinelabel[0.5em]{transducer:spec:seq}
$\seqtrans  : \ST{A}{B} \to \ST{B}{C} \to \ST{A}{C}$ \nextline
$\seqtrans \;p\;q = q \circ p$ 

\nextlinelabel[0.5em]{transducer:spec:par}
$\partrans  : \ST{A}{B} \to \ST{C}{D} \to \ST{A \times C}{B \times D}$ \nextline
$\partrans \; p\;q\;abs = zip\; (p\;(map\;\pi_1\;abs))\;(q\;(\maptrans\;\pi_2\;abs))$

\nextlinelabel[0.5em]{transducer:spec:switch}
$\switchtrans : \N \to \ST{A}{B} \to \ST{A}{B} \to \ST{A}{B}$ \nextline
$\switchtrans\;k\;p\;q = \semfun{as}{(take\;k\;(p\;as))\cdot(q\;(drop\;k\;as))}$ 

\nextlinelabel[0.5em]{transducer:spec:loop}
$\looptrans : A \to \ST{A\times B}{A \times C} \to \ST{B}{C}$ \nextline
$\looptrans\;a_0\;p = (map\;\pi_2) \circ (cycle\;a_0\;p)$ 

\nextlinelabel[0.5em]{transducer:spec:cycle}
$\cycletrans : A \to \ST{A\times B}{A \times C} \to \ST{B}{A \times C}$ \nextline
$\cycletrans\;a_0\;p = \lambda bs.\;\lambda n.\;last(gen\;a_0\;p\;bs\;n)$ 

\nextlinelabel[0.5em]{transducer:spec:gen}
$\gentrans : A $\=$\to \ST{A\times B}{A \times C} \to \stream{B}$ \nextline
                \>$\to \N \to \listtype{(A \times C)}$\nextline
$\gentrans\;a_0\;p\;bs\;0 \;\;\; = \hat{p}\; [(a_0, bs_0)]$ \nextline
$\gentrans\;a_0\;p\;bs\;(n+1) = $ \nextline
\;\;$\hat{p}\;(zip (a_0 :: (map\;\pi_1\;(\gentrans\;a_0\;p\;bs\;n)))\;
                                        (take\;(n+2)\;bs))$ 
\end{specification}
}
\caption{Semantics of Stream Transducers}
\label{transducer-semantics}
\end{figure}

The type $\ST{A}{B}$ of stream transducers (defined on
line~\ref{transducer:spec:ST}) is the set of causal functions from
$\stream{A}$ to $\stream{B}$. The operation $\lifttrans\;f$
(line~\ref{transducer:spec:lift}) creates a stream transducer that
simply maps the function $f$ over its input.  Calls to $\seqtrans\;p\;q$
(line~\ref{transducer:spec:seq}) are sequential composition: it feeds
the output of $p$ into the input of $q$. The operator $\partrans\;p\;q$
(line~\ref{transducer:spec:par}) defines parallel composition --- it
takes a stream of pairs, and feeds each component to its arguments,
respectively, and then merges the two output streams to produce the
combined output stream. The function $\switchtrans\;k\;p\;q$
(line~\ref{transducer:spec:switch}) is a very simple ``switching
combinator''.  It behaves as if it were $p$ for the first $k$ time
steps, and then behaves as if it were $q$, only starting with the
input stream beginning at time $k$.

The combinator $\looptrans\;a_0\;p$ is a feedback operation. It acts
upon a transducer $p$ which takes pairs of $A$s and $B$s, and yields
pairs of $A$s and $C$s. It turns it into a combinator that takes $B$s
to $C$s, by giving $p$ the value $a_0$ (and its $B$-input) on the
first time step, and uses the output $A$ at time $n$ as the input $A$
at time $n+1$. This is useful for constructing transducers that do
things like sum their inputs over time, and other stateful operations. 

Because this function involves feedback, it should not be surprising
that it makes use of the causal nature of its argument operation. The
$\looptrans$ function is defined in terms of $\cycletrans$, which also 
returns the sequence of output $A$s, and $\cycletrans$ is defined in 
terms of $\gentrans$, which is a function that given an argument $n$ returns a list of
outputs for the time steps from $0$ to $n$. Notice that
$gen\;a_0\;p\;bs\;n$ will always return $n+1$ elements (e.g., at
argument 0, it will return a 1 element list containing the output at
time step 0), which means that the call to $last$ in $cycle$ is
actually safe. In order to calculate $gen$, we need to recursively
calculate the outputs for all smaller time steps, and this is why $p$
must be a causal stream function --- we need to be able to call the 
approximation $\hat{p}$ to operate on the list consisting of
the first $n$ elements. 

All of these definitions are familiar to functional programmers, and
there are many techniques to prove properties of these functions ---
coinductive proofs, the $take$-lemma of \citet{bird-wadler}, arguments based on
the isomorphism between streams and functions from natural
numbers. All of these serve to make proving properties about stream
transducers very pleasant. For example, one property we will need in
the next section is the following:

\begin{lemma}{(Loop Unrolling)} We have that 
  \begin{displaymath}
    cycle\;a_0\;p\;bs = p\;(zip\;(a_0 :: (map\;\pi_1\;(cycle\;a_0\;p\;bs)))\;bs)
  \end{displaymath}
\end{lemma}

\begin{proof}
  This is easily proved using $take$-lemma of \citet{bird-wadler}, which
  states that two streams are equal if all their finite prefixes are
  equal.
\end{proof}


\subsection{Realizing Stream Transducers with Notifications}

\begin{figure}
{\small
\begin{specification}
\nextlinelabel{transduce:impl:ST}
$\ctext{ST} : \star \to \star \to \star$ \nextline
$\ST{\alpha}{\beta} \equiv \celltype{\alpha} \to \monad{\celltype{\beta}}$ 

\nextlinelabel[0.5em]{transduce:impl:lift}
$\liftop : \forall \alpha, \beta:\star.\; (\alpha \to \beta) \to \ST{\alpha}{\beta}$ \nextline
$\liftop\;\alpha\;\beta\;f\;input = $ \nextline
\;\; $\newcell \;(\bind (\readcell input)\; (\fun{x}{\alpha}{\return (f\;x)}))$ 

\nextlinelabel[0.5em]{transduce:impl:seq}
$\seqop : \forall \alpha, \beta:\star.\; \ST{\alpha}{\beta} \to \ST{\beta}{\gamma} \to \ST{\alpha}{\gamma}$ \nextline
$\seqop \alpha\;\beta\;p\;q\;input = [$\=$\letv{middle}{p\;input}$ \nextline
                            \>$\letv{output}{q\;middle}$ \nextline 
                            \>$\;output]$ 

\nextlinelabel[0.5em]{transduce:impl:par}
$\parop : \forall \alpha,$\=$\beta, \gamma, \delta:\star.$ \nextline
   \>$\ST{\alpha}{\beta} \to \ST{\gamma}{\delta} \to \ST{\alpha \times \gamma}{\beta \times \delta}$ \nextline
$\parop \alpha\;\beta\;\gamma\;\delta\; p \; q \; input = $ \nextline
\;\;$[$\=$\ctext{letv}\;a = \newcell (\bind$\=$(\readcell\;input)$ \nextline
     \>                                   \>$(\fun{x}{\alpha\times \beta}{\return (\fst{x})})) \ctext{ in}$ \nextline
     \>$\ctext{letv}\;b = {p\;a} \ctext{ in}$ \nextline
     \>$\ctext{letv}\;c = \newcell (\bind$\=$(\readcell\;input)\;$\nextline 
     \>                                   \>$(\fun{x}{\alpha\times \beta}{\return (\snd{x})})) \ctext{ in}$ \nextline
     \>$\ctext{letv}\;d = {q\;c} \ctext{ in} $ \nextline
     \>$\ctext{letv}\;output = \newcell ($\=$\bind (\readcell b)\; (\lambda b:\beta.$ \nextline
     \>                                   \>$\bind (\readcell d)\; (\lambda d:\delta.$ \nextline
     \>                                   \>$\;\;\return \pair{b}{d})))] \;\ctext{in}$ \nextline
     \>$\;output]$ 

\nextlinelabel[0.5em]{transduce:impl:switch}
$\switchop : \forall \alpha, \beta:\star.\; \N \to \ST{\alpha}{\beta} \to \ST{\alpha}{\beta} \to \ST{\alpha}{\beta}$ \nextline
$\switchop \alpha\;\beta\;k\;p\;q\; input =  $ \nextline
\;\;$[$\=$\letv{r}{\newref{\N}{0}}{}$ \nextline
    \>$\letv{a}{p\;input}$ \nextline
    \>$\letv{b}{q\;input}$ \nextline
    \>$\ctext{letv}\; out = \newcell\; ($\=$\bind (\getref r) \;(\lambda i:\N.\;$ \nextline
    \>                                 \>$\bind (\setref r\;(i+1)) \; (\lambda q:\unittype.$ \nextline
    \>                                 \>$\;\;\ctext{if}(i < k, \readcell a, \readcell b)))) \;\ctext{in}$ \nextline
    \>$\;\;out]$

\nextlinelabel[0.5em]{transduce:impl:loop}
$\loopop : \forall \alpha, \beta, \gamma:\star.\; \alpha \to \ST{\alpha\times \beta}{\alpha\times \gamma} \to \ST{\beta}{\gamma}$ \nextline
$\loopop \alpha\;\beta\;\gamma\;a_0\; p \; input = $ \nextline
\;\;$[$\=$\letv{r}{\newref{\alpha}{a_0}}{}$ \nextline
    \>$\ctext{letv}\; ab = \newcell\; ($\=$\bind (\readcell input)\; (\lambda b:\beta.$ \nextline
    \>                                \>$\bind (\getref r)\;       (\lambda a:\alpha.$ \nextline
    \>                                \>$\;\; \return \pair{a}{b}))) \;\ctext{in}$ \nextline
    \>$\letv{ac}{p\;ab}{}$ \nextline
    \>$\ctext{letv}\;c = \newcell\; ($\=$\bind (\readcell ac) \;(\lambda v:\alpha \times \gamma.$ \nextline
    \>                              \>$\bind (\setref r\;(\fst{v})) \;(\lambda q:\unittype.$ \nextline
    \>                              \>$\;\;\return (\snd{v}))))\;\ctext{in}$ \nextline
    \>$\;\;c]$ 
\end{specification}
}
\caption{Imperative Stream Transducers}
\label{imperative-transducer-semantics}
\end{figure}

While the definitions in the previous subsection yield very clean
proofs, they are not suitable as implementations --- e.g.,
$loop$ recomputes an entire history at each time step! We can derive
better implementations by looking at how imperative event loops work.

The intuition underlying event-driven programming is that a stream
transducer is implemented with the combination of a notification
network, and an \emph{event loop}.  The event loop is a
(possibly-infinite) loop which does the following on each time
step. First, it updates an input cell, to reflect any input events
that occurred on that time step. Then, it reads the output cell of the
network, to discover the output value for that time step. When the
input cell is updated, invalidations propagate throughout the
dependency network, and when the outputs are read, only the
necessary re-computations are performed.

Before formalizing this idea, we will first discuss the implementation
given in Figure~\ref{imperative-transducer-semantics} in informal
terms.  (In the definitions, we suppress the type arguments to
functions in order to make the code more readable.) We define the type
of imperative stream transducers from types $\alpha$ to $\beta$ as the
type $\celltype{\alpha} \to \monad{(\celltype{\beta})}$.  This type
should be read as saying that the implementation is a function that,
given an input cell of type $A$, will \emph{construct} a dataflow
network realizing a transducer, and which returns the output cell of
type $B$ that the event loop should read.

The simplest example of this is $\liftop f$, defined on
line~\ref{transduce:impl:lift}. It will take an input cell $input$,
and build a new cell which reads $input$, and return $f$ applied to
that value. Likewise, given two imperative implementations $p$ and $q$,
$\seqop p\;q$ (defined on line~\ref{transduce:impl:seq}) will take
an input cell, and feed the input to $p$ to build a network whose
output is named $middle$, and will then give $middle$ to $q$ to get
the final output cell. The overall network will be the network built
by the calls to both $p$ and $q$, which interact via $p$'s network
installing a value in $middle$, which $q$'s network reads and processes.


The operation $\switchop k\;p\;q$ (defined on
line~\ref{transduce:impl:switch}) is the first example that uses local
state. Given an $input$ cell, we first create a local reference $r$,
initialized to $0$. Then, we build networks corresponding to $p$ and
to $q$ (with outputs $a$ and $b$, respectively). Finally, we build a cell
$out$, whose code reads and increments $r$, and which will read $a$ or
$b$ depending on whether the reference's contents are less than or
equal to $k$. The demand-driven nature of evaluation means
that we never redundantly evaluate $p$ or $q$'s networks --- we only
ever execute one of them.

The operation $\loopop a_0\;p$ builds a feedback network by
explicitly creating a reference to hold an accumulator parameter. It
constructs a local reference initialized to $a_0$, and then constructs
a cell $ab$ which reads the input and the local reference to produce a
pair of type $A \times B$. This cell is given to $p$, to construct a
network with an output cell $ac$, yielding pairs of type $A \times
C$. Finally, we construct the overall output cell $c$, which reads
$ac$ and updates the local reference with a new value of type $A$, and
returns a value of type $C$. The use of a local reference (rather than
a cell) to store the current state of $A$ is essential, because we need
to maintain the acyclicity of the dataflow graph. 

With these ideas in mind, we come to the definition of what it means
for a dataflow network to realize a stream transducer. This property
is quite large, but despite its size is pleasant to work with.
{\small
\begin{specification}
\nextline
$Realize(i, \Phi, o, f) \triangleq$ 
\nextlinelabel{realizer:formula-stream}
\;\= $\forall v:$\=$\,\stream{A}.\;$ \fbox{$\exists \phi:\stream{\formula}, u:\stream{\powersetfin{ecell}}.$} \nextlinelabel{realizer:basic-start}
\>\> $\{\forall n:\N.\;$\=$\closed{\phi_n}{\domain{\phi_n} \cup \setof{i}} \;\land$ \nextline
\>\>\>$\domain{\phi_n} = \domain{\phi_{n+1}}\; \land$ \nextlinelabel{realizer:basic-end}
\>\>\>$\forall \psi.\; \unready{\psi}{i} \implies \unready{\psi \otimes \phi_n}{o}\}$ \nextlinelabel{realizer:init}
\>\> $\specand \; \{\Phi = \phi_0\}$ \nextlinelabel{realizer:transduce}
\>\> $\specand \;  \mathit{Transduce}(i, \phi, o, u, f, v)$ \nextlinelabel[0.5em]{transduce:transduce}


$\mathit{Transduce}(ri, \phi, o, u, f, v) \triangleq$ \nextlinelabel{transduce:quantifiers}
\> $\forall $\=$ n:\N, \phi_{in}, \phi'_{in}, u_{in}.$ \nextline
\>\> \fbox{$\astep{\phi_{in}}{\readcell i}{\phi'_{in}}{v_n}{\setof{i}}{u_{in}}$} $\; \land$ \nextline
\>\> $\{$\=$\domain{\phi_{in}} = \domain{\phi'_{in}} \;\land $ \nextline
\>\>     \>$\closed{\phi_{in}}{\domain{\phi_{in}}} \land 
            \closed{\phi_{in}}{\domain{\phi'_{in}}} \land $ \nextline
\>\>     \>$\unready{\phi_{in}}{i} \land
            i \in u_{in}\}$ \nextline
\>\> \!\!\!\!$\specimp$ \nextline
\>\> \fbox{$\astep{\phi_{in} \otimes \phi_n}{\readcell o}
              {\phi'_{in} \otimes \phi_{n+1}}{(f\;v)_n}
              {\setof{o}}{u_{in} \cup u_n}$} \nextline
\>\> $\specand\; \{u \subseteq \domain{\phi_n} \land o \in u\}$ \nextline
\end{specification}
}

We read $Realize(i, \Phi, o, f)$ as saying ``the dataflow network
$\Phi$ realizes the stream transducer $f$, when the event loop writes
inputs into $i$ and reads outputs from $o$''. 

We have highlighted the key pieces of this definition with boxes. On
line~\ref{realizer:formula-stream}, for each input stream $v$, we
existentially assert the existence of a \emph{stream} of abstract heap
formulas $\phi$. This stream represents the evolving state of the
network over time --- because our notification networks contain local
state, that state can potentially have a different value at each time
step.

Then, in the unboxed formulas, we assert some well-formedness
properties. On lines \ref{realizer:basic-start} to
\ref{realizer:basic-end}, we assert that (1) the only external cell
the network may read is $i$, (2) that the domain of the network (i.e.,
the cells whose atomic formulas are in that formula) remains constant
over time, and (3) that if the input cell is ever unready, so is the
output (i.e, the output genuinely depends on the input). Then, on
line~\ref{realizer:init}, we assert that the initial state of the
evolution of the network is $\Phi$.  (All of these conditions could be
relaxed, but in this paper there is no need. It \emph{would}
be necessary if we added combinators that dynamically created new
transducers as the program ran, since we could then create new cells
at each time step.)

Finally, we assert the network implements the stream transducer 
property on line~\ref{realizer:transduce}, using the $\mathit{Transduce}$
sub-predicate. On line~\ref{transduce:quantifiers}, we first quantify
over all times $n$, and then over $\phi_{in}$, $\phi'_{in}$, and
$u_{in}$. These extra parameters exist because reading $i$ to get the
input, may require evaluating some auxiliary network state. 

We read $\phi_{in}$ as the network state needed to read $i$, and $\phi'_{in}$ is
that state after $i$ has been read, with $u_{in}$ being the cells in
$\phi_{in}$ updated during that execution.  And indeed, on the next line
in the boxed formula, we give an abstract triple making exactly this
assumption -- that reading $i$ will give us the $v_n$, the $n$-th
element of the stream $v$, and that doing so requires the state
$\phi_{in}$, which will become $\phi'_{in}$ during the execution. (The
unboxed formulas are more syntactic conditions we push along.)

The conclusion of this implication over triples, the second
boxed formula, says that reading $o$ will give us the $(f v)_n$, the
$n$-th output of the stream transducer, and that in doing so it will 
also update the input state $\phi_{in}$ to $\phi'_{in}$, in addition to 
sending the transducer state from $\phi_n$ to $\phi_{n+1}$. 
With this definition, we can prove the following specifications: 

\begin{prop}{(FRP Correctness)}
We define the $Relate$ predicate:
{\small
\begin{tabbing}
$Relate_{A,B}(p, f) \triangleq$ \\
\;\; $\forall \psi:\formula, i:\celltype{A}.$ \\ 
\qquad $\setof{G(\psi)}\;p\;i\;\{o:\celltype{B}.\;
               \exists \Phi.\;$\=$G(\Phi \otimes \psi) \;\land$ \\
                               \>$ Realize(i, \Phi, o, f) \valid\}$
\end{tabbing}
\noindent Then, the following specifications are provable: 
\begin{tabbing}
$\forall f:A\to B.\; Relate(\liftop f, \lifttrans\;f)$ \\[0.5em]

$\forall p, f, q, g.\;$\=$Relate(p, f) \specand Relate(q, g)$ \\
                       \>$\specimp Relate(\seqop p\;q, \seqtrans\;f\;g)$ \\[0.5em]

$\forall p, f, q, g.\;$\=$Relate(p, f) \specand Relate(q, g)$ \\
                       \>$\specimp Relate(\parop p\;q, \partrans\;f\;g)$ \\[0.5em]

$\forall k, p, f, q, g.\;$\=$Relate(p, f) \specand Relate(q, g)$ \\
                          \>$\specimp Relate(\switchop k\;p\;q, \switchtrans\;k\;f\;g)$ \\[0.5em]

$\forall a_0, p, f.\; 
  Relate(p, f) \specimp Relate(\loopop\;a_0\;p, loop\;a_0\;f)$
\end{tabbing}
}
\end{prop}

These lemmas permit us to reason about our transducer implementation
as if it is a pure implementation --- for each combinator in the
interface, we have a proof that shows the corresponding implementation
combinator lifts related arguments to related results. As a result, 
we can show that, for example,  $\seqop (\liftop f) \;(\liftop g)$
and $\liftop (g \circ f)$ both realize the same pure $\lifttrans\;(g\circ f)$. 

\section{Future Work}

We foresee many further applications of the idea of ramifications.

First, there are numerous algorithms --- such as unification, the
union-find algorithm, and the chaotic iteration constraint propagation
algorithm used in dataflow analysis --- which rely on using mutation
and assignment as a way of globally broadcasting information to the
rest of the program state. These algorithms have all been resistant to
modular proof, because of the apparent need to know ``the rest of the
world'' in the program invariant. It would be interesting to see if
ramifications can help.

Second, we would like to investigate the relationship between
ramification operators and methods based on
rely-guarantee~\cite{rely-guarantee-jones}. Rely-guarantee imposes a
mutual contract between a piece of code and the rest of the world.
This is conceptually similar to the idea of a ramification, though we
see no obvious direct relationship.

Third, we introduced ramifications as a style of
specification useful for verifying a particular library. Might it be
 useful to make ramification operators part of the basic
logical framework? If so, what are their logical properties? $R(u,
\phi)$ looks like a family of modal operators on the formula $\phi$, but
we needed a number of auxiliary interaction
lemmas to make them truly useful.

\section{Related Work}

Versions of separation logic~\cite{sep-logic} supporting higher-order
languages and quantification over predicates have been proposed by
\citet{htt} with Hoare Type Theory, and by
\citet{parkinson-bierman}. It would be interesting to adapt the
proof techniques in this paper to their settings.

Prior work on verifying the observer pattern using separation logic
includes work by \citet{tldi09,ftfjp07} and work by
\citet{parkinson-iwaco-07}. Similar techniques have also been applied
in the setting of regional logic by \citet{banerjee-naumann-regions}.

In all of these works, the program invariant explicitly tracks the
observers listening to a particular subject, and the invariants for
each observer. Thus there is an invariant for a cluster of objects
(subjects and observers). For a chain of observers only one level
deep, this works reasonably well, but it breaks down when there are
chains of dependencies --- when we have an object which both observes
and is observed by yet other objects. Roughly speaking, it is
difficult to locally add an observer to a subject, since we then need
to touch the invariants of everything the subject itself observes,
forcing us to know the transitive closure of the reachability graph of
object clusters from the subject. 

Even though all of these methods are modular in the sense that
subjects and observers can be verified independently, they are
non-modular in the sense that clients will have have to track entire
dependency graphs when verifying nested uses of the observer
pattern. In fact, our present work began when we realized that this
style of observer invariant made it difficult to verify the
model-view-controller pattern.

\citet{shaner-leavens-naumann} studied using gray-box model programs
to model higher-order method calls (which can be understood as a
variant of techniques from refinement calculus) in JML, and
\citet{barnett-naumann} added a friendship system to the Boogie
system, with which it is possible to describe some forms of clusters of
collaborating objects, including the subject-observer pattern. 
These works are also
non-modular in the above sense, since they require 
knowing what all of the observers are.


\citet{history-invariants} have applied the idea of
history invariants~\cite{liskov-wing} to model observers. The use of
monotonic predicates \emph{does} give rise to a modular
proof technique, but it sharply restricts the kinds of invariants that
can be used, in ways that make it very difficult to model the
code-update-based protocol seen in our FRP example. 

\citet{self-adjusting} have proposed \emph{self-adjusting computation}
as a technique for using change propagation to write programs that
incrementally recompute answers as the inputs are
adjusted. \citet{carlsson} showed how to build a monadic combinator library for
self-adjusting computation using techniques were strikingly similar
to the monadic library described in this paper. This gives us 
confidence that this is a natural implementation style, and makes us 
hope that ramifications can help in verifying implementations of this
technique.

FRP was proposed by \citet{frp} as a declarative formalism for
interactive programming. The API in our paper differs from theirs in
two ways. First, our interface is a variation of the \emph{arrowized
  FRP} interface proposed in \citet{afrp}, and secondly, we use a
discrete rather than continuous model of time --- though we found the
idea of using a declarative semantics as a specification for the
interface an inspiring one.  Our work could also serve as a bridge
between the work on purely functional and imperative implementations,
such as the work done by \citet{superglue} on SuperGlue and by
\citet{frtime} on the FrTime system.

\acks We would like to thank Peter O'Hearn for pointing out the connection
of our work with the ramification problem of AI. This work was
partially supported by the US NSF grant CCF0916808.

\bibliographystyle{plainnat}

\begin{thebibliography}{26}
\providecommand{\natexlab}[1]{#1}
\providecommand{\url}[1]{\texttt{#1}}
\expandafter\ifx\csname urlstyle\endcsname\relax
  \providecommand{\doi}[1]{doi: #1}\else
  \providecommand{\doi}{doi: \begingroup \urlstyle{rm}\Url}\fi

\bibitem[Acar et~al.(2006)Acar, Blelloch, and Harper]{self-adjusting}
U.A. Acar, G.E. Blelloch, and R.~Harper.
\newblock {Adaptive functional programming}.
\newblock \emph{ACM Transactions on Programming Languages and Systems
  (TOPLAS)}, 28\penalty0 (6):\penalty0 990--1034, 2006.

\bibitem[Banerjee et~al.(2008)Banerjee, 2, and
  Rosenberg]{banerjee-naumann-regions}
Anindya Banerjee, David A.~Naumann 2, and Stan Rosenberg.
\newblock Regional logic for local reasoning about global invariants.
\newblock In \emph{ECOOP}, pages 387--411, 2008.

\bibitem[Barnett and Naumann(2004)]{barnett-naumann}
Mike Barnett and David~A. Naumann.
\newblock Friends need a little bit more: Maintaining invariants over shared
  state.
\newblock In \emph{MPC}, pages 54--64, 2004.

\bibitem[Biering et~al.(2007)Biering, Birkedal, and Torp-Smith]{hosl}
Bodil Biering, Lars Birkedal, and Noah Torp-Smith.
\newblock {BI}-hyperdoctrines, higher-order separation logic, and abstraction.
\newblock \emph{ACM TOPLAS}, 29\penalty0 (5):\penalty0 24, 2007.
\newblock ISSN 0164-0925.
\newblock \doi{http://doi.acm.org/10.1145/1275497.1275499}.

\bibitem[Bird and Wadler(1988)]{bird-wadler}
R.~Bird and P.~Wadler.
\newblock \emph{{An introduction to functional programming}}.
\newblock Prentice Hall International (UK) Ltd. Hertfordshire, UK, UK, 1988.

\bibitem[Birkedal et~al.(2005)Birkedal, Torp-Smith, and
  Yang]{birkedal-torpsmith-yang-lics05}
L.~Birkedal, N.~Torp-Smith, and H.~Yang.
\newblock Semantics of separation-logic typing and higher-order frame rules.
\newblock In \emph{Proc. of LICS'05}, pages 260--269, 2005.

\bibitem[Birkedal et~al.(2009)Birkedal, St{\o}vring, and
  Thamsborg]{birkedal-impred}
Lars Birkedal, Kristian St{\o}vring, and Jacob Thamsborg.
\newblock Realizability semantics of parametric polymorphism, general
  references, and recursive types.
\newblock In Luca de~Alfaro, editor, \emph{FOSSACS}, volume 5504 of
  \emph{LNCS}, pages 456--470. Springer, 2009.
\newblock ISBN 978-3-642-00595-4.

\bibitem[Carlsson(2002)]{carlsson}
Magnus Carlsson.
\newblock Monads for incremental computing.
\newblock In \emph{ICFP}, pages 26--35, 2002.

\bibitem[Cooper and Krishnamurthi(2006)]{frtime}
Gregory~H. Cooper and Shriram Krishnamurthi.
\newblock Embedding dynamic dataflow in a call-by-value language.
\newblock In Peter Sestoft, editor, \emph{ESOP}, volume 3924 of \emph{LNCS},
  pages 294--308. Springer, 2006.
\newblock ISBN 3-540-33095-X.

\bibitem[Elliott and Hudak(1997)]{frp}
C.~Elliott and P.~Hudak.
\newblock {Functional reactive animation}.
\newblock In \emph{Proceedings of ICFP'97}, pages 263--273. ACM New York, NY,
  USA, 1997.

\bibitem[Finger(1987)]{finger}
J.~J. Finger.
\newblock \emph{Exploiting constraints in design synthesis}.
\newblock PhD thesis, Stanford University, Stanford, CA, USA, 1987.

\bibitem[Hudak et~al.(2002)Hudak, Courtney, Nilsson, and Peterson]{afrp}
Paul Hudak, Antony Courtney, Henrik Nilsson, and John Peterson.
\newblock Arrows, robots, and functional reactive programming.
\newblock In Johan Jeuring and Simon L.~Peyton Jones, editors, \emph{Advanced
  Functional Programming}, volume 2638 of \emph{LNCS}, pages 159--187.
  Springer, 2002.
\newblock ISBN 3-540-40132-6.

\bibitem[Jones(1983)]{rely-guarantee-jones}
Cliff~B. Jones.
\newblock Tentative steps toward a development method for interfering programs.
\newblock \emph{ACM Transactions on Programming Languages and Systems
  (TOPLAS)}, 5\penalty0 (4):\penalty0 596--619, 1983.

\bibitem[Krishnaswami et~al.(2007)Krishnaswami, Aldrich, and Birkedal]{ftfjp07}
N.~Krishnaswami, J.~Aldrich, and L.~Birkedal.
\newblock {Modular verification of the subject-observer pattern via
  higher-order separation logic}.
\newblock In \emph{Proceedings of FTfJP: Formal Techniques for Java-like
  Programs}, 2007.

\bibitem[Krishnaswami et~al.(2009)Krishnaswami, Aldrich, Birkedal, Svendsen,
  and Buisse]{tldi09}
N.~Krishnaswami, J.~Aldrich, L.~Birkedal, K.~Svendsen, and A.~Buisse.
\newblock {Design patterns in separation logic}.
\newblock In \emph{Proceedings of TLDI'09}, pages 105--116. ACM New York, NY,
  USA, 2009.

\bibitem[Krishnaswami(2009)]{tech-report}
Neelakantan~R. Krishnaswami.
\newblock \emph{Verifying Higher-Order Programming Languages with Higher-Order
  Separation Logic}.
\newblock PhD thesis, forthcoming. Carnegie Mellon University, Pittsburgh, PA,
  USA, 2009.

\bibitem[Leino and Schulte(2007)]{history-invariants}
K.~Rustan~M. Leino and Wolfram Schulte.
\newblock Using history invariants to verify observers.
\newblock In Rocco~De Nicola, editor, \emph{ESOP}, volume 4421 of \emph{LNCS},
  pages 80--94. Springer, 2007.
\newblock ISBN 978-3-540-71314-2.

\bibitem[Liskov and Wing(2001)]{liskov-wing}
Barbara~H. Liskov and Jeannette~M. Wing.
\newblock Behavioural subtyping using invariants and constraints.
\newblock In \emph{Formal Methods for Distributed Processing: a Survey of
  Object-Oriented Approaches}, pages 254--280. Cambridge University Press, New
  York, NY, USA, 2001.
\newblock ISBN 0-521-77184-6.

\bibitem[McCarthy and Hayes(1969)]{mccarthy}
John McCarthy and Patrick~J. Hayes.
\newblock Some philosophical problems from the standpoint of artificial
  intelligence.
\newblock In B.~Meltzer and D.~Michie, editors, \emph{Machine Intelligence 4},
  pages 463--502. Edinburgh University Press, 1969.

\bibitem[McDirmid and Hsieh(2006)]{superglue}
Sean McDirmid and Wilson~C. Hsieh.
\newblock Superglue: Component programming with object-oriented signals.
\newblock In Dave Thomas, editor, \emph{ECOOP}, volume 4067 of \emph{LNCS},
  pages 206--229. Springer, 2006.
\newblock ISBN 3-540-35726-2.

\bibitem[Nanevski et~al.(2006)Nanevski, Morrisett, and Birkedal]{htt}
Aleksandar Nanevski, Greg Morrisett, and Lars Birkedal.
\newblock Polymorphism and separation in hoare type theory.
\newblock In \emph{Proceedings ICFP}, pages 62--73, New York, NY, USA, 2006.
  ACM.
\newblock ISBN 1-59593-309-3.
\newblock \doi{http://doi.acm.org/10.1145/1159803.1159812}.

\bibitem[Parkinson(2007)]{parkinson-iwaco-07}
M.~Parkinson.
\newblock {Class invariants: The end of the road}.
\newblock \emph{Proceedings IWACO}, 2007.

\bibitem[Parkinson and Bierman(2008)]{parkinson-bierman}
Matthew~J. Parkinson and Gavin~M. Bierman.
\newblock Separation logic, abstraction and inheritance.
\newblock In George~C. Necula and Philip Wadler, editors, \emph{POPL}, pages
  75--86. ACM, 2008.
\newblock ISBN 978-1-59593-689-9.

\bibitem[Pfenning and Davies(2001)]{pfenning-davies}
Frank Pfenning and Rowan Davies.
\newblock A judgmental reconstruction of modal logic.
\newblock \emph{Mathematical Structures in Computer Science}, 11\penalty0
  (4):\penalty0 511--540, 2001.
\newblock ISSN 0960-1295.

\bibitem[Reynolds(2002)]{sep-logic}
John~C. Reynolds.
\newblock Separation logic: A logic for shared mutable data structures.
\newblock In \emph{Logic in Computer Science (LICS 2002)}, pages 55--74. IEEE
  Computer Society, 2002.
\newblock ISBN 0-7695-1483-9.

\bibitem[Shaner et~al.(2007)Shaner, Leavens, and
  Naumann]{shaner-leavens-naumann}
Steve~M. Shaner, Gary~T. Leavens, and David~A. Naumann.
\newblock Modular verification of higher-order methods with mandatory calls
  specified by model programs.
\newblock In \emph{OOPSLA}, pages 351--368, 2007.

\end{thebibliography}

% \bibliography{ramified-frames}{}

% \bibliographystyle{plainnat}
% 
% \begin{thebibliography}{}
% 
% \bibitem{smith02}
% Smith, P. Q. reference text
% 
% \end{thebibliography}

\appendix

\section{Appendix: The $\cellset$ Interface}

In this section we describe the interface to the $\cellset$ type,
representing pure collections of existentially quantified cells
(terms of type $\exists \alpha:\star.\;
\celltype{\alpha}$). Specifying this interface is not entirely
trivial, because of the way equality works for this type. 

Ordinarily, we give a two-place predicate $set(v, elts)$ relating a
value $v$, and the mathematical set of elements $elts$ it represents.
However, this approach is not sufficient in our case. In order to
manage dependencies, we need to be able to test cells of
\emph{different} concrete type for equality, and the natural equality
for references only permits testing references of the same type. As a
result, we cannot unpack an existentially-quantified cell and compare
the elements in its tuple, because we do not know that the two cells
are of the same type (and indeed, they might not be). 

To deal with this problem, we assign a unique integer identifier to
each cell we create, and compare those identifiers to establish
equality. Since these identifiers are generated dynamically along with
the cells, the precise partial equivalence relation we need to use is
determined dynamically as well. So we add an additional index to the
set predicate $set(W, v, elts)$. The extra parameter $W$ is the
\emph{world}, the set of all the cells allocated so far, whose
elements must be equal if and only if their identifier fields match.

The usual set operations are supported: the expression
$\ctext{emptyset}$ represents an empty set of cells; 
$\ctext{addset}\;v\;x$ adds the element $x$ to the set $v$
represents; and $\ctext{removeset}\;v\;x$ removes $x$ from the
set $v$ represents. We also have $\ctext{iterset}\;f\;v$, which
iterates over the elements of $v$'s set and applies $f$ to each
element in some sequential order. (The specification
makes use of two auxiliary predicates: $\mathit{matches}$, which assert that a
set and a list have the same elements; and $\mathit{iterseq}$, which constructs
a command representing the sequential execution over those elements.)

We have three axioms any implementation must satisfy. First, if a
$\cellset$ value $v$ represents a set $elts$ in a world $W$, it must
also represent a set in any larger world $W'$ --- i.e., allocating
new cells cannot disturb already-existing cells.  Second, the values
in a set are always a subset of the world $W$.  Third, we require
that $set(W,v,elts)$ is a pure predicate (i.e., is not
heap-dependent). That is, we ask that $\cellset$ have a purely
functional implementation (for example, as a binary tree). This is not
necessary, but does simplify the other invariants in this paper.

{\small
\begin{specification}
\nextline
$World = $ \nextline
\;\;\;\;\=$\{D \in \powersetfin{ecell}\;|\;\forall c,d \in D.\;$\=$unique(c) = unique(d)$\nextline
        \>                                                      \>$\iff c = d\}$\nextline[0.5em]

$\exists \cellset : \star.$ \nextline
$\exists set : World \To \cellset \To \powersetfin{ecell} \To \assert.$ \nextline
$\exists \ctext{emptyset}    : cellset.$ \nextline
$\exists \ctext{addset}      : \cellset \to ecell \to \cellset.$ \nextline
$\exists \ctext{removeset}   : \cellset \to ecell \to \cellset.$ \nextline
$\exists \ctext{iterset}     : \cellset \to (ecell \to \monad{\unittype}) \to \unittype.$\nextline[0.5em]

$\setof{\forall W \in World.\; set(W, \ctext{emptyset}, \emptyset)}$ 

\specand \nextline[0.5em]

$\{\forall W \in World, v : \cellset, x : ecell, elts \in \powersetfin{ecell}.$ \nextline 
\> $set(W, v, elts) \land x \in W \implies set(W, \ctext{addset}\;v\;x, elts \cup \setof{x})\}$ \specand \nextline[0.5em]


$\{\forall W \in World, v : \cellset, x : ecell, elts \in \powersetfin{ecell}.$ \nextline 
\> $set(W, v, elts) \land x \in W \implies set(W, \ctext{removeset}\;v\;x, elts - \setof{x})\}$ \specand \nextline[0.5em]


$\{\forall W \in World, v : \cellset, elts \in \powersetfin{ecell}, 
         f : (ecell \to \monad{\unittype}).$ \nextline 
\> $set(W, v, elts) \implies \exists L : \seqsort{ecell}.\;$\=$matches\;elts\;L\; \land$ \nextline
\>                                \>$\ctext{iterset}\;f\;v = iterseq\;f\;L\}$ \specand \nextline[0.5em]


$\{\forall W, W' \in World, v, elts.$ \nextline
\>$set(W,v,elts) \land W \subseteq W' \implies set(W',v, elts)\}$\specand \nextline[0.5em]

$\{\forall W, W' \in World, v, elts.$ \nextline
\>$set(W, v, elts) \implies elts \subseteq W\}$ \specand \nextline[0.5em]

$\{\forall W, v, elts.\; \mathit{Pure}(set(W,v,elts))\}$ \nextline[0.5em]
  

$\mathit{matches}\;elts\;[] \qquad\;\;\; = elts = \emptyset$ \nextline
$\mathit{matches}\;elts\;(v :: vs) = v \in elts \land \mathit{matches}\;(elts - \setof{v})\;vs$ \nextline[0.5em]

$\mathit{iterseq}\; f\; [] \qquad\;\;\;\;$\=$= \comp{\unit}$ \nextline
$\mathit{iterseq}\; f\; (v :: vs)$\>$= \comp{\letv{\unit}{f\;v}{\ctext{run}\;\mathit{iterseq}\;f\;vs}}$ 
\end{specification}
}

\end{document}

