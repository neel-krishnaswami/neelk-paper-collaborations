\documentclass{article}
\usepackage{mathpartir}
\usepackage{amsmath}
\usepackage{amssymb}

\newcommand{\ST}[2]{{#1} \mapsto {#2}}
\newcommand{\List}[1]{\mathsf{List}\;{#1}}
\newcommand{\Set}{\mathrm{Set}}
\newcommand{\I}{\mathbf{I}}
\newcommand{\Hom}{\mathrm{Hom}}
\newcommand{\Event}{\mathsf{Event}}
\newcommand{\Events}{\Event^{\omega}}
\newcommand{\To}{\Rightarrow}
\newcommand{\setof}[1]{\left\{{#1}\right\}}
\newcommand{\comprehend}[2]{\setof{{#1}\;|\;{#2}}}
\newtheorem{prop}{Proposition}
\newtheorem{definition}{Definition}

\title{A New Semantics of Functional Reactive Programs}

\author{Neelakantan R. Krishnaswami \texttt{<neelk@microsoft.com>}}

\begin{document}
\maketitle


\section{Functional Reactive Programming}

Traditional functional reactive programming is programming with \emph{stream transducers}. That is,
we introduce a type $\ST{A}{B} \equiv A^\omega \to B^\omega$. From this, we can define an API something
like:

\begin{mathpar}
  \begin{array}{lcl}
    \mathtt{par} & : & (\ST{A}{B}) \to (\ST{X}{Y}) \to (\ST{A \times X}{B \times Y}) \\
    \mathtt{seq} & : & (\ST{A}{B}) \to (\ST{B}{C}) \to (\ST{A}{C}) \\
    \mathtt{map} & : & (A \to B) \to (\ST{A}{B}) \\
    \mathtt{loop} & : & A \to (\ST{A \times B}{A \times C}) \to (\ST{B}{C}) \\
  \end{array}
\end{mathpar}

When we try to define the semantics of loop, we realize that we didn't
mean \emph{all} stream functions, but rather only the \emph{causal} ones. These
are functions that ``depend only their history''. 

This idea can be formalized as follows: 

\begin{prop}{(Causality)}
  A function $f \in A^\omega \to B^\omega$ is causal, if for all streams $a, a' \in A^\omega$ and all $n$ 
  \begin{mathpar}
    (\forall i \leq n.\; a_i = a'_i) \implies (\forall i \leq n.\; (f\;a)_i = (f\;a')_i)
  \end{mathpar}
\end{prop}
%
This captures the idea the value of the first $n$ elements of $f\;a$ can only depend on at most the 
first $n$ elements of $a$, since we are free to vary $a$ and $a'$ past $n$. 

Now, this is a perfectly nice definition and allows giving a sensible
semantics to $\mathtt{loop}$. But it has several problems:

\begin{itemize}
\item There can be false sharing in the input.
\item The type of stream transducers is a ``first-order function type'', which limits our
  opportunities for abstraction.
\item The $\mathtt{loop}$ combinator seems a bit \emph{ad-hoc}. 
\end{itemize}

The question is, can we do better?

\section{Ultrametrics}

A \emph{1-bounded ultrametric} $(X, d)$ is a set $X$ paired with a
distance function $d \in X \times X \to \mathbb{R}$, satisfying the
following properties:

\begin{itemize}
\item $d(x, y) \in [0,1]$
\item $d(x, x) = 0$
\item $d(x, y) = d(y, x)$
\item $d(x, z) \leq \max\setof{d(x,y), d(y, z)}$
\end{itemize}

What makes it an \emph{ultra}metric is the fact that the triangle inequality is stronger than the normal
Euclidean triangle inequality. 

The category of ultrametric spaces (actually, complete 1-bounded ultrametric spaces) is cartesian closed. 

\begin{itemize}
\item The objects of this category are ultrametric spaces.
\item The morphisms are ``nonexpansive maps''. A nonexpansive map $f : (X, d_X) \to (Y, d_Y)$ is a function 
from $X \to Y$ with the property that it is non-distance-increasing:
\begin{mathpar}
  \forall x, x'.  d_Y(f\;x, f\;x') \leq d_x(x, x')
\end{mathpar}

\item The product of two ultrametric spaces $(X, d_x)$ and $(Y, d_Y)$ is the space $(X \times Y, d_{X\times Y})$, 
where 
\begin{mathpar}
  d_{X\times Y}((x,y), (x',y')) = \max \setof{d_X(x,x'), d_Y(y,y')}
\end{mathpar}
This is called the ``sup-metric''. 

\item The exponential $X \to Y$ is the set of nonexpansive maps from $X \to Y$, with the distance given by:
  \begin{mathpar}
    d_{X \to Y}(f, g) = \max \comprehend{d_Y(f\;x, g\;x)}{x \in X}
  \end{mathpar}
\end{itemize}


\subsection{Streams as Ultrametric Spaces}

The type of streams $A^\omega$ can be equipped with an ultrametric distance function: 

\begin{mathpar}
  d_{A^\omega}(a, a') = 2^{-\min\comprehend{n \in \mathbb{N}}{a_n \not= a'_n}}
\end{mathpar}

So the distance between two streams measures ``how soon'' they differ from one another. Two streams
that never differ from each other will have a distance of 0, and two streams that differ at the first
element will have a distance of 1. 

Verifying that this forms an ultrametric is straightforward. 

Now, the fun part is the following theorem: 

\begin{prop}{(Causality Coincides with the Ultrametric Function Space)}
The function $f \in A^\omega \to B^\omega$ is causal if and only if it
is a nonexpansive map according to the distance metric for streams.
\end{prop}

\noindent The proof is largely just unfolding some definitions. 

But despite its easiness, it's still really useful! It means that we now have a firm base to 
generalize FRP from. That is, we now interpret transducers as nonexpansive maps $A^\omega \to B^\omega$,
and so we can easily generalize t

\begin{itemize}
\item False sharing goes away. We can now have a sensible interpretation to $A^\omega \times B^\omega \to C^\omega$,
which lets us write functions that (for example) don't necessarily assume that the $A$s can depend on the $B$s.

\item We can curry! We now know what $A^\omega \to (B^\omega \to C^\omega)$ means, for example! 

\item The syntax of a FRP DSL greatly simplifies: it's the lambda calculus. (Actually, we need a bit more
  than this to handle feedback.) So we have a coherent, familiar story about variables and binding and
  so on. 

  Q: Can we get Simon to let us rebind application, variable references, and abstractions in GHC?
\end{itemize}

\subsection{Feedback}

One of the more fun things about functional reactive programming is \emph{feedback}. A stream's value at
time $t$ can depend on its values at earlier times. For example, consider \texttt{(fix xs. 0 :: map (+1) xs)}. 

These kinds of definitions are pervasive in reactive programming, and are probably the most distinctive
thing about it. It would be nice to have a clean, general account of them. Luckily, ultrametrics help here,
too. 

\begin{definition}{(Strictly Contractive Maps)}
  An endofunction on ultrametrics $f : A \to B$ is \emph{strictly contractive} when there is a $q \in [0,1)$
  such that for all $a, a' \in A$, we have that $d_B(f\;a, f\;a') \leq q \cdot d_A(a, a')$. 
\end{definition}

The idea behind a strictly contractive map is that applying it shrinks
the distance between any two points. (A nonexpansive map can be seen as the case when $q$ is 1.)
This suggests that when we have an endofunction, we can find fixed points. 

\begin{prop}{(Banach's Fixed Point Theorem)}
If $A$ is a complete metric space, and $f : A \to A$ is a strictly contractive map, then $f$ has a
\emph{unique} fixed point in $A$. 
\end{prop}

Now, a delay operator like \texttt{cons} (of type $A \to A^\omega \to
A^\omega$) is strictly contractive in its second argument. This is
because it adds an initial value at time 0, and so if two arguments disagreed at
time $n$, then the results will disagree at time $n+1$ -- which means the distance
between the result is  $\frac{1}{2}$ the distance of the arguments. 

Furthermore, since strictly contractive functions are preserved by
composition, application, pairing, and basically everything else, this
means that \texttt{$\lambda$xs. 0 :: map (+1) xs} is also strictly
contractive, and so by Banach's fixed point theorem we know it has a
fixed point.

Some interesting issues in implementing these fixed points, both in
the API and in the implementation. 

\section{Imperative Streams}

Define the category $\I$ as follows. The objects of $\I$ are finite
sets $D$ -- think of the elements of $D$ as a set of names for input
channels.  The $\I$-morphisms $f : D \to D'$ are injective maps from
$D \to D'$. Intuitively, the maps are embeddings up to renaming.

Next, we'll consider the functor category $[\I, \Set]$. The objects of this
category are functors from $\I$ to $\Set$, and the morphisms are the natural
transformations. That is,
\begin{mathpar}
\Hom(F, G) = \comprehend{\eta \in \Pi X \in\I_0.\; F(X) \to G(X)}
                        {\forall X,Y\in \I_0, f \in X \to Y.\; \eta_X;G f = F f;\eta_Y}
\end{mathpar}

Products and coproducts are given pointwise. The exponential $F \To G$ is also definable.
Its action on objects is:
\begin{mathpar}
(F \To G)(X) = \Hom(\I(X, -) \times F, G)
\end{mathpar}

Intuitively, this is a generalization of implication in Kripke
semantics. Recall that in a Kripke semantics, the implication $p
\implies q$ holds at a world $\tau$ when for all $\tau \sqsubseteq
\sigma$, if $p$ holds at $\sigma$, then $q$ holds at $\sigma$.
This is a uncurried version of that idea, generalized to reflect
the possibility that in $\I$, there may be multiple distinct ways
for one element to be larger than another. (More concretely, note that
$\I(X, -)$ is nonempty only when there is a map from $X$ --- and $X$
is injective.)

We now define a monad which supports reading and dynamic allocation,
whose action on objects is:
\begin{mathpar}
  T(A) (X) = \exists D.\; \Pi Y.\; \left(\I(X + D, Y) \times (Y \to \Events)) \to A(Y)
\end{mathpar}

\noindent (The $X + D \to Y$ is an injection.)

The action on morphisms is
\begin{mathpar}
  T(A)(f : X \to Y) = \lambda(D, \eta).\;(D, \lambda Z.\;\lambda (e,h).\; \eta_Z(e \circ [f, id_D], h)
\end{mathpar}

This is a strong commutative monad, which remains to be checked. 

Now, to be ambitious, we'll try to use Nick's construction of a adjoint
model from a commutative monad. (Then we'll try and do it using ultrametric-valued
covariant presheaves, whee!)

This means that we need to consider the Eilenberg-Moore category $[\I,\Set]^T$. 

The objects of this category are pairs $(F, u)$, where $F \in [\I,\Set]$, and 
$u : T(F) \to F$, such that $u$ is a $T$-action. (A $T$-action is a morphism 
such that $u \circ T(u) = u \circ \mu : T(T(F)) \to F$, and $u \circ \mu = id_F$.)

The morphisms in $\Hom((F,u), (G,v))$ are maps $\eta : F \to G$ such
that $\eta \circ u = v \circ T(\eta)$.

Now let's play the game ``unwind the definitions''. 

\end{document}
        
