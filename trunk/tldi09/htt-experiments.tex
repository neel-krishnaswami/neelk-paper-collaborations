
%
% In the tech report: include a link to the Ynot sources!
%


\section{Ynot Experiments}
\label{sec:ynot-experiments}

In this section we give a brief description of our experiments with
translating the design pattern specifications and implementations from
the earlier sections into Hoare Type Theory (HTT) and verifying them in the
Ynot implementation of HTT. More details can be found in the 
technical report~\cite{svendsen08}.

These experiments serve to (1) increase our confidence in the earlier given
specifications and implementations and their associated paper proofs; (2)
provide a starting point for a comparison of the specification logic of the
present paper and the Ynot type theory; (3) exercise the Ynot
implementation.

Ynot is an axiomatic extension to the Coq proof assistant, that supports
writing, reasoning about, and extracting higher-order, dependently-typed
programs with side-effects~\cite{nanevski08}.  Coq already includes a
powerful functional language that supports dependent types, but that
language is limited to pure, total functions.  Ynot extends Coq with
support for computations that may have effects such as non-termination,
accessing a mutable store, and throwing/catching exceptions.  The axioms of
Ynot form a small trusted computing base which has been formally justified
in previous work on Hoare Type Theory (HTT)~\cite{nanevski06separation,
  nanevski07esop, petersen08}.

As in the specification logic described in the earlier sections, Ynot
also makes use a monads, to ensure a monadic separation of effects and
pure Coq.  In Ynot specifications are types and one of the types is
the monadic type of computations $\{ P \} x:\tau \{ Q \}$; if a
computation has this type and it is run in a heap $i$ satisfying $P$
and it terminates, then it will produce a value $x$ of type $\tau$ and
result in a new heap $j$ such that the predicate $Q(i,j)$
holds. Loosely speaking, we may thus think of Ynot as a type theory
corresponding to the specification logic for Idealized ML presented
earlier under a Curry-Howard style correspondence.  Following this
intuition, we have experimented with translating the earlier described
design pattern specifications and implementations into Ynot and
formally verified them in Ynot.  We now describe the translations and
the lessons learned.

Note first, however, that in Ynot post-conditions are expressed in terms of
both the initial and the final heap. This is an alternative to the use of
logical variables as expressed by universally quantifying over variables
whose scope extends to both the pre- and post-condition.  We can thus
translate an Idealized ML specification,
{\small\begin{align*}
\forall x : \tau.\ \{ P(x) \}\ comp\ \{ a : 1.\ Q(x) \}
\end{align*}}%
into the following Ynot type
{\small\begin{align*}
&\{ \lambda i : \HEAP.\ \exists x : \tau.\ P\ x\ i \}\\
&\quad a : 1\\
&\{ \lambda i : \HEAP.\ \lambda j : \HEAP.\ \forall x : \tau.\ P\ x\ i
\rightarrow Q\ x\ j \}
\end{align*}}%
where $i$ is the initial heap and $j$ is the terminal heap. We will usually
abbreviate this type as follows: 
{\small\begin{align*}
\{ i.\ \exists x.\ P\ x\ i \}\ a : 1\ \{ i\ j.\ \forall x.\ P\ x\ i \rightarrow Q\ x\ j \}
\end{align*}}

\subsection{Flyweight in Ynot}
\label{sec:ynot-flyweight}

Besides Hoare triples, Idealized ML's specification language contains
specifications of the form $\{ P \}$, for asserting that $P$ is true. In the
flyweight specification this is used to express that calling $getdata$ with the
same character multiple times, produces the same reference. In HTT we can
express that a proposition $P$ is true by returning an element of
the subset type, $\{ x : 1 \mid P \}$, where $x$ is not free in $P$. 

The assertion language of Idealized ML also contains an expression for
asserting that a given specification holds. In the Flyweight specification
this is used in the post-condition of $make\_flyweight$, to assert that the
code returned implements a Flyweight. In HTT, we can express the same by
simply giving a more precise type for the return value of the
$make\_flyweight$ computation.

In the Ynot implementation, we have generalized the specification, such that
the computation can generate a flyweight for values of an arbitrary monotype.
The flyweight factory computation therefore also has to take as an argument, a
function, $\alpha_{eq}$, for deciding equality between $\alpha$ values. 

The rest of the specification can be translated almost directly into HTT,
however, we have made a few changes, to simplify the formal verification of the
implementation in Ynot. 
\begin{itemize}
\item In the specification of $newchar$, instead of using a set to associate
arguments with objects, we have used a partial function (i.e., a total
function from $\alpha$ to $\opttype\ \LOC$).
\item In the above specification the predicate $I$ has to specify the
representation of both the object table and the objects. We have
split $I$ into two predicates, $table$ and $refs$, and changed the precondition
of $newchar$ to the HTT equivalent of $table(...) * (refs(...) \land
chars(S))$, to make it explicit that the object table and the objects
are in separate subheaps, to simplify verification.
\end{itemize}
The final HTT type of the Flyweight factory thus looks as follows:
{\small\begin{align*}
&\Pi \alpha : \MONO.\ \Pi \alpha_{eq} : (\Pi x : \alpha.\ \Pi y : \alpha.\ \{ z : 1 \mid x = y \} + \{ z : 1 \mid x \neq y \}).\\
&\{ i.\ emp\ i \}\\
&\quad r : \Sigma table : (\alpha \rightarrow \opttype\ \LOC) \rightarrow \HEAP \rightarrow \PROP.\\
&\quad\quad \Sigma refs : (\alpha \rightarrow \opttype\ \LOC) \rightarrow \HEAP \rightarrow \PROP.\\
&\quad\quad \Sigma objat : \LOC \rightarrow \alpha \rightarrow \HEAP \rightarrow \PROP.\\
&\quad\quad\Sigma prf_1 : \{ x : 1 \mid \forall h, l, l', a, a', f.\ objat\ l\ a\ h \land objat\ l'\ a'\ h\ \land\\
&\quad\quad\quad\quad\quad\quad\quad\quad\quad\quad refs\ f\ h \rightarrow (l = l' \leftrightarrow a = a') \}.\\
&\quad\quad \Pi a : \alpha.\\
&\quad\quad\quad \{ i.\ \exists f.\ (table\ f * (\lambda h.\ allobjat(\alpha, objat, f, h) \land refs\ f\ h))\ i \}\\
&\quad\quad\quad\quad l : \LOC\\
&\quad\quad\quad \{ i\ j.\ \forall f.\ (table\ f * (\lambda h.\ allobjat(\alpha, objat, f, h) \land refs\ f\ h))\ i \rightarrow\\
&\quad\quad\quad\quad\quad ((\forall l'.\ f\ a = Some\ l' \rightarrow l = l')\ \land\\
&\quad\quad\quad\quad\quad (table\ f[a \mapsto l] * (\lambda h.\ allobjat(\alpha, objat, f[a \mapsto l], h)\ \land\\
&\quad\quad\quad\quad\quad\quad\quad\quad\quad\quad\quad\quad\quad\quad\quad\quad refs\ f[a \mapsto l]\ h))\ j) \}\ \times\\
&\quad\quad \Pi l : loc.\\
&\quad\quad\quad \{ i.\ \exists a : \alpha,\ objat\ l\ a\ i \}\\
&\quad\quad\quad\quad r : \alpha\\
&\quad\quad\quad \{ i\ j.\ \forall a : \alpha,\ objat\ l\ a\ i \rightarrow (i = j \land r = a) \}\\\
&\{ i\ j.\ ((fst\ r)\ []\ * (\lambda h.\ allobjat(\alpha, fst\ (snd\ (snd\ r)), [], h)\ \land\\
&\quad\quad\quad\quad\quad\quad\quad\quad\quad\quad\quad (fst\ (snd\ r))\ []\ h))\ j \}
\end{align*}}
where
\begin{align*}
allobjat(\alpha, objat, f, h) &\equiv \forall l : \LOC, o : \alpha.\ f\ o = Some\ l\ \rightarrow\\&\quad\quad\quad (objat\ l\ o * (\lambda h.\ \TRUE))\ h
\end{align*}
and $[] \equiv (\lambda x.\ None)$.

We were able to formally verify that the earlier given implementation of
the Flyweight pattern has the above type in Ynot.\footnote{The Coq script was 
780 lines long.}

\subsection{Iterators in Ynot}
\label{sec:iterators-in-ynot}

The translation of the Iterator specification and implemenation and the
formal verification in Ynot was straightforward, except for the
verification of {\tt next}, when the iterator is a {\tt Map2}.\footnote{The
Coq script was 2122 lines long.}  In that case, the implementation makes
two recursive calls to {\tt next} that each work on two subheaps of the
initial heap and the current Ynot implementation based on the $nextvc$
tactic (see~\cite{nanevski08}) for simplifying proof obligations forces one
to prove some preciseness properties because of the use of binary
post-conditions. It is unclear whether the preciseness problem encountered
is a limitation of binary post-conditions in general or the current Ynot
implementation; we think it is the latter. We did not finish the proof for
{\tt next} in this case, either with or without $nextvc$ (without $nextvc$
the proof became too long for us to finish by hand).  New versions of Ynot
should provide better tactic support for such examples. 
We did succeed in completing the formal verification of the
iterator pattern  without the {\tt Map2} iterator.

\subsection{Subject-Observers in Ynot}
\label{sec:subject-observer-in-ynot}

The Idealized ML implementation of the subject-observer pattern uses an
assertion $S\valid$ in the predicate $good$ to express that callback functions
are "good". HTT does not support this form of assertion. However, since
specifications are types, we can express the type of pairs of $(O, f)$ such
that $f$ is a good call function with respect to the predicate $O$ with the
following type: 
{\small\begin{align*}
T &\equiv \Sigma O : \N \rightarrow \HPROP.\\
&\quad\quad (\Pi m : \N.\ \{ i.\ \exists k : \N.\ O\ k\ i\}\ a : 1\ \{ i\ j.\ \forall k : \N.\ O\ k\ i \rightarrow O\ m\ j \})
\end{align*}}
Hence, we can restrict the quantification of $os$ in the specification of
$broadcast$ and $register$ to lists of good callback functions, $\listtype T$.
We can thus express the subject-observer pattern with the following HTT type:
{\small
\begin{align*}
&\Sigma \alpha : \TYPE.\ \Sigma sub : \alpha \times \N \times \listtype T
\rightarrow \HPROP.\\
&\quad \Pi n : \N.\ \{ i.\ emp\ i \} a : \alpha \{ i\ j.\ sub\ (a, n, [])\ j \}\ \times\\
&\quad \Pi a : \alpha.\ \Pi t : T.\\
&\quad\quad\{ i.\ \exists n : \N, os : list\ T. sub\ (a, n, os)\ i \}\\
&\quad\quad\quad r : 1\\
&\quad\quad \{ i\ j.\ \forall n : \N, os : list\ T.\ sub\ (a, n, os)\ i \rightarrow sub\ (a, n, t::os)\ j
\}\ \times\\
&\quad \Pi a : \alpha.\ \Pi m : \N.\\
&\quad\quad\{ i.\ \exists n : \N, os : list\ T.\ (sub\ (a, n, os) * obs\ os)\ i \}\\
&\quad\quad\quad r : 1\\
&\quad\quad \{ i\ j.\ \forall n : \N, os : list\ T.\ (sub\ (a, n, os) * obs\ os)\ i\\
&\quad\quad\quad\quad\quad \rightarrow (sub\ (a, m, os) * obs\_at\ (os, k))\ j \}
\end{align*}} In the Idealized ML implementation of the subject-observer,
the registered callback functions are stored in the heap. Since types and
specifications are separate in Idealized ML, the type of these computations
can be very weak, i.e., $\N \rightarrow \monad 1$, because the
specification language allows us to express that if these are "good"
callback functions then the broadcast computation will do a broadcast when
performed. In HTT there is no separate specification language, so these
callback functions have to be stored with a much stronger type, so that it
is possible to infer from their type that they are "good" callback
functions when they are retrieved from the heap.\footnote{In the technical
  report~\cite{svendsen08} we discuss an alternative translation into HTT
  based on the idea that implications between specifications in Idealized
  ML should be translated into function types in HTT, but that leads to
  an implementation that does not capture subject-observer pattern because
  it essentially results in a functional implementation.}

Ynot is based on the predicative version of HTT\cite{nanevski06separation} in
which dependent sums are predicative, i.e., for $\Sigma x : A. B$ to be a
monotype, both $A$ and $B$ have to be monotypes. Since the type of heaps in Ynot is defined as a subset of the type $\N \times \Sigma \alpha : \MONO.
\alpha$ and $\MONO$ is not a monotype, it follows that the $T$ type above is not a monotype either and that values of type $T$ cannot be stored in
the heap. It is thus unclear whether it is possible to give an implementation
of the above type in Ynot; the obvious attempt leads to a universe
inconsistency error in Coq, reflecting the predicativity issues just
discussed. 

The impredicative version of HTT \cite{petersen08} has an impredicative sum
type, $\Sigma^T x : A. B$, which is a monotype if $B$ is. Hence, in the
impredicative version of HTT, we can store values of type $T$, by using
impredicative sums. We conjecture that the implementation derived from
translating the Idealized ML implementation has the above 
type in impredicative HTT.



%%% Local Variables: 
%%% mode: latex
%%% TeX-master: patterns
%%% End: 
