\documentclass{beamer}
\usepackage{mathpartir}
\usepackage{amsmath}
\usepackage{amssymb}
\usepackage{xcolor}
\newcommand{\ctext}[1]{\mbox{\textsf{{#1}}}}
\newcommand{\unittype}{\mathsf{unit}}
\newcommand{\reftype}[1]{\ctext{ref }#1}
\newcommand{\monad}[1]{\bigcirc{#1}}
\newcommand{\cont}[1]{\ctext{cont}\;{#1}}
\newcommand{\opttype}[1]{\ctext{option }{#1}}
\newcommand{\seqsort}[1]{\ctext{seq }{#1}}
\newcommand{\listtype}[1]{\ctext{list }{#1}}

\newcommand{\pair}[2]{\left<{#1},{#2}\right>}
\newcommand{\fst}[1]{\ctext{fst } #1}
\newcommand{\snd}[1]{\ctext{snd } #1}

\newcommand{\inj}[1]{\iota_{#1}}
\newcommand{\inl}[1]{\ctext{inl }#1}
\newcommand{\inr}[1]{\ctext{inr }#1}
\newcommand{\Case}[5]{\ctext{case}(#1,\; {#2}.\; #3,\; {#4}.\; #5)}
\newcommand{\listcase}[5]{\ctext{case}(#1,\;\ctext{Nil} \to {#2},\;
                                       \ctext{Cons}({#3},{#4}) \to {#5})}
\newcommand{\Some}{\ctext{Some}}
\newcommand{\optcase}[4]{\ctext{case}(#1,\;
                                      \ctext{None} \to {#2},\;
                                      \ctext{Some}\;{#3} \to {#4})}
\newcommand{\z}{\ctext{z}}
\newcommand{\s}[1]{\ctext{s}(#1)}
\newcommand{\iter}[4]{\ctext{iter}(#1, {#2}, {#3}.\; {#4})}
\newcommand{\iterseq}[4]{\ctext{iter}_{\mathsf{seq}}(#1, {#2}, #3.\; {#4})}
\newcommand{\comp}[1]{[#1]}
\newcommand{\fun}[3]{\lambda #1:#2.\;#3}
\newcommand{\Fun}[3]{\Lambda #1:#2.\;#3}
\newcommand{\unit}{\left<\right>}
\newcommand{\pack}[2]{\ctext{pack}({#1}, {#2})}
\newcommand{\unpack}[4]{\ctext{unpack}({#1}, {#2}) = {#3} \;\ctext{in}\;{#4}}

\newcommand{\letv}[3]{\ctext{letv }#1 = #2\ctext{ in }#3}
\newcommand{\newref}[2]{\ctext{new}_{#1}(#2)}
\newcommand{\run}[1]{\ctext{run }{#1}}
\newcommand{\ok}{\ctext{ ok}}
\newcommand{\FV}[1]{\mathrm{FV}({#1})}

\newcommand{\fix}[2]{\ctext{fix}\;{#1}.\;{#2}}

\newcommand{\statecfg}[2]{\left<{#1};\;{#2}\right>}
\newcommand{\eval}[4]{\left<{#1};\;{#2}\right> \leadsto \left<{#3};\;{#4}\right>}
\newcommand{\evalabort}[2]{\left<{#1};\;{#2}\right> \leadsto \mathbf{abort}}

\newcommand{\domain}[1]{\mbox{dom}({#1})}
\newcommand{\upset}[1]{\mathcal{P}^{\uparrow}({#1})}

\newcommand{\pointsto}{\mapsto}
\newcommand{\disj}{\vee}
\renewcommand{\implies}{\supset}
\newcommand{\wand}{-\!\!*}
\newcommand{\emp}{\mathsf{emp}}
\newcommand{\validprop}[1]{{#1}\ctext{ valid}}

\newcommand{\todo}[1]{\texttt{[TODO: {#1}]}}

\newcommand{\setof}[1]{\{{#1}\}}

\newcommand{\To}{\Rightarrow}

\newcommand{\N}{\mathbb{N}}

\newcommand{\assert}{\ctext{prop}}

\newcommand{\bigstep}{\Downarrow}

\newcommand{\bigeval}[4]{\left<{#1};\;{#2}\right> \bigstep \left<{#3};\;{#4}\right>}
\newcommand{\bigevalabort}[2]{\left<{#1};\;{#2}\right> \bigstep \mathbf{abort}}

\newcommand{\spec}[4]{\{{#1}\}{#2}\{{#3}.\;{#4}\}}
\newcommand{\specX}[3]{\{{#1}\}{#2}\{{#3}\}}
\newcommand{\mspec}[4]{\langle{#1}\rangle{#2}\langle{#3}.\;{#4}\rangle}
\newcommand{\bnfalt}{\;\;|\;\;}

\newcommand{\specor}{\ctext{ or }}
\newcommand{\specand}{\;\ctext{and}\;}
\newcommand{\specimp}{\ctext{ implies }}
\newcommand{\spectype}{\ctext{spec}}
\newcommand{\valid}{\ctext{ valid}}

\newcommand{\interp}[1]{[\![{#1}]\!]}
\newcommand{\interpE}[1]{\interp{#1}^e}
\newcommand{\interpC}[1]{\interp{#1}^c}

\newcommand{\interpF}[1]{[\![{#1}]\!]_f}
\newcommand{\interpmono}[1]{\interp{#1}^{\mathrm{m}}}

\newcommand{\entails}{\models}

\newcommand{\judgeE}[4][\Theta]{{#1};\;{#2} \vdash {#3} : {#4}}
\newcommand{\judgeC}[4][\Theta]{{#1};\;{#2} \vdash {#3} \div {#4}}
\newcommand{\judgeEq}[5][\Theta]{{#1};\;{#2} \vdash {#3} \equiv {#4} : {#5}}
\newcommand{\judgeEqC}[5][\Theta]{{#1};{#2} \vdash {#3} \equiv {#4} \div {#5}}



%% semantic operations

\newcommand{\semfun}[2]{\lambda #1.\;#2}
\newcommand{\sempair}[2]{\left({#1}, {#2}\right)}
\newcommand{\powerset}[1]{\mathcal{P}(#1)}

\newcommand{\paircat}[2]{\left<{#1};{#2}\right>}
\newcommand{\sumcat}[2]{\left[{#1};{#2}\right]}
\newcommand{\abscat}[1]{\lambda({#1})}


\newcommand{\judgeP}[3]{{#1} \vdash {#2} : {#3}}
\newcommand{\judgeS}[2]{{#1} \vdash {#2} : \spectype}

\newcommand{\judgeSCtx}[2]{{#1} \vdash {#2} : \ctext{context}}

\newcommand{\judgeEqP}[4]{{#1} \vdash {#2} \equiv {#3} : {#4}}


\newcommand{\judgeWK}[3][\Theta]{{#1} \vdash {#2} : {#3}}
\newcommand{\judgeKeq}[4][\Theta]{{#1} \vdash {#2} \equiv {#3} : {#4}}

\newcommand{\entailsP}[3]{{#1} \rhd {#2} \vdash {#3}}
\newcommand{\entailsS}[3]{{#1}; {#2} \vdash {#3} \ok}

\newcommand{\chartp}{\ctext{char}}
\newcommand{\fonttp}{\ctext{font}}
%%

\newcommand{\runcmd}{\ctext{run}\;}
\newcommand{\return}{\ctext{return}\;}
\newcommand{\bind}{\ctext{bind}\;}
\newcommand{\readcell}{\ctext{read}\;}


% \newcommand{\ready}[3]{\ctext{ready}({#1}, {#2}, {#3})}
\newcommand{\unready}[2]{\ctext{unready}({#1}, {#2})}

\newcommand{\cellprop}[5]{\ctext{cell}^{#1}({#2}, {#3}, {#4}, {#5})}
\newcommand{\cellneg}[2]{\cellprop{-}{#1}{#2}{\_}{\_}}
\newcommand{\cellpos}[4]{\cellprop{+}{#1}{#2}{#3}{#4}}
\newcommand{\celleither}[2]{\cellprop{\pm}{#1}{#2}{-}{-}}
\newcommand{\localref}[2]{\ctext{local}({#1}, {#2})}
\newcommand{\localstate}{\mathit{localstate}}
\newcommand{\getref}{\ctext{getref}\;}
\newcommand{\setref}{\ctext{setref}\;}

\newcommand{\aconfig}[2]{\left<{#1};{#2}\right>}
\newcommand{\aeffect}[2]{\left[{#1}|{#2}\right]}
\newcommand{\astep}[6]{\aconfig{#1}{#2}\Downarrow\aconfig{#3}{#4}\aeffect{#5}{#6}}

\newcommand{\powersetfin}[1]{\mathcal{P}^{\mathrm{fin}}({#1})}
\newcommand{\celltype}[1]{\ctext{cell}\;{#1}}
\newcommand{\bool}{\ctext{bool}}
\newcommand{\settype}[1]{\ctext{Set}({#1})}

\newcommand{\codetype}[1]{\ctext{code}\;{#1}}

\newcommand{\newcell}{\ctext{newcell}\;}
\newcommand{\cellset}{\ctext{cellset}}
\newcommand{\addset}{\ctext{addset}}

\newcommand{\comprehend}[2]{\setof{{#1}\;|\;{#2}}}
\newcommand{\ST}[2]{\ctext{ST}({#1}, {#2})}
\newcommand{\stream}[1]{\ctext{stream}({#1})}
\newcommand{\formula}{\ctext{formula}}


\newcommand{\updatecell}{\ctext{update}}


\newcommand{\liftop}{\ctext{lift}\;}
\newcommand{\seqop}{\ctext{seq}\;}
\newcommand{\parop}{\ctext{par}\;}
\newcommand{\loopop}{\ctext{loop}\;}
\newcommand{\switchop}{\ctext{switch}\;}

\newcommand{\satisfies}[2]{\mathit{satisfies}({#1}, {#2})}
\newcommand{\sat}{\mathit{sat}}

\newcommand{\closed}[2]{\ctext{closed}({#1}, {#2})}

\newcommand{\evalAbs}[6]{\left<{#1};\;{#2}\right>\Downarrow
                         \left<{#3};\;{#4}\right>[{#5}\;|\;{#6}]}

\newcommand{\cellbad}[2]{cell^{-}({#1}, {#2}, -, -)}
\newcommand{\cellok}[4]{cell^{+}({#1}, {#2}, {#3}, {#4})}
\newcommand{\ready}[3]{ready({#1}, {#2}, {#3})}

\newcommand{\propagate}[2]{\textsf{R}({#1}, {#2})}




\newcounter{spec:linenum}

\newenvironment{specification}
               {\setcounter{spec:linenum}{0}
                \begin{tabbing}
                  \qquad \= \\[-2em]}
               {\end{tabbing}}

\newcommand{\nextline}[1][0em]{\refstepcounter{spec:linenum}\\[#1]\arabic{spec:linenum} \>}
\newcommand{\nextlinelabel}[2][0em]{\refstepcounter{spec:linenum}\label{#2}\\[#1]\arabic{spec:linenum} \>}

\newcommand{\lifttrans}{\mathit{lift}}
\newcommand{\seqtrans}{\mathit{seq}}
\newcommand{\partrans}{\mathit{par}}
\newcommand{\switchtrans}{\mathit{switch}}
\newcommand{\looptrans}{\mathit{loop}}
\newcommand{\cycletrans}{\mathit{cycle}}
\newcommand{\gentrans}{\mathit{gen}}
\newcommand{\maptrans}{\mathit{map}}



\setbeamertemplate{navigation symbols}{}
\setbeamertemplate{footline}{\strut\insertsection\hfill\insertframenumber/\inserttotalframenumber\strut\quad}

\title{Ultrametric Semantics of Higher-Order Reactive Programming}
\author{Neelakantan R. Krishnaswami \texttt{<neelk@microsoft.com>} \and
        Nick Benton \texttt{<nick@microsoft.com}}

\newcommand{\floor}[1]{\left\lfloor{#1}\right\rfloor}
\newcommand{\shrink}{\rightsquigarrow}
\newcommand{\streamtype}[1]{\celltype{\opttype{#1}}}

\begin{document}
\begin{frame}
\maketitle
\end{frame}

\begin{frame}
\frametitle{Functional Reactive Programming in a Nutshell}
\begin{itemize}
\item<+-> Goal: Write interactive programs
\item<+-> Idea: Mutable state of type $X$ become \emph{streams} of values $X^\omega$ [Eliot and Hudak 97]
\item<+-> Interactive program has type $\mathsf{Input}^\omega \to \mathsf{Output}^\omega$
\end{itemize}
\end{frame}

\begin{frame}[fragile]
\frametitle{Trouble in Paradise}
\begin{tabbing}
\texttt{profit ::} $\mathtt{stockprice}^\omega \to \mathtt{order}^\omega$ \\
\texttt{profit prices = } \\
\;\;\= \texttt{  if today $<$ tomorrow} \\
    \> \texttt{  then Buy  : (profit (tail prices))} \\
    \> \texttt{  else Sell : \!\!\!(profit (tail prices))} \\
\texttt{where} \\
    \> \texttt{  today = head prices} \\
    \> \texttt{  tomorrow = \alert<2>{head (tail prices)}} \\
\end{tabbing}
\end{frame}

\begin{frame}
  \frametitle{Causal Stream Functions}

  A function $f : A^\omega \to B^\omega$ is \emph{causal}, when for all $n, as, as'$:

  \begin{center}
    $\floor{as}_n = \floor{as'}_n$ implies $\floor{f\;as}_n = \floor{f\;as'}_n$
  \end{center}

\ \\
\begin{itemize}
\pause \item First $n$ outputs of $f$ only depend on first $n$ inputs
\pause \item \texttt{tail} not causal: element $n$ of $\mathtt{tail}\;xs$
=  element $n+1$ of $xs$
\end{itemize}
\end{frame}

\begin{frame}
  \frametitle{Arrowized FRP}

  \begin{itemize}
    \item Causal functions basis of \emph{arrowized FRP} [Nilsson \emph{et al}]\pause  
    \item Type $\mathtt{ST}\;A\;B$ are causal functions from $A^\omega \to B^\omega$ \pause
    \item Combinators for parallel/sequential composition, feedback \pause
    \item However, arrows are ``first-ordery''\ldots \pause
  \end{itemize}

  \ldots can we do even better?
\end{frame}

\begin{frame}
  \frametitle{The Category of Ultrametric Spaces}

  A pair $(X, d : X \to X \to \mathbb{R})$ is a 1-bounded ultrametric space when:

  \begin{itemize}
    \item $d(x, x') = 0$ if and only if $x = x'$
    \item $d(x, x') \in [0,1]$ 
    \item $d(x, x') = d(x', x)$
    \item $d(x, z) \leq \max(d(x,y), d(y, z))$
  \end{itemize}
  \pause
  \ \\
  \ \\
  A function $f : A \to B$ is \emph{nonexpansive}, when for all $a$ and $a'$:
    \begin{displaymath}
      d_B(f\;a, f\;a') \leq d_A(a, a')
    \end{displaymath}
  That is, $f$ is \emph{non-distance-increasing}. 
\end{frame}

\begin{frame}
  \frametitle{Streams as Ultrametric Spaces}

  Streams $X^\omega$ form an ultrametric space with the following metric

  \begin{displaymath}
    d(xs, xs') = 2^{-\min \{n \in \mathbb{N} \;|\; xs_n \;\not=\; xs'_n \}}
  \end{displaymath}

  \ \\
  Distance increases, sooner $xs$ and $xs'$ differ
  \begin{itemize}
    \item Differ at time 0 --- distance 1
    \item Differ at time 1 --- distance $\frac{1}{2}$
    \item Differ at time 2 --- distance $\frac{1}{4}$
    \item Never differ --- distance 0
 \end{itemize}
\end{frame}

\begin{frame}
  \frametitle{Nonexpansive Functions and Causality}

  \begin{block}{Theorem}
    The nonexpansive functions between streams (viewed as metric spaces) 
    are exactly the causal functions.
  \end{block}

  \pause 
  \begin{itemize}
    \item The category of (complete) 1-bounded ultrametric spaces is \emph{Cartesian closed!}
    \item The lambda calculus can be interpreted in any CCC\ldots
    \item \ldots DSL for reactive programming: functional programming!
  \end{itemize}
\end{frame}

\begin{frame}
  \frametitle{But wait, there's more\ldots}

  \begin{block}{Banach's Contraction Map Theorem}
    Every \emph{strictly} contractive function $f : A \to A$ on a metric space $A$ has a unique
    fixed point. 
  \end{block}
  \pause
  \ \\
  \begin{itemize}
    \item ``Strictly contractive'' == ``well-founded feedback''
    \item So $\mu(\lambda xs.\; 0 :: \mathit{map}\;\mathit{succ}\;xs) = 0, 1, 2, 3, \ldots$ \pause
    \item This lets us interpret feedback equations, too\ldots
    \item \ldots while ensuring they are well-defined and deterministic
  \end{itemize}
\end{frame}

\begin{frame}
  \frametitle{How Can We Implement This?}

  \begin{itemize}
    \item Imperative implementation based on dataflow propagation (TLDI 2010)
    \item Underlying library similar to self-adjusting computation [Acar \emph{et al}]
    \item Correctness proof for library uses logical relation between specs of imperative
      code and ultrametric semantics \pause
    \item Give the demo\ldots
  \end{itemize}
\end{frame}


%%
\begin{frame}[fragile]
  \frametitle{Imperative Notification Interface}
  $\mathtt{code}\;A$ monadic type built from language's pre-existing state monad:

  \begin{semiverbatim}
    code A = \(\bigcirc\)(A \(\times\) Set(\(\exists\beta.\) cell \(\beta\)))
 
    return : \(\alpha\) \(\to\) code \(\alpha\)
    bind   : code \(\alpha\) \(\to\) (\(\alpha\) \(\to\) code \(\beta\)) \(\to\) code \(\beta\)
    read   : cell \(\alpha\)  \(\to\) code \(\alpha\) 

    update : cell \(\alpha\) \(\to\) code \(\alpha\) \(\to\) \(\bigcirc(1)\)
  \end{semiverbatim}

  \begin{itemize}
    \item Computations of type $\mathtt{code}\;A$:
      \indent\begin{itemize} 
        \item compute a value of type $A$, and
        \item returns all of the cells that it directly read
      \end{itemize}
    \item Implementation manages dependencies
  \end{itemize}
\end{frame}


%%
\begin{frame}
  \frametitle{Imperative Notification Networks}

  \includegraphics<1>[height=3in]{dependency-graph.pdf}
  \includegraphics<2>[height=3in]{dependency-graph-1.pdf}
  \includegraphics<3>[height=3in]{dependency-graph-2.pdf}
  \includegraphics<4>[height=3in]{dependency-graph-3.pdf}
  \includegraphics<5>[height=3in]{dependency-graph-4.pdf}
  \includegraphics<6>[height=3in]{dependency-graph-5.pdf}
  \includegraphics<7>[height=3in]{dependency-graph-6.pdf}
\end{frame}

%%
\begin{frame}[fragile]
  \frametitle{Implementing Cells}
  \begin{semiverbatim}
    cell(\(\alpha\)) = \{
      value:     ref option(\(\alpha\));     
      expr:      ref T(\(\alpha\));          
      sources:   ref Set(\(\exists\beta.\) cell \(\beta\));  
      observers: ref Set(\(\exists\beta.\) cell \(\beta\))   
    \}
  \end{semiverbatim}
  \begin{itemize}
  \item \texttt{read} checks \texttt{value} field, may reevaluate \texttt{expr} and set \texttt{sources}
  \item \texttt{update} modifes \texttt{expr}, \texttt{value}, and transitively invalidates \texttt{observers}
  \end{itemize}
\end{frame}

%%
\begin{frame}
  \frametitle{A Custom Separation Logic for Dataflow Cells}
  \begin{displaymath}
    \begin{array}{lcll}
      \phi & ::= & I                     & \mbox{Empty Network} \\
           &  |  & \phi \otimes \psi     & \mbox{Disjoint Combination} \\ 
           &  |  & \cellbad{a}{e}        & \mbox{Un-ready Cell} \\
           &  |  & \cellok{a}{e}{v}{src} & \mbox{Possibly-Ready Cell} \\
    \end{array}
  \end{displaymath}

  \begin{itemize}
    \item \pause Goal: hide implementation complexity behind ADT
    \item \pause Abstract heap formulas describe dataflow network
    \item \pause Hoare triples for \texttt{read}, \texttt{update}, \texttt{return}, \texttt{bind} in 
      terms of these formulas
    \item (Explain on whiteboard)
  \end{itemize}
\end{frame}

%%
\begin{frame}
  \frametitle{Implementation}

  \begin{itemize}
  \item For each ultrametric type, ML implementation type:
  \begin{mathpar}
   \begin{array}{lcl}
    \interp{A^\omega}     & = & \streamtype{\interp{A}} \\
    \interp{1}           & = & \unittype \\
    \interp{A \times B}  & = & \interp{A} \star \interp{B} \\
    \interp{A \To B}     & = & \interp{A} \to \codetype{\opttype{\interp{B}}} \\
    \interp{A \shrink B} & = & \codetype{\opttype{\interp{A}}} \to \codetype{\interp{B}} 
   \end{array}
  \end{mathpar}
  \item Options come from need to implement feedback
  \item Ordinary and contractive functions represented differently 
  \item One combinator in API for each operation in the interface
  \end{itemize}
\end{frame}

%%
\begin{frame}[fragile]
\frametitle{API}
{\small
\begin{semiverbatim}
module type CCC = sig
  type (\(\alpha\),\(\beta\)) hom
  type one
  type (\(\alpha\),\(\beta\)) prod
  type (\(\alpha\),\(\beta\)) exp

  val id : (\(\alpha\),\(\alpha\)) hom
  val compose : (\(\alpha\),\(\beta\)) hom \(\to\) (\(\beta\),\(\gamma\)) hom \(\to\) (\(\alpha\),\(\gamma\)) hom

  val one : (\(\alpha\),one) hom

  val fst : ((\(\alpha\),\(\beta\)) prod, \(\alpha\)) hom 
  val snd : ((\(\alpha\),\(\beta\)) prod, \(\beta\)) hom 
  val pair : (\(\alpha\),\(\beta\)) hom \(\to\) (\(\alpha\),\(\gamma\)) hom \(\to\) (\(\alpha\), (\(\beta\),\(\gamma\)) prod) hom

  val curry : ((\(\alpha\),\(\beta\)) prod, \(\gamma\)) hom \(\to\) (\(\alpha\), (\(\beta\),\(\gamma\)) exp) hom
  val eval : (((\(\alpha\),\(\beta\)) exp, \(\alpha\)) prod, \(\beta\)) hom
end
\end{semiverbatim}
}
\end{frame}

%%
\begin{frame}
\frametitle{Implementation Details}

\begin{itemize}
\item Combinator library completely pure 
\item Intended API has strong equational principles 

(eg \texttt{pair fst snd = id})
\item \pause But they have different imperative effects..!

 \textbf{How do we know this sound?}
\item \pause Furthermore, API is higher-order

 \textbf{How can we even specify these programs?}
\end{itemize}
\end{frame}


%%
\begin{frame}
\frametitle{Logical Relations to the Rescue!}

\begin{itemize}
\item We specify the API using the technique of \emph{logical relations}
\item Define a relation $V^d_A(v, \mathtt{v}, \mu, \sigma)$
\item meaning ``$\mathtt{v}$ implements value $v$ at type $A$''. 
\item {\scriptsize(to distance $d$, in heap $\mu$, with dependencies $\sigma$)}
\end{itemize}

\end{frame}
\end{document}
