\documentclass{article}
\usepackage{mathpartir}
\usepackage{amsmath}
\usepackage{amssymb}

\title{The Linear Logic of GUIs}
\author{Neelakantan R. Krishnaswami \\ Nick Benton}

\newcommand{\ctext}[1]{\mathsf{#1}}

\newcommand{\normalize}[1]{\ctext{nf}({#1})}

\newcommand{\unittype}{\mathbf{1}}
\newcommand{\reftype}[1]{\ctext{ref}\;#1}
\newcommand{\monad}[1]{\bigcirc{#1}}
\newcommand{\cont}[1]{\ctext{cont}\;{#1}}
\newcommand{\opttype}[1]{\ctext{option }{#1}}
\newcommand{\seqsort}[1]{\ctext{seq}\;{#1}}
\newcommand{\listtype}[1]{\ctext{list}\;{#1}}

\newcommand{\pair}[2]{\left<{#1},{#2}\right>}
\newcommand{\fst}[1]{\ctext{fst}\; #1}
\newcommand{\snd}[1]{\ctext{snd}\; #1}

\newcommand{\inj}[1]{\iota_{#1}}
\newcommand{\inl}[1]{\ctext{inl }#1}
\newcommand{\inr}[1]{\ctext{inr }#1}
\newcommand{\Case}[5]{\ctext{case}(#1,\; {#2}.\; #3,\; {#4}.\; #5)}
\newcommand{\listcase}[5]{\ctext{case}(#1,\;\ctext{Nil} \to {#2},\;
                                       \ctext{Cons}({#3},{#4}) \to {#5})}
\newcommand{\optcase}[4]{\ctext{case}(#1,\;
                                      \ctext{None} \to {#2},\;
                                      \ctext{Some}\;{#3} \to {#4})}
\newcommand{\z}{\ctext{z}}
\newcommand{\s}[1]{\ctext{s}(#1)}
\newcommand{\iter}[4]{\ctext{iter}(#1, {#2}, {#3}.\; {#4})}
\newcommand{\iterseq}[4]{\ctext{iter}_{\mathsf{seq}}(#1, {#2}, #3.\; {#4})}
\newcommand{\comp}[1]{[#1]}
\newcommand{\fun}[3]{\lambda #1:#2.\;#3}
\newcommand{\Fun}[3]{\Lambda #1:#2.\;#3}
\newcommand{\unit}{\left<\right>}
\newcommand{\pack}[2]{\ctext{pack}({#1}, {#2})}
\newcommand{\unpack}[4]{\ctext{unpack}({#1}, {#2}) = {#3} \;\ctext{in}\;{#4}}
\newcommand{\alt}{\;|\;}
\newcommand{\letv}[3]{\ctext{letv}\;#1 = #2\;\ctext{in}\;#3}
\newcommand{\newref}[2]{\ctext{new}_{#1}(#2)}
\newcommand{\run}[1]{\ctext{run}\;{#1}}
\newcommand{\ok}{\ctext{ ok}}
\newcommand{\FV}[1]{\mathrm{FV}({#1})}

\newcommand{\fix}[2]{\ctext{fix}\;{#1}.\;{#2}}

\newcommand{\statecfg}[2]{\left<{#1};\;{#2}\right>}
\newcommand{\eval}[4]{\left<{#1};\;{#2}\right> \leadsto \left<{#3};\;{#4}\right>}
\newcommand{\evalabort}[2]{\left<{#1};\;{#2}\right> \leadsto \mathbf{abort}}

\newcommand{\domain}[1]{\mbox{dom}({#1})}
\newcommand{\upset}[1]{\mathcal{P}^{\uparrow}({#1})}

\newcommand{\pointsto}{\mapsto}
\newcommand{\disj}{\vee}
\renewcommand{\implies}{\supset}
\newcommand{\wand}{-\!\!*\,}
\newcommand{\emp}{\mathsf{emp}}
\newcommand{\validprop}[1]{{#1}\;\ctext{valid}}

\newcommand{\todo}[1]{\texttt{[TODO: {#1}]}}

\newcommand{\setof}[1]{\{{#1}\}}

\newcommand{\To}{\Rightarrow}
\newcommand{\From}{\Leftarrow}

\newcommand{\N}{\mathbb{N}}

\newcommand{\assert}{\ctext{prop}}

\newcommand{\bigstep}{\Downarrow}

\newcommand{\bigeval}[4]{\left<{#1};\;{#2}\right> \bigstep \left<{#3};\;{#4}\right>}
\newcommand{\bigevalabort}[2]{\left<{#1};\;{#2}\right> \bigstep \mathbf{abort}}

\newcommand{\spec}[4]{\{{#1}\}{#2}\{{#3}.\;{#4}\}}
\newcommand{\specX}[3]{\{{#1}\}{#2}\{{#3}\}}
\newcommand{\mspec}[4]{\langle{#1}\rangle{#2}\langle{#3}.\;{#4}\rangle}
\newcommand{\bnfalt}{\;\;|\;\;}

\newcommand{\specor}{\;||\;}
\newcommand{\specand}{\;\&\;}
\newcommand{\specimp}{\Rightarrow\!\!\!>}
% \newcommand{\specimp}{\Longrightarrow}
% \newcommand{\specor}{\ctext{ or }}
% \newcommand{\specand}{\ctext{ and }}
% \newcommand{\specimp}{\ctext{ implies }}
\newcommand{\spectype}{\ctext{spec}}
\newcommand{\valid}{\ctext{ valid}}

\newcommand{\interp}[1]{[\![{#1}]\!]}
\newcommand{\interpE}[1]{\interp{#1}^e}
\newcommand{\interpC}[1]{\interp{#1}^c}

\newcommand{\interpF}[1]{[\![{#1}]\!]_f}
\newcommand{\interpmono}[1]{\interp{#1}^{\mathrm{m}}}

\newcommand{\entails}{\models}

\newcommand{\judgeE}[4][\Theta]{{#1};\;{#2} \vdash {#3} : {#4}}
\newcommand{\judgeC}[4][\Theta]{{#1};\;{#2} \vdash {#3} \div {#4}}
\newcommand{\judgeEq}[5][\Theta]{{#1};\;{#2} \vdash {#3} \equiv {#4} : {#5}}
\newcommand{\judgeEqC}[5][\Theta]{{#1};{#2} \vdash {#3} \equiv {#4} \div {#5}}




%% semantic operations

\newcommand{\worldleq}{\preceq}
\newcommand{\worldgeq}{\succeq}

\newcommand{\semfun}[2]{\lambda #1.\;#2}
\newcommand{\sempair}[2]{\left({#1}, {#2}\right)}
\newcommand{\powerset}[1]{\mathcal{P}(#1)}
\newcommand{\powersetfin}[1]{\mathcal{P}^{\mathrm{fin}}(#1)}

\newcommand{\paircat}[2]{\left<{#1};{#2}\right>}
\newcommand{\sumcat}[2]{\left[{#1};{#2}\right]}
\newcommand{\abscat}[1]{\lambda({#1})}


\newcommand{\judgeP}[3]{{#1} \vdash {#2} : {#3}}
\newcommand{\judgeS}[2][\Delta]{{#1} \vartriangleright {#2} : \spectype}

\newcommand{\judgeSCtx}[2]{{#1} \vartriangleright {#2} : \ctext{context}}

\newcommand{\judgeEqP}[4]{{#1} \vdash {#2} \equiv {#3} : {#4}}


\newcommand{\judgeWK}[3][\Theta]{{#1} \vdash {#2} : {#3}}
\newcommand{\judgeKeq}[4][\Theta]{{#1} \vdash {#2} \equiv {#3} : {#4}}

\newcommand{\entailsP}[3]{{#1} \vartriangleright {#2} \vdash {#3}}
\newcommand{\entailsS}[3]{{#1}; {#2} \vdash {#3} \ok}

\newcommand{\chartp}{\ctext{char}}
\newcommand{\fonttp}{\ctext{font}}

\newcommand{\LOC}{loc}
\newcommand{\MONO}{\mathbf{mono}}
\newcommand{\HEAP}{heap}
\newcommand{\PROP}{Prop}
\newcommand{\TRUE}{True}
\newcommand{\HPROP}{Prop}
\newcommand{\TYPE}{Type}

\newtheorem{prop}{Proposition}
\newtheorem{lemma}{Lemma}

\newcommand{\comprehend}[2]{\setof{{#1}\;|\;{#2}}}


% \newenvironment{proof}{\begin{comment}}{\end{comment}}

\newenvironment{proof}{\noindent\textbf{Proof.}}{\noindent\ensuremath{\Box}}

\newcounter{prooflinenum}
\newenvironment{tabbedproof}
   {\setcounter{prooflinenum}{0}
    \begin{tabbing}\;\;\=\;\;\;\;\;\=\;\;\;\;\=\;\;\;\;\=\;\;\;\;\=\;\;\;\;\=\;\;\;\;\=\;\;\;\;\=\;\;\;\;\=\\[-2em]}
   {\end{tabbing}}

\newcommand{\oo}{\addtocounter{prooflinenum}{1}\arabic{prooflinenum}\>\>}
\newcommand{\ooo}{\addtocounter{prooflinenum}{1}\arabic{prooflinenum}\>\>\>}
\newcommand{\oooo}{\addtocounter{prooflinenum}{1}\arabic{prooflinenum}\>\>\>\>}
\newcommand{\ooooo}{\addtocounter{prooflinenum}{1}\arabic{prooflinenum}\>\>\>\>\>}
\newcommand{\oooooo}{\addtocounter{prooflinenum}{1}\arabic{prooflinenum}\>\>\>\>\>\>}
\newcommand{\ooooooo}{\addtocounter{prooflinenum}{1}\arabic{prooflinenum}\>\>\>\>\>\>\>}
\newcommand{\oooooooo}{\addtocounter{prooflinenum}{1}\arabic{prooflinenum}\>\>\>\>\>\>\>\>}
\newcommand{\ooooooooo}{\addtocounter{prooflinenum}{1}\arabic{prooflinenum}\>\>\>\>\>\>\>\>\>}

\newcommand{\ox}{\>\>}
\newcommand{\oox}{\>\>\>}
\newcommand{\ooox}{\>\>\>\>}
\newcommand{\oooox}{\>\>\>\>\>}
\newcommand{\ooooox}{\>\>\>\>\>\>}
\newcommand{\oooooox}{\>\>\>\>\>\>\>}
\newcommand{\ooooooox}{\>\>\>\>\>\>\>\>}
\newcommand{\oooooooox}{\>\>\>\>\>\>\>\>\>}


\newenvironment{eqnproof}[1][]{${#1}$\begin{displaymath}\begin{array}{lcll}}
                         {\end{array}\end{displaymath}}

\newcommand{\eline}[3][]{{#1} & = & {#2} & \mbox{{#3}} \\}

\newcommand{\elines}[3][]{{#1} & = & \begin{array}{l} #2 \end{array} & \mbox{{#3}} \\}

\newcommand{\eclaim}[3][]{{#1} &  & {#2} & \mbox{{#3}} \\}
\newcommand{\efact}[2]{{#1} & & & \mbox{#2} \\}

\newcommand{\basicspec}[4]{[{#1}]\;{#2}\;[{#3}.\;{#4}]}

% Macros for type checking assertions

\newcommand{\pfun}[3]{\hat{\lambda} #1:#2.\;#3}
\newcommand{\restrictkind}[1]{({#1})\Downarrow_K}
\newcommand{\restricttype}[1]{({#1})\Downarrow_T}

\newcommand{\restricttyenv}[2]{{#2}\Downarrow^{#1}_K}
\newcommand{\restrictvals}[2]{{#2}\Downarrow^{#1}_T}
\newcommand{\judgeACtx}[1]{\vartriangleright {#1} \;\ctext{ok}}
\newcommand{\judgeA}[3][\Delta]{{#1} \vartriangleright {#2} : {#3}}
\newcommand{\judgeSort}[2][\Delta]{\judgeA[{#1}]{{#2}}{\ctext{sort}}}
\newcommand{\judgeSortEq}[3][\Delta]{{#1} \vartriangleright {#2} \equiv {#3} : \ctext{sort}}

\newcommand{\ms}[1]{\mathcal{#1}}
\newcommand{\Frame}[2]{{#1} \otimes {#2}}

\newcommand{\Approx}[2][n]{\left\lfloor{#2}\right\rfloor^{#1}}
\newcommand{\lolli}{\multimap}
\begin{document}

\section{Introduction}

The thesis of this paper is that functional reactive programming has not taken the 
categorical structure of causal functions seriously enough, and by doing so we win 
along many axes. 

\begin{enumerate}
\item We get a much more flexible API than arrow-style FRP interfaces
  permit, while still ruling out the possibility of space/time leaks.

\item This interface still has a very rich equational theory, and a
  semantics in terms of causal functions (generalized to higher-order). 

\item Nevertheless, we can give an efficient imperative implementation of the 
  interface, which allows using well-known techniques for ``retained-mode''
  style interfaces (i.e., DOM/scenegraph) to graphics APIs.  

\item Furthermore, we can prove the correctness of this implementation, using
  Hoare logic and a logical relation between the implementation and the semantics. 
\end{enumerate}

\section{Semantics of Causal Stream Functions}

Given a function on streams $f : \stream{A} \to \stream{B}$, we say
that $f$ is causal when it is the case that the first $n$ outputs of
$f$ are determined by the first $n$ inputs. To lift this definition to
higher type, we use a simple Kripke logical relation:

\begin{mathpar}
  \begin{array}{lclcl}
    \unit & \Approx{\unittype} & \unit & = & \top \\
    \pair{a}{b} &  \Approx{A \otimes B} & \pair{a'}{b'} & = & a \Approx{A} a' \land b \Approx{B} b' \\
    v & \Approx{\stream{X}} & v' & = & \forall i \leq n.\; v_i = v'_i \\
    f & \Approx{A \lolli B} & f' & = & \forall m \leq n, v, v'.\; (v \Approx[m]{A} v') \implies (f\;v) \Approx[m]{B} (f'\;v') \\
  \end{array}
\end{mathpar}

The Kripke structure comes from the fact that this logical relation is
indexed by a time $n$, which is ordered with the usual ordering on
natural numbers. Intuitively, we can think of this as giving a family
of relations, one for each $n$, which say whether two values
approximate one another for $n$ time steps. (This is similar to the use of 
time in step-indexed models --- although here time is literally present, rather 
than just a convenient technical device!)

The causal values of a type $A$ are those values $v$ such that for all
$n$, $v \Approx{A} v$.

\begin{theorem}{(Monoidal Closure of the Category of Causal Operations)}
The category of sets of causal values of type $A$ and causal maps
between them is a symmetric monoidal closed category.
\end{theorem}

\begin{proof}
  First, we need to check that this is actually a category. To do
  this, we need to check that the identity function is causal, and
  composition of causal maps is also causal. 

  For the identity function, this is immediate. 

  To show this for composition, suppose that we have $f : A \to B$ and
  $g : B \to C$. We want to show that $g \circ f$ is causal, which
  means that we want to show that for arbitrary $m \leq n$, and for all
  $v,v'$ such that $v \Approx[m]{A} v'$, we have that $g(f(v) \Approx[m]{C} g(f(v'))$. 
  Since we know that $\forall n, m \leq n, v,v', v\Approx[m]{A}v' \implies f(v) \Approx[m]{B} f(v')$
  and that $\forall n, m \leq n, v,v', v\Approx[m]{B}v' \implies g(v) \Approx[m]{C} g(v')$,
  this follows from instantiating quantifiers appropriately and composing implications. 

  Next, we check that pairing is a bifunctor, with an action sending
  pairs of objects $(A, B)$ to the product $A \otimes B$, and sending
  pairs of functions $(f,g)$ to $\semfun{\pair{a}{b}}{\pair{f\;a}{g\;b}}$.
  
  Since pairs and functions give monoidal structure in Sets, we know
  that the functorial equations are satisfied, assuming that that the 
  action of the bifunctor also gives us a causal map. So, suppose that 
  $f : A \to X$ and $g : B \to Y$ are causal. Now we want to show that 
  $\semfun{\pair{a}{b}}{\pair{f\;a}{g\;b}}$ is a causal map between 
  $A \otimes B \to X \otimes Y$. 

  So, suppose that for all $n,m, n \leq m$, we have that $\pair{a}{b} \Approx[m]{A \otimes B}
  \pair{a'}{b'}$. By the definition of $\Approx[m]{A \otimes B}$, we know that
  $a \Approx[m]{A} a'$ and $b \Approx[m]{B} b'$ hold. From the causality of $f$ and 
  $g$, we know that $f\;a \Approx[m]{X} f\;a'$ and $g\;b \Approx[m]{Y} g\;b'$ hold, and 
  so we can conclude that $\pair{f\;a}{g\;b} \Approx[m]{X \otimes Y}
  \pair{f\;a'}{g\;b'}$ holds. 
  
  
  Next, we'll show this is a monoidal category, with associator and
  left and right unitors (what a great word!) defined by:

  \begin{mathpar}
    \begin{array}{lcl}
      \alpha_{A,B,C} & = &  \semfun{\pair{\pair{a}{b}}{c}}{\pair{a}{\pair{b}{c}}} \\
      \alpha^{-1}_{A,B,C} & = & \semfun{\pair{a}{\pair{b}{c}}}{\pair{\pair{a}{b}}{c}}  \\[1em]

      \lambda_A & = & \semfun{\pair{\unit}{a}}{a} \\
      \rho_A    & = & \semfun{\pair{a}{\unit}}{a} \\
    \end{array}
  \end{mathpar}

  Again, the equations we want are inherited from the monoidal structure
  on Set. So we just need to check that these are causal functions, which
  is pretty easy. 

  For $\alpha$, assume that $\pair{\pair{a}{b}}{c} \Approx{(A \otimes B) \otimes C} \pair{\pair{a'}{b'}}{c'}$.
  Therefore, we know that $a \Approx[m]{A} a'$, $b \Approx[m]{B} b'$, and $c \Approx[m]{C} c'$. 
  We want to show that for all $m \leq n$, $\alpha(\pair{\pair{a}{b}}{c}) \Approx[m]{A \otimes (B \otimes C)}
  \alpha(\pair{\pair{a'}{b'}}{c'})$. So this means that we want to show that 
  $\pair{a}{\pair{b}{c}} \Approx[m]{A \otimes (B \otimes C)} \pair{a'}{\pair{b'}{c'}}$. So it suffices 
  to show that $a \Approx[m]{A} a'$, $b \Approx[m]{B} b'$, and $c \Approx[m]{C} c'$, which are known. 

  The causality of $\alpha^{-1}$ is similar. 

  The causality of $\lambda_A$ can be established as follows. Suppose
  $\pair{\unit}{a} \Approx{1 \otimes A} \pair{\unit}{a'}$. Therefore
  we know that $a \Approx{A} a'$. Now we want to show that
  $\lambda_A(\pair{\unit}{a}) \Approx{A} \lambda_A(\pair{\unit}{a'})$. 
  So we need to show that $a \Approx{A} a'$, which we know. 

  The causality of $\rho_A$ is similar. 

  Finally, we need to show that this is monoidal closed. To do this, we will 
  define the evaluation map: 
  \begin{mathpar}
    \mathit{eval}_{A,B} : A \otimes (A \lolli B) \to B \triangleq (a, f) \mapsto f(a) 
  \end{mathpar}

  and define the currying operator, which takes $f : X \otimes A \to B$, and gives us a unique 
  morphism 
  
  \begin{mathpar}
    \mathit{curry}(f) : X \to (A \lolli B) \triangleq \semfun{x}{\semfun{a}{f(x,a)}}
  \end{mathpar}

  such that $f = \mathit{eval} \circ (\mathit{curry}(f) \otimes
  id)$. This equation is of course again inherited from Set, and so we just need to 
  check that these operations are causal. 

  So we want to show that $\mathit{eval}$ is causal. Suppose that we have for $m \leq n$, 
  that $(a, f) \Approx[m]{A \otimes (A \lolli B)} (a', f')$. So therefore we know that 
  $a \Approx[m]{A} a'$ and $f \Approx[m]{A \lolli B} f'$. 

  Now, we want to show that $\mathit{eval}(a, f) \Approx[m]{B}
  \mathit{eval}(a', f')$. So this means we want to show that $f(a) \Approx[m]{B} f'(a')$. 
  Now, since $f \Approx[m]{A \lolli B} f'$, we know that for all $k \leq m$ and all 
  $a$ and $a'$, that $a \Approx[k]{A} a'$ implies $f(a) \Approx[k]{B} f'(a')$. Instantiating
  $k$ with $m$, we conclude that $f(a) \Approx[m]{B} f'(a')$. 

  Finally, we want to show that if $f$ is causal then
  $\mathit{curry}(f)$ is causal. So, suppose that we have $m \leq n$ and that 
  $x \Approx[m]{X} x'$. We want to show that $\mathit{curry}{f}(x) \Approx[m]{A \lolli B} \mathit{curry}(f)(x')$. 
  So we want to show that $\semfun{a}{f(x, a)} \Approx[m]{A \lolli B} \semfun{a}{f(x',a)}$. 
  To show this, assume that we have $k \leq m$ and $a \Approx[k]{A} a'$, so that we need to 
  show $f(x, a) \Approx[k]{B} f(x',a')$. Now, since our relation is Kripke, we know that 
  $x \Approx[k]{X} x'$, which lets us conclude that $(x, a) \Approx[k]{X \otimes A} (x',a')$, 
  and since $f$ is causal, we know that $f(x, a) \Approx[k]{B} f(x',a')$. 
\end{proof}

Some notes: 
\begin{enumerate}
\item In fact, we have a cartesian closed category, but we're
  only going to use the monoidal structure in what follows, in order to
  let us add a type $\textsf{Widget}$ of linear widget streams. We'll get back
  the expressiveness by adding a contraction combinator for copyable streams. 

\item This is a rather wimpy closed structure, since we have nothing 
  corresponding to the linear exponential $!A$. It's possible that $\stream\ $
  could be pressed into this role, but the isomorphism $!!A \simeq !A$ is 
  only questionably implementable. 

\item Perhaps we should be looking at \emph{classical} linear logic, if 
  we want to model nondeterminism as well (so that we could use the category
  of sets and ``causal relations''). This might be useful for modelling the
  nondeterminism involved in clicking buttons. :)
\end{enumerate}

\section{Logical Relation on the Implementation}

All this builds on the stuff in our TLDI paper. In particular, I take
all the apparatus of abstract semantics of notifications and so on as 
given. 

\subsection{Something That Almost, But Doesn't Quite, Work}

The basic idea here is to start by saying how a stream is realized in 
terms of notifications, and likewise how stream transducers are realized. 

A stream $v : \stream{X}$ is realized by a cell $\hat{v} :
\celltype{X}$. The idea is that at each time step $i$, we read the
cell $\hat{v}$ to get the value of the stream $v_i$. Now, a cell reads
and modifies a bunch of state --- both other cells and local state --
so we also need to specify what it reads and how it changes it. To 
do this, we give two functions $\phi, \psi : \N \to \mathsf{formula}$. 
These two functions describe the pre- and the post-state at each 
time step $i$. 

(Now, you might ask why we don't simply have a \emph{single} function
$\psi$, and require that the post-state of $\psi_i$ be $\psi_{i+1}$,
and thereby capture the time-evolution of the system. The answer is
that this captures some, but not all, sorts of signal. For example, we
want to model events (such as button clicks) via the external event
loop modifying some cells. Having both $\psi$ and $\phi$ lets us
specify the case where we want to place the responsibility for change
outside the reevaluation mechanism. 

Furthermore, the changes to the state the cell reads can have an
impact on the rest of the heap, via the ramifications. So we need to
summarize this information, as well, which we do with a function $d :
\cellset$.  This set essentially describes the cells which both the
current stream and the rest of the network may depend on in common. 

So we can give the logical relation between the a mathematical stream and a
bit of code as follows: 

\begin{mathpar}
R_{\stream{X}}(\phi, \psi, u, v, \hat{v}) \triangleq 
  \forall i.\; \exists u.\; \astep{\phi_i}{\readcell{\hat{v}}}{\psi_i}{v_i}{\setof{\hat{v}}}{u}  
    \land u \subseteq v \cup \dom{\phi_i}
\end{mathpar}

Note that this means we are promising to read each cell on each time
step at least once. This is essential, since we can have local state
in our network -- for example, an accumulator needs to sum the value
of its input on \emph{every} time step, and if we fail to read the
cell we can fail to increment the sum on that step.

The way we will model functions $f : A \lolli B$ is via the use of an
imperative program which, when given an argument, builds a new piece of
dataflow graph for computing the function. So the representation type
is $\interp{A \lolli B} = \interp{A} \to \monad{\interp{B}}$.

\begin{mathpar}
R_{A \lolli B}(\phi, \phi, u, f, \hat{f}) \triangleq 
  \forall \alpha, \beta, a, \hat{a}.\; R_A(\alpha, \beta, u, a, \hat{a}) \implies 
     B_B(\alpha, \beta, u, f\;a, \hat{f}\;\hat{a}) 
\\ \\
B_A(\phi, \psi, u, v, c) \triangleq \forall \omega.\; \spec{H(\omega)}{c}{\mathit{a}}{\exists \theta.\; H(\omega \otimes \theta_0) \land R_A(\phi \otimes \theta, \psi \otimes \theta^+, u, v, \mathit{a})}
\end{mathpar}

First, note that we maintain the internal invariant that our functions
will never depend on the temporal behavior of their components. That
is, each call to such a function may build some new graph structure,
but it won't depend on the values in the cells of its arguments. This is
why the abstract state arguments are equal -- a function does not look
at the abstract state at all, and so it will work in any abstract state,
and won't change it. 

The purpose of this invariant is basically to let us call the function
whenever we want -- we don't want to have to put dataflow constraints
on calling functions. This may be worth relaxing later on, though.

That said, we then get to the specification. It says that given an 
input $\hat{a}$ that uses $\alpha$ and $\beta$ to realize a value 
$a$, then $\hat{f}\hat{a}$ will build a network using $\alpha$ and $\beta$
to realize $f\;a$. By ``build a network'', we mean the predicate $B_A$, 
defined on the next line. There, the command $c$ builds a network to 
realize $v$, when executing it (in any heap) results in some new 
graph structure $\theta_i$ which it will use together with $\alpha$
to compute values. Note that here is where we make use of the idea 
that evaluating the node takes the network from time $i$ to time
$i+1$ --- the constructed structure starts at state $\theta_0$, and
at each timestep it goes from $\theta_i$ to $\theta^+_{i} = \theta_{i+1}$. 

Finally, let's quickly look at how we handle pairs and units. 

A tensor type $A \otimes B$ is just realized by a pair $\interp{A}
\times \interp{B}$.

\begin{mathpar}
  R_{A \otimes B}(\phi, \psi, u, \pair{a}{b}, (\hat{a}, \hat{b})) \triangleq
    \exists \theta_A, \theta_B.\; 
      \begin{array}{lclc}
        R_A(\phi, \theta_A, u, a, \hat{a}) & \land & R_B(\theta_A, \psi, u, b, \hat{b}) & \land \\
        R_B(\phi, \theta_B, u, b, \hat{b}) & \land & R_A(\theta_B, \psi, u, a, \hat{a}) &  \\
      \end{array}
\end{mathpar}

This says that the pair $\pair{a}{b}$ is realized by a pair of values
$(\hat{a}, \hat{b})$.  The two lines of the definition assert that
regardless of the order in which we observe them, we ``end up in the
same place''. The unit is the unit of the tensor, so it is of the form: 

\begin{mathpar}
  R_{1}(\phi, \psi, u, \unit, ()) \triangleq (\phi = \psi)
\end{mathpar}

\subsubsection{Verification of Some Simple Programs}

Here's some examples. 

\begin{theorem}{(Mapping)}
  Define $\hat{map} f = \semfun{x}{\newcell (\bind (\readcell x)\; (\return \circ f))}$. 
Then we have that $R_{S(X) \lolli S(Y)}(\phi, \phi, \emptyset, \mathit{map}\;f, \hat{\mathit{map}}\;f)$,
for $f : X \to Y$. 
\end{theorem}

\begin{theorem}{(Eval)}
  Define $\hat{\mathit{eval}} = \semfun{\pair{f}{x}}{f(x)}$. Then we have that 
$R_{((A \lolli B) \otimes A) \lolli B}(\phi, \phi, \emptyset, \mathit{eval}, \hat{\mathit{eval}})$. 
\end{theorem}

\begin{proof}
  These are all just pretty much a matter of unwinding the definitions and seeing that 
everything is kosher. I'm too lazy to transcribe the proofs from the whiteboard to \LaTeX,
since we're going to chuck everything out soon. 
\end{proof}

\subsubsection{What Doesn't Work}

What we'd really like to do is to support a fixed point operator on this category. That
means that we need an operation with a type like 

\begin{mathpar}
  \mathsf{trace} : (A \otimes S(X) \lolli B \otimes S(X)) \lolli A \lolli B 
\end{mathpar}

The only trouble is that 

\end{document}
