\documentclass{article}

\usepackage{amsmath}
\usepackage{amssymb}
\usepackage{mathpartir}
\newcommand{\ctext}[1]{\mathsf{#1}}

\newcommand{\normalize}[1]{\ctext{nf}({#1})}

\newcommand{\unittype}{\mathbf{1}}
\newcommand{\reftype}[1]{\ctext{ref}\;#1}
\newcommand{\monad}[1]{\bigcirc{#1}}
\newcommand{\cont}[1]{\ctext{cont}\;{#1}}
\newcommand{\opttype}[1]{\ctext{option }{#1}}
\newcommand{\seqsort}[1]{\ctext{seq}\;{#1}}
\newcommand{\listtype}[1]{\ctext{list}\;{#1}}

\newcommand{\pair}[2]{\left<{#1},{#2}\right>}
\newcommand{\fst}[1]{\ctext{fst}\; #1}
\newcommand{\snd}[1]{\ctext{snd}\; #1}

\newcommand{\inj}[1]{\iota_{#1}}
\newcommand{\inl}[1]{\ctext{inl }#1}
\newcommand{\inr}[1]{\ctext{inr }#1}
\newcommand{\Case}[5]{\ctext{case}(#1,\; {#2}.\; #3,\; {#4}.\; #5)}
\newcommand{\listcase}[5]{\ctext{case}(#1,\;\ctext{Nil} \to {#2},\;
                                       \ctext{Cons}({#3},{#4}) \to {#5})}
\newcommand{\optcase}[4]{\ctext{case}(#1,\;
                                      \ctext{None} \to {#2},\;
                                      \ctext{Some}\;{#3} \to {#4})}
\newcommand{\z}{\ctext{z}}
\newcommand{\s}[1]{\ctext{s}(#1)}
\newcommand{\iter}[4]{\ctext{iter}(#1, {#2}, {#3}.\; {#4})}
\newcommand{\iterseq}[4]{\ctext{iter}_{\mathsf{seq}}(#1, {#2}, #3.\; {#4})}
\newcommand{\comp}[1]{[#1]}
\newcommand{\fun}[3]{\lambda #1:#2.\;#3}
\newcommand{\Fun}[3]{\Lambda #1:#2.\;#3}
\newcommand{\unit}{\left<\right>}
\newcommand{\pack}[2]{\ctext{pack}({#1}, {#2})}
\newcommand{\unpack}[4]{\ctext{unpack}({#1}, {#2}) = {#3} \;\ctext{in}\;{#4}}
\newcommand{\alt}{\;|\;}
\newcommand{\letv}[3]{\ctext{letv}\;#1 = #2\;\ctext{in}\;#3}
\newcommand{\newref}[2]{\ctext{new}_{#1}(#2)}
\newcommand{\run}[1]{\ctext{run}\;{#1}}
\newcommand{\ok}{\ctext{ ok}}
\newcommand{\FV}[1]{\mathrm{FV}({#1})}

\newcommand{\fix}[2]{\ctext{fix}\;{#1}.\;{#2}}

\newcommand{\statecfg}[2]{\left<{#1};\;{#2}\right>}
\newcommand{\eval}[4]{\left<{#1};\;{#2}\right> \leadsto \left<{#3};\;{#4}\right>}
\newcommand{\evalabort}[2]{\left<{#1};\;{#2}\right> \leadsto \mathbf{abort}}

\newcommand{\domain}[1]{\mbox{dom}({#1})}
\newcommand{\upset}[1]{\mathcal{P}^{\uparrow}({#1})}

\newcommand{\pointsto}{\mapsto}
\newcommand{\disj}{\vee}
\renewcommand{\implies}{\supset}
\newcommand{\wand}{-\!\!*\,}
\newcommand{\emp}{\mathsf{emp}}
\newcommand{\validprop}[1]{{#1}\;\ctext{valid}}

\newcommand{\todo}[1]{\texttt{[TODO: {#1}]}}

\newcommand{\setof}[1]{\{{#1}\}}

\newcommand{\To}{\Rightarrow}
\newcommand{\From}{\Leftarrow}

\newcommand{\N}{\mathbb{N}}

\newcommand{\assert}{\ctext{prop}}

\newcommand{\bigstep}{\Downarrow}

\newcommand{\bigeval}[4]{\left<{#1};\;{#2}\right> \bigstep \left<{#3};\;{#4}\right>}
\newcommand{\bigevalabort}[2]{\left<{#1};\;{#2}\right> \bigstep \mathbf{abort}}

\newcommand{\spec}[4]{\{{#1}\}{#2}\{{#3}.\;{#4}\}}
\newcommand{\specX}[3]{\{{#1}\}{#2}\{{#3}\}}
\newcommand{\mspec}[4]{\langle{#1}\rangle{#2}\langle{#3}.\;{#4}\rangle}
\newcommand{\bnfalt}{\;\;|\;\;}

\newcommand{\specor}{\;||\;}
\newcommand{\specand}{\;\&\;}
\newcommand{\specimp}{\Rightarrow\!\!\!>}
% \newcommand{\specimp}{\Longrightarrow}
% \newcommand{\specor}{\ctext{ or }}
% \newcommand{\specand}{\ctext{ and }}
% \newcommand{\specimp}{\ctext{ implies }}
\newcommand{\spectype}{\ctext{spec}}
\newcommand{\valid}{\ctext{ valid}}

\newcommand{\interp}[1]{[\![{#1}]\!]}
\newcommand{\interpE}[1]{\interp{#1}^e}
\newcommand{\interpC}[1]{\interp{#1}^c}

\newcommand{\interpF}[1]{[\![{#1}]\!]_f}
\newcommand{\interpmono}[1]{\interp{#1}^{\mathrm{m}}}

\newcommand{\entails}{\models}

\newcommand{\judgeE}[4][\Theta]{{#1};\;{#2} \vdash {#3} : {#4}}
\newcommand{\judgeC}[4][\Theta]{{#1};\;{#2} \vdash {#3} \div {#4}}
\newcommand{\judgeEq}[5][\Theta]{{#1};\;{#2} \vdash {#3} \equiv {#4} : {#5}}
\newcommand{\judgeEqC}[5][\Theta]{{#1};{#2} \vdash {#3} \equiv {#4} \div {#5}}




%% semantic operations

\newcommand{\worldleq}{\preceq}
\newcommand{\worldgeq}{\succeq}

\newcommand{\semfun}[2]{\lambda #1.\;#2}
\newcommand{\sempair}[2]{\left({#1}, {#2}\right)}
\newcommand{\powerset}[1]{\mathcal{P}(#1)}
\newcommand{\powersetfin}[1]{\mathcal{P}^{\mathrm{fin}}(#1)}

\newcommand{\paircat}[2]{\left<{#1};{#2}\right>}
\newcommand{\sumcat}[2]{\left[{#1};{#2}\right]}
\newcommand{\abscat}[1]{\lambda({#1})}


\newcommand{\judgeP}[3]{{#1} \vdash {#2} : {#3}}
\newcommand{\judgeS}[2][\Delta]{{#1} \vartriangleright {#2} : \spectype}

\newcommand{\judgeSCtx}[2]{{#1} \vartriangleright {#2} : \ctext{context}}

\newcommand{\judgeEqP}[4]{{#1} \vdash {#2} \equiv {#3} : {#4}}


\newcommand{\judgeWK}[3][\Theta]{{#1} \vdash {#2} : {#3}}
\newcommand{\judgeKeq}[4][\Theta]{{#1} \vdash {#2} \equiv {#3} : {#4}}

\newcommand{\entailsP}[3]{{#1} \vartriangleright {#2} \vdash {#3}}
\newcommand{\entailsS}[3]{{#1}; {#2} \vdash {#3} \ok}

\newcommand{\chartp}{\ctext{char}}
\newcommand{\fonttp}{\ctext{font}}

\newcommand{\LOC}{loc}
\newcommand{\MONO}{\mathbf{mono}}
\newcommand{\HEAP}{heap}
\newcommand{\PROP}{Prop}
\newcommand{\TRUE}{True}
\newcommand{\HPROP}{Prop}
\newcommand{\TYPE}{Type}

\newtheorem{prop}{Proposition}
\newtheorem{lemma}{Lemma}

\newcommand{\comprehend}[2]{\setof{{#1}\;|\;{#2}}}


% \newenvironment{proof}{\begin{comment}}{\end{comment}}

\newenvironment{proof}{\noindent\textbf{Proof.}}{\noindent\ensuremath{\Box}}

\newcounter{prooflinenum}
\newenvironment{tabbedproof}
   {\setcounter{prooflinenum}{0}
    \begin{tabbing}\;\;\=\;\;\;\;\;\=\;\;\;\;\=\;\;\;\;\=\;\;\;\;\=\;\;\;\;\=\;\;\;\;\=\;\;\;\;\=\;\;\;\;\=\\[-2em]}
   {\end{tabbing}}

\newcommand{\oo}{\addtocounter{prooflinenum}{1}\arabic{prooflinenum}\>\>}
\newcommand{\ooo}{\addtocounter{prooflinenum}{1}\arabic{prooflinenum}\>\>\>}
\newcommand{\oooo}{\addtocounter{prooflinenum}{1}\arabic{prooflinenum}\>\>\>\>}
\newcommand{\ooooo}{\addtocounter{prooflinenum}{1}\arabic{prooflinenum}\>\>\>\>\>}
\newcommand{\oooooo}{\addtocounter{prooflinenum}{1}\arabic{prooflinenum}\>\>\>\>\>\>}
\newcommand{\ooooooo}{\addtocounter{prooflinenum}{1}\arabic{prooflinenum}\>\>\>\>\>\>\>}
\newcommand{\oooooooo}{\addtocounter{prooflinenum}{1}\arabic{prooflinenum}\>\>\>\>\>\>\>\>}
\newcommand{\ooooooooo}{\addtocounter{prooflinenum}{1}\arabic{prooflinenum}\>\>\>\>\>\>\>\>\>}

\newcommand{\ox}{\>\>}
\newcommand{\oox}{\>\>\>}
\newcommand{\ooox}{\>\>\>\>}
\newcommand{\oooox}{\>\>\>\>\>}
\newcommand{\ooooox}{\>\>\>\>\>\>}
\newcommand{\oooooox}{\>\>\>\>\>\>\>}
\newcommand{\ooooooox}{\>\>\>\>\>\>\>\>}
\newcommand{\oooooooox}{\>\>\>\>\>\>\>\>\>}


\newenvironment{eqnproof}[1][]{${#1}$\begin{displaymath}\begin{array}{lcll}}
                         {\end{array}\end{displaymath}}

\newcommand{\eline}[3][]{{#1} & = & {#2} & \mbox{{#3}} \\}

\newcommand{\elines}[3][]{{#1} & = & \begin{array}{l} #2 \end{array} & \mbox{{#3}} \\}

\newcommand{\eclaim}[3][]{{#1} &  & {#2} & \mbox{{#3}} \\}
\newcommand{\efact}[2]{{#1} & & & \mbox{#2} \\}

\newcommand{\basicspec}[4]{[{#1}]\;{#2}\;[{#3}.\;{#4}]}

% Macros for type checking assertions

\newcommand{\pfun}[3]{\hat{\lambda} #1:#2.\;#3}
\newcommand{\restrictkind}[1]{({#1})\Downarrow_K}
\newcommand{\restricttype}[1]{({#1})\Downarrow_T}

\newcommand{\restricttyenv}[2]{{#2}\Downarrow^{#1}_K}
\newcommand{\restrictvals}[2]{{#2}\Downarrow^{#1}_T}
\newcommand{\judgeACtx}[1]{\vartriangleright {#1} \;\ctext{ok}}
\newcommand{\judgeA}[3][\Delta]{{#1} \vartriangleright {#2} : {#3}}
\newcommand{\judgeSort}[2][\Delta]{\judgeA[{#1}]{{#2}}{\ctext{sort}}}
\newcommand{\judgeSortEq}[3][\Delta]{{#1} \vartriangleright {#2} \equiv {#3} : \ctext{sort}}

\newcommand{\ms}[1]{\mathcal{#1}}
\newcommand{\Frame}[2]{{#1} \otimes {#2}}


\begin{document}

\begin{tabbing}
$Realize(i, \Phi, o, f) \triangleq$ \\
\qquad\= $\forall v:$\=$\,\stream{A}.\; \exists \phi:\stream{\formula}, u:\stream{\powersetfin{ecell}}.$ \\
\>\> $[\forall n:\N.\;$\=$\closed{\phi_n}{\domain{\phi_n} \cup \setof{i}} \;\land$ \\
\>\>\>$\domain{\phi_n} = \domain{\phi_{n+1}}\; \land$ \\
\>\>\>$\forall \psi.\; \unready{\psi}{i} \implies \unready{\psi \otimes \phi_n}{o}]$ \\
\>\> $\land \; \Phi = \phi_0$ \\
\>\> $\land \;  Sequence(i, \phi, o, u, f, v)$ \\[1em]


$Sequence(i, \phi, o, u, f, v) \triangleq$ \\
\> $\forall $\=$ n:\N, \phi_i, \phi'_i, u_i.$ \\
\>\> $\astep{\phi_i}{\readcell i}{\phi'_i}{v_n}{\setof{i}}{u_i} \; \land$ \\
\>\> $\domain{\phi_i} = \domain{\phi'_i} \land 
        \closed{\phi_i}{\domain{\phi_i}} \land 
        \closed{\phi_i}{\domain{\phi'_i}} \land
        \unready{\phi_i}{i} \land
        i \in u_i$ \\
\>\> $\Longrightarrow$ \\
\>\> $\astep{\phi_i \otimes \phi_n}{\readcell o}
              {\phi'_i \otimes \phi_{n+1}}{(f\;v)_n}
              {\setof{o}}{u_i \cup u_n}$ \\
\>\> $\land\; u \subseteq \domain{\phi_n} \land o \in u$ \\
\end{tabbing}

Now, we want to show the correctness of each of the basic operations. 

\section{Proof of $\liftop$}

We want to show:

\begin{displaymath}
\spec{G(\psi)}{\liftop f\;i}{o:B}{\exists \Phi.\; G(\Phi \otimes \psi) \land Realize(i,\Phi,o,map\;f)}
\end{displaymath}

So, executing the body of $\liftop f\;i$ gives us the post-state:

\begin{displaymath}
  G(\psi \otimes \cellneg{o}{\bind (\readcell i) \;(\semfun{v}{\return \fst{v}})})
\end{displaymath}

So we will take $\Phi = \cellneg{o}{\bind (\readcell i) \;(\semfun{v}{\return \fst{v}})}$, 
and then define: 

Now, we want to show $Realize(i, \Phi, o, map\;f)$. So we will assume
we have some $v:\stream{A}$, and then define $\phi$ and $u$ as follows:

\begin{displaymath}
  \begin{array}{lcl}
    \phi_n & = & \mbox{if}(n = 0, \Phi, \cellpos{o}{\ldots}{v_{n-1}}{\setof{i}}) \\
    u_n    & = & \setof{i} \\
  \end{array}
\end{displaymath}

Now, it's trivially the case that for arbitrary $n:\N$,
$\closed{\phi_n}{\domain{\phi_n} \cup \setof{i}}$, and that $\domain{\phi_n}
= \domain{\phi_{n+1}}$, and that $\forall \psi.\; \unready{\psi}{i}
\implies \unready{\psi \otimes \phi_n}{o}$. This deals with the first batch
of proof obligations. 

Next, we need to show $Sequence(i, \phi, o, u, f, v)$. Assume that we
have some $n$, $\phi_i$, $\phi'_i$, and $u_i$, such that

\begin{itemize}
\item $\astep{\phi_i}{\readcell i}{\phi'_i}{v_n}{\setof{i}}{u_i}$, 
\item $i \in u_i$, 
\item $\closed{\phi_i}{\domain{\phi_i}}$,
\item $\closed{\phi'_i}{\domain{\phi'_i}}$, and
\item $\domain{\phi_i} = \domain{\phi'_i}$, and
\item $\unready{\phi_i}{i}$
\end{itemize}

We have as our prestate either $\phi_i \otimes \cellneg{a}{\ldots}$,
or $\phi_i \otimes \cellpos{a}{\ldots}{v_{n-1}}{\setof{i}}$,
depending on whether or not $n$ is 0. Because $\unready{\phi_i}{i}$,
either way we know $\unready{\phi_i \otimes \celleither{a}{\ldots}}{a}$. 

Therefore, to evaluate $\readcell o$, we need to apply rule
$AUnready$. So we get the derivation tree: 

\begin{mathpar}

\inferrule*
  {\inferrule*
     {\astep{\phi_i}{\readcell i}{\phi'_i}{v_n}{\setof{i}}{u_i} \\
      \astep{\phi'_i}{\return (f\;v_n)}{\phi'_i}{f\;v_n}{\emptyset}{\emptyset}}
     {\astep{\phi_i}{cmd}{\phi'_i}{f\;v_n}{\setof{i}}{u_i}} \\
   \unready{\phi_i\otimes \phi_n}{o} }
  {\astep{\phi_i \otimes \phi_n}{\readcell o}
        {\phi'_i \otimes \cellpos{o}{\ldots}{f\;v_n}{\setof{i}}}{f\;v_n}
        {\setof{a}}{u_i \cup \setof{o}}}
\end{mathpar}


Which is exactly what we want, since $(map\;f\;v)_n = f\;v_n$, and $o \in \setof{o}$, and 
$\setof{o} \subseteq \setof{o}$. 

\section{Proof of $\parop$}

We want to show that if 

\begin{itemize}
\item $\forall a, \psi.\; \spec{G(psi)}{p\;a}{c:C}{\exists \Phi^f.\; G(\psi \otimes \Phi^f) \land
                                                   Realize(a, \Phi^f, c, f)}$ 
\item $\forall b, \psi.\; \spec{G(psi)}{p\;b}{d:D}{\exists \Phi^g.\; G(\psi \otimes \Phi^g) \land
                                                   Realize(b, \Phi^g, d, g)}$ 
\end{itemize}

then

\begin{displaymath}
  \forall i, \psi.\;\spec{G(\psi)}{\parop p\;q\;i}{o:C\times D}{\exists \Phi.\;
                                        G(\psi \otimes \Phi) \land 
                                        Realize(i, \Phi, o, g \circ f)}
\end{displaymath}

Assume $i, \psi$. After evaluating $\parop p\;q\;i$, we will have
the following state.
   
\begin{displaymath}
  G(\psi \otimes \hat{a} \otimes \hat{b} \otimes \Phi^f \otimes \Phi^g \otimes \hat{o})
  \land Realize(a, \Phi^f, c, f) \land Realize(b, \Phi^g, d, g)
\end{displaymath}

where 
\begin{displaymath}
  \begin{array}{lcl}
    \hat{a} & = & \cellneg{a}{\bind\;(\readcell i)\;(\semfun{v}{\return (\fst{v})})} \\
    \hat{b} & = & \cellneg{b}{\bind\;(\readcell i)\;(\semfun{v}{\return (\snd{v})})} \\
    \hat{o} & = & \cellneg{o}{\left[\begin{array}{l}
                               \bind (\readcell c) (\lambda v_c.\; \\
                               \bind (\readcell d) (\lambda v_d.\; \\
                               \;\;\return \pair{v_c}{v_d})) \\
                              \end{array}\right]} \\      
  \end{array}
\end{displaymath}

So we will take our $\Phi = \hat{a} \otimes \hat{b} \otimes \Phi^f \otimes \Phi^g \otimes \hat{o}$, and we want to show $Realize(i, \Phi, o, par\;g\; f)$. 

Now, we assume we have $v:\stream{A}$. Next, we'll instantiate
$Realize(a, \Phi^f, c, f)$ with $v = map\;\pi_1\;v$, and 
$Realize(b, \Phi^g, d, g)$ with $v = map\;\pi_2\;v$, which will give 
us streams $\phi^f, \phi^g, u^f, u^g$, from which we can define $\phi$ and $u$: 

\begin{displaymath}
\begin{array}{lcl}
  \phi_n & = & a_n \otimes b_n \otimes \phi^f_n \otimes \phi^g_n \otimes o_n \\
  u_n    & = & \setof{a, b, o} \cup u^f_n \cup u^g_n \\[1em]

  a_n & = & \mbox{if}(n = 0, \cellneg{a}{\ldots}, \cellpos{a}{\ldots}{(map\;\pi_1\;v)_{n-1}}{\setof{i}}) \\
  b_n & = & \mbox{if}(n = 0, \cellneg{b}{\ldots}, \cellpos{a}{\ldots}{(map\;\pi_2\;v)_{n-1}}{\setof{i}}) \\
  o_n & = & \mbox{if}(n = 0, \cellneg{o}{\ldots}, \cellpos{a}{\ldots}{(f \;v)_{n-1}}{\setof{i}}) \\
\end{array}
\end{displaymath}

Again, it's easy to see that for arbitrary $n:\N$: 

\begin{itemize}
\item $\closed{\phi_n}{\domain{\phi_n} \cup \setof{i}}$, because $\phi^f_n$ and $\phi^g_n$
are closed except for $a$ and $b$, respectively, which are within the domain of $\phi_n$,
and $o$ depends upon $c$ and $d$, which are within the domains of $\phi^f_n$ and $\phi^g_n$,
and $a$ and $b$ depend only upon $i$. 

\item $\domain{\phi_n} = \domain{\phi_{n+1}}$, because $\phi^f$ and $\phi^g$ satisfy
this property, and for all $n$ we add $a$, $b$ and $o$ to these two collections of cells. 

\item $\forall \psi.\; \unready{\psi}{i} \implies \unready{\psi \otimes \phi_n}{o}$. 
Here, if $n = 0$, $\unready{\psi \otimes \phi_0}{o}$ is automatic, since $o$ is in
a negative state. For $n > 0$, we know from $\unready{\psi}{i}$. This means that 
$\unready{\psi \otimes \celleither{a}{\ldots}}{a}$. In turn this means we know that 
$\unready{\psi \otimes a^- \otimes \phi^f_n}{c}$, and since $c$ is in the read set
of $o$, $\unready{\psi \otimes \phi_n}{o}$ holds. 
\end{itemize}

To show $Sequence(i, \phi, o, u, f, v)$, we assume an arbitrary $n,
\phi_i, \phi'_i, u_i$, such that

\begin{itemize}
\item $\astep{\phi_i}{\readcell i}{\phi'_i}{v_n}{\setof{i}}{u_i}$, 
\item $i \in u_i$, 
\item $\closed{\phi_i}{\domain{\phi_i}}$,
\item $\closed{\phi'_i}{\domain{\phi'_i}}$, and
\item $\domain{\phi_i} = \domain{\phi'_i}$, and
\item $\unready{\phi_i}{i}$
\end{itemize}

We'll first consider a $\phi^f_i$ defined as $\phi_i \otimes a_n$, and
$\phi'^f_i$ defined as $\phi'_i \otimes a_{n+1}$.  Because $\phi_i
\otimes a$ is closed with the same domain before and after, and
because $a$ is unready because $\phi_i$ is, we can use $AUnready$ to
prove that

\begin{displaymath}
  \astep{\phi^f_i}{\readcell a}{\phi'^f_i}{(map\;\pi_1\;v)_n}{\setof{a}}{u_i \cup \setof{a}}
\end{displaymath}



We'll use this to discharge the hypothesis of $Sequence(a, \phi^f, c, u^f,
f, map\;\pi_1\;v)$. This tells us that:

\begin{itemize}
\item $\astep{\phi_i \otimes a_n \otimes \phi^f_n}{\readcell c}
                          {\phi'_i \otimes a_{n+1} \otimes \phi^f_{n+1}}{(f \;(map\;\pi_1)\;v)_n}
                          {\setof{c}}{u_i \cup \setof{a} \cup u^f_n}$ 
\item $u^f_n \subseteq \domain{\phi^f_n}$
\item $c \in u^f_n$
\end{itemize}

Next, we'll take a $\phi^g_i$ defined as $\phi'_i \otimes \cellneg{b}{\ldots})$, and
$\phi'^g_i$ defined as $\phi'_i \otimes b_{n+1}$. Now, this means that 
$b$ is unready. So applying rule $AUnready$, we get: 

\begin{displaymath}
  \astep{\phi^g_i}{\readcell b}{\phi'^g_i}{(map\;\pi_2\;v)_n}{\setof{b}}{\setof{b}}
\end{displaymath}

This evidently satisfies the conditions for the implication for our
instance of $Sequence(b,\phi^g, c, u^g, g, map\;\pi_2\;v)$, which means we can conclude that: 

\begin{itemize}
\item $\astep{\phi^g_i \otimes \phi^g_n}{\readcell d}
             {\phi'^g_i \otimes \phi^g_{n+1}}{(f \;(map\;\pi_2)\;v)_n}
             {\setof{d}}{\setof{d} \cup u^g_n}$ 
\item $u^g_n \subseteq \domain{\phi^g_n}$ 
\item $d \in u^g_n$ 
\end{itemize}

With the preliminaries out of the way, assume that our state is
$G(\phi_i \otimes \phi_n)$, and that we wish to evaluate $\readcell
o$.  Now, we know that $\unready{\phi_i}{i}$, which means that
$\unready{\phi_i \otimes \phi_n}{o}$ holds, so we have to use the rule
$AUnready$, which calls for evaluating the code in $o$. The state is
updated as follows:

\begin{tabbing}
$\{G(\phi_i \otimes a_n \otimes \phi^f_n \otimes b_n \otimes \phi^g_n)\}$ \\
$\readcell c$ \\
$\{G(\phi'_i \otimes a_{n+1} \otimes \phi^f_{n+1} \otimes R(u_i \cup \setof{a} \cup u^f_n,  b_n \otimes \phi^g_n))\}$ \\
Since $i \in u_i$, we have: \\
$\{G(\phi'_i \otimes \cellneg{b}{\ldots} \otimes \phi^f_{n+1} 
   \otimes a_{n+1} \otimes
   R(u_i \cup \setof{a} \cup u^f_n, \phi^g_n))\}$ \\
Since $\closed{\phi^g_n}{\domain{\phi^g_n} \cup \setof{b}}$, we know: \\
$\{G(\phi'_i \otimes \cellneg{b}{\ldots} \otimes \phi^f_{n+1} 
   \otimes a_{n+1} \otimes \phi^g_n)\}$ \\
$\readcell d$ \\
$\{G(\phi'_i \otimes b_{n+1} \otimes \phi^g_{n+1} 
   \otimes R(\setof{b} \cup u^g_n, a_{n+1} \otimes
                                           \phi^f_{n+1}))\}$ \\
Now, because of the closure properties, we know that:\\
$\{G(\phi'_i \otimes b_{n+1} \otimes \phi^g_{n+1} 
   \otimes a_{n+1} \otimes \phi^f_{n+1})\}$ \\
\end{tabbing}
We can also see that the value returned is $\pair{(f\;(map\;\pi_1\;v))_n}{(g\;(map\;\pi_2\;v))_n}$, which is equal to $(par\;f\;g\;v)_n$, and that the update set will be $u_i \cup \setof{a,b} \cup u^f_n \cup u^g_n$. 

Then, we can finish applying the $AUnready$ rule to get the post-state of:

$$\{G(\phi'_i \otimes b_{n+1} \otimes \phi^g_{n+1} 
   \otimes a_{n+1} \otimes \phi^f_{n+1} \otimes o_{n+1})\}$$

and an update set of $u_i \cup \setof{a,b,o} \cup u^f_n \cup u^g_n$, which is what
we want. 

Notice that we implicitly used the disjointness of the different parts of the
cell heap here. 

\section{Proving $\switchop$}

We want to show that if 

\begin{itemize}
\item $\forall \psi, i.\; \spec{G(\psi)}{p\;i}{o^f:B}
                               {\exists \Phi^f.\;G(\psi \otimes \Phi^f) \land 
                                 Realize(i, \Phi^f, o^f, f)}$
\item $\forall \psi, i.\; \spec{G(\psi)}{q\;i}{o^g:B}
                               {\exists \Phi^f.\;G(\psi \otimes \Phi^g) \land 
                                 Realize(i, \Phi^g, o^g, g)}$
\end{itemize}

then 
\begin{displaymath}
\forall \psi, i, k.\; \spec{G(\psi)}{\switchop k\;p\;q\;i}{o:B}
                                 {\exists \Phi.\; G(\psi \otimes Phi) \land
                                  Realize(i, o, b, (take\;k\;(f\;v)) \cdot (g\;(drop\;k\;v)))}
\end{displaymath}

Assume we have $\psi, i, k$. Then, if we execute $\switchop k\;p\;q\;i$, we get the state:
\begin{displaymath}
  G(\psi \otimes \Phi^f \otimes \Phi^g \otimes \localref{r}{0} \otimes \cellneg{o}{code}) \land
    Realize(i, \Phi^f, o^f, f) \land Realize(i, \Phi^g, o^g, f)
\end{displaymath}

where $code$ is 

\begin{displaymath}
code = \begin{array}{l}
              \bind (\getref r)\; (\lambda i:\N. \\ 
              \bind (\setref r (i+1))\; (\lambda dummy:\unittype. \\
              \;\;\readcell \mbox{if}(i < k, o^f, o^g))) \\
       \end{array}
\end{displaymath}

Now, to prove our goal, let's take $\Phi = \Phi^f \otimes \Phi^g \otimes \localref(r, 0) \otimes \cellneg{o}{code}$. Therefore, we need to show $Realize(i, \Phi, o, switch\;k\;f\;g)$. Now, assume that 
$v:\stream{A}$. 

So, assume we have some stream $v$. 

We want to give $\phi$ and $u$. First, we instantiate the quantifiers
in $Realize(i, \Phi^f, o^f, f)$ with $v$ to get $\phi^f$ and $u^f$,
with the expected properties. Then, we instantiate the quantifier in
$Realize(i, \Phi^g, o^g, g)$ with $tail\;k\;v$ to get a $\phi^g$ and
$u^g$.  Then, we can give a definition of $\phi$ and $u$ as:

\begin{displaymath}
  \begin{array}{lcll}
    \phi_0 & = & \phi^f_0 \otimes \phi^g_0 \otimes \localref{r}{0} \otimes \cellneg{o}{code} & \\
    \phi_n & = & \phi^f_n \otimes R(\setof{i}, \phi^g_0) \otimes 
                 \localref{r}{i} \otimes \cellpos{o}{code}{(f\;v)_{i-1}}{\setof{o^f}}
           & \mbox{if } 0 < i < k  \\
    \phi_n & = & R(\setof{i}, \phi^f_k) \otimes \phi^g_{n-k} 
                \otimes \localref{r}{n} \otimes \cellpos{o}{code}{(g\;(drop\;k\;v)_{n-k}}{\setof{o^g}}
           & \mbox{if } n \geq \max(k,1) \\
  \end{array}
\end{displaymath}

Now, we need to show that for all $n$, 

\begin{itemize}
\item $\closed{\phi_n}{\domain{\phi_n}\cup\setof{i}}$ holds, because
  we know that $\phi^f_n$ and $\psi^g_n$ are closed with respect to
  their respective domains, plus $i$, and that $o$ only ever depends
  on $o^f$ or $o^g$, so the whole thing is closed. (The local ref
  doesn't matter.).

\item $\domain{\phi_n} = \domain{\phi_{n+1}}$, because we know that
  $\domain{\phi^f_n} = \domain{\phi^g_n}$, and otherwise we only have
  $o$ in our domain, so the whole thing has a fixed domain.

\item To show for all $\psi$, $\unready{\psi}{i} \implies
  \unready{\psi \otimes \phi_n}{o}$ holds, we need to do a case
  split. Assume we have a $\psi$ such that $\unready{\psi}{i}$. If $n
  = 0$, then $o$ is in a negative state, and so is immediately
  unready. If $0 < n < k$, then we have
  $\cellpos{o}{code}{v}{\setof{o^f}}$, and we know that
  $\unready{\psi}{i}$ means that $\unready{\psi \otimes
    \phi^f_n}{o^f}$, which means that $o$ is unready. If $n \geq \max(k,1)$,
  then we have $\cellpos{o}{code}{v}{\setof{o^g}}$, and we know that
  if $\unready{\psi}{i}$, then $\unready{\psi \otimes \phi^g_n}{o^g}$,
  which lets us conclude that $o$ is unready.
\end{itemize}


Now, we assume that we have $n, \phi_i, \phi'_i, u_i$ such that

\begin{itemize}
\item $\astep{\phi_i}{\readcell i}{\phi'_i}{v_n}{\setof{i}}{u_i}$, 
\item $i \in u_i$, 
\item $\closed{\phi_i}{\domain{\phi_i}}$,
\item $\closed{\phi'_i}{\domain{\phi'_i}}$, and
\item $\domain{\phi_i} = \domain{\phi'_i}$, and
\item $\unready{\phi_i}{i}$
\end{itemize}

We need to show that $\astep{\phi_i \otimes \phi_n}{\readcell o}{\phi'_i \otimes \phi_{n+1}}{(switch\;k\;f\;g)_n}{\setof{o}}{u_i \cup u_n}$

\begin{displaymath}
G(\phi_i \otimes \phi_n)  
\end{displaymath}

Now, we'll split on $k$. Suppose $k > 0$. 

Now we have two cases for $n$. 

Suppose that $n < k$. Then, if $i$ is zero then $o$ is unready
immediately, and otherwise it's unready because $\unready{\phi_i}{i}$
holds. So, we need to evaluate $o$'s code, which means reading
$o^f$. We know $Sequence(i, \phi^f, o^f, u^f, f, v)$, and can
discharge the hypotheses with the assumptions made above, giving us

\begin{itemize}
\item $\astep{\phi_i \otimes o^f_n}{\readcell o}
             {\phi'_i \otimes \o^f_{n+1}}{(f\;v)_n}
             {\setof{o^f}}{u_i \cup u^f_n}$
\item $u^f_n \subseteq \domain{o^f}$
\item $o^f \in u^f_n$
\end{itemize}


So we can evaluate the body of $o$ as follows:

\begin{tabbing}
$\setof{G(\phi_i \otimes \localref{r}{n} \phi^f_n \otimes R(i, \phi^g_0))}$ \\
$\bind (\getref r) \;(\lambda i.$ \\
$\{G(\phi_i \otimes \localref{r}{n} \otimes \phi^f_n \otimes R(i, \phi^g_0)) \land i = n\}$ \\
$\bind (\setref r\;(i+1)) \;(\lambda dummy.\;$ \\
$\{G(\phi_i \otimes \localref{r}{n+1} \otimes \phi^f_n \otimes R(i, \phi^g_0)) \land i = n\}$ \\
$\readcell \mbox{if}(i < k, o^f, o^g)$ \\
$\equiv$
$\readcell o^f$ \\
$\{G(\phi'_i \otimes \localref{r}{n+1} \otimes \phi^f_{n+1} \otimes R(i \cup u^f_n, \phi^g_0) \land i = n \land a = (f\;v)_n\}$ \\
By closure \\
$\{G(\phi'_i \otimes \localref{r}{n+1} \otimes \phi^f_{n+1} \otimes R(i, \phi^g_0) \land i = n \land a = (f\;v)_n\}$ \\
\end{tabbing}

Now, observe that $(switch\;k\;f\;g\;v)_n = (f\;v)_n$ since $n < k$.
And then we finish by adding a ready $o$-cell in the $AUnready$ rule,
which makes the post-state $\phi_{n+1}$.

Now, suppose that $n \geq k$. Now, we instantiate $Sequence(i, \phi^g, o^g, f, (drop\;k\;v))$ 
with $n - k$, and $\phi_i, \phi'_i,$ and $u_i$. This is okay because $(drop\;k\;v)_{n-k} = v_n$,
and so this is sufficient to discharge the hypothesis, giving us the conclusions:

\begin{itemize}
\item $\astep{\phi_i \otimes o^g_{n-k}}{\readcell o}
             {\phi'_i \otimes \o^f_{n-k+1}}{(g\;(drop\;k\;v))_{n-k}}
             {\setof{o^g}}{u_i \cup u^g_{n-k}}$
\item $u^f_{n-k} \subseteq \domain{o^f}$
\item $o^f \in u^f_{n-k}$
\end{itemize}

With this in hand, we can evaluate the code of $o$ as follows:

\begin{tabbing}
$\{ \phi_i \otimes
  G(R(\setof{i}, \phi^f_k) \otimes \phi^g_{n-k} \otimes 
  \localref{r}{n})
\}$ \\
$\bind (\getref r) (\lambda i.\;$ \\
$\{ \phi_i \otimes
  G(R(\setof{i}, \phi^f_k) \otimes \phi^g_{n-k} \otimes 
  \localref{r}{n})
  \land i = n
\}$ \\
$\bind (\setref r\;(i+1)) (\lambda dummy.$ \\ 
$\{ \phi_i \otimes
  G(R(\setof{i}, \phi^f_k) \otimes \phi^g_{n-k} \otimes 
  \localref{r}{n+1})
  \land i = n
\}$ \\
$\readcell \mbox{if}(i < k, o^f, o^g)$ \\
$\equiv$
$\readcell o^g$ \\
$\{ \phi'_i \otimes 
  G(R(u_i \cup u^g_{n-k} cup \setof{i}, \phi^f_k) \otimes \phi^g_{n+1-k} \otimes 
  \localref{r}{n+1})
  \land i = n \land a = g\;(drop\;k\;v)_{n-k}
\}$ \\
By closure
$\{ \phi'_i \otimes 
  G(R(\setof{i}, \phi^f_k) \otimes \phi^g_{n+1-k} \otimes 
  \localref{r}{n+1})
  \land i = n \land a = g\;(drop\;k\;v)_{n-k}
\}$ \\
\end{tabbing}

Now, observe that $(switch\;k\;f\;g\;v)_n = (g\;(drop\;k\;v))_{n-k}$ since $n \geq k$.
And then we finish by adding a ready $o$-cell in the $AUnready$ rule,
which makes the post-state $\phi_{n+1}$.

\section{Proving $\loopop$}

\subsection{Purely functional semantics}

Here, we'll have to actually define $loop$ and friends. Suppose that
$f$ is a causal stream function $\stream{A \times B} \to \stream{A
  \times C}$, with corresponding list function $\hat{f}$, and that
$a_0$ is in $A$, and $v$ is in $\stream{B}$.

\begin{tabbing}
  $loop\;a_0\;f\;v = map \; \pi_2 \; (cycle \;a_0 \; f \; v)$ \\[1em]

  $cycle\; a_0 \;f \;v \;n = last(gen\;a_0\;f\;v\;n)$ \\[1em]

  $gen\;a_0\;f\;v\;0\;\;\;\qquad$\=$= \hat{f}\;[\pair{a_0}{v_0}]$ \\
  $gen\;a_0\;f\;v\;(n+1)$  \>$= \hat{f}\;(zip\;(a_0 \cdot (map\;\pi_1\;(gen\;a_0\;f\;v\;n)))\;
                                               (take\;(n+2)\;v))$ \\
\end{tabbing}

The function $cycle$ is well-defined, because $gen$ always returns a
list one element larger than its integer argument. 
The theorem we need to prove about this loop function is the following
property:

\begin{displaymath}
  cycle\;a_0\;f\;v = f\;(zip\;(a_0 \cdot (map\;\pi_1\;(cycle\;a_0\;f\;v)))\;v)
\end{displaymath}

We'll prove this by means of the well-known $take$-lemma, which says
that two streams are equal if all of their finite approximations (by
means of $take$) are equal.  First, observe that the definition of
$cycle$ means that $take\;(n+1)\;(cycle\;a_0\;f\;v) =
gen\;a_0\;f\;v\;n$.

Now, we'll proceed by an induction on the length. 

First, the 0-th approximations of a stream are always the empty list.

Now, we'll look at what $take\;(n+1)\;(f\;(zip\;(a_0 \cdot
(map\;\pi_1\;(cycle\;a_0\;f\;v)))\;v))$ is. If the length is exactly 1, then 

\begin{tabbing}
$take\;1\;(cycle\;a_0\;f\;v) = $ \\
\;\;\=$= gen\;a_0\;f\;v\;0$ \\
    \>$= \hat{f}[\pair{a_0}{v_0}]$ \\
    \>$= \hat{f}(take\;1\;(zip\;(a_0 \cdot (map\;\pi_1\;(cycle\;a_0\;f\;v)))\;v))$ \\
    \>$= take\;1\;(f\;(zip\;(a_0 \cdot (map\;\pi_1\;(cycle\;a_0\;f\;v)))\;v))$\\
\end{tabbing}

\noindent If the length is $n + 2$ for some $n$, then 


\begin{tabbing}
$take\;(n+2)\;(f\;(zip\;(a_0 \cdot (map\;\pi_1\;(cycle\;a_0\;f\;v)))\;v))$ \\
\;\;\=$= \hat{f}(take\;(n+2)\;(zip\;(a_0 \cdot (map\;\pi_1\;(cycle\;a_0\;f\;v)))\;v))$ \\
  \>$= \hat{f}(zip\;(take\;(n+2)\;(a_0 \cdot (map\;\pi_1\;(cycle\;a_0\;f\;v))))\;
                  \;(take\;(n+2)\;v))$ \\
  \>$= \hat{f}(zip\;(a_0 \cdot (take\;(n+1)\;(map\;\pi_1\;(cycle\;a_0\;f\;v))))
                  \;(take\;(n+1)\;v))$ \\
  \>$= \hat{f}(zip\;(a_0 \cdot (map\;\pi_1\;(take\;(n+1)\;(cycle\;a_0\;f\;v))))
                  \;(take\;(n+1)\;v))$ \\
  \>$= \hat{f}(zip\;(a_0 \cdot (map\;\pi_1\;(gen\;a_0\;f\;v\;n)))
                  \;(take\;(n+1)\;v))$ \\
  \>$= gen\;a_0\;f\;v\;(n+1)$ \\
  \>$= take\;(n+2)\;(cycle\;a_0\;f\;v)$ \\
\end{tabbing}

Therefore, for all $n$, $take\;n\;(cycle\;a_0\;f\;v) = f\;(zip\;(a_0\cdot(map\;\pi_1\;(cycle\;a_0\;f\;v)))\;v)$. 

\subsection{Transducer semantics}

Now, we want to show that if 

\begin{itemize}
\item $\forall \psi, ab.\; \spec{G(\psi)}{p\;i}{ac:A\times C}{\exists \Phi^f.\; G(\psi \otimes \Phi) \land Realize(ab, \Phi^f, ac, f)}$
\end{itemize}
then $\forall \psi, i.\; \spec{G(\psi)}{\loopop\;a_0\;f}{o:C}{\exists \Phi.\; G(\psi \otimes \Phi) \land Realize(i, \Phi, o, loop\;a_0\;f)}$ \\

Assume we have $\psi$ and $i$. Executing $\loopop\;a_o\;p\;i$ will leave us in the state: 
\begin{displaymath}
  G(\psi \otimes \cellneg{ab}{cmd_1} \otimes \localref{r}{a_0} \otimes \Phi^f \otimes \cellneg{o}{cmd_2}) \land Realize(ab, \Phi^f, ac, f)
\end{displaymath}
where $cmd_1$ is 
\begin{displaymath}
  \begin{array}{l}
    \bind (\readcell i) \;(\lambda b.\;  \\
    \bind (\getref r)   \;(\lambda a.\;  \\ 
    \;\;\return \pair{a}{b})) \\
  \end{array}
\end{displaymath}
and $cmd_2$ is 
\begin{displaymath}
  \begin{array}{l}
    \bind (\readcell ac) \;(\lambda v.\; \\
    \;\;\return (\snd{ac})) \\
  \end{array}
\end{displaymath}

\noindent Now, we'll take $\Phi$ to be 
\begin{displaymath}
  \psi \otimes \cellneg{ab}{cmd_1} \otimes \localref{r}{a_0} \otimes \Phi^f \otimes \cellneg{o}{cmd_2}
\end{displaymath}
and try to show $Realize(i, \Phi, o, loop\;a_0\;f)$. 

First, assume we have $v \in \stream{B}$. Then, we'll instantiate $Realize(ab, \Phi^f, ac, f)$ with $(zip\;(a_0\cdot(map\;\pi_1\;(cycle\;a_0\;f\;v)))\;v)$, and get $\phi^f$ and $u^f$

Now, we'll define $\phi$ and $u$ as follows: 
\begin{displaymath}
  \begin{array}{lcl}
    \phi_0 & = & \Phi \\
    \phi_{n+1} & = & \cellpos{ab}{cmd_1}{(zip\;(a_0\cdot(map\;\pi_1\;(cycle\;a_0\;f\;v)))\;v)_n}{\setof{i}} \otimes \\
               & &   \localref{r}{(a_0\cdot(map\;\pi_1\;(cycle\;a_0\;f\;v))\;v)_{n+1}} \otimes \\
               & &   \phi^f_{n+1} \otimes 
                     \cellpos{o}{cmd_2}{(map\;\pi_2\;(cycle\;a_0\;f\;v))_n}{\setof{ac}} \\[1em]

    u_n & = & u^f_n \cup \setof{ab, o} \\
  \end{array}
\end{displaymath}

Now, we need to verify the properties for all $n$: 
\begin{itemize}
\item $\closed{\phi_n}{\domain{\phi_n} \cup \setof{i}}$ holds because $\phi^f_n$ is closed with
respect to itself and $ab$, and the other two cells $ab$ and $o$ only mention $i$ or things
withing $\phi_n$. 

\item $\domain{\phi_n} = \domain{\phi_{n+1}}$ because the cells $ab$ and $o$ are fixed
throughout all $n$, and we know from $Realize(ab, \Phi^f, ac, f)$ that $\phi^f$ doesn't
change its domain, either. 

\item To show that for all $\psi$, $\unready{\psi}{i}$ implies $\unready{\psi \otimes \phi_n}{o}$, 
we note that if $n = 0$, then $o$ is immediately unready. If $n > 0$, then $\unready{\psi}{i}$ implies $\unready{\psi \otimes \cellpos{ab}{cmd_1}{\ldots}{\setof{i}}}{\setof{ab}}$, which in turn
implies \\ $\unready{\psi \otimes a^+ \otimes \phi^f_n}{\setof{ac}}$, which in turn implies \\$\unready{\psi \otimes a^+ \otimes \phi^f_n \otimes \cellpos{o}{cmd_2}{\ldots}{\setof{ac}}}{\setof{ac}}$.
\end{itemize}

Now, assume that we have some $n, \phi_i, \phi'_i, u_i$, such that: 

\begin{itemize}
\item $\astep{\phi_i}{\readcell i}{\phi'_i}{v_n}{\setof{i}}{u_i}$, 
\item $i \in u_i$, 
\item $\closed{\phi_i}{\domain{\phi_i}}$,
\item $\closed{\phi'_i}{\domain{\phi'_i}}$, and
\item $\domain{\phi_i} = \domain{\phi'_i}$, and
\item $\unready{\phi_i}{i}$
\end{itemize}

Now, we'll consider $\phi^f_i = \phi_i \otimes \localref{r}{\ldots} \otimes \celleither{a}{cmd_1}$, and $\phi'^f_i = \phi'_i \otimes \localref{r}{\ldots} \otimes \celleither{a}{cmd_1}$, and $\phi'^f_i$, where the local reference and the cell $ab$ are as in $\phi_n$. Now, it's clear that the
following properties hold: 

\begin{itemize}
\item $\astep{\phi^f_i}{\readcell ab}{\phi'^f_i}{(zip\;(a_0\cdot(map\;\pi_1\;(cycle\;a_0\;f\;v)))\;v)_n}{\setof{i}}{u_i \cup \setof{ab}}$, 
\item $ab \in u_i \cup \setof{ab}$, 
\item $\closed{\phi^f_i}{\domain{\phi^f_i}}$,
\item $\closed{\phi'^f_i}{\domain{\phi'^f_i}}$, and
\item $\domain{\phi^f_i} = \domain{\phi'^f_i}$, and
\item $\unready{\phi^f_i}{ab}$
\end{itemize}

This means that we can discharge the hypotheses of the $Sequence$ property of $\phi^f$, giving
us the property: 

\begin{itemize}
\item $\astep{\phi^f_i \otimes \phi^f_n}{\readcell ac}
             {\phi'^f_i \otimes \phi^f_{n+1}}{(f\;(zip\;(a_0\cdot(map\;\pi_1\;(cycle\;a_0\;f\;v)))\;v))_n}
             {\setof{ac}}{u_i \cup \setof{ab} \cup u^f_n}$

This is the same thing as 

$\astep{\phi^f_i \otimes \phi^f_n}{\readcell ac}
             {\phi'^f_i \otimes \phi^f_{n+1}}{(cycle\;a_0\;f\;v)_n}
             {\setof{ac}}{u_i \cup \setof{ab} \cup u^f_n}$
\item $u^f_n \subseteq \domain{\phi^f_n}$
\item $ac \in u^f_n$
\end{itemize}

Now, we need to show that $\astep{\phi_i \otimes \phi_n}{\readcell o}
                                 {\phi'_i \otimes \phi_{n+1}}{(loop\;a_0\;f\;v)}
                                 {\setof{o}}{u_i \cup u_n}$. 

Since $\unready{\phi_i}{i}$, we know that $\unready{\phi_i \otimes \phi_n}{o}$, which
means we need to use the rule $AUnready$ and evaluate the code of $o$. We do this
as follows: 

\begin{tabbing}
$\{G(\phi_i \otimes ab_n 
     \otimes \localref{r}{(a_0\cdot(map\;\pi_1\;(cycle\;a_0\;f\;v))\;v)_n} 
     \otimes \phi^f_n)\}$ \\
is the same as \\
$\{G(\phi^f_i \otimes \phi^f_n)\}$ \\
$\bind (\readcell ac) \;(\lambda v.$ \\
$\{G(\phi'^f_i \otimes \phi^f_{n+1}) \land v = (cycle\;a_0\;f\;v)_n\}$ \\
which is the same as \\
$\{G(\phi'_i \otimes ab_{n+1}
     \otimes \localref{r}{(a_0\cdot(map\;\pi_1\;(cycle\;a_0\;f\;v))\;v)} 
     \otimes \phi^f_{n+1}) \land v = (cycle\;a_0\;f\;v)_n\}$ \\
$\bind (\setref r \;(\pi_1 (cycle\;a_0\;f\;v)_n)) \;(\lambda dummy.\;$ \\
$\{G(\phi'_i \otimes ab_{n+1}
     \otimes \localref{r}{(a_0\cdot(map\;\pi_1\;(cycle\;a_0\;f\;v))\;v)_{n+1}} 
     \otimes \phi^f_{n+1}) \land v = (cycle\;a_0\;f\;v)_n\}$ \\
$\;\return (\fst{v})))$ \\
$\{G(\phi'_i \otimes ab_{n+1}
     \otimes \localref{r}{(a_0\cdot(map\;\pi_1\;(cycle\;a_0\;f\;v))\;v)_{n+1}} 
     \otimes \phi^f_{n+1}) \land a = \pi_1\;(cycle\;a_0\;f\;v)_n\}$ \\
which is the same as \\
$\{G(\phi'_i \otimes ab_{n+1}
     \otimes \localref{r}{(a_0\cdot(map\;\pi_1\;(cycle\;a_0\;f\;v))\;v)_{n+1}} 
     \otimes \phi^f_{n+1}) \land a = (map\;\pi_1\;(cycle\;a_0\;f\;v))_n\}$ \\

\end{tabbing}



\end{document}
