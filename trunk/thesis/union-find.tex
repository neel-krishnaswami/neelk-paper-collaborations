\chapter{Proving the Union-Find Disjoint Set Algorithm}

\section{Introduction}

In this chapter, we introduce the technique of ``ramification'', as a
way of recovering local reasoning in the face of imperative programs
with global invariants.

The union-find disjoint set data structure~\cite{union-find} is a
technique for efficiently computing canonical representives for
equivalence classes of values. The basic technique for doing so is to
represent each value in the equivalence class as a node in a tree ---
but unlike the usual implementation of trees, each node does not
contain a pointer to its children, but rather the children each
maintain a pointer to the parent. The root of the tree has no parent
pointer, and represents the canonical representative for an
equivalence class.

The canonical representative can be found (the $\find$ operation) by
following the parent pointers to the root of the tree. Similarly, two
disjoint sets can be merged (the $\union$ operation), by finding their
canonical representatives and setting one to point to the other.

As described, this data structure is no better than using a linked
list. However, two optimizations give rise to an extremely efficient
implementation~\cite{galler-fischer-union-find}. First, the root node
can be modified to keep track of a bound on the maximum height, so
that whenever two sets are merged, the shorter tree can be made a
subtree of the deeper one. Second, the algorithm can make use of
\emph{path compression} -- whenever the $\find$ operation is called,
it can set all of the nodes on the path to the root to point directly
at the root. Together, these optimizations permit performing a
sequence of $n$ $\union$ and $\find$ operations in $O(n \cdot
\alpha(n))$ time, where $\alpha$ is the inverse Ackermann
function~\cite{tarjan-union-find-bound}.

This permitted Huet~\cite{huet-unification} to give a simple
implementation of near-linear-time unification algorithms, and
variants of this idea are used in the proofs of congruence closure
algorithms~\cite{congruence-closure}.

However, path compression is an idiom difficult to accomodate within
the framework of separation logic. In informal reasoning about the
union-find data structure, we do not explicitly track all the elements
of a union-find data structure in our reasoning --- instead, we rely
on the fact that path compression only makes changes to the heap which
our global program invariant should be insensitive to. However,
separation logic is a resource-aware logic, which demands to know the
footprint of any command. So we cannot simply leave the other elements
of the equivalence class out of the invariant, since $\union$ may read
and modify them.

The solution I propose in this chapter is to use a global invariant
structured in a way which preserves modular reasoning, both in the
style of separation logic, and in the interference-insensitive style
of the usual informal proofs. 

However, we do not only want to hide interference! One of the features
which makes union-find so elegant is that the $\union$ operation
features a well-structured use of aliased mutable state.  When merging
two equivalence classes, a single update allows efficiently
\emph{broadcasting} the change to every element of both classes in
almost constant time.  So we need a specification technique that
should also let us specify global interference in a clean,
well-structured way. We achieve this by introducing a
\emph{ramification operator} in our specification, which gives us an
abstract way of characterizing the information propagated globally.

To understand the idea of ramifications, we look back to McCarthy's
original paper introducing the frame problem~\cite{mccarthy}. There,
he described the frame problem as the problem of how to specify what
parts of a state were \emph{unaffected} by the action of a command,
which inspired the name of the frame rule in separation logic. In that
paper, he also described the \emph{qualification problem}. He observed
that many commands (such as a flipping a light switch turning on a
light bulb) have numerous implicit preconditions (such as there being
a bulb in the light socket), and dubbed the problem of identifying
these implicit preconditions the qualification problem.

Some years later, \citet{finger} observed that the qualification
problem has a dual: actions can have indirect effects that are not
explicitly stated in their specification (e.g., turning on the light can
startle the cat). He called the problem of deducing these implicit
consequences the ``ramification problem'' --- is there a simple way to
represent all of the indirect consequences of an action? If so, then 
we have a way of modularly specifying \emph{global} effects. This is
the idea we will adopt to deal with the broadcast nature of \textsf{union}
in the union-find algorithm. 

\section{The Specification}

\begin{figure}
  \begin{mathpar}
    \begin{array}{lcl}
      \formula & = & I \bnfalt \mathsf{Elt\;of\;}\tau \times \tau \bnfalt \mathsf{Tensor\;of\;}\formula \times \formula \\
      \phi, \psi & ::= & I \bnfalt \elt{x}{y} \bnfalt \phi \otimes \psi
    \end{array}
  \end{mathpar}
  \begin{mathpar}
    \begin{array}{lcl}
      R & : & (\tau \to \tau) \times \formula \to \formula \\ 
      R(\rho, I) & = & I \\
      R(\rho, \psi \otimes \phi) & = & R(\rho, \psi) \otimes R(\rho, \phi) \\
      R(\rho, \elt{x}{y}) & = & \elt{x}{\rho(y)} \\
    \end{array}
  \end{mathpar}

  \begin{mathpar}
    \boxed{\ufcontains{\phi}{x}{y}}
    \\
    \inferrule*[]
              { }
              {\ufcontains{\elt{x}{y}}{x}{y}}
    \and
    \inferrule*[]
              {\ufcontains{\phi}{x}{y}}
              {\ufcontains{\phi \otimes \psi}{x}{y}}
    \and
    \inferrule*[]
              {\ufcontains{\psi}{x}{y}}
              {\ufcontains{\phi \otimes \psi}{x}{y}}
  \end{mathpar}
  \begin{specification}
    \nextline
    $\exists \tau : \star$ 
    \nextline 
    $\exists H : \formula \to \assert$ 
    \nextline
    $\exists \newset : \monad{\tau}$ 
    \nextline
    $\exists \find : \tau \to \monad{\tau}$ 
    \nextline
    $\exists \union : \tau \times \tau \to \monad{\unittype}$
    \nextline 
      $\spec{H(\phi)}{\newset}{a:\tau}{H(\phi \otimes \elt{a}{a})}$
    \\ \> $\specand$
    \nextline 
      $\spec{H(\phi) \land (\ufcontains{\phi}{x}{y})}
            {\find(x)}
            {a:\tau}
            {H(\phi) \land a = y}$
    \\ \> $\specand$
    \nextline 
      $\spec{H(\phi) \land (\ufcontains{\phi}{x}{y})
                     \land (\ufcontains{\phi}{u}{v})}
            {\union(x, u)}
            {a:1}
            {H(R([y/v], \phi))}$ 
    \\ \> $\specand$
    \nextline
      $\setof{\forall \phi, a,b,x,y.\;H(\phi \otimes \elt{a}{b} \otimes \elt{x}{y}) \implies a \not= x}$
    \\ \> $\specand$
    \nextline
      $\setof{\forall \phi.\;H(\phi) \implies H(I)}$
    \\ \> $\specand$
    \nextline
      $\setof{\forall \phi, \psi.\;H(\phi \otimes \psi) \iff H(\psi \otimes \phi)}$
    \\ \> $\specand$
    \nextline
      $\setof{\forall \phi.\;H(I \otimes \phi) \iff H(\phi)}$
    \\ \> $\specand$
    \nextline
      $\setof{\forall \phi, \psi, \theta.\;H(\phi \otimes (\psi \otimes \theta)) \iff H((\phi \otimes \psi) \otimes \theta)}$
  \end{specification}

\caption{Specification of Union Find Algorithm}
\label{union-find:spec}
\end{figure}

In Figure~\ref{union-find:spec}, we give our specification of the
union-find algorithm. 

On lines 1-5 of the specification, we specify that there is an
abstract type $\tau$ of nodes of the disjoint set forest, and three
operations $\newset$, $\find$, and $\union$, which create new nodes,
find canonical representatives, and merge equivalence classes,
respectively. Furthermore, there is a monolithic abstract predicate
$H(\phi)$, which describes the entire disjoint-set forest all at once.

This monolithic predicate represents of one of the two tricks of this
specification. Our first trick is to replay the key idea of separation
logic, only ``one level up''. Even though we have a single abstract
predicate describing the whole forest, we can recover separation-style
modularity by indexing the abstract predicate with a formula $\phi$,
which gives a small (in fact, nearly degenerate) ``separation logic''
for describing elements of these equivalence classes.  

The datatype of formulas is given in the display above the
specification. We give it both as an inductive type, to illustrate why
it is definable in higher-order, and as a grammar (which is more
readable, and what we use in the specifications). Formulas have three
grammatical productions. First, we have the forms $I$ and $\phi
\otimes \psi$, which are an (intuitionistic) unit and separating
conjunction for the elements of this little logic. We also have an
atomic formula $\elt{x}{y}$, which says that $x$ is a term whose
canonical representative is $y$. 

These formulas have the usual resource-aware interpretation of
separation logic, so that $\elt{x}{y} \otimes \elt{a}{b}$ implies that
$x \not= a$. % \todo{Add these entailments as axioms}

We also give a simple judgement on formulas $\ufcontains{\phi}{x}{y}$,
which lets us say that $\phi$ entails drawing the conclusion that
$x$'s representative is $y$. This is of course a trivial judgement,
since the language of formulas is so simple.

In addition to this separation logic, our second trick is embodied in
the modal operator $R(\rho, \phi)$, which we call a ``ramification
operator''. Here, $\rho$ is a substitution, and $\phi$ is a
formula. $R$ is defined to operate homomorphically on the structure of
formulas, with its action on atomic formulas being $R(\rho,
\elt{x}{y}) = \elt{x}{\rho(y)}$. Intutively, a ramification $R(\rho,
\phi)$ says to replace the canonical witnesses in the domain of the
substitution with the result of the substitution. This lets us specify
the aliasing effects of global updates in a modular fashion.

So we have the tools to reason both locally and globally in our
specifications.  An example of local reasoning can be seen on line 6
of Figure~\ref{union-find:spec}.  Here, we say that given a state
$H(\phi)$, calling $\newset$ will result in a new state $H(\phi
\otimes \elt{a}{a})$. This functions a bit like a global axiom for
creating new equivalence classes, since we explicitly quantify over
the frame.

By quantifying over $\phi$, we can \emph{implement} the frame rule for
our library. This is similar to the interpretation of the frame rule
in our underlying semantics -- there, we interpret Hoare triples to mean all
the assertions that can be safely framed on, and here, we quantify over
all possible frames. 

On line 7, we see the necessity of this kind of interpretation. Our
specification for $\find$ is very simple -- it says that in any state
which entails $\elt{x}{y}$, calling $\find(x)$ will return $y$. The
$\phi$ is unchanged from precondition to postcondition, and so the
user of this library does not need to know anything about any elements
other than $x$. However, due to path compression, we can modify 
many other nodes in the forest, a fact which our domain-specific
logic conceals. 

On line 8, we give the specification of $\union$. Here, we say that if
$x$'s witness is $y$, and $u$'s witness is $v$, then calling
$\union(x,u)$ will equate the two equivalence classes, setting $u$'s
witness to $y$. Furthermore, since this is globally visible, we need
to push this ramification over the entire set of known nodes
$\phi$. Observe that unlike in the previous function we do not want
local reasoning, since the purpose the $\union$ operation is to
globally broadcast the update. But the use of a ramification operator
does \emph{structure} this update.

On lines 9-13, we add axioms corresponding to our domain-specific
logic.  On line 9, we say that disjoint elements are disjoint, and on
line 10 we say that we can forget the existence of elements (i.e., that
this logic is like intuitionistic separation logic). On lines 11-13,
we simply say that formulas are commutative, unital and associative. 

\begin{figure}
\mbox{}
\begin{specification}
\nextline $\sigma = \Child \of \reftype{\sigma} \bnfalt \Root \of \opttype\;{(\reftype{\sigma})} \times \N$ 
\nextline $\tau = \reftype{\sigma}$ 

\nextline
   $H(\phi) \triangleq \exists D \subseteq \tau, p \in D \rightharpoonup D, w \in (D-\domain{p}) \to D.\;G(D,p,w,\phi)$
\nextline[1em] $G(D, p, w, \phi) =\;$\=$p^{+} \mbox{ strict partial order} \;\land$ 
\nextline \> $w \mbox{ injective} \;\land$ 
\nextline \> $(D, p, w) \models \phi \;\land$
\nextline \> $\heap(D, p, w)$ 

\nextline[1em] $D, p, w \models I \qquad\qquad$\=$\mbox{ iff}\;\;$\=$\mbox{always}$ 
\nextline      $D, p, w \models \phi \otimes \psi$\>$\mbox{ iff}$\>$
                     \exists D_1, D_2.\; D = D_1 \uplus D_2 \mbox{ and } D_1, p, w \models \phi \mbox{ and } 
                     D_2, p, w \models \psi$
\nextline      $D, p, w \models \elt{x}{y}$ \> $\mbox{ iff }$\>$x \in D \land \exists z.\;(x,z) \in p^* \land z\mbox{ maximal} \land w(z) = y$

\nextline[1em] $\heap(D, p, w) = \;\;$\=
    $\forall^{*} l \in \domain{p}.\; l \pointsto \Child(p(l))$ 
\nextline\>  $* \; \forall^{*} l \in (D - \domain{p}).\; \exists n.\;l \pointsto \Root(\Some(w(l)), n)$ 
\end{specification}
\caption{Concrete Definition of Union Find Invariants}
\label{union-find:invariant}
\end{figure}
 
In Figure~\ref{union-find:invariant} we give the invariant for the
union-find data structure. The node type $\tau$ is a pointer to a 
term of type $\sigma$, which is either a $\Child$ value containing
the parent of the current node, or a $\Root(w, n)$ value containing
the witness plus a number to maintain balance.\footnote{We do not track the
ranks of subtrees in our invariant to avoid obscuring the essential
simplicity of the techniques underpinning ramification, though it is
straightforward to add.}

Then on line 3, we say that $H(\phi)$ holds when there are $D$, $p$, and
$w$ such that $G(D, p, w, \phi)$ holds. We require $D$ to be a finite set of
nodes $D$, $p : D \rightharpoonup D$ to be a partial map of nodes to parents, and a 
map $w : (D - \domain{p}) \to D$. The set of nodes $D$ represent all the elements that have
been allocated, and the parent map $p$ maps each node to its
parent. The function $p$ is partial since some nodes are root nodes of
the disjoint-set forest. The map $w : (D - \domain{p}) \to D$ sends
those roots to the appropriate canonical witness.

Then, on lines 4-7, we define exactly how $G$ works. In order to
ensure that the graph structure actually has no cycles, on line 4 we
impose the condition that the transitive closure of $p$ is a strict
partial order. That is, we require that $p^+$ is a transitive,
irreflexive relation. Since the domain $D$ is finite, this also 
insures that $p^+$ has no infinite ascending chains. 

In terms of graph structure, this requirement means that the nodes
form a directed acyclic graph, and the fact that it arises as the
closure of a function means that no node has more than one parent.
Together, these conditions ensure that the nodes form a forest. On
line 5, we assert that $w$ is an injective function, which ensures
that the canonical representatives for different equivalence classes
are all distinct from one another. On line 6, we assert that the
triple $(D, p, w)$ models our formulas.

These formulas are a small subset of separation logic. We have the
formula $I$, which is always satisfied. (So this logic is like an
intuitionistic rather than classical separation logic.) Then we have
the formula $\phi \otimes \psi$, which corresponds to the usual
separating conjunction, in that the resource $n$ (the collection of
disjoint-set nodes) is split into two parts, one of which must support
$\phi$ and the other of which must support $\psi$. Note that the whole
of the parent function $p$ and the canonical witness function $w$
functions are passed to both branches. This is the information that
will let us ensure that global constraints are maintained in local
invariants. Finally, the atomic proposition $\elt{x}{y}$ asserts that
$x$ is a node whose canonical witness is $y$, by saying that $x$ is in
$D$, and that there is some maximal $z$ above $x$ in the reflexive
transitive closure of $p$ (viewed as a partial order), such that $w(z)
= y$. (Note that $z$ is maximal precisely when it is not in the domain
of $p$.)

Then, the predicate $\mathit{heap}(D, p, w)$ asserts that every node
in $D$ is correctly realized by some physical node in the heap. Every
child node must point to its parent, and every root node must point 
to its witness and some count of its tree rank. 

\begin{figure}
\mbox{}
\begin{specification}
\nextline $\newset = [$\=$\letv{r}{[\newref{\sigma}{\Root(\None, 0)}]}{}$
\nextline              \>$\letv{\unit}{[r := \Root(\Some(r), 0)]}{}$
\nextline              \>$r]$

\nextline[1em] $\ctext{findroot}(x) = [$\=$
                  \letv{i}{\comp{!x}}{}$ 
\nextline \>     $\run{}\ctext{case}($\=$i,$ 
\nextline \> \>     $\Root(w, n) \to [(x, w, n)],$
\nextline \> \>     $\Child(p) \to [$\=$\letv{(r, w, n)}{\findroot(p)}{}$ 
\nextline \> \>                      \>$\letv{\_}{[x := \Child(r)]}{}$ 
\nextline \> \>                      \>$(r,w,n)]$

\nextline[1em] $\find(x) = [\letv{(\_, w, \_)}{\ctext{findroot}(x)}{w}]$ 

\nextline[1em] $\union(x, y) = [$\= 
             $\letv{(r, u, m)}{\findroot\;x}{}$ 
\nextline \> $\letv{(s, v, n)}{\findroot\;y}{}$ 
\nextline \> $\run{}$\=$\ctext{if}\;r \not= s\;\ctext{then}$ 
\nextline \>\>\;\;\=   $\ctext{if}\;m < n\;\ctext{then}$ 
\nextline \>\> \>   \;\;$[$\=$\letv{\unit}{[s := \Root(u, n)]}{}$ 
\nextline \>\> \>         \>$r := \Child(s)]$ 
\nextline \>\> \> $\ctext{else}\;\ctext{if}\;n < m\;\ctext{then}$ 
\nextline \>\> \>   \;\;$[s := \Child(r)]$
\nextline \>\> \> $\ctext{else}$ 
\nextline \>\> \>   \;\;$[\letv{\unit}{[r := \Root(u, n+1)]}{}$
\nextline \>\> \>      \>$s := \Child(r)]$
\nextline \>\> $\ctext{else}$ 
\nextline \>\> \> $[\unit]]$
\end{specification}
\caption{Implementation of Union-Find Algorithm}
\label{union-find:impl}  
\end{figure}

In Figure~\ref{union-find:impl}, we give the implementation of these functions. 
On lines 1-3, we give the code for creating a new element. This works by allocating
a pointer of type $\tau$, and then updating it so that it points to itself as its
canonical witness. 

The $\find$ function is defined on line 10, but it works by deferring
almost all of its work to an auxiliary function $\findroot$. This
function is defined on lines 4-9, and it works by recursively
following the parent pointers of each node. When it reaches a root, it
returns a triple $(r, w, n)$, containing all three of the physical
root $r$, the witness value $w$, and the tree rank $n$. On a
recursive call (i.e., when the argument is a child node), we take the
return triple, and before returning, we implement path compression, by
updating the child's parent to be the root node $r$.  The $\find$
function simply calls $\findroot$, and ignores its return values
except for the witness.

The reason we have this auxilliary function becomes evident in the
$\union$ function, on lines 11-23. Given arguments $(x,y)$, what
$\union$ does is to first call $\findroot$ on both $x$ and $y$ (on
lines 11 and 12). Then, on line 13, we compare the two physical roots
for inequality. If they are equal, then there is no work to be done
(lines 22-23). Otherwise, we compare the two returned tree ranks, and
add the smaller root as a child of the larger one (lines 14-19), and
increment the size counter if they are the same rank (lines 20-21).

\section{Correctness Proofs}

All of these proofs have a similar structure. First, we prove a lemma about how 
the properties of the parent order $p$ can change. Second, we prove how this changes
(or doesn't change) the satisfaction of a formula $\phi$. Third, we use these
two properties to verify the program itself, in an annotated specification style.




\begin{lemma}{(Soundness of Entailment)}
  If $D, p, w \models \phi$ and $\ufcontains{\phi}{u}{v}$, then $D, p, w \models \elt{u}{v}$. 
\end{lemma}

\begin{proof}
  This proof follows from an easy induction on the derivation of $\ufcontains{\phi}{u}{v}$. 
\end{proof}

\begin{lemma}{(Disjointness of Elements)}
  The formula $\forall \phi,a,b,x,y.\;H(\phi \otimes \elt{a}{b} \otimes \elt{x}{y}) \implies a \not= x$
  is valid.   
\end{lemma}
\begin{proof}
  \begin{tabbedproof}
    \oo Assume $\phi,a,b,x,y$ and $H(\phi \otimes \elt{a}{b} \otimes \elt{x}{y})$ \\
    \ooo So we know that there are $D, p,$ and $w$ such that $D, p, w \models \phi \otimes \elt{a}{b} \otimes \elt{x}{y}$ \\
    \ooo By the definition of $\otimes$, we know that there are disjoint $D_1, D_2$ and $D_3$, such that\\
    \ooox $D_1, p, w \models \phi$ \\
    \ooox $D_2, p, w \models \elt{a}{b}$ \\
    \ooox $D_3, p, w \models \elt{x}{y}$ \\
    \ooo So we know $a \in D_2$ and $x \in D_3$ \\
    \ooo Since $D_2$ and $D_3$ are disjoint, $a \not= x$ 
  \end{tabbedproof}
\end{proof}

\begin{lemma}{(Structural Properties)}
The following properties are valid:
\begin{itemize}
\item $\forall \phi.\;H(\phi) \implies H(I)$
\item $\forall \phi,\psi.\;H(\phi \otimes \psi) \iff H(\psi \otimes \phi)$
\item $\forall \phi.\;H(I \otimes \phi) \iff H(\phi)$
\item $\forall \phi,\psi,\theta.\;H(\phi \otimes (\psi \otimes \theta)) \iff H((\phi \otimes \psi) \otimes \theta)$
\end{itemize}
\end{lemma}
\begin{proof}
These properties follow immediately from the semantics of assertions. 
\end{proof}

\subsection{Proof of $\ctext{find}$}

Suppose $R \subseteq D \times D$ is a strict partial order. We say $x
\in D$ is \emph{maximal} when there is no $y$ such that $(x,y) \in R$.
Note that when we have a partial function $f : D \rightharpoonup D$
such that $f^+$ is a strict partial order, $x \in D$ is maximal
precisely when $f(x)$ is not defined. (If $f(x)$ were defined, then
$(x, f(x)) \in f^+$. Hence it cannot be defined for any maximal
element.)

\begin{lemma}{(Path Compression Lemma)}
  Suppose $D$ is a finite set and $f : D \rightharpoonup D$ is a
  partial function on $D$, such that $f^{+}$ is a strict partial
  order. Now suppose $(x,y) \in f^{+}$ with $y$ maximal.

Then, $[f|x:y]$ has the same domain as $f$, and $[f|x:y]^+$ is a
strict partial order such that for all $u,v \in D$, $(u,v) \in
[f|x:y]^{+}$ with $v$ maximal if and only if $(u,v) \in f^{+}$ with
$v$ maximal.
\end{lemma}

\begin{proof}
\begin{enumerate}
\item 
Since $(x,y) \in f^+$, we know that $x \in \domain{f}$. Hence $[f|x:y]$
has the same domain as $f$. 

\item 
For $[f|x:y]^+$ to be a strict partial order, it must be an
irreflexive transitive relation.  Since it is the transitive closure
of a function, it is immediately transitive. 

To show that it is irreflexive, assume that $(u, u) \in
[f|x:y]^+$. Therefore there is some sequence $(u, [f|x:y](u), \ldots,
[f|x:y]^{n+1}(u) = u)$.

Now consider whether $x$ is in the sequence. If it is not, then this
sequence is equal to $(u, f(u), \ldots, f^{n+1}(u) = u)$, which is a
contradiction, since $f^+$ is irreflexive. If $x$ is in the sequence,
then we know that either $x$ is the last element, or the
second-to-last element (since $f(x) = y$, and $y$ is maximal).
Suppose $x$ is the last element. Then we know that $u = x$, and hence
$[f|x:y](u) = y$, and so we have a contradiction. Suppose $y$ is the
last element. Then the first element is $y$, and there is no $f(y)$,
since $y$ is a maximal element, and this is also a
contradiction. 

Therefore there is no $(u, u) \in [f|x:y]^+$, and so $[f|x:y]^+$ is
irreflexive.

\item
\begin{itemize}
\item[$\Leftarrow$] Suppose that $(u, v)$ in $f^{+}$ with $v$
  maximal. Then there is some $k$ such that $f^k(u) = v$. So we have a
  sequence $u, f(u), \ldots, f^k(u)$.

If any of the $f^i(u) = x$ for $i < k$, then we know that $v = y$,
since both $v$ and $y$ are maximal. Therefore, it follows that the
sequence $u, [f|x:y](u), \ldots, [f|x:y]^i(u), y$ shows that $(u,v)
\in [f|x:y]^+$ with $v$ maximal.

If none of the $f^i(u) = x$, then it follows that the sequence $u,
f(u), \ldots, f^k(u)$ is exactly the same as $u, [f|x:y](u), \ldots,
[f|x:y]^k(u)$, and so $(u, v) \in [f|x:y]^+$ with $v$ maximal.

\item[$\Rightarrow$] Suppose $(u,v) \in [f|x:y]^{+}$ with $v$
  maximal. Then we know that there is some sequence $u, [f|x:y](u),
  \ldots, [f|x:y]^{k+1}(u)$ with $[f|x:y]^{k+1}(u) = v$.

  Now, either $x$ occurs in this sequence, or not. Suppose that 
  $x$ does not occur in the sequence -- that is, for every $i \leq k+1$,
  $[f|x:y]^i(u) \not= x$. Then it follows that this sequence is 
  the same as $(u, \ldots, f^{k+1}(u) = v)$. Since $[f|x:y]$ and $f$
  have the same domain, it follows that $v$ is maximal in $f$, 
  as well. Hence $(u, v) \in f^+$ with $v$ maximal. 

  Now, suppose that $x$ occurs in the sequence at position $i$. It
  cannot occur at $i = k+1$, since $f^{k+1}(u) = v$ is the last
  element, and we know $v$ is maximal. But since $[f|x:y](x) = y$, we
  know that $x$ is not maximal in $[f|x:y]$. Therefore it occurs at $i
  \leq k$.

  However, it also cannot occur at any $i < k$, since $y$ is maximal
  in $[f|x:y]$, and if $x$ occured at $i < k$, then the sequence would
  end at $i+1$ with $y$, and we know the sequence is $k+1$ elements
  long. Therefore, $x$ occurs at position $i = k$, and $(u, \ldots,
  [f|x:y]^k(u), [f|x:y]^{k+1}(u) = v) = (u, \ldots, f^k(u) = x, [f|x:y]^{k+1}(u) = y = v)$.

  Now, we know that $(x,y) \in f^+$ with $y$ maximal in $f$. Therefore,
  there is a sequence $(x, \ldots, f^{n+1}(x) = y)$. Therefore, 
  the sequence $(u, \ldots, f^k(u) = x, \ldots, f^{n+k+1}(u) = y = v)$ is 
  in $f^+$, with $v$ maximal in $f$. 

\end{itemize}
\end{enumerate}
\end{proof}

\begin{lemma}{(Satisfaction Depends on Maximality)}
Suppose $D$ is a finite set and $f,g : D \rightharpoonup D$ are
partial functions on $D$, such that $f^+$ and $g^+$ are strict partial
orders such that for all $x,y \in D$, we have that $(x,y) \in f^+$
with $y$ maximal in $f$ if and only if $(x,y) \in g^+$ with $y$
maximal in $g$.

Then for all $\phi, w$ and $D' \subseteq D$, if $D', f, w \models \phi$,
then $D', g, w \models \phi$. 
\end{lemma}

% \begin{lemma}{(Path Compression Preserves Satisfaction)}
% Suppose $D$ is a finite set and $f : D \rightharpoonup D$ is a partial function on $D$,
% such that $f^{+}$ is a strict partial order, and that $(x,y) \in f^{+}$ with $y$ maximal.
% 
% For all $\phi$, $w$ and $D' \subseteq D$, if $D', f, w \models \phi$, then $D', [f|x:y], w \models \phi$.  
% \end{lemma}


\begin{proof}
This lemma follows by induction on $\phi$. 
\begin{itemize}
\item Case $\phi = I$. This follows immediately from the definition of satisfaction. 
\item Case $\phi = \psi \otimes \theta$. 
  \begin{tabbedproof}
    \oo By assumption, we have $D', f, w \models \psi \otimes \theta$. \\
    \ooo From the definition of satisfaction, we get  $D_1, D_2$ such that $D' = D_1 \uplus D_2$ \\
    \oox and $D_1, f, w \models \psi$ and $D_2, f, w \models \theta$.  \\
    \ooo By induction, we get $D_1, g, w \models \psi$ \\
    \ooo By induction, we get $D_2, g, w \models \theta$ \\
    \ooo By definition, we get $D_1 \cup D_2, g, w \models \psi \otimes \theta$ \\
  \end{tabbedproof}
\item Case $\phi = \elt{x}{y}$.
  \begin{tabbedproof}
    \oo By assumption, $D', f, w \models \elt{x}{y}$ \\
    \oo So we know that $x \in D'$ and there exists an $z$ such that \\
    \ox $(x,z) \in f^*$ with $z$ maximal and $y = w(z)$ \\
    \oo By definition of reflexive transitive closure, either $(x, z) \in f^+$ or $x = z$ \\
    \oo Suppose $(x,z) \in f^+$:\\
    \ooo Then by hypothesis, $(x,z) \in g^+$ with $m$ maximal.  \\
    \ooo Hence $D', g, w \models \elt{x}{z}$  \\
    \oo Suppose $x = z$: \\
    \ooo Since $x$ is maximal, we know $x \not\in \domain{f}$. \\
    \ooo Suppose $x \in \domain{g}$. \\
    \oooo Then there is a sequence $(x, g(x), \ldots, u) \in g^+$ with $u$ maximal \\
    \oooo Then $(x, f(x), \ldots, u) \in f^+$ with $u$ maximal \\
    \oooo This is a contradiction, since $x$ is maximal in $f$ \\
    \ooo So $x \not\in \domain{g}$ \\
    \ooo Hence $(x, x) \in g^*$ and $x$ maximal in $g$ \\
    \ooo Hence $(x, z) \in g^*$ and $z$ maximal in $g$ \\
    \ooo Hence $D', g, w \models \elt{x}{y}$
  \end{tabbedproof}
\end{itemize}
\end{proof}


Now, we can specify and prove the correctness of $\findroot$. 

\begin{lemma}{(Correctness of $\findroot$)}
The $\findroot$ function satisfies the following specification:
\begin{specification}
\nextline $\{G(D, f, w, \phi) \land (D,f,w) \models \elt{u}{v}\}$
\nextline $\findroot(u)$
\nextline $\{(r,c,n):\tau \times \tau \times \N.$ 
\nextline\;\;\= $\exists f'.\; G(D, f', w, \phi) \land c = v \land (u,r) \in f'^{*} \land r \mbox{ maximal} \land w(r) = c \land \domain{f} = \domain{f'}$
\nextline\> $\land\; (f,f')\mbox{ have same maximal relationships}\}$
\end{specification}
\end{lemma}

Here, $f$ and $f'$ ``having the same maximal relationships'' means that $(u,v) \in f^+$ with $v$ maximal
if and only if $(u, v) \in (f')^+$ with $v$ maximal. 

\begin{proof}
\begin{tabbedproof}
\oo Assume our precondition state is $G(D, f, w, \phi) \land (D,f,w) \models \elt{u}{v}$ \\
\ooo Then we have a sequence $(u, \ldots, f^k(u) = x)$ with $x$ maximal and $w(x) = v$. \\
\ooo Then we know that $f^+$ is a strict partial order, and $D,f,w \models \phi$ and \\
\oox $\mathit{heap}(D,f,w)$ and $D,f,w \models \elt{u}{v}$ \\
\ooo Now, do a case split on whether or not $u \in \domain{f}$ (i.e., is maximal). \\
\ooo If $u \in \domain{f}$: \\
\oooo Then we have a sequence $(u, f(u), \ldots, f^k(u) = x)$ with $x$ maximal and $w(x) = v$ \\
\oooo Hence $(D, f, w) \models \elt{f(u)}{v}$ \\ 
\oooo Next, $\mathit{heap}(D,f,w) \implies (u \hookrightarrow \Child(f(u))$ \\
\oooo $[\letv{i}{[!u]}{}$ \\
\oooo So we add $i = \Child(f(u))$ to the state, and simplify the case statement \\
\oooo $\letv{(r,c,n)}{\findroot(f(u))}{}$ \\
\oooo Then $G(D, f', w, \phi) \land (f(u), r) \in f'^* \land r\mbox{ maximal} \land c = w(r) \land c = v$ for some $f'$ \\
\oooo with the same domain and maximal relationships as $f$\\
\oooo We know that $(f(u), r) \in f'^*$ and $r$ is maximal \\
\oooo If $f(u) = r$, then $(f(u), r) \in f^*$ and $r$ is maximal \\
\oooo If $f(u) \not= r$, then: \\
\ooooo $(f(u), r) \in f'^+$ and $r$ is maximal \\
\ooooo Therefore $(f(u), r) \in f^+$ and $r$ is maximal, since \\
\oooox $f$ and $f'$ have the same domain and maximal relationships \\
\oooo Therefore $(f(u), r) \in f^*$ and $r$ is maximal \\
\oooo From line 6, $(f(u), x) \in f^*$ and $x$ maximal \\
\oooo Therefore $r = x$ \\
\oooo From line 4, $(u, r) \in f^+$ and $r$ is maximal \\
\oooo Hence $(u, r) \in f'^+$ and $r$ is maximal, since \\
\oooox $f$ and $f'$ have the same domain and maximal relationships \\
\oooo So we know that $r = x$, and that $(u, r) \in f'^*$ \\ 
\oooo So $D, f', w \models \phi$ and $(f')^+$ is a strict partial order and $w$ injective and $\mathit{heap}(D, f', w)$, \\
\oooo Since we know $r$ remains maximal in $f'$, we know that $(D, f', w) \models \elt{u}{v}$, \\
\ooox since satisfaction depends only on maximality \\ 
\oooo Therefore, due to path compression, $[f'|u:r]^+$ is a strict partial order \\
\oooox with the same maximal relationships and domain as $f'$\\
\oooo Hence $D, [f'|u:r], w \models \phi$ and $\domain{[f|u:r]} = \domain{f}$ and $(u, r) \in [f'|u:r]^*$, \\
\ooox since satisfaction depends only on maximality \\ 
\oooo Now we will perform updates to make the physical heap match this logical heap \\ 
\oooo $\letv{\_}{[u := \Child(r)]}{}$ \\
\oooo Hence $\mathit{heap}(D, [f'|u:r], w)$, together with the pure formulas above \\
\oooo $(r,c,n)]$ \\
\oooo Choosing as the existential witness $[f'|u:r]$, we get \\
\oooo Hence $(r, c, n).\; \exists f'.\;G(D, f', w, \phi) \land c = v \land w(r) = c \;\land$ \\
\oooox $\domain{f} = \domain{f'} \land (u,r) \in f'^{*} \land r\mbox{ maximal} \;\land$ \\
\oooox $(f,f')$ have the same maximal relationships. \\
\ooo If $u \not\in \domain{f}$:  \\
\oooo Then $\mathit{heap}(D,f,w) \implies (u \hookrightarrow \Root(\Some(p), n))$ for some $n$ and $p = w(u)$\\
\oooo $[\letv{i}{[!u]}{}$ \\
\oooo So we add $i = \Root(\Some(w(u)), n)$ to the state, and simplify the case statement \\
\oooo $(u, p, n)]$\\
\oooo Note $(u, u) \in f^*$ and $u$ maximal, and that $p = w(u)$, and that $\domain{f} = \domain{f}$ \\
\oooo The fact that $D, f, w \models \elt{u}{v}$ implies $v = w(u)$\\ 
\oooo And obviously $f$ has the same maximal relationships as $f$ \\
\oooo Hence $\exists f''.\;G(D, f'', \phi) \land c = v \land w(r) = c \;\land$ \\
\oooox $\domain{f} = \domain{f'} \land (u,r) \in f''^{*} \land r\mbox{ maximal} \;\land$ \\
\oooox $(f,f')$ have the same maximal relationships. \\
\oooo (with the choice of $f$ as witness for $f''$) \\
\end{tabbedproof}
\end{proof}

\begin{lemma}{(The $\find$ Function is Correct)}
  The $\find$ function meets the specification in Figure~\ref{union-find:spec}. 
\end{lemma}

\begin{proof}
This proof is easy, since $\findroot$ does almost all the work.
\begin{tabbedproof}
\oo Assume a precondition of $H(\phi)$ and $\ufcontains{\phi}{u}{v}$ \\
\ooo This means we have $G(D, f, w, \phi)$ for some $D$, $f$, and $ w$. \\
\ooo Furthermore, we also know that $D, f, w \models \elt{u}{v}$. \\
\ooo Now, expand the call $\find(u)$: \\
\ooo $[\letv{(\_, v', \_)}{\findroot(u)}{}$ \\
\ooo Now we know $G(D, f', w, \phi) \land \domain{f} = \domain{f'} \land v' = v$ \\
\ooo $v']$ \\
\ooo Now we know $G(D, f', w, \phi) \land \domain{f} = \domain{f'} \land a = v$ 
\end{tabbedproof}
\end{proof}

\subsection{Proof of $\newset$}

\begin{lemma}{(Satisfaction and Allocation)}
Suppose $D, f, w \models \phi$ and $x \not \in D$. 

Then $D \cup \setof{x}, f, [w|x:x] \models \phi \otimes \elt{x}{x}$
\end{lemma}

\begin{proof}
\begin{tabbedproof}
\oo To prove this, we want to exhibit $D_1, D_2$ such that $D \cup \setof{x} = D_1 \uplus D_2$, \\
\ox and $D_1, f, [w|x:x] \models \phi$ and $D_2, f, [w|x:x] \models \elt{x}{x}$. \\ 
\oo Take $D_1 = D$ and $D_2 = \setof{x}$, which are disjoint since $x \not\in D$. \\ 
\oo Note that $\setof{x}, f, [w|x:x] \models \elt{x}{x}$, since:  \\
\oox $x \in \setof{x}$  \\
\oox $(x, x) \in f^*$ (since this is a reflexive relation) \\
\oox $x$ is maximal (since it is not in the domain of $f$) \\
\oox $[w|x:x](x) = x$. \\
\oo Now it remains to be shown that $D, f, [w|x:x] \models \phi$ \\
\oo By induction on $\phi$, we will show  \\
\oox $\forall D, \phi,$ if $x \not\in D$ and $D, f, w \models \phi$ then $D, f, [w|x:x] \models \phi$ \\
\ooo Case $\phi = I$:\\ 
\oooo This case is immediate, since $D, f, [w|x:x] \models I$ by definition \\ 
\ooo Case $\phi = \psi \otimes \theta$: \\    
\oooo Since $D, f, w, \models \psi \otimes \theta$, there are $D_1, D_2$ so $D_1, f, w \models \psi$ and $D_2, f, w \models \theta$ \\
\ooox and $D = D_1 \uplus D_2$ \\
\oooo Since $x \not \in D$, we know $x \not\in D_1$ and $x \not\in D_2$ \\
\oooo By induction, we know $D_1, f, [w|x:x] \models \psi$ \\ 
\oooo By induction, we know $D_2, f, [w|x:x] \models \theta$ \\
\oooo Hence $D, f, [w|x:x] \models \phi$ \\
\ooo Case $\phi = \elt{u}{v}$ \\
\oooo So we know $u \in D$ and that $(u,z) \in f^{*}$ and $z$ maximal and $w(z) = v$ for some $z$ \\
\oooo Since $x \not \in D$, $x \not\in D - \domain{f}$, and so it follows that $[w|x:x](z) = v$ \\
\oooo Hence $D, f, [w|x:x] \models \elt{u}{v}$ \\
\oo Hence $D, f, [w|x:x] \models \elt{u}{v}$ 
\end{tabbedproof}
\end{proof}

\begin{lemma}
The $\newset$ procedure meets the specification in Figure~\ref{union-find:spec}. 
\end{lemma}

\begin{proof}
\begin{tabbedproof}
\oo Assume we have a precondition state $H(\phi)$, and consider the body of $\newset$. \\
\ooo Then we know that $f^+$ is a strict partial order, and $D,f,w \models \phi$ and \\
\oox $\mathit{heap}(D,f,w)$ and $D,f,w \models \elt{u}{v}$ \\
\ooo $[\letv{r}{[\newref{\sigma}{\Root(\None, 0)}]}{}$ \\
\ooo Now the state is $\mathit{heap}(D,f,w) * r \pointsto \Root(\None, 0)$, plus the pure predicates. \\
\ooo $\letv{\_}{[r := \Root(\Some(r), 0)]}{}$ \\
\ooo Now the state is $\mathit{heap}(D,f,w) * r \pointsto \Root(\Some(r), 0)$, plus the pure predicates. \\
\ooo Since $\mathit{heap}(D,f,w)$ has a pointer for each $l \in D$, it follows that $r \not \in D$. \\
\ooo Thus, we know that $D', f, [w|r:r] \models \phi \otimes \elt{r}{r}$ where $= D \uplus \setof{r}$ \\ 
\ooo $f^+$ is still a strict partial order which is a subset of $D' \times D'$ \\
\ooo So $f \in D' \rightharpoonup D'$ \\
\ooo $[w|r:r] \in (D' - \domain{f}) \to D'$ \\
\ooo It is clear that $\mathit{heap}(D,f,w) * r \pointsto \Root(\Some(r), 0)$ is \\
\oox equivalent to $\mathit{heap}(D', f, [w|r:r])$ \\
\ooo Hence $G(D', f, [w|r:r], \phi \otimes \elt{r}{r})$ \\
\ooo $r]$ \\
\ooo Hence $a.\; G(D', f, [w|a:a], \phi \otimes \elt{a}{a})$ \\
\ooo Hence $a.\; H(\phi \otimes \elt{a}{a})$ 
\end{tabbedproof}
\end{proof}

\subsection{Proof of $\mathsf{union}$}

\begin{lemma}
If $f^+$ is a strict partial order, $(u, v) \in f^+$ and $v$ maximal,
and $(x, y) \in f^+$ and $y$ maximal, and $v \not= y$, then it follows that for $g = [f|v:y]$,

\begin{enumerate}
\item $\domain{g} = \domain{f} \uplus \setof{v}$
\item $g^+$ is a strict partial order 
\item If $(a,b) \in f^*$ with $b$ maximal, either $b \not= v$ and $(a,b) \in g^*$ with $b$ maximal,
  or $b = v$ and $(a,y) \in g^*$ with $y$ maximal. 
\end{enumerate}
\end{lemma}

\begin{proof}
\begin{enumerate}
\item Since $v$ is maximal in $f$, it follows that $v \not\in \domain{f}$, and hence $\domain{[f|v:y]} = \domain{f} \uplus \setof{v}$

\item To be a strict partial order, it suffices that there is no $(z,z) \in g^+$. 
\begin{tabbedproof}
\oo Assume $(z, z) \in g^+$ \\
\oo Then there is a $k \geq 1$ such that $g^k(z) = z$ \\
\oo Now, we'll show that $v$ does not occur in the sequence $z, g(z), \ldots, g^k(z)$ \\
\oo To do this, we'll first show that $y$ does not occur in the sequence $z, g(z), \ldots, g^k(z)$ \\
\oo Since $y$ is maximal for $f$ and $v \not= y$, we know $y$ is maximal for $g = [f|v:y]$ \\
\oo As a result, $y$ can only occur as the last element $g^k(z)$ \\
\oo But since $g^k(z) = z$ and $z \in \domain{g}$, we know that $y \not= v$, \\
\ox and so it cannot occur as the last element, either. \\
\oo As a result, $v$ cannot occur at any $i < k$, since then $g^{i+1}(z) = y$, \\
\oo and we know this cannot happen \\
\oo We also know $v$ cannot occur at $i = k$, since $g^k(z) = z = g^0(z)$, and \\
\oo we know $v$ cannot occur at $i = 0$ \\
\oo Therefore for $i \leq k$, we have $g^i(z) \not= v$.  \\
\oo Hence we have $g^i(z) = f^i(z)$ \\
\oo Therefore $z, f(z), \ldots, f^k(z) = z$ shows that $(z, z) \in f^+$ \\
\oo But $f^+$ is a strict partial order, which is a contradiction. \\
\end{tabbedproof}

\item If $(a,b) \in f^*$ with $b$ maximal, either $b \not= v$ and $(a,b) \in g^*$ with $b$ maximal,
  or $b = v$ and $(a,y) \in g^*$ with $y$ maximal.
\begin{tabbedproof}
\oo Assume $(a,b) \in f^*$ with $b$ maximal. \\
\ooo Then either $a = b$, or $(a,b) \in f^+$ \\
\ooo Suppose $a = b$: \\
\oooo Then either $b = v$ or not \\
\oooo Suppose $b = v$: \\
\ooooo Then $(a,y) = (v,y) \in g \subseteq g^*$, and  \\
\ooooo $y$ is maximal since $y \not\in \domain{g} = \domain{f} \cup \setof{v}$ \\
\oooo Suppose $b \not= v$: \\
\ooooo Then $(a,y) = (b, b) \in g^*$, since $b$ is maximal since $b \not\in \domain{g} = \domain{f} \cup \setof{v}$ \\ 
\ooo Suppose $(a,b) \in f^+$: \\
\oooo Then we have a $k > 0$ such that $f^k(a) = b$ \\
\oooo Now, either there is an $i \leq k$ such that $f^i(a) = v$, or not. \\
\oooo Suppose $f^i(a) = v$: \\
\ooooo Since $v$ is maximal, it is the last element, so $b = v$ \\
\ooooo Then we know that for all $j < i$, $f^j(a) \not= v$, since $v$ is maximal in $f$ \\
\ooooo Therefore for all $j \leq i$, $f^j(a) = g^j(a)$ \\
\ooooo Therefore $g^{i+1}(a) = y$ \\
\ooooo Hence $(a,y) \in g^*$ and $y$ is maximal, since $y \not\in \domain{g}$ \\
\oooo Suppose there is no $i$ such that $f^i(a) = v$: \\
\ooooo Then we know that for all $j \leq k$, $f^j(a) = g^j(a)$, and $b \not= v$ \\
\ooooo Hence $(a,b) \in g^*$ and $b$ is maximal since $b \not\in \domain{f} \cup \setof{v} = \domain{g}$\\
\end{tabbedproof}
\end{enumerate}
\end{proof}

\begin{lemma}{(Ramification)}
Suppose $D, f, w \models \phi$ and $(x,y) \in f^*$ with $y$ maximal and $(u, v) \in f^*$ with $v$ maximal,
and $y \not= v$, and $w$ injective. Let $w'$ be the restriction of $[w|v:z]$ to exclude $y$, and let $g = [f|y:v]$. 

Then, $D, g,  w' \models R([z/w(v), z/w(y)], \phi)$.
\end{lemma}

\begin{proof}
This proof follows by induction on the structure of $\phi$. 

\begin{tabbedproof}
\oo Case $\phi = I$: \\
\ooo This case is immediate since $D, g, w' \models I$  and $R([z/w(v), z/w(y)],I) = I$ \\
\oo Case $\phi = \psi \otimes \theta$: \\
\ooo By satisfaction of $\phi$, we have $D_1, D_2$ such that $D = D_1 \uplus D_2$ and \\
\ooox $D_1, f, w \models \psi$ and\\
\ooox $D_2, f, w \models \theta$ \\
\ooo By induction, we have $D_1, g, w' \models R([z/w(v), z/w(y)], \psi)$ \\ 
\ooo By induction, we have $D_2, g, w' \models R([z/w(v), z/w(y)], \theta)$ \\ 
\ooo By the definition of satisfaction, $D, g, w' \models  R([z/w(v), z/w(y)], \psi)\otimes R([w(v)/w(y)], \theta)$ \\ 
\ooo By the definition of $R$, we have $D, g, w' \models  R([z/w(v), z/w(y)], \phi)$  \\
\oo Case $\phi = \elt{a}{b}$: \\
\ooo We know that $a \in D$, $(a,c) \in f^*$ with $c$ maximal, and $w(c) = b$ with $v$ maximal \\
\ooo Therefore either $c \not= y$ and $(a,c) \in g^*$ with $c$ maximal or $c = y$ and $(a, v) \in g^*$ \\
\ooo Suppose $c \not= y$: \\
\oooo Since $w$ is injective, we know that $w(c) \not= w(y)$. Hence $[z/w(y)]w(c) = w(c)$ \\
\oooo Consider whether $c$ is $v$.  \\
\oooo If $c \not= v$:  \\
\ooooo Since $w$ is injective, we know that $w(c) \not= w(v)$. Hence $[z/w(v)]w(c) = w(c)$ \\
\ooooo Hence $[z/w(v), z/w(y)]w(c) = w(c)$ \\
\ooooo So $\elt{a}{b} = R([z/w(v),z/w(y)], \elt{a}{b})$ \\
\ooooo Furthermore $w'(c) = w(c)$ \\
\ooooo Hence $D, g, w' \models \elt{a}{b}$ \\
\ooooo Hence $D, g, w' \models R([z/w(v), z/w(y)], \elt{a}{b})$ \\
\oooo If $c = v$: \\
\ooooo Since $w(c) = w(v)$, we have $[z/w(v)]w(c) = z$ \\
\ooooo Hence $[z/w(v), z/w(y)]w(c) = z$ \\
\ooooo Note $w'(v) = z$ \\
\ooooo Hence $D, g, w' \models \elt{a}{z}$ \\
\ooooo Hence $D, g, w' \models R([z/w(v), z/w(y)], \elt{a}{b})$ \\
\ooo Suppose $c = y$: \\
\oooo Then, $b = w(c) = w(y)$, so that $[z/w(v), z/w(y)]b = z$\\ 
\oooo So $R([z/w(v), z/w(y)], \elt{a}{b}) = \elt{a}{z}$ \\
\oooo Furthermore, we know that $(a, v) \in g^*$ and $v$ maximal and $w'(v) = z$ \\
\oooo So $D, g, w' \models \elt{a}{z}$ \\
\oooo So $D, g, w' \models R([z/w(v),z/w(y)], \elt{a}{b})$ 
\end{tabbedproof}
\end{proof}

\begin{lemma}{(Correctness of $\union$)}
The $\union$ function meets the specification in Figure~\ref{union-find:spec}. 
\end{lemma}

\begin{proof}
\begin{tabbedproof}
\oo Assume we have a precondition $H(\phi)$ and $\ufcontains{\phi}{x}{y}$ and $\ufcontains{\phi}{u}{v}$\\
\ooo Then we have $D$, $f$, $w$, such that $G(D, f, w, \phi)$ \\
\ooo Furthermore $D, f, w \models \elt{x}{y}$ \\
\ooo Now consider the body of $\union(x,u)$ \\
\ooo $[\letv{(r,y',m)}{\findroot(x)}{}$ \\
\ooo So we have $f'$ such that $G(D, f', w, \phi)$ and \\
\ooox  $(x, r) \in f'^*$ and $r$ maximal in $f'$ \\
\ooox  $f$ and $f'$ have the same maximal relationships and domain \\
\ooox  $y = w(r)$ \\
\ooox  $y = y'$ \\
\ooo Furthermore $D, f', w \models \elt{u}{v}$, so \\
\ooo $\letv{(s,v',n)}{\findroot(u)}{}$ \\
\ooo So we have $f''$ such that $G(D, f'', w, \phi)$ and \\
\ooox  $(u, s) \in f''^*$ and $s$ maximal in $f''$ \\
\ooox  $f'$ and $f''$ have the same maximal relationships and domain \\
\ooox  $v = w(s)$ \\
\ooox  $v = v'$ \\
\ooo So we can substitute $y$ for $y'$ and $v$ for $v'$ \\
\ooo Since $f$ and $f'$ have the same maximal relationships, and \\
\oox since $f'$ and $f''$ have the same maximal relationships,  \\
\oox we know that $f$ and $f''$ have the same maximal relationships \\
\ooo Since $(x,r) \in f'^*$ and $r$ maximal in $f'$, we know $(x, r) \in f''^*$ and $r$ maximal in $f''$ \\
\ooo Now case analyze on whether $r = s$: \\
\ooo If $r = s$: \\
\oooo Then we can simplify the remaining program to $\unit$ \\
\oooo Now, note that since $r = s$, $w(r) = w(s)$, and so $y = v$ and $R([y/v], \phi) = \phi$ \\
\oooo Hence we can hide $D, f'', w$ to get $H(\phi)$ \\
\ooo If $r \not= s$: \\
\oooo Now, case analyze on the ranks $m$ and $n$: \\
\oooo If $m < n$: \\
\ooooo We can simplify the if-then-else, and continue \\
\ooooo $\letv{\unit}{[s := \Root(y, n)]}{}$ \\
\ooooo $r := \Child(s)$ \\
\ooooo Now take $w'$ to be the restriction of $[w|s:y]$ to exclude $r$ \\
\ooooo Now take $g = [f''|r:s]$ \\
\ooooo Since $w(r) = y$, and $w$ is injective, $w'$ is still injective \\
\ooooo From the ramification lemma, we know $D, g, w' \models R([y/y, y/v], \phi)$ \\
\ooooo This is the same as $R([y/v], \phi)$ \\
\ooooo The two updates ensure that $\mathit{heap}(D, g, w')$ hold \\
\ooooo Hiding $D, g, w'$, we get $H(R([y/v], \phi)$ \\
\oooo If $n < m$: \\
\ooooo We can simplify the if-then-else, and continue \\
\ooooo $s := \Child(r)$ \\
\ooooo Now take $w'$ to be the restriction of $[w|r:y]$ to exclude $s$ \\
\ooooo Now take $g = [f''|s:r]$ \\
\ooooo By the ramification lemma, $D, g, w' \models R([y/y, y/v], \phi)$ \\
\ooooo This is the same as $R([y/v], \phi)$ \\
\ooooo The update ensures that $\mathit{heap}(D, g, w')$ holds \\
\ooooo Hiding $D, g, w'$, we get $H(R([y/v], \phi)$ \\
\oooo If $m = n$: \\
\ooooo $\letv{\unit}{[r := \Root(y, m+1)]}{}$ \\
\ooooo $s := \Child(r)$ \\
\ooooo Now take $w'$ to be the restriction of $[w|r:y]$ to exclude $s$ \\
\ooooo Now take $g = [f''|s:r]$ \\
\ooooo By the ramification lemma, $D, g, w' \models R([y/y, y/v], \phi)$ \\
\ooooo This is the same as $R([y/v], \phi)$ \\
\ooooo The update ensures that $\mathit{heap}(D, g, w')$ holds \\
\ooooo Hiding $D, g, w'$, we get $H(R([y/v], \phi)$ \\

\end{tabbedproof}
\end{proof}

\paragraph{Acknowledgements}

I would like to thank Peter O'Hearn for pointing out the connection
of our work with the ramification problem of AI.


% \begin{lemma}{(Existence of Upper Fixed Points)}
% Suppose $A$ is a finite set, and $f : A \to A$ is a function whose
% transitive closure is a partial order, and let $\mu(f)$ be the
% set of fixed points of $f$. Then, there is a function 
% $\mathit{root}_f : A \to \mu(f)$ such that for each $x$ there is a 
% $k$ such that $f^{k} = \mathit{root}_f(x)$.
% \end{lemma}
% 
% \begin{proof}
% Observe that the transitive closure gives an order $\sqsubseteq$ such
% that for each $i \leq j$ and $x$ in $A$, $f^i(x) \sqsubseteq
% f^j(x)$. Since $A$ is finite, it follows that the size of the longest
% chain of distinct elements is at most $|A|$ -- so any sequence longer
% than that must repeat elements of $A$.
% 
% Suppose there is a repeated element, occuring at the $i$-th and
% $j$-th iterations of applying $f$ to $x$, where $i < j$. Since the
% transitive closure is a partial order, we know that if $a \sqsubseteq
% b$ and $a \sqsubseteq b$, then $a = b$. Therefore every element 
% of the sequence from $i$ to $j$ is equal to $f^i(x)$. Since $i$ is
% strictly less than $j$, this means that $f(f^i(x)) = f^i(x)$, and 
% is hence a fixed point of $f$. 
% \end{proof}
% 
% This property justifies the recursion in the implementation of
% $\find$: it says that the root of a node's parent is still the root of
% the node. It is worth contrasting the style of specification here with
% the usual inductive specification of child-pointing trees in
% separation logic. When proceeding from parent to child, we use that
% fact that list or tree predicates are inductively defined, and try to
% work directly with the inductive structure of the predicate.  Here, we
% specify the invariant on the whole heap as a relation property of the
% heap, and then have to prove that the measures we use are suitably
% well-founded.
% 
% 
% However, $\find$ also updates parents as it goes, and so at each
% iteration it works on a \emph{different} global order structure.
% We need to ensure that the answers we return are preserved under
% this update, which is what the following lemma will ensure. 
% 
% \begin{lemma}{(Fixed Point Update)}
% Suppose $A$ is a finite set, and $f : A \to A$ is a function whose
% transitive closure is a partial order, and let $\mu(f)$ be the set of
% fixed points of $f$. For a given $z$ in $A$, if we define $f' =
% [f|z:\mathit{root}_f(z)]$, then the transitive closure of $f'$ is also
% a partial order and $\mathit{root}_f = \mathit{root}_{f'}$.
% \end{lemma}
% 
% \begin{proof}
% Suppose that we have an arbitrary $x \in A$. Now, consider the sets of
% elements $U = \comprehend{y}{y \sqsupseteq_f x}$, and $U' =
% \comprehend{y}{y \sqsupseteq_{f'} x}$. (Note that $\sqsupseteq_{f'}$
% is a well-defined relation, but that we do not yet know that it is a
% partial order --- we only know that it defines a preorder.)
% 
% First, observe that $U$ is totally ordered with respect to
% $\sqsubseteq_f$. We know that every $u \in U$ is equal to $u
% = f^k(x)$ for some $k$, and so each pair of $u$ and $v$ is $f^j(x)$
% and $f^k(x)$ for some $j$ and $k$. Since the natural numbers are
% totally ordered, and since $j \leq k$ implies that $f^j(x)
% \sqsubseteq_f f^k(x)$, it follows that $U$ is totally ordered with
% respect to $\sqsubseteq_f$.
% 
% Now, $U$ either contains $z$ or not.
% 
% Suppose it does not contain $z$. Then $U = \comprehend{y}{y
%   \sqsupseteq_{f'} x} = U'$, since $f = f'$ for all arguments not
% equal to $z$. Therefore, $\mathit{root}_{f'}(x) = \mathit{root}_f(x)$.
% 
% On the other hand, suppose $U$ does contain $z$. Then we know that $U$
% contains $\mathit{root}_f(z)$, and that this is an upper bound of
% every element of $U$. 
% 
% Each distinct element of $U$ is $f^i(x)$ for some $0 \leq i <
% |U|$. Therefore, we know that $z = f^k(x)$ for some $k$. Furthermore,
% for each $j < k$, we know that $z \not= f^j(x)$. Therefore, these
% elements are the $j$ smallest elements for both $U$ and $U'$. However,
% we know that $f'(z) = \mathit{root}_f(z)$, and so at this point the
% ordered enumeration of the elements of $U'$ ends.
% 
% Therefore, $U'$ is a subset of $U$ , whose maximum element (relative
% to $f'$) is the same as the maximum element of $U$ --- in both cases,
% it is $\mathit{root}_f(x)$. Since $U'$ is a subset of $U$, it further
% follows that the transitive closure of $f'$ remains a partial
% order: there can be no cycles since we have only removed paths from
% the graph, and there were no cycles to begin with.
% \end{proof}
% 
% Besides $\find$, our API also includes the $\union$ and $\newset$ operations. 
% To create new disjoint sets, we need to that the order structure of two distinct
% set forests can be merged sensibly. 
% 
% \begin{lemma}{(Order Extension)}
% Suppose $A$ and $B$ are disjoint finite sets, and $f : A \to A$ and $g
% : B \to B$ are functions whose transitive closures are partial
% orders. Let $\mu(f)$ and $\mu(g)$ be the set of fixed points of $f$
% and $g$, respectively. 
% 
% Then $h : A \cup B \to A \cup B = f \cup g$ is also a function whose 
% transitive closure is a partial order, with $\mu(h) = \mu(f) \cup \mu(g)$, 
% and $\mathit{root}_h = \mathit{root}_f \cup \mathit{root}_g$. 
% \end{lemma}
% 
% 
% \begin{proof}
% \begin{enumerate}
% \item We need to show that the transitive closure of $h$ is a partial
% order. To do this, we need to show that the antisymmetry axiom holds, 
% since reflexivity and transitivity arise immediately from taking the
% transitive closure. 
% 
% So, suppose that $(x,y) \in h^{*}$, and likewise $(y, x) \in
% h^{*}$. From the definition of the transitive closure, we know there
% is some $m$ and $n$ such that $y = h^m(x)$, and $x = h^n{y}$.  Since
% $A$ and $B$ are disjoint, we know that either $x \in A$ or $x \in
% B$. Furthermore, from the definition of $h$ we know that $h$ takes
% elements of $A$ to elements of $A$ according to $f$, and takes
% elements of $B$ to elements of $B$ according to $g$. So if $x \in A$,
% then $y = f^m(x)$, and so $y \in A$, and so $x = f^n(y)$.  Then by the
% antisymmetry of $f^{*}$, we know that $x = y$. So if $x \in B$, then
% $y = g^m(x)$, and so $y \in B$, and so $x = g^n(y)$.  Then by the
% antisymmetry of $g^{*}$, we know that $x = y$.
% 
% \item Next, we need to show that the set of fixed points of $h$ is the
%   union of the fixed points of $f$ and $g$. This follows immediately
%   from the fact that $A$ and $B$ are disjoint, and $f$ and $g$ are
%   endofunctions.
% 
% \item Finally, we need to show that $\mathit{root}_h = \mathit{root}_f
%   \cup \mathit{root}_g$. Again, since $A$ and $B$ are disjoint, there
%   are no increasing chains of elements of $A \cup B$ that include
%   elements of both $A$ and $B$.  Therefore, the upper fixed points of
%   $h$ on its $A$ sub-domain will be given by $\mathit{root}_f$ and the
%   upper fixed points of $h$ on its $B$ sub-domain will be given by
%   $\mathit{root}_g$.
% \end{enumerate}
% \end{proof}
% 
% Finally, we need a lemma to let us prove that when we take the union
% of two sets, we will again have a graph which has the desired properties.
% 
% \begin{lemma}{(Abstract Union)}
% Suppose $A$ is a finite set, and $f : A \to A$ is a function whose
% transitive closure is a partial order, and let $\mu(f)$ be the set of
% fixed points of $f$. For any $y, z$ in $\mu(A)$, if we define $f' =
% [f|y:z]$, then the transitive closure of $f'$ is a partial order, and 
% $\mathit{root}_{f'} = \semfun{a}{(\IfTE{\mathit{root}_f(a)=y}{z}{\mathit{root}_f(a)})}$.
% \end{lemma}
% 
% \begin{proof}
% Suppose that $x \in A$. Now, consider the sets of elements $U =
% \comprehend{y}{y \sqsupseteq_f x}$, and $U' = \comprehend{y}{y
%   \sqsupseteq_{f'} x}$. We know that $U$ can be ordered into elements
% $x, \ldots, f^k(x)$, with $k = |U|$ and with $f^k(x) = \mu(x)$. 
% 
% Now, consider whether $y \in A$ or not. If it is not, then we know
% that $f'^{i}(x) = f^{i}(x)$ for all $i \leq k$. Therefore $U'$ is
% equal to this set, and is also totally ordered and has the maximal
% element $\mathit{root}_f(x)$. 
% 
% On the other hand, if $y$ is in $A$, then we know that $f'^i(x) =
% f^i(x)$ until index $k$, at which point $f'^k(x) = z$. However, we can
% easily see that $f'^{k+1}(x) = f'(f'^k(x)) = z$, which means that $U'$
% has a unique totally order with respect to $\sqsubseteq_{f'}$, and has
% maximum $z$. 
% 
% Furthermore, since each set of upper bounds of $x$ is uniquely totally
% ordered by $\sqsubseteq_{f'}$, we know that the transitive closure of
% $f'$ is a partial order, and not just a preorder. 
% \end{proof}
% 
% 
% \begin{lemma}{(Monotonicity of $\ufcontains{\phi}{x}{y}$)}
% Suppose $\ufcontains{\phi}{x}{y}$. Then $\ufcontains{\phi \otimes \psi}{x}{y}$. 
% \end{lemma}
% 
% \begin{proof}
% Take the derivation of $\ufcontains{\phi}{x}{y}$. Apply the tensor rule 
% to it, and derive $\ufcontains{\phi \otimes \psi}{x}{y}$.
% \end{proof}
% 
% \begin{lemma}
%   Suppose $\ufcontains{\phi}{x}{y}$. Then for all $\rho$, $\ufcontains{R(\rho, \phi)}{x}{\rho(y)}$. 
% \end{lemma}
% 
% \begin{proof}
% This follows by induction on the derivation of $\ufcontains{\phi}{x}{y}$. 
% \begin{itemize}
% \item Suppose $\ufcontains{\elt{x}{y}}{x}{y}$. 
% 
% In this case $R(\rho, \elt{x}{y}) = \elt{x}{\rho(y)}$. Then apply the 
% unit rule to derive $\ufcontains{R(\rho, \phi)}{x}{\rho(y)}$. 
% 
% \item Suppose that we use the tensor-left rule, so that by inversion we have
% $\phi = \psi \otimes \theta$, and $\ufcontains{\psi}{x}{y}$. 
% 
% Then, by induction we have $\ufcontains{R(\rho, \psi)}{x}{\rho(y)}$. 
% By the tensor-left rule, we have $\ufcontains{R(\rho, \psi) \otimes R(\rho, \theta)}{x}{\rho(y)}$. Then, by the definition of $R$, we have $\ufcontains{R(\rho, \psi \otimes \theta)}{x}{\rho(y)}$.
% 
% \item The tensor-right case is symmetric. 
% \end{itemize}
% \end{proof}
% 
% \begin{lemma}{(Model Extension)}
%   Suppose $(n, p, w) \models \phi$, and $n'$ is disjoint from $n$, and $p' \in n' \to n'$ such that $p'^*$ is a partial order, and $w' \in n' \to \mu(n')$. Then $(n, p \cup p', w \cup w') \models \phi$. 
% \end{lemma}
% 
% \begin{proof}
%  We proceed by induction on $\phi$. 
% 
%  \begin{itemize}
%    \item Case $\phi = I$
% 
%      This case is immediate. 
% 
%    \item Case $\phi = \elt{x}{y}$. 
% 
%      In this case, we know that $p'$ is completely disjoint from
%      $p$. As a result, the transitive closure of $p \cup p'$ is just
%      the unions of the transitive closures of $p$ and $p'$, and any
%      upper fixed point of an element of $n \cup n'$ is just the upper
%      fixed point of $p$ or of $p'$. Therefore $\mathit{root}(x)$
%      remains unchanged, and since $w'$ is an extension of $w$,
%      $w'(\mathit{root}(x)) = y$.
%      
%      \item $\phi = \psi \otimes \theta$
% 
%        This follows by induction. Suppose that $(n_1, p, w) \models
%        \psi$ and $(n_2, p, w) \models \theta$. Then, by induction
%        hypothesis, we know that $(n_1, p \cup p', w \cup w') \models
%        \psi$ and $(n_2, p \cup p', w \cup w') \models \theta$. Then,
%        by the definition of satisfaction, we know that $(n, p \cup p',
%        w \cup w') \models \psi \otimes \theta$.
%  \end{itemize}
% \end{proof}
% 
% \begin{lemma}{(Shrinking)}
% For suitably-typed $n, p, w$, suppose $(n, p, w) \models \phi \otimes \psi$. Then 
% we know that $(n, p, w) \models \phi$ and $(n, p, w) \models \psi$. 
% \end{lemma}
% \begin{proof}
%   From the definition of the model, we know that there are smaller $n_1$ and $n_2$ 
% such that $(n_1, p, w) \models \phi$ and $(n_2, p, w) \models \psi$. Then, by 
% the model extension lemma, we know that $(n, p, w) \models \phi$ and $(n, p, w) \models \psi$. 
% \end{proof}
% 
% \begin{lemma}{(Soundness of Witness Values)}
% For suitably-typed $n, p, w$, suppose $(n, p, w) \models \phi$, and that $\ufcontains{\phi}{x}{y}$. Then $w(\mathit{root}(x)) = y$.  
% \end{lemma}
% \begin{proof}
% This follows by induction on the derivation of
% $\ufcontains{\phi}{x}{y}$, with the two inductive cases following
% trivially using shrinking. The only interesting case is the base case,
% when $\ufcontains{\elt{x}{y}}{x}{y}$.  In this case, we know that $(n,
% p, w) \models \elt{x}{y}$, so we know $w(\mathit{root}(x)) = y$. 
% 
% 
% \end{proof}
% 
% \section{Correctness Proofs}
% 
% \subsection{Proof of $\newset$}
% 
% \begin{specification}
% \nextline $\{H(\phi)\}$
% \nextline $\{G(n, p, w, \phi)\}$
% \nextline $\letv{r}{[\newref{\sigma}{\Root(\None, 0)}]}{}$
% \nextline $\{G(n, p, w, \phi) * r \pointsto \Root(\None, 0)\}$
% \nextline $\letv{\unit}{[r := \Root(\Some(r), 0)]}{}$
% \nextline $\{G(n, p, w, \phi) * r \pointsto \Root(\Some(r), 0)\}$
% \nextline $r]$
% \nextline $\{G(n, p, w, \phi) * a \pointsto \Root(\Some(a), 0)\}$
% \nextline $\{p^{*} \mbox{ partial order} \land (n, p, w) \models \phi \land
%              \heap(n, p, w) * a \pointsto \Root(\Some(a), 0)\}$
% \nextline $\{p^{*} \mbox{ partial order} \land (n, p, w) \models \phi \land
%              \heap(n, p, w) * a \pointsto \Root(\Some(a), 0) \land a \not\in n\}$
% \nextline Let $n' = n \cup \setof{a}$, and let $p' = [p|a:a]$, and let $w' = [w|a:a]$. 
% \nextline By the order extension lemma,
% \nextline $\{p'^{*} \mbox{ partial order} \land (n, p', w') \models \phi \land
%              \heap(n', p', w')\}$
% \nextline By definition of modelling, $(\setof{a}, p', w') \models \elt{a}{a}$
% \nextline So $(n', p', w') \models \phi \otimes \elt{a}{a}$. Therefore
% \nextline $\{p'^{*} \mbox{ partial order} \land (n', p', w') \models \phi \otimes \elt{a}{a} \land \heap(n', p', w')\}$
% \nextline $\{G(n', p', w', \phi \otimes \elt{a}{a})\}$
% \nextline $\{H(\phi \otimes \elt{a}{a})\}$
% \end{specification}
% 
% \subsection{Proof of $\find$}
% 
% In order to prove this function correct, we will need to appeal to the
% correctness of the $\findroot$, function, which we will specifiy as
% follows:
% 
% \begin{prop}{(Specification of $\findroot$)}
%   The specification of the $\findroot$ function is 
% 
%   \begin{specification}
%     \nextline $\{G(n, p, w, \phi) \land \ufcontains{\phi}{x}{y}\}$
%     \nextline $\findroot(x)$ 
%     \nextline $\{(u, y', n).\; \exists p'.\;G(n, p', w, \phi) \;\land
%                        u = \mathit{root}(x) \land y' = \Some(w(u)) 
%                        \land p'(x) = u\}$
%   \end{specification}
% \end{prop}
% 
% \begin{proof}
% Assume we begin in the precondition $\ufcontains{\phi}{x}{y} \land
% G(n, p, w, \phi)$.  By definition, we know that $G(n, p, w, \phi) =
% p^{*}\mbox{ partial order} \land (n,p,w) \models \phi \land
% \heap(n,p,w)$, by the soundness of the models relation, we
% additionally know that $w(\mathit{root}_p(w(x))) = y$.
% 
% 
% \end{proof}
% 
% 
% 
% \begin{specification}
% \nextline $\{\ufcontains{\phi}{x}{y} \land G(n, p, w, \phi)\}$
% \nextline $\{p^{*} \mbox{ partial order} \land (n, p, w) \models \phi \land
%              \heap(n, p, w) \land \ufcontains{\phi}{x}{y}\}$
% \nextline From the soundness of the models relationship, we know 
% \nextline $\{p^{*} \mbox{ partial order} \land (n, p, w) \models \phi \land
%              \heap(n, p, w) \land x \in n \land w(\mathit{root}(x)) = y\}$
% \nextline $\{p^{*} \mbox{ partial order} \land (n, p, w) \models \phi \land
%              x \in n \land w(\mathit{root}(x)) = y $ 
% \nextline $\;\;\land\; \heap(n - \setof{x}, p, w) * \IfTE{x = p(x)}{\exists k.\; x \pointsto \Root(\Some(w(x)), k)}{x \pointsto \Child(p(x))}\}$
% 
% \nextline $\{p^{*} \mbox{ partial order} \land (n, p, w) \models \phi \land
%              x \in n \land w(\mathit{root}(x)) = y $ 
% \nextline $\;\;\land\; \heap(n - \setof{x}, p, w) *  x \pointsto (\IfTE{x = p(x)}{\Root(\Some(w(x)), k)}{\Child(p(x))})\}$
% 
% \nextline $[$\=$\letv{v}{[!x]}{}$ 
% \nextline $\{p^{*} \mbox{ partial order} \land (n, p, w) \models \phi \land
%              x \in n \land w(\mathit{root}(x)) = y $ 
% \nextline $\;\;\land\; \heap(n, p, w) \land v = (\IfTE{x = p(x)}{\Root(\Some(w(x)), k)}{\Child(p(x))})\}$
% \nextline\> $\run{}\ctext{case}(v,$\=
% \nextline\> \> $\Root(z, k) \to$
% 
% \nextline\> \> $\{p^{*} \mbox{ partial order} \land (n, p, w) \models \phi \land
%              x \in n \land w(\mathit{root}(x)) = y $ 
% \nextline\> \> $\;\;\land\; \heap(n, p, w) \land v = \Root(z, k) \land x = p(x) \land z = \Some(w(x))\}$
% 
% \nextline\> \> $\{p^{*} \mbox{ partial order} \land (n, p, w) \models \phi \land
%              x \in n \land w(\mathit{root}(x)) = y $ 
% \nextline\> \> $\;\;\land\; \heap(n, p, w) x = p(x) \land \mathit{root}(x) = x \land z = \Some(w(x))\}$
% 
% \nextline\> \> $\{p^{*} \mbox{ partial order} \land (n, p, w) \models \phi \land
%              x \in n \land w(x) = y $ 
% \nextline\> \> $\;\;\land\; \heap(n, p, w) \land x = p(x) \land \mathit{root}(x) = x \land z = \Some(w(x))\}$
% 
% \nextline\> \> $\{p^{*} \mbox{ partial order} \land (n, p, w) \models \phi \land
%              x \in n $
% \nextline\> \> $\;\;\land\; \heap(n, p, w) \land x = p(x) \land \mathit{root}(x) = x \land z = \Some(y)\}$
% \nextline \> \> $[(w, x, k)]$
% 
% \nextline\> \> $\{a.\;p^{*} \mbox{ partial order} \land (n, p, w) \models \phi \land
%              x \in n $
% \nextline\> \> $\;\;\land\; \heap(n, p, w) \land x = p(x) \land \mathit{root}(x) = x \land z = \Some(y) land a = (x, \Some(y), k)\}$
% 
% \nextline\> \> $\{a.\;G(n, p, w) \land a = (x, \Some(y), k)\}$
% \nextline\> \> $\{a.\;\exists n, p, w.\; G(n, p, w) \land land a = (x, \Some(y), k)\}$
% 
% \nextline\> \> $\Child(m) \to$
% \nextline\> \> $\{p^{*} \mbox{ partial order} \land (n, p, w) \models \phi \land
%              x \in n \land w(\mathit{root}(x)) = y $ 
% \nextline\> \> $\;\;\land\; \heap(n, p, w) \land v = \Child(p(x)) \land m = p(x) \}$
% 
% \nextline\> \> $\{p^{*} \mbox{ partial order} \land (n, p, w) \models \phi\bullet\elt{p(x)}{y} \land
%              x \in n \land w(\mathit{root}(x)) = y $ 
% \nextline\> \> $\;\;\land\; \heap(n, p, w) \land v = \Child(p(x)) \land m = p(x) \}$
% 
% 
% \nextline\> \> $[$\=$\letv{(u, y, k)}{\ctext{findroot}(m)}{}$
% 
% \nextline\> \> $\{p'^{*} \mbox{ partial order} \land (n, p', w) \models \phi\bullet\elt{p(x)}{y} \land
%              x \in n \land w(\mathit{root}(x)) = y $ 
% \nextline\> \> $\;\;\land\; \heap(n, p, w) \land v = \Child(p(x)) \land u = \mathit{root}(p(x)) \}$
% 
% 
% 
% \end{specification}
% 
% 
% 
% 
% This specification asserts that calling $\findroot$ will give us a new
% heap which (a) models the same formula as before, and (b) will return
% the upper fixed point of $x$, as well as its witness. If the witness
% is the root itself, we return $\None$, and otherwise return the
% witness wrapped in a $\Some$. (We also return the height, whose value
% we are not tracking in our invariant.)
% 
% The key difficulty in this proof is that $\findroot$ implements
% \emph{path compression} --- each recursive call modifies the heap in
% nonlocal ways. This nonlocality is evident in the fact that the parent
% and witness functions $p$ changes from the pre- to the post-condition.
% 
% The correctness of this program relies on the path compression lemma
% we proved earlier. 
% 
% 
% 
% 
% 
% \begin{lemma}{(External Membership and Modelling)}
% Suppose $n \subseteq \tau, p \in n \to n, w \in \mu(n) \to n$. Furthermore,
% suppose $(p, n, w) \models \phi$, and $\ufcontains{\phi}{x}{y}$. 
% Then $(n, p, w) \models \elt{x}{y}$. 
% \end{lemma}
% 
% \begin{proof}
% We proceed by induction on $\phi$: 
% \begin{itemize}
%   \item Case $\phi = \elt{x'}{y'}$
%     By inversion on $\ufcontains{\phi}{x}{y}$, we know that the only 
%     rule which can apply is the element rule, and so $x = x'$ and $y = y'$. 
% 
%     Then by hypothesis we know that $(n, p, w) \models \elt{x}{y}$. 
% 
%   \item Case $\phi = I$
% 
%     This case is impossible, since there is no rule from which we can
%     conclude $\ufcontains{I}{x}{y}$. 
% 
%   \item Case $\phi = \psi \otimes \theta$. 
% 
%     By inversion, we know that either $\ufcontains{\psi}{x}{y}$, or 
%     that $\ufcontains{\theta}{x}{y}$. 
% 
%     Then, since $(n, p, w) \models \psi \otimes \theta$, we know 
%     that $(n_1, p, w) \models \psi$ and $(n_2, p, w) \models \theta$,
%     where $n = n_1 \uplus n_2$. Without loss of generality, 
%     suppose that $\ufcontains{\psi}{x}{y}$. Then by induction we know
%     that $(n_1, p, w) \models \elt{x}{y}$, and then by monotonicity 
%     we know that $(n, p, w) \models \elt{x}{y}$. 
% \end{itemize}
% \end{proof}
% 
% \begin{lemma}{(Relating Membership and Heaps)}
% Suppose $n = n_1 \uplus n_2$. Then $\heap(n, p, w) \iff \heap(n_1, p, w) * \heap(n_2, p, w)$. 
% \end{lemma}
% 
% \begin{proof}
%   \begin{specification}
%     \nextline By definition, $\heap(n, p, w) =  \forall^{*} l \in n. F(l)$, where
%     \nextline  $F(l) = \IfTE{p(l) = l}{\exists n.\; l \pointsto \Root(\IfTE{w(x) = x}{\None}{\Some(w(x))}, n)}{l \pointsto \Child(p(l))}$.
%     \nextline So $\heap(n, p, w) =  \forall^{*} l \in n_1 \uplus n_2. F(l)$. 
%     \nextline So $\heap(n, p, w) =  \forall^{*} l \in n_1 F(l) * \forall^{*} l \in n_2 F(l)$. 
%     \nextline So $\heap(n, p, w) = \heap(n_1, p, w) * \heap(n_2, p, w)$. 
%   \end{specification}
% \end{proof}
% 
% \begin{specification}
% \nextline $\setof{H(\phi) \land \ufcontains{\phi}{x}{y}}$
% \nextline Since $\ufcontains{\phi}{x}{y}$ is pure, we assume it outside
%   the precondition. 
% \nextline 
%   $\{$\=$\exists n \subseteq \tau, p \in n \to n, w \in \mu(n) \to n.\;
%       p^{*} \mbox{ partial order} \land 
%              (n, p, w) \models \phi \;\land \heap(n, p, w) \}$
% \nextline 
%   $\{$\=$n \subseteq \tau, p \in n \to n, w \in \mu(n) \to n \land
%    p^{*} \mbox{ partial order} \land 
%              (n, p, w) \models \phi \;\land \heap(n, p, w) \}$
% \nextline 
%   Since $\ufcontains{\phi}{x}{y}$ implies $(n, p, w) \models \elt{x}{y}$, it 
%   follows that:
% \nextline 
%     $\{$\=$n \subseteq \tau, p \in n \to n, w \in \mu(n) \to n \land
%       p^{*} \mbox{ partial order} \land 
%              (n, p, w) \models \phi \;\land \heap(n - {x}, p, w) \;*$ 
% \nextline 
%     \> $\IfTE{p(x) = x}{\exists n.\; x \pointsto \Root(\IfTE{w(x) = x}{\None}{\Some(w(x))}, n)}{l \pointsto \Child(p(x))}
% \}$
% \nextline $\letv{v}{[!x]}{}$
% \nextline
%     $\{$\=$n \subseteq \tau, p \in n \to n, w \in \mu(n) \to n \land
%       p^{*} \mbox{ partial order} \land 
%              (n, p, w) \models \phi \;\land \heap(n - {x}, p, w)\;*$ 
% \nextline
%     \> $x \pointsto v\;\land$ 
% \nextline
%     \> $v = \IfTE{p(x) = x}{\exists n.\; \Root(\IfTE{w(x) = x}{\None}{\Some(w(x))}, n)}{\Child(p(x))} \}$
% \nextline $\run{}\ctext{case}(v,$
% \nextline \;\;$\Root(u,k) \to\;$
% \nextline \;\;
%     $\{$\=$n \subseteq \tau, p \in n \to n, w \in \mu(n) \to n \land
%       p^{*} \mbox{ partial order} \land 
%              (n, p, w) \models \phi \;\land \heap(n - {x}, p, w)\;*$ 
% \nextline
%     \> $x \pointsto v \;\land$
% \nextline
%     \> $v = \IfTE{p(x) = x}{\exists n.\; \Root(\IfTE{w(x) = x}{\None}{\Some(w(x))}, n)}{\Child(p(x))} \;\land$ 
% \nextline \> $v = \Root(u, k)\}$
% \nextline \;\;
%     $\{$\=$n \subseteq \tau, p \in n \to n, w \in \mu(n) \to n \land
%       p^{*} \mbox{ partial order} \land 
%              (n, p, w) \models \phi \;\land \heap(n - {x}, p, w)\;*$ 
% \nextline
%     \> $x \pointsto v \;\land$
% \nextline
%     \> $u = \IfTE{w(x) = x}{\None}{\Some(w(x))} \;\land$ 
% \nextline
%     \> $p(x) = x \;\land$
% \nextline \;\;\=$\ctext{case}(u,$ 
% \nextline \>\;\;\=$\None \to [(x, x, k)]$ 
% 
%   
% 
% \end{specification}
