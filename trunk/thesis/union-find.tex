\chapter{Proving the Union-Find Disjoint Set Algorithm}

In this chapter, we introduce the technique of ``ramification'', as a
way of recovering local reasoning in the face of imperative programs
with global invariants.

The union-find disjoint set data structure is a technique for
efficiently computing canonical representing for equivalence classes
of values. The basic technique for doing so is to represent each value
in the equivalence class as a node in a tree --- but unlike the usual
implementation of trees, each node does not contain a pointer to its
children, but rather the children each maintain a pointer to the
parent. The root of the tree has no parent pointer, and represents the
canonical representative for an equivalence class.

The canonical representative can be found (the FIND operation) by
following the parent pointers to the root of the tree. Similarly, two
disjoint sets can be merged (the UNION operation), by finding their
canonical representatives and setting one to point to the other.

As described, this data structure is no better than using a linked
list. However, two optimizations give rise to an extremely efficient
implementation. First, the root node can be modified to keep track of
a bound on the maximum height, so that whenever two sets are merged,
the shorter tree can be made a subtree of the deeper one. Second, the
algorithm can make use of \emph{path compression} -- whenever the FIND
operation is called, it can set all of the nodes on the path to the
root to point directly at the root. Together, these optimizations
permit performing a sequence of $n$ UNION and FIND operations in $O(n
\cdot \alpha(n))$ time, where $\alpha$ is the inverse Ackermann
function.

This permits simple implementation of near-linear-time unification
algorithms, and a variant of this idea is used in the proofs of
congruence closure algorithms.

However, path compression has been an idiom difficult to accomodate
within the framework of separation logic. In informal reasoning about
the union-find data structure, we do not explicitly track all the
elements of a union-find data structure in our reasoning --- instead,
we rely on the fact that path compression makes changes to the heap
which ``will not change the answer''. However, since separation logic
is a resource-aware logic, we can't simply leave the other elements of
the equivalence class out of the invariant, since UNION may read and
modify them.

The solution I propose in this chapter is to use a global invariant,
but structured in a way which preserves modular reasoning.

\begin{figure}
  \begin{mathpar}
    \begin{array}{lcl}
      \phi, \psi & ::= & I \bnfalt \elt{x}{y} \bnfalt \phi \otimes \psi
    \end{array}
  \end{mathpar}
  \begin{mathpar}
    \begin{array}{lcl}
      R(\rho, I) & = & I \\
      R(\rho, \psi \otimes \phi) & = & R(\rho, \psi) \otimes R(\rho, \phi) \\
      R(\rho, \elt{x}{y}) & = & \elt{x}{\rho(x)} \\
    \end{array}
  \end{mathpar}

  \begin{mathpar}
    \boxed{\ufcontains{\phi}{x}{y}}
    \\
    \inferrule*[]
              { }
              {\ufcontains{\elt{x}{y}}{x}{y}}
    \and
    \inferrule*[]
              {\ufcontains{\phi}{x}{y}}
              {\ufcontains{\phi \otimes \psi}{x}{y}}
    \and
    \inferrule*[]
              {\ufcontains{\psi}{x}{y}}
              {\ufcontains{\phi \otimes \psi}{x}{y}}
  \end{mathpar}
  \begin{specification}
    \nextline
    $\exists \tau : \star$ 
    \nextline 
    $\exists H : \formula \to \assert$ 
    \nextline
    $\exists \newset : \monad{\tau}$ 
    \nextline
    $\exists \find : \tau \to \monad{\tau}$ 
    \nextline
    $\exists \union : \tau \times \tau \to \monad{\unittype}$
    \nextline 
      $\spec{H(\phi)}{\newset}{a:\tau}{H(\phi \otimes \elt{a}{a})}$
    \\ \> $\specand$
    \nextline 
      $\spec{H(\phi) \land (\ufcontains{\phi}{x}{y})}
            {\find(x)}
            {a:\tau}
            {H(\phi) \land a = y}$
    \\ \> $\specand$
    \nextline 
      $\spec{H(\phi) \land (\ufcontains{\phi}{x}{y})
                     \land (\ufcontains{\phi}{u}{v})}
            {\union(x, u)}
            {a:1}
            {H(R([y/v], \phi))}$
  \end{specification}

\caption{Specification of Union Find Algorithm}
\label{union-find:spec}
\end{figure}

In Figure~\ref{union-find:spec}, we give a specification of the
union-find algorithm. In this specification, we essentially replay the
key trick of separation logic, only ``one level up''. That is, we give
a monolithic abstract predicate $H(\phi)$, which describes all of the
trees at once. We recover modularity in the index of the predicate
$\phi$, which gives a small (in fact, nearly degenerate) ``separation
logic'' for describing nodes. 

\begin{figure}
\mbox{}
\begin{specification}
\nextline $\sigma = \Child \of \reftype{\sigma} \bnfalt \Root \of \opttype\;{\reftype{\sigma}} \times \N$ 
\nextline $\tau = \reftype{\sigma}$ 

\nextlinelabel{union-find:invariant} 
   $H(\phi) \triangleq \exists$\=
     $n \subseteq \tau, p \in n \to n, w \in \mu(n) \to n.$
\nextline \> $p^{*} \mbox{ partial order} \;\land$ 
\nextline \> $(n, p, w) \models \phi \;\land$
\nextline \> $\heap(n, p, w)$ 

\nextline[1em] $n, p, w \models I \qquad\qquad$\=$\mbox{ iff}\;\;$\=$\mbox{always}$ 
\nextline      $n, p, w \models \phi \otimes \psi$\>$\mbox{ iff}$\>$
                     \exists n', n''.\; n = n' \uplus n'' \mbox{ and } n', p, w \models \phi \mbox{ and } 
                     n'', p, w \models \psi$
\nextline      $n, p, w \models \elt{x}{y}$ \> $\mbox{ iff }$\>$
                 w(\mathit{root}(x)) = y$

\nextline[1em] $\heap(n, p, w) =  \forall^{*} l \in n. \IfThenElse{p(l) = l}{\exists n.\; l \pointsto \Root(w(l), n)}{l \pointsto \Child(p(l))}$

\nextline[1em] $\mathit{root} : n \to \mu(p) = \mu(\semfun{f}{f \circ p})$
\end{specification}
\caption{Implementation of Union Find Algorithm}
\label{union-find:impl}
\end{figure}
 
On line~\ref{union-find:invariant} of Figure~\ref{union-find:impl} we
give the invariant for the union-find data structure. We assert the
existence of a finite set of nodes $n$, a map $p : n \to n$, and a map
$w : \mu(p) \to p$ (where $\mu(p)$ denotes the set of fixed points of
$p$). The set of nodes $n$ are all the sets that have been allocated,
and the parent map $p$ maps each node to its parent. The fixed points
of $p$ are the roots of the union-find trees, and the map $w : \mu(p)
\to p$ identifies the canonical witness associated with each root.

In order to ensure that the graph structure actually forms a tree, we
impose the condition that the transitive closure of $p$ is a partial
order. The antisymmetry condition means that there are no nontrivial
cycles in the relation, and since every node has (a) a single parent,
and (b) there are a finite number of nodes, we will reach a root node
in a finite number of steps.

Then, we assert that the triple $(n, p, w)$ models our formulas.
These formulas are a small subset of separation logic. We have the
formula $I$, which is always satisfied. Then we have the formula $\phi
\otimes \psi$, which corresponds to the usual separating conjunction,
in that the resource $n$ (the collection of disjoint-set nodes) is
split into two parts, one of which must support $\phi$ and the other
of which must support $\psi$. Note that the whole of the parent
function $p$ and the canonical witness function $w$ functions are
passed to both branches. This is the information that will let us
ensure that global constraints are maintained in local invariants.

In particular, the atomic proposition $\elt{x}{y}$ asserts that $x$ is
a node whose canonical witness is $y$, which is checked by consulting
$p$ to get the root, and then passing that to $w$.

\begin{figure}
\mbox{}
\begin{specification}
\nextline $\newset = \comp{\newref{\sigma}{\Root(\None, 0)}}$

\nextline[1em] $\ctext{findroot}(x) = [$\=$
                  \letv{v}{\comp{!x}}{}$ 
\nextline \>     $\run{}\ctext{case}($\=$v,$ 
\nextline \> \>     $\Root(w, n) \to [(x, w, n)],$
\nextline \> \>     $\Child(p) \to \ctext{findroot}(p))]$

\nextline[1em] $\find(x) = [$\=
             $\letv{(\_, w, \_)}{\ctext{findroot}(x)}{}$ 
\nextline \> $\ctext{case}($\=$w,$ 
\nextline \> \> $\None \to x,$  
\nextline \> \> $\Some(w) \to w)]$

\nextline[1em] $\union(x, y) = [$\= 
                 $\letv{(r, u, m)}{\find\;x}{}$ 
\nextline \>     $\letv{(s, v, n)}{\find\;y}{}$ 
\nextline \>     
\end{specification}
\caption{Implementation of Union-Find Algorithm}
\label{union-find:impl}  
\end{figure}

\begin{lemma}{(Existence of Upper Fixed Points)}
Suppose $A$ is a finite set, and $f : A \to A$ is a function whose
transitive closure is a partial order, and let $\mu(f)$ be the
set of fixed points of $f$. Then, there is a function 
$\mathit{root}_f : A \to \mu(f)$ such that for each $x$ there is a 
$k$ such that $f^{k} = \mathit{root}_f(x)$.
\end{lemma}

\begin{proof}
Observe that the transitive closure gives an order such that for
each $i \leq j$ and $x$ in $A$, $f^i(x) \sqsubseteq f^j(x)$. Since
$A$ is finite, it follows that the size of the longest chain of
distinct elements is at most $|A|$ -- so any sequence longer 
than that must repeat elements of $A$. 

Suppose there is a repeated element, occuring at the $i$-th and
$j$-th iterations of applying $f$ to $x$, where $i < j$. Since the
transitive closure is a partial order, we know that if $a \sqsubseteq
b$ and $a \sqsubseteq b$, then $a = b$. Therefore every element 
of the sequence from $i$ to $j$ is equal to $f^i(x)$. Since $i$ is
strictly less than $j$, this means that $f(f^i(x)) = f^i(x)$, and 
is hence a fixed point of $f$. 
\end{proof}

\begin{lemma}{(Fixed Point Update)}
Suppose $A$ is a finite set, and $f : A \to A$ is a function whose
transitive closure is a partial order, and let $\mu(f)$ be the set of
fixed points of $f$. For a given $z$ in $A$, if we define $f' =
[f|z:\mathit{root}_f(z)]$, then $\mathit{root}_f = \mathit{root}_{f'}$. 
\end{lemma}

\begin{proof}
Suppose that we have an arbitrary $x \in A$. Now, consider the totally
ordered set of elements $U = \comprehend{y}{y \sqsupseteq_f x}$, and 
$U' = \comprehend{y}{y \sqsupseteq_{f'} x}$. Now, $U$ either contains 
$z$ or not.

Suppose it does not contain $z$. Then $U = \comprehend{y}{y
  \sqsupseteq_{f'} x} = U'$, and so $\mathit{root}_{f'}(x) =
\mathit{root}_f(x)$.

Suppose $U$ does contain $z$. Then we know that $U$ contains
$\mathit{root}_f(z)$, and that this is an upper bound of every element
of $U$. Furthermore, since $U$ is totally ordered, we know that $U'$
must be smaller than $U$ -- it will contain none of the elements of
$U$ that were in between $z$ and $\mathit{root}_f(z)$.  That is, $U' =
\comprehend{y}{y \sqsupseteq_{f'} x} = \comprehend{y}{ z \sqsupseteq y
  \sqsupseteq x }$. But $U'$ still has a maximum equal to
$\mathit{root}_f(z)$. So $\mathit{root}_{f'}(x) =
\mathit{root}_{f}(x)$.
\end{proof}

\begin{lemma}{(Order Extension)}
Suppose $A$ and $B$ are disjoint finite sets, and $f : A \to A$ and $g
: B \to B$ are functions whose transitive closures are partial
orders. Let $\mu(f)$ and $\mu(g)$ be the set of fixed points of $f$
and $g$, respectively. 

Then $h : A \cup B \to A \cup B = f \cup g$ is also a function whose 
transitive closure is a partial order, with $\mu(h) = \mu(f) \cup \mu(g)$, 
and $\mathit{root}_h = \mathit{root}_f \cup \mathit{root}_g$. 
\end{lemma}


\begin{proof}
\begin{enumerate}
\item We need to show that the transitive closure of $h$ is a partial
order. To do this, we need to show that the antisymmetry axiom holds, 
since reflexivity and transitivity arise immediately from taking the
transitive closure. 

So, suppose that $(x,y) \in h^{*}$, and likewise $(y, x) \in
h^{*}$. From the definition of the transitive closure, we know there
is some $m$ and $n$ such that $y = h^m(x)$, and $x = h^n{y}$.  Since
$A$ and $B$ are disjoint, we know that either $x \in A$ or $x \in
B$. Furthermore, from the definition of $h$ we know that $h$ takes
elements of $A$ to elements of $A$ according to $f$, and takes
elements of $B$ to elements of $B$ according to $g$. So if $x \in A$,
then $y = f^m(x)$, and so $y \in A$, and so $x = f^n(y)$.  Then by the
antisymmetry of $f^{*}$, we know that $x = y$. So if $x \in B$, then
$y = g^m(x)$, and so $y \in B$, and so $x = g^n(y)$.  Then by the
antisymmetry of $g^{*}$, we know that $x = y$.

\item Next, we need to show that the set of fixed points of $h$ is the
  union of the fixed points of $f$ and $g$. This follows immediately
  from the fact that $A$ and $B$ are disjoint, and $f$ and $g$ are
  endofunctions.

\item Finally, we need to show that $\mathit{root}_h = \mathit{root}_f
  \cup \mathit{root}_g$. Again, since $A$ and $B$ are disjoint, there
  are no increasing chains of elements of $A \cup B$ that include
  elements of both $A$ and $B$.  Therefore, the upper fixed points of
  $h$ on its $A$ sub-domain will be given by $\mathit{root}_f$ and the
  upper fixed points of $h$ on its $B$ sub-domain will be given by
  $\mathit{root}_g$.
\end{enumerate}
\end{proof}

\begin{lemma}{(Monotonicity of $\ufcontains{\phi}{x}{y}$)}
Suppose $\ufcontains{\phi}{x}{y}$. Then $\ufcontains{\phi \otimes \psi}{x}{y}$. 
\end{lemma}

\begin{proof}
Take the derivation of $\ufcontains{\phi}{x}{y}$. Apply the tensor rule 
to it, and derive $\ufcontains{\phi \otimes \psi}{x}{y}$.
\end{proof}

\begin{lemma}
  Suppose $\ufcontains{\phi}{x}{y}$. Then for all $\rho$, $\ufcontains{R(\rho, \phi)}{x}{\rho(y)}$. 
\end{lemma}

\begin{proof}
This follows by induction on the derivation of $\ufcontains{\phi}{x}{y}$. 
\begin{itemize}
\item Suppose $\ufcontains{\elt{x}{y}}{x}{y}$. 

In this case $R(\rho, \elt{x}{y}) = \elt{x}{\rho(y)}$. Then apply the 
unit rule to derive $\ufcontains{R(\rho, \phi)}{x}{\rho(y)}$. 

\item Suppose that we use the tensor-left rule, so that by inversion we have
$\phi = \psi \otimes \theta$, and $\ufcontains{\psi}{x}{y}$. 

Then, by induction we have $\ufcontains{R(\rho, \psi)}{x}{\rho(y)}$. 
By the tensor-left rule, we have $\ufcontains{R(\rho, \psi) \otimes R(\rho, \theta)}{x}{\rho(y)}$. Then, by the definition of $R$, we have $\ufcontains{R(\rho, \psi \otimes \theta)}{x}{\rho(y)}$.

\item The tensor-right case is symmetric. 
\end{itemize}
\end{proof}

\begin{lemma}
  Suppose $(n, p, w) \models \phi$, and $n'$ is disjoint from $n$, and $p' \in n' \to n'$ such that $p'^*$ is a partial order, and $w' \in n' \to \mu(n')$. Then $(n, p \cup p', w \cup w') \models \phi$. 
\end{lemma}

\begin{proof}
 We proceed by induction on $\phi$. 

 \begin{itemize}
   \item Case $\phi = I$

     This case is immediate. 

   \item Case $\phi = \elt{x}{y}$. 

     In this case, we know that $p'$ is completely disjoint from $p$. As a result, the transitive closure of $p \cup p'$ is just the unions of the transitive closures of $p$ and $p'$, and any upper fixed point of an element of $n \cup n'$ is just the upper fixed point of $p$ or of $p'$. Therefore $\mathit{root}(x)$ remains unchanged, and since $w'$ is an extension of $w$, $w'(\mathit{root}(x)) = y$. 
     
     \item $\phi = \psi \otimes \theta$

       This follows by induction -- suppose $(n_1, p, w) \models \psi$ and 
       $(n_2, p, w) models \theta$. Then by induction $(n_1 \cup n', p', w') \models \theta$, and $(n_2, p', w') \models \theta$. (FIX THE INDUCTION HYP TO MAKE THIS WORK.)
       
 \end{itemize}
\end{proof}

\section{Correctness Proofs}

\subsection{Proof of $\newset$}

\begin{specification}
\nextline
  $\setof{H(\phi)}$ 
\nextline 
  $\{$\=$\exists n \subseteq \tau, p \in n \to n, w \in \mu(n) \to n.\;
      p^{*} \mbox{ partial order} \land 
             (n, p, w) \models \phi \;\land \heap(n, p, w) \}$
\nextline 
  $\{n \subseteq \tau \land p \in n \to n \land w \in \mu(n) \to n \land
      p^{*} \mbox{ partial order} \land 
             (n, p, w) \models \phi \;\land \heap(n, p, w) \}$
\nextline
  $\newref{\sigma}{\Root(\None, 0)}$ 
\nextline
  $\{$\=$n \subseteq \tau \land p \in n \to n \land w \in \mu(n) \to n \land
      p^{*} \mbox{ partial order} \land 
             (n, p, w) \models \phi \;\land$
\nextline \>
        $\heap(n, p, w) * a \pointsto \Root(\None, 0) \}$ 
\nextline
  $\{$\=$n \subseteq \tau \land p \in n \to n \land w \in \mu(n) \to n \land
      p^{*} \mbox{ partial order} \land 
             (n, p, w) \models \phi \;\land$
\nextline \>
        $\heap(n, p, w) * a \pointsto \Root(\None, 0) \}$ 
\nextlinelabel{newset:witness} Take $n' = n \cup {a}$, take $p' = [p|a:a]$, and $w = [w|a:a]$. 
\nextline
  $\{$\=$n' \subseteq \tau \land p' \in n' \to n' \land w' \in \mu(n') \to n' \land
      p'^{*} \mbox{ partial order} \land 
             (n', p', w') \models \phi \otimes \elt{a}{a} \;\land$
\nextline \>
        $\heap(n', p', w')\}$ 
\nextline 
  $\{$\=$\exists n' \subseteq \tau, p' \in n' \to n', w' \in \mu(n') \to n'.\;$ 
\nextline \>$p^{*} \mbox{ partial order} \land 
             (n', p', w') \models \phi \otimes \elt{a}{a} \;\land \heap(n', p', w') \}$
\nextline
  $\setof{H(\phi \otimes \elt{a}{a})}$ 

\end{specification}

The key step in the correctness proof of $\newset$ is on line~\ref{newset:witness}, where we define new witnesses and show that it satisfies the invariant. 

\begin{enumerate}
\item Since we allocate a new pointer, and each of the elements in $n$ is also a pointer in the current heap (due to the $\heap(n,p,w)$ assertion), it follows that the new value $a$ is disjoint from all of the values in $n$. 

\item Due to the disjointness, $\mu(n') = \mu(n) \cup \mu([a:a]) = \mu(n) \cup \setof{a}$.

\item Also due to the disjointness, the transitive closure of $p' = [p|a:a]$ is equal to the union of the transitive closure of $p$, (i.e., $p^{*}$) and the transitive closure of $[a:a]$ (i.e., $\setof{(a,a)}$). Since each of these is a partial order, and are disjoint from one another, their union is itself a partial order. So $p'^{*}$ is a partial order. 

\item We know that $(\setof{a}, p', w') \models \elt{a}{a}$, and that 
$(n, p', w') \models \phi$ (since $(n, p, w) \models \phi$), so it follows
that $(n', p', w') \models \phi \otimes \elt{a}{a}$. 

\item Since $p'(a) = a$, and $w'(a) = a$, it follows that $\heap(n, p, w) * 
* a \pointsto \Root(\None, 0)$ is the same as $\heap(n', p', w')$. 
\end{enumerate}

