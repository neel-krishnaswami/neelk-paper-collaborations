%for a more compact document, add the option openany to avoid
%starting all chapters on odd numbered pages
\documentclass[12pt]{cmuthesis}

% This is a template for a CMU thesis.  It is 18 pages without any content :-)
% The source for this is pulled from a variety of sources and people.
% Here's a partial list of people who may or may have not contributed:
%
%        bnoble   = Brian Noble
%        caruana  = Rich Caruana
%        colohan  = Chris Colohan
%        jab      = Justin Boyan
%        josullvn = Joseph O'Sullivan
%        jrs      = Jonathan Shewchuk
%        kosak    = Corey Kosak
%        mjz      = Matt Zekauskas (mattz@cs)
%        pdinda   = Peter Dinda
%        pfr      = Patrick Riley
%        dkoes = David Koes (me)

% My main contribution is putting everything into a single class files and small
% template since I prefer this to some complicated sprawling directory tree with
% makefiles.

% some useful packages
\usepackage{times}
\usepackage{fullpage}
\usepackage{graphicx}
\usepackage[numbers,sort]{natbib}
%  \usepackage[backref,pageanchor=true,plainpages=false, pdfpagelabels, bookmarks,bookmarksnumbered,
%  %pdfborder=0 0 0,  %removes outlines around hyper links in online display
%  ]{hyperref}
\usepackage{subfigure}
\usepackage{mathpartir}
\usepackage{amsmath}
\usepackage{amssymb}
\usepackage{amsthm}
\usepackage{verbatim}
% \usepackage[margin=1in]{geometry}
\usepackage{float}
\usepackage{stackrel}
\usepackage{xr}
% \usepackage{microtype}
\usepackage{fancyhdr}
\usepackage{multirow}
\setlength{\headheight}{15.2pt}

\newcommand{\ctext}[1]{\mathsf{#1}}

\newcommand{\normalize}[1]{\ctext{nf}({#1})}

\newcommand{\unittype}{\mathbf{1}}
\newcommand{\reftype}[1]{\ctext{ref}\;#1}
\newcommand{\monad}[1]{\bigcirc{#1}}
\newcommand{\cont}[1]{\ctext{cont}\;{#1}}
\newcommand{\opttype}[1]{\ctext{option }{#1}}
\newcommand{\seqsort}[1]{\ctext{seq}\;{#1}}
\newcommand{\listtype}[1]{\ctext{list}\;{#1}}

\newcommand{\pair}[2]{\left<{#1},{#2}\right>}
\newcommand{\fst}[1]{\ctext{fst}\; #1}
\newcommand{\snd}[1]{\ctext{snd}\; #1}

\newcommand{\inj}[1]{\iota_{#1}}
\newcommand{\inl}[1]{\ctext{inl }#1}
\newcommand{\inr}[1]{\ctext{inr }#1}
\newcommand{\Case}[5]{\ctext{case}(#1,\; {#2}.\; #3,\; {#4}.\; #5)}
\newcommand{\listcase}[5]{\ctext{case}(#1,\;\ctext{Nil} \to {#2},\;
                                       \ctext{Cons}({#3},{#4}) \to {#5})}
\newcommand{\optcase}[4]{\ctext{case}(#1,\;
                                      \ctext{None} \to {#2},\;
                                      \ctext{Some}\;{#3} \to {#4})}
\newcommand{\z}{\ctext{z}}
\newcommand{\s}[1]{\ctext{s}(#1)}
\newcommand{\iter}[4]{\ctext{iter}(#1, {#2}, {#3}.\; {#4})}
\newcommand{\iterseq}[4]{\ctext{iter}_{\mathsf{seq}}(#1, {#2}, #3.\; {#4})}
\newcommand{\comp}[1]{[#1]}
\newcommand{\fun}[3]{\lambda #1:#2.\;#3}
\newcommand{\Fun}[3]{\Lambda #1:#2.\;#3}
\newcommand{\unit}{\left<\right>}
\newcommand{\pack}[2]{\ctext{pack}({#1}, {#2})}
\newcommand{\unpack}[4]{\ctext{unpack}({#1}, {#2}) = {#3} \;\ctext{in}\;{#4}}
\newcommand{\alt}{\;|\;}
\newcommand{\letv}[3]{\ctext{letv}\;#1 = #2\;\ctext{in}\;#3}
\newcommand{\newref}[2]{\ctext{new}_{#1}(#2)}
\newcommand{\run}[1]{\ctext{run}\;{#1}}
\newcommand{\ok}{\ctext{ ok}}
\newcommand{\FV}[1]{\mathrm{FV}({#1})}

\newcommand{\fix}[2]{\ctext{fix}\;{#1}.\;{#2}}

\newcommand{\statecfg}[2]{\left<{#1};\;{#2}\right>}
\newcommand{\eval}[4]{\left<{#1};\;{#2}\right> \leadsto \left<{#3};\;{#4}\right>}
\newcommand{\evalabort}[2]{\left<{#1};\;{#2}\right> \leadsto \mathbf{abort}}

\newcommand{\domain}[1]{\mbox{dom}({#1})}
\newcommand{\upset}[1]{\mathcal{P}^{\uparrow}({#1})}

\newcommand{\pointsto}{\mapsto}
\newcommand{\disj}{\vee}
\renewcommand{\implies}{\supset}
\newcommand{\wand}{-\!\!*\,}
\newcommand{\emp}{\mathsf{emp}}
\newcommand{\validprop}[1]{{#1}\;\ctext{valid}}

\newcommand{\todo}[1]{\texttt{[TODO: {#1}]}}

\newcommand{\setof}[1]{\{{#1}\}}

\newcommand{\To}{\Rightarrow}
\newcommand{\From}{\Leftarrow}

\newcommand{\N}{\mathbb{N}}

\newcommand{\assert}{\ctext{prop}}

\newcommand{\bigstep}{\Downarrow}

\newcommand{\bigeval}[4]{\left<{#1};\;{#2}\right> \bigstep \left<{#3};\;{#4}\right>}
\newcommand{\bigevalabort}[2]{\left<{#1};\;{#2}\right> \bigstep \mathbf{abort}}

\newcommand{\spec}[4]{\{{#1}\}{#2}\{{#3}.\;{#4}\}}
\newcommand{\specX}[3]{\{{#1}\}{#2}\{{#3}\}}
\newcommand{\mspec}[4]{\langle{#1}\rangle{#2}\langle{#3}.\;{#4}\rangle}
\newcommand{\bnfalt}{\;\;|\;\;}

\newcommand{\specor}{\;||\;}
\newcommand{\specand}{\;\&\;}
\newcommand{\specimp}{\Rightarrow\!\!\!>}
% \newcommand{\specimp}{\Longrightarrow}
% \newcommand{\specor}{\ctext{ or }}
% \newcommand{\specand}{\ctext{ and }}
% \newcommand{\specimp}{\ctext{ implies }}
\newcommand{\spectype}{\ctext{spec}}
\newcommand{\valid}{\ctext{ valid}}

\newcommand{\interp}[1]{[\![{#1}]\!]}
\newcommand{\interpE}[1]{\interp{#1}^e}
\newcommand{\interpC}[1]{\interp{#1}^c}

\newcommand{\interpF}[1]{[\![{#1}]\!]_f}
\newcommand{\interpmono}[1]{\interp{#1}^{\mathrm{m}}}

\newcommand{\entails}{\models}

\newcommand{\judgeE}[4][\Theta]{{#1};\;{#2} \vdash {#3} : {#4}}
\newcommand{\judgeC}[4][\Theta]{{#1};\;{#2} \vdash {#3} \div {#4}}
\newcommand{\judgeEq}[5][\Theta]{{#1};\;{#2} \vdash {#3} \equiv {#4} : {#5}}
\newcommand{\judgeEqC}[5][\Theta]{{#1};{#2} \vdash {#3} \equiv {#4} \div {#5}}




%% semantic operations

\newcommand{\worldleq}{\preceq}
\newcommand{\worldgeq}{\succeq}

\newcommand{\semfun}[2]{\lambda #1.\;#2}
\newcommand{\sempair}[2]{\left({#1}, {#2}\right)}
\newcommand{\powerset}[1]{\mathcal{P}(#1)}
\newcommand{\powersetfin}[1]{\mathcal{P}^{\mathrm{fin}}(#1)}

\newcommand{\paircat}[2]{\left<{#1};{#2}\right>}
\newcommand{\sumcat}[2]{\left[{#1};{#2}\right]}
\newcommand{\abscat}[1]{\lambda({#1})}


\newcommand{\judgeP}[3]{{#1} \vdash {#2} : {#3}}
\newcommand{\judgeS}[2][\Delta]{{#1} \vartriangleright {#2} : \spectype}

\newcommand{\judgeSCtx}[2]{{#1} \vartriangleright {#2} : \ctext{context}}

\newcommand{\judgeEqP}[4]{{#1} \vdash {#2} \equiv {#3} : {#4}}


\newcommand{\judgeWK}[3][\Theta]{{#1} \vdash {#2} : {#3}}
\newcommand{\judgeKeq}[4][\Theta]{{#1} \vdash {#2} \equiv {#3} : {#4}}

\newcommand{\entailsP}[3]{{#1} \vartriangleright {#2} \vdash {#3}}
\newcommand{\entailsS}[3]{{#1}; {#2} \vdash {#3} \ok}

\newcommand{\chartp}{\ctext{char}}
\newcommand{\fonttp}{\ctext{font}}

\newcommand{\LOC}{loc}
\newcommand{\MONO}{\mathbf{mono}}
\newcommand{\HEAP}{heap}
\newcommand{\PROP}{Prop}
\newcommand{\TRUE}{True}
\newcommand{\HPROP}{Prop}
\newcommand{\TYPE}{Type}

\newtheorem{prop}{Proposition}
\newtheorem{lemma}{Lemma}

\newcommand{\comprehend}[2]{\setof{{#1}\;|\;{#2}}}


% \newenvironment{proof}{\begin{comment}}{\end{comment}}

\newenvironment{proof}{\noindent\textbf{Proof.}}{\noindent\ensuremath{\Box}}

\newcounter{prooflinenum}
\newenvironment{tabbedproof}
   {\setcounter{prooflinenum}{0}
    \begin{tabbing}\;\;\=\;\;\;\;\;\=\;\;\;\;\=\;\;\;\;\=\;\;\;\;\=\;\;\;\;\=\;\;\;\;\=\;\;\;\;\=\;\;\;\;\=\\[-2em]}
   {\end{tabbing}}

\newcommand{\oo}{\addtocounter{prooflinenum}{1}\arabic{prooflinenum}\>\>}
\newcommand{\ooo}{\addtocounter{prooflinenum}{1}\arabic{prooflinenum}\>\>\>}
\newcommand{\oooo}{\addtocounter{prooflinenum}{1}\arabic{prooflinenum}\>\>\>\>}
\newcommand{\ooooo}{\addtocounter{prooflinenum}{1}\arabic{prooflinenum}\>\>\>\>\>}
\newcommand{\oooooo}{\addtocounter{prooflinenum}{1}\arabic{prooflinenum}\>\>\>\>\>\>}
\newcommand{\ooooooo}{\addtocounter{prooflinenum}{1}\arabic{prooflinenum}\>\>\>\>\>\>\>}
\newcommand{\oooooooo}{\addtocounter{prooflinenum}{1}\arabic{prooflinenum}\>\>\>\>\>\>\>\>}
\newcommand{\ooooooooo}{\addtocounter{prooflinenum}{1}\arabic{prooflinenum}\>\>\>\>\>\>\>\>\>}

\newcommand{\ox}{\>\>}
\newcommand{\oox}{\>\>\>}
\newcommand{\ooox}{\>\>\>\>}
\newcommand{\oooox}{\>\>\>\>\>}
\newcommand{\ooooox}{\>\>\>\>\>\>}
\newcommand{\oooooox}{\>\>\>\>\>\>\>}
\newcommand{\ooooooox}{\>\>\>\>\>\>\>\>}
\newcommand{\oooooooox}{\>\>\>\>\>\>\>\>\>}


\newenvironment{eqnproof}[1][]{${#1}$\begin{displaymath}\begin{array}{lcll}}
                         {\end{array}\end{displaymath}}

\newcommand{\eline}[3][]{{#1} & = & {#2} & \mbox{{#3}} \\}

\newcommand{\elines}[3][]{{#1} & = & \begin{array}{l} #2 \end{array} & \mbox{{#3}} \\}

\newcommand{\eclaim}[3][]{{#1} &  & {#2} & \mbox{{#3}} \\}
\newcommand{\efact}[2]{{#1} & & & \mbox{#2} \\}

\newcommand{\basicspec}[4]{[{#1}]\;{#2}\;[{#3}.\;{#4}]}

% Macros for type checking assertions

\newcommand{\pfun}[3]{\hat{\lambda} #1:#2.\;#3}
\newcommand{\restrictkind}[1]{({#1})\Downarrow_K}
\newcommand{\restricttype}[1]{({#1})\Downarrow_T}

\newcommand{\restricttyenv}[2]{{#2}\Downarrow^{#1}_K}
\newcommand{\restrictvals}[2]{{#2}\Downarrow^{#1}_T}
\newcommand{\judgeACtx}[1]{\vartriangleright {#1} \;\ctext{ok}}
\newcommand{\judgeA}[3][\Delta]{{#1} \vartriangleright {#2} : {#3}}
\newcommand{\judgeSort}[2][\Delta]{\judgeA[{#1}]{{#2}}{\ctext{sort}}}
\newcommand{\judgeSortEq}[3][\Delta]{{#1} \vartriangleright {#2} \equiv {#3} : \ctext{sort}}

\newcommand{\ms}[1]{\mathcal{#1}}
\newcommand{\Frame}[2]{{#1} \otimes {#2}}



% Approximately 1" margins, more space on binding side
%\usepackage[letterpaper,twoside,vscale=.8,hscale=.75,nomarginpar]{geometry}
%for general printing (not binding)
\usepackage[letterpaper,twoside,vscale=.8,hscale=.75,nomarginpar,hmarginratio=1:1]{geometry}

% Provides a draft mark at the top of the document. 
% \draftstamp{\today}{DRAFT}


\begin{document} 
\frontmatter

%initialize page style, so contents come out right (see bot) -mjz
\pagestyle{fancyplain}

\title{ %% {\it \huge Thesis Proposal}\\
{\bf Verifying Higher-Order Imperative Programs with Higher-Order Separation Logic}}
\author{Neelakantan R. Krishnaswami}
\date{1 June 2012}
\Year{2012}
\trnumber{CMU-CS-12-127}

\committee{
Jonathan Aldrich, Co-chair \\
John C. Reynolds, Co-chair \\
Robert Harper \\
Lars Birkedal, IT University of Copenhagen
}

\support{This research was partially supported by National Science Foundation Grant
CCF-0916808.}
\disclaimer{The views and conclusions presented in this document are those of the author
and do not reflect the official opinions of the NSF or the U.S. government.}

% copyright notice generated automatically from Year and author.
% permission added if \permission{} given.

\keywords{Separation Logic, Design Patterns, Verification, Specification, Type Theory,
          Denotational Semantics, Program Logic, Program Correctness}

\maketitle

\begin{dedication}
I would like to thank my wife, Rachel, for her love, patience, and
logistical support. 

\ \\

My parents encouraged me to study computers.

I am not at all sure this was what they had in mind. 
\end{dedication}

% \pagestyle{plain} % for toc, was empty
 

\pagestyle{fancyplain}
\renewcommand{\chaptermark}[1]{\markboth{#1}{}}
 
% \lhead{\fancyplain{}{\thepage} --- \texttt{DRAFT 2.167}}
% \lhead{\fancyplain{}{\thepage} --- \texttt{\inp}
\lhead{\fancyplain{}{\thepage}}
\chead{}
\rhead{\fancyplain{}{\textit{\leftmark}}}
\lfoot{}
\cfoot{}
\rfoot{}



%% Obviously, it's probably a good idea to break the various sections of your thesis
%% into different files and input them into this file...

\begin{abstract}
In this thesis I show is that it is possible to give modular
correctness proofs of interesting higher-order imperative programs
using higher-order separation logic.

To do this, I develop a model higher-order imperative programming
language, and develop a program logic for it. I demonstrate the power
of my program logic by verifying a series of examples. This includes
both realistic patterns of higher-order imperative programming such as
the subject-observer pattern, as well as examples demonstrating the
use of higher-order logic to reason modularly about highly aliased
data structures such as the union-find disjoint set algorithm.
\end{abstract}

\begin{acknowledgments}
I would like to begin by thanking my advisors for giving me perhaps
more rope than was advisable. John and Jonathan were both very
tolerant of my sudden enthusiasms, my half-baked ideas, and general
tendency to wander off in peculiar directions. 

Lars was an invaluable source of support and advice, and after every
trip to Denmark I returned significantly more knowledgeable than
before I left. I would like to thank Bob for an act he may not even
recall any more: as a new graduate student, he informed me that I
would be coming to ConCert group meetings, despite technically not
even being part of the ConCert project.
 
As a consequence, I have the need and the pleasure to thank the
whiteboard gang for their willingness to talk. Noam Zeilberger and
Jason Reed were particularly inspiring examples of people following
their mathematical muse, and William Lovas, Daniel Licata, Kevin
Watkins, Joshua Dunfield, Carsten Varming, and Rob Simmons were
reliable sources of type-theoretic and semantic inspiration.
Whenever I needed a different perpsective, the Plaid group was
there. I'd like to thank Donna Malayeri, Kevin Bierhof, and Nels
Beckman in particular.

Hongseok Yang, Peter O'Hearn, Philippa Gardner, and Matthew Parkinson
were all encouraging above and beyond the call of duty to a random
student from thousands of miles away. 

\end{acknowledgments}


\tableofcontents
\listoffigures
% \listoftables

\mainmatter

%% Double space document for easy review:
%\renewcommand{\baselinestretch}{1.66}\normalsize

% The other requirements Catherine has:
%
%  - avoid large margins.  She wants the thesis to use fewer pages, 
%    especially if it requires colour printing.
%
%  - The thesis should be formatted for double-sided printing.  This
%    means that all chapters, acknowledgements, table of contents, etc.
%    should start on odd numbered (right facing) pages.
%
%  - You need to use the department standard tech report title page.  I
%    have tried to ensure that the title page here conforms to this
%    standard.
%
%  - Use a nice serif font, such as Times Roman.  Sans serif looks bad.
%
% Other than that, just make it look good...
% bib

\begin{sloppypar}
\chapter{Introduction}

My thesis is that it is possible to give modular correctness proofs of
interesting higher-order imperative programs using higher-order
separation logic.

\section{Motivation}

It is more difficult to reason about programs that use aliasing than
ones that do not use mutable shared data. It is more difficult to
reason about programs that use higher order features than first order
programs. Put both together, and matters become both challenging and
interesting for formal verification, since the combination yields
languages more complex than the sum of their parts. 

Techniques to reason about purely functional programs, which
extensively use higher-order methods but eschew mutable state, are an
extremely well-developed branch of programming language theory, with a
long and successful history.

Historically, techniques to reason about mutable state have lagged
behind, but some years ago O'Hearn and Reynolds introduced separation
logic~\citep{sep-logic}, which has proven extremely successful at
enabling correctness proofs of even such intricate imperative programs
as garbage collectors~\citep{gc-proof}. However, separation logic has
historically focused on low-level programming languages that lack a
strong type discipline and allow the use of techniques such as casting
pointers to integers.

However, even high level languages that allow the use of state are
prone to aliasing errors, since (with a few exceptions) type systems
do not track interference properties. Some languages, such as
Haskell~\cite{haskell-report}, isolate all side-effects within a
particular family of monadic types. While this preserves reasoning
principles for functional code, we still face the problem that
reasoning about monadic code remains as difficult as ever. Haskell's
type discipline imprisons mutation and state, but does not
rehabilitate them.

Of course, we can combine language features combinatorially, and if
this particular combination had no applications, then it would be only
of technical interest. In fact, though, higher-order state pervades
the design of many common programs. For example, graphical user
interfaces (GUIs) are typically structured as families of callbacks
that operate over the shared data structures representing the
interface and program. These callbacks are structured using the
subject-observer pattern~\cite{gof}, which uses collections of
callbacks to implicitly synchronize mutable data structures across a
program. So higher-order state not only poses a technical challenge,
but also offers a tantalizingly concrete example to motivate the
technical development.

In this dissertation, I develop a model higher-order imperative
programming language, and develop a program logic for it. I
demonstrate the power of my program logic by verifying many
interesting examples, culminating in the correctness proof of a
library for event-driven programming, which despite an extremely
imperative (and higher-order) implementation, nevertheless permits
clients to reason about it using simple and powerful equational
reasoning principles.

\section{Programming Language}

The first contribution of my dissertation is to give a denotational
semantics for a predicative version of System
$F^\omega$~\cite{girard-thesis}, extended with a monadic type constructor for
state in Pfenning-Davies~\cite{pfenning-davies} style. This language
can be considered an instance of Moggi's~\cite{moggi-monads} monadicf
metalanguage, as all heap effects and nontermination are encapsulated
within the monadic type constructor.

Higher-order imperative programs usually contain a very substantial
functional part in addition to whatever imperative operations they
perform. As a result, it is very convenient to be able to reason
equationally about programs when possible (including induction
principles for inductive types), and by giving a denotational
semantics, I can easily show the soundness of these equational
properties.

Ths semantics of this language is the focus of Chapter 2. In
particular, I show that a programming language including references to
polymorphic values, still gives rise to domain equations which can
still be solved using the classical techniques of Smyth and
Plotkin~\cite{smyth-plotkin}.

\section{Higher Order Separation Logic}

The main tool I develop to prove programs correct is a higher order
separation logic for a higher-order imperative programming
language. The details of the logic are given in Chapter 3, but 
I will say a few words to set the stage now. 

As I mentioned earlier, we can reason about the functional part of
programs using the standard $\beta\eta$-theory of the lambda calculus,
but to reason about imperative computations, I introduce a
specification logic, in the style of Reynolds' specification logic for
Algol~\citep{spec-logic}.

In this logic, the basic form of specification is the Hoare triple
over an imperative computation $\spec{P}{c}{a:A}{Q}$. Here, $P$ is the
precondition, $c$ is the computation that we are specifying, and $Q$
is the post- condition. The binder $a:A$ names the return value of the
computation, so that we can mention it in the post-condition.

The assertion logic for the pre- and post-conditions is higher order
separation logic~\citep{hosl}. This is a substructural logic that
extends ordinary logic with three spatial connectives to enable
reasoning about the aliasing behavior of data. The Hoare triples
themselves form the atomic propositions of a first-order
intuitionistic logic of specifications. The quantifiers range over the
sorts of the assertion logic, so that we can universally and
existentially quantify over assertions and expressions.

As a teaser example, consider the specification of a counter module.

\begin{tabbing}
$\exists \alpha : \star$ \\
$\exists \mathtt{create} : \unittype \to \monad{\alpha}$ \\
$\exists \mathtt{next} : \alpha \to \monad{\N}$ \\
$\exists $\= $\mathit{counter} : \alpha \times \N \To \assert$ \\
\> $\mspec{\emp}{\mathtt{create}()}{a:\alpha}{\mathit{counter}(a, 0)}$ \\
\> $\specand$ \\
\> $\forall c:\alpha, n:\N.\; \mspec{counter(c, n)}{\mathtt{next}(c)}{a:\N}{a = n \land \mathit{counter}(c, n+1)}$
\\
\end{tabbing} 

The idea is that in our program logic, we can assert the existence of
an abstract type of counters $\alpha$, operations $\mathtt{create}$
and $\mathtt{next}$ (which create and advance counters, respectively),
and a state predicate $\mathit{counter}(c, n)$ (which asserts that the
counter $c$ has value $n$).

The specifications are Hoare triples --- the triple for $\mathtt{create}$ asserts
that from an empty heap, calling $\mathtt{create}$ creates a counter initialized
to 0, and the triple for $\mathtt{next}(c)$ asserts that if $\mathtt{next}$ is called
on $c$, in a state where it has value $n$, then the return value will be $n$ and
the state of the counter will be advanced to $n+1$. 

Now, here are two possible implementations for the existential witnesses:

\begin{displaymath}
\begin{array}{lcl}
\alpha & \equiv & \reftype{\N} \\[0.5em]
counter & \equiv & \lambda (r, n).\; (r \pointsto n) \\[0.5em]
\mathtt{create} & \equiv & \lambda ().\; [\newref{\N}{0}] \\[0.5em]
\mathtt{next}  & \equiv & \lambda r.\; [\letv{n}{[!r]}{
                                        \letv{()}{[r := n + 1]}{n}}] \\
\end{array}
\end{displaymath}

and also                                        
\begin{displaymath}
\begin{array}{lcl}
\alpha & \equiv & \reftype{\N} \\[0.5em]
counter & \equiv & \lambda (r, n).\; (r \pointsto n + 7) \\[0.5em]
\mathtt{create} & \equiv & \lambda ().\; [\newref{\N}{7}] \\[0.5em]
\mathtt{next}  & \equiv & \lambda r.\; [\letv{n}{[!r]}{
                                        \letv{()}{[r := n + 1]}{n - 7}}] \\
\end{array}
\end{displaymath}

So in our program logic, the implementations of modules consist of the
witnesses to the existential formulas. This is morally an application
of Mitchell and Plotkin's identification of data abstraction and
existential types~\cite{mitchell-plotkin} --- in this view, linking a
module with a client is nothing but an application of the existential
elimination rule of ordinary logic.  However, note that instead of
abstracting only over types (as, we also abstract over the \emph{heap}.

In order for a verification methodology to scale up to even modestly-
sized programs, it must be modular. There are three informal senses in
which I use the word ``modular'', each of which is supported by
different features of this logic, and are (mostly) illustrated in this
example. 

\begin{enumerate}
\item First, we should be able to verify programs library by
  library. That is, we should be able to give the specification of a
  library, and prove both that implementations satisfy the
  specification without knowing anything about the clients that use
  it, and likewise prove the correctness of a client program without
  knowing anything about the details of the implementation (beyond
  what is promised by the specification). 

  In the example above, I tackle this problem by making use of the
  fact that the presence of existential specifications in our
  specification logic lets use the Mitchell-Plotkin encoding of
  modules as existentials. So once I prove a client against this
  specification, I can swap between implementations without having
  to re-verify the client. 

\item Related to this is a modular treatment of aliasing. Ordinary
  Hoare logic becomes non-modular when asked to treat mutable, shared
  state, because we must explicitly specify the aliasing or
  non-aliasing of every variable and mutable data structure in our
  pre- and post-conditions. Besides the quadratic dependence on the
  size of the program, the relevant set of variables grows whenever a
  subprogram is embedded in a larger. Separation logic resolves this
  problem by introducing via the frame rule, a feature which we carry
  forward in our logic.

  In our example, counters have state, and so the predicates for
  distinct counters need to held distinct. So even in the simplest
  possible example, it is useful to be able to have the power of 
  separation logic available. 

\item Finally, it is important to ensure that the abstractions we
  introduce compose. Benton\cite{benton-modularity} described a
  situation where he was able to prove that a memory allocator and the
  rest of the program interacted only through a particular interface,
  but when he tried to divide the rest of the program into further
  modules, he encountered difficulties, because it was unclear how to
  share the resources and responsibility for upholding the contract
  with the allocator.

  While the previous example was too simple to illustrate these kinds
  of problems, I do exploit the expressiveness of my program logic to
  introduce several new \emph{specification patterns} for verifying
  these kinds of programs. In the following section, I will describe
  some of the patterns I discovered.
\end{enumerate}

\section{Verifying Programs}

The final test of a program logic is what we can prove with it, and so
accordingly I have devoted a great deal of effort to not only prove
metatheorems \emph{about} the logic, but also theorems \emph{in} my
logic.

\subsection{Design Patterns}

One prominent source of examples of higher-order state arises in
object-oriented programs. One way of translating objects into
functional language is by viewing objects as records of functions
(methods) accessing common hidden state (fields). As a result, common
programming idioms in object-oriented languages can be viewed as
patterns of higher-order imperative programs.

Over the years, object-oriented programmers have documented many of
idioms which repeatedly arise in practice, calling them \emph{design
  patterns}~\cite{gof}. While originally intended to help practitioners
communicate with each other, design patterns also offer a nice collection 
of small, but realistic and widely-occurring, programs to study for
verification purposes. 

In Chapter 4, I translate many common design patterns into my
programming language\footnote{Unsurprisingly, they turn into idioms
  that many ML programmers will find familiar.} and then give
specifications and correctness proofs for these programs. I want to
emphasize that I view the specifications as even more important a
contribution as the correctness proofs themselves: programmers have
informal reasons for believing in the correctness of their programs,
which are often surprisingly subtle to formalize. 

\subsection{The Union-Find Algorithm}

One of the reasons for the success of separation logic (and related
substructural logics like linear logic) in program verification is the
fact that aliasing turns out to be only used lightly in many
programs. 

In Chapter 5 of my dissertation, I study the
union-find~\cite{union-find} algorithm. This algorithm is very
challenging to verify, because its correctness and efficiency relies
intimately upon the ability to use shared mutable data to broadcast
updates to the whole program.

To deal with this problem, I make use of the expressive power of
higher-order separation logic, and introduce a \emph{domain-specific
  logic} to describe the interface between a union-find library and
its clients. This allows the implementation to keep a global view of
the union-find data structure, while still permitting clients to be
proved via local reasoning.

Furthermore, those operations which have genuinely global effects
(such as taking the union of two disjoint sets) can be specified using
custom modal operators I call \emph{ramifications}. These operators
commute with the custom separating conjunction, and so allow local
reasoning even when global effects are in play.

\subsection{Verifying Event-Driven Programs}

In Chapter 6, I give the culminating examples of the proof methodology
I have developed in support of my thesis. This chapter divides into 
two main parts. 

In the first part, I implement and prove the correctness of an
imperative dataflow graph implementation. In a dataflow graph, we have
a collection of cells, which each contain a piece of code, which is
executed when the cell is asked for a value. Cells memoize their
values and remember their dependencies, so that when they are asked
again for a value, they will not unnecessarily recompute a
value. Conversely, if a cell is modified, it will invalidate its
memoized value and transitively invalidate anything that dependend on
that value.  The specification of the dataflow graph relies crucially
on ramifications in order to modularly specify dependencies and the
effects of modifications to the graph.

Then, in the second part, I use this dataflow graph implementation to
in turn implement a library for functional reactive
programming~\cite{frp}. I prove the correctness of the imperative
implementation relative to a mathematically simple (but unusably
inefficient) semantics somewhat in the style of realizability, which
allows programmers to reason about the imperative implementation as if
it were a pure one. 

Furthermore, this correctness proof relies only on the interface to
the the dataflow graph library, and not to its implementation. This
demonstrates that very sophisticated libraries can still be factored
into modular parts, and moreover that the correctness proofs can be
cleanly factored as well. 

% \subsection{Imperative Modularity}
% 
% While separating modules and clients in this way is obviously useful,
% for imperative programs it is not entirely a sufficient mechanism on
% its own.
% 
% In a purely functional program, all communication between an
% implementation and a client of a module is mediated by the signature
% of the module. The addition of state makes it possible to create an
% extra channel of communication through the heap, and in our formalism
% this is captured by adding pre- and post-conditions to the
% specifications of imperative actions.
% 
% However, suppose we have a client program with specification $C$,
% which makes use of a particular module implementation with a
% specification $S$. Furthermore, suppose that \emph{both} the client
% program implemeting $C$, and the module implementing the $S$
% interface, rely upon yet another module with interface $F$. As an
% example, imagine an $F$ that has a purely functional interface, but
% which makes use of an imperative cache for efficiency reasons (for
% example, to implement memoization). Now, even though there is aliased
% state between $C$ and $I$, they cannot interfere with one one another
% through it.
% 
% Therefore, a properly modular specification \emph{should not require}
% $S$ to mention the use of the caching module. Otherwise we will be
% forced to mention more and more extraneous details, as we layer
% modules on top of one another, with the total size of the
% specifications growing with the size of the reachable module graph.
% 
% The language I am proposing is able to properly describe this
% situation, thanks to our ability to use the frame rule. Concretely,
% suppose we have the following interface for our third module.
% 
% \begin{tabbing}
% $\forall n:\N.\; \spec{cache}{\mathtt{fib}(n)}{a:\N}{cache \land a = fibonacci(n)}$ \\
% \end{tabbing} 
% 
% Here, we can imagine \texttt{fib} produces fibonacci numbers using the
% naive exponential-time recursive implementation, but adds caching to
% ``optimize'' it to linear time. The $cache$ predicate appears in the
% pre- and post-conditions of $\mathtt{fib}$, so at first it would seem
% that a $C$ and $S$ that used this function would have to mention that
% they require the heap associated with $cache$.
% 
% However, note that $cache$ is the same in both the pre- and
% post-condition. This means that we can prove $C$ and $I$ using a
% specification 
% $$F \equiv \spec{\emp}{\mathtt{fib}(n)}{a:\N}{\emp \land a =
%   fibonacci(n)}$$
% and then we can frame $cache$ on at the very end, after
% completing the proofs of $C$ and $I$. Concretely, the proof tree 
% looks like this:
% 
% \begin{mathpar}
% \inferrule*[right=Cut]
%            {\vdash F \otimes cache \\ 
%             \inferrule*[right=Frame]
%                        {\inferrule*{F \vdash S \\
%                                   F, S \vdash C}
%                                  {F \vdash C}}
%                       {F \otimes cache \vdash C}
%           }
%           {\vdash C \otimes cache}
% \end{mathpar}
% 
% Here, I write $R \otimes p$ to suggest that $p$ is framed onto the
% triples in the specification $R$. Note that in the subderivation above
% the use of the Frame rule, we do not need to carry around $cache$ in
% our specification of $\mathtt{fib}$, which means that when we
% introduce $S$, we will not need to mention $cache$ in its specification.
% 
% Simple caching schemes are very easy to hide -- all we need is to make
% use of the frame rule~\cite{sep-inf}. A stronger test of our ability
% to hide irrelevant detail will be if we can hide Liskov and Wing's
% \emph{monotonic constraints}~\cite{liskov-wing}.
% 
% Monotonic invariants are program invariants that hold monotonically
% throughout a program's execution. Suppose we have a partial order $(S,
% \sqsubseteq)$ and a predicate $P$, such that if $s' \sqsubseteq s$
% then $P(s') \implies P(s)$. Now, if we then ensure that any triples
% that mention $P$ are of the form $\spec{P(s)}{c}{a:\tau}{\exists s'.\;
%   P(s') \land s' \sqsubseteq s}$, then we know that no unknown uses of
% this module can break our invariant, since they can only make $P$ ``more
% true''. 
% 
% This property is used in several verification methodologies to prove
% programs with potentially unknown calls~\cite{boogie-sub-obs}, but 
% using it for information hiding purposes may be novel. 
% 
% Related to monotonic effects are \emph{commutative effects}~\cite{idioms}. 
% These are side-effects in which the order they happen in does not affect
% correctness. For example, a gensym operation is a commutative effect, since 
% all we care about is the fact that two different invocations return distinct
% symbols. 

% \section{Interesting Higher-Order Imperative Programs}
% 
% Validating my thesis means that I have to demonstrate that it will
% work on interesting imperative programs. So what do I mean with the
% word ``interesting''? There are several senses in which I use it:
% 
% \begin{itemize}
% \item One kind of interesting program are programs that pose
%   particular difficulties for verification, such as ``Landin's knot'',
%   which implements recursion by backpatching a function pointer. These
%   kinds of procedures are typically small, and may not be found in
%   actual software very often\footnote{Recently, I found a natural use
%     of this technique in a real program, so I do not mean ``not very
%     often'' as a polite way of saying ``never''.}, but help
%   demonstrate the capabilities and limits of a verification
%   methodology.
% 
% \item A second kind of interesting program are example programs that
%   capture a particular stylized patterns of use that arise frequently
%   in practice. Design patterns~\cite{gof} are a good example of this
%   kind of program -- they represent what might be called engineering
%   wisdom, and formalizing these patterns can help demonstrate that our
%   verification methodology can accomodate typical good design, and
%   moreover its formal version is not overly cumbersome to use. 
% 
% \item A third kind of interesting program are big programs. Even if a
%   verification methodology works successfully on small examples, it
%   remains to be seen whether it will work on larger examples, because
%   in a larger example we will have to show that the different patterns
%   of use can coexist peacefully, and that they compose gracefully. 
% 
%   Also, I only mean ``big'' relative to the size of typical examples
%   (which are usually in the tens of lines of code). I will be happy
%   with verifying a program that is on the order of high hundreds to
%   low thousands of lines of code. This is, of course, very small by
%   industrial standards, but it is large enough to learn whether or
%   not the different pieces of this methodology fit together. 
% \end{itemize}
% 
% 
% \subsubsection{Design Patterns}
% 
% I will begin by analyzing individual design patterns, giving a
% specification, an implementation, and a small client program for each
% of the design patterns with a substantially stateful part.
% 
% The intuition here is that 1) object-oriented software is the biggest
% repository of higher-order imperative code out there, and 2) the Gang
% of Four design patterns are well-respected engineering wisdom about
% how to manage such designs. So if I can show that I can formalize that
% practice, I'll have taken a big step towards showing I can formally
% capture the sorts of reasoning that software engineers use to manage
% such programs.
% 
% Furthermore, there's also a \emph{negative} component here. I want to
% figure out what patterns of reasoning are difficult to capture in my
% framework, and systematically analyzing the design patterns should
% help me search the design space.
% 
% One further thing that's worth noting is that a design pattern
% \emph{isn't} a particular formula of specification logic. Any
% specification we write represent particular balances between
% generality (that is, more possible implementations) and specificity
% (that is, more invariants for a client to use). For example, suppose
% we have an iterator specification which supports deletion. For some
% collection types (such as stretchable arrays), deleting an element
% invalidates only those iterators that have are further along the array
% than the one that did the deletion. If we put this fact into the
% specification, then clients can use this fact to not invalidate some
% iterators. But the price is that this specification rules out
% collection implementations such as binary trees, where any deletion
% will invalidate all other iterators.\footnote{This means that any spec
% will be criticized by both those who think it is too liberal, and
% those who think it is too restrictive. I think this is a feature,
% not a bug -- making assumptions explicit in a way that permits
% detailed argument is a big win!}
% 
% 
% \begin{itemize}
% \item[Iterator]
% \item[Subject Observer]
% 
% These two I have written about extensively. I need to pull in the text
% from those here.
% 
% \item[Flyweight]
% 
% The basic idea here is similar to memoization: cache the construction
% of (immutable) objects, and on an object request only return a new one
% if a satisfactory one is not in the cache. The cache, of course, is a
% global mutable data structure that \emph{we don't want to see} in the
% proof of any clients. This makes it a good use case for the frame
% rule, and the issues that arise here will be a good test of whether we
% can actually scale proofs.
% 
% \item[Memento]
% 
% This pattern is commonly used to support undo. It's basically an
% opaque token representing an object's old state, and the object has a
% method that the client can use (with the memento) to reset the state.
% We can implement this with higher-order state -- the token can be a
% command the user can invoke, and there are some nice sequencing issues
% in when you can invoke the undo operation or not. We should also look
% at multilevel undo, and compare this to an internal undo operation.
% 
% \item[Chain of Responsibility]
% 
% The Chain of Responsibility pattern is a pattern for composing
% commands from other commands. Rather than building a monolithic
% command, we break it up into processing units that take a message and
% handle the parts of the update they can, delegating the rest to the
% rest of the chain. This doesn't have to be a strict chain, so you can
% have a ``tree of responsibility'' as well.
% 
% There's probably a very slick specification of this in terms of state
% transformers, and I think there might be a nice ownership transfer
% idiom lurking in here that's worth unearthing. Then again, maybe not
% -- I've had trouble finding any really complex uses of aliasing yet.
% 
% \item[State]
% 
% This is a pattern in which an object is made from other objects which
% provide abstract state -- it does not know the precise implementation
% of its components. In fact, it's okay for the implementation of the
% concrete state to \emph{change} as the program runs. Verifying this
% example is primarily useful as a diagnostic test; of course we can
% handle it, but even the slightest hint of ugliness in our development
% indicates trouble in the logic.
% 
% \item[Composite]
% 
% At first, I thought that the Composite pattern didn't have any
% interesting issues, but then I realized I was wrong, because we are
% often composing \emph{stateful} objects. For example, consider an
% iterator library (such as the \texttt{itertools} library in Python)
% which constructs new iterators out of existing ones.
% 
% Concretely, suppose we have \texttt{PairIterator} that takes two
% iterators and returns a new one that produces pairs of values,
% iterating over the two arguments in lockstep. This seems
% straightforward, until you think that this implies certain sharing
% constraints -- in particular, the arguments have to be two
% \emph{different} iterators.
% 
% So we need to be able to say something about sharing here. This may be
% easiest if we use multiple specifications for the same function (for
% example, if the two iterators come from the same underlying
% collection, or not).
% 
% \item[Proxy] 
% 
% The Proxy pattern does not have very much complexity in its
% implementation; it's essentially just a (possibly-imperative) wrapper
% function -- its ML type could be \texttt{proxy : (a -> b) -> (a -> b)}.
% However, analyzing it will still offer some interesting evidence for 
% or against my thesis. That's because at heart a Proxy transforms 
% the specification of its argument, and we can examine how hard it is
% to model that. 
% 
% Note that state can play an important in implementing a real
% Proxy, because it provides us with a ``back channel'' for
% communication.  For example, a security Proxy that checks to see
% whether the owner of a piece of data is authorized before invoking a
% method might have a data structure it consults for that authorization
% information.
% 
% The Decorator pattern is similar to this, only we can add methods
% to our input in addition to transforming old ones. 
% 
% \item[Strategy]
% 
% In a language with first-class functions, there's not much to the
% Strategy pattern -- we just need a pointer to a function. But the very
% simplicity of this pattern means that we ought to be able to prove
% some fun programs, like showing the correctness of a program that
% updates its strategy as it runs.
% 
% 
% \item[Facade]
% 
% The Facade pattern takes some existing modules and presents a nice
% abstraction of them to the programmer. The way I conceptualize it, we
% want to compose imperative modules in a way that turns their state
% into a single new predicate. This is pretty straightforward in easy
% cases, but I'm interesting in finding out if there's a way to make
% this \emph{hard} I'd like to stress-test our ability to abstract over
% state.
% 
% In particular, the test of whether this works will be seeing how big
% the predicates grow. If I composing two modules with $n$-place
% predicates describing their heap yields a Facade with a $2n$-place
% predicate, it's pretty clear that information hiding isn't genuinely
% happening. This contrasts with the Flyweight, where the test of
% modularity is seeing how many starred subformulas show up in the
% pre- and post-conditions. 
% 
% \end{itemize}
% 
% \section{Contributions}
% 
% To demonstrate my thesis, I make a number of concrete contributions.
% In Chapter 2, I define a higher-order imperative programming language
% based on a predicate variant of $F_\omega$~\citep{fomega}, augmented
% with reference types which uses a monadic language to encapsulate its
% side-effects (including both modification of the heap, and
% nontermination). I give this language a domain-theoretic denotational
% semantics based on the techniques of \citet{smyth-plotkin}, which lets
% me validate strong equational reasoning principles --- including both
% the $\beta-$ and $\eta$-rules of the lambda calculus ---
% 
% Equational reasoning is less helpful for imperative programs, and to
% support reasoning about this part of the programming language, I
% define and prove the soundness of a program logic in Chapter 3. The
% program logic combines ideas from specification logic and higher-order
% separation logic to give an expressive program logic 

\chapter{Type Structure}

\section{Syntax of Types and Kinds}

In this section, I will describe the type structure of the programming
language I will use in this dissertation. The language is a pure,
total, predicatively polymorphic programming language (with
quantification over higher kinds), augmented with a monadic type
constructor that permits nontermination and higher-order state. The
syntax of types is given in figure~\ref{lang-type-syntax}.


\begin{figure}[h]
\begin{displaymath}
  \begin{array}{lcll}
    \mbox{Kinds} & 
      \kappa & ::= & \star \bnfalt \kappa \to \kappa 
    \\[1em]
    \mbox{Monotypes} & 
      \tau & ::= & 
         \unittype \bnfalt 
         \tau \times \tau \bnfalt 
         \tau \to \tau \bnfalt 
         \tau + \tau \bnfalt
         \N \bnfalt \\
     &&& \reftype{A} \bnfalt
         \monad{\tau} \bnfalt 
         \\ % \cont{\tau} \\
     &&& \alpha \bnfalt
         \tau\;\tau \bnfalt 
         \fun{\alpha}{\kappa}{\tau} 
    \\[1em]
    \mbox{Polytypes} & 
      A,B & ::= & 
         \unittype \bnfalt 
         A \times B \bnfalt 
         A \to B \bnfalt 
         A + B \bnfalt
         \N \bnfalt  \\
    &&&  \reftype{A} \bnfalt
         \monad{A} \bnfalt \tau \bnfalt \\
    &&&  \forall \alpha:\kappa.\; A \bnfalt 
         \exists \alpha:\kappa.\; A \\[1em]
    \mbox{Type Contexts} & 
      \Theta & ::= & \cdot \bnfalt \Theta, \alpha:\kappa \\
  \end{array}
\end{displaymath}
\caption{Syntax of the Types and Kinds of the Programming Language}
\label{lang-type-syntax}
\end{figure}

The basic kind structure of the language is given with the kinds
$\kappa$, which range over the kind $\star$, the kind of ground
monotypes, and higher kinds $\kappa \to \kappa$ built from it. We write
all of these constructors with the letters $\tau$ and $\sigma$. The
monotype constructors are the unit type $\unittype$, pair types $\tau
\times \sigma$, sums $\tau + \sigma$, function space $\tau \to
\sigma$, natural numbers $\N$, computation types $\monad{\tau}$, and
finally (ML-style) references $\reftype{A}$. Also, within open types
we may also use type variables $\alpha$ to refer to monotype constructors, and we
can also define lambda-abstractions and applications to inhabit the
higher kinds of this language.

The type $\reftype{A}$ is not merely a pointer to a value of
monomorphic type; it also permits storing a pointer to a value of
polymorphic types $A$. This seemingly violates the usual stratification
between monotypes and polytypes, since quantifiers can occur within
$A$. The intuition for viewing $\reftype{A}$ as a monotype is that a
reference to a value of polymorphic type is itself merely a location
with no additional structure, and so it is safe to treat a reference
to a value of polymorphic type as a monomorphic value. Our
denotational semantics of references will formalize this intuition
and make it precise.

The polytypes $A$ themselves extend the monotypes with universal
quantification $\forall \alpha:\kappa.\;A$ as well as existential
types $\exists \alpha:\kappa.\;A$. Each of the simple type
constructors --- sums, products, functions, computations --- also may
contain polymorphic types as subexpressions within it. However, this
is actually only a modest generalization of classical ML-style type
schemes. Because the universal and existential quantifiers range over
the kinds $\kappa$, it is impossible to instantiate them with a
polytype, thereby limiting us to predicative
polymorphism. Nevertheless, being able to quantifier over higher kinds
and instantiate quantifiers with them is sufficient to model many
useful idioms (for example, quantifying not just over the element type
of a list, but also quantifying over the collection type constructor).
For the occasional cases we need to write impredicatively polymorphic
programs, we will simulate true impredicativity by passing references.

The kinding judgments $\judgeWK{\tau}{\kappa}$ and
$\judgeWK{A}{\bigstar}$ determine well-formedness of monotypes and
polytypes, respectively. The judgment $\judgeWK{\tau}{\kappa}$ asserts
that the monotype constructor $\tau$ has the kind $\kappa$, and as can
be seen in Figure~\ref{monotype-kinding}, the type constructors all
have the expected structure. The rule for references calls out to the
judgment $\judgeWK{A}{\bigstar}$, defined in
Figure~\ref{polytype-kinding}, which gives well-formedness conditions
for polytypes. The reason we need a second judgment is that there is
no kind of polytypes, and so we simply directly judge whether a
polytype is well-formed. 

These judgments, and all others in this thesis, follow the usual
Barendregt variable convention, in that variable names do not occur
repeated in contexts, and that bound variables are renamed (according
to the usual rules of alpha-equivalence) so that they are different
from the free variables.

\begin{figure}[h]
\begin{mathpar}
\inferrule*[right=KUnit]
          { }
          {\judgeWK{\unittype}{\star}}
\and
\inferrule*[right=KProd]
          {\judgeWK{\tau}{\star} \\
           \judgeWK{\sigma}{\star}}
          {\judgeWK{\tau \times \sigma}{\star}}
\and
\inferrule*[right=KArrow]
          {\judgeWK{\tau}{\star} \\
           \judgeWK{\sigma}{\star}}
          {\judgeWK{\tau \to \sigma}{\star}}
\and
\inferrule*[right=KSum]
          {\judgeWK{\tau}{\star} \\
           \judgeWK{\sigma}{\star}}
          {\judgeWK{\tau + \sigma}{\star}}
\and
\inferrule*[right=KNat]
          { }
          {\judgeWK{\N}{\star}}
\and
\inferrule*[right=KRef]
          {\judgeWK{A}{\bigstar}}
          {\judgeWK{\reftype{A}}{\star}}
\and
\inferrule*[right=KComp]
          {\judgeWK{\tau}{\star}}
          {\judgeWK{\monad{\tau}}{\star}}
% \and
% \inferrule*[right=KCont]
%           {\judgeWK{\tau}{\star}}
%           {\judgeWK{\cont{\tau}}{\star}}
\\
\inferrule*[right=KHyp]
          { \alpha:\kappa \in \Theta }
          { \judgeWK{\alpha}{\kappa} }
\and
\inferrule*[right=KApp]
          { \judgeWK{\tau}{\kappa' \to \kappa} \\
            \judgeWK{\tau'}{\kappa'} }
          { \judgeWK{\tau\;\tau'}{\kappa} }
\and
\inferrule*[right=KLam]
          { \judgeWK[\Theta, \alpha:\kappa']{\tau}{\kappa} }
          { \judgeWK{\fun{\alpha}{\kappa'}{\tau}}{\kappa' \to \kappa} }
\end{mathpar}
\caption{Kinding Rules for Monotypes}
\label{monotype-kinding}
\end{figure}

\begin{figure}[h]
\begin{mathpar}
\inferrule*[right=KUnit]
          { }
          {\judgeWK{\unittype}{\bigstar}}
\and
\inferrule*[right=KProd]
          {\judgeWK{A}{\bigstar} \\
           \judgeWK{B}{\bigstar}}
          {\judgeWK{A \times B}{\bigstar}}
\and
\inferrule*[right=KArrow]
          {\judgeWK{A}{\bigstar} \\
           \judgeWK{B}{\bigstar}}
          {\judgeWK{A \to B}{\bigstar}}
\and
\inferrule*[right=KSum]
          {\judgeWK{A}{\bigstar} \\
           \judgeWK{B}{\bigstar}}
          {\judgeWK{A + B}{\bigstar}}
\and
\inferrule*[right=KNat]
          { }
          {\judgeWK{\N}{\bigstar}}
\and
\inferrule*[right=KComp]
          {\judgeWK{A}{\bigstar}}
          {\judgeWK{\monad{A}}{\bigstar}}
% \and
% \inferrule*[right=KCont]
%           {\judgeWK{A}{\bigstar}}
%           {\judgeWK{\cont{A}}{\bigstar}}
\\
\inferrule*[right=KMono]
          { \judgeWK{\tau}{\star} }
          { \judgeWK{\tau}{\bigstar} }
\\
\inferrule*[right=KForall]
           { \judgeWK[\Theta, \alpha:\kappa]{A}{\bigstar} }
           { \judgeWK{\forall \alpha:\kappa.\;A}{\bigstar} }
\and
\inferrule*[right=KExists]
           { \judgeWK[\Theta, \alpha:\kappa]{A}{\bigstar} }
           { \judgeWK{\exists \alpha:\kappa.\;A}{\bigstar} }
\end{mathpar}
\caption{Kinding Rules for Polytypes}
\label{polytype-kinding}
\end{figure}

\subsection{The Syntactic Theory of Kinds}

Types also support a pair of equality judgments
$\judgeKeq{\tau}{\tau'}{\kappa}$ and $\judgeKeq{A}{B}{\bigstar}$, shown
in Figures \ref{monotype-equality} and \ref{polytype-equality}.  The
equality judgment for monotypes implements the $\beta$- and
$\eta$-equality principles of the lambda calculus, along with
congruence rules for all of the type constructors of our language.
The only rules we have for the equality judgment for polytypes are
simple congruence rules, plus a recursive call back to the other
equality judgment whenever we need to compare monotyped terms.

\begin{figure}[h]
\begin{mathpar}
\inferrule[]
          { }
          {\judgeKeq{1}{1}{\star}}
\and
\inferrule[]
          {\judgeKeq{\tau}{\tau'}{\star} \\
           \judgeKeq{\sigma}{\sigma'}{\star}}
          {\judgeKeq{\tau \times \sigma}{\tau' \times \sigma'}{\star}}
\and
\inferrule[]
          {\judgeKeq{\tau}{\tau'}{\star} \\
           \judgeKeq{\sigma}{\sigma'}{\star}}
          {\judgeKeq{\tau \to \sigma}{\tau' \to \sigma'}{\star}}
\and
\inferrule[]
          {\judgeKeq{\tau}{\tau'}{\star} \\
           \judgeKeq{\sigma}{\sigma'}{\star}}
          {\judgeKeq{\tau + \sigma}{\tau' + \sigma'}{\star}}
\and
\inferrule[]
          { }
          {\judgeKeq{\N}{\N}{\star}}
\and
\inferrule[]
          { \judgeKeq{A}{A'}{\bigstar} }
          { \judgeKeq{\reftype{A}}{\reftype{A'}}{\star} }
\and
\inferrule[]
          { \judgeKeq{\tau}{\tau'}{\star} }
          { \judgeKeq{\monad{\tau}}{\monad{\tau'}}{\star} }
\and
% \inferrule[]
%           { \judgeKeq{\tau}{\tau'}{\star} }
%           { \judgeKeq{\cont{\tau}}{\cont{\tau'}}{\star} }
% \and
\inferrule[]
          { \alpha:\kappa \in \Theta }
          { \judgeKeq{\alpha}{\alpha}{\kappa} }
\and
\inferrule[]
          { \judgeKeq{\tau}{\sigma}{\kappa' \to \kappa} \\
            \judgeKeq{\tau'}{\sigma'}{\kappa'}}
          { \judgeKeq{\tau \; \tau'}{\sigma \; \sigma'}{\kappa} }
\and
\inferrule[]
         { \judgeKeq[\Theta, \alpha:\kappa']{\tau}{\sigma}{\kappa} }
         { \judgeKeq{\fun{\alpha}{\kappa'}{\tau}}
                    {\fun{\alpha}{\kappa'}{\sigma}}
                    {\kappa' \to \kappa} }
\and
\inferrule[]
          { \judgeWK{(\fun{\alpha}{\kappa'}{\tau})\;\tau'}{\kappa} }
          { \judgeKeq{(\fun{\alpha}{\kappa'}{\tau})\;\tau'}
                     {[\tau'/\alpha]\tau}
                     {\kappa} }
\and
\inferrule[]
          { \judgeKeq[\Theta, \alpha:\kappa']
                     {\tau\;\alpha}{\sigma\;\alpha}{\kappa} \\ \alpha \not\in \FV{\tau,\sigma}}
          { \judgeKeq{\tau}{\sigma}{\kappa' \to \kappa} }
\\
\inferrule[]
          { \judgeWK{\tau}{\kappa} }
          { \judgeKeq{\tau}{\tau}{\kappa} }
\and
\inferrule[]
          { \judgeKeq{\tau}{\sigma}{\kappa} }
          { \judgeKeq{\sigma}{\tau}{\kappa} }
\and
\inferrule[]
          { \judgeKeq{\tau}{\tau'}{\kappa} \\
            \judgeKeq{\tau'}{\sigma}{\kappa} }
          { \judgeKeq{\tau}{\sigma}{\kappa} }
% \and
% \inferrule[]
%           { \judgeKeq{\tau}{\tau'}{\kappa'} \\ 
%             \judgeKeq[\Theta, \alpha:\kappa']{\sigma}{\sigma'}{\kappa} }
%           { \judgeKeq{[\tau/\alpha]\sigma}{[\tau'/\alpha]\sigma'}{\kappa} }
\end{mathpar}
\caption{Equality Rules for Monotypes}
\label{monotype-equality}
\end{figure}


\begin{figure}
\begin{mathpar}
\inferrule[]
          { }
          {\judgeKeq{1}{1}{\bigstar}}
\and
\inferrule[]
          {\judgeKeq{A}{A'}{\bigstar} \\
           \judgeKeq{B}{B'}{\bigstar}}
          {\judgeKeq{A \times B}{A' \times B'}{\bigstar}}
\and
\inferrule[]
          {\judgeKeq{A}{A'}{\bigstar} \\
           \judgeKeq{B}{B'}{\bigstar}}
          {\judgeKeq{A \to B}{A' \to B'}{\bigstar}}
\and
\inferrule[]
          {\judgeKeq{A}{A'}{\bigstar} \\
           \judgeKeq{B}{B'}{\bigstar}}
          {\judgeKeq{A + B}{A' + B'}{\bigstar}}
\and
\inferrule[]
          { }
          {\judgeKeq{\N}{\N}{\bigstar}}
\and
\inferrule[]
          { \judgeKeq{A}{A'}{\bigstar} }
          { \judgeKeq{\reftype{A}}{\reftype{A'}}{\bigstar} }
\and
\inferrule[]
          { \judgeKeq{A}{A'}{\bigstar} }
          { \judgeKeq{\monad{A}}{\monad{A'}}{\bigstar} }
\and
% \inferrule[]
%           { \judgeKeq{A}{A'}{\bigstar} }
%           { \judgeKeq{\cont{A}}{\cont{A'}}{\bigstar} }
% \and
\inferrule[]
          { \judgeKeq{\tau}{\tau'}{\star} }
          { \judgeKeq{\tau}{\tau'}{\bigstar} }
\and
\inferrule[]
          { \judgeKeq[\Theta, \alpha:\kappa]{A}{B}{\bigstar} }
          { \judgeKeq{\forall \alpha:\kappa.\;A}
                     {\forall \alpha:\kappa.\;B}
                     {\bigstar} }
\and
\inferrule[]
          { \judgeKeq[\Theta, \alpha:\kappa]{A}{B}{\bigstar} }
          { \judgeKeq{\exists \alpha:\kappa.\;A}
                     {\exists \alpha:\kappa.\;B}
                     {\bigstar} }
\\
\inferrule[]
          {\judgeWK{A}{\bigstar}}
          {\judgeKeq{A}{A}{\bigstar}}
\and
\inferrule[]
          {\judgeKeq{A}{B}{\bigstar} \\ \judgeKeq{B}{C}{\bigstar}}
          {\judgeKeq{A}{C}{\bigstar}}
\and
\inferrule[]
          {\judgeKeq{A}{B}{\bigstar}}
          {\judgeKeq{B}{A}{\bigstar}}
\end{mathpar}
\caption{Equality Rules for Polytypes}
\label{polytype-equality}
\end{figure}

\begin{prop}{(Weakening)}
If $\judgeWK[\Theta]{\tau}{\kappa}$  then $\judgeWK[\Theta, \alpha:\kappa']{\tau}{\kappa}$. 
\end{prop}
\begin{proof}
This follows from structural induction on the derivation of 
$\judgeWK[\Theta]{\tau}{\kappa}$. 
\end{proof}\\


\begin{prop}{(Substitution)}
Suppose $\judgeWK{\tau'}{\kappa'}$. Then 
  \begin{itemize}
  \item If $\judgeWK[\Theta, \alpha:\kappa']{\tau}{\kappa}$, then $\judgeWK{[\tau'/\alpha]\tau}{\kappa}$.
  \item If $\judgeWK[\Theta, \alpha:\kappa']{A}{\bigstar}$, then $\judgeWK{[\tau'/\alpha]A}{\bigstar}$.
  \end{itemize}
\end{prop}
\begin{proof}
This follows from mutual structural induction on the derivation of 
$\judgeWK[\Theta, \alpha:\kappa']{\tau}{\kappa}$ and $\judgeWK[\Theta, \alpha:\kappa']{A}{\bigstar}$. 
\end{proof}\\


\begin{prop}{(Well-Kindedness of Equality)}
If we have that $\judgeKeq{\tau}{\sigma}{\kappa}$, then we know that 
$\judgeWK{\tau}{\kappa}$ and $\judgeWK{\sigma}{\kappa}$.  Likewise, 
$\judgeKeq{A}{B}{\bigstar}$, then we know that $\judgeWK{a}{\bigstar}$ and $\judgeWK{b}{\bigstar}$.  

\end{prop}
\begin{proof}
This follows from mutual structural inductions on the derivation of   
$\judgeKeq{\tau}{\sigma}{\kappa}$ and $\judgeKeq{A}{B}{\bigstar}$. 
\end{proof}\\

\begin{prop}{(Substitution into Equality)}
  If we have that $\judgeKeq{\sigma}{\sigma'}{\kappa_1}$, then 
  \begin{itemize}
    \item If $\judgeKeq[\Theta, \alpha:\kappa_1]{\tau}{\tau'}{\kappa_2}$, then
             $\judgeKeq{[\sigma/\alpha]\tau}{[\sigma'/\alpha]\tau'}{\kappa_2}$.
    \item If $\judgeKeq[\Theta, \alpha:\kappa_1]{A}{B}{\bigstar}$, then
             $\judgeKeq{[\sigma/\alpha]A}{[\sigma'/\alpha]B}{\bigstar}$.
  \end{itemize}
\end{prop}
\begin{proof}
This follows by mutual structural induction on the two derivations of the 
equality judgment.
\end{proof}

\subsection{Semantics of Kinds}

Because types and kinds form an instance of the simply typed lambda
calculus, we would like to argue that there is a unique
$\beta$-normal, $\eta$-long form for each well-kinded type expression.
If we consider only the monotype constructors excluding the reference
types, this is immediate.  However, the presence of quantifiers in
reference types slightly complicates this story. Luckily, redices can
only occur in subterms of polytypes which are monotype constructors, and
hence by an induction on the structure of a polytype we can deduce that
it has unique normal forms as well. 

As a result, when we quotient the set of well-kinded terms by the
equality judgment, we know that each equivalence class contains a
single long normal term, which we can take as the canonical
representative of that class. We can formalize this by giving the
set-theoretic semantics of the kinds as follows.

\begin{displaymath}
  \begin{array}{l}
    \interpset{\kappa} = \comprehend{\tau}{\tau \mbox{ is $\beta$-normal, $\eta$-long} 
                                        \mbox{ and } \judgeWK[\cdot]{\tau}{\kappa} } \\
    \interpset{\bigstar} = \comprehend{A}{A \mbox{ is $\beta$-normal, $\eta$-long} 
                                       \mbox{ and } \judgeWK[\cdot]{A}{\bigstar} }
  \end{array}
\end{displaymath}

So we take the meaning of a kind $\kappa$ to be exactly the closed,
$\beta$-normal, $\eta$-long terms of that kind. Then, we can give a semantics of
typing derivations as follows. First, we define the interpretation of kinding 
contexts to be a tuple of kinds, in the usual way:

\begin{displaymath}
  \begin{array}{lcl}
    \interpset{\cdot} & = & \unittype \\
    \interpset{\Theta, \alpha:\kappa} & = & \interpset{\Theta} \times \interpset{\kappa} \\
  \end{array}
\end{displaymath}

Then, the meaning of a derivation $\judgeWK{\tau}{\kappa}$ is a set-theoretic
function taking the environment into the interpretation of kinds: 

\begin{displaymath}
  \begin{array}{lcl}
    \interpset{\judgeWK{\tau}{\kappa}} & \in & \interpset{\Theta} \to \interpset{\kappa} \\
    \interpset{\judgeWK{\tau}{\kappa}} & = & \semfun{\theta \in \interpset{\Theta}}{\normalize{\theta_\Theta(\tau)}} \\[1em]
    \interpset{\judgeWK{A}{\bigstar}} & \in & \interpset{\Theta} \to \interpset{\bigstar} \\
    \interpset{\judgeWK{A}{\bigstar}} & = & \semfun{\theta \in \interpset{\Theta}}{\normalize{\theta_\Theta(A)}} \\
  \end{array}
\end{displaymath}

Here, $\theta(\tau)$ is a tuple in $\interp{\Theta}$, and the notation $\theta_\Theta$
denotes a function which turns it back into a substitution:
\begin{displaymath}
  \begin{array}{lcl}
    ()_\cdot & = & [] \\
    (\theta, \tau)_{\Theta, \alpha:\kappa} & = & [\theta_\Theta, \tau/\alpha] \\
  \end{array}
\end{displaymath}

We then take the substitution $\theta_\Theta$ and substitutes for each
of the free variables in $\tau$ (or $A$), and $\normalize{\tau}$
computes the normal form of the type constructor $\tau$.

This means that the following theorems about the equality theory on
type constructors are true: 

\begin{prop}{(Equality Judgment is Sound)}
We have that 
\begin{enumerate}
\item If $\judgeKeq{\tau}{\tau'}{\kappa}$, then $\tau =_{\beta\eta} \tau'$ 
\item If $\judgeKeq{A}{A'}{\bigstar}$, then $A =_{\beta\eta} A'$ 
\end{enumerate}
\end{prop}

\begin{proof}
This proof is an easy structural induction on the equality judgment. 
\end{proof}

\section{Semantics of Types}

Now, we want to give a semantics for these types, which we will then
use to interpret terms of our programming language. Repeating the
design criteria mentioned earlier, we need to obey the two principles
below:

\begin{itemize}
\item Our interpretation of types should make all non-monadic types 
  pure. 
\item In particular, I want to treat even nontermination as a side
  effect, in addition to the more-obvious effects of control and state. 
\end{itemize}

The purpose of this choice is twofold. First, it will give us a rich
subset of the language which is total and pure, which will be
convenient when writing assertions about programs --- we will be able
to use any pure function directly in assertions, without having to
worry about side effects in predicates. Second, purity means that we
will get a very rich equality theory for the language --- both the
$\beta$ and $\eta$ laws will hold for all of the types of the
programming language, which will facilitate equational reasoning about
the pure part of the programming language.

Because we count nontermination as an effect, our denotational
semantics is in CPO, the category of complete partial orders and
continuous functions between them. In particular, we do not demand
that all domains have least elements --- that is, we only require that
the objects of this category be \emph{predomains}, rather than
domains. This permits us to model pure types as predomains lacking a
bottom element.

\subsection{The Overall Plan}

The central problem in giving the semantics of types is the
interpretation of the monadic type constructor. Since the monadic type
is the type of imperative computations, it must be a kind of heap or
predicate transformer. However, heaps contain values of arbitrary
polymorphic type, which defeats simple attempts to give semantics by
induction on types.

Our strategy to resolve this circularity is to give successive domain
equations for monotypes, polytypes, and heaps, suitably parametrized
so as to let us avoid reference to the semantics of a later stage in
an earlier one. Then, we will solve the domain equation, and tie the
knot.

\begin{enumerate}
\item For each ground monotype $\tau$, we give a functor in $CPO_\bot
  \times CPO^{op}_\bot \to CPO$.  The parameters to this functor will
  eventually be the basic continuations.
\item Using this interpretation of monotypes, for each polymorphic
  kinding judgment, we will give another functor in $CPO_\bot \times
  CPO^{op}_\bot \to CPO$. Again, the parameters to this functor will
  eventually be the basic continuations.
\item For heaps, a functor $CPO_\bot \times CPO^{op}_\bot \to
  CPO$. Again, the parameters to this functor will eventually be the
  basic continuations.
\item A domain equation in $CPO_\bot \times CPO^{op}_\bot \to
  CPO_\bot$, whose least fixed point we will take as our domain of
  basic continuations.
\end{enumerate}

Roughly speaking, the basic continuations will be the (continuous)
maps from heaps to primitive observations (the 2-point Sierpinski
space), and by interpreting the monadic type as a CPS type with answer
type equal to the basic continuations, we can view monadic terms as
heap transformers.

\subsection{Interpreting Monotypes}

In Figure~\ref{interp-monotypes}, we give an interpretation of the
closed, canonical monotypes (i.e., monotypes of kind $\star$, with no
free occurrences of type variables within them, and no $\beta$-redexes
within them) as a functor in $CPO_\bot \times CPO^{op}_\bot \to CPO$.
This semantics is parametrized with two arguments $K^+$ and $K^-$,
which separate the positive and negative occurrences of the basic 
continuations. 

\begin{figure}
\begin{displaymath}
\interpmono{-} : \mbox{Monotype} \to CPO_\bot \times CPO^{op}_\bot \to CPO   
\end{displaymath}
\begin{displaymath}
\begin{array}{lcl}
\mbox{Monotype} & = & \interpset{\star} \\
Loc    & = & \N \times \interpset{\bigstar} \\
Loc(A) & = & \comprehend{(n, B) \in Loc}{A = B} \\[1em]
% Loc    & = & \bigcup \setof{X \;|\; \exists A.\; (\judgeWK[\cdot]{A}{\bigstar}) \;\land\;X = Loc(A)} \\[1em]

\interpmono{\judgeWK[\cdot]{\unittype}{\star}}(K_+, K_-) & = & \setof{*} \\

\interpmono{\judgeWK[\cdot]{\N}{\star}}(K_+, K_-) & = &  \N \\

\interpmono{\judgeWK[\cdot]{\tau \times \sigma}{\star}}(K_+, K_-) & = & 
  \interpmono{\judgeWK[\cdot]{\tau}{\star}}(K_+, K_-) \times 
  \interpmono{\judgeWK[\cdot]{\sigma}{\star}}(K_+, K_-) \\

\interpmono{\judgeWK[\cdot]{\tau + \sigma}{\star}}(K_+, K_-) & = & 
  \interpmono{\judgeWK[\cdot]{\tau}{\star}}(K_+, K_-) + 
  \interpmono{\judgeWK[\cdot]{\sigma}{\star}}(K_+, K_-) \\


\interpmono{\judgeWK[\cdot]{\tau \to \sigma}{\star}}(K_+, K_-) & = & 
  \interpmono{\judgeWK[\cdot]{\tau}{\star}}(K_-, K_+) \to
  \interpmono{\judgeWK[\cdot]{\sigma}{\star}}(K_+, K_-) \\

\interpmono{\judgeWK[\cdot]{\reftype{A}}{\star}}(K_+, K_-) & = & Loc(\normalize{A}) \\

% \interpmono{\judgeWK[\cdot]{\cont{\tau}}{\star}}(K_+, K_-) & = & 
%     \interpmono{\judgeWK[\cdot]{\tau}{\star}}(K_-,K_+) \to K_+ \\

\interpmono{\judgeWK[\cdot]{\monad{\tau}}{\star}}(K_+, K_-) & = & 
   (\interpmono{\judgeWK[\cdot]{\tau}{\star}}(K_+, K_-) \to K_-) \to K_+ \\[1em]
\end{array}
\end{displaymath}
\caption{Locally Continuous Functor Interpreting Monotypes}
\label{interp-monotypes}
\end{figure}
Before explaining the clauses in detail, I will explain why it is
well-defined at all. First, because we are considering closed terms of
kind $\star$, the normalization theorem tells us that any such term
will normalize to one of the cases listed above. In particular, we
will never bottom out at a variable, because the context is closed. We
will never bottom out at a lambda, because we are considering only the
kind $\star$, and we will never bottom out an application, because
there will be room for further beta-reduction in this case, and by
hypothesis we are only considering the normal forms.

This means we cover all of the possibilities in this definition, and
furthermore we know it is well-founded, because all of the recursive
calls to $\interpmono{-}$ are always on immediate subterms of the
type.

Most of the clauses of this definition should be relatively
straightforward --- the main mystery is that we have parametrized this
interpretation by two arguments $K_+$ and $K_-$, whose meaning I will
explain when we reach the monadic type constructor.  We interpret the
unit type as the one-element, discretely ordered predomain, the
natural number type as the natural numbers with a discrete order,
pairs as the categorical products of $CPO$, sums as coproducts, and
functions via the exponentials of $CPO$.

Reference types $\reftype{A}$ are interpreted as pairs consisting of
natural numbers and the representative syntactic object $A$. The
intuition is that a reference is just a number, together with a type
tag saying what type of value the reference will point to. It is
important that we do \emph{not} interpret the type tag $A$ in this
definition --- a ref cell is a number plus the purely syntactic object
$A$, acting as a label. This is because we have no interpretation
function that can interpret the quantifiers yet: the current
definition interprets only the monotypes.

The first time we use $K$ is when we interpret the type
$\monad{\tau}$.  The monadic type $\monad{\tau}$ is interpreted in
continuation-passing style, as $(\interpmono{\tau}(K_+, K_-) \to K_-)
\to K_+$, and the the $K_+$ and $K_-$ arguments are revealed as the
positive and negative occurrences of the ``answer type'' of the
continuation. One minor fact worth pointing out is that the positive
and negative occurrences do \emph{not} trade places on the recursive
call to $\interp{\tau}$, since it occurs on the left-hand-side of the
left-hand-side of a function space (which is hence a positive
occurrence).

Note how this works: $\monad{\tau}$ is to be the type of stateful
computations, but there is (apparently) no state in this definition;
it seems like an ordinary continuation semantics.  The way that we
will re-introduce state is in the definition of $K$. We will
ultimately interpret the answer type $K$ as maps from heaps $H$ to the
two-point Sierpinski lattice $O = \setof{\top, \bot}$, so that $K = H
\to O$. Then, the monadic type will mean $(\interp{\tau} \to (H \to
O)) \to (H \to O)$, which can be understood as a sort of predicate
transformer.

\subsection{Interpreting Polytypes}

At this stage of the definition, we cannot yet define what heaps mean,
since heaps can contain references to values of polymorphic type, and
we have not yet defined the semantics of polymorphic types. This is
why we have carefully parametrized our functorial semantics of
monotypes so that we do not needed to mention them explicitly.

To continue closing this gap, we will give an interpretation of
polymorphic types as an indexed function from a context of closed
mono-kinded type constructors to a CPO.  As before, we parametrize by
the continuation arguments, and again define a functor. This
definition is given in Figure~\ref{interp-polytypes}.

\begin{figure}
\begin{displaymath}
\begin{array}{lcl}\interp{\judgeWK{A}{\bigstar}} &\in& \interpset{\Theta} \to (CPO_\bot \times CPO^{op}_\bot) \to CPO 
\\[1em]
\interp{\judgeWK{\tau}{\bigstar}}\;\theta\;(K_+,K_-) & = & 
    \interpmono{\judgeWK[\cdot]{\theta(\tau)}{\star}}(K_+, K_-) 
\\
\interp{\judgeWK{\forall \alpha:\kappa.\;A}{\bigstar}}\;\theta\;(K_+,K_-) & = & 
    \Pi \tau:\kappa.\; 
        \interp{\judgeWK[\Theta, \alpha:\kappa]{A}{\bigstar}}\;(\theta,\tau)\;(K_+,K_-) 
\\
\interp{\judgeWK{\exists \alpha:\kappa.\;A}{\bigstar}}\;\theta\;(K_+,K_-) & = & 
    \Sigma \tau:\kappa.\; 
        \interp{\judgeWK[\Theta, \alpha:\kappa]{A}{\bigstar}}\;(\theta,\tau)\;(K_+,K_-) 
\\
\interp{\judgeWK{A \times B}{\bigstar}}\;\theta\;(K_+,K_-) & = & 
   \interp{\judgeWK{A}{\bigstar}}\;\theta\;(K_+,K_-) \times
   \interp{\judgeWK{B}{\bigstar}}\;\theta\;(K_+,K_-) 
\\
\interp{\judgeWK{A + B}{\bigstar}}\;\theta\;(K_+,K_-) & = & 
   \interp{\judgeWK{A}{\bigstar}}\;\theta\;(K_+,K_-) +
   \interp{\judgeWK{B}{\bigstar}}\;\theta\;(K_+,K_-) 
\\
\interp{\judgeWK{A \to B}{\bigstar}}\;\theta\;(K_+,K_-) & = & 
   \interp{\judgeWK{A}{\bigstar}}\;\theta\;(K_-,K_+) \to
   \interp{\judgeWK{B}{\bigstar}}\;\theta\;(K_+,K_-) 
\\
\interp{\judgeWK{\unittype}{\bigstar}}\;\theta\;(K_+, K_-) & = & \setof{*} 
\\

\interp{\judgeWK{\N}{\bigstar}}\;\theta\;(K_+, K_-) & = &  \N 
\\
% \interp{\judgeWK{\cont{A}}{\bigstar}}\;\theta\;(K_+,K_-) & = & 
%    \interp{\judgeWK{A}{\bigstar}}\;\theta\;(K_-,K_+) \to K_+
% \\
\interp{\judgeWK{\monad{A}}{\bigstar}}\;\theta\;(K_+,K_-) & = & 
   (\interp{\judgeWK{A}{\bigstar}}\;\theta\;(K_+,K_-) \to K_-) \to K_+
\\
\interp{\judgeWK{\reftype{A}}{\star}}\;\theta\;(K_+, K_-) & = & Loc(\interpset{\judgeWK{A}{\bigstar}}\;\theta)
\\
\end{array}
\end{displaymath}
\caption{Locally Continuous Functor Interpreting Polytypes}
\label{interp-polytypes}
\end{figure}

As explained earlier, the interpretation of the type constructor
context $\interp{\Theta}$ are the tuples of the interpretations of
each kind in the environment $\Theta$, which are in turn interpreted
as merely the closed canonical forms of that kind. (So $\star$ are
just the closed monotypes, $\star \to \star$ the closed lambda-terms
of that kind, and so on.)

This definition is also well-founded, since it is defined by a
structural recursion over the derivation of the kinding derivation of
$\judgeWK{A}{\bigstar}$.

Whenever we reach a variable or application case, we can apply the
substitution and invoke the interpretation function for the
monotypes. Universal types $\forall \alpha:\kappa.\;A$ are interpreted
as set-indexed predomains or dependent functions from the set of
closed canonical objects of the kind $\kappa$ into
predomains. Existential types $\exists \tau:\kappa.\;A$ are
interpreted as pairs or dependent sums: we pair a syntactic monotype
with the predomain interpreting the second component.

The remaining cases essentially repeat the clauses of the definitions
for monotypes, to allow for the possibility that there may be
sub-components of pairs/sums/functions/etc that contain universal or
existential types.

As a result, we need to verify that this definition is coherent, For
example, a type like $\N + \N : \bigstar$ can be derived two ways,
depending on whether the ``$+$'' is viewed as constructor combining
polytypes or monotypes. Fortunately, the lack of interesting structure
on polytypes makes this easy.

\begin{prop}{(Coherence of Polymorphic Type Interpretation)}
  For any two derivations of $D, D' :: \judgeWK{A}{\bigstar}$, we have that 
  $\interp{D} = \interp{D'}$. 
\end{prop}

\begin{proof}
  This follows via induction on $A$. For each $A$, there are at most two
rules from which it can be derived. 
\end{proof}


\subsection{Interpreting Heaps and Basic Continuations}

We are finally in a position to define heaps and the recursive domain
equation we would like to solve:

\begin{displaymath}
\begin{array}{lcl}
H(K_+,K_-) & = & \Sigma L \subseteq^{fin} Loc.\; 
                    \left(\Pi (l,A) \in L.\; 
                             \interp{\judgeWK[\cdot]{A}{\bigstar}}(\cdot)\;(K_+,K_-) 
                    \right)\\

\mathcal{K}(K_+,K_-) & = & H(K_-,K_+) \to O \\    
\end{array}
\end{displaymath}

We use $H$ to define what heaps mean. A heap is a dependent sum whose
first component is a finite set of allocated locations, together with
a map that takes each location in the set of allocated locations, and
returns a value of the appropriate type.

The continuation type is defined by the solution to the equation
$\mathcal{K}$, which of type $(CPO^{op}_\bot \times CPO_\bot) \to
CPO_\bot$. We know that this goes into $CPO_\bot$, since it defines a
map into the two point domain, and a function space into a domain is
itself a domain. So if we can solve the equation $K \simeq \mathcal{K}(K,
K)$, then we can plug $K$ into all our other definitions to interpret
all of the types in our language.

Filling this in, we can understand how basic continuations work: they
receive a heap, and then either loop or terminate. The monadic type
$\monad{\tau}$ now can be seen as the state-and-continuation monad,
which combines the effects of the continuation monad and the state
monad via a domain which looks like $(\tau \to H \to O) \to H \to O$.
(Though our model supports it, we will never actually use the possibly
of adding control operators to the language in this thesis.)

To show that this equation actually has a solution, it suffices to
show that $F$ is a locally-continuous functor. We will prove this by
induction over $\interp{-}$ and $\interpmono{-}$, and then at each
case appeal to a series of lemmas showing that each construction we
use preserves local continuity.

\section{Solving Domain Equations}

\subsection{Local Continuity}

A functor $F : CPO_\bot \times CPO^{op}_\bot \to CPO$ is \emph{locally
  continuous} if it preserves the order structure on its arguments.
That is, it must be monotone --- if $f \sqsubseteq f'$ and $g
\sqsubseteq g'$, then $F(f, g) \sqsubseteq F(f', g')$ --- and it
must also preserve limits --- given a pair of chains $f_i, g_i$, 
$\bigsqcup_i F(f_i, g_i) = F(\bigsqcup_i f_i, \bigsqcup_i g_i)$. 

Now, we will show that the functors we used earlier are all locally
continuous. All of this is standard, but I give the proofs to make the
presentation self-contained.

\begin{lemma}{Local Continuity}
\begin{enumerate}
\item If $F,G : CPO_\bot \times CPO^{op}_\bot \to CPO$ are locally continuous,
      then $\lambda A,B.\; F(A,B) \times G(A,B)$ is locally continuous.  
\item If $F,G : CPO_\bot \times CPO^{op}_\bot \to CPO$ are locally continuous,
      then $\lambda A,B.\; F(A,B) + G(A,B)$ is locally continuous.  
\item If $F,G : CPO_\bot \times CPO^{op}_\bot \to CPO$ are locally continuous,
      then $\lambda A,B.\;F(A,B) \to G(A,B)$ is locally continuous.  
\item The constant functor $K_C$ is locally continuous. 
\item If X is a set, and $F(x) : CPO_\bot \times CPO^{op}_\bot \to CPO$ is an
      $X$-indexed family of locally continuous functors, then 
      $\lambda (A,B).\; \Pi x:X.\;F(x)(A,B)$ is a locally continuous functor. 
\item If X is a set, and $F(x) : CPO_\bot \times CPO^{op}_\bot \to CPO$ is an
      $X$-indexed family of locally continuous functors, then 
      $\lambda (A,B).\;\Sigma x:X.\;F(x)(A,B)$ is a locally continuous functor. 
\end{enumerate}
\end{lemma}
 
\begin{proof}

We elide proofs of monotonicity, and give only the proofs of preservation of
limits. The notations $\left<f;g\right>$ and notation $[f;g]$ are the unique
mediating maps for products and sums, and mean:
\begin{displaymath}
\begin{array}{lcl}
  [f;g] & = & \fun{x}{A+B}{\left\{\begin{array}{ll}
                         f\;a & \mbox{when }x = \inl{\;a} \\
                         g\;b & \mbox{when }x = \inl{\;b} \\
                         \end{array}\right.} \\[1em]
  \left<f;g\right> & = & \fun{x}{A \times B}{(f\;x,g\;x)} 
\end{array}
\end{displaymath}
We also lift these to $n$-ary versions.

\begin{enumerate}
\item Suppose $F$ and $G$ are locally continuous. Now, for $A$ and $B$, we have 
the functor which takes $(A,B)$ to $F(A,B) \times G(A,B)$
on the object part, and which takes $(f,g)$ to $F(f,g) \times G(f,g)$ on
the arrow part.

Next, suppose that we have a chain of functions $\left<f_i\right> : A \to B$ and
$\left<g_i\right> : X \to Y$. Now, we calculate:

\begin{displaymath}
\begin{array}{lcl}
  \sqcup_i (F \times G (f_i,g_i)) 
   & = & \sqcup_i (F(f_i,g_i) \times G(f_i,g_i)) \\
   & = & \sqcup_i \left<F(f_i,g_i) \circ \pi_1; 
                        G(f_i,g_i) \circ \pi_2\right>\\
   & = & \left<\sqcup_i (F(f_i,g_i) \circ \pi_1);
               \sqcup_i (G(f_i,g_i) \circ \pi_2)\right> \;\;\;\;(*)\\
   & = & \left<\sqcup_i F(f_i,g_i) \circ \sqcup_i \pi_1;
               \sqcup_i G(f_i,g_i) \circ \sqcup_i \pi_2\right>\\
   & = & \left<\sqcup_i F(f_i,g_i) \circ \pi_1);
               \sqcup_i G(f_i,g_i) \circ \pi_2)\right>\\
   & = & (\sqcup_i F(f_i,g_i)) \times (\sqcup_i G(f_i,g_i))\\
\end{array}
\end{displaymath}

The interesting step is marked with (*); it is justified by the fact
that we know that $\pi_j \circ (\sqcup_i \left<h^1_i;h^2_i\right>) = 
\sqcup_i (\pi_j \circ \left<h^1_i;h^2_i\right>) = 
\sqcup_i h^j_i$,
and that $\pi_j \circ \left<\sqcup_i h^1_i; \sqcup_i h^2_i\right> = \sqcup_i h^j_i$,
and that the mediating morphism is unique. 


\item Suppose $F$ and $G$ are locally continuous. Now, for $A$ and $B$
we have the functor which takes $(A,B)$ to $F(A,B) + G(A,B)$
on the object part, and which takes $(f,g)$ to $F(f,g) + G(f,g)$ on
the arrow part.

Next, suppose that we have a chain of functions $\left<f_i\right> : C \to A$ and
$\left<g_i\right> : B \to D$. Now, we calculate:

\begin{displaymath}
\begin{array}{lcl}
  \sqcup_i (F + G (f_i,g_i)) 
   & = & \sqcup_i (F(f_i,g_i) + G(f_i,g_i)) \\
   & = & \sqcup_i \left[\inl \circ F(f_i,g_i); 
                        \inr \circ G(f_i,g_i)\right]\\
   & = & \left[\inl \circ \sqcup_i (F(f_i,g_i));
               \inr \circ \sqcup_i (G(f_i,g_i))\right] \;\;\;\;(*)\\
   & = & \left[\sqcup_i \inl \circ F(f_i,g_i);
               \sqcup_i \inr \circ G(f_i,g_i) \right]\\
   & = & \left[\inl \circ \sqcup_i F(f_i,g_i));
               \inr \circ \sqcup_i G(f_i,g_i))\right]\\
   & = & (\sqcup_i F(f_i,g_i)) + (\sqcup_i G(f_i,g_i))\\
\end{array}
\end{displaymath}

The interesting step is marked with (*); it is justified by the fact
that we know that $(\sqcup_i \left[h^1_i;h^2_i\right]) \circ
\mathsf{in}_j = 
\sqcup_i (\left[h^1_i;h^2_i\right] \circ \mathsf{in}_j) = 
\sqcup_i h^j_i$, and that $\left<\sqcup_i h^1_i; \sqcup_i
h^2_i\right> \circ \mathsf{in}_j = \sqcup_i h^j_i$, and that the mediating
morphism is unique.

\item Suppose $F$ and $G$ are locally continuous. Now, for $A$ and $B$
we have the functor which takes $(A,B)$ to $F(B,A) \to G(A,B)$
on the object part, and which takes $(f,g)$ to $F(g,f) \to G(f,g)$ on
the arrow part.
 
Next, suppose that we have a chain of functions $\left<f_i\right> : A \to C$ 
and $\left<g_i\right> : D \to B$. Now, we calculate:
 
\begin{displaymath}
\begin{array}{lcl}
\sqcup_i [F \to G](f_i,g_i) & = & F(g_i,f_i) \to G(f_i,g_i) \\
& = & 
  \sqcup_i \lambda h.\; [G(f_i,g_i) \circ h \circ F(g_i, f_i)] \\
& = & 
  \lambda h.\; \sqcup_i [G(f_i,g_i) \circ h \circ F(g_i, f_i)]  \\
& = & 
  \lambda h.\; [(\sqcup_i G(f_i,g_i)) \circ h \circ (\sqcup_i F(g_i, f_i))] \\
& = & 
  \lambda h.\; [G(\sqcup_i f_i, \sqcup_ig_i) \circ h \circ 
                F(\sqcup_i g_i, \sqcup_i f_i)] \\
\end{array}
\end{displaymath}

% The interesting step is at (*). 



% F(g : D -> B, f : A -> C) : F(D,C) -> F(B,A)
% G(f : A -> C, g : D -> B) : G(A,B) -> G(C,D)
% 
% F(g,f) : F(D,C) -> F(B,A)
% G(f,g) : G(A,B) -> G(C,D)
% 
% F(g,f) -> G(f,g) : (F(B,A) -> G(A,B)) -> F(D,C) -> G(C,D)
%                  = \lambda h. G(f,g) \circ h \circ F(g,f)
% 





\item The constant functor is locally continuous because it maps all 
morphisms to the identity morphism, and hence trivially preserves 
limits. 

\item Suppose $X$ is a set, and $F(x)$ is an $X$-indexed family of
locally continuous functors. 

First, given objects $(A,B)$, we have the object part of this as the
dependent function space $\Pi x:X.\; F(x)(A,B)$, with elements $u
\sqsubseteq v$ if and only if for all $x \in X$, $u(x)
\sqsubseteq_{F(x)(A,B)} v(x)$. 

This object is a true product over $X$. Given $\Pi x:X.\; F(x)(A,B)$,
we can define the $x$-th projection as $\pi_x = \lambda f.\;
f(x)$. Then, it's clear that given a family of morphisms $f_{x\in X} : Y \to
F(x)(A,B)$, we can define a function $\left<f_{x\in X}\right> : Y \to \Pi
x:X.\;F(x)(A,B) = \lambda y.\; \lambda x.\; f_x(y)$, which means that
for all $x$, and that $\pi_x \circ \left<f_{x\in X}\right> = f_x$.

We show uniqueness by supposing that that there is some $g$ such
that $\pi_x \circ g = f_x$. Then, we know that 
$g = \lambda x.\; \pi_x \circ g$, which means that $g = \lambda x.\; f_x$,
which is exactly $\left<f_{x\in X}\right>$. 


Next, given morphisms $f \in A \to C$ and $g \in D \to B$, we have
$[\Pi x:X.\; F(x)](f,g) \in [\Pi x:X.\; F(x)](A,B) \to [\Pi x:X.\; F(x)](C,D)$
as:
  
\begin{displaymath}
  [\Pi x:X.\; F(x)](f,g) = \lambda x.\; F(x)(f,g)
\end{displaymath}

Clearly, this preserves identities and composition, and is hence a
functor. 

Now, suppose that $(f,g) \sqsubseteq (f',g')$, and that $x$ is an
arbitrary element of $X$. Then $[\Pi x:X.\; F(x)](f,g) = F(x)(f,g)$
and $[\Pi x:X.\; F(x)](f',g') = F(x)(f',g')$. Since $F(x)$ is a
locally continuous functor, we know that $F(x)(f,g) \sqsubseteq
F(x)(f',g')$, and so $[\Pi x:X.\; F(x)]$ preserves ordering.

Now, suppose that $(f_i,g_i)$ form a chain. So, we know that 

\begin{displaymath}
\begin{array}{lcl}
\sqcup_i ([\Pi x:X.\; F(x)](f_i,g_i)) & = & 
  \sqcup_i (\lambda x:X.\; (F(x) (f_i, g_i))) \\
& = & 
  \lambda x:X.\; (\sqcup_i (F(x) (f_i, g_i))) \;\;\;\;(*) \\
& = & 
  \lambda x:X.\; (F(x) (\sqcup_i (f_i, g_i))) \\
& = & 
[\Pi x:X.\; F(x)] (\sqcup_i (f_i,g_i)) \\
\end{array}
\end{displaymath}

The interesting step is (*); it is justified by the fact
that we know that $\pi_x \circ \sqcup_i \left<h^x_i\right> =  
\sqcup_i (\pi_x \circ_i \left<h^x_i\right>) = 
\sqcup_i h^x_i$, 
and that $\pi_x \circ \left<\sqcup_i h^x_i\right> 
          = \sqcup_i h^x_i$, 
and that the mediating morphism is unique.

As a result, we can conclude that this functor is locally 
continuous.

\item Suppose $X$ is a set, and $F(x)$ is an $X$-indexed family
of locally continuous functors. 

First, given objects $(A,B)$, we have the object part of the functor
yielding the dependent sum $\Sigma x:X.\; F(x)(A,B)$. Ordering is
given pairwise, equipping the set $X$ with the trivial ordering. 
That is $(x,o) \sqsubseteq (x', o')$ if and only if $x = x'$ and 
$o \sqsubseteq_{F(x)(A,B)} o'$. 

This is a true coproduct over $X$. Given $\Sigma x:X.\; F(x)(A,B)$, we
can define the injections $\inj{x} \in F(x)(A,B) \to \Sigma x:X.\;
F(x)(A,B)$ as $\lambda v. (x,v)$.  Next, suppose we have a family of
functions $f_x : F(x)(A,B) \to Y$.  We can define a function $[f_{x\in X}]
\in (\Sigma x:X.\; F(x)(A,B)) \to Y$ as $\lambda (x,v).\;
(f_x\;v)$. It's clear that $[f_{x\in X}]\circ \inj{i} = f_i$.

Finally, we can establish uniqueness as follows. Suppose that there 
is a $g$ such that $g \circ \inj{i} = f_i$. Next, we know that 
$g = \lambda (x,v).\;g (x, v) = \lambda (x,v).\;(g\circ\inj{x})(v)$,
which is clearly $\lambda (x,v).\; (f_x\;v)$, which is just $[f_{x \in X}]$


Next, given morphisms $f \in A \to C$ and $g \in D \to B$, we have
$[\Sigma x:X.\; F(x)](f,g) \in [\Pi x:X.\; F(x)](A,B) \to [\Sigma x:X.\; F(x)](C,D)$
as:

\begin{displaymath}
  [\Sigma x:X.\; F(x)](f,g) = \lambda (x, v).\; (x, F(x)(f,g)(v))
\end{displaymath}

This clearly preserves identities and composition, and hence defines
a functor. 

Now, suppose that $(f,g) \sqsubseteq (f',g')$ and that $(x,v)$ is
an element of $\Sigma x:X.\; F(x)(A,B)$. Then we have that
$[\Sigma x:X.\;F(x)](f,g)](x,v) = (x,F(x)(f,g)(v))$ and that
$[\Sigma x:X.\;F(x)](f',g')](x,v) = (x,F(x)(f',g')(v))$. So we
know that $x=x$, and by the local continuity of $F(x)$, we know
that $F(x)(f,g)(v) \sqsubseteq F(x)(f',g')(v)$. So this functor
preserves ordering. 

Finally, suppose $(f_i, g_i)$ form a chain. 

\begin{displaymath}
\begin{array}{lcl}
\sqcup_i [\Sigma x:X.\;F(x)](f_i,g_i) 
& = & \sqcup_i \lambda (x,v).\; (F(x)(f_i,g_i)(v)) \\
& = & \lambda (x,v). \sqcup_i (F(x)(f_i,g_i)(v)) \;\;\;\; (*) \\
& = & \lambda (x,v).\; F(x)(\sqcup_i f_i, \sqcup_i g_i))(v) \\
& = & [\Sigma x:X.\;F(x)](\sqcup_i f_i, \sqcup_i g_i) \\
\end{array}
\end{displaymath}

The interesting step is (*); it is justified by the fact
that we know that 
$(\sqcup_i [h^x_i]) \circ \inj{x} =  
\sqcup_i ([h^x_i] \circ \inj{x}) =  \sqcup_i h^x_i$, 
and that $[\sqcup_i h^x_i] \circ \inj{x} = \sqcup_i h^x_i$, 
and that the mediating morphism is unique.
\end{enumerate}
\end{proof}

\subsection{The Inverse Limit Construction}

Once we have a locally continuous functor, we would like to find a
solution to the fixed point equation it defines.

\begin{prop}{(Smyth and Plotkin)}
Any locally-continuous functor $F : CPO_\bot \times CPO^{op}_\bot \to
CPO_\bot$ has a solution to the equation $X \cong F(X, X)$. Moreover, there
is also a \emph{minimal} isomorphism solving this equation. 
\end{prop}

The existence of a solution follows from Scott's inverse limit
construction, together with Smyth and Plotkin's characterization of
such solution~\cite{smyth-plotkin}. We give an explicit construction
of their solution in the following subsection, in which we will always
take $F$ to be a locally-continuous functor of the type mentioned
above.

\subsubsection{Embeddings and Projections}

First, recall that an \emph{embedding} $e : C \to D$ between pointed
CPOs is a continuous function such that there exists a function $p : D
\to C$ (called a \emph{projection}) with the properties that $p \circ
e = id_C$ and $e \circ p \sqsubseteq id_D$. 

Now, we'll introduce the category $CPO_\bot^{O}$, which is the
category whose objects are the pointed domains, and whose morphisms
from $D$ to $E$ are the embedding-projection pairs. The identity
morphism from domain $D$ to $D$ is the pair $\left<id, id\right>$, and
the composition operation on $\left<e, p\right>$ and $\left<e',
p'\right>$ is $\left<e' \circ e, p \circ p'\right>$. To verify that
this is indeed a category, we check that:


\begin{itemize}
\item The identity $\left<id, id\right> : D \to D$ is an 
embedding-projection pair because $id \circ id = id$ and 
$id \sqsubseteq id$. 

\item The composition $\left<e, p\right> \circ \left<e',
p'\right>$ is an embedding-projection pair because it 
is defined to be equal to $\left<e \circ e', p' \circ p\right>$, 
and we have that embedding followed by projection is:

\begin{displaymath}
  \begin{array}{lcl}
    (p' \circ p) \circ (e \circ e') 
       & = & p' \circ (p \circ e) \circ e' \\ 
       & = & p' \circ id \circ e' \\ 
       & = & p' \circ e' \\
       & = & id \\                 
  \end{array}
\end{displaymath}

and likewise we have for a projection followed by an embedding:

\begin{displaymath}
  \begin{array}{lcll}
    (e \circ e')  \circ (p' \circ p) 
       & = & e \circ (e' \circ p') \circ p \\
       & \sqsubseteq & e \circ id \circ p \\ 
       & \sqsubseteq & e \circ p \\
       & \sqsubseteq & id \\                 
  \end{array}
\end{displaymath}

\item Finally, it's clear that composition is associative and has
  identities as units because it inherits these properties from the
  underlying composition operations.
\end{itemize}


Now, consider the one-point domain $\emptyset_\bot = \setof{\bot}$,
and the sequence of domains $X_i$, defined inductively by $X_0 =
\emptyset_\bot$ and $X_{i+1} = F(X_i, X_i)$. Next, we will define
embeddings and projections $e_i : X_i \to X_{i+1}$ and $p_i : X_{i+1}
\to X_i$ as follows:

\begin{displaymath}
  \begin{array}{llcl}
    e_0     & : X_0 \to X_1 & = &\lambda x.\; \bot \\
    e_{i+1} & : X_{i+1} \to X_{i+2} & = & F(e_i, p_i) \\[0.5em]
    p_0 &  : X_1 \to X_0 & = & \lambda x.\; \bot \\
    p_{i+1} & : X_{i+2} \to X_{i+1} & = & F(p_i, e_i) \\[0.5em]
  \end{array}
\end{displaymath}

\begin{lemma}{(Embeddings and Projections)} Each $\left<e_i, p_i\right>$ forms
an arrow from $X_i$ to $X_{i+1}$ in $CPO_\bot^O$.
\end{lemma}

\begin{proof}
This proof proceeds by induction on $i$. 
\begin{itemize}
\item Case $i=0$: Obviously $e_0 \circ p_0 = id$, since $X_0 = \setof{\bot}$.  
  Likewise, since $p_0(e_0(x)) = \bot$, and $\bot \sqsubseteq x$, it follows that
  $p_0 \circ e_0 \sqsubseteq id$. 

\item Case $i=n+1$: 

  First, we'll show that $e_i \circ p_i$ is the identity: 

  \begin{displaymath}
    \begin{array}{lcll}
      e_i \circ p_i & = & e_{n+1} \circ p_{n+1}            & \mbox{Def.}\\ 
                    & = & F(e_n, p_n) \circ F(p_n, e_n)   & \mbox{Def.}\\ 
                    & = & F(e_n \circ p_n, e_n \circ p_n) & \mbox{Functor property}\\ 
                    & = & F(id, id)                       & \mbox{Ind. hyp.}\\  
                    & = & id                              & \mbox{Functor property} \\
    \end{array}
  \end{displaymath}

  \noindent Now, we'll show that $p_i \circ e_i \sqsubseteq id$: 

  \begin{displaymath}
    \begin{array}{lcll}
      p_i \circ e_i & = & p_{n+1} \circ e_{n+1}            & \mbox{Def.}\\ 
                    & = & F(p_n, e_n) \circ F(e_n, p_n)   & \mbox{Def.}\\ 
                    & = & F(p_n \circ e_n, p_n \circ e_n) & \mbox{Functor property}\\ 
    \end{array}
  \end{displaymath}

  \noindent By induction, we know that $p_n \circ e_n \sqsubseteq id$, and
  because locally continuous functors are also monotone, we know that 
  $F(p_n \circ e_n, p_n \circ e_n) \sqsubseteq F(id, id) \equiv id$.
\end{itemize}
\end{proof}

\ \\

\subsubsection{Construction of the Domain}

Now, we'll define the domain $X$ to be the domain with the underlying set:
\begin{displaymath}
X \equiv \left\{ x \in (\Pi n:\N.\; X_n) \;|\; \forall m:\N.\; x_m = p_m(x_{m+1}) \right\}
\end{displaymath}
with the ordering being the usual component-wise ordering. (As a
notational convenience, we will write $x_n$ to indicate the $n$-th
component of $x$, or $x(n)$.)  To be in $CPO_\bot$, it needs a least
element, which is just $\lambda n:\N.\;\bot$.  

We claim that this pointed CPO $X$ is the colimit of the chain of
domains $X_i$ in $CPO_\bot^O$. To prove it, we must proceed in two
stages.

\begin{lemma}{($X$ is a cocone)} $X$ is a cocone of the
diagram $X_0 \longrightarrow X_1 \longrightarrow \ldots$. 
\end{lemma}


\begin{proof}
To show this, we must give morphisms $\left<\hat{e}_i,
\hat{p}_i\right> : X_i \to X$. To do so, we'll define:

\begin{displaymath}
  \hat{e}_n : X_n \to X \equiv 
    \lambda x:X_n.\; \lambda m:\N.\; 
       \left\{ 
          \begin{array}{ll}
            p_{m,n}(x) & \mbox{if } m < n \\
            x         & \mbox{if } m = n \\
            e_{n,m}(x) & \mbox{if } m > n \\
          \end{array}
       \right.
\end{displaymath}

We define $e_{i,j}$ to be the composition $e_{j-1} \circ e_{j-2} \circ \ldots \circ e_i$,
which will have the type $X_i \to X_j$. Likewise, we define $p_{i,j}$ to be the 
composition $p_i \circ \ldots \circ p_{j-1}$, which will have the type $X_j \to X_i$. 

\noindent The projection $\hat{p}_n : X \to X_n$ is much simpler. It's just

\begin{displaymath}
  \hat{p}_n : X \to X_n \equiv \lambda x:X.\; x_n
\end{displaymath}

Now, we'll verify that these do form an embedding-projection pair. 

  \begin{itemize}
  \item First, we'll show that $\hat{p}_n \circ \hat{e}_n = id$. 

    \begin{displaymath}
      \begin{array}{lcl}
        \hat{p}_n \circ \hat{e}_n & = & \lambda x:X_n.\; (\hat{p}_n \circ \hat{e}_n)\;x \\
                                  & = & \lambda x:X_n.\; \hat{p}_n(\hat{e}_n \;x) \\
                                  & = & \lambda x:X_n.\; (\hat{e}_n\;x)\;n \\
                                  & = & \lambda x:X_n.\; x \\
                                  & = & id \\
      \end{array}
    \end{displaymath}

  \item Now, we'll show that $\hat{e}_n \circ \hat{p}_n \sqsubseteq id$. 
    
    \begin{displaymath}
      \begin{array}{lcl}
        \hat{e}_n \circ \hat{p}_n & = & \lambda x:X.\; (\hat{e}_n \circ \hat{p}_n) \; x \\
                                  & = & \lambda x:X.\; \hat{e}_n (\hat{p}_n \; x) \\ 
                                  & = & \lambda x:X.\; \hat{e}_n (x_n) \\ 
      \end{array}
    \end{displaymath}

    Now, when applied to an argument $x \in X$, it's clear that the
    result element is component-wise equal to $x$ for the components
    less than or equal to $n$, and less than that for components
    bigger than $n$, which makes the result smaller than $x$.
  \end{itemize}
This establishes that there are morphisms $\left<\hat{e}_i,
\hat{p}_i\right> : X_i \to X$. Now, we need 1) to show that the 
equation $\left<\hat{e}_i, \hat{p}_i\right> : X_i \to X = 
\left<\hat{e}_{i+1}, \hat{p}_{i+1}\right> \circ \left<e_i, p_i\right>$ 
holds, and 2) that $\bigsqcup_i \hat{e}_i \circ \hat{p}_i = id$, 
which will establish that the diagram commutes appropriately. 

Expanding the definition of composition, we want to show that 
$\left<\hat{e}_i, \hat{p}_i\right> = \left<\hat{e}_{i+1} \circ e_i,
                                          p_i \circ \hat{p}_{i+1} \right>$.
So, we have that
\begin{displaymath}
  \begin{array}{lcl}
    \hat{e}_{i+1} \circ e_i
      & = &     
      \lambda x:X_i.\; \lambda m:\N.\; 
       \left\{ 
          \begin{array}{ll}
            p_{m,{i+1}}(e_i\;x) & \mbox{if } m < i+1 \\
            e_i\;x         & \mbox{if } m = i+1 \\
            e_{{i+1},m}(e_i\;x) & \mbox{if } m > i+1 \\
          \end{array}
       \right. 
   \\[1em]
      & = &     
      \lambda x:X_i.\; \lambda m:\N.\; 
       \left\{ 
          \begin{array}{ll}
            p_{m,i}(p_i(e_i\;x)) & \mbox{if } m < i+1 \\
            e_{i, i+1}\;x         & \mbox{if } m = i+1 \\
            e_{{i+1},m}(e_i\;x) & \mbox{if } m > i+1 \\
          \end{array}
       \right. 
   \\[1em]
     & = & 
      \lambda x:X_i.\; \lambda m:\N.\; 
       \left\{ 
          \begin{array}{ll}
            p_{m,i}(x) & \mbox{if } m < i+1 \\
            e_{i,m}(x) & \mbox{if } m > i \\
          \end{array}
       \right. 
   \\[1em]
     & = & 
      \lambda x:X_i.\; \lambda m:\N.\; 
       \left\{ 
          \begin{array}{ll}
            p_{m,i}(x) & \mbox{if } m < i \\
            x             & \mbox{if } m = i \\
            e_{i,m}(x) & \mbox{if } m > i \\
          \end{array}
       \right. 
   \\[1em]
     & = & \hat{e}_i \\
  \end{array}
\end{displaymath}

\noindent In the other direction, we show that 
\begin{displaymath}
  \begin{array}{lcl}
    p_i \circ \hat{p}_{i+1} & = & \lambda x:X.\; p_i(x_{i+1}) \\ 
                           & = & \lambda x:X.\; x_i \\
                           & = & \hat{p}_i \\
  \end{array}
\end{displaymath}
The second step follows from the definition of $X$. 

Now, we need to show that $\bigsqcup_i \hat{e}_i \circ \hat{p}_i = id$. 

\begin{displaymath}
  \begin{array}{lcl}
    \bigsqcup_i \hat{e}_i \circ \hat{p}_i 
       & = & 
       \bigsqcup_i \lambda x:X.\; \hat{e}_i(\hat{p}_i \;x)
    \\
       & = & 
       \bigsqcup_i \lambda x:X.\; \hat{e}_i(x_i)
    \\
       & = & 
       \bigsqcup_i \lambda x:X.\;\lambda m:\N.\; 
          \left\{ 
          \begin{array}{ll}
            p_{m,i}(x_i) & \mbox{if } m < i \\
            x_i         & \mbox{if } m = i \\
            e_{i,m}(x_i) & \mbox{if } m > i \\
          \end{array}
       \right.
    \\
       & = & 
       \bigsqcup_i \lambda x:X.\;\lambda m:\N.\; 
          \left\{ 
          \begin{array}{ll}
            x_m         & \mbox{if } m \leq i \\
            e_{i,m}(x_i) & \mbox{if } m > i \\
          \end{array}
       \right.
    \\
       & = & 
       \bigsqcup_i \lambda x:X.\;\lambda m:\N.\; 
          \left\{ 
          \begin{array}{ll}
            x_m         & \mbox{if } m \leq i \\
            e_{i,m}(p_{i,m}\;x_m) & \mbox{if } m > i \mbox{\;\;\;}(*)\\
          \end{array}
       \right.
    \\
       & = & 
       \lambda x:X.\;\lambda m:\N.\; x_m
    \\
       & = & 
       id
  \end{array}
\end{displaymath}

In $(*)$, we use the definition of $X$ to see that the components
greater than $i$ are smaller than $x$'s component at that index. So
for each given $i$, the first $i$ components of $\hat{e}_i \circ
\hat{p}_i$ are the identity function, and below the identity for
anything bigger than that. Thus, the limit as $i$ goes to infinity is
the identity function for all components. 


\end{proof}

To show that $X$ is the colimit of this diagram, we need to show there
is a unique map from it to any other cocone. 

\begin{lemma}{(Universality of $X$)} 
Suppose that there is a $Y$ with morphisms $\left<f_n, q_n\right> : X_n
\to Y$ and $q_n : Y \to X_n$, forming a cocone over $X_0 \longrightarrow
X_1 \longrightarrow \ldots$. Then, there is a unique 
$\left<h_e,h_p\right> : X \to Y$ such that for all $n$, 
$\left<f_n,q_n\right> = \left<h_e,h_p\right> \circ \left<\hat{e}_n, \hat{p}_n\right>$.
\end{lemma}

\begin{proof}
To show this, we need to explicitly construct $h_e$ and $h_p$, and show
that they form an embedding-projection pair. We'll define $h_e : X \to Y = 
\bigsqcup_i f_i \circ \hat{p}_i$, and define 
$h_p : Y \to X = \lambda y:Y.\; \lambda i:\N.\; q_i\; y$.

Before we can proceed any further, we need to establish that $h_e$
actually defines a morphism --- that is, we have to establish that
$f_i \circ \hat{p}_i$ is a chain in $i$. So, assume we have some
arbitrary $i$, and some arbitrary $x : X$.

\begin{enumerate}
\item Now, by the properties of embedding-projection pairs, we know
$e_i(p_i\;x_{i+1}) \sqsubseteq x_{i+1}$, 
\item By the continuity of $f_{i+1}$, 
this means $f_{i+1}(e_i(p_i\;x_{i+1})) \sqsubseteq f_{i+1}(x_{i+1})$. 
\item By the fact that $Y$ is a cocone, this
means $f_i(p_i\;x_{i+1})
\sqsubseteq f_{i+1}(x_{i+1})$. 
\item By the
definition of $X$, this is 
the same as showing $f_i(x_i) \sqsubseteq f_{i+1}(x_{i+1})$. 
\item By the definition of $\hat{p}$, this is the same as 
   $f_i(\hat{p}_i\;x) \sqsubseteq f_{i+1}(\hat{p}_{i+1}\;x)$. 
\item Since this holds for all $x$, we have 
shown $f_i \circ \hat{p}_i \sqsubseteq f_{i+1} \circ \hat{p}_{i+1}$. 
\end{enumerate}

Next, let's establish that $h_e$ and $h_p$ form an embedding-projection
pair. To show that $h_e \circ h_p \sqsubseteq id$, we use equational reasoning:

\begin{displaymath}
  \begin{array}{lcl}
    h_e \circ h_p  
     & = & 
      (\bigsqcup_i f_i \circ \hat{p}_i) \circ (\lambda y:Y.\; \lambda i:\N.\; (q_i\; y)) 
\\
      & = & 
      \bigsqcup_i (f_i \circ \hat{p}_i \circ (\lambda y:Y.\; \lambda i:\N.\; (q_i\; y)) 
\\      
%      & = & 
%      \bigsqcup_i \lambda y:Y.\;(f_i \circ \hat{p}_i \circ (\lambda y:Y.\; \lambda i:\N.\; (q_i\; y))\;y 
% \\
      & = & 
      \bigsqcup_i (\lambda y:Y.\;f_i(\hat{p}_i\;(\lambda i:\N.\; (q_i\; y))))
\\
      & = & 
      \bigsqcup_i (\lambda y:Y.\;f_i(q_i\; y))
\\
      & \sqsubseteq & 
      \bigsqcup_i \lambda y:Y.\;y
\\
      & \sqsubseteq & 
      \lambda y:Y.\;y
\\
  \end{array}
\end{displaymath}

In other direction, we need to show $h_p \circ h_e = id$.

\begin{displaymath}
  \begin{array}{lcl}
    h_p \circ h_e
     & = &  
      (\lambda y:Y.\; \lambda j:\N.\; (q_j\; y)) \circ (\bigsqcup_i f_i \circ \hat{p}_i)
\\
     & = &  
      \bigsqcup_i ((\lambda y:Y.\; \lambda j:\N.\; (q_j\; y)) \circ f_i \circ \hat{p}_i)
\\
     & = &  
      \lambda x:X.\; \bigsqcup_i ((\lambda y:Y.\; \lambda j:\N.\; (q_j\; y)) \circ f_i \circ \hat{p}_i))\;x
\\
     & = &  
      \lambda x:X.\; \bigsqcup_i ((\lambda y:Y.\; \lambda j:\N.\; (q_j\; y))\;(f_i\;(\hat{p}_i\;x)))
\\   
     & = &  
      \lambda x:X.\; \bigsqcup_i (\lambda j:\N.\; (q_j\; (f_i\;(\hat{p}_i\;x))))
\\   
     & = &  
      \lambda x:X.\; \lambda j:\N.\; \bigsqcup_i ((q_j\; (f_i\;(\hat{p}_i\;x))))
  \end{array}
\end{displaymath}

To finish this calculation, consider an arbitrary $x:X$ and $j:\N$. Now, consider
the tail of the chain, where $i > j$. Now, since we know that $q_k = p_k \circ q_{k+1}$, 
it follows that:

\begin{displaymath}
  \begin{array}{lcl}
   q_j\; (f_i\;(\hat{p}_i\;x))
     & = & p_{j,i}(q_i\; (f_i\;(\hat{p}_i\;x))) \\
     & = & p_{j,i}(\hat{p}_i\;x) \\
     & = & \hat{p}_j\;x \\
     & = & x_j \\
  \end{array}
\end{displaymath}

Which means that the least upper bound of the chain has to be 
$\lambda x:X.\;\lambda j:\N.\;x_j$ -- which means that it is the identity.


So we have established that $\left<h_e,h_p\right>$ is a morphism
between $X$ and $Y$. Next, let's see whether it commutes:
$\left<f_n,q_n\right> = \left<h_e,h_p\right> \circ \left<\hat{e}_n,
\hat{p}_n\right>$. Unfolding the definition of composition, we 
get two proof obligations. First, 

\begin{displaymath}
\begin{array}{lcl}
   h_e \circ \hat{e}_n 
   & = & 
     (\bigsqcup_i f_i \circ \hat{p}_i) \circ \hat{e}_n 
\\
   & = & 
     \bigsqcup_i (f_i \circ \hat{p}_i \circ \hat{e}_n)
\\
   & = & 
     \lambda x:X_n.\; \bigsqcup_i (f_i (\hat{p}_i \; (\hat{e}_n\;x)))
\\
   & = & 
     \lambda x:X_n.\; \bigsqcup_i (f_i (\hat{e}_n\;x\;i))
\\
   & = & 
     \lambda x:X_n.\; \bigsqcup_i f_i 
       \left(\left\{ 
          \begin{array}{ll}
            p_{i,n}(x) & \mbox{if } i < n \\
            x         & \mbox{if } i = n \\
            e_{n,i}(x) & \mbox{if } i > n \\
          \end{array}
       \right.\right)
\end{array}
\end{displaymath}

To find the limit of this chain, consider any $i > n$. Because $f_{k+1} \circ e_k = f_k$, 
we can see that $f_i(e_{n,i}\;x) = f_n\;x$, which means that the limit is $f_n\;x$, and
hence $h_e \circ \hat{e}_n = f_n$. 

Next, consider $\hat{p}_n \circ h_p$: 

\begin{displaymath}
\begin{array}{lcl}
   \hat{p}_n \circ h_p
   & = & 
     \hat{p}_n \circ (\lambda y:Y.\;\lambda i:\N.\;(q_i\;y))
\\
   & = & 
     \lambda y:Y.\; \hat{p}_n (\lambda i:\N.\;(q_i\;y))
\\
   & = & 
     \lambda y:Y.\; (q_n\;y)
\\
   & = & 
     q_n 
\\
\end{array}
\end{displaymath}

At this point, we have established that $X$ is a weak colimit --
there's a morphism from it to any other cone, but we still have yet to
show that it is a unique morphism. So, suppose that we have some 
other mediating morphism $\left<h'_e, h'_p\right> : X \to Y$. 

For the embedding $h'_e$, we proceed as follows:

\begin{enumerate}
\item Now, 
it must be the case that $\left<h'_e, h'_p\right> \circ \left<\hat{e}_n, \hat{p}_n\right> = 
\left<f_n, q_n\right>$. 

\item So $h'_e \circ \hat{e}_n = f_n$. 

\item Composing both sides with $\hat{p}_n$, we get 
   $h'_e \circ \hat{e}_n \circ \hat{p}_n = f_n \circ \hat{p}_n$. 

\item Taking limits of chains on both sides, we get $h'_e \circ \bigsqcup_n \hat{e}_n \circ{p}_n = \bigsqcup_n f_n \circ \hat{p}_n$

\item Simplifying, we get $h'_e = h_e$.
\end{enumerate}

For the projection $h'_p$, we have

\begin{enumerate}
\item We have $\hat{p}_n \circ h'_p = q_n$. 
\item Composing on both sides with $\hat{e}_n$, we have $\hat{e}_n \circ \hat{p}_n \circ h'_p = \hat{e}_n \circ q_n$. 
\item Taking limits on both sides, we have $h'_p = \bigsqcup_n \hat{e}_n \circ q_n$. 
\item Simplifying the limit expression, we get $h'_p = \lambda y:Y.\;\lambda n:\N.\; q_n(y)$. 
\item So $h'_p = h_p$. 
\end{enumerate}

\end{proof}


\subsubsection{Showing $X$ is a Solution to $F(X, X) \cong X$}

\begin{lemma}{(X is a fixed point)} The isomorphism $F(X, X) \cong X$
is valid.
\end{lemma}
\begin{proof}
First, note that applying $F$ to each of the $X_i$ and $\left<e_i,
p_i\right>$ yields $X_{i+1}$ and $\left<e_{i+1}, p_{i+1}\right>$.  In
other words, applying $F$ to our old diagram gives us the same thing
as before, only with the first element chopped off.

Therefore, $X$ is still a colimit for this diagram, because if we
replicate the colimit construction for this diagram, we can establish
an isomorphism between the ``new'' construction and $X$, since the
leading elements of the infinite product are determined by the
requirement that $x_i = p_i(x_{i+1})$.

Since $F$ is a locally continuous functor, it preserves colimits of
chains $\bot \longrightarrow F(\bot) \longrightarrow \ldots$, so $F(X,
X)$ is itself a colimiting object. 

Since colimits are unique up to isomorphism, it follows that $F(X, X)
\cong X$. 
\end{proof}

\section{Solving Our Recursive Domain Equation}

Given that we know that the basic operations we use in our
interpretation are locally continuous, we can show that our
interpretation function gives rise to a locally continuous functor.

\begin{lemma}{Functoriality of $\interpmono{-}$ and $\interp{-}$}
  \begin{enumerate}
  \item For all canonical derivations $\judgeWK[\cdot]{\tau}{\star}$, 
    $\interpmono{\judgeWK[\cdot]{\tau}{\star}}$ is locally continuous. 
  \item For all canonical derivations $\judgeWK[\cdot]{A}{\bigstar}$, 
    $\interp{\judgeWK[\cdot]{A}{\bigstar}}$ is locally continuous. 
  \item $H$ is a locally continuous functor
  \item $\mathcal{K}$ is a locally continuous functor. 
  \end{enumerate}
\end{lemma}

\begin{proof}
 The proof of the first case follows by structural
induction on the canonical derivations of monotypes. This is then used
as a lemma in the proof of the second case, which is done via a
structural induction on the canonical derivations of polytypes. This
then lets us prove the third case, that $H$ is a locally-continuous
functor, because we can work from the inside out, using the fact that
set-indexed sums and products of locally-continuous functors are
themselves locally continuous. Finally, since $\mathcal{K}$ is just 
$H \to O$, we know it is locally continuous also. 
\end{proof}

\ \\\noindent Observe that $\mathcal{K}$, applied to any arguments,
yields a pointed domain, since the Sierpinski domain is pointed, and a
continuous function space into a pointed domain is itself
pointed. Hence our functor $K$ is also a functor into $CPO_\bot$, the
category of complete \emph{pointed} partial orders and continuous
functions. Now, we can appeal to the existence of solutions to our
recursive domain equation to solve for the solution to the equation 

\begin{displaymath}
K \cong \mathcal{K}(K, K)
\end{displaymath}

\subsection{Computations form a Monad}

Most of the base type constructors are obviously interpreted in terms
of the underlying categorical constructions: pair types are
categorical products, sums are categorical sums, functions are
exponentials, and natural numbers are interpreted as a natural numbers
object. However, we will need to check that the type of computations
actually forms a monad.

\begin{lemma}{(Computations form a Monad)} 
The functor $T(A) = (A \to K) \to K$ forms a monad in CPO. 
\end{lemma}

\begin{proof}
We need to give a unit (a family of arrows $\eta_A : A \to T(A)$) and a lift
operations (given $f : A \to T(B)$, we need to give $f^* : T(A) \to T(B)$, 
such that the following equations hold:
\begin{enumerate}
\item $\eta_A^* = id_{T(A)}$
\item $f^* \circ \eta_A = f$ 
\item $f^* \circ g^* = (f^* \circ g)^*$
\end{enumerate}
(Technically, these conditions are the conditions for a Kleisli
triple, which is equivalent to a monad.)  We can now define $\eta_A$
and $f^*$ in terms of the internal language of CPO as follows:

\begin{displaymath}
  \begin{array}{lcl}
    \eta_A(a) & = & \lambda k.\; k\;a \\
    f^*       & = & \lambda a' : (A \to K) \to K. \lambda k_b : (B \to K).\;
                         a' (\lambda a.\; f\;a\;k_b)
  \end{array}
\end{displaymath}

\begin{enumerate}
\item The proof of the first equation is as follows: 
  \begin{displaymath}
    \begin{array}{lcl}
      \eta_A^* 
        & = & \lambda a'.\;\lambda k_{A}.\; 
                 a'(\lambda a.\; \eta_A\;a\;k_{A}) \\ 
        & = & \lambda a'.\;\lambda k_{A}.\; 
                 a'(\lambda a.\; k_A\;a) \\ 
        & = & \lambda a'.\;\lambda k_{A}.\; a'\;k_A \\
        & = & \lambda a'.\;a' \\
        & = & id_{T(A)} \\
    \end{array}
  \end{displaymath}

\item The proof of the second equation is as follows: 
  \begin{displaymath}
    \begin{array}{lcl}
      f^* \circ \eta_A 
        & = & \lambda a.\; f^*(\eta_A(a)) \\
        & = & \lambda a.\; f^*(\lambda k.\;k\;a) \\
        & = & \lambda a.\; \lambda k_a.\; 
                (\lambda k.\;k\;a)\;(\lambda a.\;f\;a\;k_a) \\
        & = & \lambda a.\;\lambda k_a.\;
                  (\lambda a.\;f\;a\;k_a)\;a \\
        & = & \lambda a.\;\lambda k_a.\; f\;a\;k_a \\
        & = & \lambda a.\; f\;a \\
        & = & f \\                       
    \end{array}
  \end{displaymath}

\item The proof of the third equation is as follows: 
  \begin{displaymath}
    \begin{array}{lcl}
      f^* \circ g^* 
      & = & \lambda a''.\; f^*(g^*(a'')) \\
      & = & \lambda a''.\; f^*(\lambda k.\;a''(\lambda a.\;g\;a\;k)) \\
      & = & \lambda a''.\;\lambda k.\;
              (\lambda k.\;a''(\lambda a.\;g\;a\;k))\;
              (\lambda a_1.\; f\;a\;k) \\
      & = & \lambda a''.\;\lambda k.\;
              a''(\lambda a.\;g\;a\;(\lambda a_1.\;f\;a_1\;k)) \\
      & = & \lambda a''.\;\lambda k.\;
               a''(\lambda a.\;f^*\; (g a)\; k) \\
      & = & \lambda a''.\;\lambda k.\;
               a''(\lambda a.\;(f^* \circ g)\;a\;k) \\
      & = & (f^* \circ g)^* \\       
    \end{array}
  \end{displaymath}
\end{enumerate}
\end{proof}

\section{The Programming Language}

We have given the semantics of types in domain-theoretic terms. Now,
we'll give the syntax and typing of the programming language. Then,
we'll use the domain-theoretic semantics just given to first give a
denotational semantics for the programming language, and second, to
give an interesting equality theory for it. The syntax of terms is
given in figure~\ref{lang-syntax}.


\begin{figure}
\begin{displaymath}
  \begin{array}{llcl}
    \mbox{Pure expressions} & 
     e & ::= & 
         \unit \bnfalt
         \pair{e}{e'} \bnfalt
         \fst{e} \bnfalt
         \snd{e} \bnfalt 
\\
     &&& \inl{\;e} \bnfalt
         \inr{\;e} \bnfalt
         \Case{e_0}{x_1}{e_1}{x_2}{e_2} \bnfalt
\\
     &&& \z \bnfalt 
         \s{e} \bnfalt 
         \iter{e}{e_0}{x}{e_1}
\\ 
     &&& x \bnfalt \fun{x}{A}{e} \bnfalt e\;e' \bnfalt
\\ 
     &&& \Fun{\alpha}{\kappa}{e} \bnfalt e\;\tau \bnfalt
\\ 
     &&& \pack{\tau}{e} \bnfalt \unpack{\alpha}{x}{e}{e'} \bnfalt
\\
     &&& \comp{c} \bnfalt \fix{x:D}{e}
\\[1em]
  \mbox{Computations} & 
    c & ::= & e \bnfalt \letv{x}{e}{c} \bnfalt
              \newref{A}{e} \bnfalt !e \bnfalt e := e'
\\[1em]
  \mbox{Contexts} & 
    \Gamma & ::= & \cdot \bnfalt \Gamma, x:A 
\\[1em]
  \mbox{Pointed Types} & 
     D & ::= & \monad{A} \bnfalt A \to D \bnfalt \forall \alpha:\kappa.\; D 
               \bnfalt \unittype \bnfalt D \times D % \bnfalt \cont{A} 
\\[1em] 
  \end{array}
\end{displaymath}
\caption{Syntax of the Programming Language}
\label{lang-syntax}
\end{figure}



\begin{figure}
\begin{mathpar}
\inferrule*[right=EUnit]
          { }
          {\judgeE{\Gamma}{\unit}{\unittype}}
\and
\inferrule*[right=EPair]
          {\judgeE{\Gamma}{e_1}{A} \\ 
           \judgeE{\Gamma}{e_2}{B}}
          {\judgeE{\Gamma}{\pair{e_1}{e_2}}{A \times B}}
\and
\inferrule*[right=EFst]
          {\judgeE{\Gamma}{e}{A \times B}}
          {\judgeE{\Gamma}{\fst{e}}{A}}
\and
\inferrule*[right=ESnd]
          {\judgeE{\Gamma}{e}{A \times B}}
          {\judgeE{\Gamma}{\snd{e}}{B}}
\\
\inferrule*[right=EInl]
          {\judgeE{\Gamma}{e}{A}}
          {\judgeE{\Gamma}{\inl{\;e}}{A + B}}
\and
\inferrule*[right=EInr]
          {\judgeE{\Gamma}{e}{B}}
          {\judgeE{\Gamma}{\inr{\;e}}{A + B}}
\and
\inferrule*[right=ECase]
          {\judgeE{\Gamma}{e}{A+B} \\
           \judgeE{\Gamma, x:A}{e_1}{C} \\
           \judgeE{\Gamma, y:B}{e_2}{C}}
          {\judgeE{\Gamma}{\Case{e}{x}{e_1}{y}{e_2}}{C}}
\\
\inferrule*[right=EZero]
          { }
          {\judgeE{\Gamma}{\z}{\N}}
\and
\inferrule*[right=ESucc]
          {\judgeE{\Gamma}{e}{\N}}
          {\judgeE{\Gamma}{\s{e}}{\N}}
\and
\inferrule*[right=EIter]
          {\judgeE{\Gamma}{e}{\N} \\ 
           \judgeE{\Gamma}{e_0}{A} \\
           \judgeE{\Gamma,x:A}{e_1}{A}}
          {\judgeE{\Gamma}{\iter{e}{e_0}{x}{e_1}}{A}}
\end{mathpar}
\begin{mathpar}
\inferrule*[right=EVar]
          { x:A \in \Gamma }
          {\judgeE{\Gamma}{x}{A}}
\and
\inferrule*[right=ELam]
          {\judgeE{\Gamma, x:A}{e}{B}}
          {\judgeE{\Gamma}{\fun{x}{A}{e}}{A \to B}}
\and
\inferrule*[right=EApp]
          {\judgeE{\Gamma}{e}{A \to B} \\
           \judgeE{\Gamma}{e'}{A} }
          {\judgeE{\Gamma}{e\;e'}{B}}
\\
\inferrule*[right=ETLam]
          {\judgeE[\Theta, \alpha:\kappa]{\Gamma}{e}{A}}
          {\judgeE{\Gamma}{(\Fun{\alpha}{\kappa}{e})}{(\forall \alpha:\kappa.\;A)}}
\and
\inferrule*[right=ETapp]
          {\judgeE{\Gamma}{e}{\forall \alpha:\kappa.\;A} \\
           \judgeWK{\tau}{\kappa}}
          {\judgeE{\Gamma}{e\;\tau}{[\tau/\alpha]A}}
\\
\inferrule*[right=EPack]
          {\judgeE{\Gamma}{\tau}{\kappa} \\ 
           \judgeWK[\Theta, \alpha:\kappa]{A}{\bigstar} \\
           \judgeE{\Gamma}{e}{[\tau/\alpha]A}}
          {\judgeE{\Gamma}{\pack{\tau}{e}}{\exists \alpha:\kappa.\; A}}
\and
\inferrule*[right=EUnpack]
          {\judgeE{\Gamma}{e}{\exists \alpha:\kappa.\;A} \\
           \judgeE[\Theta, \alpha:\kappa]{\Gamma, x:A}{e'}{B} \\
           \judgeWK{B}{\bigstar}}
          {\judgeE{\Gamma}{\unpack{\alpha}{x}{e}{e'}}{B}}
\\
\inferrule*[right=EMonad]
          {\judgeC{\Gamma}{c}{A}}
          {\judgeE{\Gamma}{\comp{c}}{\monad{A}}}
\and
\inferrule*[right=EFix]
          {\judgeE{\Gamma, x:D}{e}{D}}
          {\judgeE{\Gamma}{\fix{x:D}{e}}{D}}
\and
\inferrule*[right=EKeq]
          {\judgeE{\Gamma}{e}{A} \\
           \judgeKeq{A}{B}{\bigstar}}
          {\judgeE{\Gamma}{e}{B}}

\end{mathpar}
\caption{Typing of the Pure Expressions}
\label{lang-typing-pure}
\end{figure}

The pure terms of the language include the unit value $\unit$; pairs
$\pair{e}{e'}$ and projections $\fst{e}$ and $\snd{e}$; injections
into sum types $\inl{\;e}$ and $\inr{\;e}$, and a case form
$\Case{e}{x}{e'}{y}{e''}$; lambda abstractions $\fun{x}{A}{e}$ and
applications $e\;e'$; type abstraction $\Fun{\alpha}{\kappa}{e}$ and
type application $e\;\tau$; and existential packing $\pack{\tau}{e}$
and unpacking $\unpack{\alpha}{x}{e}{e'}$. Suspended monadic
computations $\comp{c}$ are terms of the type $\monad{A}$.  Natural
numbers are given by zero $\z$ and successor $\s{e}$ constructors, and
are eliminated by primitive iteration $\iter{e}{e_0}{x}{\!e_1}$. The
restriction to primitive iteration ensures that infinitely looping
programs cannot be defined by eliminating natural numbers, while still
permitting us to define total functions such as addition and
multiplication. (We will make free use of other inductive datatypes in
following chapters, using the naturals as a prototypical example of how 
to handle them.)

For general recursion, we have a term-level fixed point operator,
$\fix{x:D}{e}$. It is not defined over all types; it is only permitted
to range over \emph{well-pointed} types. That is, fixed points are
only defined over types whose interpretations are domains with least
elements. The pointed types include the monadic types, functions into
pointed types, products of pointed types, and polymorphic
quantification over pointed types. Functions into pointed types
corresponds to ML's usual fixed point, and taking fixed points of
products corresponds to mutual recursion.  Taking fixed points of
polymorphic types corresponds to \emph{polymorphic recursion}. The
absence of nontermination (or any other effect) at other types ensures
that the full beta and eta rules will be available for reasoning with
them.

The typing rules for all of these forms are given in
figure~\ref{lang-typing-pure}, and the computation forms of the
language are given in figure~\ref{lang-typing-monadic}.


\begin{figure}
\begin{mathpar}
\inferrule*[right=CReturn]
          {\judgeE{\Gamma}{e}{A}}
          {\judgeC{\Gamma}{e}{A}}
\and
\inferrule*[right=CLet]
          {\judgeE{\Gamma}{e}{\monad{A}} \\
           \judgeC{\Gamma, x:A}{c}{B}}
          {\judgeC{\Gamma}{\letv{x}{e}{c}}{B}}
\and
\inferrule*[right=CGet]
          {\judgeE{\Gamma}{e}{\reftype{A}}}
          {\judgeC{\Gamma}{!e}{A}}
\and
\inferrule*[right=CSet]
          {\judgeE{\Gamma}{e}{\reftype{A}} \\
           \judgeE{\Gamma}{e'}{A}}
          {\judgeC{\Gamma}{e := e'}{\unittype}}
\and
\inferrule*[right=CNew]
          {\judgeE{\Gamma}{e}{A}}
          {\judgeC{\Gamma}{\newref{A}{e}}{\reftype{A}}}
\end{mathpar}
\caption{Typing of Monadic Expressions}
\label{lang-typing-monadic}
\end{figure}

\section{Denotational Semantics}

We give the semantics of the expression and command languages
with functions $\interpE{\judgeE{\Gamma}{e}{A}}$ and
$\interpC{\judgeC{\Gamma}{c}{A}}$. Since we have two contexts, we will
need two environments, one a type environment (as before, a tuple of
closed type expressions), and the other, a value environment
(consisting of a tuple of values of the appropriate type). In other
words, the interpretation of a term is a type-indexed family of
morphisms in $CPO$.

We give the interpretation of contexts in
Figure~\ref{lang-context-interp}. In this definition, we take the
usual liberties in not explicitly giving the isomorphisms necessary to
implement structural rules like Exchange. The definitions of the two
mutually recursive interpretation functions, $\interpE{\judgeE{\Gamma}{e}{A}}$
and $\interpC{\judgeC{\Gamma}{c}{A}}$, which interpret pure expressions and
computations respectively, are given in
figure~\ref{lang-pure-interp} and figure~\ref{lang-monadic-interp}.

We also give a set of rules for deriving equalities between program
expressions, in the figures from \ref{lang-pure-eq-1} to
\ref{lang-cong-eq}. These rules include the $\beta$- and
$\eta$-equalities of the lambda calculus, for sums, products, and
function spaces, the monad laws for computation types, as well as the
$\beta$- and $\eta$-equalities for numbers, existentials and
universals. The $\eta$-rule for numbers justifies reasoning about
iterative programs via induction on the natural numbers. The
$\eta$-rule for universals and existentials arises from simple
extensionality over types: since our model is not parametric, it
does not justify parametric principles of reasoning.

Below, we collect the theorems describing the properties of these two
judgments, and give their proofs in the following subsection.

\begin{figure}
\begin{mathpar}
\inferrule*[right=CtxNil]
          { }
          {\judgectx{\Theta}{\cdot}}
\and
\inferrule*[right=CtxCons]
          {\judgectx{\Theta}{\Gamma} \\ \judgeWK{A}{\bigstar}}
          {\judgectx{\Theta}{\Gamma, x:A}}
\\
\begin{array}{lcl}
 \interp{\Theta \vdash \cdot}\;\theta & = & 1 \\
 \interp{\Theta \vdash \Gamma, x:A}\;\theta & = & 
     \interp{\Theta \vdash \Gamma}\;\theta 
     \times 
     \interp{\Theta \vdash A:\bigstar}\theta \\
\end{array}
\end{mathpar}
\caption{Interpretation of Program Contexts}
\label{lang-context-interp}
\end{figure}

\begin{figure}
\begin{displaymath}
  \begin{array}{lcl}
    \interpE{\judgeE{x_1:A_1, \ldots, x_n:A_n}{x_i}{A_i}}\;\theta\;\gamma
       & = & \pi_i(\gamma)
    \\
    \interpE{\judgeE{\Gamma}{\fun{x}{A}{e}}{A \to B}}\;\theta\;\gamma
       & = & \semfun{v}{\interpE{\judgeE{\Gamma, x:A}{e}{B}}\;\theta\;(\gamma, v)} \\
    \interpE{\judgeE{\Gamma}{e\;e'}{B}}\;\theta\;\gamma
       & = & 
       (\interpE{\judgeE{\Gamma}{e}{A\to B}}\;\theta\;\gamma)\;
       (\interpE{\judgeE{\Gamma}{e'}{A}}\theta\;\gamma)
    \\
    \interpE{\judgeE{\Gamma}{\unit}{\unittype}}\;\theta\;\gamma
       & = & 
       *
    \\
    \interpE{\judgeE{\Gamma}{\pair{e_1}{e_2}}{A_1\times A_2}}\;\theta\;\gamma
       & = & 
          \sempair{\interpE{\judgeE{\Gamma}{e_1}{A_1}}\;\theta\;\gamma}
                  {\interpE{\judgeE{\Gamma}{e_2}{A_2}}\;\theta\;\gamma}
    \\
    \interpE{\judgeE{\Gamma}{\fst{e}}{A_1}}\;\theta\;\gamma
       & = & 
       \pi_1(\interpE{\judgeE{\Gamma}{e}{A_1\times A_2}}\;\theta\;\gamma)
    \\
    \interpE{\judgeE{\Gamma}{\snd{e}}{A_2}}\;\theta\;\gamma
       & = & 
       \pi_2(\interpE{\judgeE{\Gamma}{e}{A_1\times A_2}}\;\theta\;\gamma)
    \\
    \interpE{\judgeE{\Gamma}{\inl{\;e}}{A_1+A_2}}\;\theta\;\gamma 
       & = & 
       \iota_1(\interpE{\judgeE{\Gamma}{e}{A_1}}\;\theta\;\gamma)
    \\
    \interpE{\judgeE{\Gamma}{\inr{\;e}}{A_1+A_2}}\;\theta\;\gamma 
       & = & 
       \iota_2 (\interpE{\judgeE{\Gamma}{e}{A_2}}\;\theta\;\gamma)
    \\
    \interpE{\judgeE{\Gamma}{\Case{e}{x}{e_1}{y}{e_2}}{C}}\;\theta\;\gamma
       & = & 
          \mbox{let } a = \interpE{\judgeE{\Gamma}{e}{A_1 + A_2}}\theta\;\gamma
          \mbox{ in}\\
       &   & 
          \mbox{let } f_1 = \semfun{v_1}{\interpE{\judgeE{x:A_1, \Gamma}{e_1}{C}}\theta\;(\gamma, v_1)}
          \mbox{ in}\\
       &   & 
          \mbox{let } f_2 = \semfun{v_2}{\interpE{\judgeE{y:A_2, \Gamma}{e_2}{C}}\theta\;(\gamma, v_2)} 
          \mbox{ in}\\
       &   &\;\;
           [f_1, f_2](a)
    \\
    \interpE{\judgeE{\Gamma}{\z}{\N}}\;\theta\;\gamma
       & = & 
         0 
    \\
    \interpE{\judgeE{\Gamma}{\s{e}}{\N}}\;\theta\;\gamma
       & = & 
         1 + (\interpE{\judgeE{\Gamma}{e}{\N}}\;\theta\;\gamma)
    \\
    \interpE{\judgeE{\Gamma}{\iter{e}{e_0}{x}{e_1}}{A}}\;\theta\;\gamma
    &=& 
       \mbox{let } 
          a = \interpE{\judgeE{\Gamma}{e}{\N}}\;\theta\;\gamma  
       \mbox{ in}\\
    & &\mbox{let } 
          i = \interpE{\judgeE{\Gamma}{e_0}{A}}\;\theta\;\gamma
       \mbox{ in}\\
    & &\mbox{let } 
          s = \semfun{v}{\interpE{\judgeE{\Gamma,x:A}{e_1}{A}}\;\theta\;(\gamma,v)}  
       \mbox{ in}\\
    & &\mbox{ }iter_A[i, s](a)
   \\
   \interpE{\judgeE{\Gamma}{\comp{c}}{\monad{A}}}\;\theta\;\gamma
   & = & 
     \interpC{\judgeC{\Gamma}{c}{A}}\;\theta\;\gamma
   \\
   \interpE{\judgeE{\Gamma}{\fix{x:D}{e}}{D}}\;\theta\;\gamma
   & = & 
     fix(\semfun{v}{(\interpE{\judgeE{\Gamma, x:D}{e}{D}}\;\theta\;(\gamma,v))})
   \\
   \interpE{\judgeE{\Gamma}{\Fun{\alpha}{\kappa}{e}}
                           {\forall \alpha:\kappa.\; A}}\;\theta\;\gamma 
   & = & 
     \lambda \tau:\interp{\kappa}.\; 
        (\interpE{\judgeE[\Theta, \alpha:\kappa]{\Gamma}{e}{A}}\;(\theta, \tau)\;\gamma
   \\
   \interpE{\judgeE{\Gamma}{e\;\tau}{A[\tau/\alpha]}}\;\theta\;\gamma
   & = & 
     \interp{\judgeE{\Gamma}{e}{\forall \alpha:\kappa.\;A}}\;\theta\;\gamma\;
            [\theta(\tau)]
   \\
   \interpE{\judgeE{\Gamma}{\pack{\tau}{e}}{\exists \alpha:\kappa.\;A}}
           \;\theta\;\gamma
   & = & 
           ([\theta(\tau)], \interpE{\judgeE{\Gamma}{e}{A[\tau/\alpha]}}\;\theta\;\gamma)
   \\
   \interpE{\judgeE{\Gamma}{\unpack{\alpha}{x}{e}{e'}}{B}}\;\theta\;\gamma
   & = & 
   (\semfun{\pair{\tau}{v}}
           {\interpE{\judgeE[\Theta, \alpha:\kappa]{\Gamma, x:A}{e'}{B}}
                    \;(\theta, \tau)\;(\gamma,v)})
   \\
   & & \;\;
   (\interpE{\judgeE{\Gamma}{e}{\exists \alpha:\kappa.\;A}}\;\theta\;\gamma)
   \\
   \interpE{\judgeE{\Gamma}{e}{B}}\;\theta\;\gamma 
   & = & 
     \interpE{\judgeE{\Gamma}{e}{A}}\;\theta\;\gamma \mbox{ when } \judgeKeq{A}{B}{\bigstar}
     \mbox{ by EKeq}
   \\
  \end{array}
\end{displaymath}
\caption{Interpretation of Pure Terms}
\label{lang-pure-interp}
\end{figure}

\begin{figure}
\begin{displaymath}
  \begin{array}{lcl}
    \interpC{\judgeC{\Gamma}{e}{A}}\;\theta
    & = & 
    \eta_{\interp{A}\;\theta} \circ \interpE{\judgeE{\Gamma}{e}{A}}\;\theta
    \\
    \interpC{\judgeC{\Gamma}{\letv{x}{e}{c}}{B}}\;\theta\;\gamma
    & = & 
       \mbox{let }c : \interp{\monad{A}}\theta = \interpE{\judgeE{\Gamma}{e}{\monad{A}}}\;\theta\;\gamma \mbox{ in}\\
    && \mbox{let }f : \interp{A}\theta \to \interp{\monad{B}}\theta = 
                  \lambda v.\;\interpE{\judgeE{\Gamma, x:A}{c}{B}}\;\theta\; (\gamma, v) \mbox{ in}\\
    && \;\; f^{*}(c)
    \\
    \interpC{\judgeC{\Gamma}{!e}{A}}\;\theta\;\gamma
    & = &
       \mbox{let }l = \interpE{\judgeE{\Gamma}{e}{\reftype{A}}}\;\theta\;\gamma 
       \mbox{ in}\\
    && \;\;\lambda k.\;\lambda (L, h).\; 
        \left\{ \begin{array}{ll}
                  k\; (h\;l)\; (L, h) & \mbox{when } l \in L \\
                  \top                & \mbox{otherwise} \\
        \end{array}
        \right.
    \\
    \interpC{\judgeC{\Gamma}{e := e'}{\unittype}}\;\theta\;\gamma 
    & = & 
       \mbox{let }l = \interpE{\judgeE{\Gamma}{e}{\reftype{A}}}\;\theta\;\gamma 
    \mbox{ in}\\
    && \mbox{let }v = \interpE{\judgeE{\Gamma}{e'}{A}}\;\theta\;\gamma \mbox{ in}\\
    && \;\; \lambda k.\;\lambda (L, h).\; 
              \left\{ \begin{array}{ll}
                       k \unit (L, [h|l:v]) & \mbox{when } l \in L \\
                      \top                & \mbox{otherwise} \\
              \end{array}
              \right.
    \\
    \interpC{\judgeC{\Gamma}{\newref{A}{e}}{\reftype{A}}}\;\theta\;\gamma 
    & = & 
      \mbox{let }v = \interpE{\judgeE{\Gamma}{e}{A}}\;\theta\;\gamma \mbox{ in}\\
    && \;\;\lambda k.\;\lambda (L, h).\; \\
    && \;\;\qquad \mbox{let }l = \newloc{L}{A} \mbox{ in}\\
    && \;\;\qquad k\;l\;(L \cup \setof{l}, [h|l:v])
    \\[1em]
    \newloc{L}{A} & = & (1 + \sup\comprehend{ n \in \N }{\exists B.\;(n,B) \in L}, A)
  \end{array}
\end{displaymath}
\caption{Interpretation of Computations}
\label{lang-monadic-interp}
\end{figure}



\begin{figure}
\begin{mathpar}
\inferrule*[right=EqUnit]
          {\judgeE{\Gamma}{e}{\unittype} \\
           \judgeE{\Gamma}{e'}{\unittype} }
          {\judgeEq{\Gamma}{e}{e'}{\unittype}}
\\
% \inferrule*[right=EqPairCong]
%           {\judgeEq{\Gamma}{e_1}{e'_1}{A_1} \\
%            \judgeEq{\Gamma}{e_2}{e'_2}{A_2}}
%           {\judgeEq{\Gamma}{\pair{e_1}{e_2}}{\pair{e'_1}{e'_2}}{A_1\times A_2}}
% \and
% \inferrule*[right=EqFstCong]
%           {\judgeEq{\Gamma}{e}{e'}{A_1\times A_2}}
%           {\judgeEq{\Gamma}{\fst{e}}{\fst{e'}}{A_1}}
% \and
% \inferrule*[right=EqFstCong]
%           {\judgeEq{\Gamma}{e}{e'}{A_1\times A_2}}
%           {\judgeEq{\Gamma}{\snd{e}}{\snd{e'}}{A_2}}
% \\
\inferrule*[right=EqPairFst]
          {\judgeE{\Gamma}{\pair{e_1}{e_2}}{A_1\times A_2}}
          {\judgeEq{\Gamma}{\fst{\pair{e_1}{e_2}}}{e_1}{A_1}}
\and
\inferrule*[right=EqPairSnd]
          {\judgeE{\Gamma}{\pair{e_1}{e_2}}{A_1\times A_2}}
          {\judgeEq{\Gamma}{\snd{\pair{e_1}{e_2}}}{e_2}{A_2}}
\and
\inferrule*[right=EqPairEta]
          {\judgeE{\Gamma}{e}{A_1\times A_2}}
          {\judgeEq{\Gamma}{e}{\pair{\fst{e}}{\snd{e}}}{A_1\times A_2}}
\\
% \inferrule*[right=EqFunCong]
%           {\judgeEq{\Gamma, x:A}{e}{e'}{B} \\
%            \judgeKeq{A}{A'}{\bigstar}}
%           {\judgeEq{\Gamma}{\fun{x}{A}{e}}{\fun{x}{A'}{e'}}{B}}
% \and
% \inferrule*[right=EqAppCong]
%           {\judgeEq{\Gamma}{e_1}{e'_1}{A \to B} \\
%            \judgeEq{\Gamma}{e_2}{e'_2}{A}}
%           {\judgeEq{\Gamma}{e_1\;e_2}{e'_1\;e'_2}{B}}
% \and
\inferrule*[right=EqFunEta]
          {\judgeEq{\Gamma, x:A}{e\;x}{e'\;x}{B} \\ x \not\in \FV{e,e'}}
          {\judgeEq{\Gamma}{e}{e'}{A \to B}}
\and
\inferrule*[right=EqFunBeta]
          {\judgeE{\Gamma}{(\fun{x}{A}{e})\;e'}{B} \\ x \not\in \Gamma}
          {\judgeEq{\Gamma}{(\fun{x}{A}{e})\;e'}{[e'/x]e}{B}}
\\
% \inferrule*[right=EqInlCong]
%           {\judgeEq{\Gamma}{e}{e'}{A}}
%           {\judgeEq{\Gamma}{\inl{e}}{\inl{e'}}{A+B}}
% \and
% \inferrule*[right=EqInrCong]
%           {\judgeEq{\Gamma}{e}{e'}{B}}
%           {\judgeEq{\Gamma}{\inr{e}}{\inr{e'}}{A+B}}
% \and
% \inferrule*[right=EqCaseCong]
%           {\judgeEq{\Gamma}{e}{e'}{A+B} \\
%            \judgeEq{\Gamma, x:A}{e_1}{e'_1}{C} \\
%            \judgeEq{\Gamma, y:B}{e_2}{e'_2}{C} }
%           {\judgeEq{\Gamma}{\Case{e}{x}{e_1}{y}{e_2}}{\Case{e'}{x}{e'_1}{y}{e'_2}}{C}}
% \and
\inferrule*[right=EqSumInlBeta]
          {\judgeE{\Gamma}{\Case{\inl{\;e}}{x}{e_1}{y}{e_2}}{C}}
          {\judgeEq{\Gamma}{\Case{\inl{\;e}}{x}{e_1}{y}{e_2}}{[e/x]e_1}{C}}
\and
\inferrule*[right=EqSumInrBeta]
          {\judgeE{\Gamma}{\Case{\inr{\;e}}{x}{e_1}{y}{e_2}}{C}}
          {\judgeEq{\Gamma}{\Case{\inr{\;e}}{x}{e_1}{y}{e_2}}{[e/y]e_2}{C}}
\and
\inferrule*[right=EqSumEta]
          {\judgeE{\Gamma}{e}{A+B} \\
           \judgeE{\Gamma, z:A+B}{e'}{C}}
          {\judgeEq{\Gamma}{[e/z]e'}{\Case{e}{x}{[\inl{\;x}/z]e'}
                                             {y}{[\inr{\;x}/z]e'}}{C}}
\\
\inferrule*[right=EqMonad]
          {\judgeEqC{\Gamma}{c}{c'}{A}}
          {\judgeEq{\Gamma}{\comp{c}}{\comp{c'}}{\monad{A}}}
\and
\inferrule*[right=EqFix]
          {\judgeE{\Gamma}{\fix{x:D}{e}}{D}}
          {\judgeEq{\Gamma}{\fix{x:D}{e}}{[\fix{x:D}{e}/x]e}{D}}
\end{mathpar}
\caption{Equality Rules for Sums, Products, Exponentials, and Suspended Computations}
\label{lang-pure-eq-1}
\end{figure}

\begin{figure}
\begin{mathpar}
\inferrule*[right=EqNatZBeta]
          {\judgeE{\Gamma}{\iter{\z}{e_0}{x}{e_1}}{A}}
          {\judgeEq{\Gamma}{\iter{\z}{e_0}{x}{e_1}}{e_0}{A}}
\and
\inferrule*[right=EqNatSBeta]
          {\judgeE{\Gamma}{\iter{\s{e}}{e_0}{x}{e_1}}{A}}
          {\judgeEq{\Gamma}{\iter{\s{e}}{e_0}{x}{e_1}}{[\iter{e}{e_0}{x}{e_1}/x]e_1}{A}}
\and
\inferrule*[right=EqNatEta]
          {\judgeE{\Gamma}{e_0}{A} \\
           \judgeE{\Gamma, x:A}{e_1}{A} \\
           \judgeE{\Gamma}{e}{A} \\
           \judgeEq{\Gamma}{[\z/n]e}{e_0}{A} \\
           \judgeEq{\Gamma, m:\N}{[\s{m}/n]e}{[[m/n]e/x]e_1}{A}}
          {\judgeEq{\Gamma, n:\N}{e}{\iter{n}{e_0}{x}{e_1}}{A}}
\and
\inferrule*[right=EqAllBeta]
          {\judgeE{\Gamma}{\Fun{\alpha}{\kappa}{e}}
                          {\forall \alpha:\kappa.\;A} \\
           \judgeWK{\tau}{\kappa}}
          {\judgeEq{\Gamma}{(\Fun{\alpha}{\kappa}{e})\;\tau}
                           {[\tau/\alpha]e}
                           {[\tau/\alpha]A}}
\and
\inferrule*[right=EqAllEta]
          {\judgeEq[\Theta, \alpha:\kappa]{\Gamma}{e\;\alpha}{e'\;\alpha}{A} \\
           \judgeE{\Gamma}{e}{\forall \alpha:\kappa.\;A} \\
           \judgeE{\Gamma}{e'}{\forall \alpha:\kappa.\;A} \\
           \Theta \vdash \Gamma}
          {\judgeEq{\Gamma}{e}{e'}{\forall \alpha:\kappa.\;A}}
\and
\inferrule*[right=EqExistsBeta]
          {\judgeE{\Gamma}{\pack{\tau}{e}}{\exists \alpha:\kappa.\;A} \\
           \judgeE[\Theta, \alpha:\kappa]{\Gamma, x:A}{e'}{B}}
          {\judgeEq{\Gamma}{\unpack{\alpha}{x}{\pack{\tau}{e}}{e'}}
                          {[\tau/\alpha][e/x]e'}
                          {B}}
\and
\inferrule*[right=EqExistsEta]
          {\judgeE{\Gamma, z:\exists \alpha:\kappa.\;A}{e'}{B} \\ 
           \judgeE{\Gamma}{e}{\exists \alpha:\kappa.\;A}}
          {\judgeEq{\Gamma}{\unpack{\alpha}{x}{e}{[\pack{\alpha}{x}/z]e'}}{[e/z]e'}{B}}
\end{mathpar}
\caption{Equality Rules for Numbers, Universals, and Existentials}
\label{lang-pure-eq-2} 
\end{figure}

\begin{figure}
\begin{mathpar}
\inferrule*[right=EqCommandEta]
          {\judgeC{\Gamma}{c}{A}}
          {\judgeEqC{\Gamma}{c}{\letv{x}{[c]}{x}}{A}}
\and
\inferrule*[right=EqCommandBeta]
          {\judgeC{\Gamma}{\letv{x}{\comp{e}}{c}}{A}}
          {\judgeEqC{\Gamma}{\letv{x}{\comp{e}}{c}}{[e/x]c}{A}}
\and
\inferrule*[right=EqCommandAssoc]
          {\judgeC{\Gamma}{\letv{x}{\comp{\letv{y}{e}{c_1}}}{c_2}}{A}}
          {\judgeEqC{\Gamma}{\letv{x}{\comp{\letv{y}{e}{c_1}}}{c_2}}
                            {\letv{y}{e}{\letv{x}{\comp{c_1}}}{c_2}}{A}}
\end{mathpar}
\caption{Equality Rules for Computations}
\label{lang-monad-eq}  
\end{figure}


\begin{figure}
\begin{mathpar}
\inferrule*[right=EqRefl]
          {\judgeE{\Gamma}{e}{A}}
          {\judgeEq{\Gamma}{e}{e}{A}}
\and
\inferrule*[right=EqSymm]
          {\judgeEq{\Gamma}{e}{e'}{A}}
          {\judgeEq{\Gamma}{e'}{e}{A}}
\and
\inferrule*[right=EqTrans]
          {\judgeEq{\Gamma}{e}{e'}{A} \\
           \judgeEq{\Gamma}{e'}{e''}{A}}
          {\judgeEq{\Gamma}{e}{e''}{A}}
\and
\inferrule*[right=EqSubst]
          {\judgeEq{\Gamma, x:A}{e_1}{e_2}{B} \\
           \judgeEq{\Gamma}{e'_1}{e'_2}{A}}
          {\judgeEq{\Gamma}{[e'_1/x]e_1}{[e'_2/x]e_2}{B}}
\and
\inferrule*[right=EqCommandRefl]
          {\judgeC{\Gamma}{c}{A}}
          {\judgeEqC{\Gamma}{c}{c}{A}}
\and
\inferrule*[right=EqCommandSymm]
          {\judgeEqC{\Gamma}{c}{c'}{A}}
          {\judgeEqC{\Gamma}{c'}{c}{A}}
\and
\inferrule*[right=EqCommandTrans]
          {\judgeEqC{\Gamma}{c}{c'}{A} \\
           \judgeEqC{\Gamma}{c'}{c''}{A}}
          {\judgeEqC{\Gamma}{c}{c''}{A}}
\and
\inferrule*[right=EqCommandSubst]
          {\judgeEqC{\Gamma, x:A}{c_1}{c_2}{B} \\
           \judgeEq{\Gamma}{e_1}{e_2}{A}}
          {\judgeEqC{\Gamma}{[e_1/x]c_1}{[e_2/x]c_2}{B}}
\end{mathpar}
\caption{Congruence Rules for Equality}
\label{lang-cong-eq}
\end{figure}

\begin{lemma}{(Soundness of Weakening)}
  Suppose $\judgeE[\Theta]{\Gamma}{e}{A}$. 
  \begin{enumerate}
   \item It is the case that 
    $\interpE{\judgeE[\Theta, \alpha:\kappa]{\Gamma}{e}{A}}\;(\theta, \tau)\;\gamma$ is equal to 
    $\interpE{\judgeE[\Theta]{\Gamma}{e}{A}}\;\theta\;\gamma$. 
   \item It is the case that 
    $\interpE{\judgeE[\Theta]{\Gamma, x:B}{e}{A}}\;\theta\;(\gamma,v)$ is equal to 
    $\interpE{\judgeE[\Theta]{\Gamma}{e}{A}}\;\theta\;\gamma$. 
  \end{enumerate}
\end{lemma}\ \\


\begin{lemma}{(Soundness of Type Substitution)}
If we know that $\judgeWK{\tau}{\kappa}$, then 
\begin{itemize}
\item If $\judgeE[\Theta, \alpha:\kappa]{\Gamma}{e}{A}$, then 
       $\interpE{\judgeE[\Theta, \alpha:\kappa]{\Gamma}{e}{A}}\;(\theta,[\theta(\tau)])\;\gamma$ is
       equal to \\ $\interpE{\judgeE{[\tau/\alpha]\Gamma}{[\tau/\alpha]e}{[\tau/\alpha]A}}\;\theta\;\gamma$
\item If $\judgeC[\Theta, \alpha:\kappa]{\Gamma}{c}{A}$, then 
       $\interpC{\judgeC[\Theta, \alpha:\kappa]{\Gamma}{c}{A}}\;(\theta,[\theta(\tau)])\;\gamma$ 
       is equal to \\ $\interpC{\judgeC{[\tau/\alpha]\Gamma}{[\tau/\alpha]c}{[\tau/\alpha]A}}\;\theta\;\gamma$
\end{itemize}
\end{lemma}\ \\


\begin{lemma}{(Soundness of Substitution)}
\begin{enumerate}
\item If we know that $\judgeE{\Gamma, y:A, \Gamma'}{e}{B}$ and $\judgeE{\Gamma}{e'}{A}$
  and $\Theta \vdash \theta$, 
  then \\ $\interpE{\judgeE{\Gamma}{[e'/y]e}{B}}\;\theta\;(\gamma,\gamma') = 
        \interpE{\judgeE{\Gamma, y:A}{e}{B}}\;\theta 
                \;\left(\gamma, 
                        \interpE{\judgeE{\Gamma}{e'}{A}}\;\theta\;\gamma,
                        \gamma'
                \right)$

\item If we know that $\judgeC{\Gamma, y:A, \Gamma'}{c}{B}$ and $\judgeE{\Gamma}{e'}{A}$
  and $\Theta \vdash \theta$, 
  then \\ $\interpC{\judgeE{\Gamma}{[e'/y]c}{B}}\;\theta\;(\gamma,\gamma') = 
        \interpC{\judgeC{\Gamma, y:A}{c}{B}}\;\theta \;
                \left(\gamma,
                      \interpC{\judgeE{\Gamma}{e'}{A}}\;\theta\;\gamma,
                      \gamma'
                \right)$
\end{enumerate}
\end{lemma}\ \\

\begin{lemma}{(Soundness of Equality Rules)}
We have that:
\begin{enumerate}
\item If $\judgeEq{\Gamma}{e}{e'}{A}$, then $\judgeE{\Gamma}{e}{A}$ and 
$\judgeE{\Gamma}{e'}{A}$ and 
$\interpE{\judgeE{\Gamma}{e}{A}} = \interpE{\judgeE{\Gamma}{e'}{A}}$.

\item If $\judgeEqC{\Gamma}{c}{c'}{A}$, then $\judgeC{\Gamma}{c}{A}$ and 
$\judgeC{\Gamma}{c'}{A}$ and 
$\interpC{\judgeC{\Gamma}{c}{A}} = \interpC{\judgeC{\Gamma}{c'}{A}}$.
\end{enumerate}
\end{lemma}\ \\



\subsection{Proofs}

\begin{lemma*}{(Soundness of Weakening)}
  Suppose $\judgeE[\Theta]{\Gamma}{e}{A}$. 
  \begin{enumerate}
   \item It is the case that 
    $\interpE{\judgeE[\Theta, \alpha:\kappa]{\Gamma}{e}{A}}\;(\theta, \tau)\;\gamma$ is equal to \\
    $\interpE{\judgeE[\Theta]{\Gamma}{e}{A}}\;\theta\;\gamma$. 
   \item It is the case that 
    $\interpE{\judgeE[\Theta]{\Gamma, x:B}{e}{A}}\;\theta\;(\gamma,v)$ is equal to \\
    $\interpE{\judgeE[\Theta]{\Gamma}{e}{A}}\;\theta\;\gamma$. 
  \end{enumerate}
\end{lemma*}

\begin{proof}
  The proof is by induction on the typing derivation of $\judgeE[\Theta]{\Gamma}{e}{A}$. 
\end{proof}\\


\begin{lemma*}{(Soundness of Substitution)}
\begin{enumerate}
\item If we know that $\judgeE{\Gamma, y:A, \Gamma'}{e}{B}$ and $\judgeE{\Gamma}{e'}{A}$
  and $\Theta \vdash \theta$, 
  then \\ $\interpE{\judgeE{\Gamma}{[e'/y]e}{B}}\;\theta\;(\gamma,\gamma') = 
        \interpE{\judgeE{\Gamma, y:A}{e}{B}}\;\theta 
                \;\left(\gamma, 
                        \interpE{\judgeE{\Gamma}{e'}{A}}\;\theta\;\gamma,
                        \gamma'
                \right)$

\item If we know that $\judgeC{\Gamma, y:A, \Gamma'}{c}{B}$ and $\judgeE{\Gamma}{e'}{A}$
  and $\Theta \vdash \theta$, 
  then \\ $\interpC{\judgeE{\Gamma}{[e'/y]c}{B}}\;\theta\;(\gamma,\gamma') = 
        \interpC{\judgeC{\Gamma, y:A}{c}{B}}\;\theta \;
                \left(\gamma,
                      \interpE{\judgeE{\Gamma}{e'}{A}}\;\theta\;\gamma,
                      \gamma'
                \right)$
\end{enumerate}
\end{lemma*}

\begin{proof}
This property follows by a mutual structural induction on the derivations 
$\judgeE{\Gamma, y:A}{e}{B}$ and $\judgeC{\Gamma, y:A}{c}{C}$. 

First, we'll do the cases for pure terms. (When there's no confusion,
we'll write $\interpE{e}$ for
$\interpE{\judgeE{\Gamma}{e}{A}}\;\theta$.)

\begin{itemize}
\item case EVar: 
  There are two cases, depending on whether the $e = x_i$, or $e = y$

  \begin{enumerate}
  \item If $e = x_i$ for some $i$, then $[e'/y]x_i = x_i$, and so we have
   $\interpE{\judgeE{\Gamma, \Gamma'}{[e'/y]x_i}{B}}\;\theta\;(\gamma, \gamma')$

    \begin{eqnproof}
      \eline{\interpE{\judgeE{\Gamma, \Gamma'}{x_i}{B}}\;\theta\;(\gamma, \gamma')}
            { Subst.}
      \eline{\pi_i(\gamma, \gamma')}
            { Semantics}
      \eline{\pi_i(\gamma, \interpE{\judgeE{\Gamma}{e'}{A}}\;\theta\;\gamma, \gamma')}
            { Adjusting $i$ }
      \eline{\interpE{\judgeE{\Gamma, y:A, \Gamma'}{x_i}{B}}\;\theta\;(\gamma, \interpE{\judgeE{\Gamma}{e'}{A}}\;\theta, \gamma')}
            { Semantics}
    \end{eqnproof}

  \item If $e = y$, then we have $\interpE{\judgeE{\Gamma, \Gamma'}{[e'/y]y}{A}\;\theta\;(\gamma, \gamma')}$

    \begin{eqnproof}
      \eline
            {\interpE{\judgeE{\Gamma, \Gamma'}{e'}{A}}\;\theta\;(\gamma,\gamma')}
            { Subst. }
      \eline{\interpE{\judgeE{\Gamma}{e'}{A}}\;\theta\;\gamma}
            { Since $FV(e') \cap \Gamma = \emptyset$}
      \eline{\interpE{\judgeE{\Gamma, y:A, \Gamma'}{y}{A}}\;\theta
              \;(\gamma, \interpE{\judgeE{\Gamma}{y}{A}}\;\theta\;\gamma, \gamma')}
            { Semantics}
    \end{eqnproof}
  \end{enumerate}

\item case ELam: We have $\interpE{\judgeE{\Gamma, \Gamma'}{[e'/y](\fun{x}{B}{e})}{B \to B'}\;\theta\;(\gamma, \gamma')}$

  \begin{eqnproof}
    \eline
          {\interpE{\judgeE{\Gamma, \Gamma'}{\fun{x}{B}{[e'/y]e}}{B \to B'}}\;\theta\;(\gamma, \gamma')}
          {Subst.}
    \eline{\semfun{v}{\interpE{\judgeE{\Gamma, \Gamma', x:B}{[e'/y]e}{B'}}\;\theta\;(\gamma, \gamma', v)}}
          {Semantics}

    \eline{\semfun{v}
            {\interpE{\judgeE{\Gamma, y:A, \Gamma', x:B}{e}{B'}}\;\theta\;
                       (\gamma, \interpE{\judgeE{\Gamma}{e'}{A}}\;\theta\;\gamma, 
                        \gamma', v)}}     
          {IH}
    \eline{\interpE{\judgeE{\Gamma, y:A, \Gamma'}{\fun{x}{B}{e}}{B \to B'}}
           \;\theta\;(\gamma, 
                      \interpE{\judgeE{\Gamma}{e'}{A}}\;\theta\;\gamma, 
                      \gamma')}
          {Semantics}
  \end{eqnproof}

  Note that we need to use the Barendregt convention, implicitly
  renaming $x$ to ensure the avoidance of captures. We will not mention this point
  further in remaining proofs. 

\item Case EApp: We have 
   $\interpE{\judgeE{\Gamma, \Gamma'}{[e'/y](e_1\;e_2)}{B}}\;\theta\;
       (\gamma, \gamma')$

  \begin{eqnproof}
    \eline{\interpE{\judgeE{\Gamma, \Gamma'}{[e'/y]e_1\;[e'/y]e_2}{B}}\;\theta\;
           (\gamma, \gamma')}
          {Subst.}
    \eline{\interpE{\judgeE{\Gamma, \Gamma'}{[e'/y]e_1}{B_2 \to B}}\;\theta\;(\gamma, \gamma')\;\;
           \interpE{\judgeE{\Gamma, \Gamma'}{[e'/y]e_2}{B_2}}\;\theta\;(\gamma, \gamma')}
          {Semantics}
     \eline{\interpE{\judgeE{\Gamma, y:A, \Gamma'}{e_1}{B_2 \to B}}\;\theta\;
              \gamma''\;\;
            \interpE{\judgeE{\Gamma, y:A, \Gamma'}{e_2}{B_2}}\;\theta\;\gamma''}
           {IH, IH}
     \eline{\interpE{\judgeE{\Gamma, y:A, \Gamma'}{e_1\;e_2}{B}}\;\theta\;\gamma''}
           {Semantics}
  \end{eqnproof}

  (Where $\gamma'' = (\gamma, \interpE{\judgeE{\Gamma}{e'}{A}}\;\theta\;\gamma, \gamma')$)

\item case EUnit: We have $\interpE{\judgeE{\Gamma, \Gamma'}{[e'/y]\unit}{\unittype}}\;\theta\;(\gamma, \gamma')$
  \begin{eqnproof}
   \eline{\interpE{\judgeE{\Gamma, \Gamma'}{\unit}{\unittype}}\;\theta\;(\gamma, \gamma')}
         {Substitution}
   \eline{*}
         {Semantics}
   \eline{\interpE{\judgeE{\Gamma, y:A, \Gamma'}{\unit}{\unittype}}\;\theta\;
          (\gamma, \interpE{\judgeE{\Gamma}{e'}{A}}\;\theta\;\gamma, \gamma')}
         {Semantics}
  \end{eqnproof}


\item case EPair: We have $\interpE{\judgeE{\Gamma, \Gamma'}{[e'/y]\pair{e_1}{e_2}}{B}}\;\theta\;
       (\gamma, \gamma')$ 

  \begin{eqnproof}
    \eline{\interpE{\judgeE{\Gamma, \Gamma'}{\pair{[e'/y]e_1}{[e'/y]e_2}}{B_1\times B_2}}\;\theta\;(\gamma, \gamma')}
          {Subst.}
    \eline{\sempair{\interpE{\judgeE{\Gamma, \Gamma'}{[e'/y]e_1}{B_1}}\;\theta\;(\gamma, \gamma')}
                   {\interpE{\judgeE{\Gamma, \Gamma'}{[e'/y]e_2}{B_2}}\;\theta\;(\gamma, \gamma')}}
          {Semantics}
     \eline{\sempair{\interpE{\judgeE{\Gamma, y:A, \Gamma'}{e_1}{B_1}}\;\theta\;
                     \gamma''}
                    {\interpE{\judgeE{\Gamma, y:A, \Gamma'}{e_2}{B_2}}\;\theta\;\gamma''}}
           {IH, IH}
     \eline{\interpE{\judgeE{\Gamma, y:A, \Gamma'}{\pair{e_1}{e_2}}{B}}\;\theta\;\gamma''}
           {Semantics}
  \end{eqnproof}

  (Where $\gamma'' = (\gamma, \interpE{\judgeE{\Gamma}{e'}{A}}\;\theta\;\gamma, \gamma')$)


\item Case EFst: We have $\interpE{\judgeE{\Gamma, \Gamma'}{[e'/y]\fst{e}}{B_1}}\;\theta\;(\gamma, \gamma')$

  \begin{eqnproof}
    \eline{\interpE{\judgeE{\Gamma, \Gamma'}{\fst{[e'/y]e}}{B_1}}\;\theta\;(\gamma, \gamma')}
          {Substitution}
    \eline{\pi_1(\interpE{\judgeE{\Gamma, \Gamma'}{[e'/y]e}{B_1\times B_2}}\;\theta\;(\gamma, \gamma'))}
          {Semantics}
    \eline{\pi_1(\interpE{\judgeE{\Gamma, y:A, \Gamma'}{e}{B_1\times B_2}}\;\theta\;\gamma'')}
          {IH}
    \eline{\interpE{\judgeE{\Gamma, y:A, \Gamma'}{\fst{e}}{B_1\times B_2}}\;\theta\;\gamma''}
          {Semantics}
  \end{eqnproof}

  (Where $\gamma'' = (\gamma, \interpE{\judgeE{\Gamma}{e'}{A}}\;\theta\;\gamma, \gamma')$)

\item Case ESnd: We have $\interpE{\judgeE{\Gamma, \Gamma'}{[e'/y]\snd{e}}{B_1}}\;\theta\;(\gamma, \gamma')$

  \begin{eqnproof}
    \eline{\interpE{\judgeE{\Gamma, \Gamma'}{\snd{[e'/y]e}}{B_2}}\;\theta\;(\gamma, \gamma')}
          {Substitution}
    \eline{\pi_2(\interpE{\judgeE{\Gamma, \Gamma'}{[e'/y]e}{B_1\times B_2}}\;\theta\;(\gamma, \gamma'))}
          {Semantics}
    \eline{\pi_2(\interpE{\judgeE{\Gamma, y:A, \Gamma'}{e}{B_1\times B_2}}\;\theta\;\gamma'')}
          {IH}
    \eline{\interpE{\judgeE{\Gamma, y:A, \Gamma'}{\snd{e}}{B_1\times B_2}}\;\theta\;\gamma''}
          {Semantics}
  \end{eqnproof}

  (Where $\gamma'' = (\gamma, \interpE{\judgeE{\Gamma}{e'}{A}}\;\theta\;\gamma, \gamma')$)

\item Case EInl: We have $\interpE{\judgeE{\Gamma, \Gamma'}{[e'/y]\inl{e}}{B_1 + B_2}}\;\theta\;(\gamma, \gamma')$

  \begin{eqnproof}
    \eline{\interpE{\judgeE{\Gamma, \Gamma'}{\inl{([e'/y]e)}}{B_1+B_2}}\;\theta\;(\gamma, \gamma')}
          {Substitution}
    \eline{\iota_1(\interpE{\judgeE{\Gamma, \Gamma'}{[e'/y]e}{B_1}}\;\theta\;(\gamma, \gamma'))}
          {Semantics}
    \eline{\iota_1(\interpE{\judgeE{\Gamma, y:A, \Gamma'}{e}{B_1}}\;\theta\;\gamma'')}
          {IH}
    \eline{\interpE{\judgeE{\Gamma, y:A, \Gamma'}{\inl{e}}{B_1+ B_2}}\;\theta\;\gamma''}
          {Semantics}
  \end{eqnproof}

  (Where $\gamma'' = (\gamma, \interpE{\judgeE{\Gamma}{e'}{A}}\;\theta\;\gamma, \gamma')$)


\item Case EInr: We have $\interpE{\judgeE{\Gamma, \Gamma'}{[e'/y]\inl{e}}{B_1 + B_2}}\;\theta\;(\gamma, \gamma')$

  \begin{eqnproof}
    \eline{\interpE{\judgeE{\Gamma, \Gamma'}{\inr{([e'/y]e)}}{B_1+B_2}}\;\theta\;(\gamma, \gamma')}
          {Substitution}
    \eline{\iota_2(\interpE{\judgeE{\Gamma, \Gamma'}{[e'/y]e}{B_2}}\;\theta\;(\gamma, \gamma'))}
          {Semantics}
    \eline{\iota_2(\interpE{\judgeE{\Gamma, y:A, \Gamma'}{e}{B_2}}\;\theta\;\gamma'')}
          {IH}
    \eline{\interpE{\judgeE{\Gamma, y:A, \Gamma'}{\inr{e}}{B_1+ B_2}}\;\theta\;\gamma''}
          {Semantics}
  \end{eqnproof}

  (Where $\gamma'' = (\gamma, \interpE{\judgeE{\Gamma}{e'}{A}}\;\theta\;\gamma, \gamma')$)

\item Case ECase: We have $\judgeE{\Gamma,\Gamma'}{[e'/y]\Case{e}{x_1}{e_1}{x_2}{e_2}}{C}$

  \begin{eqnproof}
    \eline{\judgeE{\Gamma, \Gamma'}{\Case{[e'/y]e}{x_1}{[e'/y]e_1}{x_2}{[e'/y]e_2}}{C}}
          {Substitution}[1em]

    \eclaim[(1)]{\judgeE{\Gamma, \Gamma'}{[e'/y]e}{B_1+B_2}}
           {Inversion}
    \eclaim[(2)]{\judgeE{\Gamma, \Gamma', x_1:B_1}{[e'/y]e_1}{B_1}}
           {Inversion}
    \eclaim[(3)]{\judgeE{\Gamma, \Gamma', x_2:B_2}{[e'/y]e_2}{B_2}}
           {Inversion}
    \eline[\interpE{(1)}]
          {\lambda \theta\;(\gamma, \gamma').\; \interpE{\judgeE{\Gamma, y:A, \Gamma'}{e}{B_1+B_2}}\;\theta\;(\gamma, \interpE{e'}\theta\gamma, \gamma')}
          {IH}
    \eline[\interpE{(2)}]
          {\lambda \theta\;(\gamma, \gamma', v).\; \interpE{\judgeE{\Gamma, y:A, \Gamma', x_1:B_1}{e_1}{C}}\;\theta\;(\gamma, \interpE{e'}\theta\gamma, \gamma', v)}
          {IH}
    \eline[\interpE{(3)}]
          {\lambda \theta\;(\gamma, \gamma', v).\; \interpE{\judgeE{\Gamma, y:A, \Gamma', x_2:B_2}{e_2}{C}}\;\theta\;(\gamma, \interpE{e'}\theta\gamma, \gamma', v)}
          {IH}
  \end{eqnproof}

  Now, the interpretation $\interpE{\judgeE{\Gamma, \Gamma'}{\Case{[e'/y]e}{x_1}{[e'/y]e_1}{x_2}{[e'/y]e_2}}{C}}\;\theta\;(\gamma, \gamma')$ will be equal to 
$[f_1, f_2]\;a$, where 

  \begin{eqnproof}
    \eline[a]{\interpE{\judgeE{\Gamma, \Gamma'}{[e'/y]e}{B_1+B_2}}\;\theta\;(\gamma, \gamma')}{}
    \eline{\interpE{\judgeE{\Gamma, y:A, \Gamma'}{e}{B_1+B_2}}\;\theta\;(\gamma, \interpE{e'}\;\theta\;\gamma, \gamma')}{}
    \eline[f_1]{\lambda v.\;\interpE{\judgeE{\Gamma, \Gamma', x_1:B_1}{e_1}{C}}\;\theta\;(\gamma,\gamma', v)}{}
    \eline{\lambda v.\;\interpE{\judgeE{\Gamma, y:A, \Gamma', x_1:B_1}{e_1}{C}}\;\theta\;(\gamma, \interpE{e'}\;\theta\;\gamma, \gamma', v)}
               {}
    \eline[f_2]{\lambda v.\;\interpE{\judgeE{\Gamma, \Gamma', x_2:B_2}{e_2}{C}}\;\theta\;(\gamma,\gamma', v)}{}
    \eline{\lambda v.\;\interpE{\judgeE{\Gamma, y:A, \Gamma', x_2:B_2}{e_1}{C}}\;\theta\;(\gamma, \interpE{e'}\;\theta\;\gamma, \gamma', v)}
         {}
  \end{eqnproof}

  Which means that $[f_1, f_2]\;a = \interpE{\judgeE{\Gamma, y:A, \Gamma'}{\Case{e}{x_1}{e_1}{x_2}{y_2}}{C}}\;\theta\;(\gamma, \interpE{e'}\;\theta\;\gamma, \gamma')$

\item case EZero: We have $\interpE{\judgeE{\Gamma, \Gamma'}{[e'/y]\z}{\N}}\;\theta\;(\gamma, \gamma')$
  \begin{eqnproof}
   \eline{\interpE{\judgeE{\Gamma, \Gamma'}{\z}{\N}}\;\theta\;(\gamma, \gamma')}
         {Substitution}
   \eline{0}
         {Semantics}
   \eline{\interpE{\judgeE{\Gamma, y:A, \Gamma'}{\z}{\N}}\;\theta\;
          (\gamma, \interpE{\judgeE{\Gamma}{e'}{A}}\;\theta\;\gamma, \gamma')}
         {Semantics}
  \end{eqnproof}

\item case ESucc: We have $\interpE{\judgeE{\Gamma, \Gamma'}{[e'/y]\s{e}}{\N}}\;\theta\;(\gamma, \gamma')$

  \begin{eqnproof}
    \eline{\interpE{\judgeE{\Gamma, \Gamma'}{\s{[e'/y]e}}{\N}}\;\theta\;(\gamma, \gamma')}
          {Substitution}
    \eline{s(\interpE{\judgeE{\Gamma, \Gamma'}{[e'/y]e}{\N}}\;\theta\;(\gamma, \gamma'))}
          {Semantics}
    \eline{s(\interpE{\judgeE{\Gamma, y:A, \Gamma'}{e}{\N}}\;\theta\;\gamma'')}
          {IH}
    \eline{\interpE{\judgeE{\Gamma, y:A, \Gamma'}{\s{e}}{\N}}\;\theta\;\gamma''}
          {Semantics}
  \end{eqnproof}

  (Where $\gamma'' = (\gamma, \interpE{\judgeE{\Gamma}{e'}{A}}\;\theta\;\gamma, \gamma')$)

\item case EIter: We have $\judgeE{\Gamma, \Gamma'}{[e'/y]\iter{e}{e_0}{x}{e_1}}{C}$
  \begin{eqnproof}
    \eline{\judgeE{\Gamma, \Gamma'}{\iter{[e'/y]e}{[e'/y]e_0}{x}{[e'/y]e_1}}{C}}
          {Substitution}
          [1em]
    \eclaim[(1)]
          {\judgeE{\Gamma, \Gamma'}{[e'/y]e}{\N}}
          {Inversion}
    \eclaim[(2)]
          {\judgeE{\Gamma, \Gamma'}{[e'/y]e_0}{C}}
          {Inversion}
    \eclaim[(3)]
          {\judgeE{\Gamma, \Gamma', x:C}{[e'/y]e_1}{C}}
          {Inversion}
          [1em]
    \eline[\interpE{(1)}]
          {\lambda \theta\;(\gamma, \gamma').\;
            \interpE{\judgeE{\Gamma, y:A, \Gamma'}{e}{\N}}
              \;\theta
              \;(\gamma, \interpE{e'}\;\theta\;\gamma, \gamma')}
          {IH}
    \eline[\interpE{(2)}]
          {\lambda \theta\;(\gamma, \gamma').\;
            \interpE{\judgeE{\Gamma, y:A, \Gamma'}{e_0}{C}}
              \;\theta
              \;(\gamma, \interpE{e'}\;\theta\;\gamma, \gamma')}
          {IH}               
    \eline[\interpE{(3)}]
          {\lambda \theta\;(\gamma, \gamma'').\;
            \interpE{\judgeE{\Gamma, y:A, \Gamma', x:C}{e_0}{C}}
              \;\theta
              \;(\gamma, \interpE{e'}\;\theta\;\gamma, \gamma'')}
          {IH}               
  \end{eqnproof}
  Now, we assume suitable $\theta, \gamma, \gamma'$, and consider the 
  interpretation of 

  $$\interpE{\judgeE{\Gamma, \Gamma'}{\iter{[e'/y]e}{[e'/y]e_0}{x}{[e'/y]e_1}}{C}}\;\theta\;(\gamma, \gamma')$$ 

  This is equal to $iter_C[i,s]\;a$, where:

  \begin{eqnproof}
    \eline[a]{\interpE{(1)}\;\theta\;(\gamma,\gamma')}{}
    \eline{\interpE{\judgeE{\Gamma, y:A, \Gamma'}{e}{\N}}
              \;\theta
              \;(\gamma, \interpE{e'}\;\theta\;\gamma, \gamma')}
          {}
          [1em]
    \eline[i]{\interpE{(2)}\;\theta\;(\gamma, \gamma')}{}
    \eline{\interpE{\judgeE{\Gamma, y:A, \Gamma'}{e_0}{C}}
              \;\theta
              \;(\gamma, \interpE{e'}\;\theta\;\gamma, \gamma')}
          {}
    \eline[s]{\semfun{v}{\interpE{(3)}\;\theta\;(\gamma,\gamma',v)}}
             {}
    \eline{\semfun{v}{
           \interpE{\judgeE{\Gamma, y:A, \Gamma', x:C}{e_1}{C}}
           \;\theta
           \;(\gamma, \interpE{e'}\;\theta\;\gamma, \gamma', v)}}
          {}
  \end{eqnproof}

  Which means that
  \begin{displaymath}
    iter_C[i,s]\;a = \interpE{\judgeE{\Gamma, y:A, \Gamma'}{\iter{e}{[e'/y]e_0}{x}{[e'/y]e_1}}{C}}    \;\theta\;(\gamma, \interpE{e'}\;\theta\;\gamma, \gamma')
  \end{displaymath}

\item case EMonad: We have $\judgeE{\Gamma, \Gamma'}{[e'/y]\comp{c}}{\monad{B}}$. 

  \begin{eqnproof}
    \eline{\judgeE{\Gamma, \Gamma'}{\comp{[e'/y]c}}{\monad{B}}}
          {Substitution}
          [1em]
    \eclaim[\mbox{We have}]
           {\judgeC{\Gamma, \Gamma'}{[e'/y]c}{B}}
           {Inversion}
  \end{eqnproof}

  By mutual induction, we know that for all suitable $\theta, \gamma, \gamma'$,
  \begin{displaymath}
   \interpC{\judgeC{\Gamma,\Gamma'}{[e'/y]c}{B}}\;\theta\;(\gamma,\gamma') = 
   \interpC{\judgeE{\Gamma, y:A, \Gamma'}{c}{B}}
           \;\theta\;
           \;(\gamma, \interpE{e'}\;\theta\;\gamma, \gamma')
  \end{displaymath}

  Therefore we know that $\interpE{\judgeE{\Gamma, \Gamma'}{[e'/y]\comp{c}}{\monad{B}}}\;\theta\;(\gamma, \gamma')$ is

  \begin{eqnproof}
    \eline{\interpC{\judgeC{\Gamma,\Gamma'}{[e'/y]c}{B}}\;\theta\;(\gamma,\gamma)}
          {Semantics}
    \eline{\interpC{\judgeE{\Gamma, y:A, \Gamma'}{c}{B}}
           \;\theta\;
           \;(\gamma, \interpE{e'}\;\theta\;\gamma, \gamma')}
          {See above}
    \eline{\interpE{\judgeE{\Gamma, y:A, \Gamma'}{\comp{c}}{\monad{B}}}\;
           \;\theta\;
           \;(\gamma, \interpE{e'}\;\theta\;\gamma, \gamma')}
          {Semantics}
  \end{eqnproof}

\item case EFix: We have that $\interpE{\judgeE{\Gamma, \Gamma'}{[e'/y](\fix{x:D}{e})}{D}}\;\theta\;(\gamma,\gamma')$ is

  \begin{eqnproof}
    \eline{\interpE{\judgeE{\Gamma, \Gamma'}{\fix{x:D}{([e'/y]e)}}{D}}
           \;\theta\;(\gamma,\gamma')}
          {Substitution}
    \eline{fix(\semfun{v}{
             (\interpE{\judgeE{\Gamma, \Gamma', x:D}{[e'/y]e}{D}}
               \;\theta\;(\gamma,\gamma',v))})}
          {Semantics}
     \eline{fix(\semfun{v}{
             (\interpE{\judgeE{\Gamma, y:A, \Gamma', x:D}{e}{D}}
               \;\theta\;(\gamma,\interpE{e'}\;\theta\;\gamma, \gamma',v))})}
          {IH }
     \eline{\interpE{\judgeE{\Gamma, y:A, \Gamma'}{\fix{x:D}{e}}{D}}
             \;\theta
             \;(\gamma,\interpE{e'}\;\theta\;\gamma, \gamma',v)}
           {Semantics}
  \end{eqnproof}

\item case ETLam: We have that $\interpE{\judgeE{\Gamma,\Gamma'}{[e'/y](\Fun{\alpha}{\kappa}{e})}{\forall \alpha:\kappa.\;B}}\;\theta\;(\gamma,\gamma')$ 

  \begin{eqnproof}
    \eline{\interpE{
             \judgeE{\Gamma,\Gamma'}{\Fun{\alpha}{\kappa}{[e'/y]e}}{\forall \alpha:\kappa.\;B}}\;\theta\;(\gamma, \gamma')}
          {Substitution}
    \eline{\semfun{\tau}{\interpE{
             \;\judgeE[\Theta, \alpha:\kappa]{\Gamma,\Gamma'}{[e'/y]e}{B}}\;(\theta, \tau)\;(\gamma, \gamma')}}
          {Semantics}
    \eline{\semfun{\tau}{\interpE{
             \;\judgeE[\Theta, \alpha:\kappa]{\Gamma,\Gamma'}{e}{B}}
      \;(\theta, \tau)\;(\gamma, \interpE{e'}\;(\theta, \tau)\;\gamma, \gamma')}}
          {IH}
    \eline{\semfun{\tau}{\interpE{
             \;\judgeE[\Theta, \alpha:\kappa]{\Gamma,\Gamma'}{e}{B}}
      \;(\theta, \tau)\;(\gamma, \interpE{e'}\;\theta\;\gamma, \gamma')}}
          {Since $\alpha \not\in FTV(e')$}

    \eline{\interpE{\judgeE{\Gamma, y:A, \Gamma'}
                           {\Fun{\alpha}{\kappa}{e}}
                           {\forall \alpha:\kappa.\;B}}
           \;\theta\;(\gamma, \interpE{e'}\;\theta\;\gamma, \gamma')}
          {Semantics}                 
  \end{eqnproof}

\item case ETApp: We have that $\interpE{\judgeE{\Gamma,\Gamma'}{[e'/y](e\;\tau)}{[\tau/\alpha]B}}\;\theta\;(\gamma, \gamma')$ is 

  \begin{eqnproof}
    \eline{\interpE{
           \judgeE{\Gamma,\Gamma'}{([e'/y]e)\;\tau}{[\tau/\alpha]B}}
           \;\theta\;(\gamma,\gamma')}
          {Substitution}
    \eline{\left(\interpE{\judgeE{\Gamma,\Gamma'}{[e'/y]e}{\forall \alpha:\kappa.\;B}}\;\theta\;(\gamma,\gamma')\right)\;\tau(\theta)}
          {Semantics}
    \eline{\left(
             \interpE{\judgeE{\Gamma,y:A,\Gamma'}{e}{\forall \alpha:\kappa.\;B}}
                 \;\theta\;(\gamma,\interpE{e'}\;\theta\;\gamma,\gamma')
           \right)\; \tau(\theta)}
          {IH}
    \eline{\interpE{
           \judgeE{\Gamma,y:A,\Gamma'}{e\;\tau}{[\tau/\alpha]B}}
           \;\theta\;(\gamma,\interpE{e'}\;\theta\;\gamma, \gamma')}
          {Substitution}
  \end{eqnproof}

\item case EPack: We have that $\interpE{\judgeE{\Gamma,\Gamma'}{[e'/y]\pack{\tau}{e}}{\exists \alpha:\kappa.\;B}}\;\theta\;(\gamma, \gamma')$ is

  \begin{eqnproof}
    \eline{\interpE{
           \judgeE{\Gamma,\Gamma'}
                  {[e'/y]\pack{\tau}{e}}
                  {\exists \alpha:\kappa.\;B}}
           \;\theta\;(\gamma, \gamma')}
          {Substitution}
    \eline{(\tau(\theta), 
            \interpE{
            \judgeE{\Gamma,\Gamma'}
                   {[e'/y]e}
                   {[\tau/\alpha]B}}
            \;\theta\;(\gamma,\gamma'))}
          {Semantics}
    \eline{(\tau(\theta), 
            \interpE{
            \judgeE{\Gamma,y:A,\Gamma'}
                   {e}
                   {[\tau/\alpha]B}}
            \;\theta\;(\gamma,\interpE{e'}\;\theta\;\gamma, \gamma'))}
          {IH}
    \eline{\interpE{
           \judgeE{\Gamma,y:A, \Gamma'}
                  {\pack{\tau}{e}}
                  {\exists \alpha:\kappa.\;B}}
           \;\theta\;(\gamma,\interpE{e'}\;\theta\;\gamma, \gamma')}
          {Semantics}
  \end{eqnproof}
\item case EUnpack: We have $\interpE{\judgeE{\Gamma,\Gamma'}{[e'/y](\unpack{\alpha}{x}{e_1}{e_2})}{B}}\;\theta\;(\gamma,\gamma')$ as

  \begin{eqnproof}
    \eline{\interpE{
           \judgeE{\Gamma,\Gamma'}
                  {\unpack{\alpha}{x}{[e'/y]e_1}{[e'/y]e_2}}
                  {B}}
             \;\theta\;(\gamma,\gamma')}
          {Substitution}
    \eline{(\semfun{\pair{\tau}{v}}
                   {(\interpE{\judgeE[\Theta, \alpha:\kappa]
                             {\Gamma,\Gamma', x:C}
                             {[e'/y]e_2}
                             {B}}
                     \;(\theta,\tau)
                     \;(\gamma, \gamma', v))})}
           {}
    \eclaim{\;\;(\interpE{
            \judgeE{\Gamma,\Gamma'}
                   {[e'/y]e_1}
                   {\exists \alpha:\kappa.\;C}}
            \;\theta\;(\gamma,\gamma'))}
          {Semantics}
    \eline{(\semfun{\pair{\tau}{v}}
                   {(\interpE{\judgeE[\Theta, \alpha:\kappa]
                             {\Gamma, y:A, \Gamma', x:C}
                             {e_2}
                             {B}}
                     \;(\theta,\tau)
                     \;(\gamma, \interpE{e'}\;(\theta,\tau)\;\gamma, \gamma', v))})}
           {}
    \eclaim{\;\;(\interpE{
            \judgeE{\Gamma, y:A, \Gamma'}
                   {e_1}
                   {\exists \alpha:\kappa.\;C}}
            \;\theta\;(\gamma,\interpE{e'}\;\theta\;\gamma, \gamma'))}
          {IH}
    \eline{(\semfun{\pair{\tau}{v}}
                   {(\interpE{\judgeE[\Theta, \alpha:\kappa]
                             {\Gamma, y:A, \Gamma', x:C}
                             {e_2}
                             {B}}
                     \;(\theta,\tau)
                     \;(\gamma, \interpE{e'}\;\theta\;\gamma, \gamma', v))})}
           {}
    \eclaim{\;\;(\interpE{
            \judgeE{\Gamma, y:A, \Gamma'}
                   {e_1}
                   {\exists \alpha:\kappa.\;C}}
            \;\theta\;(\gamma,\interpE{e'}\;\theta\;\gamma, \gamma'))}
          {$\alpha \not\in FTV(e')$}
    \eline{\interpE{
           \judgeE{\Gamma, y:A, \Gamma'}
                  {\unpack{\alpha}{x}{e_1}{e_2}}
                  {B}}
           \;\theta
           \;(\gamma,\interpE{e'}\;\theta\;\gamma, \gamma')}
          {Semantics}
  \end{eqnproof}

\item case EKeq: We have $\interpE{\judgeE{\Gamma,\Gamma'}{[e'/y]e}{B}}\;\theta\;(\gamma,\gamma')$
  \begin{eqnproof}
    \eline{\interpE{\judgeE{\Gamma,\Gamma'}{[e'/y]e}{C}}\;\theta\;(\gamma,\gamma')}
          {Semantics, $B = C$}
    \eline{\interpE{\judgeE{\Gamma,y:A,\Gamma'}{e}{C}}
           \;\theta
           \;(\gamma,\interpE{e'}\;\theta\;\gamma,\gamma')}
          {IH}
    \eline{\interpE{\judgeE{\Gamma,y:A,\Gamma'}{e}{B}}
           \;\theta
           \;(\gamma,\interpE{e'}\;\theta\;\gamma,\gamma')}
          {Semantics}
  \end{eqnproof}
\end{itemize}

Now, the cases for the computation terms follow.

\begin{itemize}
\item case CReturn: We have that $\interpC{\judgeC{\Gamma,\Gamma'}{[e'/y]e}{B}}\;\theta\;(\gamma,\gamma')$
  \begin{eqnproof}
    \eline{\eta_{\interp{B}\theta}(
             \interpE{\judgeE{\Gamma,\Gamma'}{[e'/y]e}{B}}
               \;\theta
               \;(\gamma,\gamma'))}
          {Semantics}
    \eline{\eta_{\interp{B}\theta}(
             \interpE{\judgeE{\Gamma,y:A,\Gamma'}{e}{B}}
               \;\theta
               \;(\gamma,\interpE{e'}\;\theta\;\gamma, \gamma'))}
          {Mutual IH}
    \eline{\eta_{\interp{B}\theta}(
             \interpE{\judgeE{\Gamma,y:A,\Gamma'}{e}{B}}
               \;\theta
               \;(\gamma,\interpE{e'}\;\theta\;\gamma, \gamma'))}
          {Semantics}
  \end{eqnproof}

\item case CLet: We have that $\interpC{\judgeC{\Gamma,\Gamma'}{[e'/y](\letv{x}{e}{c})}{C}} \;\theta\;(\gamma,\gamma')$

  \begin{eqnproof}
    \eline{\interpC{
           \judgeC{\Gamma,\Gamma'}{\letv{x}{[e'/y]e}{[e'/y]c}}{C}} 
           \;\theta\;(\gamma,\gamma')}
          {Substitution}
    \eline{(\semfun{v}{\interpC{\judgeC{\Gamma,\Gamma',x:B}{[e'/y]c}{C}}
                       \;\theta
                       \;(\gamma,\gamma',v)})^*}
          {}
    \eclaim{\;\;(\interpE{\judgeE{\Gamma,\Gamma'}{[e'/y]e}{\monad{B}}}\;
                \theta\;(\gamma,\gamma'))}
           {Semantics}
    \eline{(\semfun{v}{\interpC{\judgeC{\Gamma,y:A,\Gamma',x:B}{c}{C}}
                       \;\theta
                       \;(\gamma,\interpE{e'}\;\theta\;\gamma,\gamma',v)})^*}
          {}
    \eclaim{\;\;(\interpE{\judgeE{\Gamma,y:A,\Gamma'}{e}{\monad{B}}}\;
                \theta\;(\gamma,\interpE{e'}\;\theta\;\gamma,\gamma'))}
           {IH,IH}
    \eline{\interpC{
           \judgeC{\Gamma,y:A,\Gamma'}{\letv{x}{e}{c}}{C}} 
           \;\theta\;(\gamma,\interpE{e'}\;\theta\;\gamma,\gamma')}
          {Semantics}
  \end{eqnproof}

\item case CGet: We have that $\interpC{\judgeC{\Gamma,\Gamma'}
                                               {[e'/y](!e)}{B}}
                                       \;\theta\;(\gamma,\gamma')$

  \begin{eqnproof}
    \eline{\interpC{
           \judgeC{\Gamma,\Gamma'}
                  {!([e'/y]e)}{B}}
                  \;\theta\;(\gamma,\gamma')}
          {Substitution}
    \eline{\lambda k.\;\lambda (L, h).\; 
              \left\{ \begin{array}{ll}
                        k\; (h\;l)\; (L, h) & \mbox{when } l \in L \\
                        \top                & \mbox{otherwise} \\
                      \end{array}
              \right.}
          {Semantics}
    \eclaim
           {\mbox{where }l = \interpE{\judgeE{\Gamma,\Gamma'}
                                             {[e'/y]e}{\reftype{B}}}
                  \;\theta\;(\gamma,\gamma')}
           {}[1em]

    \eline{\interpC{
           \judgeC{\Gamma,y:A,\Gamma'}
                  {!e}{B}}
                  \;\theta\;(\gamma,\interpE{e'}\;\theta\;\gamma, \gamma')}
          {}
    \eclaim[]
          {\mbox{because }l = \interpE{\judgeE{\Gamma,y:A,\Gamma'}
                                              {e}{\reftype{B}}}
                              \;\theta\;(\gamma,
                                         \interpE{e'}\;\theta\;\gamma, 
                                         \gamma')}
          {IH}
  \end{eqnproof}

\item case CSet: We have $\interpC{\judgeC{\Gamma,\Gamma'}
                                          {[e'/y](e_1 := e_2)}{\unittype}}
                                  \;\theta\;(\gamma,\gamma')$
  \begin{eqnproof}
    \eline{\interpC{\judgeC{\Gamma,\Gamma'}
                                          {[e'/y]e_1 := [e'/y]e_2}{\unittype}}
                                  \;\theta\;(\gamma,\gamma')}
          {Substitution}
    \eline{\lambda k.\;\lambda (L, h).\; 
              \left\{ \begin{array}{ll}
                       k \unit (L, [h|l:v]) & \mbox{when } l \in L \\
                      \top                & \mbox{otherwise} \\
              \end{array}
              \right.}
          {Semantics}
    \eclaim{\mbox{where }l = \interpE{\judgeE{\Gamma,\Gamma;}{[e'/y]e_1}{\reftype{B}}}
                               \;\theta\;(\gamma,\gamma')}
           {}
    \eclaim{\mbox{where }v = \interpE{\judgeE{\Gamma,\Gamma;}{[e'/y]e_2}{B}}
                               \;\theta\;(\gamma,\gamma')}
           {}
           [1em]
    \eline{\interpC{\judgeC{\Gamma,y:A,\Gamma'}{e_1 := e_2}{\unittype}}
                   \;\theta
                   \;(\gamma,\interpE{e'}\;\theta\;\gamma,\gamma')}
          {}
    \eclaim{\mbox{because }l = \interpE{\judgeE{\Gamma,y:A,\Gamma}{e_1}{\reftype{B}}}
                               \;\theta\;(\gamma,\interpE{e'}\;\theta\;\gamma,\gamma')}
           {IH}
    \eclaim{\mbox{because }v = \interpE{\judgeE{\Gamma,y:A,\Gamma}{e_2}{B}}
                               \;\theta\;(\gamma,\interpE{e'}\;\theta\;\gamma,\gamma')}
           {IH}
  \end{eqnproof}

\item case CNew: We have that $\interpC{\judgeC{\Gamma,\Gamma'}{[e'/y]\newref{B}{e}}
                                               {\reftype{B}}}
                                       \;\theta\;(\gamma,\gamma')$.
  \begin{eqnproof}
    \eline{\interpC{\judgeC{\Gamma,\Gamma'}{\newref{B}{([e'/y]e)}}{\reftype{B}}}
            \;\theta
            \;(\gamma,\gamma')}
          {Substitution}
    \eline{\lambda k.\;\lambda (L, h).\;\left(
             \begin{array}{l}
               \mbox{let }l = \newloc{L}{A}  \mbox{ in} \\
               k\;l\;(L \cup \setof{l}, [h|l:v]) \\
             \end{array}\right)}
          {}
    \eclaim{\mbox{where }v = \interpE{\judgeE{\Gamma,\Gamma'}
                                             {[e'/y]e}{A}}\;\theta\;(\gamma,\gamma')}
           {Semantics}
    \eline{\interpC{\judgeC{\Gamma,y:A,\Gamma'}{\newref{B}{e}}{\reftype{B}}}
            \;\theta
            \;(\gamma,\interpE{e'}\;\theta\;\gamma,\gamma')}
          {}
    \eclaim{\mbox{because }v = \interpE{\judgeE{\Gamma,\Gamma'}
                                             {e}{A}}\;\theta\;(\gamma,\interpE{e'}\;\theta\;\gamma,\gamma')}
           {IH}
  \end{eqnproof}

    
\end{itemize}
\end{proof}


\begin{lemma*}{(Soundness of Type Substitution)}
If we know that $\judgeWK{\tau}{\kappa}$, then 
\begin{itemize}
\item If $\judgeE[\Theta, \alpha:\kappa]{\Gamma}{e}{A}$, then 
       $\interpE{\judgeE[\Theta, \alpha:\kappa]{\Gamma}{e}{A}}\;(\theta,\tau(\theta))\;\gamma$ is
       equal to \\ $\interpE{\judgeE{[\tau/\alpha]\Gamma}{[\tau/\alpha]e}{[\tau/\alpha]A}}\;\theta\;\gamma$
\item If $\judgeC[\Theta, \alpha:\kappa]{\Gamma}{c}{A}$, then 
       $\interpC{\judgeC[\Theta, \alpha:\kappa]{\Gamma}{c}{A}}\;(\theta,\tau(\theta))\;\gamma$ 
       is equal to \\ $\interpC{\judgeC{[\tau/\alpha]\Gamma}{[\tau/\alpha]c}{[\tau/\alpha]A}}\;\theta\;\gamma$
\end{itemize}
\end{lemma*}

\begin{proof}
  The proof is by induction on the structure of the typing derivation. The
interesting case is: 

\begin{itemize}
\item case EHyp: 

  \begin{eqnproof}
        [\interpE{\judgeE[\Theta, \alpha:\kappa]
                         {\Gamma}
                         {x_i}{A_i}}\;(\theta, {\tau(\theta)})\;\gamma]
    \eline{\pi_i(\gamma)}
          {Semantics}
  \end{eqnproof}

\noindent Now, observe that the tuple $\gamma$ is an element of $\interp{\Theta,\alpha:\kappa \vdash \Gamma}\;(\theta, \tau(\theta))$.

  \begin{eqnproof}[\interp{\Theta,\alpha:\kappa \vdash \Gamma}\;(\theta, {\tau(\theta)}) =]
     \eline{\interp{\interp{\judgeWK[\Theta,\alpha:\kappa]{A_1}{\bigstar}}\;
                       (\theta, \tau(\theta)) 
                    \times \ldots \times 
                    \interp{\judgeWK[\Theta,\alpha:\kappa]{A_n}{\bigstar}}\;
                       (\theta, \tau(\theta))}}
           {}
     \eline{\interp{\judgeWK{[\tau/\alpha]A_1}{\bigstar}}\;\theta 
            \times \ldots \times 
            \interp{\judgeWK{[\tau/\alpha]A_n}{\bigstar}}\;\theta}
           {}
     \eline{\interp{\Theta \vdash [\tau/\alpha]\Gamma}\;\theta}
           {}
  \end{eqnproof}

So $\gamma$ is also an element of $\interp{\Theta \vdash [\tau/\alpha]\Gamma}\;\theta$ as well, and so 

  \begin{eqnproof}[\interpE{\judgeE{[\tau/\alpha]\Gamma}{x_i}{[\tau/\alpha]A_i}}\;\theta\;\gamma=]
    \eline{\pi_i(\gamma)}
          {Semantics}
    \eline{\interpE{\judgeE[\Theta, \alpha:\kappa]
                         {\Gamma}
                         {x_i}{A_i}}\;(\theta, \left<\theta(\tau)\right>)\;\gamma}
          {}
  \end{eqnproof}

\item other cases: 
These all follow the structure of the derivation. 
\end{itemize}
\end{proof}\\



\begin{lemma*}{(Soundness of Equality Rules)}
We have that:
\begin{enumerate}
\item If $\judgeEq{\Gamma}{e}{e'}{A}$, then $\judgeE{\Gamma}{e}{A}$ and 
$\judgeE{\Gamma}{e'}{A}$ and 
$\interpE{\judgeE{\Gamma}{e}{A}} = \interpE{\judgeE{\Gamma}{e'}{A}}$.

\item If $\judgeEqC{\Gamma}{c}{c'}{A}$, then $\judgeC{\Gamma}{c}{A}$ and 
$\judgeC{\Gamma}{c'}{A}$ and 
$\interpC{\judgeE{\Gamma}{c}{A}} = \interpC{\judgeC{\Gamma}{c'}{A}}$.
\end{enumerate}
\end{lemma*}


\begin{proof}
The proof of this theorem is by induction on the derivations of $\judgeEq{\Gamma}{e}{e'}{A}$
and $\judgeEqC{\Gamma}{c}{c'}{A}$. So, assuming we have suitable $\theta$ and $\gamma$, we
proceed as follows:

\begin{itemize}
\item case EqUnit: From $\judgeEq{\Gamma}{e}{e'}{\unittype}$, we have

  \begin{eqnproof}
    \eclaim{\judgeE{\Gamma}{e}{\unittype}}{By inversion}
    \eclaim{\judgeE{\Gamma}{e'}{\unittype}}{By inversion}
  \end{eqnproof}

  \begin{eqnproof}
    \eline[\interpE{\judgeE{\Gamma}{e}{\unittype}}\;\theta\;\gamma]
          {(*)}
          {Semantics}
    \eline{\interpE{\judgeE{\Gamma}{e'}{\unittype}}\;\theta\;\gamma}
          {Semantics}
  \end{eqnproof}

\item case EqPairFst: From $\judgeEq{\Gamma}{\fst{\pair{e_1}{e_2}}}{e_1}{A_1}$, we have:

  \begin{eqnproof}
    \eclaim[1]{\judgeEq{\Gamma}{\fst{\pair{e_1}{e_2}}}{e_1}{A_1}}
              {Hypothesis}
    \eclaim[2]{\judgeE{\Gamma}{\pair{e_1}{e_2}}{A_1\times A_2}}
           {By inversion on 1}
    \eclaim[3]{\judgeE{\Gamma}{e_1}{A_1}}
              {By inversion on 2}
    \eclaim[4]{\judgeE{\Gamma}{\fst{\pair{e_1}{e_2}}}{A_1}}
              {By rule EFst on 2}
  \end{eqnproof}

  \begin{eqnproof}[\interpE{\judgeE{\Gamma}{\fst{\pair{e_1}{e_2}}}{A_1}}\;\theta\;\gamma =]
    \eline{\pi_1(\interpE{\judgeE{\Gamma}{\pair{e_1}{e_2}}{A_1\times A_2}}\;\theta\;\gamma)}
          {Semantics}
    \eline{\pi_1(\sempair{\interpE{\judgeE{\Gamma}{e_1}{A_1}}\;\theta\;\gamma}
                         {\interpE{\judgeE{\Gamma}{e_1}{A_2}}\;\theta\;\gamma})}
          {Semantics}
    \eline{\interpE{\judgeE{\Gamma}{e_1}{A_1}}\;\theta\;\gamma}
          {Products}
  \end{eqnproof}

\item case EqPairSnd: From $\judgeEq{\Gamma}{\snd{\pair{e_1}{e_2}}}{e_1}{A_2}$, we have:

  \begin{eqnproof}
    \eclaim[1]{\judgeEq{\Gamma}{\snd{\pair{e_1}{e_2}}}{e_1}{A_2}}
              {Hypothesis}
    \eclaim[2]{\judgeE{\Gamma}{\pair{e_1}{e_2}}{A_1\times A_2}}
           {By inversion on 1}
    \eclaim[3]{\judgeE{\Gamma}{e_2}{A_2}}
              {By inversion on 2}
    \eclaim[4]{\judgeE{\Gamma}{\snd{\pair{e_1}{e_2}}}{A_2}}
              {By rule EFst on 2}
  \end{eqnproof}

  \begin{eqnproof}[\interpE{\judgeE{\Gamma}{\snd{\pair{e_1}{e_2}}}{A_2}}\;\theta\;\gamma =]
    \eline{\pi_2(\interpE{\judgeE{\Gamma}{\pair{e_1}{e_2}}{A_1\times A_2}}\;\theta\;\gamma)}
          {Semantics}
    \eline{\pi_2(\sempair{\interpE{\judgeE{\Gamma}{e_1}{A_1}}\;\theta\;\gamma}
                         {\interpE{\judgeE{\Gamma}{e_1}{A_2}}\;\theta\;\gamma})}
          {Semantics}
    \eline{\interpE{\judgeE{\Gamma}{e_2}{A_2}}\;\theta\;\gamma}
          {Products}
  \end{eqnproof}

\item case EqPairEta: 
  \begin{eqnproof}
     \eclaim[1]{\judgeEq{\Gamma}{e}{\pair{\fst{e}}{\snd{e}}}{A_1\times A_2}}
               {Hypothesis}
     \eclaim[2]{\judgeE{\Gamma}{e}{A_1\times A_2}}
               {Inversion}
     \eclaim[3]{\judgeE{\Gamma}{\fst{e}}{A_1}}
               {Rule EFst on 2}
     \eclaim[4]{\judgeE{\Gamma}{\snd{e}}{A_2}}
               {Rule ESnd on 2}
     \eclaim[5]{\judgeE{\Gamma}{\pair{\fst{e}}{\snd{e}}}{A_1\times A_2}}
               {Rule EPair on 3, 4}
  \end{eqnproof}

  \begin{eqnproof}[\interpE{\judgeE{\Gamma}{e}{A_1\times A_2}}\;\theta\;\gamma =]
     \eline{\sempair{\pi_1(\interpE{\judgeE{\Gamma}{e}{A_1\times A_2}}\;\theta\;\gamma)}
                     {\pi_2(\interpE{\judgeE{\Gamma}{e}{A_1\times A_2}}\;\theta\;\gamma)}}
            {Products}
     \eline{\sempair{\interpE{\judgeE{\Gamma}{\fst{e}}{A_1}}\;\theta\;\gamma}
                     {\interpE{\judgeE{\Gamma}{\snd{e}}{A_2}}\;\theta\;\gamma}}
            {Semantics $\times$ 2}
     \eline{\interpE{\judgeE{\Gamma}{\pair{\fst{e}}{\snd{e}}}{A_1\times A_2}}\;\theta\;\gamma}
            {Semantics}    
  \end{eqnproof}

\item case EqFunBeta:

  \begin{eqnproof}
    \eclaim[1]{\judgeEq{\Gamma}{(\fun{x}{A}{e})\;e'}{[e'/x]e}{B}}
              {Hypothesis}
    \eclaim[2]{\judgeE{\Gamma}{(\fun{x}{A}{e})\;e'}{B}}
              {Inversion on 1}
    \eclaim[3]{\judgeE{\Gamma}{e'}{A}}
              {Inversion on 2}
    \eclaim[4]{\judgeE{\Gamma}{\fun{x}{A}{e}}{A \to B}}
              {Inversion on 2}
    \eclaim[5]{\judgeE{\Gamma,x:A}{e}{B}}
              {Inversion on 4}
    \eclaim[6]{\judgeE{\Gamma}{[e'/x]e}{B}}
              {Substitution 3 into 5}
  \end{eqnproof}

  \begin{eqnproof}[\interpE{\judgeE{\Gamma}{(\fun{x}{A}{e})\;e'}{B}}\;\theta\;\gamma = ]
    \eline{(\interpE{\judgeE{\Gamma}{\fun{x}{A}{e}}{A \to B}}\;\theta\;\gamma)\;
           (\interpE{\judgeE{\Gamma}{e'}{B}}\;\theta\;\gamma)}
          {Semantics}
    \eline{(\semfun{v}{(\interpE{\judgeE{\Gamma,x:A}{e}{B}}\;\theta\;(\gamma,v))}\;
           \interpE{\judgeE{\Gamma}{e'}{B}\;\theta\;\gamma}}
          {Semantics}
    \eline{\interpE{\judgeE{\Gamma,x:A}{e}{B}}\;\theta\;
             (\gamma,\interpE{\judgeE{\Gamma}{e'}{B}}\;\theta\;\gamma)}
          {Functions}
    \eline{\interpE{\judgeE{\Gamma}{[e'/x]e}{B}}\;\theta\;\gamma}
          {Substitution}
  \end{eqnproof}

\item case EqFunEta: 

  \begin{eqnproof}
    \eclaim[1]{\judgeEq{\Gamma}{e}{e'}{A \to B}}
           {Hypothesis}
    \eclaim[2]{\judgeEq{\Gamma, x:A}{e\;x}{e'\;x}{B}}
              {Inversion}
    \eclaim[3]{\judgeE{\Gamma,x:A}{e}{B}}
              {Induction}
    \eclaim[4]{\judgeE{\Gamma,x:A}{e'}{B}}
              {Induction}
    \eclaim[5]{\judgeE{\Gamma}{e}{B}}
              {since $x\not \in FV(e)$}
    \eclaim[6]{\judgeE{\Gamma}{e'}{B}}
              {since $x\not \in FV(e')$}
  \end{eqnproof}
  \begin{eqnproof}[\mbox{For arbitrary }v,]
    \eline[\interpE{\judgeE{\Gamma,x:A}{e'}{B}}\;\theta\;(\gamma,v)]
          {\interpE{\judgeE{\Gamma,x:A}{e}{B}}\;\theta\;(\gamma,v)}
          {Induction}
    \eline[\interpE{\judgeE{\Gamma,x:A}{e}{B}}\;\theta\;(\gamma,v)]
          {\interpE{\judgeE{\Gamma}{e}{B}}\;\theta\;\gamma}
          {since $x\not \in FV(e)$}
    \eline[\interpE{\judgeE{\Gamma,x:A}{e'}{B}}\;\theta\;(\gamma,v)]
          {\interpE{\judgeE{\Gamma}{e'}{B}}\;\theta\;\gamma}
          {since $x\not \in FV(e')$}
    \eline[\interpE{\judgeE{\Gamma}{e'}{B}}\;\theta\;\gamma]
          {\interpE{\judgeE{\Gamma}{e}{B}}\;\theta\;\gamma}
          {Transitivity}
  \end{eqnproof}

\item case EqSumInlBeta
  \begin{eqnproof}
    \eclaim[1]{\judgeEq{\Gamma}{\Case{\inl{\;e}}{x}{e_1}{y}{e_2}}{[e/x]e_1}{C}}
              {Hypothesis}
    \eclaim[2]{\judgeE{\Gamma}{\Case{\inl{\;e}}{x}{e_1}{y}{e_2}}{C}}
              {Inversion on 1}
    \eclaim[3]{\judgeE{\Gamma}{\inl{\;e}}{A + B}}
              {Inversion on 2}
    \eclaim[4]{\judgeE{\Gamma, x:A}{e_1}{C}}
              {Inversion on 2}
    \eclaim[5]{\judgeE{\Gamma, y:B}{e_2}{C}}
              {Inversion on 2}
    \eclaim[6]{\judgeE{\Gamma}{e}{A}}
              {Inversion on 3}
    \eclaim[7]{\judgeE{\Gamma}{[e/x]e_1}{C}}
              {Substitute 6 into 4}
  \end{eqnproof}
  \begin{eqnproof}[\interpE{\judgeE{\Gamma}{\Case{\inl{\;e}}{x}{e_1}{y}{e_2}}{C}}\;\theta\;\gamma =]
    \eline{[f_1,f_2](a)}
          {Semantics}
    \eclaim[\mbox{where}]
           {\begin{array}{lcl}
               a & = & \interpE{\judgeE{\Gamma}{\inl{\;e}}{A + B}}\;\theta\;\gamma \\
                 & = & \iota_1(\interpE{\judgeE{\Gamma}{e}{A}}\;\theta\;\gamma) \\
               f_1 & = & \semfun{v}{\interpE{\judgeE{\Gamma,x:A}{e_1}{C}}\;\theta\;(\gamma,v)} \\
               f_2 & = & \semfun{v}{\interpE{\judgeE{\Gamma,y:B}{e_2}{C}}\;\theta\;(\gamma,v)} \\
            \end{array}}
           {}[2em]
    \eline[{[f_1,f_2](a)}]
          {[f_1,f_2](\iota_1(\interpE{\judgeE{\Gamma}{e}{A}}\;\theta\;\gamma))}
          {Sums}
    \eline{f_1(\interpE{\judgeE{\Gamma}{e}{A}}\;\theta\;\gamma)}
          {}
    \eline{\interpE{\judgeE{\Gamma,x:A}{e_1}{C}}\;\theta\;(\gamma,\interpE{\judgeE{\Gamma}{e}{A}})}
          {Def of $f_1$}
    \eline{\interpE{\judgeE{\Gamma}{[e/x]e_1}{C}}\;\theta\;\gamma}
          {Substitutition}
  \end{eqnproof}

\item case EqSumInrBeta:
  \begin{eqnproof}
    \eclaim[1]{\judgeEq{\Gamma}{\Case{\inr{\;e}}{x}{e_1}{y}{e_2}}{[e/y]e_2}{C}}
              {Hypothesis}
    \eclaim[2]{\judgeE{\Gamma}{\Case{\inr{\;e}}{x}{e_1}{y}{e_2}}{C}}
              {Inversion on 1}
    \eclaim[3]{\judgeE{\Gamma}{\inr{\;e}}{A + B}}
              {Inversion on 2}
    \eclaim[4]{\judgeE{\Gamma, x:A}{e_1}{C}}
              {Inversion on 2}
    \eclaim[5]{\judgeE{\Gamma, y:B}{e_2}{C}}
              {Inversion on 2}
    \eclaim[6]{\judgeE{\Gamma}{e}{B}}
              {Inversion on 3}
    \eclaim[7]{\judgeE{\Gamma}{[e/y]e_2}{C}}
              {Substitute 6 into 5}
  \end{eqnproof}
  \begin{eqnproof}[\interpE{\judgeE{\Gamma}{\Case{\inr{\;e}}{x}{e_1}{y}{e_2}}{C}}\;\theta\;\gamma =]
    \eline{[f_1,f_2](a)}
          {Semantics}
    \eclaim[\mbox{where}]
           {\begin{array}{lcl}
               a & = & \interpE{\judgeE{\Gamma}{\inr{\;e}}{A + B}}\;\theta\;\gamma \\
                 & = & \iota_2(\interpE{\judgeE{\Gamma}{e}{B}}\;\theta\;\gamma) \\
               f_1 & = & \semfun{v}{\interpE{\judgeE{\Gamma,x:A}{e_1}{C}}\;\theta\;(\gamma,v)} \\
               f_2 & = & \semfun{v}{\interpE{\judgeE{\Gamma,y:B}{e_2}{C}}\;\theta\;(\gamma,v)} \\
            \end{array}}
           {}[2em]
    \eline[{[f_1,f_2](a)}]
          {[f_1,f_2](\iota_2(\interpE{\judgeE{\Gamma}{e}{B}}\;\theta\;\gamma))}
          {Sums}
    \eline{f_2(\interpE{\judgeE{\Gamma}{e}{B}}\;\theta\;\gamma)}
          {}
    \eline{\interpE{\judgeE{\Gamma,y:B}{e_2}{C}}\;\theta\;(\gamma,\interpE{\judgeE{\Gamma}{e}{B}})}
          {Def of $f_1$}
    \eline{\interpE{\judgeE{\Gamma}{[e/y]e_2}{C}}\;\theta\;\gamma}
          {Substitution}
  \end{eqnproof}

\item case EqSumEta:
  \begin{eqnproof}
    \eclaim[1]{\judgeEq{\Gamma}{\Case{e}{x}{[\inl{\;x}/z]e'}{y}{[\inr{\;y}/z]e'}}{[e/z]e'}{C}}
              {Hypothesis}
    \eclaim[2]{\judgeE{\Gamma}{e}{A+B}}
              {Inversion on 1}
    \eclaim[3]{\judgeE{\Gamma, z:A+B}{e'}{C}}
              {Inversion on 1}
    \eclaim[4]{\judgeE{\Gamma, x:A, z:A+B}{e'}{C}}
              {Weakening on 3}
    \eclaim[5]{\judgeE{\Gamma, x:A}{\inl{\;x}}{A+B}}
              {By rules}
    \eclaim[6]{\judgeE{\Gamma, x:A}{[\inl{\;x}/z]e'}{C}}
              {Substitution of 5 into 4}
    \eclaim[7]{\judgeE{\Gamma, y:B, z:A+B}{e'}{C}}
              {Weakening on 3}
    \eclaim[8]{\judgeE{\Gamma, y:B}{\inr{\;y}}{A+B}}
              {By rules}
    \eclaim[9]{\judgeE{\Gamma, y:B}{[\inr{\;y}/z]e'}{C}}
              {Substitution of 8 into 7}
    \eclaim[10]{\judgeE{\Gamma}{\Case{e}{x}{[\inl{\;x}/z]e'}{y}{[\inr{\;y}/z]e'}}{C}}
               {By ECase on 2, 6, 9}
    \eclaim[11]{\judgeE{\Gamma}{[e/z]e'}{C}}
               {Substitution of 2 into 3}

  \end{eqnproof}

Now from the semantics, we know that $\interpE{\judgeE{\Gamma}{e}{A+B}}\;\theta\;\gamma$ is
either equal to some $\iota_i(v_A)$ or some $\iota_2(v_B)$. 

Suppose it is equal $\iota_1(v_A)$. Then,  $\interpE{\judgeE{\Gamma}{\Case{e}{x}{[\inl{\;x}/z]e'}{y}{[\inr{\;y}/z]e'}}{C}}\;\theta\;\gamma$ is equal to 
\begin{eqnproof}
  \eline{\left[
           \begin{array}{l}
            \semfun{v}{\interpE{\judgeE{\Gamma,x:A}{[\inl{\;x}/z]e'}{C}}\;\theta\;(\gamma,v)}, \\
            \semfun{v}{\interpE{\judgeE{\Gamma,y:B}{[\inr{\;y}/z]e'}{C}}\;\theta\;(\gamma,v)}  \\
           \end{array}\right]
         (\iota_1(v_A))}
        {Semantics}
  \eline{\interpE{\judgeE{\Gamma,x:A}{[\inl{\;x}/z]e'}{C}}\;\theta\;(\gamma,v_A)}
        {Sums}  
  \eline{\begin{array}{l}
           \interpE{\judgeE{\Gamma,x:A,z:A+B}{e'}{C}}\;\theta \\
           \qquad (\gamma, v_A, \interpE{\judgeE{\Gamma,x:A}{\inl{\;x}}{A+B}}\;\theta\;(\gamma, v_A)) \\
         \end{array}}
        {Substitution}
  \eline{\interpE{\judgeE{\Gamma,x:A,z:A+B}{e'}{C}}\;\theta\;(\gamma, v_A, \iota_1(v_A))}
        {Semantics}
  \eline{\interpE{\judgeE{\Gamma, z:A+B}{e'}{C}}\;\theta\;(\gamma, \iota_1(v_A))}
        {Since $x \not \in FV(e')$}
  \eline{\interpE{\judgeE{\Gamma, z:A+B}{e'}{C}}\;\theta\;(\gamma, \interpE{\judgeE{\Gamma}{e}{A+B}}\;\theta\;\gamma)}
        {Meaning of $\iota_1(v_A)$}
  \eline{\interpE{\judgeE{\Gamma}{[e/z]e'}{C}}\;\theta\;\gamma}
        {Substitution}
\end{eqnproof}

Suppose it is $\iota_2(v_B)$. Then,
$\interpE{\judgeE{\Gamma}{\Case{e}{x}{[\inl{\;x}/z]e'}{y}{[\inr{\;y}/z]e'}}{C}}\;\theta\;\gamma$
is equal to
\begin{eqnproof}
  \eline{\left[
           \begin{array}{l}
            \semfun{v}{\interpE{\judgeE{\Gamma,x:A}{[\inl{\;x}/z]e'}{C}}\;\theta\;(\gamma,v)}, \\
            \semfun{v}{\interpE{\judgeE{\Gamma,y:B}{[\inr{\;y}/z]e'}{C}}\;\theta\;(\gamma,v)}  \\
           \end{array}\right]
         (\iota_2(v_B))}
        {Semantics}
  \eline{\interpE{\judgeE{\Gamma,y:B}{[\inr{\;y}/z]e'}{C}}\;\theta\;(\gamma,v_B)}
        {Sums}  
  \eline{\begin{array}{l}
           \interpE{\judgeE{\Gamma,y:B,z:A+B}{e'}{C}}\;\theta \\
           \qquad (\gamma, v_B, \interpE{\judgeE{\Gamma,y:B}{\inr{\;y}}{A+B}}\;\theta\;(\gamma, v_B)) \\
         \end{array}}
        {Substitution}
  \eline{\interpE{\judgeE{\Gamma,y:B,z:A+B}{e'}{C}}\;\theta\;(\gamma, v_B, \iota_2(v_B))}
        {Semantics}
  \eline{\interpE{\judgeE{\Gamma, z:A+B}{e'}{C}}\;\theta\;(\gamma, \iota_2(v_B))}
        {Since $x \not \in FV(e')$}
  \eline{\interpE{\judgeE{\Gamma, z:A+B}{e'}{C}}\;\theta\;(\gamma, \interpE{\judgeE{\Gamma}{e}{A+B}}\;\theta\;\gamma)}
        {Meaning of $\iota_2(v_B)$}
  \eline{\interpE{\judgeE{\Gamma}{[e/z]e'}{C}}\;\theta\;\gamma}
        {Substitution}
\end{eqnproof}

\item case EqMonad:

  \begin{eqnproof}
    \eclaim[1]{\judgeEq{\Gamma}{\comp{c}}{\comp{c'}}{\monad{A}}}
             {Hypothesis}
    \eclaim[2]{\judgeEqC{\Gamma}{c}{c'}{A}}
             {Inversion on 1}
    \eclaim[3]{\judgeC{\Gamma}{c}{A}}
             {Mutual Induction on 2}
    \eclaim[4]{\judgeC{\Gamma}{c'}{A}}
             {Induction on 2}
    \eclaim[5]{\judgeE{\Gamma}{\comp{c}}{\monad{A}}}
             {By rule EMonad on 3}
    \eclaim[6]{\judgeE{\Gamma}{\comp{c'}}{\monad{A}}}
             {By rule EMonad on 4}
  \end{eqnproof}

  \begin{eqnproof}
    \eline[\interpE{\judgeE{\Gamma}{\comp{c}}{\monad{A}}}\;\theta\;\gamma]
          {\interpC{\judgeC{\Gamma}{c}{A}}\;\theta\;\gamma}
          {Semantics}
    \eline{\interpC{\judgeC{\Gamma}{c'}{A}}\;\theta\;\gamma}
          {Mutual Induction}
    \eline{\interpE{\judgeE{\Gamma}{\comp{c'}}{\monad{A}}}\;\theta\;\gamma}
          {Semantics}
  \end{eqnproof}

\item case EqFix
  \begin{eqnproof}
    \eclaim[1]{\judgeEq{\Gamma}{\fix{x:D}{e}}{[(\fix{x:D}{e})/x]e}{D}}
              {Hypothesis}
    \eclaim[2]{\judgeE{\Gamma}{\fix{x:D}{e}}{D}}
              {Inversion on 1}
    \eclaim[3]{\judgeE{\Gamma, x:D}{e}{D}}
              {Inversion on 2}
    \eclaim[4]{\judgeE{\Gamma}{[(\fix{x:D}{e})/x]e}{D}}
              {Substitution of 2 into 3}
  \end{eqnproof}
  \begin{eqnproof}[\interpE{\judgeE{\Gamma}{\fix{x:D}{e}}{D}}\;\theta\;\gamma]
    \eline{fix(\semfun{v}{(\interpE{\judgeE{\Gamma, x:D}{e}{D}}\;\theta\;(\gamma,v))})}
          {Semantics}
    \eline{\interpE{\judgeE{\Gamma, x:D}{e}{D}}\;\theta\;(\gamma,
           fix(\semfun{v}{(\interpE{\judgeE{\Gamma, x:D}{e}{D}}\;\theta\;(\gamma,v))}))}
          {Unroll $fix$}
    \eline{\interpE{\judgeE{\Gamma, x:D}{e}{D}}\;\theta\;(\gamma,
            \interpE{\judgeE{\Gamma}{\fix{x:D}{e}}{D}}\;\theta\;\gamma)}
          {Definition}
    \eline{\interpE{\judgeE{\Gamma}{[(\fix{x:D}{e})/x]e}{D}}\;\theta\;\gamma}
          {Substitution}
  \end{eqnproof}

\item case EqNatZBeta:
  \begin{eqnproof}
    \eclaim[1]{\judgeEq{\Gamma}{\iter{\z}{e_0}{x}{e_1}}{e_0}{A}}
              {Hypothesis}
    \eclaim[2]{\judgeE{\Gamma}{\iter{\z}{e_0}{x}{e_1}}{A}}
              {Inversion on 1}
    \eclaim[3]{\judgeE{\Gamma}{\z}{\N}}
              {Inversion on 2}
    \eclaim[4]{\judgeE{\Gamma,x:A}{e_1}{A}}
              {Inversion on 2}
    \eclaim[5]{\judgeE{\Gamma}{e_0}{A}}
              {Inversion on 2}
  \end{eqnproof}
  \begin{eqnproof}[\interpE{\judgeE{\Gamma}{\iter{\z}{e_0}{x}{e_1}}{A}}\;\theta\;\gamma =]
    \eline{\begin{array}{l}
             iter[\interpE{\judgeE{\Gamma}{e_0}{A}}\;\theta\;\gamma,
                \semfun{v}{\interpE{\judgeE{\Gamma,x:A}{e_1}{A}}\;\theta\;(\gamma,v)}] \\
            (\interpE{\judgeE{\Gamma}{\z}{\N}}\;\theta\;\gamma) \\
           \end{array}}
          {Semantics}
    \eline{iter[\interpE{\judgeE{\Gamma}{e_0}{A}}\;\theta\;\gamma,
                \semfun{v}{\interpE{\judgeE{\Gamma,x:A}{e_1}{A}}\;\theta\;(\gamma,v)}]
            (z)}
          {Semantics}
    \eline{\interpE{\judgeE{\Gamma}{e_0}{A}}\;\theta\;\gamma}
          {Iter properties}
  \end{eqnproof}

\item EqNatSBeta
  \begin{eqnproof}
    \eclaim[1]{\judgeEq{\Gamma}{\iter{\s{e}}{e_0}{x}{e_1}}{[\iter{e}{e_0}{x}{e_1}/x]e_1}{A}}
              {Hypothesis}
    \eclaim[2]{\judgeE{\Gamma}{\iter{\s{e}}{e_0}{x}{e_1}}{A}}
              {Inversion on 1}
    \eclaim[3]{\judgeE{\Gamma}{\s{e}}{\N}}
              {Inversion on 2}
    \eclaim[4]{\judgeE{\Gamma}{e_0}{A}}
              {Inversion on 2}
    \eclaim[5]{\judgeE{\Gamma,x:A}{e_1}{A}}
              {Inversion on 2}
    \eclaim[6]{\judgeE{\Gamma}{e}{\N}}
              {Inversion on 3}
    \eclaim[7]{\judgeE{\Gamma}{\iter{e}{e_0}{x}{e_1}}{A}}
              {Rule EIter on 6, 4, 5}
  \end{eqnproof}
  \begin{eqnproof}[\interpE{\judgeE{\Gamma}{\iter{\s{e}}{e_0}{x}{e_1}}{A}}\;\theta\;\gamma =]
    \eline{\begin{array}{l}
              iter[\interpE{\judgeE{\Gamma}{e_0}{A}}\;\theta\;\gamma,
                   \semfun{v}{\interpE{\judgeE{\Gamma,x:A}{e_1}{A}}\;\theta\;(\gamma,v)}] \\
              \;\;(\interpE{\judgeE{\Gamma}{\s{e}}{\N}}\;\theta\;\gamma)
           \end{array}}
          {Semantics}
    \eline{\begin{array}{l}
              iter[\interpE{\judgeE{\Gamma}{e_0}{A}}\;\theta\;\gamma,
                   \semfun{v}{\interpE{\judgeE{\Gamma,x:A}{e_1}{A}}\;\theta\;(\gamma,v)}] \\
              \;\;(s(\interpE{\judgeE{\Gamma}{e}{\N}}\;\theta\;\gamma))
           \end{array}}
          {Semantics}
    \eline{\begin{array}{l}
             \interpE{\judgeE{\Gamma,x:A}{e_1}{A}}\;\theta \\
             \left(\gamma,
              \begin{array}{l}
                iter[\interpE{\judgeE{\Gamma}{e_0}{A}}\;\theta\;\gamma,
                     \semfun{v}{\interpE{\judgeE{\Gamma,x:A}{e_1}{A}}\;\theta\;(\gamma,v)}] \\
                  \;\;(\interpE{\judgeE{\Gamma}{e}{\N}}\;\theta\;\gamma) \\
              \end{array}\right) \\
           \end{array}}
          {Iter}
     \eline{\interpE{\judgeE{\Gamma,x:A}{e_1}{A}}\;\theta
            (\gamma, \interpE{\judgeE{\Gamma}{\iter{e}{e_0}{x}{e_1}}{A}}\;\theta\;\gamma)}
           {Semantics}
     \eline{\interpE{\judgeE{\Gamma}{[\iter{e}{e_0}{x}{e_1}/x]e_1}{A}}\;\theta\;\gamma}
           {Substitution}
  \end{eqnproof}

\item case EqNatEta:
  \begin{eqnproof}
    \eclaim[1]{\judgeEq{\Gamma, n:\N}{\iter{n}{e_0}{x}{e_1}}{e}{A}}
              {Hypothesis}
    \eclaim[2]{\judgeE{\Gamma}{e}{A}}
              {Inversion on 1}
    \eclaim[3]{\judgeE{\Gamma}{e_0}{A}}
              {Inversion on 1}
    \eclaim[4]{\judgeE{\Gamma,n:\N}{e_0}{A}}
              {Weakening on 3}
    \eclaim[5]{\judgeE{\Gamma,x:A}{e_1}{A}}
              {Inversion on 1}
    \eclaim[6]{\judgeE{\Gamma,n:\N,x:A}{e_1}{A}}
              {Weakening on 5}
    \eclaim[7]{\judgeE{\Gamma,n:\N}{n}{\N}}
              {Rule Hyp}
    \eclaim[8]{\judgeE{\Gamma,n:\N}{\iter{n}{e_0}{x}{e_1}}{A}}
              {Rule EIter on 7, 4, 6}
  \end{eqnproof}

Now, assume we have some suitable environment $(\gamma, v)$. So $v$ is a natural
number, and we shall proceed by induction on it. 

\begin{itemize}
  \item case $v = 0$. 

    \begin{eqnproof}[\interpE{\judgeE{\Gamma,n:\N}{e}{A}}\;\theta\;(\gamma,0) = ]
      \eline{\interpE{\judgeE{\Gamma,n:\N}{e}{A}}\;\theta\;(\gamma, \interpE{\judgeE{\Gamma}{\z}{\N}}\;\theta\;\gamma)}
            {Semantics}
      \eline{\interpE{\judgeE{\Gamma}{[\z/n]e}{A}}\;\theta\;\gamma}
            {Substitution}
      \eline{\interpE{\judgeE{\Gamma}{e}{A}}\;\theta\;\gamma}
            {Induction Hypothesis}
      \eline{\interpE{\judgeE{\Gamma,n:\N}{e}{A}}\;\theta\;(\gamma, 0)}
            {Weakening}
    \end{eqnproof}

\item case $v = s(k)$
  By induction, we know \\
$\interpE{\judgeE{\Gamma, n:\N}{e}{A}}\;\theta\;(\gamma, k)$ $=$ $\interpE{\judgeE{\Gamma,n:\N}{\iter{n}{e_0}{x}{e_1}}{A}}\;\theta\;(\gamma, k)$

\ \\

  \begin{eqnproof}[\interpE{\judgeE{\Gamma,n:\N}{\iter{n}{e_0}{x}{e_1}}{A}}\;\theta\;(\gamma,s(k)) =]
    \eline{iter\left[
            \begin{array}{l}
              \interpE{\judgeE{\Gamma, n:\N}{e_0}{A}}\;\theta\;(\gamma,s(k)), \\
              \semfun{v}{\interpE{\judgeE{\Gamma,n:\N,x:A}{e_1}{A}}\;\theta\;(\gamma,s(k),v)} \\
            \end{array}\right](s(k))}
          {Semantics}
    \eline{iter\left[
            \begin{array}{l}
              \interpE{\judgeE{\Gamma}{e_0}{A}}\;\theta\;(\gamma), \\
              \semfun{v}{\interpE{\judgeE{\Gamma,x:A}{e_1}{A}}\;\theta\;(\gamma,v)} \\
            \end{array}\right](s(k))}
          {since $x \not \in FV(e_0), FV(e_1)$}
    \eline{iter\left[
            \begin{array}{l}
              \interpE{\judgeE{\Gamma, m:\N}{e_0}{A}}\;\theta\;(\gamma,k), \\
              \semfun{v}{\interpE{\judgeE{\Gamma,m:\N,x:A}{e_1}{A}}\;\theta\;(\gamma,k,v)} \\
            \end{array}\right](s(k))}
          {By weakening}
    \eline{\interpE{\judgeE{\Gamma,n:\N,x:A}{e_1}{A}}\;\theta\;(\gamma, iter[\ldots](k))}
          {By $iter$}
    \eline{\begin{array}{l}
             \interpE{\judgeE{\Gamma,n:\N,x:A}{e_1}{A}}\;\theta \\
             \;\;(\gamma, k,
               \interpE{\judgeE{\Gamma,n:\N}{\iter{n}{e_0}{x}{e_1}}{A}}\;\theta\;(\gamma, k)) \\
           \end{array}}
          {Semantics}
    \eline{\begin{array}{l}
             \interpE{\judgeE{\Gamma,n:\N,x:A}{e_1}{A}}\;\theta \\
             \;\;(\gamma, k,
                  \interpE{\judgeE{\Gamma, n:\N}{e}{A}}\;\theta\;(\gamma, k)) \\
           \end{array}}
          {Inner Induction}
    \eline{\interpE{\judgeE{\Gamma,n:\N}{[e/x]e_1}{A}}\;\theta\;(\gamma, k)} 
          {Substitution}
  \end{eqnproof}

  \begin{eqnproof}[\interpE{\judgeE{\Gamma, n:\N}{e}{A}}\;\theta\;(\gamma, s(k)) =]
    \eline{\interpE{\judgeE{\Gamma, m:\N, n:\N}{e}{A}}\;\theta\;(\gamma, k, s(k))}
          {Weakening}
    \eline{\interpE{\judgeE{\Gamma, m:\N, n:\N}{e}{A}}\;\theta\;(\gamma, k, \interpE{\judgeE{\Gamma, m:\N}{\s{m}}{\N}}\;\theta\;(\gamma, k))}
          {Semantics}
    \eline{\interpE{\judgeE{\Gamma, m:\N}{[\s{m}/n]e}{A}}\;\theta\;(\gamma, k)}
          {Substitution}
  \end{eqnproof}
\end{itemize}

  These two are equal by appeal to the outer induction hypothesis, which we get via 
  inversion on the original judgement. 

\item case EqAllBeta: 

  \begin{eqnproof}
    \eclaim[1]{\judgeEq{\Gamma}{(\Fun{\alpha}{\kappa}{e})\;\tau}{[\tau/\alpha]e}{[\tau/\alpha]A}}
              {Hypothesis}
    \eclaim[2]{\judgeE{\Gamma}{\Fun{\alpha}{\kappa}{e}}{\forall \alpha:\kappa.\;A}}
              {Inversion on 1}
    \eclaim[3]{\judgeWK{\tau}{\kappa}}
              {Inversion on 1}
    \eclaim[4]{\judgeE[\Theta,\alpha:\kappa]{\Gamma}{e}{A}}
              {Inversion on 2}
    \eclaim[5]{\judgeE{\Gamma}{[\tau/\alpha]e}{[\tau/\alpha]A}}
              {Substitute 3 into 4}
    \eclaim[6]{\judgeE{\Gamma}{(\Fun{\alpha}{\kappa}{e})\;\tau}{[\tau/\alpha]A}}
              {Rule ETApp on 2, 3}
  \end{eqnproof}

  \begin{eqnproof}[\interpE{\judgeE{\Gamma}{(\Fun{\alpha}{\kappa}{e})\;\tau}
                                   {[\tau/\alpha]A}}\;\theta\;\gamma = ]
    \eline{\interpE{\judgeE{\Gamma}{\Fun{\alpha}{\kappa}{e}}{\forall \alpha:\kappa.\;A}}
                   \;\theta\;\gamma
           \;
           \tau(\theta)}
          {Semantics}
    \eline{(\semfun{\sigma}{\interpE{\judgeE[\Theta, \alpha:\kappa]{\Gamma}{e}{A}}\;(\theta, \sigma)\;\gamma}) \tau(\theta)}
          {Semantics}
    \eline{\interpE{\judgeE[\Theta, \alpha:\kappa]{\Gamma}{e}{A}}\;(\theta, \tau(\theta))\;\gamma}
          {}
    \eline{\interpE{\judgeE{\Gamma}{[\tau/\alpha]e}{[\tau/\alpha]A}}\;\theta\;\gamma}
          {Type Substitution}
  \end{eqnproof}

\item case EqAllEta:

  \begin{eqnproof}
    \eclaim[1]{\judgeEq{\Gamma}{e}{e'}{\forall \alpha:\kappa.\;A}}
              {Hypothesis}
    \eclaim[2]{\judgeE{\Gamma}{e}{\forall \alpha:\kappa.\;A}}
              {Inversion on 1}
    \eclaim[3]{\judgeE{\Gamma}{e'}{\forall \alpha:\kappa.\;A}}
              {Inversion on 1}
  \end{eqnproof}

  \begin{eqnproof}
    \eline[\interpE{\judgeE[\Theta, \alpha:\kappa]{\Gamma}{e\;\alpha}{A}}\;(\theta, \sigma)\;\gamma]        {\interpE{\judgeE[\Theta, \alpha:\kappa]{\Gamma}{e'\;\alpha}{A}}\;(\theta, \sigma)\;\gamma}
          {Induction}[1em]

    \eline[\interpE{\judgeE[\Theta, \alpha:\kappa]{\Gamma}{e\;\alpha}{A}}\;(\theta, \sigma)\;\gamma]
          {(\interpE{\judgeE[\Theta, \alpha:\kappa]{\Gamma}{e}{\forall \alpha:\kappa.\;A}}\;(\theta, \sigma)\;\gamma)\;\sigma}
          {Semantics}
    \eline{(\interpE{\judgeE{\Gamma}{e}{\forall \alpha:\kappa.\;A}}\;\theta\;\gamma)\;\sigma}
          {Strengthening}[1em]

    \eline[\interpE{\judgeE[\Theta, \alpha:\kappa]{\Gamma}{e'\;\alpha}{A}}\;(\theta, \sigma)\;\gamma]
          {(\interpE{\judgeE[\Theta, \alpha:\kappa]{\Gamma}{e'}{\forall \alpha:\kappa.\;A}}\;(\theta, \sigma)\;\gamma)\;\sigma}
          {Semantics}
    \eline{(\interpE{\judgeE{\Gamma}{e'}{\forall \alpha:\kappa.\;A}}\;\theta\;\gamma)\;\sigma}
          {Strengthening}[1em]

    \eline[(\interpE{\judgeE{\Gamma}{e}{\forall \alpha:\kappa.\;A}}\;\theta\;\gamma)\;\sigma]
          {(\interpE{\judgeE{\Gamma}{e'}{\forall \alpha:\kappa.\;A}}\;\theta\;\gamma)\;\sigma}
          {Transitivity}[1em]

    \eline[\interpE{\judgeE{\Gamma}{e}{\forall \alpha:\kappa.\;A}}\;\theta\;\gamma]
          {\interpE{\judgeE{\Gamma}{e'}{\forall \alpha:\kappa.\;A}}\;\theta\;\gamma}
          {Extensionality}[1em]
  \end{eqnproof}

\item case EqExistsBeta: 

  \begin{eqnproof}
    \eclaim[1]{\judgeEq{\Gamma}{\unpack{\alpha}{x}{\pack{\tau}{e}}{e'}}{[\tau/\alpha, e/x]e'}{C}}
              {Hypothesis}
    \eclaim[2]{\judgeE{\Gamma}{\pack{\tau}{e}}{\exists \alpha:\kappa.\;A}}
              {Inversion on 1}
    \eclaim[3]{\judgeE[\Theta, \alpha:\kappa]{\Gamma, x:A}{e'}{C}}
              {Inversion on 1}
    \eclaim[4]{\judgeWK{\tau}{\kappa}}
              {Inversion on 2}
    \eclaim[5]{\judgeE{\Gamma}{e}{[\tau/\alpha]A}}
              {Inversion on 2}
    \eclaim[6]{\judgeE{\Gamma}{[\tau/\alpha, e/x]e'}{C}}
              {Substitution of 4,5 into 3}
    \eclaim[7]{\judgeE{\Gamma}{\unpack{\alpha}{x}{\pack{\tau}{e}}{e'}}{C}}
              {By rule EUnpack on 2, 3}
  \end{eqnproof}

  \begin{eqnproof}[\interpE{\judgeE{\Gamma}{\unpack{\alpha}{x}{\pack{\tau}{e}}{e'}}{C}}\;\theta\;\gamma = ]
    \eline{\begin{array}{l}
             (\semfun{(\sigma,v)}{\interpE{\judgeE[\Theta,\alpha:\kappa]{\Gamma,x:A}{e'}{C}}\;(\theta,\sigma)\;(\gamma,v)}) \\
             \;\;\interpE{\judgeE{\Gamma}{\pack{\tau}{e}}{\exists \alpha:\kappa.\;A}}\;\theta\;\gamma \\
           \end{array}}
          {Semantics}
    \eline{\begin{array}{l}
             (\semfun{(\sigma,v)}{\interpE{\judgeE[\Theta,\alpha:\kappa]{\Gamma,x:A}{e'}{C}}\;(\theta,\sigma)\;(\gamma,v)}) \\
             \;\;\left(\tau(\theta), \interpE{\judgeE{\Gamma}{e}{[\tau/\alpha]A}}\;\theta\;\gamma\right) \\
           \end{array}}
          {Semantics}
    \eline{\begin{array}{l}
             \interpE{\judgeE[\Theta,\alpha:\kappa]{\Gamma,x:A}{e'}{C}} \\
             (\theta,\tau(\theta))\;(\gamma, \interpE{\judgeE{\Gamma}{e}{[\tau/\alpha]A}}\;\theta\;\gamma) \\
           \end{array}}
          {Simplify}
    \eline{\interpE{\judgeE{\Gamma,x:A}{[\tau/\alpha, e/x]e'}{C}} \theta\;\gamma}
          {Substitution}
  \end{eqnproof}

\item case EqExistsEta: 
  \begin{eqnproof}
    \eclaim[1]{\judgeEq{\Gamma}{\unpack{\alpha}{x}{e}{[\pack{\alpha}{x}/z]e'}}{[e/z]e'}{B}}
              {Hypothesis}
    \eclaim[2]{\judgeE{\Gamma}{e}{\exists \alpha:\kappa.\;A}}
              {Inversion on 1}
    \eclaim[3]{\judgeE{\Gamma, z:\exists \alpha:\kappa.\;A}{e'}{B}}
              {Inversion on 1}
    \eclaim[4]{\judgeE{\Gamma}{[e/z]e'}{B}}
              {Substitution of 2 into 3}
    \eclaim[5]{\judgeE[\Theta,\alpha:\kappa]{\Gamma, x:A, z:\exists \alpha:\kappa.\;A}{e'}{B}}
              {Weakening on 3}
    \eclaim[6]{\judgeE[\Theta,\alpha:\kappa]{\Gamma, x:A}{\pack{\alpha}{x}}{\exists \alpha:\kappa.\;A}}
              {Rule EPack}
    \eclaim[7]{\judgeE[\Theta,\alpha:\kappa]{\Gamma, x:A}{[\pack{\alpha}{x}/z]e'}{B}}
              {Substitution of 6 into 5}
    \eclaim[8]{\judgeE{\Gamma}{\unpack{\alpha}{x}{e}{[\pack{\alpha}{x}/z]e'}}{B}}
              {Rule EUnpack on 2, 7}
  \end{eqnproof}

  \begin{eqnproof}[\interpE{\judgeE{\Gamma}{\unpack{\alpha}{x}{e}{[\pack{\alpha}{x}/z]e'}}{B}}
                   \;\theta\;\gamma =]
    \eline{\begin{array}{l}
             (\semfun{(\sigma,v)}{\interpE{\judgeE[\Theta, \alpha:\kappa]
                                                  {\Gamma, x:A}{[\pack{\alpha}{x}/z]e'}{B}}
                                \;(\theta,\sigma)\;(\gamma,v)}) \\
           \;\interpE{\judgeE{\Gamma}{e}{\exists \alpha:\kappa.\;A}}
                     \;\theta\;\gamma \\
           \end{array}}
          {Semantics}
    \eline{\begin{array}{l}
             (\lambda (\sigma,v).\;\interpE{\judgeE[\Theta,\alpha:\kappa]
                                                  {\Gamma, x:A, z:\exists \alpha:\kappa.\;A}
                                                  {e'}{B}} \\
             \;\;  (\theta,\sigma) \\
             \;\;  (\gamma,v, 
                      \interpE{\judgeE[\Theta, \alpha:\kappa]
                                      {\Gamma, x:A}{\pack{\alpha}{x}}{\exists \alpha:\kappa.\;A}}
                      \;(\theta,\sigma)\;(\gamma, v)) \\
           \;\interpE{\judgeE{\Gamma}{e}{\exists \alpha:\kappa.\;A}}
                     \;\theta\;\gamma \\
           \end{array}}
          {Substitution}
    \eline{\begin{array}{l}
             (\lambda (\sigma,v).\;\interpE{\judgeE[\Theta,\alpha:\kappa]
                                                  {\Gamma, x:A, z:\exists \alpha:\kappa.\;A}
                                                  {e'}{B}} 
             \;(\theta,\sigma)\; (\gamma,v, (\sigma, v))) \\
           \;\interpE{\judgeE{\Gamma}{e}{\exists \alpha:\kappa.\;A}}
                     \;\theta\;\gamma \\
           \end{array}}
          {Semantics}
    \eline{\begin{array}{l}
             (\lambda (\sigma,v).\;\interpE{\judgeE[\Theta]
                                                  {\Gamma, z:\exists \alpha:\kappa.\;A}
                                                  {e'}{B}} 
             \;\theta\; (\gamma, (\sigma, v))) \\
           \;\interpE{\judgeE{\Gamma}{e}{\exists \alpha:\kappa.\;A}}
                     \;\theta\;\gamma \\
           \end{array}}
          {Strengthening}
    \eline{\begin{array}{l}
             \interpE{\judgeE[\Theta]
                              {\Gamma, z:\exists \alpha:\kappa.\;A}
                              {e'}{B}} 
             \;\theta\; (\gamma, \interpE{\judgeE{\Gamma}{e}{\exists \alpha:\kappa.\;A}}
                                    \;\theta\;\gamma) 
           \end{array}}
          {Semantics}
    \eline{\begin{array}{l}
             \interpE{\judgeE[\Theta]
                              {\Gamma}
                              {[e/z]e'}{B}}\;\theta \;\gamma
           \end{array}}
          {Semantics}
% 
% 
%     \eline{\begin{array}{l}
%              (\lambda (\sigma,v).\;\interpE{\judgeE[\Theta,\alpha:\kappa]
%                                                   {\Gamma, x:A, z:\exists \alpha:\kappa.\;A}
%                                                   {e'}{B}} 
%              \;(\theta,\sigma)\; (\gamma,v, (\sigma, v))) \\
%            \;(\tau(\theta), \interpE{\judgeE{\Gamma}{e}{[\tau/\alpha]A}}\;\theta\;\gamma) \\
%            \end{array}}
%           {Semantics}
%     \eline{\begin{array}{l}
%              \interpE{\judgeE[\Theta,\alpha:\kappa]
%                              {\Gamma, x:A, z:\exists \alpha:\kappa.\;A}
%                              {e'}{B}} \\
%              \; (\theta, \tau(\theta)) \\
%              \; (\gamma, \interpE{\judgeE{\Gamma}{e}{[\tau/\alpha]A}}\;\theta\;\gamma,
%                  (\tau(\theta), \interpE{\judgeE{\Gamma}{e}{[\tau/\alpha]A}}\;\theta\;\gamma))\\
%            \end{array}}
%           {Simplify}
%     \eline{\begin{array}{l}
%              \interpE{\judgeE[\Theta,\alpha:\kappa]
%                              {\Gamma, x:A, z:\exists \alpha:\kappa.\;A}
%                              {e'}{B}} \\
%              \; (\theta, \tau(\theta)) \\
%              \; (\gamma, \interpE{\judgeE{\Gamma}{e}{[\tau/\alpha]A}}\;\theta\;\gamma,
%                  \interpE{\judgeE{\Gamma}{e}{\exists \alpha:\kappa.\;A}}
%                      \;\theta\;\gamma) \\
%            \end{array}}
%           {Semantics}
%     \eline{\begin{array}{l}
%              \interpE{\judgeE[\Theta,\alpha:\kappa]{\Gamma, x:A}{[e/z]e'}{B}} \\
%              \; (\theta, \tau(\theta)) 
%              \; (\gamma, \interpE{\judgeE{\Gamma}{e}{[\tau/\alpha]A}}\;\theta\;\gamma) \\
%            \end{array}}
%           {Substitution}
%     \eline{\interpE{\judgeE{\Gamma}{[e/z]e'}{B}}\;\theta\;\gamma}
%           {$\alpha,x \not \in FV([e/z]e')$}
  \end{eqnproof}

\item case EqCommandEta: 

  \begin{eqnproof}
    \eclaim[1]{\judgeEqC{\Gamma}{c}{\letv{x}{\comp{c}}{x}}{A}}
              {Hypothesis}
    \eclaim[2]{\judgeC{\Gamma}{c}{A}}
              {Inversion on 1}
    \eclaim[3]{\judgeE{\Gamma}{\comp{c}}{\monad{A}}}
              {Rule EMonad on 2}
    \eclaim[4]{\judgeE{\Gamma, x:A}{x}{A}}
              {Rule EHyp}
    \eclaim[5]{\judgeC{\Gamma, x:A}{x}{A}}
              {Rule CReturn on 4}
    \eclaim[6]{\judgeC{\Gamma}{\letv{x}{\comp{c}}{x}}{A}}
              {By Rule CLet on 3,5}
  \end{eqnproof}

  \begin{eqnproof}[\interpC{\judgeC{\Gamma}{\letv{x}{\comp{c}}{x}}{A}}\;\theta\;\gamma =]
    \eline{\begin{array}{l}
             (\semfun{v}{\interpC{\judgeC{\Gamma,x:A}{x}{A}}\;\theta\;(\gamma,v)})^* \\
           \; \interpE{\judgeE{\Gamma}{\comp{c}}{\monad{A}}}\;\theta\;\gamma \\ 
           \end{array}}
          {Semantics}
    \eline{\begin{array}{l}
            (\semfun{v}{\eta(\interpE{\judgeE{\Gamma,x:A}{x}{A}}\;\theta\;(\gamma,v))})^* \\
            \;\interpC{\judgeC{\Gamma}{c}{A}}\;\theta\;\gamma \\
           \end{array}}
          {Semantics}
    \eline{(\semfun{v}{(\eta(v))})^*\;(\interpC{\judgeC{\Gamma}{c}{A}}\;\theta\;\gamma)}
          {Simplify}
    \eline{id(\interpC{\judgeC{\Gamma}{c}{A}}\;\theta\;\gamma)}
          {Monad law}
    \eline{\interpC{\judgeC{\Gamma}{c}{A}}\;\theta\;\gamma}
          {Simplify}
  \end{eqnproof}

\item EqCommandBeta:

  \begin{eqnproof}
    \eclaim[1]{\judgeEqC{\Gamma}{\letv{x}{\comp{e}}{c}}{[e/x]c}{B}}
              {Hypothesis}
    \eclaim[2]{\judgeC{\Gamma}{\letv{x}{\comp{e}}{c}}{B}}
              {Inversion on 1}
    \eclaim[3]{\judgeE{\Gamma}{\comp{e}}{\monad{A}}}
              {Inversion on 2}
    \eclaim[4]{\judgeC{\Gamma,x:A}{c}{B}}
              {Inversion on 2}
    \eclaim[5]{\judgeC{\Gamma}{e}{A}}
              {Inversion on 3}
    \eclaim[6]{\judgeE{\Gamma}{e}{A}}
              {Inversion on 5}
    \eclaim[7]{\judgeC{\Gamma}{[e/x]c}{B}}
              {Substitution of 6 into 4}
  \end{eqnproof}

  \begin{eqnproof}[\interpC{\judgeC{\Gamma}{\letv{x}{\comp{e}}{c}}{B}}\;\theta\;\gamma =]
    \eline{\begin{array}{l}
             (\semfun{v}{\interpC{\judgeC{\Gamma,x:A}{c}{B}}\;\theta\;(\gamma,v)})^* \\
             \;\; (\interpE{\judgeE{\Gamma}{\comp{e}}{\monad{A}}}\;\theta\;\gamma) \\
           \end{array}}
          {Semantics}
    \eline{\begin{array}{l}
             (\semfun{v}{\interpC{\judgeC{\Gamma,x:A}{c}{B}}\;\theta\;(\gamma,v)})^* \\
             \;\; (\interpC{\judgeC{\Gamma}{e}{A}}\;\theta\;\gamma) \\
           \end{array}}
          {Semantics}
    \eline{\begin{array}{l}
             (\semfun{v}{\interpC{\judgeC{\Gamma,x:A}{c}{B}}\;\theta\;(\gamma,v)})^* \\
             \;\; \eta(\interpE{\judgeE{\Gamma}{e}{A}}\;\theta\;\gamma) \\
           \end{array}}
          {Semantics}
    \eline{\begin{array}{l}
             (\semfun{v}{\interpC{\judgeC{\Gamma,x:A}{c}{B}}\;\theta\;(\gamma,v)}) \\
             \;\; (\interpE{\judgeE{\Gamma}{e}{A}}\;\theta\;\gamma) \\
           \end{array}}
          {Monad laws}
    \eline{\interpC{\judgeC{\Gamma,x:A}{c}{B}}
             \;\theta\;(\gamma, \interpE{\judgeE{\Gamma}{e}{A}}\;\theta\;\gamma)}
          {Simplify}
    \eline{\interpC{\judgeC{\Gamma}{[e/x]c}{B}}\;\theta\;\gamma}
          {Substitution}
  \end{eqnproof}

\item EqCommandAssoc:
  \begin{eqnproof}
    \eclaim[1]{\scriptsize \judgeEqC{\Gamma}{\letv{x}{\comp{\letv{y}{e}{c_1}}}{c_2}}
                                {\letv{y}{e}{\letv{x}{\comp{c_1}}{c_2}}}{C}}
              {Hypothesis}
    \eclaim[2]{\judgeC{\Gamma}{\letv{x}{\comp{\letv{y}{e}{c_1}}}{c_2}}{C}}
              {Inversion on 1}
    \eclaim[3]{\judgeE{\Gamma}{\comp{\letv{y}{e}{c_1}}}{\monad{B}}}
              {Inversion on 2}
    \eclaim[4]{\judgeC{\Gamma,x:B}{c_2}{C}}
              {Inversion on 2}
    \eclaim[5]{\judgeC{\Gamma}{\letv{y}{e}{c_1}}{B}}
              {Inversion on 3}
    \eclaim[6]{\judgeE{\Gamma}{e}{\monad{A}}}
              {Inversion on 5}
    \eclaim[7]{\judgeC{\Gamma,y:A}{c_1}{B}}
              {Inversion on 5}
    \eclaim[8]{\judgeC{\Gamma,y:A,x:B}{c_2}{C}}
              {Weakening on 4}
    \eclaim[9]{\judgeE{\Gamma,y:A}{\comp{c_1}}{\monad{B}}}
              {Rule EMonad on 7}
    \eclaim[10]{\judgeC{\Gamma,y:A}{\letv{x}{\comp{c_1}}{c_2}}{C}}
               {Rule CLet on 9, 8}
    \eclaim[11]{\judgeC{\Gamma}{\letv{y}{e}{\letv{x}{\comp{c_1}}{c_2}}}{C}}
               {Rule CLet on 6, 10}
  \end{eqnproof}

  \begin{eqnproof}[\interpC{\judgeC{\Gamma}{\letv{x}{\comp{\letv{y}{e}{c_1}}}{c_2}}{C}}
                   \;\theta\;\gamma =]
    \elines{(\semfun{v_2}{\interpC{\judgeC{\Gamma,x:B}{c_2}{C}}\;\theta\;(\gamma,v_2)})^* \\
            \; \interpE{\judgeE{\Gamma}{\comp{\letv{y}{e}{c_1}}}{\monad{B}}}\;\theta\;\gamma \\}
           {Semantics}
    \elines{(\semfun{v_2}{\interpC{\judgeC{\Gamma,x:B}{c_2}{C}}\;\theta\;(\gamma,v_2)})^* \\
            \; \interpC{\judgeC{\Gamma}{\letv{y}{e}{c_1}}{\monad{B}}}\;\theta\;\gamma \\}
           {Semantics}
    \elines{(\semfun{v_2}{\interpC{\judgeC{\Gamma,x:B}{c_2}{C}}\;\theta\;(\gamma,v_2)})^* \\
            \; ((\semfun{v_1}{\interpC{\judgeC{\Gamma,y:A}{c_1}{B}}\;\theta\;(\gamma,v_1)})^* \\
            \;\;\; \interpE{\judgeE{\Gamma}{e}{\monad{A}}}\;\theta\;\gamma) \\ }
           {Semantics}
    \elines{(\lambda v_1.\; (\semfun{v_2}{\interpC{\judgeC{\Gamma,x:B}{c_2}{C}}\;\theta\;(\gamma,v_2)})^* \\
            \;\;(\interpC{\judgeC{\Gamma,y:A}{c_1}{B}}\;\theta\;(\gamma,v_1)))^* \\  
            \qquad \interpE{\judgeE{\Gamma}{e}{\monad{A}}}\;\theta\;\gamma) \\}
           {Monad Laws}
    \elines{(\lambda v_1.\; (\semfun{v_2}{\interpC{\judgeC{\Gamma,y:A,x:B}{c_2}{C}}\;\theta\;(\gamma,v_1, v_2)})^* \\
            \;\;(\interpC{\judgeC{\Gamma,y:A}{c_1}{B}}\;\theta\;(\gamma,v_1)))^* \\  
            \qquad \interpE{\judgeE{\Gamma}{e}{\monad{A}}}\;\theta\;\gamma) \\}
           {Weakening}
    \elines{(\lambda v_1.\; (\semfun{v_2}{\interpC{\judgeC{\Gamma,y:A,x:B}{c_2}{C}}\;\theta\;(\gamma,v_1, v_2)})^* \\
            \;\;(\interpE{\judgeE{\Gamma,y:A}{\comp{c_1}}{\monad{B}}}\;\theta\;(\gamma,v_1)))^* \\  
            \qquad \interpE{\judgeE{\Gamma}{e}{\monad{A}}}\;\theta\;\gamma) \\}
           {Semantics}
    \elines{(\lambda v_1.\;
              \interpC{\judgeC{\Gamma,y:A}{\letv{x}{\comp{c_1}}{c_2}}{C}}
                      \;\theta\;(\gamma,v_1))^* \\
            \qquad \interpE{\judgeE{\Gamma}{e}{\monad{A}}}\;\theta\;\gamma \\}
           {Semantics}
    \eline{\interpC{\judgeC{\Gamma}{\letv{y}{e}{\letv{x}{c_1}{c_2}}}{C}}\;\theta\;\gamma}
          {Semantics}
  \end{eqnproof}

\item EqRefl
  \begin{eqnproof}
    \eclaim[1]{\judgeEq{\Gamma}{e}{e}{A}}
              {Hypothesis}
    \eclaim[2]{\judgeE{\Gamma}{e}{A}}
              {Inversion on 1}
  \end{eqnproof}

  \begin{eqnproof}
    \eline[\interpE{\judgeE{\Gamma}{e}{A}}\;\theta\;\gamma] 
          {\interpE{\judgeE{\Gamma}{e}{A}}\;\theta\;\gamma}
          {Reflexivity}
  \end{eqnproof}

\item EqSymm
   \begin{eqnproof}
     \eclaim[1]{\judgeEq{\Gamma}{e}{e'}{A}}
               {Hypothesis}
     \eclaim[2]{\judgeEq{\Gamma}{e'}{e}{A}}
               {Inversion on 1}
     \eclaim[3]{\judgeE{\Gamma}{e}{A}}
               {Induction on 2}
     \eclaim[4]{\judgeE{\Gamma}{e'}{A}}
               {Induction on 2}
   \end{eqnproof}

   \begin{eqnproof}
     \eline[\interpE{\judgeE{\Gamma}{e'}{A}}\;\theta\;\gamma]
           {\interpE{\judgeE{\Gamma}{e}{A}}\;\theta\;\gamma}
           {Induction on 2, above}
     \eline[\interpE{\judgeE{\Gamma}{e}{A}}\;\theta\;\gamma]
           {\interpE{\judgeE{\Gamma}{e'}{A}}\;\theta\;\gamma}
           {Symmetry on prev step}
   \end{eqnproof}

\item case EqTrans
  \begin{eqnproof}
    \eclaim[1]{\judgeEq{\Gamma}{e}{e''}{A}}
              {Hypothesis}
    \eclaim[2]{\judgeE{\Gamma}{e}{e'}{A}}
              {Inversion on 1}
    \eclaim[3]{\judgeE{\Gamma}{e'}{e''}{A}}
              {Inversion on 1}
    \eclaim[4]{\judgeE{\Gamma}{e}{A}}
              {Induction on 2}
    \eclaim[5]{\judgeE{\Gamma}{e''}{A}}
              {Induction on 3}
  \end{eqnproof}

  \begin{eqnproof}
    \eline[\interpE{\judgeE{\Gamma}{e}{A}}\;\theta\;\gamma]
          {\interpE{\judgeE{\Gamma}{e'}{A}}\;\theta\;\gamma}
          {Induction}
    \eline[\interpE{\judgeE{\Gamma}{e'}{A}}\;\theta\;\gamma]
          {\interpE{\judgeE{\Gamma}{e''}{A}}\;\theta\;\gamma}
          {Induction}
    \eline[\interpE{\judgeE{\Gamma}{e}{A}}\;\theta\;\gamma]
          {\interpE{\judgeE{\Gamma}{e''}{A}}\;\theta\;\gamma}
          {Transitivity}
  \end{eqnproof}

\item case EqSubst:
  \begin{eqnproof}
    \eclaim[1]{\judgeEq{\Gamma}{[e_2/x]e_1}{[e'_2/x]e'_1}{B}}
              {Hypothesis}
    \eclaim[2]{\judgeEq{\Gamma,x:A}{e_1}{e'_1}{B}}
              {Inversion on 1}
    \eclaim[3]{\judgeEq{\Gamma}{e_2}{e'_2}{A}}
              {Inversion on 1}
    \eclaim[4]{\judgeE{\Gamma,x:A}{e_1}{B}}
              {Induction on 2}
    \eclaim[5]{\judgeE{\Gamma,x:A}{e'_1}{B}}
              {Induction on 2}
    \eclaim[6]{\judgeE{\Gamma}{e_2}{A}}
              {Induction on 3}
    \eclaim[7]{\judgeE{\Gamma}{e'_2}{A}}
              {Induction on 3}
    \eclaim[8]{\judgeE{\Gamma}{[e_2/x]e_1}{B}}
              {Substitution of 6 into 4}
    \eclaim[9]{\judgeE{\Gamma}{[e'_2/x]e'_1}{B}}
              {Substitution of 7 into 5}
  \end{eqnproof}

  \begin{eqnproof}[\interpE{\judgeE{\Gamma}{[e_2/x]e_1}{B}}\;\theta\;\gamma =]
     \eline{\interpE{\judgeE{\Gamma,x:A}{e_1}{B}}\;\theta\;
              (\gamma,\interpE{\judgeE{\Gamma}{e_2}{A}}\;\theta\;\gamma)} 
           {Substitution}
     \eline{\interpE{\judgeE{\Gamma,x:A}{e_1}{B}}\;\theta\;
              (\gamma,\interpE{\judgeE{\Gamma}{e'_2}{A}}\;\theta\;\gamma)} 
           {Induction}
     \eline{\interpE{\judgeE{\Gamma,x:A}{e'_1}{B}}\;\theta\;
              (\gamma,\interpE{\judgeE{\Gamma}{e'_2}{A}}\;\theta\;\gamma)} 
           {Induction}
     \eline{\interpE{\judgeE{\Gamma}{[e'_2/x]e'_1}{B}}\;\theta\;\gamma}
           {Substitution}
  \end{eqnproof}

\item EqCommandRefl
  \begin{eqnproof}
    \eclaim[1]{\judgeEqC{\Gamma}{c}{c}{A}}
              {Hypothesis}
    \eclaim[2]{\judgeC{\Gamma}{c}{A}}
              {Inversion on 1}
  \end{eqnproof}

  \begin{eqnproof}
    \eline[\interpC{\judgeC{\Gamma}{c}{A}}\;\theta\;\gamma] 
          {\interpC{\judgeC{\Gamma}{c}{A}}\;\theta\;\gamma}
          {Reflexivity}
  \end{eqnproof}

\item EqCommandSymm
   \begin{eqnproof}
     \eclaim[1]{\judgeEqC{\Gamma}{c}{c'}{A}}
               {Hypothesis}
     \eclaim[2]{\judgeEqC{\Gamma}{c'}{c}{A}}
               {Inversion on 1}
     \eclaim[3]{\judgeC{\Gamma}{c}{A}}
               {Induction on 2}
     \eclaim[4]{\judgeC{\Gamma}{c'}{A}}
               {Induction on 2}
   \end{eqnproof}

   \begin{eqnproof}
     \eline[\interpC{\judgeC{\Gamma}{c'}{A}}\;\theta\;\gamma]
           {\interpC{\judgeC{\Gamma}{c}{A}}\;\theta\;\gamma}
           {Induction on 2, above}
     \eline[\interpC{\judgeC{\Gamma}{c}{A}}\;\theta\;\gamma]
           {\interpC{\judgeC{\Gamma}{c'}{A}}\;\theta\;\gamma}
           {Symmetry on prev step}
   \end{eqnproof}

\item case EqCommandTrans
  \begin{eqnproof}
    \eclaim[1]{\judgeEqC{\Gamma}{c}{c''}{A}}
              {Hypothesis}
    \eclaim[2]{\judgeC{\Gamma}{c}{c'}{A}}
              {Inversion on 1}
    \eclaim[3]{\judgeC{\Gamma}{c'}{c''}{A}}
              {Inversion on 1}
    \eclaim[4]{\judgeC{\Gamma}{c}{A}}
              {Induction on 2}
    \eclaim[5]{\judgeC{\Gamma}{c''}{A}}
              {Induction on 3}
  \end{eqnproof}

  \begin{eqnproof}
    \eline[\interpC{\judgeC{\Gamma}{c}{A}}\;\theta\;\gamma]
          {\interpC{\judgeC{\Gamma}{c'}{A}}\;\theta\;\gamma}
          {Induction}
    \eline[\interpC{\judgeC{\Gamma}{c'}{A}}\;\theta\;\gamma]
          {\interpC{\judgeC{\Gamma}{c''}{A}}\;\theta\;\gamma}
          {Induction}
    \eline[\interpC{\judgeC{\Gamma}{c}{A}}\;\theta\;\gamma]
          {\interpC{\judgeC{\Gamma}{c''}{A}}\;\theta\;\gamma}
          {Transitivity}
  \end{eqnproof}

\item case EqCommandSubst:
  \begin{eqnproof}
    \eclaim[1]{\judgeEqC{\Gamma}{[e_2/x]c_1}{[e'_2/x]c'_1}{B}}
              {Hypothesis}
    \eclaim[2]{\judgeEqC{\Gamma,x:A}{c_1}{c'_1}{B}}
              {Inversion on 1}
    \eclaim[3]{\judgeEq{\Gamma}{e_2}{e'_2}{A}}
              {Inversion on 1}
    \eclaim[4]{\judgeC{\Gamma,x:A}{c_1}{B}}
              {Induction on 2}
    \eclaim[5]{\judgeC{\Gamma,x:A}{c'_1}{B}}
              {Induction on 2}
    \eclaim[6]{\judgeE{\Gamma}{e_2}{A}}
              {Induction on 3}
    \eclaim[7]{\judgeE{\Gamma}{e'_2}{A}}
              {Induction on 3}
    \eclaim[8]{\judgeC{\Gamma}{[e_2/x]c_1}{B}}
              {Substitution of 6 into 4}
    \eclaim[9]{\judgeC{\Gamma}{[e'_2/x]c'_1}{B}}
              {Substitution of 7 into 5}
  \end{eqnproof}

  \begin{eqnproof}[\interpC{\judgeC{\Gamma}{[e_2/x]c_1}{B}}\;\theta\;\gamma =]
     \eline{\interpC{\judgeC{\Gamma,x:A}{c_1}{B}}\;\theta\;
              (\gamma,\interpE{\judgeE{\Gamma}{e_2}{A}}\;\theta\;\gamma)} 
           {Substitution}
     \eline{\interpC{\judgeC{\Gamma,x:A}{c_1}{B}}\;\theta\;
              (\gamma,\interpE{\judgeE{\Gamma}{e'_2}{A}}\;\theta\;\gamma)} 
           {Induction}
     \eline{\interpC{\judgeC{\Gamma,x:A}{c'_1}{B}}\;\theta\;
              (\gamma,\interpE{\judgeE{\Gamma}{e'_2}{A}}\;\theta\;\gamma)} 
           {Induction}
     \eline{\interpC{\judgeC{\Gamma}{[e'_2/x]c'_1}{B}}\;\theta\;\gamma}
           {Substitution}
  \end{eqnproof}

\end{itemize}
\end{proof}

\chapter{The Semantics of Separation Logic}

In this chapter, I will describe the semantics of our separation
logic. Rather than working directly with the heap model of separation
logic, we will approach the semantics in a somewhat more modular
style.

First, we will define what we mean by ``semantics of separation logic''
in terms of \emph{BI algebras}, which give an algebraic semantics of
separation logic in the same way that the Heyting algebras give
semantics to intuitionistic logics. Then, we will look at how we can
construct a BI-algebra from sets of elements of an arbitrary partial
commutative monoid, proving that we can satisfy each of the axioms of
a BI algebra.

Then, we will show that our predomain of heaps actually forms a
partial commutative monoid, which means that we can now apply the
theorems and definitions of the previous sections to immediately get
the heap model we want.

With a semantic definition of assertions in hand, we will then move on
to the semantics of specifications. We will again play the algebraic
game, and give a Kripke semantics for specifications. We will also
give a semantic interpretation of Hoare triples which validates the
frame rule and fixed point induction.

After defining the semantics of assertions and specifications, I will
give their syntax.

\section{BI Algebras}

A \emph{BI algebra} is a Heyting algebra with additional residuated
monoidal structure to model the separating conjunction and wand. This
means that a BI algebra is a partial order $(B, \leq)$ with operations
$(\top, \land, \implies, \bot, \vee, I, *, \wand)$ satisfying:

\begin{enumerate}
\item $\forall p \in B.\; p \leq \top$
\item $\forall p \in B.\; \bot \leq p$
\item $\forall p,q,r \in B.$ if $r \leq p$ and $r \leq q$, then
      $r \leq p \land q$ and 
      $p \land q \leq p$ and $p \land q \leq q$
\item $\forall p,q,r \in B.$ if $p \leq r$ and $q \leq r$, then
      $p \vee q \leq r$ and
      $p \leq p \vee q$ and $q \leq p \vee q$.
\item $\forall p, q, r.\; p \land q \leq r \iff p \leq q \implies r$
\item $\forall p.\; p * I = p$
\item $\forall p, q.\; p * q = q * p$
\item $\forall p, q, r.\; (p * q) * r = p * (q * r)$
\item $\forall p, q, r.\; p * q \leq r \iff p \leq q \wand r$
\end{enumerate}

The first five conditions are just the usual conditions for a Heyting
algebra (that it have greatest and least elements, greatest lower
bounds and least upper bounds, and that it have an implication). The
next three are the monoid structure axioms, that say $I$ is the unit,
and that $*$ is commutative and associative. The last axiom asserts
the existence of a wand that is adjoint to the separating conjunction
the same way that the implication is adjoint to the ordinary
conjunction.

In addition, we will also ask that this algebra be \emph{complete},
which means that meets and joins of arbitrary sets of elements be
well-defined.

\begin{enumerate}
\item[10.] $\forall r\in B, P \subseteq B.$ if $(\forall p \in P.\; r \leq p)$, then  
      $r \leq \bigwedge P $ and 
      $\forall p \in P.\; \bigwedge P \leq p$
\item[11.] $\forall r \in B, P \subseteq B.$ if $(\forall p \in P.\; p \leq r)$, then  
      $\bigvee P \leq r$ and 
      $\forall p \in P.\; p \leq \bigvee P$
\end{enumerate}

We will eventually use completeness in order to interpret quantifiers as 
possibly-infinitary conjunctions or disjunctions.
 
\subsection{Partial Commutative Monoids}

A partial commutative monoid is a triple $(M, e, \cdot)$, where $e \in
M$, and $(\cdot)$ is a partial operation from $M \times M$ to $M$. We
will write $m \# m'$ to mean that $m \cdot m'$ is defined. 

Furthermore, the following properties must hold:

\begin{itemize}
\item $e$ is a unit, so that $e \cdot m = m$ and $e \# m$. 
\item $(\cdot)$ is commutative. If $m_1 \# m_2$, then $m_2 \# m_1$ and $m_1 \cdot m_2 = m_2 \cdot m_1$. 
\item $(\cdot)$ is associative. 
  \begin{itemize}
  \item If $m_1 \# m_2$ and $(m_1 \cdot m_2) \# m_3$, then $m_2 \# m_3$ and $m_1 \# (m_2 \cdot m_3)$ and $(m_1 \cdot m_2) \cdot m_3 = m_1 \cdot (m_2 \cdot m_3)$.
  \item If $m_2 \# m_3$ and $m_1 \# (m_2 \cdot m_3)$, then $m_1 \# m_2$ and $(m_1 \cdot m_2) \# m_3$ and $(m_1 \cdot m_2) \cdot m_3 = m_1 \cdot (m_2 \cdot m_3)$.
  \end{itemize}
  
  
\end{itemize}

\subsection{BI-Algebras over Partial Commutative Monoids}

Given a partial commutative monoid, we can show that the powerset $\powerset{M}$ 
forms a BI-algebra. 

\begin{lemma}{(Powersets of Partial Commutative Monoids)}
Given a partial commutative monoid $(M, e, \cdot)$, its powerset
$(\powerset{M}, \subseteq)$ forms a BI-algebra
with the following operations:

\begin{itemize}
\item $\top = M$
\item $p \land q = p \cap q$
\item $p \implies q = \setof{m \in M \;|\;\mbox{if }m \in p \mbox{ then } m \in q}$
\item $\bot = \emptyset$ 
\item $p \vee q = p \cup q$
\item $I = \setof{e}$
\item $p * q = \setof{m \in M\;|\; \exists m_1, m_2.\; 
                       m_1 \# m_2 \mbox{ and } m_1 \in p \mbox{ and } m_2 \in q \mbox{ and } m_1\cdot m_2 = m}$
\item $p \wand q = \{ m \in M \;|\; \forall m' \in p.\; \mbox{if } m \# m' \mbox{ then } m \cdot m' \in q \}$
\item $\bigwedge P = \bigcap P$ 
\item $\bigvee P = \bigcup P$
\end{itemize}
\end{lemma}


\begin{proof}
\begin{enumerate}
\item  We want to show $\forall p \in \powerset{M}.\; p \leq \top$
  \begin{tabbedproof}
        Assume $p \in \powerset{M}$ \\
    \oo By definition of powerset, we have $p \subseteq M$ \\
    \oo By definition of $\top$, we have $p \subseteq \top$ \\
    \oo By definition of $\leq$, we have $p \leq \top$ \\
  \end{tabbedproof}

\item We want to show $\forall p \in B.\; \bot \leq p$
  \begin{tabbedproof}
        Assume $p \in \powerset{M}$ \\ 
    \oo By definition of $\emptyset$, we have $\emptyset \subseteq p$ \\
    \oo By definition of $\bot$, $\leq$, we have $\bot \leq p$ \\
  \end{tabbedproof}

\item We want to show $\forall p,q,r \in \powerset{M}.$ if $r \leq p$ and $r \leq q$, then
      $r \leq p \land q$ and 
      $p \land q \leq p$ and $p \land q \leq q$

   \begin{tabbedproof}
      Assume $p,q,r \in \powerset{M}$ \\
      Assume $r \leq p$, $r \leq q$ \\[1em]

      \oo Expanding definition of $\leq$, we have $r \subseteq p$ and $r \subseteq q$ \\
      \oo By properties of $\cap$, we have $r \cap r \subseteq p \cap q$ \\
      \oo Hence $r \subseteq p \cap q$ \\
      \oo By definition of $\leq$ and $\land$, we have $r \leq p \land q$ \\[1em]
      \oo By properties of $\cap$, we have $p \cap q \subseteq p$ \\
      \oo By definition of $\leq$ and $\land$, we have $p \wedge q \leq p$ \\[1em]
      \oo By properties of $\cap$, we have $p \cap q \subseteq q$ \\
      \oo By definition of $\leq$ and $\land$, we have $p \wedge q \leq q$ \\
   \end{tabbedproof}

\item We want to show $\forall p,q,r \in \powerset{M}.$ if $p \leq r$ and $q \leq r$, then
      $p \vee q \leq r$ and
      $p \leq p \vee q$ and $q \leq p \vee q$.

  \begin{tabbedproof}
    Assume $p, q, r \in \powerset{M}$, $p \leq r$, $q \leq $ \\[1em]
    \oo By definition of $\leq$, we have $p \subseteq r$, $q \subseteq r$ \\
    \oo By set properties, we have $p \cup q \subseteq r$ \\
    \oo By definition of $\leq$, we have $p \vee q \leq r$ \\[1em]

    \oo By set properties we have $p \subseteq p \cup q$ \\
    \oo By definitions of $\leq$ and $\vee$, we have $p \leq p \vee q$ \\[1em]
  
    \oo By set properties, we have $q \subseteq p \cup q$ \\
    \oo By definitions of $\leq$ and $\vee$, we have  $q \leq p \vee q$ \\
  \end{tabbedproof}

\item We want to show $\forall p, q, r \in \powerset{M}.\; p \land q \leq r \iff p \leq q \implies r$

  \begin{tabbedproof}
    Assume $p,q,r \in \powerset{M}$ \\[1em]

    $\To$ direction: \\
    \oo Assume $p \land q \leq r$ \\
    \ooo By definitions of $\leq$ and $\land$, we have $p \cap q \subseteq r$ \\
    \ooo We want to show $p \leq q \implies r$ \\
    \ooo So we want to show $p \subseteq (q \implies r)$ \\
    \ooo So we want to show $\forall m$, if $m \in p$ then $m \in (q \implies r)$ \\
    \ooo Assume $m$ and $m \in p$ \\
    \oooo Want to show $m \in q \implies r$ \\
    \oooo Equivalent to showing if $m \in q$, then $m \in r$ \\
    \oooo Assume $m \in q$ \\
    \ooooo Since $m \in p$ and $m \in q$, we know $m \in p \cap q$. \\
    \ooooo Since $p \cap q \subseteq r$, we know $m \in r$ \\
    \ooo Therefore $p \subseteq q \implies r$ \\
    \ooo By definition of $\leq$, we see $p \leq q \implies r$ \\[1em]

    $\Leftarrow$ direction:\\
    \oo Assume $p \leq q \implies r$ \\
    \ooo By definition of $\leq$, we know $p \subseteq q \implies r$ \\
    \ooo We want to show $p \land q \leq r$ \\
    \ooo So we want to show $p \cap q \subseteq r$ \\
    \ooo So we want to show $\forall m$, if $m \in p \cap q$ then $m \in r$ \\
    \ooo Assume $m \in p \cap q$ \\
    \oooo Hence $m \in p$ and $m \in q$ \\
    \oooo Since $p \subseteq q \implies r$, we know for all $m$, if $m \in p$, then if $m \in q$, then $m \in r$ \\
    \oooo Hence $m \in r$ \\
    \ooo Hence $p \cap q \subseteq r$ \\
    \ooo By definition of $\leq$ and $\land$, we conclude $p \land q \leq r$ \\
  \end{tabbedproof}

\item We want to show $\forall p \in \powerset{M}.\; p * I = p$
  \begin{tabbedproof}
    \oo Assume $p \in \powerset{M}$ \\
    \ooo We want to show $p * I = p$ \\ 
    \ooo This is equivalent to showing for all $m \in M$, that $m \in p * I$ if and only if $m \in p$ \\
    \ooo Assume $m \in M$ \\
    \oooo $\To$ direction: \\
    \ooooo Assume $m \in p * I$ \\
    \ooooo Therefore $\exists m_1, m_2 \in M$ such that \\
    \ooooox $m_1 \# m_2$ and $m_1 \in p$ and $m_2 \in I$ and $m = m_1 \cdot m_2$\\
    \ooooo Let $m_1, m_2$ be witnesses, so that \\
    \oooooo $m_1 \# m_2$ and $m_1 \in p$ and $m_2 \in I$ and $m = m_1 \cdot m_2$\\
    \oooooo Since $I = \setof{e}$, we know $m_2 = e$ \\
    \oooooo By unit property, $m = m_1 \cdot e = m_1$ \\
    \oooooo Since $m_1 \in p$, we know $m \in p$ \\
    \ooo $\From$ direction: \\
    \oooo Assume $m \in p$ \\
    \ooooo We want to show $m \in p * I$ \\ 
    \ooooo So we want to show there are $m_1, m_2$ \\
    \ooooox such that $m_1 \# m_2$ and $m_1 \in p$ and $m_2 \in I$ and $m = m_1 \cdot m_2$ \\
    \ooooo Choose $m_1$ to be $m$, and $m_2$ to be $e$ \\
    \ooooo So we want to show $m \# e$ and $m \in p$ and $e \in I$ and $m = m \cdot e$ \\
    \ooooo By properties of unit, $m \# e$ and $m = m \cdot e$ \\
    \ooooo By definition of $I$, we know $e \in I$ \\
    \ooooo We know $m \in p$ by hypothesis \\
    \ooooo Therefore goal in line 16 met. \\ 
  \end{tabbedproof}

\item We want to show $\forall p, q \in \powerset{M}.\; p * q = q * r$

  \begin{eqnproof}[\mbox{Assume }p,q \in \powerset{M}]
    \eline[p * q]
          {\setof{m \in M\;|\; \exists m_1 \in p, m_2 \in q.\; 
                       m_1 \# m_2 \land m_1\cdot m_2 = m}}
          {Definition}
    \eline{\setof{m \in M\;|\; \exists m_2 \in q, m_1 \in p.\; 
                       m_1 \# m_2 \land m_1\cdot m_2 = m}}
          {Logical manipulation}
    \eline{\setof{m \in M\;|\; \exists m_2 \in q, m_1 \in p.\; 
                       m_2 \# m_1 \land m_2\cdot m_1 = m}}
          {Commutativity}
    \eline{q * p}
          {Definition}    
  \end{eqnproof}

\item We want to show $\forall p, q, r \in \powerset{M}.\; (p * q) * r = p * (q * r)$

  \begin{tabbedproof}
    \oo Assume $p, q, r \in \powerset{M}$ \\
    \oo We want to show $(p * q) * r = p * (q * r)$ \\
    \oo This is equivalent to showing for all $m \in M$, that \\
    \ox $m \in (p * q) * r$ if and only if $ m \in p * (q * r)$ \\
    \oo Assume $m \in M$ \\
    \ooo $\To$ direction: \\
    \oooo Assume $m \in (p * q) * r$ \\
    \ooooo Therefore there are $m_{pq}, m_r$ such that \\
    \ooooox $m_{pq} \# m_r$ and $m = m_{pq} \cdot m_r$ and $m_{pq} \in p * q$ and $m_r \in r$ \\
    \ooooo From $m_{pq} \in p * q$, we know there are $m_p$, $m_q$ such that \\
    \ooooox $m_p \# m_q$ and $m_{pq} = m_p \cdot m_q$ and $m_p \in p$ and $m_q$ in $r$ \\
    \ooooo By $m_{pq} = m_p \cdot m_q$, we know $m = (m_p \cdot m_q) \cdot m_r$ and $(m_p \cdot m_q) \# m_r$\\
    \ooooo Since $m_p \# m_q$ and $(m_p \cdot m_q) \# m_r$, by associativity we know \\
    \ooooox $m_q \# m_r$ and $m_p \# (m_q \cdot m_r)$ and 
            $m = (m_p \cdot m_q) \cdot m_r = m_p \cdot (m_q \cdot m_r)$ \\
    \ooooo Since $m_q \# m_r$ and $m_q \cdot m_r = m_q \cdot m_r$ and $m_q \in q$ and $m_r \in r$, \\
    \ooooox we know $m_q \cdot m_r \in q * r$ \\
    \ooooo Since $m_p \# (m_q \cdot m_r)$ and $m_p \cdot (m_q \cdot m_r) = m_p \cdot (m_q \cdot m_r)$ \\
    \ooooox and $m_p \in p$ and $m_q \cdot m_r \in q * r$,  we know $m_p \cdot (m_q \cdot m_r) \in p * (q * r)$ \\
    \ooooo Therefore $m \in p * (q * r)$ \\
    \ooo $\From$ direction: \\
    \oooo Assume we have $m_p, m_{qr}$ such that \\
    \oooox $m_p \# m_{qr}$ and $m = m_p \cdot m_{qr}$ and $m_p \in p$ and $m_{qr} \in q * r$ \\
    \ooooo Therefore we have $m_q, m_r$ such that \\
    \ooooox $m_q \# m_{r}$ and $m_{qr} = m_q \cdot m_r$ and $m_q \in q$ and $m_r \in r$ \\
    \ooooo From $m_{qr} = m_q \cdot m_r$ we have \\
    \ooooox $m = m_p \cdot (m_q \cdot m_r)$ and $m_p \# (m_q \cdot m_r)$ \\
    \ooooo By associativity with $m_q \# m_{r}$ and $m_p \# (m_q \cdot m_r)$, we have \\
    \ooooox $m_p \# m_q$ and $(m_p \cdot m_q) \# m_r$ and 
            $m = m_p \cdot (m_q \cdot m_r) = (m_p \cdot m_q) \cdot m_r$ \\
    \ooooo Since $m_p \# m_q$ and $m_p \in p$ and $m_q \in q$, we know \\
    \ooooox $(m_p \cdot m_q) \in p * q$ \\
    \ooooo Since $(m_p \cdot m_q) \# m_r$ and $(m_p \cdot m_q) in p * q$ and $m_r \in r$, we have \\ 
    \ooooox $(m_p \cdot m_q) \cdot m_r \in (p * q) * r$ \\
    \ooooo Therefore $m \in (p * q) * r$ \\
  \end{tabbedproof}

\item We want to show that for all $p, q, r \in \powerset{M}$, $p * q \leq r$ if and
only if $p \leq q \wand r$. 

  \begin{tabbedproof}
    Assume $p, q, r \in \powerset{M}$. \\
    \oo First, we will give the $\To$ direction. \\
    \oo Assume $p * q \leq r$ \\
    \oo So we know 
          $\comprehend{m}{\exists m_1, m_2.\; m_1 \# m_2 \mbox{ and } m_1 \in p \mbox{ and } m_2 \in q 
                                             \mbox{ and } m_1 \cdot m_2 = m}
           \subseteq r$ \\
    \oo Thus, $\forall m.\; (\exists m_1, m_2.\; m_1 \# m_2 \mbox{ and } m_1 \in p \mbox{ and } m_2 \in q 
                                             \mbox{ and } m_1 \cdot m_2 = m)$ implies $m \in r$. \\
    \oo By turning existentials on the left into universals, and instantiating $m$, \\
    \ooo
             $\forall m_1, m_2.\; (m_1 \# m_2 \mbox{ and } m_1 \in p \mbox{ and } m_2 \in q) \mbox{ implies } (m_1 \cdot m_2) \in r$ [HYP1] \\
    \oo Now, to show $p \leq q \wand r$, we must show
         for all $m$, if $m \in p$, that $m \in q \wand r$. \\
    \oo Assume $m$, $m \in p$. [HYP2] \\
    \ooo Now, we want to show $m \in \comprehend{m}{\forall m' \in q.\; m\# m' \mbox{ and } m' \mbox{ implies } (m \cdot m') \in r}$ \\
    \ooo This is equivalent to showing $\forall m' \in q.\; m\# m' \mbox{ and } m' \mbox{ implies } (m \cdot m') \in r$ \\
    \ooo Assume $m'$, $m' \in q$, $m \# m'$. [HYP3] \\ 
    \oooo Instantiating [HYP1] with $m$ and $m'$ and the hypotheses in [HYP2] and [HYP3], \\
    \oooo we can conclude $(m \cdot m') \in r$.  \\[1em]

    \oo Now, we will show the $\Leftarrow$ direction. \\
    \oo Assume $p \leq q \wand r$. \\
    \oo So we know, $p \subseteq \comprehend{m}{\forall m' \in q. m \# m' \mbox{ and } m' \in q \mbox{ implies } (m \cdot m') \in r}$ \\
    \oo Therefore $\forall m. m \in p \mbox{ implies } \forall m' \in q.\; m \# m' \mbox{ and } m' \in q \mbox{ implies } (m \cdot m') \in r$.  \\
    \oo Therefore $\forall m, m'.\; m \in p \mbox{ and } m' \in q \mbox{ and } m \# m' \mbox{ implies } (m \cdot m') \in r$ \\
    \oo Therefore $\forall m_o, m, m'.\; m \in p \mbox{ and } m' \in q \mbox{ and } m \# m' \mbox{ and } m_o = m\cdot m' \mbox{ implies } m_o \in r$ \\
    \oo Therefore $\forall m_o, (\exists m, m'.\; m \in p \mbox{ and } m' \in q \mbox{ and } m \# m' \mbox{ and } m_o = m\cdot m') \mbox{ implies } m_o \in r$ \\
    \oo Therefore $\forall m_o, m_o \in (p * q) \mbox{ implies } m_o \in r$ \\
    \oo Therefore $p * q \subseteq r$ \\
    \oo Therefore $p * q \leq r$ \\

  \end{tabbedproof}

\item We want to show $\forall r, P \subseteq B$, if $(\forall p \in P.\; r \leq p)$, then
$r \leq \bigwedge P$ and $\forall p \in P.\; \bigwedge P \leq p$. 

  \begin{tabbedproof}
    \oo Assume $r$, $P$, $P \subseteq B$, and $(\forall p \in P.\; r \leq p)$ \\[1em]
    \ooo First, we want to show $r \leq \bigwedge P$. \\
    \oooo This is equivalent to showing $r \subseteq \bigcap P$ \\
    \oooo This is equivalent to showing that for all $m \in r$, $m \in \bigcap P$. \\
    \oooo Assume $m$, $m \in r$.  \\
    \ooooo Showing $m \in \bigcap P$ is equivalent to $\forall p \in P. m \in p$ \\
    \ooooo Assume $p \in P$.  \\
    \oooooo Instantiating hypothesis with $p$, $r \subseteq p$. \\
    \oooooo This means $\forall m'. m' \in r \implies m' \in p$. \\
    \oooooo Instantiating $m'$ with $m$, we learn $m \in p$. \\
    \oooo Therefore, $r \leq \bigwedge P$. \\[1em]
    \ooo Second, we want to show that $\forall p \in P.\; \bigwedge P \leq p$. \\
    \ooo Assume $p$, $p \in P$. \\
    \oooo Now, we want to show $\bigwedge P \leq p$. \\
    \oooo This is equivalent to showing $\bigcap P \subseteq p$. \\
    \oooo This is equivalent to showing $\forall m.\; m \in \bigcap P \implies m \in p$ \\
    \oooo Assume $m$, $m \in \bigcap P$.  \\
    \ooooo Therefore, $\forall p' \in P.\; m \in p'$. \\
    \ooooo Instantiating $p'$ with $p$, we get $m \in p$. \\
  \end{tabbedproof}

\item We want to show $\forall r, P \subseteq B$, if $(\forall p \in P.\; p \leq r)$, then
$\bigvee P \leq r$ and $\forall p \in P.\; p \leq \bigvee P$. 

  \begin{tabbedproof}
    \oo Assume $r, P \subseteq B$, and $(\forall p \in P.\; p \leq r)$ \\[1em]
    \ooo First, we want to show $\bigvee P \leq r$ \\
    \oooo This is equivalent to showing $\bigcup P \subseteq r$ \\
    \oooo This is equivalent to showing $\forall m.\; m \in \bigcup P \implies m \in r$ \\
    \oooo Assume $m$, $m \in \bigcup P$. \\
    \ooooo $m \in \bigcup P$ is equivalent to $\exists p \in P.\; m \in p$ \\
    \ooooo Suppose $p' \in P$ is the witness, and that $m \in p'$ \\ 
    \oooooo Instantiating the quantifier $p$ in the hypothesis, we get $p' \leq r$ \\
    \oooooo This means $\forall m', m' \in p' \implies m' \in r$ \\
    \oooooo Instantiating the quantifier $m'$ with $m$,  we conclude $m \in r$.  \\
    \oooo Therefore, $\forall m.\; m \in \bigcup P \implies m \in r$ \\[1em]
    \ooo Second, we want to show $\forall p \in P.\; p \leq \bigvee P$. \\
    \oooo Assume $p$, $p \in P$ \\
    \ooooo We want to show $p \leq \bigvee P$ \\
    \ooooo This is equivalent to showing $p \subseteq \bigcup P$ \\
    \ooooo This is equivalent to showing $\forall m.\; m \in p \implies m \in \bigcup P$ \\
    \ooooo This is equivalent to showing $\forall m.\; m \in p \implies \exists p' \in P.\; m \in p'$ \\
    \ooooo Assume $m$, $m \in p$ \\
    \oooooo Take $p'$ to be $p$, since $p \in P$. \\
    \oooooo Thus, $m \in p$ by hypothesis \\
    \oooo Therefore, $p \leq \bigvee P$ \\
  \end{tabbedproof}

\end{enumerate}


\end{proof}

\subsection{Sets of Heaps form a BI algebra}

Now, we will take our predomain $H$ and form a BI algebra from it. To do so, we will
map it with the forgetful functor $U$, so that we forget the partial order structure 
and let $U(H)$ be an ordinary set. 

Now, $U(H) = \sum L \in \powersetfin{Loc}.\; (\prod \sempair{n}{A} \in L.\; \interp{\judgeWK[\cdot]{A}{\bigstar}}\;(*)\;(K,K))$

In what follows, we will usually suppress the $U$, in order to reduce clutter. 

Now, we can define the operations on it as follows: 

\begin{itemize}
\item The unit element $e \triangleq \sempair{\emptyset}{\emptyset}$
\item The operation $(L, f) \cdot (L', g)$ is defined when $L \cap L' = \emptyset$, and
      is equal to 

\begin{displaymath}
 \left(L \cup L', \lambda x.\;\left\{\begin{array}{ll}
                                 f(x) & \mathsf{when}\; x \in L \\
                                 g(x) & \mathsf{when}\; x \in L'
                               \end{array}
                         \right.\right)
 \end{displaymath}
\end{itemize}

The lambda-expression in the operation definition actually defines a
function. Since we know that since $L$ and $L'$ are disjoint, this
means that any element of $x \in L \cup L'$ is exclusively either in
$L$ or $L'$, which means that the definition is unambiguous. Since $f$
and $g$ are well-typed with respect to the index sets $L$ and $L'$
respectively, our new function must be as well.

Now, we can check the properties. 

\begin{itemize}

\item First, we will check that $e$ is a unit. Suppose we have $m = (L, f) \in H$. 

\begin{eqnproof}
  \eline[(\emptyset, \emptyset) \cdot (L, f)]
        {\left(\emptyset \cup L, \lambda x.\; \left\{\begin{array}{ll}
                                                      \emptyset(x) & \mathsf{when}\; x \in \emptyset \\
                                                       f(x) & \mathsf{when}\; x \in L \\
                                                     \end{array}
                                              \right.\right)}
        {Definition}
  \eline{\left(L, \lambda x.\; \begin{array}{ll}
                                  f(x) & \mathsf{when}\; x \in L \\
                               \end{array} \right)}
        {Simplification}
  \eline{\sempair{L}{f}}
        {Simplification}
\end{eqnproof}

\item Second, we will check commutativity. Suppose we have $(L,f) \in H$ and $(L',g) \in H$. 

  First, it's obviously the case that if $L \cap L' = \emptyset$, then $L' \cap L = \emptyset$. 

  \begin{eqnproof}
    \eline[(L,f) \cdot (L',g)] 
          { \left(L \cup L', \lambda x.\;\left\{\begin{array}{ll}
                                 f(x) & \mathsf{when}\; x \in L \\
                                 g(x) & \mathsf{when}\; x \in L'
                               \end{array}
                         \right.\right)}
          {Definition}
    \eline{\left(L' \cup L, \lambda x.\;\left\{\begin{array}{ll}
                                 f(x) & \mathsf{when}\; x \in L \\
                                 g(x) & \mathsf{when}\; x \in L' \\
                               \end{array}
                         \right.\right)}
          {Commutativity of $\cup$}
    \eline{\left(L' \cup L, \lambda x.\;\left\{\begin{array}{ll}
                                 g(x) & \mathsf{when}\; x \in L' \\
                                 f(x) & \mathsf{when}\; x \in L \\
                               \end{array}
                         \right.\right)}
          {Reordering Cases}
    \eline{(L',g) \cdot (L,f)}
          {Definition}
  \end{eqnproof}

\item Now, we will check associativity. Suppose $(L,f)$, $(L',g)$, and $(L'',h)$ are in $H$. 

  First, assume that $L \cap L' = \emptyset$ and that $(L \cup L') \cap L'' = \emptyset$. 
  Then we know that $L \cap L'' = \emptyset$ and $L' \cap L'' = \emptyset$, so we can 
  conclude that $L \cap (L' \cup L'') = \emptyset$. 

  Second, assume that $L' \cap L'' = \emptyset$ and that $L \cap (L' \cup L'') = \emptyset$. 
  Then we know that $L \cap L' = \emptyset$ and $L \cap L'' = \emptyset$, so we can 
  conclude that $(L \cup L') \cap L'' = \emptyset$. 

  \begin{eqnproof}
    \eline[(L,f)\cdot((L',g)\cdot(L'',h))]
          {(L,f)\cdot \left(L' \cup L'', 
                            \lambda x.\; \left\{
                                  \begin{array}{ll}
                                     g(x) & \mathsf{when}\; x \in L' \\
                                     h(x) & \mathsf{when}\; x \in L'' \\
                                  \end{array}\right.\right)}
          {Definition}
     \eline{\left(L \cup (L' \cup L''), 
                            \lambda x.\; \left\{
                                  \begin{array}{ll}
                                     f(x) & \mathsf{when}\; x \in L \\
                                     g(x) & \mathsf{when}\; x \in L' \\
                                     h(x) & \mathsf{when}\; x \in L'' \\
                                  \end{array}\right.\right)}
           {Definition}
     \eline{\left((L \cup L') \cup L'', 
                            \lambda x.\; \left\{
                                  \begin{array}{ll}
                                     f(x) & \mathsf{when}\; x \in L \\
                                     g(x) & \mathsf{when}\; x \in L' \\
                                     h(x) & \mathsf{when}\; x \in L'' \\
                                  \end{array}\right.\right)}
           {Associativity}
      \eline{\left(L \cup L', 
                    \lambda x.\; \left\{
                                  \begin{array}{ll}
                                     f(x) & \mathsf{when}\; x \in L \\
                                     g(x) & \mathsf{when}\; x \in L' \\
                                  \end{array}\right.\right) \cdot (L'', h)}
            {Definition}
    \eline{((L,f)\cdot(L',g))\cdot(L'',h)}
          {Definition}
  \end{eqnproof}
\end{itemize}

Therefore, we can conclude that sets of heaps form a complete BI algebra. 

\section{Semantics of Specifications}

Before we can give the semantics of specifications, we are going to run
into a very sticky issue: our denotational semantics does not validate
the frame property. The formal semantics of the $\newref{A}{e}$
command allocates a new reference by finding the largest numeric id of
any reference in the heap's domain, and then allocating a reference
whose numeric id is one greater than that.

This means that the behavior of the memory allocator is deterministic,
which means that our semantic domain can include awkward programs
which crash if the heap is larger than a certain size. Obviously,
safety monotonicity property cannot hold for such programs, because
extending the heap enough will cause these programs to crash.  
 
On the other hand, we do not actually want to prove the correctness of
any of these pathological programs: all the programs we actually want
to write and prove correct are actually well-behaved.

To do this, we will adapt an idea of Birkedal and
Yang~\cite{birkedal-yang}. They proposed changing the interpretation
of program specifications from a boolean semantics (in which each
specification is either true or false) into a Kripke interpretation.

The modal frame they proposed was one in which worlds are sets of
assertions of separation logic (i.e., elements of a BI algebra).
Intuitively, we can think of each world as ``the set of assertions
that can be safely framed onto this specification''. A specification
is then true when all assertions can be framed onto it, which is how
we will end up justifying the frame rule.

As we did for assertions, we will give this semantics in a modular
way. We will first define a preorder on elements of a BI algebra --- the
extension ordering --- and then use this ordering to give the truth
values as upwards-closed sets of assertions.

\section{World Preorders}

We can define a world preorder $W(B, \worldleq)$ over a BI algebra $B$ as
follows.  The elements of $W(B, \leq)$ are the elements of $B$, and for any
two elements $p$ and $q$, the ordering $p \worldleq q$ is defined as
follows:

\begin{displaymath}
p \worldleq q \iff \exists r.\; p * r = q
\end{displaymath}

To verify the relation $\worldleq$ is a preorder, we need to 
show it is reflexive and transitive. 

$p \worldleq p$ holds because we can take $r$ to be $I$. 


To show transitivity, we must show that $p_1 \worldleq p_3$, given
that $p_1 \worldleq p_2$ and $p_2 \worldleq p_3$.

\begin{tabbedproof}
\oo Assume $p_1 \worldleq p_2$  \\
\oo Assume $p_2 \worldleq p_3$  \\
\ooo By definition of $\worldleq$, $\exists r.\; p_1 * r = p_2$ \\
\ooo By definition of $\worldleq$, $\exists r'.\; p_2 * r' = p_3$ \\
\ooo Let $r$ and $r'$ be the witnesses in lines 3 and 4, so we have \\
\oooo $p_1 * r = p_2$ \\
\oooo $p_2 * r' = p_3$ \\
\oooo Substituting for $p_2$, we get $p_1 * r * r' = p_3$ \\
\oooo Taking as witness $r'' = r * r'$, we show $\exists r''.\; p_1 * r'' = p_3$ \\
\ooo By definition of $\worldleq$, $p_1 \worldleq p_3$ \\
\end{tabbedproof}

\section{Heyting Algebras over Preorders}

Given any preorder $(P, \worldleq)$, we can construct a complete
Heyting algebra by considering the set of its upward-closed subsets
$(\upset{P}, \subseteq)$:

\begin{displaymath}
\upset{P} = 
  \{ S \in \mathcal{P}(P) \;|\;
     \forall p \in S, \forall q \in B. \mbox{ if } p \worldleq q \mbox{ then } q \in S 
  \}
\end{displaymath}

The ordering relation for the Heyting algebra is set inclusion, and
the operations are:

\begin{itemize}
\item $\top = P$
\item $\bot = \emptyset$
\item $S \land S' = S \cap S'$
\item $S \vee S' = S \cup S'$
\item $\bigwedge_{i \in I} S_i = \bigcap_{i \in I} S_i$
\item $\bigvee_{i \in I} S_i = \bigcup_{i \in I} S_i$
\item $S \implies S' = \{ r \in P \;|\; \forall r' \worldgeq r.\; 	
                          \mbox{if } r' \in S \mbox{ then } r' \in S' \}$
\end{itemize}

The idea here is that we will take the world preorder over assertions
(ordered by the extension order $\worldleq$) and use it to define a 
Heyting algebra, whose elements will become the semantics of specifications.


To show that these operations actually form a Heyting algebra, we need
to show that they satisfy the Heyting algebra axioms. First, we need
to show that the meet and join are the greatest lower bounds and least
upper bounds respectively. To do this, we will just show that arbitrary
meets and joins exist, and then the nullary and binary meets and joins
will fall out as a special case.

\begin{lemma}{(Meets in the algebra of specifications)}
If $X \subseteq \upset{P}$, then $\bigwedge X$ defines a meet. 
\end{lemma}

\begin{proof}
We need to show that if for all $i \in I$, $S_i \in \upset{P}$, 
then $\bigwedge_{i \in I} S_i \in \upset{P}$. First, we will verify that 
the intersection of a family of upwards-closed subsets is itself an
upwards-closed subset. 

\begin{tabbedproof}
\oo Assume $\forall i \in I.\; S_i \in \upset{P}$ \\
\ooo We want to show $\bigwedge_{i \in I} S_i \in \upset{P}$ \\ 
\ooo So we want to show for all $x \in \bigwedge_{i \in I} S_i$ and for all $y \in P$, if $x \worldleq y$ then $y \in \bigwedge_{i \in I} S_i$ \\
\ooo Assume $x$, $x \in \bigwedge_{i \in I} S_i$, $y$, $y \in P$, $x \worldleq y$ \\
\oooo Since $\bigwedge_{i \in I} S_i = \bigcap_{i \in I} S_i$, we know 
      $\forall i \in I.\; x \in S_i$ \\
\oooo Assume $i \in I$ \\
\ooooo Since $x \in S_i$, $x \worldleq y$, and $S_i$ is upwards-closed, $y \in S_i$ \\
\oooo Therefore, $\forall i \in I$, $y \in S_i$ \\
\ooo Therefore for all $x \in \bigwedge_{i \in I} S_i$ and for all $y \in P$, if $x \worldleq y$ then $y \in \bigwedge_{i \in I} S_i$ \\
\ooo Which means $\bigwedge_{i \in I} S_i \in \upset{P}$ \\ 
\end{tabbedproof}

\noindent Next, we need to show that the Heyting algebra axiom for meets. 
Stated formally,
this is $\forall S \in \upset{P}$, if $X \subseteq
\upset{P}$ and $(\forall S' \in X.\; S \leq S')$, then $S
\leq \bigwedge X$ and $\forall S' \in X, \bigwedge X \leq S'$.
\\

\begin{tabbedproof}
\oo Assume $S \in \upset{P}, X \subseteq \upset{P},$ and   
           $(\forall S' \in X.\; S \leq S')$ \\
\ooo First, we want to show $S \leq \bigwedge X$ \\
\oooo This is equivalent to showing $\forall p.\; p \in S \implies p \in \bigwedge X$ \\
\oooo Assume $p \in S$ \\
\ooooo We want to show $p \in \bigwedge X$, so we want to show $\forall S' \in X.\; p \in S'$. \\
\ooooo Assume $S' \in X$ \\
\oooooo From the hypothesis in 1, we know $S \leq S'$ \\
\oooooo This means $\forall p.\; p \in S \implies p \in S'$ \\
\oooooo Instantiate the quantifier with $p$ and use hypothesis 4 to conclude $p \in S'$ \\
\oooo Therefore, $\forall p.\; p \in S \implies p \in \bigwedge X$, so $S \leq \bigwedge X$ \\[1em]
\ooo Second, we want to show $\forall S' \in X.\; \bigwedge X \leq S'$ \\
\ooo Assume $S' \in X$ \\
\oooo We want to show $\bigwedge X \leq S'$, so we must show $\forall p.\; p \in \bigwedge X \implies p \in S'$ \\
\oooo Assume $p \in \bigwedge X$ \\
\ooooo Therefore, we know $\forall S' \in X.\; p \in S'$ \\
\ooooo Instantiate the quantifier with $S'$ to conclude $p \in S'$ \\
\oooo Therefore $\bigwedge X \leq S'$ \\
\ooo Therefore $\forall S' \in X.\; \bigwedge X \leq S'$ 
\end{tabbedproof}
\end{proof}



\begin{lemma}{(Joins in the algebra of specifications)}
If $X \subseteq \upset{P}$, then $\bigvee X$ defines an arbitrary join. 
\end{lemma}
\begin{proof}
First, we need to verify the join we defined actually gives us an upward closed set ---
that is, if $X \subseteq \upset{P}$, then $\bigvee{X} \in \upset{P}$

\vspace{0.5em}

\begin{tabbedproof}
\oo Assume  $X \subseteq \upset{P}$ \\
\ooo We want to show $\bigvee{X} \in \upset{P}$ \\
\ooo This means for all $x \in \bigvee{X}, y \in P,$ if $x \worldleq y$ then $y \in \bigvee{X}$\\
\ooo Assume $x \in \bigvee X$, $y \in P$, $x \worldleq y$ \\
\oooo Since $x \in \bigvee X$, we know $\exists S \in X.\; x \in S$ \\
\oooo Let $S$ be the witness of the existential, so $S \in X$ and $x \in S$ \\ 
\ooooo  Since $S$ is upward closed, $x \in S$, and $x \worldleq y$, we know $y \in S$ \\
\ooo Therefore we can conclude $\exists S \in X.\; y \in S$ \\
\ooo Therefore $y \in \bigvee X$ \\
\end{tabbedproof}

\noindent Now, we need to show the Heyting axioms for disjunction. Formally stated, it is $\forall S \in \upset{P}, X \subseteq
\upset{P}$, if $(\forall S' \in X.\; S' \leq S)$, then $\bigvee X \leq
S$ and $\forall S' \in X.\; S' \leq \bigvee X$.

\vspace{0.5em}

\begin{tabbedproof}
\oo Assume $S \in \upset{P}, X \subseteq \upset{P}$, and $\forall S' \in X.\; S' \leq S$ \\
\ooo First, we want to show $\bigvee X \leq S$ \\
\oooo This is the same as $\forall p, p \in \bigvee X \implies p \in S$ \\
\oooo Assume $p \in \bigvee X$ \\
\ooooo This means $\exists S' \in X.\; p \in S'$ \\
\ooooo Let $S'$ be the witness to the existential, so $S' \in X$ and $p \in S'$ \\
\oooooo Instantiating the quantifier in the hypothesis with $S'$, we get $S' \leq S$ \\
\oooooo This means $\forall p \in S', p \in S$ \\
\oooooo Instantiating the quantifier with $p$, we  get $p \in S$ \\
\oooo Therefore $\forall p, p \in \bigvee X \implies p \in S$ \\
\oooo This is equivalent to $\bigvee X \leq S$ \\[1em]

\ooo Second, we want to show $\forall S' \in X.\; S' \leq \bigvee X$ \\
\oooo Assume $S' \in X$ \\
\ooooo We want to show $S' \leq \bigvee X$ \\
\ooooo This means $\forall p \in S', p \in \bigvee X$ \\
\ooooo Assume $p \in S'$ \\
\oooooo We want to show $p \in \bigvee X$ \\
\oooooo This means we must show $\exists S' \in X.\; p \in S'$ \\
\oooooo Witness the existential with $S'$, so we can show $p \in S'$ by hypothesis \\
\oooo  Therefore $\forall S' \in X.\; S' \leq \bigvee X$ \\
\end{tabbedproof}
\end{proof}




\begin{lemma}{(Implication)}
If $S_1, S_2 \in \upset{P}$, then
$S_1 \implies S_2 \in \upset{P}$. 
\end{lemma}

\begin{proof}
First, we will check that the definition gives us an upward closed set.

\vspace{0.5em}

\begin{tabbedproof}
\oo Assume $x \in S_1 \implies S_2$, and that $y \worldgeq x$ \\
\ooo From this, we know $\forall r' \worldgeq x, $ if $r' \in S_1$ then $r' \in S_2$ \\
\ooo We want to show $\forall r' \worldgeq y, $ if $r' \in S_1$ then $r' \in S_2$ \\
\ooo Assume $r' \worldgeq y$ and $r' \in S_1$ \\
\oooo Since $r' \worldgeq y$ and $r' \worldgeq x$, we know $r' \worldgeq x$ \\
\oooo From this  and $r' \in S_1$, we can use the hypothesis in line 2 to get $r' \in S_2$ \\
\ooo Therefore $\forall r' \worldgeq y, $ if $r' \in S_1$ then $r' \in S_2$ \\
\end{tabbedproof}

\noindent Now that we know that $\implies$ has the correct codomain, we need to
verify that it satisfies the adjoint relationship between conjunction and 
implication: 
\begin{displaymath}
S_1 \land S_2 \subseteq R \iff S_1 \subseteq S_2 \implies R
\end{displaymath}

\noindent This is equivalent to showing that 

\begin{displaymath}
(\forall x.\; x \in S_1 \land S_2 \Rightarrow x \in R) \iff
(\forall x.\; x \in S_1 \Rightarrow x \in S_2 \implies R)
\end{displaymath}

\noindent First, let's show the $\Rightarrow$ direction.
\\

\begin{tabular}{ll}
Assume for all $x.\; x \in S_1 \land S_2 \Rightarrow x \in R$ &
(1)
\\

Assume $x \in S_1$ &
(2)
\\

Assume $r' \worldgeq x$ &
(3)
\\

Assume $r' \in S_2$ & 
(4)
\\

$r' \in S_1$ & 
Since $x \in S_1$ and $r' \worldgeq x$ \\

$r' \in S_1 \cap S_2$ & 
Since $r' \in S_1$ and $r' \in S_2$ \\

$r' \in R$ &
By assumption (1) \\

$r' \in S_2 \Rightarrow r' \in R$ &
Implication intro (4) \\

$\forall r' \worldgeq x.\; r' \in S_2 \Rightarrow r' \in R$ &
Universal intro (3) \\

$x \in \{ r \in P \;|\; \forall r' \worldgeq r.\; r' \in S_2 \Rightarrow r' \in R$ &
Comprehension intro \\

$x \in S_2 \implies R$ &
Definition of $\implies$ \\

$\forall x.\; x \in S_1. \Rightarrow x \in S_2 \implies R$ &
Universal intro (2) \\

$(\forall x.\; x \in S_1 \land S_2 \Rightarrow x \in R) \Rightarrow (\forall x.\; x \in S_1 \Rightarrow x \in S_2 \implies R)$ &
Implication intro (1) \\
\end{tabular}
\\

Next, let's show the $\Leftarrow$ direction. 
\\

\begin{tabular}{ll}
Assume $\forall x.\; x \in S_1 \Rightarrow x \in S_2 \implies R$ &
(1) \\

Assume $x \in S_1 \land S_2$ & 
(2) \\

$x \in S_1$ & 
Since $x \in S_1 \land S_2$ (2)\\

$x \in S_2$ & 
Since $x \in S_1 \land S_2$ (3) \\

$x \in S_2 \implies R$ & 
By (2) and (1) \\

$x \in \{ r \in P \;|\; \forall r' \worldgeq r.\; \mbox{if } r' \in S_2 \mbox{ then } r' \in R \}$ &
Definition of $\implies$ \\

$\forall r' \worldgeq x.\; \mbox{if } r' \in S_2 \mbox{ then } r' \in R$ &
Comprehension instantiation (4) \\

$x \worldgeq x$ & 
Reflexivity (5) \\

$\mbox{if } x \in S_2 \mbox{ then } x \in R$ & 
Instantation of (4) with (5) \\

$x \in R$ & 
Implication elim via (3) \\

$\forall x.\; x \in S_1 \land S_2 \Rightarrow x \in R$ & 
Universal/Implication intro (2) \\

$(\forall x.\; x \in S_1 \Rightarrow x \in S_2 \implies R) \Rightarrow (\forall x.\; x \in S_1 \land S_2 \Rightarrow x \in R)$ & 
Implication intro (1) \\
\end{tabular}
\end{proof}

These lemmas establish that $\upset{P}$ forms a complete Heyting algebra. 

\section{Semantic Hoare Triples}

In this section, we will define \emph{semantic Hoare triples}, which will be the basic
elements we will use to define our specification logic. As always, the definition
will come in a piece-wise fashion. We will start by defining ``basic Hoare triples'',
which will be given semantics in a relatively familiar boolean fashion. We will then
use these to define our true Kripke-style semantic Hoare triples. 

\subsection{Basic Hoare Triples}

Since we have a continuation semantics, it is natural to define a
continuation style interpretation of basic Hoare triples, as well.

\subsubsection{Approximating Postconditions}

Given a $Q \in \interp{A} \to \powerset{H}$, we define $Approx(Q)$ as the set:
\begin{displaymath}
  Approx(Q) \triangleq \comprehend{k \in \interp{A} \to H \to O}
                         {\forall v \in \interp{A}, h \in Q(v).\; k\;v\;h = \bot}
\end{displaymath}

These define a set of continuations which ``continuously approximate''
the postcondition $Q$ -- they are the set of continuations which run
forever when given a value and heap in $Q$.  From this set we will
define the function $Best(Q)$, which will be the ``best continuous
approximation'' to Q. (Intuitively, think of this as being like a
closure operator from topology, which finds the smallest open set
containing the given set.)

\begin{displaymath}
  Best(Q) \triangleq \lambda v \in \interp{A}.\; \lambda h \in H.\; 
    \left\{\begin{array}{ll}
             \top & \mbox{when } \exists k \in Approx(Q).\; k\;v\;h = \top \\
             \bot & \mbox{otherwise}
           \end{array}
    \right.
\end{displaymath}
Of course, we have to verify that $Best(Q)$ is actually a continuous function. 

\begin{itemize}
\item First, we need to check that $Best(Q)$ is a monotone function. 

\begin{tabbedproof}
\oo Suppose we have $v \sqsubseteq v'$ and $h \sqsubseteq h'$. \\
\ooo We know $Best(Q)\;v\;h \in O$. Analyzing this by cases, we see \\
\ooo Suppose $Best(Q)\;v\;h = \bot$ \\
\oooo Since $\forall o \in O.\; \bot \sqsubseteq o$, it follows that 
      $\bot \sqsubseteq Best(Q)\;v'\;h'$ \\
\oooo So $Best(Q)\;v\;h \sqsubseteq Best(Q)\;v'\;h'$ \\
\ooo Suppose $Best(Q)\;v\;h = \top$ \\
\oooo By definition of $Best(Q)$, $\exists k \in Approx(Q).\; k\;v\;h = \top$ \\
\oooo Let $k \in Approx(Q)$ be the witness such that $k\;v\;h = \top$ \\ 
\ooooo Since $k$ is monotone, $k\;v\;h \sqsubseteq k\;v'\;h'$ \\
\ooooo So $\top \sqsubseteq k\;v'\;h'$ \\
\ooooo Since $\top$ is maximal in $O$, $k\;v'\;h' = \top$ \\
\ooooo So we can take $k$ to be the witness such that $\exists k.\; k\;v'\;h' = \top$ \\
\oooo Therefore $Best(Q)\;v'\;h' = \top$ \\
\oooo Therefore $Best(Q)\;v\;h \sqsubseteq Best(Q)\;v'\;h'$ \\
\end{tabbedproof}

\item Second, we need to show that $Best(Q)$ preserves limits. 

\begin{tabbedproof}
\oo Suppose we have two chains $v_i$ and $h_i$ such that $i \leq j$
implies $v_i \sqsubseteq v_j$ and $h_i \sqsubseteq h_j$. \\
\ooo We want to show that $\bigsqcup_i Best(Q)\;v_i\;h_i = Best(Q)\;(\sqcup v_i)\;(\sqcup h_i)$ \\
\ooo By excluded middle, either some $k$ in $Approx(Q)$ such that $k\;(\sqcup v_i)\;(\sqcup h_i) = \top$, or not. \\
\ooo Suppose that $\exists k \in Approx(Q).\; k\;(\sqcup v_i)\;(\sqcup h_i) = \top$ \\
\oooo Therefore $Best(Q)\;(\sqcup v_i)\;(\sqcup h_i) = \top$ \\
\oooo By continuity of $k$, $\bigsqcup_i k\;v_i\;h_i = \top$ \\
\oooo Since $O$ is discrete, there is an $n$ such that $k\;v_n\;h_n = \top$ \\
\oooo Therefore, for all $j \geq n$, $Best(Q)\;v_n\;h_n = \top$ \\
\oooo This means $\bigsqcup_i Best(Q)\;v_i\;h_n = \top$ \\
\oooo Therefore $\bigsqcup_i Best(Q)\;v_i\;h_n = Best(Q)\;(\sqcup v_i)\;(\sqcup h_i)$ \\
\ooo Suppose that $\lnot(\exists k \in Approx(Q).\; k\;(\sqcup v_i)\;(\sqcup h_i) = \top$ \\
\oooo This is equivalent to $\forall k \in Approx(Q).\; k\;(\sqcup v_i)\;(\sqcup h_i) = \bot$ \\
\oooo This means $Best(Q)\;(\sqcup v_i)\;(\sqcup h_i) = \bot$ \\
\oooo Now, assume $k \in Approx(Q)$ \\
\ooooo So $k\;(\sqcup v_i)\;(\sqcup h_i) = \bot$ \\
\ooooo By continuity, $\bigsqcup_i k\;v_i\;h_i = \bot$ \\
\ooooo Therefore for all $i$, $k\;v_i\;h_i = \bot$ \\
\oooo So for all $k \in Approx(Q)$ and $i$, we know $k\;v_i\;h_i = \bot$ \\
\oooo This is equivalent to $\forall i.\; \lnot(\exists k \in Approx(Q).\; k\;v_i\;h_i = \top)$\\
\oooo Therefore, for all $i$, we know $Best(Q)\;v_i\;h_i = \bot$ \\
\oooo Therefore, we know $\bigsqcup_i Best(Q)\;v_i\;h_i = \bot$ \\
\oooo So we conclude $\bigsqcup_i Best(Q)\;v_i\;h_i = Best(Q)\;(\sqcup v_i)\;(\sqcup h_i)$\\
\end{tabbedproof}
\end{itemize}
This establishes that $Best(Q)$ is a continuous function. 

Now we will prove a minor lemma about this function, which shows that we
are interpreting the Sierpinski lattice such that the bottom element
is truth, and the top element is falsehood.

\begin{lemma}{(Lattice Order Reverses Approximation Order)}
Suppose $Q$ and $Q'$ are $A$-predicates in $\interp{A} \to \powerset{H}$, and that
for all $a \in \interp{A}$, $Q(a) \subseteq Q'(a)$. Then $\mathit{Best}(Q') \sqsubseteq \mathit{Best}(Q)$. 
\end{lemma}

\begin{tabbedproof}
\oo We want to show $\mathit{Best}(Q') \sqsubseteq \mathit{Best}(Q)$ \\
\oo First, we will observe that $\mathit{Approx}(Q') \subseteq \mathit{Approx}(Q)$ \\
\oo So, suppose that $k \in \mathit{Approx}(Q')$. We want to show $k \in \mathit{Approx}(Q)$. \\
\ooo We want to show that for all $a \in \interp{A}$ and $h \in Q(a)$, $k\;a\;h = \bot$.  \\
\ooo Assume that $a \in \interp{A}$ and $h \in Q(a)$. \\ 
\oooo We know that since $Q(a) \subseteq Q'(a)$, we have $h \in Q'(a)$.  \\
\oooo Therefore since $k \in \mathit{Approx}(Q')$, we know $k\;a\;h = \bot$ \\
\oo Now, we want to show that $\mathit{Approx}(Q') \subseteq \mathit{Approx}(Q)$ \\
\oo So we want to show that for all $a \in \interp{A}$ and $h \in H$, $\mathit{Approx}(Q')\;a\;h \subseteq \mathit{Approx}(Q)\;a\;h$ \\
\oo Assume we have $a \in \interp{A}$ and $h \in H$ \\
\ooo Suppose $\mathit{Approx}(Q')\;a\;h = \top$: \\
\oooo Therefore there is a $k\in\mathit{Approx}(Q')$ such that $k\;a\;h = \top$ \\
\oooo Therefore $k \in \mathit{Approx}(Q)$, and so $\mathit{Approx}(Q)\;a\;h = \top$ \\
\ooo Suppose $\mathit{Approx}(Q')\;a\;h = \bot$: \\
\oooo Therefore for all $k \in \mathit{Approx}(Q')$, we have $k\;a\;h = \top$ \\
\oooo Therefore for all $k \in \mathit{Approx}(Q)$, we have $\mathit{Approx}(Q)\;a\;h = \bot$ \\

\end{tabbedproof}

\subsubsection{Basic Boolean Hoare Triples}

Supposing that $p$ is an element of the BI algebra $\powerset{H}$, $c$ is an element
of the domain of commands $(A \to K) \to K$, and $q$ is an $A$-indexed predicate, of
type $A \to \powerset{H}$, then we can define the basic boolean Hoare triple 
$\basicspec{p}{c}{a:A}{q}$:

\begin{displaymath}
  \basicspec{p}{c}{a:A}{q(a)} \triangleq
    \forall h \in p.\; \left(c\;Best(q)\;h = \bot\right)
\end{displaymath}

If $c$ is given a continuation which will run forever (i.e., equals
$\bot$) whenever it receives a heap in $q$, then given a heap $h \in
p$, $c$ with that continuation and that heap will also run forever.

The reason that we interpret triples this way is to make the fixed
point induction rule a sound rule of inference.



\subsection{Kripke Interpretation}

Now, we can give the Kripke interpretation of semantic triples. Given a semantic
assertion $p$, an element $c$ of the domain of $A$-commands $(A \to K) \to K$, 
and a $A$-predicate $q \in A \to \powerset{H}$, we define the meaning of a 
semantic triple as: 

\begin{displaymath}
  \spec{p}{c}{a:A}{q(a)} \triangleq
    \comprehend{ r \in \powerset{H} }
               {\forall s \sqsupseteq r.\; \basicspec{p * s}{c}{a:A}{q(a) * s}}
\end{displaymath}

\begin{lemma}{(Semantic Triples are Specifications)}
  For suitable $p,c,A$, and $q$, we have that 
$\spec{p}{c}{a:A}{q(a)} \in \upset{\powerset{H}}$
\end{lemma}

\begin{proof}
\begin{tabbedproof}
\oo We want to show $\spec{p}{c}{a:A}{q(a)} \in \upset{\powerset{H}}$ \\
\oo This is equivalent to $\forall r,s$ if $r \in \spec{p}{c}{a:A}{q(a)}$ and $s \sqsupseteq r$,
then $s \in \spec{p}{c}{a:A}{q}$ \\
\oo Assume $r,s, r \in \spec{p}{c}{a:A}{q(a)}$, and $s \sqsupseteq r$ \\
\ooo $r \in \spec{p}{c}{a:A}{q(a)}$ is equivalent to 
     $\forall s \sqsupseteq r.\; \basicspec{p * s}{c}{a:A}{q(a) * s}$ \\
\ooo We want to show $s \in \spec{p}{c}{a:A}{q}$ \\
\ooo This is equivalent to showing 
     $\forall t \sqsupseteq s.\; \basicspec{p * t}{c}{a:A}{q(a) * t}$ \\
\ooo Assume $t$, $t \sqsupseteq s$ \\
\oooo By transitivity with $t \sqsupseteq s$ and $s \sqsupseteq r$, we know $t \sqsupseteq r$ \\
\oooo Instantiating quantifier in line 4 with $t$, $\basicspec{p * t}{c}{a:A}{q(a) * t}$ \\
\ooo Therefore $\forall t \sqsupseteq s.\; \basicspec{p * t}{c}{a:A}{q(a) * t}$ \\
\ooo Therefore $s \in \spec{p}{c}{a:A}{q}$ \\
\oo Therefore $\forall r,s$ if $r \in \spec{p}{c}{a:A}{q(a)}$ and $s \sqsupseteq r$,
    then $s \in \spec{p}{c}{a:A}{q}$ \\
\oo We have shown $\spec{p}{c}{a:A}{q(a)} \in \upset{\powerset{H}}$ \\ 
\end{tabbedproof}
\end{proof}

\subsubsection{Fixed Point Induction}

The reason we have gone to all this trouble of using continuous
approximations to the postcondition is to create an admissibility
property which will allow us to justify a fixed point induction rule.

\begin{lemma}{(Bottom Satisfies All Specifications)}
We have that $\spec{p}{\bot}{a:A}{q(a)} = \powerset{H}$. 
\end{lemma}
\begin{proof}
\begin{tabbedproof}
\oo We want to show $\spec{p}{\bot}{a:A}{q(a)} = \powerset{H}$ \\
\oo It suffices to show $\forall r \in \powerset{H}, r \in \spec{p}{\bot}{a:A}{q(a)}$ \\
\oo Assume $r \in \powerset{H}$ \\
\ooo We want to show $r \in \spec{p}{\bot}{a:A}{q(a)}$, which is equivalent to 
     $\forall s \sqsupseteq r.\; \basicspec{p*s}{\bot}{a:A}{q(a) * s}$ \\
\ooo Assume $s \sqsupseteq r$ \\
\oooo We want to show $\basicspec{p*s}{\bot}{a:A}{q(a) * s}$ \\
\oooo This is equivalent to $\forall h \in p. \bot\;Best(q)\;h = \bot$ \\
\oooo Assume $h \in p$ \\
\ooooo  By definition of least element, $\bot\;Best(q)\;h = \bot$ \\
\oooo Therefore $\forall h \in p. \bot\;Best(q)\;h = \bot$ \\
\oooo Therefore $\basicspec{p*s}{\bot}{a:A}{q(a) * s}$ \\
\ooo Therefore $\forall s \sqsupseteq r.\; \basicspec{p*s}{\bot}{a:A}{q(a) * s}$ \\
\ooo Therefore $r \in \spec{p}{\bot}{a:A}{q(a)}$ \\
\oo Therefore $\forall r \in \powerset{H}, r \in \spec{p}{\bot}{a:A}{q(a)}$ \\
\oo Therefore $\spec{p}{\bot}{a:A}{q(a)} = \powerset{H}$ \\
\end{tabbedproof}
\end{proof}

\begin{lemma}{(Admissibility of Triple Subsets)}
Define $\spec{p}{-}{a:A}{q(a)}$ to be 

\begin{displaymath}
\spec{p}{-}{a:A}{q(a)} \triangleq
         \comprehend{c \in (A \to K) \to K}
                   { \spec{p}{c}{a:A}{q(a)} = \powerset{H} }
\end{displaymath}

Then, $\spec{p}{-}{a:A}{q(a)}$ forms an admissible subset of $(A \to K) \to K$. 
That is, given a chain $c_i \in \spec{p}{-}{a:A}{q(a)}$, we know that 
$\spec{p}{\sqcup_i c_i}{a:A}{q(a)}$. 
\end{lemma}

\begin{proof}
\begin{tabbedproof}
\oo Suppose we have a chain $c_i \in \spec{p}{-}{a:A}{q(a)}$. \\
\ooo We want to show that $\sqcup_i c_i \in \spec{p}{\sqcup_i c_i}{a:A}{q(a)}$ \\
\ooo This is equivalent to $\spec{p}{\sqcup c_i}{a:A}{q(a)} = \powerset{H}$ \\
\ooo This is equivalent to 
     $\forall r \in \powerset{H}, s \sqsupseteq r.\; 
         \basicspec{p * s}{\sqcup c_i}{a:A}{q(a) * s}$ \\
\ooo Assume $r \in \powerset{H}, s \sqsupseteq r$ \\
\oooo We want to show $\basicspec{p * s}{\sqcup c_i}{a:A}{q(a) * s}$ \\
\oooo This is equivalent to $\forall h \in p * s.\; (\sqcup c_i)\; Best(\semfun{a}{q(a)*s})\;h = \bot$ \\
\oooo Assume $h \in p * s$ \\ 
\ooooo By continuity, we know $(\sqcup c_i)\; Best(\semfun{a}{q(a)*s})\;h = 
                               \bigsqcup (c_i\;Best(\semfun{a}{q(a)*s})\;h)$ \\
\ooooo Suppose $c$ is an element of the chain of $c_i$ \\
\oooooo Then we know $c \in \spec{p}{-}{a:A}{q}$ \\
\oooooo This is equivalent to $\spec{p}{c}{a:A}{q} = \powerset{H}$ \\
\oooooo This is equivalent to $\forall r, s \sqsupseteq r.\; \basicspec{p*s}{c}{a:A}{q(a)*s}$ \\
\oooooo This is equivalent to $\forall r, s \sqsupseteq r, h \in p*s, c\;Best(\semfun{a}{q(a)*s})\;h = \bot$ \\
\oooooo Instantiating quantifiers with $r$, $s$, and $h$, we get $c\;Best(\semfun{a}{q(a)*s})\;h = \bot$\\
\ooooo Therefore $\forall c \in \comprehend{c_i}{i \in \N}$, $c\;Best(\semfun{a}{q(a)*s})\;h = \bot$ \\
\ooooo Therefore $\bigsqcup (c_i\;Best(\semfun{a}{q(a)*s})\;h) = \bot$ \\
\ooooo Therefore $(\sqcup c_i)\;Best(\semfun{a}{q(a)*s})\;h = \bot$ \\
\oooo Therefore $\forall h \in p * s.\; (\sqcup c_i)\; Best(\semfun{a}{q(a)*s})\;h = \bot$ \\
\oooo Therefore $\basicspec{p * s}{\sqcup c_i}{a:A}{q(a) * s}$ \\
\ooo Therefore $\forall r \in \powerset{H}, s \sqsupseteq r.\; 
                   \basicspec{p * s}{\sqcup c_i}{a:A}{q(a) * s}$ \\
\ooo Therefore $\spec{p}{\sqcup c_i}{a:A}{q(a)} = \powerset{H}$ \\
\ooo Therefore $\sqcup_i c_i \in \spec{p}{\sqcup_i c_i}{a:A}{q(a)}$ \\
\end{tabbedproof}
\end{proof}

\begin{lemma}{(Semantic Fixed Point Induction)}
If we know that for all $x$, $\spec{p}{x}{a:A}{q(a)} = \powerset{H}$ implies $\spec{p}{f(x)}{a:A}{q(a)}) = \powerset{H}$, then we know that $\spec{p}{\mathit{fix}(f)}{a:A}{q(a)} = \powerset{H}$
\end{lemma}

\begin{proof}
First, observe that $f^n(\bot)$ forms a chain -- that is, for all $i$, $f^i(\bot) \sqsubseteq f^{i+1}(\bot)$.
\begin{tabbedproof}
\oo We want to show $\forall i$, $f^i(\bot) \sqsubseteq f^{i+1}(\bot)$ \\
\oo We proceed by induction on $i$ \\
\ooo Case $i = 0$: \\
\oooo We want to show $\bot \sqsubseteq f(\bot)$ \\
\oooo This follows immediately from the fact that $\bot$ is the least element of a domain. \\
\ooo Case $i = j + 1$ \\
\oooo We want to show $f^j(\bot) \sqsubseteq f^{j+1}(\bot) \implies f^i(\bot) \sqsubseteq f^{i+1}(\bot)$ \\
\oooo Assume $f^j(\bot) \sqsubseteq f^{j+1}(\bot)$ \\
\ooooo By monotonicity of $f$, $f(f^j(\bot)) \sqsubseteq f(f^{j+1}(\bot))$ \\
\ooooo Therefore $f^i(\bot) \sqsubseteq f^{i+1}(\bot)$ \\
\oo Therefore $\forall i$, $f^i(\bot) \sqsubseteq f^{i+1}(\bot)$ \\
\end{tabbedproof}

\noindent Now, observe that for every $n$, $f^n(\bot) \in \spec{p}{-}{a:A}{q}$. 

\begin{tabbedproof}
\oo Assume for all $x$, $\spec{p}{x}{a:A}{q(a)} = \powerset{H}$ implies $\spec{p}{f(x)}{a:A}{q(a)}) = \powerset{H}$ \\
\ooo We want to show $\forall n, f^n(\bot) \in \spec{p}{-}{a:A}{q(a)}$ \\ 
\ooo We proceed by induction on $n$: \\
\oooo Case $n = 0$ \\
\ooooo We want to show $\bot \in \spec{p}{-}{a:A}{q(a)}$ \\
\ooooo This follows from the fact that bottom satisfies all specifications. \\
\oooo Case $n = m + 1$ \\
\ooooo We want to show $f^m(\bot) \in \spec{p}{-}{a:A}{q} \implies
                        f^{m+1}(\bot) \in \spec{p}{-}{a:A}{q}$ \\
\ooooo Assume $f^m(\bot) \in \spec{p}{-}{a:A}{q(a)}$ \\
\oooooo This means $\spec{p}{f^m(\bot)}{a:A}{q(a)} = \powerset{H}$ \\
\oooooo Instantiate line 1 with $f^m(\bot)$, to conclude
          $\spec{p}{f(f^m(\bot))}{a:A}{q(a)}) = \powerset{H}$ \\
\oooooo This means $\spec{p}{f^{m+1}(\bot)}{a:A}{q(a)} = \powerset{H}$ \\
\oooooo This means $f^{m+1}(\bot) \in \spec{p}{-}{a:A}{q(a)}$ \\
\ooo Therefore $\forall n, f^n(\bot) \in \spec{p}{-}{a:A}{q(a)}$ \\ 
\end{tabbedproof}
Finally, by the admissibility of $\spec{p}{-}{a:A}{q(a)}$, we know that
$\sqcup f^n(\bot) \in \spec{p}{-}{a:A}{q(a)}$. Since $\sqcup f^n(\bot) = fix(f)$, we 
know that $\spec{p}{\mathit{fix}(f)}{a:A}{q(a)} = \powerset{H}$. 
\end{proof}



\section{The Framing Operator}

One nice feature of the Kripke-style interpretation of specifications
is that it naturally validates higher-order frame rules.  We define
the operation $S \otimes p$, where $S \in \upset{W(\powerset{H})}$ and
$p \in W(\powerset{H})$:
\begin{displaymath}
S \otimes p = \comprehend{ r \in W(\powerset{H}) }{ r * p \in S }
\end{displaymath}
%
The way to understand this is that $S \otimes p$ restricts $S$ to only those
elements which can be extended by $p$. That is, if we think of $S$ as the
set of propositions that can be framed on to a basic triple, then we want
$S \otimes p$ to be only the frames in $S$, such that if we added $p$ to
them we continue to have a frame in $S$. So this operation gives a semantic
interpretation of the frame rule. 

So we need to show that first, $S \otimes p$ actually is an element of
our Heyting algebra of specification truth values, and second, we want
the frame rule to be sound -- we want $S$ to always imply $S \otimes
p$.

\begin{lemma}{(Framing is a Lattice Operation)}
For all $S$ and $p$, we have that $S \otimes p$ is in $\upset{W(\powerset{H}})$,
and that $S \subseteq S \otimes p $.
\end{lemma}
\begin{proof}
To show that $S \otimes p \in \upset{W(\powerset{H})}$, we need to show that for all 
$x,y$, if $x \in S \otimes p$ and $y \sqsupseteq x$, then $y \in S \otimes p$. 

\begin{tabbedproof}
\oo We want to show for all $x,y$, if $x \in S \otimes p$ and $y \sqsupseteq x$, then $y \in S \otimes p$ \\
\oo Assume $x, y, x \in S \otimes p, y \sqsupseteq x$ \\
\ooo From $x \in S \otimes p$, we know $x * p \in S$ \\
\ooo From $y \sqsupseteq x$, we know $\exists r.\; y = x * r$ \\
\ooo Let $r$ be the witness so that $y = x * r$ \\
\oooo Since $S$ is upward closed, $x * p * r \in S$ \\
\oooo Since $x * p * r = (x * r) *p$, we know  $y * p \in S$ \\
\oooo Therefore $y \in S \otimes p$ \\
\end{tabbedproof}

\noindent To show $S \subseteq S \otimes p$, we need to show that if $x \in S$, then $x \in S \otimes p$. 
\begin{tabbedproof}
\oo Assume $x \in S$ \\
\ooo Since $\sqsubseteq$ is the extension ordering, $x \sqsubseteq x * p$ \\   
\ooo Since $S$ is upward closed, $x * p \in S$\\
\ooo Therefore $x \in S \otimes p$ 
\end{tabbedproof}
\end{proof}

\subsection{Framing Commutes With Logical Operators}

\begin{lemma}{(Framing onto Semantic Triples)}
We have that 
\begin{displaymath}
\spec{p}{c}{a:A}{q(a)} \otimes r = \spec{p * r}{c}{a:A}{q(a) * r}  
\end{displaymath}
\end{lemma}

\begin{proof}
\begin{tabbedproof}
\oo We want to show $\spec{p}{c}{a:A}{q(a)} \otimes r = \spec{p * r}{c}{a:A}{q(a) * r}$. \\
\oo This means $\forall s \in W(\powerset{H}).\; s \in (\spec{p}{c}{a:A}{q(a)} \otimes r)$ if and
only if $s \in \spec{p * r}{c}{a:A}{q(a) * r}$. \\
\ooo Assume $s \in W(\powerset{H})$ \\
\oooo $\To$ direction: \\
\ooooo Assume $s \in (\spec{p}{c}{a:A}{q(a)} \otimes r)$ \\
\oooooo This means $s * r \in \spec{p}{c}{a:A}{q(a)}$ \\
\oooooo This means $\forall t \sqsupseteq s * r.\; \basicspec{p * t}{c}{a:A}{q * t}$ \\
\oooooo We want to show $s \in \spec{p * r}{c}{a:A}{q(a) * r}$ \\
\oooooo So we want $\forall t' \sqsupseteq s.\; \basicspec{p * r * t'}{c}{a:A}{q(a) * r * t'}$\\
\oooooo Assume $t' \in \sqsupseteq s$ \\
\ooooooo Clearly, $t' * r \sqsupseteq s * r$ \\
\ooooooo Instantiate quantifier in 7 with $t' * r$ to conclude
         $\basicspec{p * t' * r}{c}{a:A}{q * t' * r}$ \\
\ooooooo Rearranging, we get $\basicspec{p * r * t'}{c}{a:A}{q * r * t'}$ \\
\oooooo Therefore $\forall t' \sqsupseteq s.\; \basicspec{p * r * t'}{c}{a:A}{q(a) * r * t'}$\\
\oooooo Therefore $s \in \spec{p * r}{c}{a:A}{q(a) * r}$ \\[0.5em]

\oooo $\Leftarrow$ direction: \\
\ooooo Assume $s \in \spec{p * r}{c}{a:A}{q(a) * r}$. \\
\oooooo This means $\forall t \sqsupseteq s.\; \basicspec{p * r * t}{c}{a:A}{q(a) * r * t}$ \\
\oooooo We want to show $s \in (\spec{p}{c}{a:A}{q(a)} \otimes r)$ \\
\oooooo So we want $s * r \in \spec{p}{c}{a:A}{q(a)}$ \\
\oooooo So we want $\forall t \sqsupseteq s * r.\; \basicspec{p * t}{c}{a:A}{q * t}$ \\
\oooooo Assume $t \sqsupseteq s * r$ \\
\ooooooo Since $t \sqsupseteq s * r$, we know $\exists u.\; t = s * r * u$ \\
\ooooooo Let $u$ be the witness such that $t = s * r * u$ \\
\ooooooo Now, note that $s * u \sqsupseteq s$ \\
\ooooooo Instantiate quantifier in 18 with $s * u$, so 
         $\basicspec{p * r * s * u}{c}{a:A}{q(a) * r * s * u}$ \\
\ooooooo Rearranging, $\basicspec{p * s * r * u}{c}{a:A}{q(a) * s * r * u}$ \\
\ooooooo By equality in 24, $\basicspec{p * t}{c}{a:A}{q * t}$ \\
\oooooo Therefore $\forall t \sqsupseteq s * r.\; \basicspec{p * t}{c}{a:A}{q * t}$ \\
\oooooo Therefore $s * r \in \spec{p}{c}{a:A}{q(a)}$ \\
\oooooo Therefore $s \in (\spec{p}{c}{a:A}{q(a)} \otimes r)$ 
\end{tabbedproof}
\end{proof}

\begin{lemma}{(Framing Commutes with Meets)}
We have that 
\begin{displaymath}
\left(\bigwedge_{i \in I} S_i\right) \otimes p = \bigwedge_{i \in I} (S_i \otimes p)
\end{displaymath}
\end{lemma}

\begin{proof}
To show $\left(\bigwedge_{i \in I} S_i\right) \otimes p = \bigwedge_{i \in I} (S_i \otimes p)$,
we need to show $\forall r.\; r \in \left(\bigwedge_{i \in I} S_i\right) \otimes p$ if
and only if $r \in \bigwedge_{i \in I} (S_i \otimes p)$. 

\begin{tabbedproof}
\oo Assume $r \in W(\powerset{H})$ \\[0.5em]

\ooo $\To$ direction: \\
\oooo Assume $r \in \left(\bigwedge_{i \in I} S_i\right) \otimes p$ \\
\ooooo From assumption, we know $r * p \in \bigwedge_{i \in I} S_i$ \\
\ooooo This means that $\forall i \in I.\; r * p \in S_i$ \\
\ooooo We want to show $r \in \bigwedge_{i \in I} (S_i \otimes p)$ \\
\ooooo So we want $\forall i \in I.\; r \in (S_i \otimes p)$ \\
\ooooo So we want $\forall i \in I.\; r * p \in S_i$ \\
\ooooo Assume $i \in I$ \\
\oooooo We want to show $r * p \in S_i$ \\
\oooooo Instantiating line 5 with $i$, we get $r * p \in S_i$ \\[0.5em]

\ooo $\Leftarrow$ direction:  \\
\oooo Assume $r \in \bigwedge_{i \in I} (S_i \otimes p)$  \\
\ooooo From this, we know that $\forall i \in I.\; r \in (S_i \otimes p)$ \\
\ooooo We want to show $r \in \left(\bigwedge_{i \in I} S_i\right) \otimes p$ \\
\ooooo So we want to show $r * p \in \bigwedge_{i \in I} S_i$ \\
\ooooo So we want to show $\forall i \in I.\; r * p \in S_i$ \\
\ooooo Assume $i \in I$ \\
\oooooo Instantiating line 14 with $i$, we get $r \in (S_i \otimes p)$ \\
\oooooo This means $r * p \in S_i$ \\
\ooooo Therefore $\forall i \in I.\; r * p \in S_i$ \\
\ooooo Therefore $r * p \in \bigwedge_{i \in I} S_i$ \\
\ooooo Therefore $r \in \left(\bigwedge_{i \in I} S_i\right) \otimes p$ 
\end{tabbedproof}
\end{proof}

\begin{lemma}{(Framing Commutes with Joins)}
We have that 
\begin{displaymath}
\left(\bigvee_{i \in I} S_i\right) \otimes p = \bigvee_{i \in I} (S_i \otimes p)
\end{displaymath}
\end{lemma}

\begin{proof}
Showing this is equivalent to showing $\forall r.\; r \in
\left(\bigvee_{i \in I} S_i\right) \otimes p$ if and only if $r \in
\bigvee_{i \in I} (S_i \otimes p)$.

\begin{tabbedproof}
\oo Assume $r$ \\[0.5em]
\ooo  $\To$ direction: \\
\oooo Assume $r \in \left(\bigvee_{i \in I} S_i\right) \otimes p$ \\
\oooo This means $r * p \in \bigvee_{i \in I} S_i$ \\
\oooo This means $\exists i \in I.\; r * p \in S_i$ \\
\oooo We want to show $r \in \bigvee_{i \in I} (S_i \otimes p)$ \\
\oooo So we want to show $\exists i \in I.\; r \in (S_i \otimes p)$ \\
\oooo Let $i$ be the witness in 5, such that $r * p \in S_i$ \\
\ooooo From this, we see $r \in (S_i \otimes p)$ \\
\ooooo From this and $i$, we conclude $\exists i \in I.\; r \in (S \otimes p)$ \\
\oooo Therefore $r \in \bigvee_{i \in I} (S_i \otimes p)$ \\[0.5em]

\ooo $\From$ direction: \\
\oooo Assume $r \in \bigvee_{i \in I} (S_i \otimes p)$. \\
\ooooo From this, $\exists i \in I.\; r \in (S_i \otimes p)$ \\
\ooooo We want to show $r \in \left(\bigvee_{i \in I} S_i\right) \otimes p$ \\
\ooooo So we want $r * p \in \bigvee_{i \in I} S_i$ \\
\ooooo So we want $\exists i \in I.\; r * p \in S_i$ \\
\ooooo Let $i \in I$ be the witness in 14, so that $r \in (S_i \otimes p)$ \\
\oooooo From this, $r * p \in S_i$ \\
\oooooo With this and $i \in I$, we know $\exists i \in I.\; r * p \in S_i$ \\
\ooooo Therefore $r * p \in \bigvee_{i \in I} S_i$ \\
\ooooo Therefore $r \in \left(\bigvee_{i \in I} S_i\right) \otimes p$ \\
\end{tabbedproof}
\end{proof}

\begin{lemma}{(Framing Commutes Through Implication)}
We have that
\begin{displaymath}
  (S_1 \implies S_2) \otimes p = (S_1 \otimes p) \implies (S_2 \otimes p) 
\end{displaymath}
\end{lemma}

\begin{proof}
This is equivalent to showing that $\forall r, r \in [(S_1 \implies S_2) \otimes p]$ if and
only if $r \in [(S_1 \otimes p) \implies (S_2 \otimes p)]$. 

\begin{tabbedproof}
\oo Assume $r$ \\
\ooo $\To$ direction:\\
\oooo Assume $r \in [(S_1 \implies S_2) \otimes p]$ \\
\ooooo This means $r * p \in (S_1 \implies S_2)$ \\
\ooooo This means $\forall s \sqsupseteq r * p,$ if $s \in S_1$ then $s \in S_2$ \\
\ooooo We want to show $r \in [(S_1 \otimes p) \implies (S_2 \otimes p)]$ \\
\ooooo So we want $\forall s \sqsupseteq r,$ if  $s \in S_1 \otimes p$ then $s \in S_2 \otimes p$ \\
\ooooo Assume $s \sqsupseteq r$ and $s \in S_1 \otimes p$ \\
\oooooo From this $s * p \in S_1$ \\ 
\oooooo Since $s * p \sqsupseteq s$ and $s \sqsupseteq r$, we have $s * p \sqsupseteq r$ \\
\oooooo Instantiating line 5 with $s * p$, we have if $s * p \in S_1$ then $s * p \in S_2$ \\
\oooooo Using this and line 9, we have $s * p \in S_2$ \\
\oooooo From this, we have $s \in S_2 \otimes p$ \\
\ooooo Therefore $\forall s \sqsupseteq r,$ if  $s \in S_1 \otimes p$ then $s \in S_2 \otimes p$ \\
\ooooo Therefore $r \in [(S_1 \otimes p) \implies (S_2 \otimes p)]$ \\

\ooo $\From$ direction: \\
\oooo Assume $r \in [(S_1 \otimes p) \implies (S_2 \otimes p)]$ \\
\ooooo From this, $\forall s \sqsupseteq r,$ if $s \in (S_1 \otimes p)$, then $s \in (S_2 \otimes p)$ \\
\ooooo We want to show $r \in [(S_1 \implies S_2) \otimes p]$ \\
\ooooo So we want $r * p \in (S_1 \implies S_2)$ \\
\ooooo So we want $\forall s \sqsupseteq r * p$, if $s \in S_1$ then $s \in S_2$ \\
\ooooo Assume $s \sqsupseteq r * p$ and $s \in S_1$ \\
\oooooo From this, $\exists t.\; s = t * r * p$ \\
\oooooo Let $t$ be the witness such that $s = t * r * p$ \\ 
\ooooooo Note $t * r \sqsupseteq r$ \\
\ooooooo Instantiating 18 with $t * r$, we get if $t * r \in (S_1 \otimes p)$, then $t * r \in (S_2 \otimes p)$ \\
\ooooooo From this, we have if $t * r * p \in S_1$ then $t * r * p \in S_2$ \\
\ooooooo So we have if $s \in S_1$, then $s \in S_2$ \\
\ooooooo From this and 22, we have $s \in S_2$ \\
\ooooo Therefore $\forall s \sqsupseteq r * p$, if $s \in S_1$ then $s \in S_2$ \\
\ooooo Therefore $r * p \in (S_1 \implies S_2)$ \\
\ooooo Therefore $r \in [(S_1 \implies S_2) \otimes p]$ 
\end{tabbedproof}
\end{proof}

\section{Syntax of Assertions and Specifications}

In this section, we will give the syntax of specifications and
assertions, and then we will give their semantics. The syntactic
categories are given in Figure~\ref{logic-syntax}. The sorts of our
logic are ranged over by $\omega$, and include the kinds $\kappa$, the
types $A$, and the propositional sorts $\upsilon$.  The propositional
sorts $\omega$ include $\assert$, the sort $\upsilon \To \omega$,
which are the sort of propositional functions, and the sort
$\forallsort{\alpha}{\kappa}{\upsilon}$, which are
type-constructor-indexed families.
 
Note that we syntactically identify a family of propositional sorts
$\upsilon$. These sorts all end in $\assert$, and by distinguishing
them from kinds $\kappa$ and types $A$, we forbid the formation of
sorts like $\assert \To A$. This will ensure that program terms will
never depend on purely logical facts, though the converse (logical
terms depending on program terms) is allowed. This restriction is a
slight variation of the usual convention in higher-order logic, where
sorts must bottom out in the assertion type.

However, general sorts $\omega$ include both types and kinds. This
lets us define the assertions we will need for asserting facts about
polymorphic programs. For example, the sort of the $\mathsf{list}$
predicate for polymorphic lists can be given as $\mathsf{listprop} :
\forallsort{\alpha}{\star}{\listtype{\alpha} \To \seqsort{\alpha} \To
  \assert}$.


The terms (for which we will use $p$ $q$ as metavariables) which are
categorized by our sorts are also given in Figure~\ref{logic-syntax}.
They include lambda-abstraction and application for the two function
space sorts $\omega \To \upsilon$ and
$\forallsort{\alpha}{\kappa}{\upsilon}$, terms $e$ for the sorts $A$,
type expressions $\tau$ for the sorts $\kappa$.

Finally, we also have the assertions of separation logic for the sort
$\assert$.  These include the usual propositional logical connectives
--- $\top, p \land q, p \implies q, \bot, p \vee q$ -- as well as the
spatial connectives $\emp, p * q, p \wand q$, and $e \pointsto_A
e'$. Note that the points-to proposition is typed; it indicates that
$e$ is a reference of type $\reftype{A}$ with contents $e'$ of type
$A$.

The quantifiers $\forall u:\omega.\; p$ and $\exists u:\omega.\; p$
are higher-order quantifiers. They can range over all sorts, including
the sort of assertions $\assert$, and so we have the full power of
higher-order separation logic available.  Finally, we have the
\emph{specification embedding assertion} $\validprop{S}$.  This is an
\emph{assertion} that the specification $S$ is true, and is useful for
writing assertions that include facts about the behavior of code.

The specifications $S$ begin with the basic Hoare triple
$\spec{p}{c}{a:A}{q}$, which says that the computation $c$, when run
from a pre-state in $p$, will end in a post-state in $q$, with its
return value named by $a$. Similarly, we have the monadic Hoare triple
form $\mspec{p}{e}{a:A}{q}$ which says that the suspended monadic
computation $e$ (of type $\monad{A}$), will take a pre-state $p$ to a
post-state $q$ if it were to be run. The specification $\setof{p}$ is
the assertion-embedding specification, which says that $p$ is a
tautology of separation logic. 

This does mean that assertions and specifications are mutually
recursive, which means that we will have to give the semantics of
these two syntaxes simultaneously. This is one of the reasons we spent
the first half of this chapter developing the semantics with no
reference to the intended syntax at all -- I wanted to make sure the
semantic domains were all well-defined before giving the
mutually-recursive semantics of the program logic.

\begin{figure}
\begin{displaymath}
\begin{array}{llcl}
\mbox{Propositional Sorts} & 
\upsilon & ::= & \assert \bnfalt \omega \To \upsilon \bnfalt \forallsort{\alpha}{\kappa}{\upsilon}
\\[1em]
\mbox{Sorts} & \omega & ::= & \upsilon \bnfalt \kappa \bnfalt A 
\\[1em]

\mbox{Terms} & 
p,q & ::= & u \bnfalt \pfun{u}{\omega}{p} \bnfalt \pfunall{\alpha}{\kappa}{p} \bnfalt p\;q \bnfalt p\;[\tau] \bnfalt \tau \bnfalt e \\
&   &  |  & \top \bnfalt p \land q \bnfalt p \implies q \bnfalt \bot \bnfalt p \vee q \\
&   &  |  & \emp \bnfalt p * q \bnfalt p \wand q \bnfalt p \pointsto_{A} q \\
&   &  |  & \forall u:\omega.\; p \bnfalt \exists u:\omega.\; p \bnfalt p =_\omega q \bnfalt
            \validprop{S} \\[1em]

\mbox{Specifications} & 
S & ::= & \spec{p}{c}{a:A}{q} \bnfalt \mspec{p}{e}{a:A}{q} \bnfalt \setof{P} \\
& &  |  & S \specand S' \bnfalt S \specimp S' \bnfalt S \specor S' \\
& &  |  & \forall u:\omega.\; S \bnfalt \exists u:\omega.\; S \\[1em]

\mbox{Propositional Contexts} & 
\Delta & ::= & \cdot \bnfalt \Delta, u:\upsilon \\[1em]
\end{array}
\end{displaymath}
\caption{Syntax of Assertions and Specifications}
\label{logic-syntax}  
\end{figure}



A propositional context $\Delta$ is a sequence of sorted variables of
the form $u:\omega$. However, due to the fact that the sorts contain
types, we need a judgement to establish whether a sort is well-formed
with respect to a context of kind variables, and likewise we will need
a judgement to establish whether the propositional context is
well-formed with respect to a context of kind variables.

In Figure~\ref{logic-sort-ok}, we give the judgement
$\judgeSort{\omega}$ used to decide whether or not a given sort is
well-formed or not. The judgement $\judgeACtx{\Delta}$, also given in
Figure~\ref{logic-sort-ok}, then uses the well-sorting judgment to
establish whether a particular context is well-formed or not. 

Finally, we need an equality judgement for sorts, since we have an
equality theory for types. The judgement
$\judgeSortEq{\omega}{\omega'}$, defined in
Figure~\ref{logic-sort-ok}, judges whether two sorts are equal, by
means of an almost-congruence. That is, the rules of this judgement
are all congruence rules, except for the single case
\textsc{SortEqType}, which inherits the equality judgement for
polytypes.

Now that we have defined what the sorts are, we define what it means
for a term to be well-sorted in context with the judgement
$\judgeA{p}{\omega}$, defined in Figure~\ref{logic-prop-ok}.

\begin{figure}
\begin{mathpar}
% \boxed{\judgeACtx{\Delta}} 
% \\
% \inferrule*[right=CtxNil]
%           { }
%           {\judgeACtx{\cdot}}
% \and
% \inferrule*[right=CtxCons]
%           {\judgeACtx{\Delta} \\
%            \judgeSort{\omega}}
%           {\judgeACtx{\Delta, u:\omega}}
% \\
\boxed{\judgeSort{\omega}}
\\
\inferrule*[right=SortKind]
          { }
          {\judgeSort{\kappa}}
% \and
% \inferrule*[right=SortPolyKind]
%           {\judgeACtx{\Delta}}
%           {\judgeSort{\bigstar}}
\and
\inferrule*[right=SortA]
          {\judgeWK{A}{\bigstar}}
          {\judgeSort{A}}
\and
\inferrule*[right=SortProp]
          { }
          {\judgeSort{\assert}}
\and
\inferrule*[right=SortImp]
          {\judgeSort{\omega} \\ \judgeSort{\upsilon}}
          {\judgeSort{\omega \To \upsilon}}
\and
\and
\inferrule*[right=SortImp]
          {\judgeSort[\Theta, \alpha:\kappa]{\upsilon}}
          {\judgeSort{\forallsort{\alpha}{\kappa}{\upsilon}}}

\\
\boxed{\judgeSortEq{\omega}{\omega'}}
\\
\inferrule*[right=SortEqProp]
           { } 
           {\judgeSortEq{\assert}{\assert}}
\and
\inferrule*[right=SortEqKind]
          { }
          {\judgeSortEq{\kappa}{\kappa}}
\and
% \inferrule*[right=SortEqPolyKind]
%           { }
%           {\judgeSortEq{\bigstar}{\bigstar}}
% \and
\inferrule*[right=SortEqImp]
          {\judgeSortEq{\omega}{\omega'} \\
           \judgeSortEq{\upsilon}{\upsilon'} }
          {\judgeSortEq{\omega \To \upsilon}{\omega' \To \upsilon'}}
\and
\inferrule*[right=SortEqType]
          {\judgeKeq{A}{B}{\bigstar}}
          {\judgeSortEq{A}{B}}
\and
\inferrule*[right=SortEqAll]
          {\judgeSortEq[\Theta,\alpha:\kappa]{\upsilon}{\upsilon'}}
          {\judgeSortEq{\forallsort{\alpha}{\kappa}{\upsilon}}{\forallsort{\alpha}{\kappa}{\upsilon'}}}
\end{mathpar}
\caption{Well-sorting of Sorts and Contexts}
\label{logic-sort-ok}
\end{figure}

\begin{figure}
\begin{mathpar}
\boxed{\judgeA{p}{\omega}}
\\
\inferrule*[right=TType]
          {\judgeACtx{\Delta} \\
           \judgectx{\Theta}{\Gamma} \\
           \judgeWK{\tau}{\kappa}}
          {\judgeA{\tau}{\kappa}}
\and
\inferrule*[right=TExpr]
          {\judgeACtx{\Delta} \\
           \judgeE{\Gamma}{e}{A}}
          {\judgeA{e}{A}}
\and
% \inferrule*[right=TPolytype]
%           {\judgeACtx{\Delta} \\
%            \judgeWK[\restrictkind{\Delta}]{A}{\bigstar}}
%           {\judgeA{A}{\bigstar}}
\inferrule*[right=THyp]
          {\judgeACtx{\Delta} \\
           u:\omega \in \Delta }
          {\judgeA{u}{\omega}}
\and
\inferrule*[right=TAbs1]
          {\judgeA[\Theta; \Gamma; \Delta, u:\upsilon']{p}{\upsilon}}
          {\judgeA{\pfun{u}{\upsilon'}{p}}{\upsilon' \To \upsilon}}
\and
\inferrule*[right=TAbs2]
          {\judgeA[\Theta; \Gamma, x:A; \Delta]{p}{\upsilon}}
          {\judgeA{\pfun{x}{A}{p}}{A \To \upsilon}}
\and
\inferrule*[right=TAbs3]
          {\judgeA[\Theta, \alpha:\kappa; \Gamma; \Delta]{p}{\upsilon} \\ \alpha\not\in\FV{\Gamma,\Delta,\upsilon}}
          {\judgeA{\pfun{\alpha}{\kappa}{p}}{\kappa \To \upsilon}}
\and
\inferrule*[right=TApp]
          {\judgeA{p}{\omega \To \upsilon} \\
           \judgeA{q}{\omega}}
          {\judgeA{p\;q}{\upsilon}}
\and
\inferrule*[right=TAbsAll]
          {\judgeA[\Theta, \alpha:\kappa; \Gamma; \Delta]{p}{\upsilon} \\
           \alpha \not\in \FV{\Gamma,\Delta}}
          {\judgeA{\pfunall{\alpha}{\kappa}{p}}{\forallsort{\alpha}{\kappa}{\upsilon}}}
\and
\inferrule*[right=TAppAll]
          {\judgeA{p}{\forallsort{\alpha}{\kappa}{\upsilon}} \\
           \judgeA{\tau}{\kappa}}
          {\judgeA{p\;[\tau]}{[\tau/\alpha]\upsilon}}

\and
\inferrule*[right=TConst]
          {\judgeACtx{\Delta} \\ \judgectx{\Theta}{\Gamma} \\
           c \in \setof{\top, \bot, \emp}}
          {\judgeA{c}{\assert}}
\and
\inferrule*[right=TBinary]
          {\judgeA{p}{\assert} \\
           \judgeA{q}{\assert} \\
           \oplus \in \setof{\land, \vee, \implies, *, \wand}}
          {\judgeA{p \oplus q}{\assert}}
\and
\inferrule*[right=TQuantify1]
          {\judgeA[\Theta; \Gamma; \Delta, u:\omega]{p}{\assert} \\
           Q \in \setof{\forall, \exists}}
          {\judgeA{Q u:\upsilon.\; p}{\assert}}
\and
\inferrule*[right=TQuantify2]
          {\judgeA[\Theta; \Gamma, x:A; \Delta]{p}{\assert} \\
           Q \in \setof{\forall, \exists}}
          {\judgeA{Q x:A.\; p}{\assert}}
\and
\inferrule*[right=TQuantify3]
          {\judgeA[\Theta, \alpha:\kappa; \Gamma; \Delta, u:\omega]{p}{\assert} \\
           Q \in \setof{\forall, \exists} \\
           \alpha \not\in \FV{\Gamma, \Delta}}
          {\judgeA{Q \alpha:\kappa.\; p}{\assert}}
\and
\inferrule*[right=TPointsTo]
          {\judgeA{e}{\reftype{A}} \\
           \judgeA{e'}{A}}
          {\judgeA{e \pointsto_A e'}{\assert}}
\and
\inferrule*[right=TEqual]
          {\judgeA{p}{\omega} \\
           \judgeA{q}{\omega} \\
           \judgeSort{\omega}}
          {\judgeA{p =_\omega q}{\assert}}
\and
\inferrule*[right=TSpec]
          {\judgeS{S}}
          {\judgeA{\validprop{S}}{\assert}}
\and
\inferrule*[right=TEqSort]
          {\judgeSortEq{\omega}{\omega'} \\
           \judgeA{p}{\omega'}}
          {\judgeA{p}{\omega}}
\end{mathpar}
\caption{Well-sorting of Assertions}
\label{logic-prop-ok}
\end{figure}

In the rules \textsc{TType} and \textsc{TExpr}, we simply
inherit well-kindedness and well-typedness from the corresponding
judgements for types and terms.  The rule \textsc{THyp} is the
hypothesis rule for propositional variables. (Type and program
expression variables can only be referenced through the
\textsc{TType} and \textsc{TExpr} rules.)


%
% \begin{figure}
% \begin{displaymath}
% \begin{array}{lcl}
% \restrictkind{\cdot}                 & = & \cdot \\
% \restrictkind{\Delta, \alpha:\kappa} & = & \restrictkind{\Delta}, \alpha:\kappa \\
% \restrictkind{\Delta, x:A}           & = & \restrictkind{\Delta} \\
% \restrictkind{\Delta, u:\assert}     & = & \restrictkind{\Delta} \\
% \restrictkind{\Delta, u:\omega \To \omega'} & = & \restrictkind{\Delta} \\[1em]
% 
% \restricttype{\cdot}                 & = & \cdot \\
% \restricttype{\Delta, \alpha:\kappa} & = & \restricttype{\Delta} \\
% \restricttype{\Delta, x:A}           & = & \restricttype{\Delta}, x:A \\
% \restricttype{\Delta, u:\assert}     & = & \restricttype{\Delta} \\
% \restricttype{\Delta, u:\omega \To \omega'} & = & \restricttype{\Delta} \\[1em]
% 
% \restricttyenv{\cdot}{\unit}                     & = & \unit \\ 
% \restricttyenv{\Delta, u:\kappa}{(\delta, x)}    & = & (\restricttyenv{\Delta}{\delta}, x) \\
% \restricttyenv{\Delta, u:\assert}{(\delta, x)}   & = & \restricttyenv{\Delta}{\delta} \\
% \restricttyenv{\Delta, u:A}{(\delta, x)}         & = & \restricttyenv{\Delta}{\delta} \\
% \restricttyenv{\Delta,u:\omega \To \omega'}{(\delta, x)} &=& \restricttyenv{\Delta}{\delta} \\[1em]
% 
% \restrictvals{\cdot}{\unit}                     & = & \unit                             \\ 
% \restrictvals{\Delta, u:\kappa}{(\delta, \tau)} & = & \restrictvals{\Delta}{\delta}      \\
% \restrictvals{\Delta, u:\assert}{(\delta, x)}   & = & \restrictvals{\Delta}{\delta}      \\
% \restrictvals{\Delta, u:A}{(\delta, x)}         & = & (\restrictvals{\Delta}{\delta}, x) \\
% \restrictvals{\Delta, u:\omega \To \omega'}{(\delta, x)} &=& \restrictvals{\Delta}{\delta} \\[1em]
% \end{array}
% \end{displaymath}
% \caption{Auxilliary Context Operations}
% \label{context-ops}  
% \end{figure}


% 
% (There is a minor technical issue here: we can refer to a program
% variable either via the \textsc{TermExpr} rule, or using the
% \textsc{TermHyp} rule. Strictly speaking, this means that well-sorting
% derivations are not unique, and so when we give a semantics, we would
% need to prove a coherence theorem. However, this could easily be
% avoided by adding some syntactic annotation to the use of program
% terms (for instance, by writing $\left\lfloor e \right\rfloor$ or
% similarly for variables), so we will simply let this issue slide.
% The \textsc{TermAbs} and \textsc{TermApp} rules give the
lambda-abstraction and application rules for the function sort $\omega
\To \upsilon$ -- there are no surprises here. Likewise, the
\textsc{TAbsAll} and \textsc{TAppAll} rules allow abstracting
applying kind-indexed products.

Finally, there are all
the rules giving the sorting of propositions. The nullary propositions
$\top, \bot$, and $\emp$ are typed with the \textsc{TConst}
rule, and the binary propositions $\land, \vee, \implies, *$, and
$\wand$ are typed with the \textsc{TBinary} rule, requiring
their two arguments to both be of sort $\assert$.


The two quantifiers $\forall u:\omega.\;p$ and $\exists u:\omega.\;p$
are sorted with the \textsc{TQuantify} rules. We have three
variants of this rule, putting a new variable into different contexts
depending on whether the variable is a type, term or propositional
variable.  The points-to $e \pointsto_A e'$ and equality $p =_\omega
q$ each require that their arguments be of the correct sort. Note that
$e \pointsto_A e'$ is restricted to program types, as expected,
whereas equality is valid at any sort.

Finally, we have the rule $\textsc{TSpec}$ for the
specification-embedding assertion $\validprop{S}$, which recursively
invokes the well-sorted specification $\judgeS{S}$. This judgement is
defined in Figure~\ref{logic-spec-ok}, and consists of a handful of
rules. The \textsc{SpecTriple} rule asserts that $\spec{p}{c}{a:A}{q}$
is well-kinded when $A$ is a type, $p$ is an assertion, $c$ is a
computation yielding an $A$, and $q$ is an assertion with $a$ as an
extra free variable.  Likewise the \textsc{SpecMTriple} rule does the
same job for monadic expressions, saying that $\mspec{p}{e}{a:A}{q}$,
saying that $e$ must be a term of monadic type $\monad{A}$, but
otherwise as in the \textsc{SpecTriple} rule.  Finally, the remaining
atomic proposition \textsc{SpecAssert} rule recursively calls back into the
assertion well-kinding judgement.

The \textsc{SpecBinary} rules gives well-formedness conditions for the
conjunction ($S \specand S'$), disjunction ($S \specor S'$), and
implication ($S \specimp S'$) over specifications, in each case asking
the subterms to be well-formed specifications. The quantifier rules
\textsc{SpecQuantify} simply extend the context with the newly
quantified variable. As with assertions, we have three versions of
this rule, one for each context.

Since $\spectype$ is not a sort, this means that the language of
specifications is a multi-sorted first-order logic, rather than a
higher-order logic. There are no technical obstacles to extending it
in this fashion, but we have not felt any strong need to do so.


\begin{figure}
\begin{mathpar}
\boxed{\judgeS{S}} \\
\inferrule*[right=SpecTriple]
          {\judgeA{p}{\assert} \\
           \judgeA{\comp{c}}{\monad{A}} \\
           \judgeA[\Theta; \Gamma, a:A; \Delta]{q}{\assert}}
          {\judgeS{\spec{p}{c}{a:A}{q}}}
\and
\inferrule*[right=SpecMTriple]
          {\judgeA{p}{\assert} \\
           \judgeA{e}{\monad{A}} \\
           \judgeA[\Theta; \Gamma, a:A; \Delta]{q}{\assert}} 
          {\judgeS{\mspec{p}{e}{a:A}{q}}}
\and
\inferrule*[right=SpecAssert]
          {\judgeA{p}{\assert}}
          {\judgeS{\setof{p}}}
\and
\inferrule*[right=SpecQuantify1]
          {\judgeS[\Theta; \Gamma; \Delta, u:\upsilon]{S} \\ u \in \setof{\forall, \exists}}
          {\judgeS{Q u:\upsilon.\; S}}
\and
\inferrule*[right=SpecQuantify2]
          {\judgeS[\Theta, \alpha:\kappa, \Gamma, \Delta]{S} \\ u \in \setof{\forall, \exists} \\
           \alpha \not\in \FV{\Gamma,\Delta}}
          {\judgeS{Q \alpha:\kappa.\; S}}
\and
\inferrule*[right=SpecQuantify3]
          {\judgeS[\Theta; \Gamma, x:A; \Delta]{S} \\ u \in \setof{\forall, \exists}}
          {\judgeS{Q x:A.\; S}}
\and
\inferrule*[right=SpecBinary]
          {\judgeS{S_1} \\ \judgeS{S_2} \\ \oplus \in \setof{\specand, \specor, \specimp}}
          {\judgeS{S_1 \oplus S_2}}
\end{mathpar}
\caption{Well-sorting of Specifications}
\label{logic-spec-ok}
\end{figure}

\subsection{Substitution Properties}

In the syntax for the program logic, we have systematically made the
decision to forbid the appearance of logical expressions within
program expressions. We accomplish this by prohibiting variables from
having sorts like $P : \assert \To \N$. As a result, it is not
possible to form program expressions which depend on subexpressions of
logical type, such as writing an expression of monadic type
$\monad{\N}$ like $\comp{P(x \pointsto y)}$.

This is because the restriction operations ensure that program
expressions never use variables whose of logical type, and since the
sort of $P$ is neither a kind nor a type, it cannot appear within a
term typed by the \textsc{TExpr} rule.

By requiring program expressions to only contain programmatic
subexpressions, we effectively forbid the use of ``ghost expressions''
in our programs. This restriction greatly simplify the semantics of
the program logic. I intend for the sorts of our logic to be
interpreted in Set, but the terms of our programming language are
interpreted in CPO. It is not trivial to intermingle logical and
program expressions, and so the simplest course is to forbid this
mixing.

However, this does not restrict the expressiveness of our program
logic reasoning, since our program logic is a full specification
logic. We have no need of ghost state, since we can place quantifiers
outside of Hoare triples. This lets us relate variables across a pre-
and post-condition without any dubious manipulations of binding
structure or extensions of the heap semantics.

The price we must pay for this simplicity is an increase in the
complexity of the substitution theorems of the logic: since we have
three contexts, we need six substitution theorems --- three each for
assertions and specifications. We will give these theorems after we
have given the proofs, since it will be convenient to state the syntactic
and semantic substitution properties together. 

\section{Semantics}

Now, we will consider the interpretation of the syntax. To do this,
we will proceed in stages. First, we will show how to interpret sorts and
contexts by sets, and show that the equality judgement for sorts is
sound. Using this, we will then give interpretation of terms and
specifications, and show that the interpretation is sound with respect
to substitution.

\subsection{Interpretation of Sorts}

We will start by explaining how to interpret the sorts. Essentially,
we are going to define assertions to be the powerset of heaps, and
specifications to be upward closed sets of assertions, but because we
allow types as sorts, and types may have free type variables, we will
need to give a pair of indexed, recursive definitions to make
everything work properly. We give these definitions in
Figure~\ref{sort-interpretation}. 

\begin{figure}
\begin{displaymath}
  \begin{array}{lcl}
    \interp{\judgeSort{\omega}} & \in & \interp{\Theta} \to \mathrm{Set} \\[0.5em]
    \interp{\judgeSort{\kappa}} \;\theta
    & = & \interp{\kappa} 
    \\
    \interp{\judgeSort{A}}\;\theta
    & = & U(\interp{\judgeWK{A}{\bigstar}}\;\theta)
    \\
    \interp{\judgeSort{\assert}}\;\theta
    & = & \powerset{H}
    \\
    \interp{\judgeSort{\omega \To \upsilon}}\;\theta
    & = & \interp{\judgeSort{\omega}}\;\theta \to \interp{\judgeSort{\omega}}\;\theta
    \\
    \interp{\judgeSort{\forallsort{\alpha}{\kappa}{\upsilon}}}\;\theta
    & = & \Pi \tau \in \interp{\kappa}.\;\interp{\judgeSort[\Theta, \alpha:\kappa]{\upsilon}}\;(\theta,\tau)
    \\[2em]
    \interp{\judgeACtx{\Delta}} & \in & \interp{\Theta} \to \mathrm{Set} \\[0.5em]
    \interp{\judgeACtx{\cdot}}\;\theta & = & 1 \\
    \interp{\judgeACtx{\Delta, u:\upsilon}}\;\theta & = & \interp{\judgeACtx{\Delta}}\;\theta \times
                                                          \interp{\judgeSort{\upsilon}\;\theta}

  \end{array}
\end{displaymath}
\caption{Interpretation of Sorts and Contexts}
\label{sort-interpretation}  
\end{figure}

A well-sorting derivation is interpreted as a function from a tuple
representing the type constructor environment into a set. The
interpretation of a kind $\kappa$ is just its set-theoretic semantics,
defined in the previous chapter. The interpretation of a type $A$ is
its domain-theoretic interpretation, hit with the forgetful functor
$U$ to get only its underlying set of points.  The interpretation of
propositions $\assert$ is the powerset of heaps, and the function
space $\omega \To \upsilon$ is just the set-theoretic function space
between the two sorts. The type-indexed sort
$\forallsort{\alpha}{\kappa}{\upsilon}$ is interpreted by an indexed
family of sets, with the index extendeding the context.

We show that this definition is compatible with sort-equality below:

\begin{lemma}{(Soundness of Sort Equality)}
If $\judgeSortEq{\omega}{\omega'}$ is derivable, then
$\interp{\judgeSort{\omega}} = \interp{\judgeSort{\omega'}}$.
\end{lemma}
\begin{proof}
  The proof is a routine induction.
\end{proof}


% Therefore, we will need to give a mutually-recursive definition
% between the interpretation the two judgements $\judgeACtx{\Delta}$ and
% $\judgeA{\Delta}{\omega}$.
% 
% \begin{displaymath}
%   \begin{array}{lcl}
%     \interp{\judgeACtx{\Delta}} & \in & \mbox{Set} \\
%     \interp{\judgeSort{\omega}} & \in & \interp{\judgeACtx{\Delta}} \to \mbox{Set} \\[1em]
% 
%     \interp{\judgeACtx{\cdot}} & = & \unittype \\
%     \interp{\judgeACtx{\Delta, u:\omega}} & = & \sum \delta \in \interp{\judgeACtx{\Delta}}.\; 
%                                                   (\interp{\judgeSort{\omega}}\;\delta) \\[1em]
%  
%     \interp{\judgeSort{\kappa}}\;\delta & = & \interp{\kappa} \\
%     \interp{\judgeSort{A}}\;\delta & = & 
%        U(\interp{\judgeWK[\restrictkind{\Delta}]{A}{\bigstar}}\;(\restricttyenv{\Delta}{\delta})) \\
%     \interp{\judgeSort{\assert}}\;\delta & = & \powerset{H} \\
%     \interp{\judgeSort{\omega \To \omega'}}\;\delta & = &  
%        \interp{\judgeSort{\omega}}\;\delta \to \interp{\judgeSort{\omega}}\;\delta \\
%   \end{array}
% \end{displaymath}
% 
% The semantic brackets are slightly overloaded in this definition. To
% interpret a type $A$, we take it to be the set of the elements of the
% domain-theoretic interpretation of $A$, and to interpret a kind
% $\kappa$, we take it to be the set of closed canonical forms of that
% type. Both of these use the interpretations given in the previous
% chapter. 
% 
% In order to establish that this actually defines a function, we will
% need to prove some supporting lemmas. First, we need to establish that
% we an also prove that the context $\Delta$ is well-formed is
% $\judgeSort{\omega}$ is derivable:
% 
% \begin{lemma}{(Well-sortedness implies well-formed contexts)}
% If it is the case that $\judgeSort{\omega}$, then $\judgeACtx{\Delta}$.
% \end{lemma}
% \begin{proof}
% The proof is by a trivial structural induction over the derivation $\judgeSort{\omega}$. 
% \end{proof}
% 
% Now, this means that the domain of the interpretation function
% $\interp{\judgeACtx{\Delta}}$ is always well-defined, given a $\judgeSort{\omega}$. Next,
% we need to establish that we get the same result, regardless of \emph{which} derivation
% of this judgement we found. 
% 
% \begin{lemma}{(Uniqueness of Interpretations)}
% We have that:
% \begin{enumerate}
% \item If we have two derivations $\mathcal{D}_1 :: \judgeACtx{\Delta}$
%       and $\mathcal{D}_2 :: \judgeACtx{\Delta}$, 
%       then $\interp{\mathcal{D}_1} = \interp{\mathcal{D}_2}$
% \item If we have two derivations $\mathcal{D}_1 ::\; \judgeSort{\omega}$
%       and $\mathcal{D}_2 ::\; \judgeSort{\omega}$, 
%       then $\interp{\mathcal{D}_1} = \interp{\mathcal{D}_2}$
% \end{enumerate}
% \end{lemma}
% \begin{proof}
%   The proof is at the end of the chapter.
% \end{proof}
% 
% \subsubsection{Interpretation of Equality Judgement on Sorts}
% 
% Now, we will show that our equality judgement on sorts $\judgeSortEq{\omega}{\omega'}$ is
% sound. 
% 
% \begin{lemma}{(Soundness of Sort Equality)}
% If $\judgeACtx{\Delta}$, $\judgeSort{\omega}$, and $\judgeSort{\omega'}$ are derivable, then if 
% $\judgeSortEq{\omega}{\omega'}$ is derivable, then $\interp{\judgeSort{\omega}} = \interp{\judgeSort{\omega'}}$. 
% \end{lemma}
% \begin{proof}
%   The proof is at the end of the chapter. 
% \end{proof}

\subsection{Interpretation of Terms and Specifications}

The term judgement $\judgeA{p}{\omega}$ is mutually recursively
defined with the specification judgement $\judgeS{S}$. Therefore, when
we give the semantics of these judgements, we need to define them
together. The definition of the interpretation of these two judgments
is given in Figures~\ref{term-interpretation} and
\ref{spec-interpretation}.

There are no surprises in the interpretations -- everything is
interpreted straightforwardly. The only unusual feature is that
lambda-abstractions and quantifiers have three cases, corresponding to
their three typing rules. However, the interpretation is still
syntax-directed, since types, kinds, and propositional sorts are
syntactically distinct.

\subsection{Substitution Properties}

We state the two main soundness theorems below. 

\begin{figure}
\begin{displaymath}
\begin{array}{lcl}
\interp{\judgeA{p}{\omega}} & \in & \prod \theta \in \interp{\theta}, \gamma \in \interp{\judgectx{\Theta}{\Gamma}}\;\theta, \delta \in \interp{\judgeACtx{\Delta}}\;\theta.\\
                            &     & \qquad \to \interp{\judgeSort{\omega}}\;\theta \gamma \delta \\
\interp{\judgeS{S}}         & \in & \prod \theta \in \interp{\theta}, \gamma \in \interp{\judgectx{\Theta}{\Gamma}}\;\theta, \delta \in \interp{\judgeACtx{\Delta}}\;\theta.\\
                            &     &\qquad\to \upset{W(\powerset{H})} \\[1em]

\interp{\judgeA{\tau}{\kappa}}\;\theta\,\gamma\,\delta & = & \interp{\judgeWK{\tau}{\kappa}}\;\theta \\

\interp{\judgeA{e}{A}}\;\theta\,\gamma\,\delta & = &  U(\interp{\judgeE{\Gamma}{e}{A}}\;\theta)\;\gamma \\

\interp{\judgeA{\pfun{u}{\upsilon'}{p}}{\upsilon' \To \upsilon}}\;\theta\,\gamma\,\delta & = & 
   \semfun{v \in \interp{\judgeSort{\upsilon'}}}
          {\interp{\judgeA[\Theta; \Gamma; \Delta,u:\upsilon']{p}{\upsilon}}\;\theta\,\gamma\,(\delta, v)} \\

\interp{\judgeA{\pfun{x}{A}{p}}{A \To \upsilon}}\;\theta\,\gamma\,\delta & = & 
   \semfun{v \in \interp{\judgeSort{A}}}
          {\interp{\judgeA[\Theta;\Gamma,x:A;\Delta]{p}{\upsilon}}\;\theta\,(\gamma,v)\,\delta} \\

\interp{\judgeA{\pfun{\alpha}{\kappa}{p}}{\kappa \To \upsilon}}\;\theta\,\gamma\,\delta & = & 
   \semfun{\tau \in \interp{\judgeSort{\kappa}}}
          {\interp{\judgeA[\Theta, \alpha:\kappa;\Gamma;\Delta]{p}{\upsilon}}\;(\theta,\tau)\,\gamma\,\delta} \\

\interp{\judgeA{p\;q}{\upsilon}}\;\delta & = & 
   (\interp{\judgeA{p}{\omega \To \upsilon}}\;\delta)\;(\interp{\judgeA{q}{\omega}}\;\delta) \\

\interp{\judgeA{\pfunall{\alpha}{\kappa}{p}}{\forallsort{\alpha}{\kappa}{\upsilon}}}\;\theta\,\gamma\,\delta & = &
   \semfun{\tau \in \interp{\judgeSort{\kappa}}\;\theta}
          {\interp{\judgeA[\Theta, \alpha:\kappa; \Gamma; \Delta]{p}{\upsilon}}\;(\theta,\tau)\,\gamma\,\delta}
\\
\interp{\judgeA{p\;[\tau]}{[\tau/\alpha]\upsilon}}\;\theta\,\gamma\,\delta & = & 
   (\interp{\judgeA{p}{\forallsort{\alpha}{\kappa}{\upsilon}}}\;\theta\,\gamma\,\delta)\;
   (\interp{\judgeWK{\tau}{\kappa}}\;\theta) \\


\interp{\judgeA{u}{\omega}}\;\theta\,\gamma\,\delta & = & \pi_u(\delta) \\

\interp{\judgeA{c}{\assert}}\;\theta\,\gamma\,\delta & = & \interp{c}^0 \\

\interp{\judgeA{p \oplus q}{\assert}}\;\theta\,\gamma\,\delta & = & 
    \interp{\judgeA{p}{\assert}}\;\theta\,\gamma\,\delta \\
& & \interp{\oplus}^2\;\;\interp{\judgeA{q}{\assert}}\;\theta\,\gamma\,\delta \\

\interp{\judgeA{e \pointsto_A e'}{\assert}}\;\theta\,\gamma\,\delta & = & 
    \mbox{let } l = \interp{\judgeE{\Gamma}{e}{\reftype{A}}}\;\theta\,\gamma \mbox{ in}\\
& & \mbox{let } v = \interp{\judgeE{\Gamma}{e'}{A}}\;\theta\,\gamma \mbox{ in} \\
& & l \pointsto v \\


\interp{\judgeA{p =_\omega q}{\assert}}\;\theta\,\gamma\,\delta & = & 
   \mbox{if }\interp{\judgeA{p}{\omega}}\;\theta\,\gamma\,\delta = \interp{\judgeA{q}{\omega}}\;\theta\,\gamma\,\delta \\
   & & \mbox{ then } \top \mbox{ else} \bot \\

\interp{\judgeA{Q u:\upsilon.\; p}{\assert}}\;\theta\,\gamma\,\delta & = & 
  \interp{Q}^\infty_{v \in \interp{\judgeSort{\upsilon}}\;\theta}
    \interp{\judgeA[\Theta; \Gamma; \Delta, u:\omega]{p}{\assert}}\;\theta\,\gamma\,(\delta, v) \\

\interp{\judgeA{Q x:A.\; p}{\assert}}\;\theta\,\gamma\,\delta & = & 
  \interp{Q}^\infty_{v \in \interp{\judgeSort{A}}\;\theta}
    \interp{\judgeA[\Theta; \Gamma, x:A; \Delta]{p}{\assert}}\;\theta\,(\gamma,v)\,\delta \\

\interp{\judgeA{Q \alpha:\kappa.\; p}{\assert}}\;\theta\,\gamma\,\delta & = & 
  \interp{Q}^\infty_{\tau \in \interp{\judgeSort{\kappa}}\;\theta}
    \interp{\judgeA[\Theta, \alpha:\kappa; \Gamma; \Delta]{p}{\assert}}\;(\theta, \tau)\,\gamma\,\delta \\


\interp{\judgeA{\validprop{S}}{\assert}}\;\theta\,\gamma\,\delta & = & 
   \mbox{if } \interp{\judgeS{S}}\;\theta\,\gamma\,\delta = \top_{\upset{W(\powerset{H})}}
   \mbox{ then } \top
   \mbox{ else } \bot \\

\interp{\judgeA{p}{\omega}}\;\theta\,\gamma\,\delta & = & 
   \interp{\judgeA{p}{\omega'}}\;\delta \mbox{ when } \judgeSortEq{\omega}{\omega'} \\[1em]

\interp{\top}^0 & = & \top \\
\interp{\bot}^0 & = & \bot \\
\interp{\emp}^0 & = & I \\[1em]

\interp{\land}^2    & = & \land \\
\interp{\implies}^2 & = & \implies \\
\interp{\vee}^2     & = & \vee \\
\interp{*}^2        & = & * \\
\interp{\wand}^2    & = & \wand \\[1em]

\interp{\forall}^\infty & = & \bigwedge \\
\interp{\exists}^\infty & = & \bigvee \\[1em]

l \pointsto v & = & \setof{(\setof{l}, \semfun{loc}{v \mbox{ when } loc = l})}
\end{array}
\end{displaymath}
\caption{ Interpretation of Terms }
\label{term-interpretation}  
\end{figure}

\begin{figure}
\begin{displaymath}
\begin{array}{lcl}
\interp{\judgeS{\spec{p}{c}{a:A}{q}}}\;\theta\,\gamma\,\delta & = & 
   \begin{array}{l}
     \setof{\interp{\judgeA{p}{\assert}}\;\theta\,\gamma\,\delta} \\
      \interp{\judgeC{\Gamma}{c}{A}}\;\theta\;\gamma\;\delta \\
     \setof{v.\; \interp{\judgeA[\Delta, a:A]{q}{\assert}}\;\theta\,(\gamma,v)\,\delta} \\
   \end{array} 
\\[2em]

\interp{\judgeS{\mspec{p}{e}{a:A}{q}}}\;\theta\,\gamma\,\delta & = & 
   \begin{array}{l}
     \setof{\interp{\judgeA{p}{\assert}}\;\theta\,\gamma\,\delta} \\
      \interp{\judgeE{\Gamma}{e}{\monad{A}}}\;\theta\;\gamma\;\delta \\
     \setof{v.\; \interp{\judgeA[\Delta, a:A]{q}{\assert}}\;\theta\,(\gamma,v)\,\delta} \\
   \end{array} 
\\[2em]

\interp{\judgeS{\setof{p}}}\;\theta\,\gamma\,\delta & = & 
  \mbox{if } \interp{\judgeA{p}{\assert}}\;\theta\,\gamma\,\delta = \top_{\powerset{H}}
  \mbox{ then } \top 
  \mbox{ else } \bot \\

\interp{\judgeS{S_1 \oplus S_2}}\;\theta\,\gamma\,\delta & = & 
  \interp{\judgeS{S_1}}\;\theta\,\gamma\,\delta \;\;\interp{\oplus} \\ 
&& \interp{\judgeS{S_2}}\;\theta\,\gamma\,\delta \\

\interp{\judgeS{Q u:\upsilon.\; S}}\;\theta\,\gamma\,\delta & = & 
  \interp{Q}_{v \in \interp{\judgeSort{\upsilon}}\;\theta} 
     \interp{\judgeS[\Theta;\Gamma;\Delta, u:\upsilon]{S}}\;\theta\,\gamma\,(\delta,v) \\

\interp{\judgeS{Q x:A.\; S}}\;\theta\,\gamma\,\delta & = & 
  \interp{Q}_{v \in \interp{\judgeSort{A}}\;\theta} 
     \interp{\judgeS[\Theta;\Gamma,x:A;\Delta]{S}}\;\theta\,(\gamma,v)\,\delta \\

\interp{\judgeS{Q \alpha:\kappa.\; S}}\;\theta\,\gamma\,\delta & = & 
  \interp{Q}_{\tau \in \interp{\judgeSort{\kappa}}\;\theta} 
     \interp{\judgeS[\Theta,\alpha:\kappa;\Gamma;\Delta]{S}}\;(\theta,\tau)\,\gamma\,\delta \\[1em]


\interp{\specand} & = & \land \\
\interp{\specor}  & = & \vee  \\
\interp{\specimp} & = & \implies \\[1em]

\interp{\forall} & = & \bigwedge \\
\interp{\exists} & = & \bigvee \\

\end{array}
\end{displaymath}
\caption{Interpretation of Specifications}
\label{spec-interpretation}  
\end{figure}

\subsubsection{Semantics of Substitution}

\begin{lemma}{(Substitution for Sorts)}
Suppose $\judgeWK{\tau}{\kappa}$.
\begin{enumerate}
\item If $\judgeSort[\Theta, \alpha:\kappa]{\omega}$, then $\judgeSort[\Theta]{[\tau/\alpha]\omega}$ and \\
      $\interp{\judgeSort[\Theta]{[\tau/\alpha]\omega}}\;\theta$ equals
      $\interp{\judgeSort[\Theta, \alpha:\kappa]{\omega}}\;(\theta, \interp{\judgeWK{\tau}{\kappa}}\;\theta)$
\item If $\judgeACtx[\Theta, \alpha:\kappa]{\Delta}$, then $\judgeACtx{{[\tau/\alpha]}\Delta}$ and \\
      $\interp{\judgeACtx{{[\tau/\alpha]}\Delta}}\;\theta$ equals 
      $\interp{\judgeACtx[\Theta, \alpha:\kappa]{\Delta}}(\theta, \interp{\judgeWK{\tau}{\kappa}}\;\theta)$ 
\item If $\judgeSortEq[\Theta, \alpha:\kappa]{\omega}{\omega'}$, then
$\judgeSortEq{{[\tau/\alpha]}\omega}{{[\tau/\alpha]}\omega'}$. 
\end{enumerate}
\end{lemma}
\begin{proof}
  These lemmas follow from a routine induction.
\end{proof}


\begin{lemma}{(Substitution for Terms and Specifications)}
\begin{enumerate}
\item Suppose $\judgeWK{\tau}{\kappa}$.
  \begin{enumerate}
  \item If $\judgeA[\Theta, \alpha:\kappa; \Gamma; \Delta]{p}{\omega}$, 
        then 
        \begin{enumerate}
        \item $\judgeA[\Theta; {[\tau/\alpha]}\Gamma; {[\tau/\alpha]}\Delta]{{[\tau/\alpha]}p}{{[\tau/\alpha]}\omega}$
        \item $\interp{\judgeA[\Theta; {[\tau/\alpha]}\Gamma; {[\tau/\alpha]}\Delta]{{[\tau/\alpha]}p}{{[\tau/\alpha]}\omega}}\;\theta\,\gamma\,\delta$ equals \\
          $\interp{\judgeA[\Theta, \alpha:\kappa; \Gamma, \Delta]{p}{\omega}}\;(\theta,\interp{\judgeWK{\tau}{\kappa}}\;\theta)\,\gamma\,\delta$
        \end{enumerate}
  \item If $\judgeS[\Theta, \alpha:\kappa; \Gamma; \Delta]{S}$, 
        then 
        \begin{enumerate}
        \item $\judgeS[\Theta; {[\tau/\alpha]}\Gamma; {[\tau/\alpha]}\Delta]{{[\tau/\alpha]}S}$
        \item $\interp{\judgeS[\Theta; {[\tau/\alpha]}\Gamma; {[\tau/\alpha]}\Delta]{{[\tau/\alpha]}S}}\;\theta$ equals \\
              $\interp{\judgeS[\Theta, \alpha:\kappa; \Gamma; \Delta]{S}}(\theta, \interp{\judgeWK{\tau}{\kappa}}\;\theta)$
        \end{enumerate}
  \end{enumerate}
\item Suppose $\judgeE{\Gamma}{e}{A}$.
  \begin{enumerate}
  \item If $\judgeA[\Theta; \Gamma, x:A; \Delta]{p}{\omega}$, 
        then 
        \begin{enumerate}
        \item $\judgeA[\Theta; \Gamma; \Delta]{[e/x]p}{\omega}$, 
        \item $\interp{\judgeA[\Theta; \Gamma; \Delta]{[e/x]p}{\omega}}\;\theta\,\gamma\,\delta$ equals \\
              $\interp{\judgeA[\Theta; \Gamma, x:A; \Delta]{p}{\omega}}\;\theta\,(\gamma, \interp{\judgeE{\Gamma}{e}{A}}\;\theta\;\gamma)\,\delta$               
        \end{enumerate}
  \item If $\judgeS[\Theta; \Gamma, x:A; \Delta]{S}$, 
        then 
        \begin{enumerate}
        \item $\judgeS[\Theta; \Gamma; \Delta]{[e/x]S}$.
        \item $\interp{\judgeS[\Theta; \Gamma; \Delta]{[e/x]S}}\;\theta\,\gamma\,\delta$ equals \\
              $\interp{\judgeS[\Theta; \Gamma, x:A; \Delta]{S}}\;\theta\,(\gamma, \interp{\judgeE{\Gamma}{e}{A}}\;\theta\;\gamma)\,\delta$
        \end{enumerate}
  \end{enumerate}
\item Suppose $\judgeA{q}{\upsilon}$.
  \begin{enumerate}
  \item If $\judgeA[\Theta; \Gamma; \Delta, u:\upsilon]{p}{\omega}$, 
        then 
        \begin{enumerate}
          \item $\judgeA[\Theta; \Gamma; \Delta]{[q/u]p}{\omega}$
          \item $\interp{\judgeA[\Theta; \Gamma; \Delta]{[q/u]p}{\omega}}\;\theta\,\gamma\,\delta$ equals \\
                $\interp{\judgeA[\Theta; \Gamma; \Delta, u:\upsilon]{p}{\omega}}\;\theta\,\gamma\,(\delta, \interp{\judgeA{q}{\upsilon}}\;\theta\,\gamma\,\delta)$
        \end{enumerate}
      \item If $\judgeS[\Theta; \Gamma:A; \Delta, u:\upsilon]{S}$, 
        then 
        \begin{enumerate}
          \item $\judgeS[\Theta; \Gamma; \Delta]{[q/u]S}$ 
          \item $\interp{\judgeS[\Theta; \Gamma; \Delta]{[q/u]S}}\;\theta\,\gamma\,\delta$ equals \\
                $\interp{\judgeS[\Theta; \Gamma:A; \Delta, u:\upsilon]{S}}\;\theta\,\gamma\,(\delta, \interp{\judgeA{q}{\upsilon}}\;\theta\,\gamma\,\delta)$
        \end{enumerate}
  \end{enumerate}

\end{enumerate}
\end{lemma}
\begin{proof}
  The proof is at the end of the chapter.
\end{proof}


\section{The Program Logic}

\begin{figure}
\begin{mathpar}
  \inferrule*[right=AxEquiv1]
         {\judgeS{\spec{p}{c}{a:A}{q} \specimp \mspec{p}{\comp{c}}{a:A}{q}}} 
         {\validS{\spec{p}{c}{a:A}{q} \specimp \mspec{p}{\comp{c}}{a:A}{q}}} 
\and 
  \inferrule*[right=AxEquiv2]
    {\judgeS{\mspec{p}{c}{a:A}{q} \specimp \spec{p}{\comp{c}}{a:A}{q}}} 
    {\validS{\mspec{p}{c}{a:A}{q} \specimp \spec{p}{\comp{c}}{a:A}{q}}} 
\and

  \inferrule*[right=AxReturn]
    {\judgeS{\spec{P}{e}{a:A}{P \land a = e}}}
    {\validS{\spec{P}{e}{a:A}{P \land a = e}}}
\and

  \inferrule*[right=AxAssign]
    {\judgeS{\spec{e \pointsto_A -}{e := e'}{a:\unittype}{e \pointsto_A e'}}}
    {\validS{\spec{e \pointsto_A -}{e := e'}{a:\unittype}{e \pointsto_A e'}}}
\and

  \inferrule*[right=AxAlloc]
    {\judgeS{\spec{\emp}{\newref{A}{e}}{a:\reftype{A}}{a \pointsto e}}}
    {\validS{\spec{\emp}{\newref{A}{e}}{a:\reftype{A}}{a \pointsto e}}}
\and

  \inferrule*[right=AxDeref]
    {\judgeS{\spec{e \pointsto_A e'}{!e}{a:A}{e \pointsto_A e' \land a = e'}}}
    {\validS{\spec{e \pointsto_A e'}{!e}{a:A}{e \pointsto_A e' \land a = e'}}}
\and

  \inferrule*[right=AxBind]
            {\validS{\mspec{p}{e}{x:A}{q}} \\
             \validS[\Delta,x:A]{\spec{q}{c}{a:B}{r}} \\
             \judgeA[\Delta,a:A]{q}{\assert}}
            {\validS{\spec{p}{\letv{x}{e}{c}}{a:B}{r}}}
\and

  \inferrule*[right=AxFix]
            {\validS{(\forall x:\monad{A}.\; \mspec{p}x{a:A}{q(a)} \specimp \mspec{p}{e}{a:A}{q})}}
            {\validS{\mspec{p}{\fix{x:\monad{A}}{e}}{a:A}{q}}}
\end{mathpar}
\caption{Basic Axioms of Specification Logic}
\label{spec-logic-axioms-basic}
\end{figure}

\begin{figure}
\begin{mathpar}
  \inferrule*[right=AxExtract]
             {\validS{\setof{r} \specimp \spec{p}{c}{a:A}{q}} \\
              \mbox{$r$ is a pure formula} \\
              \validP{p \implies r}}
             {\validS{\spec{p}{c}{a:A}{q}}}
\and

  \inferrule*[right=AxEmbed]
             {\validS{\setof{r} \specimp \spec{p \land r}{c}{a:A}{q}}}
             {\validS{\setof{r} \specimp \spec{p}{c}{a:A}{q}}}
\and

  \inferrule*[right=AxUseValid]
             { }
             {\validS{\setof{\validprop{S}} \specimp S}}
\and

  \inferrule*[right=AxForgetEx]
             {\validS[\Delta, u:\omega]{\spec{p}{c}{a:A}{q}} \\ u\not\in\mathrm{FV}(c) \\ u\not\in\mathrm{FV}(q)}
             {\validS{\spec{\exists u:\omega.\;p}{c}{a:A}{q}}}
\and
  \inferrule*[right=AxEquality]
             {\validS{\setof{r} \specimp \spec{p}{c[e/x]}{a:A}{q}} \\
              \validP{r \implies e =_A e'} }
             {\judgeS{\setof{r} \specimp \spec{p}{c[e'/x]}{a:A}{q}}}

\and
  \inferrule*[right=AxConsequence]
             {\validS{\spec{p'}{c}{a:A}{q'}} \\ \validP{p \implies p'} \\ \validP{q' \implies q}}
             {\validS{\spec{p}{c}{a:A}{q}}}
  
\and

  \inferrule*[right=AxDisjunction]
             {\validS{\spec{p}{c}{a:A}{q}} \\
              \validS{\spec{p'}{c}{a:A}{q'}}}
             {\validS{\spec{p \vee p'}{c}{a:A}{q \vee q'}}}
\and

  \inferrule*[right=AxEmbedAssert]
             {\validP{p}}
             {\validS{\setof{p}}}

\and
\inferrule*[right=AxFrame]
          {\judgeS{S} \\ \judgeA{r}{\assert} }
          {\validS{S \specimp S \otimes r}}
\and
\mbox{(+ all of the axioms of intuitionistic logic)}
\end{mathpar}
\caption{Structural Axioms of the Program Logic}
\label{spec-logic-axioms-structural}
\end{figure}

\begin{figure}
\begin{mathpar}
  \inferrule*{\validS{S}}
             {\validP{\validprop{S}}}
\and
  \inferrule*{\judgeP{\Delta}{p}{\assert}}
             {\validP{\validprop{\setof{p}} \implies p}}
\and 
  \inferrule*{\judgeEqA{p}{q}{\omega}}
             {\validP{p =_\omega q}}
\\
  \mbox{(plus axioms of higher-order separation logic)}
\end{mathpar}
\caption{Axioms of Assertion Logic}
\label{assertion-logic-axioms}  
\end{figure}

\begin{figure}
\begin{mathpar}
\inferrule*[]
          {\judgeA{p}{\omega}}
          {\judgeEqA{p}{p}{\omega}}
\and
\inferrule*[]
          {\judgeEqA{p}{q}{\omega}}
          {\judgeEqA{q}{p}{\omega}}
\and
\inferrule*[]
          {\judgeEqA{p}{q}{\omega} \\ \judgeEqA{q}{r}{\omega}}
          {\judgeEqA{p}{r}{\omega}}
\and
\inferrule*[]
          {\judgeA{(\fun{x}{\omega}{p})\;q}{\omega'} \\
           \mbox{$q$ satisfies conditions of substitution theorem}}
          {\judgeEqA{(\fun{x}{\omega}{p})\;q}{[q/x]p}{\omega'}}
\and
\inferrule*[]
          {\judgeEqA[\Delta, x:\omega]{p\;x}{p'\;x}{\omega'}}
          {\judgeEqA{\fun{x}{\omega}{p}}{\fun{x}{\omega}{p'}}{\omega \To \omega'}}
\and 
\inferrule*[]
          {\judgeEqA{p}{p'}{\omega \To \omega'} \\
           \judgeEqA{q}{q'}{\omega}}
          {\judgeEqA{p\;q}{p'\;q'}{\omega'}}
\and
\inferrule*[]
          {x:\omega \in \Delta}
          {\judgeEqA{x}{x}{\omega}}
\and
\inferrule*[]
          {\judgeKeq[\restrictkind{\Delta}]{\tau}{\tau'}{\kappa}}
          {\judgeEqA{\tau}{\tau'}{\kappa}}
\and
\inferrule*[]
          {\judgeEq[\restrictkind{\Delta}]{\restricttype{\Delta}}{e}{e'}{A}}
          {\judgeEqA{e}{e'}{A}}
\and 
\inferrule*[]
          {\validP{p \iff q}}
          {\judgeEqA{p}{q}{\assert}}
\and
\inferrule*[]
          {\judgeEqA{p}{q}{\omega} \\ \judgeSortEq{\omega}{\omega'}}
          {\judgeEqA{p}{q}{\omega'}}
\end{mathpar}
\caption{Equality Judgement for Assertions}
\label{assertion-equality}  
\end{figure}


My program logic consists of three judgments:

\begin{enumerate}
\item $\validP{p}$, which asserts that a proposition $p$ is valid.

\item $\validS{S}$, which asserts that a specification $S$ is valid.

\item $\judgeEqA{p}{q}{\omega}$, which asserts that $p$ and $q$ are validly equal.
\end{enumerate}
%

\noindent The semantics of these three judgments is given as follows:
\begin{enumerate}
\item An assertion $\judgeP{\Delta}{p}{\assert}$ is valid, if and only if
for all $\delta \in \interp{\judgeACtx{\Delta}}$,
$\interp{\judgeP{\Delta}{p}{\assert}}\;\delta = \top$.

\item Similarly, a specification $\judgeS{S}$ is valid if and only
if, for all $\delta \in \interp{\judgeACtx{\Delta}}$,
$\interp{\judgeS{S}}\;\delta = \top$.

\item Finally, two assertion terms $\judgeA{p}{\omega}$ and
$\judgeA{q}{\omega}$ are validly equal, if and only if for all $\delta
\in \interp{\judgeACtx{\Delta}}$,
$\interp{\judgeP{\Delta}{p}{\assert}}\;\delta$ is equal to
$\interp{\judgeP{\Delta}{p}{\assert}}\;\delta$
\end{enumerate}

So this says that for any assignment of the free variables, a valid
assertion $p$ must be true, and similarly a valid specification $S$
must be true. Likewise, two syntactic terms $p$ and $q$ are validly
equal if their interpretations are equal under all environments. 

The program logic consists of a set of three mutually-recursive
judgments for deriving valid assertions, specifications, and
equalities. In Figure~\ref{spec-logic-axioms-basic}, I give the
primitive rules for deriving valid specifications of commands, and
Figure~\ref{spec-logic-axioms-structural}, I give a collection of
structural axioms.  In Figure~\ref{assertion-logic-axioms}, I give
some of the rules for deriving valid assertions, and in
Figure~\ref{assertion-equality}, I give rules for deriving equalities
between terms.

In Figure~\ref{spec-logic-axioms-basic}, the \textsc{AxEquiv1} and
\textsc{AxEquiv2} axioms assert the equivalence of Hoare triples over
commands and monadic expressions.  The \textsc{AxReturn} axiom asserts
that when we return a value $e$, we can strengthen the postcondition
with the assertion that the return value is $e$. The \textsc{AxAssign}
axiom asserts that if $e$ points to something in the precondition, the
command $e := e'$ ensures that in the postcondition $e \pointsto
e'$. The \textsc{AxAlloc} axiom asserts that allocating a new
reference $\newref{A}{e}$ will allocate a new pointer which is the
return value $a$, and that $a \pointsto e$ will be star'd onto the
postcondition. The \textsc{AxDeref} rule asserts that if $e \pointsto
e'$ in the precondition, then in the postcondition $e \pointsto e'$
will continue to hold, and that the return value $a = e'$.

The \textsc{AxBind} rule is the sequential composition rule of this
logic.  Because of the monadic nature of our programming language, we
can name and bind intermediate results, but of course there is a side-condition
to ensure that variables do not escape their scopes. 

Finally the \textsc{AxFix} rule is a fixed point induction rule for
this calculus. As in LCF, this rule looks like an induction without a
base case, which makes sense since this program logic is a partial
correctness calculus. This rule generalizes to all the pointed types,
but the syntactic overhead for stating the rules gets higher as the
types get larger.

In Figure~\ref{spec-logic-axioms-structural}, the \textsc{AxExtract}
rule takes a pure consequence of a precondition, and lifts it to
hypothetical assertion of the validity of an assertion. The intuition
for this rule is that to show the Hoare triple, we must assume $p$,
and so this means we might as well assume $r$ generally while proving
this triple. Conversely, The \textsc{AxEmbed} rule says that if we
have assumed the validity of an assertion, it does no harm to assume
it while proving a Hoare triple. The \textsc{AxUseValid} axiom states
that if we know that the assertion $\validprop{S}$ is valid, then we
may as well assume that $S$ is valid, and the \textsc{AxEmbedAssert}
says that valid assertions can be embedded in specifications. 

The \textsc{AxEquality} rule lets us rewrite programs using equational
reasoning, and the \textsc{AxForgetEx} axiom lets us drop existentials
from preconditions. The \textsc{AxConsequence} rule is just the rule
of consequence, and the \textsc{AxDisjunction} rule is just the rule
of disjunction, both familiar from Hoare logic. However, it is worth
pointing out that my language does \emph{not} validate the rule of
conjunction, as a consequence of having a language with a continuation
semantics.

Finally, the \textsc{AxFrame} schema gives the frame rule. My logic
supports the higher-order frame rule, and so we need to define a
syntactic frame operator $\Frame{S}{r}$ to act on arbitrary
specifications.
\begin{displaymath}
  \begin{array}{lcl}
    \Frame{\spec{p}{c}{a:A}{q}}{r}    & = & \spec{p * r}{c}{a:A}{q * r} \\
    \Frame{\mspec{p}{c}{a:A}{q}}{r}   & = & \mspec{p * r}{c}{a:A}{q * r} \\
    \Frame{\setof{p}}{r}              & = & \setof{p} \\
    \Frame{(S_1 \specand S_2)}{r}      & = & \Frame{S_1}{r} \specand \Frame{S_2}{r} \\
    \Frame{(S_1 \specor S_2)}{r}       & = & \Frame{S_1}{r} \specor \Frame{S_2}{r} \\
    \Frame{(S_1 \specimp S_2)}{r}      & = & \Frame{S_1}{r} \specimp \Frame{S_2}{r} \\
    \Frame{(\forall x:\omega.\; S)}{r} & = & \forall x:\omega.\; (\Frame{S}{r}) \\
    \Frame{(\exists x:\omega.\; S)}{r} & = & \exists x:\omega.\; (\Frame{S}{r}) \\
  \end{array}
\end{displaymath}
This is just the familiar action of the frame rule on Hoare triples. It
does nothing to the validity assertion $\setof{p}$, and distributes over all 
the other connectives. 

Framing is well-defined, and corresponds precisely to the semantic frame rule
we considered earlier in the chapter. 

\begin{prop}{(Syntactic Well-Formedness of Frame Operator)}
If $\judgeS{S}$ and $\judgeA{p}{\assert}$, then
we define $\judgeS{\Frame{S}{p}}$.  
\end{prop}
\begin{proof}
  By structural induction on specifications.
\end{proof}

More interesting than the syntactic well-formedness of the frame
operator is its semantic well-formedness. 

\begin{lemma}{(Syntactic Framing is Semantic Framing)}
If $\judgeS{S}$ and $\judgeA{r}{\assert}$, then for all $\delta \in \interp{\Delta}$, 
$\interp{\judgeS{\Frame{S}{r}}}\;\delta = 
\Frame{\interp{\judgeS{S}}\;\delta}{\interp{\judgeA{r}{\assert}}\;\delta}$ \\
\end{lemma}
\begin{proof}
  The proof is at the end of the chapter.
\end{proof}\\

The distinctive axioms of the assertion logic are far fewer. In
Figure~\ref{assertion-logic-axioms}, I give only three rules tailored
to this particular program logic. One for embedding specifications in
assertions, another for extracting assertions out of specifications,
and the last says that equality derivations entail equality
assertions. All the other rules are just the axioms of higher-order
separation logic.

Likewise, the valid equality judgment in
Figure~\ref{assertion-equality} is fairly minimal: it contains the
$\beta$ and $\eta$ rules for functions, as well as inheriting the
equality judgments for types and terms.

With all the buildup so far, it is easy to prove the following soundness
theorem:

\begin{theorem}{(Soundness of the Program Logic)}
  \begin{enumerate}
  \item If $\validP{p}$, then $\judgeA{p}{\assert}$ is valid.

  \item If $\validS{S}$, the $\judgeS{S}$ is valid.

  \item If $\judgeEqA{p}{q}{\omega}$, then $p$ and $q$ are validly equal.
  \end{enumerate}
\end{theorem}

\begin{proof}
  The proof of these three theorems follows from a mutual structural
  induction.  The soundness of each atomic rule for the two validity
  judgements is proven in the next section. The soundness of the rules
  for equality are a consequence of the equality rules we have already
  proven, and the cartesian closure of Set.  
\end{proof}

\section{Related Work}

There are two immediate predecessors to the proof system in this
dissertation. 

First, there is the work of Reynolds on specification logic for
Algol~\cite{spec-logic}. Idealized Algol~\cite{idealized-algol}
combines a call-by-name functional language, together with a type of
imperative computations whose operations correspond to the
while-language (i.e., it does not have higher-order store). This
stratification ensures that Algol features a powerful equational
theory which validates both the $\beta$- and $\eta$-rules of the
lambda-calculus. 

Specification logic extends Hoare logic~\cite{hoare-logic} to deal
with these new features, by viewing Hoare triples as the atomic
propositions of a first-order logic of specifications. Then, the user
of the logic can specify the behavior of a higher-order function using
an implications over triples, using the hypothesis of an implication
to specify the behavior of arguments of command type. In short,
specification logic can be viewed as a synthesis of LCF~\cite{lcf} with
  Hoare logic.

One of the motivations for the language design in my dissertation was
to see if the analogy between the computation type in Algol and the
monadic type discipline could be extended to a full higher-order
programming language, with features like higher-order store. This has
worked fantastically well. The semantics of the logic of this
dissertation is (suprisingly) \emph{simpler} than original
specification logic, even though the programming language itself is a
much more complex one than idealized Algol.

In particular, one of the main sources of complexity in usages of
specification logic have simply vanished from this logic. Algol has
assignable variables (like C or Java, though of course this reverses
the chronology), and so specification logic had to account for their
update with assertions about so-called ``good variable''
conditions. Since ML-like languages do not have assignable variables,
this means that all aliasing is confined to the heap. As a result, the
entire machinery of good variables is simply absent from my system.
So all aliasing and interference is confined to the heap, and
separation logic works quite well for specifying and reasoning about
this interference.

The other line of work I make use is on algebraic models of separation
logic. I give a very concrete model of separation logic in this
chapter, and am able to interpret higher-order quantification due to
the fact that the lattice of propositions is complete. This is
probably best understood as an instantiation of Biering~\emph{et
  al.}'s hyperdoctrine model of higher-order separation
logic. Likewise, my semantic domain for specifications makes use of
the techniques introduced by Birkedal and Yang~\cite{birkedal-yang} to
model higher-order frame rules.

Happily, though, most of this machinery stays ``under the hood'' to
ensure that the logic looks simple to the user.  As an example of this
phenomenon, we use TT-closure~\cite{tt-closure} to force the
admissibility of Hoare triples. This lets us give a simple rule for
fixed-point induction, without having to specify conditions on
admissible predicates (which is especially problematic in a setting
with predicate variables).

Finally, nested triples are very useful for specifying the properties
of higher-order programs: it is very useful to be able put the
specification of the contents of a reference into an assertion.  In
this chapter, I have given a ``cheap and cheerful'' embedding of
assertions in specifications and vice-versa, become true
specifications. This approach has the virtues of first, being very
technically easy to implement, and second, sufficient for all of the
examples I have considered.

However, more sophisticated approaches are certainly
possible. \citet{nested-hoare-triples} describe how to unify the
assertion and specification languages into a single logic.  This
offers the technical benefit that it extends the power of the
higher-order frame rule by letting it operate in a hereditary way, to
act on nested triples contained within assertions. In contrast, the
frame rule I give stops at the boundary of an assertion.

\subsection{Other Program Logics}

\subsubsection{Parkinson's Separation Logic for Java}

Parkinson developed a version of separation logic for Java in his
doctoral dissertation~\cite{parkinson-thesis}. His logic does not
feature a sophisticated specification logic; instead, he tries to use
the traditional notion of behavioral
subtyping~\cite{behavioral-subtyping} in order to deviate less
dramatically from the standard syntax of Hoare logic. However, strict
behavioral subtyping is very restrictive --- it essentially forces an
object-oriented program to ``look like'' a first-order program --- and
many object-oriented idioms (i.e., which correspond to higher-order
programming patterns) cannot be specified or proven correct in this 
framework 

Parkinson recovers flexibility through the use of a very clever
non-standard model of Hoare logic. The idea, described in Parkinson
and Bierman~\cite{parkinson-bierman-05}, is to admit a form of
second-order existential quantification, with a very unusual semantic
interpretation. Their ``abstract predicate families'' are predicates
defined on objects, whose definition varies according to the
\emph{concrete} class of the predicate. That is, their abstract
predicates ``dynamically dispatch'' to different definitions,
depending on the class of the object indexing the predicate. As a
result, they can re-use the proof rules of behavioral subtyping, even
though different classes can be given entirely different invariants.

While this is clearly a very natural idea for the specification of
object-oriented programs, I think that a richer language of
specifications and a more conventional interpretation of existentials
and universals is likely to be desirable in practice. This is because
modern OO languages like C\# and Scala also contain support for
first-class functions, which means that program logics should directly
support reasoning about higher-order programs.

\subsubsection{Hoare Type Theory}

Nanevski, Morisett and Birkedal have developed Hoare Type
Theory~\cite{htt, nanevski08}, which is an elegant and sophisticated
dependently-typed functional language that, like our system, uses a
monadic discipline to control effects.  Unlike our work, HTT takes
advantage of type dependency to directly integrate specifications into
the types of computation terms.

Nanevski, Ahmad, Morisett and Birkedal~\cite{abstract-htt} have
proposed using the existential quantification of their type theory to
hide data representations, giving examples such as a malloc/free style
memory allocator. \citet{nanevski-victor-10} also give
a proof of a modern congruence closure algorithm, illustrating that
this system can be used to verify some of the most complex imperative
algorithms known.

My sense is that this line of work represents the future of the design
of high-level imperative programming languages, but the particular
choices made in HTT's design have some shortcomings which make a
direct comparison with my own work somewhat more challenging than it
should be.

In particular, HTT uses an indexed family of monads to represent
computations.  That is, a computation has a type $e :
\spec{P}{-}{a:A}{Q(a)}$, where $P$ and $Q$ are predicates on the
heap. This means that higher-order imperative programs may need to
refer to terms of computation type in their pre- and post-conditions.
In the original formulation of HTT, which is predicative, this causes
size issues to arise which can make writing higher-order imperative
programs difficult, since any code mentioned in the indexes has to be
at a lower universe level than the code being verified.

This difficulty has been overcome by Petersen~\emph{et al.}, who give
an impredicative model of Hoare Type Theory. However, Svendsen
[personal communication] says that in an impredicative setting, the
weaker elimination rules for impredicative existentials means that
proofs involving higher-order imperative code become more difficult to
do. (This was discovered during an attempt to port the example program
in Chapter 6 to HTT.)

These issues simply do not arise in my system, since specifications
are kept strictly separate from types. This suggests that an
interesting future direction would be to study a version of Hoare Type
Theory in which there is an ordinary monadic type of computations,
together with a predicates on terms of monadic type which equip it
with pre- and post-conditions.

\subsubsection{Regional Logic and Ownership}

In addition to systems based on separation, there is also a line of
research based on the concept of object invariants and ownership.  The
Java modeling language (JML)~\cite{jml} and the Boogie
methodology~\cite{boogie} are two of the most prominent systems based
on this research stream. In Boogie, each object tracks its owner
object with a ghost field, and the ownership discipline enforces that
the heap have a tree structure. This allows the calculation of frame
properties without explosions due to aliasing, even though the
specification language remains ordinary first-order
logic. \citet{banerjee-naumann-regions} give a logic, which they name
``regional logic'', which formalizes these ideas in a small core logic
more tractable than industrial-strength systems like JML or Boogie.

One of the most interesting features of this line of work is that
\citet{banerjee-naumann-rep} were able to use an ownership discipline to
prove a representation independence (i.e., relational parametricity)
result for Java-like programs.  This is something that neither my
system, nor any of the other ones described earlier is capable of.
The equality relation I use is simply the equality inherited from the
(highly non-abstract) denotational semantics I give.



\section{Correctness Proofs}

\subsection{Substitution Theorems}

\begin{lemma*}{(Substitution for Terms and Specifications)}
\begin{enumerate}
\item Suppose $\judgeWK{\tau}{\kappa}$.
  \begin{enumerate}
  \item If $\judgeA[\Theta, \alpha:\kappa; \Gamma; \Delta]{p}{\omega}$, 
        then 
        \begin{enumerate}
        \item $\judgeA[\Theta; {[\tau/\alpha]}\Gamma; {[\tau/\alpha]}\Delta]{{[\tau/\alpha]}p}{{[\tau/\alpha]}\omega}$
        \item $\interp{\judgeA[\Theta; {[\tau/\alpha]}\Gamma; {[\tau/\alpha]}\Delta]{{[\tau/\alpha]}p}{{[\tau/\alpha]}\omega}}\;\theta\,\gamma\,\delta$ equals \\
          $\interp{\judgeA[\Theta, \alpha:\kappa; \Gamma, \Delta]{p}{\omega}}\;(\theta,\interp{\judgeWK{\tau}{\kappa}}\;\theta)\,\gamma\,\delta$
        \end{enumerate}
  \item If $\judgeS[\Theta, \alpha:\kappa; \Gamma; \Delta]{S}$, 
        then 
        \begin{enumerate}
        \item $\judgeS[\Theta; {[\tau/\alpha]}\Gamma; {[\tau/\alpha]}\Delta]{{[\tau/\alpha]}S}$
        \item $\interp{\judgeS[\Theta; {[\tau/\alpha]}\Gamma; {[\tau/\alpha]}\Delta]{{[\tau/\alpha]}S}}\;\theta$ equals \\
              $\interp{\judgeS[\Theta, \alpha:\kappa; \Gamma; \Delta]{S}}(\theta, \interp{\judgeWK{\tau}{\kappa}}\;\theta)$
        \end{enumerate}
  \end{enumerate}
\item Suppose $\judgeE{\Gamma}{e}{A}$.
  \begin{enumerate}
  \item If $\judgeA[\Theta; \Gamma, x:A; \Delta]{p}{\omega}$, 
        then 
        \begin{enumerate}
        \item $\judgeA[\Theta; \Gamma; \Delta]{[e/x]p}{\omega}$, 
        \item $\interp{\judgeA[\Theta; \Gamma; \Delta]{[e/x]p}{\omega}}\;\theta\,\gamma\,\delta$ equals \\
              $\interp{\judgeA[\Theta; \Gamma, x:A; \Delta]{p}{\omega}}\;\theta\,(\gamma, \interp{\judgeE{\Gamma}{e}{A}}\;\theta\;\gamma)\,\delta$               
        \end{enumerate}
  \item If $\judgeS[\Theta; \Gamma, x:A; \Delta]{S}$, 
        then 
        \begin{enumerate}
        \item $\judgeS[\Theta; \Gamma; \Delta]{[e/x]S}$.
        \item $\interp{\judgeS[\Theta; \Gamma; \Delta]{[e/x]S}}\;\theta\,\gamma\,\delta$ equals \\
              $\interp{\judgeS[\Theta; \Gamma, x:A; \Delta]{S}}\;\theta\,(\gamma, \interp{\judgeE{\Gamma}{e}{A}}\;\theta\;\gamma)\,\delta$
        \end{enumerate}
  \end{enumerate}
\item Suppose $\judgeA{q}{\upsilon}$.
  \begin{enumerate}
  \item If $\judgeA[\Theta; \Gamma; \Delta, u:\upsilon]{p}{\omega}$, 
        then 
        \begin{enumerate}
          \item $\judgeA[\Theta; \Gamma; \Delta]{[q/u]p}{\omega}$
          \item $\interp{\judgeA[\Theta; \Gamma; \Delta]{[q/u]p}{\omega}}\;\theta\,\gamma\,\delta$ equals \\
                $\interp{\judgeA[\Theta; \Gamma; \Delta, u:\upsilon]{p}{\omega}}\;\theta\,\gamma\,(\delta, \interp{\judgeA{q}{\upsilon}}\;\theta\,\gamma\,\delta)$
        \end{enumerate}
      \item If $\judgeS[\Theta; \Gamma:A; \Delta, u:\upsilon]{S}$, 
        then 
        \begin{enumerate}
          \item $\judgeS[\Theta; \Gamma; \Delta]{[q/u]S}$ 
          \item $\interp{\judgeS[\Theta; \Gamma; \Delta]{[q/u]S}}\;\theta\,\gamma\,\delta$ equals \\
                $\interp{\judgeS[\Theta; \Gamma:A; \Delta, u:\upsilon]{S}}\;\theta\,\gamma\,(\delta, \interp{\judgeA{q}{\upsilon}}\;\theta\,\gamma\,\delta)$
        \end{enumerate}
  \end{enumerate}

\end{enumerate}
\end{lemma*}
\begin{proof}
First, assume $\judgeWK{\tau}{\kappa}$, and $\judgeA[\Theta, \alpha:\kappa; \Gamma;\Delta]{p}{\omega}$, and
$\judgeS[\Theta, \alpha:\kappa; \Gamma; \Delta]{S}$. 

Now, we proceed by mutual induction on the derivation of $p$ and $S$: 
\begin{enumerate}

\item Case \textsc{TType}: $\judgeA[\Theta, \alpha:\kappa; \Gamma;\Delta]{\tau'}{\kappa'}$

  First, the syntax:
  \begin{tabbedproof}
    \oo By inversion, we know $\judgectx{\Theta, \alpha:\kappa}{\Delta}$ \\
    \oo By inversion, we know $\judgectx{\Theta, \alpha:\kappa}{\Gamma}$ \\
    \oo By inversion, we know $\judgeWK[\Theta, \alpha]{\tau'}{\kappa'}$ \\
    \oo By substitution, $\judgectx{\Theta}{{[\tau/\alpha]}\Delta}$ \\
    \oo By substitution, $\judgectx{\Theta}{{[\tau/\alpha]}\Gamma}$ \\
    \oo By substitution, $\judgeWK{{[\tau/\alpha]}\tau'}{\kappa'}$ \\
    \oo By rule, $\judgeA[\Theta; {[\tau/\alpha]}\Gamma;{[\tau/\alpha]}\Delta]{{[\tau/\alpha]}\tau'}{\kappa'}$
  \end{tabbedproof}

  For semantics,     $\interp{\judgeA[\Theta; {[\tau/\alpha]}\Gamma;{[\tau/\alpha]}\Delta]{{[\tau/\alpha]}\tau'}{\kappa'}}\;\theta\;\gamma\;\delta$
  \begin{eqnproof}
    \eline{\interp{\judgeWK[\Theta]{{[\tau/\alpha]}\tau'}{\kappa'}}\;\theta}
          {Semantics}
    \eline{\interp{\judgeWK[\Theta,\alpha:\kappa]{\tau'}{\kappa'}}\;(\theta, \interp{\judgeWK{\tau}{\kappa}}\;\theta)}
          {Substitution thm}
    \eline{\interp{\judgeA[\Theta, \alpha:\kappa;\Gamma;\Theta]{\tau'}{\kappa'}}\;(\theta, \interp{\judgeWK{\tau}{\kappa}}\;\theta)\;\gamma\;\delta}
          {Semantics}
  \end{eqnproof}
  The correctness of the application of $\gamma$ and $\delta$ follows from the equations for contexts
  under substitution. 

  \item Case \textsc{TExpr}: $\judgeA[\Theta, \alpha:\kappa; \Gamma; \Delta]{e}{A}$

    First, the syntax:
    \begin{tabbedproof}
      \oo By inversion, we know $\judgectx{\Theta, \alpha:\kappa}{\Delta}$ \\
      \oo By inversion, we know $\judgeE[\Theta, \alpha:\kappa]{\Gamma}{e}{A}$ \\
      \oo By substitution, we know $\judgectx{\Theta}{{[\tau/\alpha]}\Delta}$ \\
      \oo By substitution, we know $\judgeE{{[\tau/\alpha]}\Gamma}{{[\tau/\alpha]}e}{{[\tau/\alpha]}A}$ \\
      \oo By rule, we know $\judgeA[\Theta; {[\tau/\alpha]}\Gamma; {[\tau/\alpha]}\Delta]{{[\tau/\alpha]}e}{{[\tau/\alpha]}A}$
    \end{tabbedproof}

    For semantics, consider $\interp{\judgeA[\Theta; {[\tau/\alpha]}\Gamma; {[\tau/\alpha]}\Delta]{{[\tau/\alpha]}e}{{[\tau/\alpha]}A}}\;\theta\;\gamma\;\delta$
    \begin{eqnproof}
      \eline{\interp{\judgeE{{[\tau/\alpha]}\Gamma}{{[\tau/\alpha]}e}{{[\tau/\alpha]}A}}\;\theta\;\gamma}
            {Semantics}
      \eline{\interp{\judgeE{\Gamma}{e}{A}}\;(\theta, \interp{\judgeWK{\tau}{\kappa}}\;\theta)\;\gamma}
            {Substitution}
      \eline{\interp{\judgeA[\Theta, \alpha:\kappa; \Gamma; \Delta]{e}{A}}\;(\theta, \interp{\judgeWK{\tau}{\kappa}}\;\theta)\;\gamma\;\delta}
            {Semantics}
    \end{eqnproof}
  The correctness of the application of $\gamma$ and $\delta$ follows from the equations for contexts
  under substitution. 

\item Case \textsc{THyp}: $\judgeA[\Theta, \alpha:\kappa; \Gamma; \Delta]{u}{\upsilon}$

  First, the syntax:
  \begin{tabbedproof}
    \oo By inversion, we know $\judgectx{\Theta, \alpha:\kappa}{\Gamma}$\\
    \oo By inversion, we know $\judgectx{\Theta, \alpha:\kappa}{\Delta}$\\
    \oo By substitution, we know $\judgectx{\Theta}{{[\tau/\alpha]}\Gamma}$\\
    \oo By substitution, we know $\judgectx{\Theta}{{[\tau/\alpha]}\Delta}$\\
    \oo By rule, we know $\judgeA[\Theta; {[\tau/\alpha]}\Gamma; {[\tau/\alpha]}\Delta]{u}{{[\tau/\alpha]}A}$\\
  \end{tabbedproof}

  Next, consider $\interp{\judgeA[\Theta; {[\tau/\alpha]}\Gamma; {[\tau/\alpha]}\Delta]{u}{{[\tau/\alpha]}A}}\;\theta\;\gamma\;\delta$\\
  \begin{eqnproof}
    \eline{\pi_u(\delta)}{Semantics}
    \eline{\interp{\judgeA[\Theta, \alpha:\kappa; \Gamma; \Delta]{u}{A}}\;(\theta, \interp{\judgeWK{\tau}{\kappa}}\;\theta)\;\gamma\;\delta}
          {Semantics}
  \end{eqnproof}
  The correctness of the application of $\gamma$ and $\delta$ follows from the equations for contexts
  under substitution. 

\item Case \textsc{TAbs1}: $\judgeA[\Theta, \alpha:\kappa; \Gamma; \Delta]{\pfun{u}{\upsilon'}{p}}{\upsilon' \To \upsilon}$
  
  First, the syntax:
  \begin{tabbedproof}
    \oo By inversion, $\judgeA[\Theta, \alpha:\kappa; \Gamma; \Delta, u:\upsilon']{p}{\upsilon}$ \\
    \oo By induction, $\judgeA[\Theta; {[\tau/\alpha]}\Gamma; {[\tau/\alpha]}\Delta, u:{[\tau/\alpha]}\upsilon']{{[\tau/\alpha]}p}{{[\tau/\alpha]}\upsilon}$ \\
    \oo By rule, $\judgeA[\Theta; {[\tau/\alpha]}\Gamma; {[\tau/\alpha]}\Delta]{\pfun{u}{{[\tau/\alpha]}\upsilon'}{{[\tau/\alpha]}p}}{{[\tau/\alpha]}\upsilon' \To {[\tau/\alpha]}\upsilon}$ \\
    \oo By def of subst, $\judgeA[\Theta; {[\tau/\alpha]}\Gamma; {[\tau/\alpha]}\Delta]{{[\tau/\alpha]}(\pfun{u}{\upsilon'}{p})}{{[\tau/\alpha]}(\upsilon' \To \upsilon)}$ \\
  \end{tabbedproof}
  %
  For semantics, consider $\interp{\judgeA[\Theta; {[\tau/\alpha]}\Gamma; {[\tau/\alpha]}\Delta]{{[\tau/\alpha]}(\pfun{u}{\upsilon'}{p})}{{[\tau/\alpha]}(\upsilon' \To \upsilon)}}\;\theta\;\gamma\;\delta$ \\
  \begin{eqnproof}
    \eline{\semfun{v}{\interp{\judgeA[\Theta; {[\tau/\alpha]}\Gamma; {[\tau/\alpha]}\Delta, u:{[\tau/\alpha]}\upsilon']{{[\tau/\alpha]}p}{{[\tau/\alpha]}\upsilon}}\;\theta\;\gamma\;(\delta,v)}}
          {Semantics}
    \eline{\semfun{v}{\interp{\judgeA[\Theta, \alpha:\kappa; \Gamma; \Delta, u:\upsilon']{p}{\upsilon}}\;(\theta, \interp{\judgeWK{\tau}{\kappa}}\;\theta)\;\gamma\;(\delta,v)}}
          {Induction}
    \eline{\interp{\judgeA[\Theta, \alpha:\kappa; \Gamma; \Delta]{\pfun{u}{\upsilon'}{p}}{\upsilon' \To \upsilon}}\;(\theta, \interp{\judgeWK{\tau}{\kappa}}\;\theta)\;\gamma\;\delta}
          {Semantics}
  \end{eqnproof}
  The correctness of the application of $\gamma$ and $\delta$ follows from the equations for contexts
  under substitution. 

\item Case \textsc{TAbs2}: $\judgeA[\Theta, \alpha:\kappa; \Gamma; \Delta]{\pfun{x}{A}{p}}{A \To \upsilon}$
  
  First, the syntax:
  \begin{tabbedproof}
    \oo By inversion, $\judgeA[\Theta, \alpha:\kappa; \Gamma, x:A; \Delta]{p}{\upsilon}$ \\
    \oo By induction, $\judgeA[\Theta; {[\tau/\alpha]}\Gamma, x:{[\tau/\alpha]}A; {[\tau/\alpha]}\Delta]{{[\tau/\alpha]}p}{{[\tau/\alpha]}\upsilon}$ \\
    \oo By rule, $\judgeA[\Theta; {[\tau/\alpha]}\Gamma; {[\tau/\alpha]}\Delta]{\pfun{x}{{[\tau/\alpha]}A}{{[\tau/\alpha]}p}}{{[\tau/\alpha]}A \To {[\tau/\alpha]}\upsilon}$ \\
    \oo By def of subst, $\judgeA[\Theta; {[\tau/\alpha]}\Gamma; {[\tau/\alpha]}\Delta]{{[\tau/\alpha]}(\pfun{x}{A}{p})}{{[\tau/\alpha]}(A \To \upsilon)}$ \\
  \end{tabbedproof}
  %
  For semantics, consider $\interp{\judgeA[\Theta; {[\tau/\alpha]}\Gamma; {[\tau/\alpha]}\Delta]{{[\tau/\alpha]}(\pfun{x}{A}{p})}{{[\tau/\alpha]}(A \To \upsilon)}}\;\theta\;\gamma\;\delta$ \\
  \begin{eqnproof}
    \eline{\semfun{v}{\interp{\judgeA[\Theta; {[\tau/\alpha]}\Gamma, x:{[\tau/\alpha]}A; {[\tau/\alpha]}\Delta]{{[\tau/\alpha]}p}{{[\tau/\alpha]}\upsilon}}\;\theta\;(\gamma,v)\;\delta}}
          {Semantics}
    \eline{\semfun{v}{\interp{\judgeA[\Theta, \alpha:\kappa; \Gamma, x:A; \Delta]{p}{\upsilon}}\;(\theta, \interp{\judgeWK{\tau}{\kappa}}\;\theta)\;(\gamma,v)\;\delta}}
          {Induction}
    \eline{\interp{\judgeA[\Theta, \alpha:\kappa; \Gamma; \Delta]{\pfun{x}{A}{p}}{A \To \upsilon}}\;(\theta, \interp{\judgeWK{\tau}{\kappa}}\;\theta)\;\gamma\;\delta}
          {Semantics}
  \end{eqnproof}
  The correctness of the application of $\gamma$ and $\delta$ follows from the equations for contexts
  under substitution. 

\item Case \textsc{TAbs3}: $\judgeA[\Theta, \alpha:\kappa; \Gamma; \Delta]{\pfun{\beta}{\kappa'}{p}}{\kappa' \To \upsilon}$
  
  First, the syntax:
  \begin{tabbedproof}
    \oo By inversion, $\judgeA[\Theta, \alpha:\kappa, \beta:\kappa'; \Gamma; \Delta]{p}{\upsilon}$ \\
    \oo By induction, $\judgeA[\Theta, \beta:\kappa'; {[\tau/\alpha]}\Gamma; {[\tau/\alpha]}\Delta]{{[\tau/\alpha]}p}{{[\tau/\alpha]}\upsilon}$ \\
    \oo By rule, $\judgeA[\Theta; {[\tau/\alpha]}\Gamma; {[\tau/\alpha]}\Delta]{\pfun{\beta}{\kappa'}{{[\tau/\alpha]}p}}{\kappa' \To {[\tau/\alpha]}\upsilon}$ \\
    \oo By def of subst, $\judgeA[\Theta; {[\tau/\alpha]}\Gamma; {[\tau/\alpha]}\Delta]{{[\tau/\alpha]}(\pfun{\beta}{\kappa'}{p})}{{[\tau/\alpha]}(\kappa' \To \upsilon)}$ \\
  \end{tabbedproof}
  %
  For semantics, consider $\interp{\judgeA[\Theta; {[\tau/\alpha]}\Gamma; {[\tau/\alpha]}\Delta]{{[\tau/\alpha]}(\pfun{\beta}{\kappa'}{p})}{{[\tau/\alpha]}(\kappa' \To \upsilon)}}\;\theta\;\gamma\;\delta$ \\
  \begin{eqnproof}
    \eline{\semfun{\tau}{\interp{\judgeA[\Theta, \beta:\kappa'; {[\tau/\alpha]}\Gamma; {[\tau/\alpha]}\Delta]{{[\tau/\alpha]}p}{{[\tau/\alpha]}\upsilon}}\;(\theta,\tau)\;\gamma\;\delta}}
          {Semantics}
    \eline{\semfun{\tau}{\interp{\judgeA[\Theta, \alpha:\kappa, \beta:\kappa'; \Gamma; \Delta]{p}{\upsilon}}\;(\theta, \interp{\judgeWK{\tau}{\kappa}}\;\theta, \tau)\;\gamma\;\delta}}
          {Induction}
    \eline{\interp{\judgeA[\Theta, \alpha:\kappa; \Gamma; \Delta]{\pfun{\beta}{\kappa'}{p}}{\kappa' \To \upsilon}}\;(\theta, \interp{\judgeWK{\tau}{\kappa}}\;\theta)\;\gamma\;\delta}
          {Semantics}
  \end{eqnproof}
  The correctness of the application of $\gamma$ and $\delta$ follows from the equations for contexts
  under substitution. We also silently permuted the context at the second step, and made use of the fact
  that $\beta$ is not free in $\Gamma$ or $\Delta$.

\item Case \textsc{TAbsAll}: $\judgeA[\Theta, \alpha:\kappa; \Gamma; \Delta]{\pfunall{\beta}{\kappa'}{p}}{\forallsort{\beta}{\kappa'}{\upsilon}}$
  
  First, the syntax:
  \begin{tabbedproof}
    \oo By inversion, $\judgeA[\Theta, \alpha:\kappa, \beta:\kappa'; \Gamma; \Delta]{p}{\upsilon}$ \\
    \oo By induction, $\judgeA[\Theta, \beta:\kappa'; {[\tau/\alpha]}\Gamma; {[\tau/\alpha]}\Delta]{{[\tau/\alpha]}p}{{[\tau/\alpha]}\upsilon}$ \\
    \oo By rule, $\judgeA[\Theta; {[\tau/\alpha]}\Gamma; {[\tau/\alpha]}\Delta]{\pfunall{\beta}{\kappa'}{{[\tau/\alpha]}p}}{\forallsort{\beta}{\kappa'}{{[\tau/\alpha]}\upsilon}}$ \\
    \oo By def of subst, $\judgeA[\Theta; {[\tau/\alpha]}\Gamma; {[\tau/\alpha]}\Delta]{{[\tau/\alpha]}(\pfunall{\beta}{\kappa'}{p})}{{[\tau/\alpha]}\forallsort{\beta}{\kappa'}{\upsilon}}$ \\
  \end{tabbedproof}
  %
  For semantics, consider $\interp{\judgeA[\Theta; {[\tau/\alpha]}\Gamma; {[\tau/\alpha]}\Delta]{{[\tau/\alpha]}(\pfun{\beta}{\kappa'}{p})}{{[\tau/\alpha]}(\forallsort{\beta}{\kappa'}{\upsilon})}}\;\theta\;\gamma\;\delta$ \\
  \begin{eqnproof}
    \eline{\semfun{\tau}{\interp{\judgeA[\Theta, \beta:\kappa'; {[\tau/\alpha]}\Gamma; {[\tau/\alpha]}\Delta]{{[\tau/\alpha]}p}{{[\tau/\alpha]}\upsilon}}\;(\theta,\tau)\;\gamma\;\delta}}
          {Semantics}
    \eline{\semfun{\tau}{\interp{\judgeA[\Theta, \alpha:\kappa, \beta:\kappa'; \Gamma; \Delta]{p}{\upsilon}}\;(\theta, \interp{\judgeWK{\tau}{\kappa}}\;\theta, \tau)\;\gamma\;\delta}}
          {Induction}
    \eline{\interp{\judgeA[\Theta, \alpha:\kappa; \Gamma; \Delta]{\pfun{\beta}{\kappa'}{p}}{\forallsort{\beta}{\kappa'}{\upsilon}}}\;(\theta, \interp{\judgeWK{\tau}{\kappa}}\;\theta)\;\gamma\;\delta}
          {Semantics}
  \end{eqnproof}
  The correctness of the application of $\gamma$ and $\delta$ follows from the equations for contexts
  under substitution. We also silently permuted the context at the second step, and made use of the fact
  that $\beta$ is not free in $\Gamma$ or $\Delta$.

\item Case \textsc{TApp}: $\judgeA[\Theta, \alpha:\kappa; \Gamma; \Delta]{p\;q}{\upsilon}$
  
  First, the syntax:
  \begin{tabbedproof}
    \oo By inversion, $\judgeA[\Theta, \alpha:\kappa; \Gamma; \Delta]{p}{\omega \To \upsilon}$\\
    \oo By inversion, $\judgeA[\Theta, \alpha:\kappa; \Gamma; \Delta]{q}{\omega}$\\
    \oo By induction, $\judgeA[\Theta; {[\tau/\alpha]}\Gamma; {[\tau/\alpha]}\Delta]{{[\tau/\alpha]}p}{{[\tau/\alpha]}(\omega \To \upsilon)}$\\
    \oo By induction, $\judgeA[\Theta; {[\tau/\alpha]}\Gamma; {[\tau/\alpha]}\Delta]{{[\tau/\alpha]}q}{{[\tau/\alpha]}\omega}$\\
    \oo By rule, $\judgeA[\Theta; {[\tau/\alpha]}\Gamma; {[\tau/\alpha]}\Delta]{{[\tau/\alpha]}p\;{[\tau/\alpha]}q}{{[\tau/\alpha]}\omega \To {[\tau/\alpha]}\upsilon}$\\
    \oo By subst def, $\judgeA[\Theta; {[\tau/\alpha]}\Gamma; {[\tau/\alpha]}\Delta]{{[\tau/\alpha]}(p\;q)}{{[\tau/\alpha]}(\omega \To \upsilon)}$\\
  \end{tabbedproof}

  For semantics, consider $\interp{\judgeA[\Theta; {[\tau/\alpha]}\Gamma; {[\tau/\alpha]}\Delta]{{[\tau/\alpha]}(p\;q)}{{[\tau/\alpha]}\upsilon}}\;\theta\;\gamma\;\delta$
  \begin{eqnproof}
    \eline{\begin{array}{l}
             (\interp{\judgeA[\Theta; {[\tau/\alpha]}\Gamma; {[\tau/\alpha]}\Delta]{{[\tau/\alpha]}p}{{[\tau/\alpha]}(\omega \To \upsilon)}}\;\theta\;\gamma\;\delta) \\
             \;\;(\interp{\judgeA[\Theta; {[\tau/\alpha]}\Gamma; {[\tau/\alpha]}\Delta]{{[\tau/\alpha]}q}{{[\tau/\alpha]}\omega}}\;\theta\;\gamma\;\delta) 
      \end{array}}
    {Semantics}
    \eline{\begin{array}{l}
             (\interp{\judgeA[\Theta, \alpha:\kappa; \Gamma; \Delta]{p}{\omega \To \upsilon}}\;(\theta, \interp{\judgeWK{\tau}{\kappa}}\;\theta)\;\gamma\;\delta) \\
             \;\;(\interp{\judgeA[\Theta, \alpha:\kappa; \Gamma; \Delta]{q}{\omega}}\;(\theta, \interp{\judgeWK{\tau}{\kappa}}\;\theta)\;\gamma\;\delta) 
      \end{array}}
    {Induction}
    \eline{\interp{\judgeA[\Theta, \alpha:\kappa; \Gamma; \Delta]{p\;q}{\upsilon}}\;(\theta, \interp{\judgeWK{\tau}{\kappa}}\;\theta)\;\gamma\;\delta}
          {Semantics}
  \end{eqnproof}


\item Case \textsc{TAppAll}: $\judgeA[\Theta, \alpha:\kappa; \Gamma; \Delta]{p\;[\tau']}{[\tau'/\beta]\upsilon}$
  
  First, the syntax:
  \begin{tabbedproof}
    \oo By inversion, $\judgeA[\Theta, \alpha:\kappa; \Gamma; \Delta]{p}{\forallsort{\beta}{\kappa'}{\upsilon}}$\\
    \oo By inversion, $\judgeA[\Theta, \alpha:\kappa; \Gamma; \Delta]{\tau'}{\kappa'}$\\
    \oo By induction, $\judgeA[\Theta; {[\tau/\alpha]}\Gamma; {[\tau/\alpha]}\Delta]{{[\tau/\alpha]}p}{{[\tau/\alpha]}(\forallsort{\beta}{\kappa'}{\upsilon})}$\\
    \oo By induction, $\judgeA[\Theta; {[\tau/\alpha]}\Gamma; {[\tau/\alpha]}\Delta]{{[\tau/\alpha]}\tau'}{\kappa'}$\\
    \oo By rule, $\judgeA[\Theta; {[\tau/\alpha]}\Gamma; {[\tau/\alpha]}\Delta]{{[\tau/\alpha]}p\;{[\tau/\alpha]}[\tau']}{{[\tau/\alpha, [\tau/\alpha]\tau'/\beta]}\upsilon}$\\
    \oo By subst def, $\judgeA[\Theta; {[\tau/\alpha]}\Gamma; {[\tau/\alpha]}\Delta]{{[\tau/\alpha]}(p\;[\tau'])}{{[\tau/\alpha, [\tau/\alpha]\tau'/\beta]}\upsilon}$\\
  \end{tabbedproof}

  For semantics, consider $\interp{\judgeA[\Theta; {[\tau/\alpha]}\Gamma; {[\tau/\alpha]}\Delta]{{[\tau/\alpha]}(p\;[\tau'])}{{[\tau/\alpha]}\upsilon}}\;\theta\;\gamma\;\delta$
  \begin{eqnproof}
    \eline{\begin{array}{l}
             (\interp{\judgeA[\Theta; {[\tau/\alpha]}\Gamma; {[\tau/\alpha]}\Delta]{{[\tau/\alpha]}p}{{[\tau/\alpha]}(\forallsort{\beta}{\kappa'}{\upsilon})}}\;\theta\;\gamma\;\delta) \\
             \;\;(\interp{\judgeA[\Theta; {[\tau/\alpha]}\Gamma; {[\tau/\alpha]}\Delta]{{[\tau/\alpha]}\tau'}{\kappa'}}\;\theta\;\gamma\;\delta) 
      \end{array}}
    {Semantics}
    \eline{\begin{array}{l}
             (\interp{\judgeA[\Theta, \alpha:\kappa; \Gamma; \Delta]{p}{\forallsort{\beta}{\kappa'}{\upsilon}}}\;(\theta, \interp{\judgeWK{\tau}{\kappa}}\;\theta)\;\gamma\;\delta) \\
             \;\;(\interp{\judgeA[\Theta, \alpha:\kappa; \Gamma; \Delta]{\tau'}{\kappa'}}\;(\theta, \interp{\judgeWK{\tau}{\kappa}}\;\theta)\;\gamma\;\delta) 
      \end{array}}
    {Induction}
    \eline{\interp{\judgeA[\Theta, \alpha:\kappa; \Gamma; \Delta]{p\;[\tau']}{\upsilon}}\;(\theta, \interp{\judgeWK{\tau}{\kappa}}\;\theta)\;\gamma\;\delta}
          {Semantics}
  \end{eqnproof}


\item Case \textsc{TConst}:

  First, the syntax: 
  \begin{tabbedproof}
    \oo By inversion, $\judgectx{\Theta, \alpha:\kappa}{\Gamma}$ \\
    \oo By inversion, $\judgectx{\Theta, \alpha:\kappa}{\Delta}$ \\
    \oo By substitution, $\judgectx{\Theta}{[\tau/\alpha]\Gamma}$ \\
    \oo By substitution, $\judgectx{\Theta}{[\tau/\alpha]\Delta}$ \\
    \oo By rule, $\judgeA[\Theta; {[\tau/\alpha]}\Gamma; {[\tau/\alpha]}\Delta]{c}{\assert}$
  \end{tabbedproof}

  For semantics consider $\interp{\judgeA[\Theta; {[\tau/\alpha]}\Gamma; {[\tau/\alpha]}\Delta]{c}{\assert}}\;\theta;\gamma\;\delta$ 
  \begin{eqnproof}
    \eline{\interp{c}^0}{Semantics}
    \eline{\interp{\judgeA[\Theta, \alpha:\kappa; \Gamma; \Delta]{c}{\assert}}\;(\theta, \interp{\judgeWK{\tau}{\kappa}}\;\theta);\gamma\;\delta}
          {Semantics}
  \end{eqnproof}
  The correctness of the application of $\gamma$ and $\delta$ follows from the equations for contexts
  under substitution. 

\item Case \textsc{TBinary}: $\judgeA[\Theta, \alpha:\kappa; \Gamma; \Delta]{p \oplus q}{\assert}$
  
  First, the syntax:
  \begin{tabbedproof}
    \oo By inversion, $\judgeA[\Theta, \alpha:\kappa; \Gamma; \Delta]{p}{\assert}$\\
    \oo By inversion, $\judgeA[\Theta, \alpha:\kappa; \Gamma; \Delta]{q}{\assert}$\\
    \oo By induction, $\judgeA[\Theta; {[\tau/\alpha]}\Gamma; {[\tau/\alpha]}\Delta]{{[\tau/\alpha]}p}{\assert}$\\
    \oo By induction, $\judgeA[\Theta; {[\tau/\alpha]}\Gamma; {[\tau/\alpha]}\Delta]{{[\tau/\alpha]}q}{\assert}$\\
    \oo By rule, $\judgeA[\Theta; {[\tau/\alpha]}\Gamma; {[\tau/\alpha]}\Delta]{{[\tau/\alpha]}p \oplus {[\tau/\alpha]}q}{{[\tau/\alpha]}\assert}$\\
    \oo By subst def, $\judgeA[\Theta; {[\tau/\alpha]}\Gamma; {[\tau/\alpha]}\Delta]{{[\tau/\alpha]}(p \oplus q)}{\assert}$\\
  \end{tabbedproof}

  For semantics, consider $\interp{\judgeA[\Theta; {[\tau/\alpha]}\Gamma; {[\tau/\alpha]}\Delta]{{[\tau/\alpha]}(p \oplus q)}{\assert}}\;\theta\;\gamma\;\delta$
  \begin{eqnproof}
    \eline{\begin{array}{l}
             (\interp{\judgeA[\Theta; {[\tau/\alpha]}\Gamma; {[\tau/\alpha]}\Delta]{{[\tau/\alpha]}p}{\assert}}\;\theta\;\gamma\;\delta) \;\interp{\oplus}^2 \\
             \;\;(\interp{\judgeA[\Theta; {[\tau/\alpha]}\Gamma; {[\tau/\alpha]}\Delta]{{[\tau/\alpha]}q}{\assert}}\;\theta\;\gamma\;\delta) 
      \end{array}}
    {Semantics}
    \eline{\begin{array}{l}
             (\interp{\judgeA[\Theta, \alpha:\kappa; \Gamma; \Delta]{p}{\assert}}\;(\theta, \interp{\judgeWK{\tau}{\kappa}}\;\theta)\;\gamma\;\delta) \interp{\oplus}^2 \\
             \;\;(\interp{\judgeA[\Theta, \alpha:\kappa; \Gamma; \Delta]{q}{\assert}}\;(\theta, \interp{\judgeWK{\tau}{\kappa}}\;\theta)\;\gamma\;\delta) 
      \end{array}}
    {Induction}
    \eline{\interp{\judgeA[\Theta, \alpha:\kappa; \Gamma; \Delta]{p \oplus q}{\assert}}\;(\theta, \interp{\judgeWK{\tau}{\kappa}}\;\theta)\;\gamma\;\delta}
          {Semantics}
  \end{eqnproof}

\item Case \textsc{TQuantify1}: $\judgeA[\Theta, \alpha:\kappa; \Gamma; \Delta]{Q u:\upsilon.\;p}{\assert}$
  
  First, the syntax:
  \begin{tabbedproof}
    \oo By inversion, $\judgeA[\Theta, \alpha:\kappa; \Gamma; \Delta, u:\upsilon]{p}{\assert}$ \\
    \oo By induction, $\judgeA[\Theta; {[\tau/\alpha]}\Gamma; {[\tau/\alpha]}\Delta, u:{[\tau/\alpha]}\upsilon]{{[\tau/\alpha]}p}{\assert}$ \\
    \oo By rule, $\judgeA[\Theta; {[\tau/\alpha]}\Gamma; {[\tau/\alpha]}\Delta]
                         {Q u:{[\tau/\alpha]}\upsilon.\;{[\tau/\alpha]}p}{\assert}$ \\
    \oo By def of subst, $\judgeA[\Theta; {[\tau/\alpha]}\Gamma; {[\tau/\alpha]}\Delta]
                                 {{[\tau/\alpha]}(Q u:\upsilon.\;p)}{\assert}$ 
  \end{tabbedproof}

  For semantics, consider
  $\interp{\judgeA[\Theta; {[\tau/\alpha]}\Gamma; {[\tau/\alpha]}\Delta]
                  {{[\tau/\alpha]}(Q u:\upsilon.\;p)}{\assert}}\;\theta\;\gamma\;\delta$ 
  \begin{eqnproof}
    \eline{\begin{array}{l}
            \interp{Q}_{v \in \interp{\judgeSort{{[\tau/\alpha]}\upsilon}}\;\theta} \\
            \;\;\interp{\judgeA[\Theta;{[\tau/\alpha]}\Gamma;{[\tau/\alpha]}\Delta, u:{[\tau/\alpha]}\upsilon]{{[\tau/\alpha]}p}{\assert}}\;\theta\,\gamma\,(\delta,v)
           \end{array}}
          {Semantics}
    \eline{\begin{array}{l}
            \interp{Q}_{v \in \interp{\judgeSort[\Theta, \alpha:\kappa]{\upsilon}}\;(\theta, \interp{\judgeWK{\tau}{\kappa}}\;\theta)} \\
            \;\;\interp{\judgeA[\Theta, \alpha:\kappa;\Gamma;\Delta, u:\upsilon]{p}{\assert}}\;(\theta, \interp{\judgeWK{\tau}{\kappa}}\;\theta)\,\gamma\,(\delta,v)
           \end{array}}
          {Induction}
    \eline{\interp{\judgeA[\Theta, \alpha:\kappa;\Gamma;\Delta]{Q u:\upsilon.\;p}{\assert}}\;(\theta, \interp{\judgeWK{\tau}{\kappa}}\;\theta)\,\gamma\,\delta}
          {Semantics}
  \end{eqnproof}

  The correctness of the application of $\gamma$ and $\delta$ follows from the equations for
  contexts under substitution.


\item Case \textsc{TQuantify2}: $\judgeA[\Theta, \alpha:\kappa; \Gamma; \Delta]{Q x:A.\;p}{\assert}$
  
  First, the syntax:
  \begin{tabbedproof}
    \oo By inversion, $\judgeA[\Theta, \alpha:\kappa; \Gamma, x:A; \Delta]{p}{\assert}$ \\
    \oo By induction, $\judgeA[\Theta; {[\tau/\alpha]}\Gamma, x:{[\tau/\alpha]}A; {[\tau/\alpha]}\Delta]{{[\tau/\alpha]}p}{\assert}$ \\
    \oo By rule, $\judgeA[\Theta; {[\tau/\alpha]}\Gamma; {[\tau/\alpha]}\Delta]
                         {Q x:{[\tau/\alpha]}A.\;{[\tau/\alpha]}p}{\assert}$ \\
    \oo By def of subst, $\judgeA[\Theta; {[\tau/\alpha]}\Gamma; {[\tau/\alpha]}\Delta]
                                 {{[\tau/\alpha]}(Q x:A.\;p)}{\assert}$ 
  \end{tabbedproof}

  For semantics, consider
  $\interp{\judgeA[\Theta; {[\tau/\alpha]}\Gamma; {[\tau/\alpha]}\Delta]
                  {{[\tau/\alpha]}(Q x:A.\;p)}{\assert}}\;\theta\;\gamma\;\delta$ 
  \begin{eqnproof}
    \eline{\begin{array}{l}
            \interp{Q}_{v \in \interp{\judgeSort{{[\tau/\alpha]}A}}\;\theta} \\
            \;\;\interp{\judgeA[\Theta;{[\tau/\alpha]}\Gamma, x:{[\tau/\alpha]}A;{[\tau/\alpha]}\Delta]{{[\tau/\alpha]}p}{\assert}}\;\theta\,(\gamma,v)\,\delta
           \end{array}}
          {Semantics}
    \eline{\begin{array}{l}
            \interp{Q}_{v \in \interp{\judgeSort[\Theta, \alpha:\kappa]{A}}\;(\theta, \interp{\judgeWK{\tau}{\kappa}}\;\theta)} \\
            \;\;\interp{\judgeA[\Theta, \alpha:\kappa;\Gamma, x:A;\Delta]{p}{\assert}}\;(\theta, \interp{\judgeWK{\tau}{\kappa}}\;\theta)\,(\gamma,v)\,\delta
           \end{array}}
          {Induction}
    \eline{\interp{\judgeA[\Theta, \alpha:\kappa;\Gamma;\Delta]{Q x:A.\;p}{\assert}}\;(\theta, \interp{\judgeWK{\tau}{\kappa}}\;\theta)\,\gamma\,\delta}
          {Semantics}
  \end{eqnproof}

  The correctness of the application of $\gamma$ and $\delta$ follows from the equations for
  contexts under substitution.

\item Case \textsc{TQuantify3}: $\judgeA[\Theta, \alpha:\kappa; \Gamma; \Delta]{Q \beta:\kappa'.\;p}{\assert}$
  
  First, the syntax:
  \begin{tabbedproof}
    \oo By inversion, $\judgeA[\Theta, \alpha:\kappa, \beta:\kappa'; \Gamma; \Delta]{p}{\assert}$ \\
    \oo By induction, $\judgeA[\Theta, \beta:\kappa'; {[\tau/\alpha]}\Gamma; {[\tau/\alpha]}\Delta]{{[\tau/\alpha]}p}{\assert}$ \\
    \oo By rule, $\judgeA[\Theta; {[\tau/\alpha]}\Gamma; {[\tau/\alpha]}\Delta]
                         {Q \beta:\kappa'.\;{[\tau/\alpha]}p}{\assert}$ \\
    \oo By def of subst, $\judgeA[\Theta; {[\tau/\alpha]}\Gamma; {[\tau/\alpha]}\Delta]
                                 {{[\tau/\alpha]}(Q \beta:\kappa'.\;p)}{\assert}$ 
  \end{tabbedproof}

  For semantics, consider
  $\interp{\judgeA[\Theta; {[\tau/\alpha]}\Gamma; {[\tau/\alpha]}\Delta]
                  {{[\tau/\alpha]}(Q \beta:\kappa'.\;p)}{\assert}}\;\theta\;\gamma\;\delta$ 
  \begin{eqnproof}
    \eline{\begin{array}{l}
            \interp{Q}_{\tau' \in \interp{\judgeSort{\kappa'}}\;\theta} \\
            \;\;\interp{\judgeA[\Theta, \beta:\kappa';{[\tau/\alpha]}\Gamma;{[\tau/\alpha]}\Delta]{{[\tau/\alpha]}p}{\assert}}\;(\theta,\tau')\,\gamma\,\delta
           \end{array}}
          {Semantics}
    \eline{\begin{array}{l}
            \interp{Q}_{\tau' \in \interp{\judgeSort[\Theta, \alpha:\kappa]{\kappa'}}\;(\theta, \interp{\judgeWK{\tau}{\kappa}}\;\theta)} \\
            \;\;\interp{\judgeA[\Theta, \alpha:\kappa, \beta:\kappa';\Gamma;\Delta]{p}{\assert}}\;(\theta, \interp{\judgeWK{\tau}{\kappa}}\;\theta, \tau')\,\gamma\,(\delta,v)
           \end{array}}
          {Induction}
    \eline{\interp{\judgeA[\Theta;\Gamma;\Delta]{Q \beta:\kappa'.\;p}{\assert}}\;(\theta, \interp{\judgeWK{\tau}{\kappa}}\;\theta)\,\gamma\,\delta}
          {Semantics}
  \end{eqnproof}

  The correctness of the application of $\gamma$ and $\delta$ follows from the equations for
  contexts under substitution. We also silently permute the context in the second line.

\item Case \textsc{TEqual}: $\judgeA[\Theta, \alpha:\kappa; \Gamma; \Delta]{p =_\omega q}{\assert}$
  
  First, the syntax:
  \begin{tabbedproof}
    \oo By inversion, $\judgeA[\Theta, \alpha:\kappa; \Gamma; \Delta]{p}{\omega}$ \\
    \oo By inversion, $\judgeA[\Theta, \alpha:\kappa; \Gamma; \Delta]{q}{\omega}$ \\
    \oo By inversion, $\judgeSort[\Theta, \alpha:\kappa]{\omega}$ \\
    \oo By induction, $\judgeA[\Theta; {[\tau/\alpha]}\Gamma; {[\tau/\alpha]}\Delta]{{[\tau/\alpha]}p}{{[\tau/\alpha]}\omega}$ \\
    \oo By induction, $\judgeA[\Theta; {[\tau/\alpha]}\Gamma; {[\tau/\alpha]}\Delta]{{[\tau/\alpha]}q}{{[\tau/\alpha]}\omega}$ \\
    \oo By substitution, $\judgeSort[\Theta]{{[\tau/\alpha]}\omega}$ \\
    \oo By rule, $\judgeA[\Theta; {[\tau/\alpha]}\Gamma; {[\tau/\alpha]}\Delta]{{[\tau/\alpha]}(p =_\omega q)}{\assert}$
  \end{tabbedproof}

  For the semantics, consider $\interp{\judgeA[\Theta; {[\tau/\alpha]}\Gamma; {[\tau/\alpha]}\Delta]{{[\tau/\alpha]}(p =_\omega q)}{\assert}}\;\theta\;\gamma\;\delta$
  \begin{eqnproof}
    \eline{\mbox{if }\interp{{[\tau/\alpha]}p}\theta\;\gamma\;\delta = \interp{{[\tau/\alpha]}q}\theta\;\gamma\;\delta\mbox{ then }\top \mbox{ else } \bot}
          {Semantics}
    \eline{\mbox{if }\interp{p}(\theta, \interp{\judgeWK{\tau}{\kappa}}\theta)\;\gamma\;\delta = \interp{q}(\theta, \interp{\judgeWK{\tau}{\kappa}}\theta)\;\gamma\;\delta\mbox{ then }\top \mbox{ else } \bot}
          {Induction}
    \eline{\interp{\judgeA[\Theta, \alpha:\kappa; \Gamma; \Delta]{(p =_\omega q)}{\assert}}\;(\theta, \interp{\judgeWK{\tau}{\kappa}}\theta)\;\gamma\;\delta}
          {Semantics}
  \end{eqnproof}
  The correctness of the application of $\gamma$ and $\delta$ follows from the equations for
  contexts under substitution. We also need to use the equality of sorts under substitution to
  justify forming the equality in the third line. 

\item Case \textsc{TPointsto}: $\judgeA[\Theta, \alpha:\kappa; \Gamma; \Delta]{e \pointsto_A e'}{\assert}$

  First, the syntax:
  \begin{tabbedproof}
    \oo By inversion, $\judgeA[\Theta, \alpha:\kappa; \Gamma; \Delta]{e}{\reftype{A}}$ \\
    \oo By inversion, $\judgeA[\Theta, \alpha:\kappa; \Gamma; \Delta]{e'}{A}$ \\
    \oo By induction, $\judgeA[\Theta; {[\tau/\alpha]}\Gamma; {[\tau/\alpha]}\Delta]{{[\tau/\alpha]}e}{{[\tau/\alpha]}\reftype{A}}$ \\
    \oo By induction, $\judgeA[\Theta; {[\tau/\alpha]}\Gamma; {[\tau/\alpha]}\Delta]{{[\tau/\alpha]}e'}{{[\tau/\alpha]}A}$ \\
    \oo By rule, $\judgeA[\Theta; {[\tau/\alpha]}\Gamma; {[\tau/\alpha]}\Delta]{{[\tau/\alpha]}(e \pointsto_A e')}{\assert}$
  \end{tabbedproof}

  For the semantics, consider $\interp{\judgeA[\Theta; {[\tau/\alpha]}\Gamma; {[\tau/\alpha]}\Delta]{{[\tau/\alpha]}(e \pointsto_A e')}{\assert}}\;\theta\;\gamma\;\delta$
  \begin{eqnproof}
    \eline{\begin{array}{l}
           \interp{\judgeE[\Theta]{{[\tau/\alpha]}\Gamma}{{[\tau/\alpha]}e}{{[\tau/\alpha]}\reftype{A}}}\;\theta\;\gamma \\ 
           \pointsto \\
           \interp{\judgeE[\Theta]{{[\tau/\alpha]}\Gamma}{{[\tau/\alpha]}e'}{{[\tau/\alpha]}A}}\;\theta\;\gamma 
           \end{array}}
          {Semantics}
    \eline{\begin{array}{l}
           \interp{\judgeE[\Theta, \alpha:\kappa]{\Gamma}{e}{\reftype{A}}}\;(\theta, \interp{\judgeWK{\tau}{\kappa}}\;\theta)\;\gamma \\ 
           \pointsto \\
           \interp{\judgeE[\Theta, \alpha:\kappa]{\Gamma}{e'}{A}}\;(\theta, \interp{\judgeWK{\tau}{\kappa}}\;\theta)\;\gamma 
           \end{array}}
          {Induction}
    \eline{\interp{\judgeA[\Theta, \alpha:\kappa; \Gamma; \Delta]{e \pointsto_A e'}{\assert}}\;(\theta, \interp{\judgeWK{\tau}{\kappa}}\;\theta)\;\gamma }
          {Semantics}
  \end{eqnproof}
  The correctness of the application of $\gamma$ and $\delta$ follows from the equations for
  contexts under substitution. 

\item Case \textsc{TEqSort}: $\judgeA[\Theta, \alpha:\kappa; \Gamma; \Delta]{p}{\omega}$
  
  First, the syntax:
  \begin{tabbedproof}
    \oo By inversion, $\judgeSortEq[\Theta, \alpha:\kappa]{\omega}{\omega'}$ \\
    \oo By inversion, $\judgeA[\Theta, \alpha:\kappa; \Gamma; \Delta]{p}{\omega'}$ \\
    \oo By substitution, $\judgeSortEq{{[\tau/\alpha]}\omega}{{[\tau/\alpha]}\omega'}$ \\
    \oo By induction, $\judgeA[\Theta; {[\tau/\alpha]}\Gamma; {[\tau/\alpha]}\Delta]{{[\tau/\alpha]}p}{{[\tau/\alpha]}\omega'}$ \\
    \oo By rule, $\judgeA[\Theta; {[\tau/\alpha]}\Gamma; {[\tau/\alpha]}\Delta]{{[\tau/\alpha]}p}{{[\tau/\alpha]}\omega}$ \\
  \end{tabbedproof}

  For the semantics, consider $\interp{\judgeA[\Theta; {[\tau/\alpha]}\Gamma; {[\tau/\alpha]}\Delta]{{[\tau/\alpha]}p}{{[\tau/\alpha]}\omega}}\;\theta\;\gamma\;\delta$ \\
  \begin{eqnproof}
    \eline{\interp{\judgeA[\Theta; {[\tau/\alpha]}\Gamma; {[\tau/\alpha]}\Delta]{{[\tau/\alpha]}p}{{[\tau/\alpha]}\omega'}}\;\theta\;\gamma\;\delta}
          {Semantics}
    \eline{\interp{\judgeA[\Theta, \alpha:\kappa; \Gamma; \Delta]{p}{\omega'}}\;(\theta, \interp{\judgeWK{\tau}{\kappa}}\;\theta)\;\gamma\;\delta}
          {Induction}
    \eline{\interp{\judgeA[\Theta, \alpha:\kappa; \Gamma; \Delta]{p}{\omega}}\;(\theta, \interp{\judgeWK{\tau}{\kappa}}\;\theta)\;\gamma\;\delta}
          {Semantics}
  \end{eqnproof}
  The correctness of the application of $\gamma$ and $\delta$ follows from the equations for
  contexts under substitution. 

\item Case \textsc{TSpec}: $\judgeA[\Theta, \alpha:\kappa; \Gamma; \Delta]{\validprop{S}}{\assert}$:
  
  First, the syntax:
  \begin{tabbedproof}
    \oo By inversion, $\judgeS[\Theta, \alpha:\kappa; \Gamma; \Delta]{S}$ \\
    \oo By mutual induction $\judgeS[\Theta; {[\tau/\alpha]}\Gamma; {[\tau/\alpha]}\Delta]{{[\tau/\alpha]}S}$\\
    \oo By rule, $\judgeA[\Theta; {[\tau/\alpha]}\Gamma; {[\tau/\alpha]}\Delta]{\validprop{{[\tau/\alpha]}S}}{\assert}$
  \end{tabbedproof}

  For the semantics, consider $\interp{\judgeA[\Theta; {[\tau/\alpha]}\Gamma; {[\tau/\alpha]}\Delta]{\validprop{{[\tau/\alpha]}S}}{\assert}}\;\theta\;\gamma\;\delta$ 
  \begin{eqnproof}
    \eline{\mbox{if } \interp{{[\tau/\alpha]}S}\;\theta\;\gamma\;\delta = \top \mbox{ then } \top \mbox{ else } \bot}
          {Semantics}
    \eline{\mbox{if } \interp{S}\;(\theta, \interp{\judgeWK{\tau}{\kappa}}\;\theta)\;\gamma\;\delta = \top \mbox{ then } \top \mbox{ else } \bot}
          {Induction}
    \eline{\interp{\judgeA[\Theta, \alpha:\kappa; \Gamma; \Delta]{\validprop{S}}{\assert}}\; (\theta, \interp{\judgeWK{\tau}{\kappa}}\;\theta)\;\gamma\;\delta}
          {Semantics}
  \end{eqnproof}
  The correctness of the application of $\gamma$ and $\delta$ follows from the equations for
  contexts under substitution. 

\item Case \textsc{SpecTriple}: $\judgeS[\Theta, \alpha:\kappa; \Gamma; \Delta]{\spec{p}{c}{a:A}{q}}$
  
  First, the syntax:
  \begin{tabbedproof}
    \oo By inversion, $\judgeA[\Theta, \alpha:\kappa; \Gamma; \Delta]{p}{\assert}$ \\
    \oo By inversion, $\judgeA[\Theta, \alpha:\kappa; \Gamma; \Delta]{\comp{c}}{\monad{A}}$ \\
    \oo By inversion, $\judgeA[\Theta, \alpha:\kappa; \Gamma; \Delta, a:A]{q}{\assert}$ \\
    \oo By induction, $\judgeA[\Theta; {[\tau/\alpha]}\Gamma; {[\tau/\alpha]}\Delta]{{[\tau/\alpha]}p}{\assert}$ \\
    \oo By induction, $\judgeA[\Theta; {[\tau/\alpha]}\Gamma; {[\tau/\alpha]}\Delta]{{[\tau/\alpha]}\comp{c}}{{[\tau/\alpha]}\monad{A}}$ \\
    \oo By induction, $\judgeA[\Theta; {[\tau/\alpha]}\Gamma; {[\tau/\alpha]}\Delta, {[\tau/\alpha]}a:A]{{[\tau/\alpha]}q}{\assert}$ \\
    \oo By rule, $\judgeS[\Theta; {[\tau/\alpha]}\Gamma; {[\tau/\alpha]}\Delta]{{[\tau/\alpha]}(\spec{p}{c}{a:A}{q})}$
  \end{tabbedproof}

  For the semantics, consider $\interp{\judgeS[\Theta; {[\tau/\alpha]}\Gamma; {[\tau/\alpha]}\Delta]{{[\tau/\alpha]}(\spec{p}{c}{a:A}{q})}}\;\theta;\gamma\;\delta$
  \begin{eqnproof}
    \eline{\begin{array}{l}
           \{\interp{\judgeA[\Theta; {[\tau/\alpha]}\Gamma; {[\tau/\alpha]}\Delta]{p}{\assert}}\;\theta\;\gamma\;\delta\} \\
           \interp{\judgeC{{[\tau/\alpha]}\Gamma}{{[\tau/\alpha]}c}{A}}\;\theta\;\gamma\;\delta \\
           \{v.\;\interp{\judgeA[\Theta; {[\tau/\alpha]}\Gamma, a:A; {[\tau/\alpha]}\Delta]{q}{\assert}}\;\theta\;(\gamma,v)\;\delta\} 
           \end{array}}
          {Semantics}
    \eline{\begin{array}{l}
           \{\interp{\judgeA[\Theta, \alpha:\kappa \Gamma; \Delta]{p}{\assert}}\;(\theta, \interp{\judgeWK{\tau}{\kappa}}\;\theta)\;\gamma\;\delta\} \\
           \interp{\judgeC[\Theta, \alpha:\kappa]{\Gamma}{c}{A}}\;(\theta, \interp{\judgeWK{\tau}{\kappa}}\;\theta)\;\gamma\;\delta \\
           \{v.\;\interp{\judgeA[\Theta, \alpha:\kappa \Gamma, a:A; \Delta]{q}{\assert}}\;(\theta, \interp{\judgeWK{\tau}{\kappa}}\;\theta)\;(\gamma,v)\;\delta\} 
           \end{array}}
          {Induction}
    \eline{\interp{\judgeS[\Theta; \Gamma; \Delta]{(\spec{p}{c}{a:A}{q})}}\;(\theta, \interp{\judgeWK{\tau}{\kappa}}\;\theta);\gamma\;\delta}
          {Semantics}
  \end{eqnproof}
  The correctness of the application of $\gamma$ and $\delta$ follows from the equations for
  contexts under substitution. 

\item Case \textsc{SpecMTriple}: $\judgeS[\Theta, \alpha:\kappa; \Gamma; \Delta]{\mspec{p}{e}{a:A}{q}}$
  
  First, the syntax:
  \begin{tabbedproof}
    \oo By inversion, $\judgeA[\Theta, \alpha:\kappa; \Gamma; \Delta]{p}{\assert}$ \\
    \oo By inversion, $\judgeA[\Theta, \alpha:\kappa; \Gamma; \Delta]{e}{\monad{A}}$ \\
    \oo By inversion, $\judgeA[\Theta, \alpha:\kappa; \Gamma; \Delta, a:A]{q}{\assert}$ \\
    \oo By induction, $\judgeA[\Theta; {[\tau/\alpha]}\Gamma; {[\tau/\alpha]}\Delta]{{[\tau/\alpha]}p}{\assert}$ \\
    \oo By induction, $\judgeA[\Theta; {[\tau/\alpha]}\Gamma; {[\tau/\alpha]}\Delta]{{[\tau/\alpha]}e}{{[\tau/\alpha]}\monad{A}}$ \\
    \oo By induction, $\judgeA[\Theta; {[\tau/\alpha]}\Gamma; {[\tau/\alpha]}\Delta, {[\tau/\alpha]}a:A]{{[\tau/\alpha]}q}{\assert}$ \\
    \oo By rule, $\judgeS[\Theta; {[\tau/\alpha]}\Gamma; {[\tau/\alpha]}\Delta]{{[\tau/\alpha]}(\mspec{p}{e}{a:A}{q})}$
  \end{tabbedproof}

  For the semantics, consider $\interp{\judgeS[\Theta; {[\tau/\alpha]}\Gamma; {[\tau/\alpha]}\Delta]{{[\tau/\alpha]}(\mspec{p}{e}{a:A}{q})}}\;\theta;\gamma\;\delta$
  \begin{eqnproof}
    \eline{\begin{array}{l}
           \{\interp{\judgeA[\Theta; {[\tau/\alpha]}\Gamma; {[\tau/\alpha]}\Delta]{p}{\assert}}\;\theta\;\gamma\;\delta\} \\
           \interp{\judgeE{{[\tau/\alpha]}\Gamma}{{[\tau/\alpha]}e}{\monad{A}}}\;\theta\;\gamma\;\delta \\
           \{v.\;\interp{\judgeA[\Theta; {[\tau/\alpha]}\Gamma, a:A; {[\tau/\alpha]}\Delta]{q}{\assert}}\;\theta\;(\gamma,v)\;\delta\} 
           \end{array}}
          {Semantics}
    \eline{\begin{array}{l}
           \{\interp{\judgeA[\Theta, \alpha:\kappa \Gamma; \Delta]{p}{\assert}}\;(\theta, \interp{\judgeWK{\tau}{\kappa}}\;\theta)\;\gamma\;\delta\} \\
           \interp{\judgeE[\Theta, \alpha:\kappa]{\Gamma}{e}{\monad{A}}}\;(\theta, \interp{\judgeWK{\tau}{\kappa}}\;\theta)\;\gamma\;\delta \\
           \{v.\;\interp{\judgeA[\Theta, \alpha:\kappa \Gamma, a:A; \Delta]{q}{\assert}}\;(\theta, \interp{\judgeWK{\tau}{\kappa}}\;\theta)\;(\gamma,v)\;\delta\} 
           \end{array}}
          {Induction}
    \eline{\interp{\judgeS[\Theta; \Gamma; \Delta]{(\mspec{p}{e}{a:A}{q})}}\;(\theta, \interp{\judgeWK{\tau}{\kappa}}\;\theta);\gamma\;\delta}
          {Semantics}
  \end{eqnproof}
  The correctness of the application of $\gamma$ and $\delta$ follows from the equations for
  contexts under substitution. 

\item Case \textsc{SpecQuantify1}: $\judgeS[\Theta, \alpha:\kappa; \Gamma; \Delta]{Q u:\upsilon.\;S}$
  
  First, the syntax:
  \begin{tabbedproof}
    \oo By inversion, $\judgeS[\Theta, \alpha:\kappa; \Gamma; \Delta, u:\upsilon]{S}$ \\
    \oo By induction, $\judgeS[\Theta; {[\tau/\alpha]}\Gamma; {[\tau/\alpha]}\Delta, u:{[\tau/\alpha]}\upsilon]{{[\tau/\alpha]}S}$ \\
    \oo By rule, $\judgeS[\Theta; {[\tau/\alpha]}\Gamma; {[\tau/\alpha]}\Delta]
                         {Q u:{[\tau/\alpha]}\upsilon.\;{[\tau/\alpha]}S}$ \\
    \oo By def of subst, $\judgeS[\Theta; {[\tau/\alpha]}\Gamma; {[\tau/\alpha]}\Delta]
                                 {{[\tau/\alpha]}(Q u:\upsilon.\;S)}$ 
  \end{tabbedproof}

  For semantics, consider
  $\interp{\judgeS[\Theta; {[\tau/\alpha]}\Gamma; {[\tau/\alpha]}\Delta]
                  {{[\tau/\alpha]}(Q u:\upsilon.\;S)}}\;\theta\;\gamma\;\delta$ 
  \begin{eqnproof}
    \eline{\begin{array}{l}
            \interp{Q}_{v \in \interp{\judgeSort{{[\tau/\alpha]}\upsilon}}\;\theta} \\
            \;\;\interp{\judgeS[\Theta;{[\tau/\alpha]}\Gamma;{[\tau/\alpha]}\Delta, u:{[\tau/\alpha]}\upsilon]{{[\tau/\alpha]}S}}\;\theta\,\gamma\,(\delta,v)
           \end{array}}
          {Semantics}
    \eline{\begin{array}{l}
            \interp{Q}_{v \in \interp{\judgeSort[\Theta, \alpha:\kappa]{\upsilon}}\;(\theta, \interp{\judgeWK{\tau}{\kappa}}\;\theta)} \\
            \;\;\interp{\judgeS[\Theta, \alpha:\kappa;\Gamma;\Delta, u:\upsilon]{S}}\;(\theta, \interp{\judgeWK{\tau}{\kappa}}\;\theta)\,\gamma\,(\delta,v)
           \end{array}}
          {Induction}
    \eline{\interp{\judgeS[\Theta, \alpha:\kappa;\Gamma;\Delta]{Q u:\upsilon.\;S}}\;(\theta, \interp{\judgeWK{\tau}{\kappa}}\;\theta)\,\gamma\,\delta}
          {Semantics}
  \end{eqnproof}

  The correctness of the application of $\gamma$ and $\delta$ follows from the equations for
  contexts under substitution.


\item Case \textsc{SpecQuantify2}: $\judgeS[\Theta, \alpha:\kappa; \Gamma; \Delta]{Q x:A.\;S}$
  
  First, the syntax:
  \begin{tabbedproof}
    \oo By inversion, $\judgeS[\Theta, \alpha:\kappa; \Gamma, x:A; \Delta]{S}$ \\
    \oo By induction, $\judgeS[\Theta; {[\tau/\alpha]}\Gamma, x:{[\tau/\alpha]}A; {[\tau/\alpha]}\Delta]{{[\tau/\alpha]}S}$ \\
    \oo By rule, $\judgeS[\Theta; {[\tau/\alpha]}\Gamma; {[\tau/\alpha]}\Delta]
                         {Q x:{[\tau/\alpha]}A.\;{[\tau/\alpha]}S}$ \\
    \oo By def of subst, $\judgeS[\Theta; {[\tau/\alpha]}\Gamma; {[\tau/\alpha]}\Delta]
                                 {{[\tau/\alpha]}(Q x:A.\;S)}$ 
  \end{tabbedproof}

  For semantics, consider
  $\interp{\judgeS[\Theta; {[\tau/\alpha]}\Gamma; {[\tau/\alpha]}\Delta]
                  {{[\tau/\alpha]}(Q x:A.\;S)}}\;\theta\;\gamma\;\delta$ 
  \begin{eqnproof}
    \eline{\begin{array}{l}
            \interp{Q}_{v \in \interp{\judgeSort{{[\tau/\alpha]}A}}\;\theta} \\
            \;\;\interp{\judgeS[\Theta;{[\tau/\alpha]}\Gamma, x:{[\tau/\alpha]}A;{[\tau/\alpha]}\Delta]{{[\tau/\alpha]}S}}\;\theta\,(\gamma,v)\,\delta
           \end{array}}
          {Semantics}
    \eline{\begin{array}{l}
            \interp{Q}_{v \in \interp{\judgeSort[\Theta, \alpha:\kappa]{A}}\;(\theta, \interp{\judgeWK{\tau}{\kappa}}\;\theta)} \\
            \;\;\interp{\judgeS[\Theta, \alpha:\kappa;\Gamma, x:A;\Delta]{S}}\;(\theta, \interp{\judgeWK{\tau}{\kappa}}\;\theta)\,(\gamma,v)\,\delta
           \end{array}}
          {Induction}
    \eline{\interp{\judgeS[\Theta, \alpha:\kappa;\Gamma;\Delta]{Q x:A.\;S}}\;(\theta, \interp{\judgeWK{\tau}{\kappa}}\;\theta)\,\gamma\,\delta}
          {Semantics}
  \end{eqnproof}

  The correctness of the application of $\gamma$ and $\delta$ follows from the equations for
  contexts under substitution.

\item Case \textsc{SpecQuantify3}: $\judgeS[\Theta, \alpha:\kappa; \Gamma; \Delta]{Q \beta:\kappa'.\;S}$
  
  First, the syntax:
  \begin{tabbedproof}
    \oo By inversion, $\judgeS[\Theta, \alpha:\kappa, \beta:\kappa'; \Gamma; \Delta]{S}$ \\
    \oo By induction, $\judgeS[\Theta, \beta:\kappa'; {[\tau/\alpha]}\Gamma; {[\tau/\alpha]}\Delta]{{[\tau/\alpha]}S}$ \\
    \oo By rule, $\judgeS[\Theta; {[\tau/\alpha]}\Gamma; {[\tau/\alpha]}\Delta]
                         {Q \beta:\kappa'.\;{[\tau/\alpha]}S}$ \\
    \oo By def of subst, $\judgeS[\Theta; {[\tau/\alpha]}\Gamma; {[\tau/\alpha]}\Delta]
                                 {{[\tau/\alpha]}(Q \beta:\kappa'.\;S)}$ 
  \end{tabbedproof}

  For semantics, consider
  $\interp{\judgeS[\Theta; {[\tau/\alpha]}\Gamma; {[\tau/\alpha]}\Delta]
                  {{[\tau/\alpha]}(Q \beta:\kappa'.\;S)}}\;\theta\;\gamma\;\delta$ 
  \begin{eqnproof}
    \eline{\begin{array}{l}
            \interp{Q}_{\tau' \in \interp{\judgeSort{\kappa'}}\;\theta} \\
            \;\;\interp{\judgeS[\Theta, \beta:\kappa';{[\tau/\alpha]}\Gamma;{[\tau/\alpha]}\Delta]{{[\tau/\alpha]}S}}\;(\theta,\tau')\,\gamma\,\delta
           \end{array}}
          {Semantics}
    \eline{\begin{array}{l}
            \interp{Q}_{\tau' \in \interp{\judgeSort[\Theta, \alpha:\kappa]{\kappa'}}\;(\theta, \interp{\judgeWK{\tau}{\kappa}}\;\theta)} \\
            \;\;\interp{\judgeS[\Theta, \alpha:\kappa, \beta:\kappa';\Gamma;\Delta]{S}}\;(\theta, \interp{\judgeWK{\tau}{\kappa}}\;\theta, \tau')\,\gamma\,(\delta,v)
           \end{array}}
          {Induction}
    \eline{\interp{\judgeS[\Theta;\Gamma;\Delta]{Q \beta:\kappa'.\;S}}\;(\theta, \interp{\judgeWK{\tau}{\kappa}}\;\theta)\,\gamma\,\delta}
          {Semantics}
  \end{eqnproof}

  The correctness of the application of $\gamma$ and $\delta$ follows from the equations for
  contexts under substitution. We also silently permute the context in the second line.

\item Case \textsc{SpecBinary}: $\judgeS[\Theta, \alpha:\kappa; \Gamma; \Delta]{S \oplus S'}$
  
  First, the syntax:
  \begin{tabbedproof}
    \oo By inversion, $\judgeS[\Theta, \alpha:\kappa; \Gamma; \Delta]{S}$\\
    \oo By inversion, $\judgeS[\Theta, \alpha:\kappa; \Gamma; \Delta]{S'}$\\
    \oo By induction, $\judgeS[\Theta; {[\tau/\alpha]}\Gamma; {[\tau/\alpha]}\Delta]{{[\tau/\alpha]}S}$\\
    \oo By induction, $\judgeS[\Theta; {[\tau/\alpha]}\Gamma; {[\tau/\alpha]}\Delta]{{[\tau/\alpha]}S'}$\\
    \oo By rule, $\judgeS[\Theta; {[\tau/\alpha]}\Gamma; {[\tau/\alpha]}\Delta]{{[\tau/\alpha]}S \oplus {[\tau/\alpha]}S'}$\\
    \oo By subst def, $\judgeS[\Theta; {[\tau/\alpha]}\Gamma; {[\tau/\alpha]}\Delta]{{[\tau/\alpha]}(S \oplus S')}$\\
  \end{tabbedproof}

  For semantics, consider $\interp{\judgeS[\Theta; {[\tau/\alpha]}\Gamma; {[\tau/\alpha]}\Delta]{{[\tau/\alpha]}(S \oplus S')}}\;\theta\;\gamma\;\delta$
  \begin{eqnproof}
    \eline{\begin{array}{l}
             (\interp{\judgeS[\Theta; {[\tau/\alpha]}\Gamma; {[\tau/\alpha]}\Delta]{{[\tau/\alpha]}S}}\;\theta\;\gamma\;\delta) \;\interp{\oplus} \\
             \;\;(\interp{\judgeS[\Theta; {[\tau/\alpha]}\Gamma; {[\tau/\alpha]}\Delta]{{[\tau/\alpha]}S'}}\;\theta\;\gamma\;\delta) 
      \end{array}}
    {Semantics}
    \eline{\begin{array}{l}
             (\interp{\judgeS[\Theta, \alpha:\kappa; \Gamma; \Delta]{S}}\;(\theta, \interp{\judgeWK{\tau}{\kappa}}\;\theta)\;\gamma\;\delta) \interp{\oplus} \\
             \;\;(\interp{\judgeS[\Theta, \alpha:\kappa; \Gamma; \Delta]{S'}}\;(\theta, \interp{\judgeWK{\tau}{\kappa}}\;\theta)\;\gamma\;\delta) 
      \end{array}}
    {Induction}
    \eline{\interp{\judgeS[\Theta, \alpha:\kappa; \Gamma; \Delta]{S \oplus S'}}\;(\theta, \interp{\judgeWK{\tau}{\kappa}}\;\theta)\;\gamma\;\delta}
          {Semantics}
  \end{eqnproof}
  The correctness of the application of $\gamma$ and $\delta$ follows from the equations for
  contexts under substitution. 

\item Case \textsc{TSpec}: $\judgeS[\Theta, \alpha:\kappa; \Gamma; \Delta]{\setof{p}}$:
  
  First, the syntax:
  \begin{tabbedproof}
    \oo By inversion, $\judgeA[\Theta, \alpha:\kappa; \Gamma; \Delta]{p}{\assert}$ \\
    \oo By mutual induction $\judgeA[\Theta; {[\tau/\alpha]}\Gamma; {[\tau/\alpha]}\Delta]{{[\tau/\alpha]}p}{\assert}$\\
    \oo By rule, $\judgeS[\Theta; {[\tau/\alpha]}\Gamma; {[\tau/\alpha]}\Delta]{\setof{{[\tau/\alpha]}p}}$
  \end{tabbedproof}

  For the semantics, consider $\interp{\judgeS[\Theta; {[\tau/\alpha]}\Gamma; {[\tau/\alpha]}\Delta]{\setof{{[\tau/\alpha]}p}}}\;\theta\;\gamma\;\delta$ 
  \begin{eqnproof}
    \eline{\mbox{if } \interp{{[\tau/\alpha]}p}\;\theta\;\gamma\;\delta = \top \mbox{ then } \top \mbox{ else } \bot}
          {Semantics}
    \eline{\mbox{if } \interp{p}\;(\theta, \interp{\judgeWK{\tau}{\kappa}}\;\theta)\;\gamma\;\delta = \top \mbox{ then } \top \mbox{ else } \bot}
          {Induction}
    \eline{\interp{\judgeS[\Theta, \alpha:\kappa; \Gamma; \Delta]{\setof{p}}}\; (\theta, \interp{\judgeWK{\tau}{\kappa}}\;\theta)\;\gamma\;\delta}
          {Semantics}
  \end{eqnproof}
  The correctness of the application of $\gamma$ and $\delta$ follows from the equations for
  contexts under substitution. 
\end{enumerate}

\ \\

First, assume $\judgeE{\Gamma}{e''}{B}$, and $\judgeA[\Theta; \Gamma, y:B;\Delta]{p}{\omega}$, and
$\judgeS[\Theta; \Gamma, y:B; \Delta]{S}$. 

Now, we proceed by mutual induction on the derivation of $p$ and $S$: 
\begin{enumerate}

\item Case \textsc{TType}: $\judgeA[\Theta; \Gamma, y:B;\Delta]{\tau'}{\kappa'}$

  First, the syntax:
  \begin{tabbedproof}
    \oo By inversion, we know $\judgectx{\Theta}{\Delta}$ \\
    \oo By inversion, we know $\judgectx{\Theta}{\Gamma, y:B}$ \\
    \oo By inversion, we know $\judgeWK[\Theta]{\tau'}{\kappa'}$ \\
    \oo By inversion, we know $\judgectx{\Theta}{\Gamma}$ \\
    \oo By rule, $\judgeA[\Theta; \Gamma;\Delta]{\tau'}{\kappa'}$
  \end{tabbedproof}

  For semantics,  $\interp{\judgeA[\Theta; \Gamma; \Delta]{[e''/y]\tau'}{\kappa'}}\;\theta\;\gamma\;\delta$
  \begin{eqnproof}
    \eline{\interp{\judgeWK[\Theta]{\tau'}{\kappa'}}\;\theta}
          {Semantics}
    \eline{\interp{\judgeA[\Theta;\Gamma, y:B;\Theta]{\tau'}{\kappa'}}\;\theta\;(\gamma, \interp{\judgeE{\Gamma}{e''}{B}}\;\theta\;\gamma)\;\delta}
          {Semantics}
  \end{eqnproof}

  \item Case \textsc{TExpr}: $\judgeA[\Theta; \Gamma, y:B; \Delta]{e}{A}$

    First, the syntax:
    \begin{tabbedproof}
      \oo By inversion, we know $\judgectx{\Theta}{\Delta}$ \\
      \oo By inversion, we know $\judgeE{\Gamma, y:B}{e}{A}$ \\
      \oo By substitution, we know $\judgeE{\Gamma}{[e''/y]e}{A}$ \\
      \oo By rule, we know $\judgeA[\Theta; \Gamma; \Delta]{[e''/y]e}{A}$
    \end{tabbedproof}

    For semantics, consider $\interp{\judgeA[\Theta; \Gamma; \Delta]{[e''/y]e}{A}}\;\theta\;\gamma\;\delta$
    \begin{eqnproof}
      \eline{\interp{\judgeE{\Gamma}{[e''/y]e}{A}}\;\theta\;\gamma}
            {Semantics}
      \eline{\interp{\judgeE{\Gamma}{e}{A}}\;\theta\;(\gamma, \interp{\judgeE{\Gamma}{e''}{B}}\;\theta\;\gamma)\;\delta}
            {Substitution}
      \eline{\interp{\judgeA[\Theta; \Gamma, y:B; \Delta]{e}{A}}\;\theta\;(\gamma, \interp{\judgeE{\Gamma}{e''}{B}}\;\theta\;\gamma)\;\delta}
            {Semantics}
    \end{eqnproof}

\item Case \textsc{THyp}: $\judgeA[\Theta; \Gamma, y:B; \Delta]{u}{\upsilon}$

  First, the syntax:
  \begin{tabbedproof}
    \oo By inversion, we know $\judgectx{\Theta}{\Gamma, y:B}$\\
    \oo By inversion, we know $\judgectx{\Theta}{\Delta}$\\
    \oo By inversion, we know $\judgectx{\Theta}{\Gamma}$\\
    \oo By rule, we know $\judgeA[\Theta; \Gamma; \Delta]{u}{A}$\\
    \oo Hence, we know $\judgeA[\Theta; \Gamma; \Delta]{[e''/y]u}{A}$\\
  \end{tabbedproof}

  Next, consider $\interp{\judgeA[\Theta; \Delta; \Gamma]{u}{A}}\;\theta\;\gamma\;\delta$\\
  \begin{eqnproof}
    \eline{\pi_u(\delta)}{Semantics}
    \eline{\interp{\judgeA[\Theta; \Gamma, y:B; \Delta]{u}{A}}\;\theta\;(\gamma, \interp{\judgeE{\Gamma}{e''}{B}}\;\theta\;\gamma)\;\delta}
          {Semantics}
  \end{eqnproof}

\item Case \textsc{TAbs1}: $\judgeA[\Theta; \Gamma, y:B; \Delta]{\pfun{u}{\upsilon'}{p}}{\upsilon' \To \upsilon}$
  
  First, the syntax:
  \begin{tabbedproof}
    \oo By inversion, $\judgeA[\Theta; \Gamma, y:B; \Delta, u:\upsilon']{p}{\upsilon}$ \\
    \oo By induction, $\judgeA[\Theta; \Gamma; \Delta, u:\upsilon']{[e''/y]p}{\upsilon}$ \\
    \oo By rule, $\judgeA[\Theta; \Gamma; \Delta]{\pfun{u}{\upsilon'}{[e''/y]p}}{\upsilon' \To \upsilon}$ \\
    \oo By def of subst, $\judgeA[\Theta; \Gamma; \Delta]{[e''/y](\pfun{u}{\upsilon'}{p})}{\upsilon' \To \upsilon}$ \\
  \end{tabbedproof}
  %
  For semantics, consider $\interp{\judgeA[\Theta; \Gamma; \Delta]{(\pfun{u}{\upsilon'}{[e''/y]p})}{\upsilon' \To \upsilon}}\;\theta\;\gamma\;\delta$ \\
  \begin{eqnproof}
    \eline{\semfun{v}{\interp{\judgeA[\Theta; \Gamma; \Delta, u:\upsilon']{[e''/y]p}{\upsilon}}\;\theta\;\gamma\;(\delta,v)}}
          {Semantics}
    \eline{\semfun{v}{\interp{\judgeA[\Theta; \Gamma, y:B; \Delta, u:\upsilon']{p}{\upsilon}}\;\theta\;(\gamma, \interp{\judgeE{\Gamma}{e''}{B}}\;\theta\;\gamma)\;(\delta,v)}}
          {Induction}
    \eline{\interp{\judgeA[\Theta; \Gamma, y:B; \Delta]{\pfun{u}{\upsilon'}{p}}{\upsilon' \To \upsilon}}\;\theta\;(\gamma, \interp{\judgeE{\Gamma}{e''}{B}}\;\theta\;\gamma)\;\delta}
          {Semantics}
  \end{eqnproof}

\item Case \textsc{TAbs2}: $\judgeA[\Theta; \Gamma, y:B; \Delta]{\pfun{x}{A}{p}}{A \To \upsilon}$
  
  First, the syntax:
  \begin{tabbedproof}
    \oo By inversion, $\judgeA[\Theta; \Gamma, y:B, x:A; \Delta]{p}{\upsilon}$ \\
    \oo By induction, $\judgeA[\Theta; \Gamma, x:A; \Delta]{[e''/y]p}{\upsilon}$ \\
    \oo By rule, $\judgeA[\Theta; \Gamma; \Delta]{\pfun{x}{A}{[e''/y]p}}{A \To \upsilon}$ \\
    \oo By def of subst, $\judgeA[\Theta; \Gamma; \Delta]{[e''/y](\pfun{x}{A}{p})}{(A \To \upsilon)}$ \\
  \end{tabbedproof}
  %
  For semantics, consider $\interp{\judgeA[\Theta; \Gamma; \Delta]{[e''/y](\pfun{x}{A}{p})}{(A \To \upsilon)}}\;\theta\;\gamma\;\delta$ \\
  \begin{eqnproof}
    \eline{\semfun{v}{\interp{\judgeA[\Theta; \Gamma, x:A; \Delta]{[e''/y]p}{\upsilon}}\;\theta\;(\gamma,v)\;\delta}}
          {Semantics}
    \eline{\semfun{v}{\interp{\judgeA[\Theta; \Gamma, y:B, x:A; \Delta]{p}{\upsilon}}\;\theta\;(\gamma, \interp{\judgeE{\Gamma}{e''}{B}}\;\theta\;\gamma, v)\;\delta}}
          {Induction}
    \eline{\interp{\judgeA[\Theta; \Gamma, y:B; \Delta]{\pfun{x}{A}{p}}{A \To \upsilon}}\;\theta\;(\gamma, \interp{\judgeE{\Gamma}{e''}{B}}\;\theta\;\gamma)\;\delta}
          {Semantics}
  \end{eqnproof}
  We silently permute arguments in the second line. 

\item Case \textsc{TAbs3}: $\judgeA[\Theta; \Gamma, y:B; \Delta]{\pfun{\beta}{\kappa'}{p}}{\kappa' \To \upsilon}$
  
  First, the syntax:
  \begin{tabbedproof}
    \oo By inversion, $\judgeA[\Theta, \beta:\kappa'; \Gamma, y:B; \Delta]{p}{\upsilon}$ \\
    \oo By weakening, $\judgeE[\Theta, \beta:\kappa']{\Gamma}{e''}{B}$ \\
    \oo By induction, $\judgeA[\Theta, \beta:\kappa'; \Gamma; \Delta]{{[e''/y]}p}{\upsilon}$ \\
    \oo By rule, $\judgeA[\Theta; \Gamma; \Delta]{\pfun{\beta}{\kappa'}{{[e''/y]}p}}{\kappa' \To \upsilon}$ \\
    \oo By def of subst, $\judgeA{{[e''/y]}(\pfun{\beta}{\kappa'}{p})}{\kappa' \To \upsilon}$ \\
  \end{tabbedproof}
  %
  For semantics, consider $\interp{\judgeA{[e''/y](\pfun{\beta}{\kappa'}{p})}{\kappa' \To \upsilon}}\;\theta\;\gamma\;\delta$ \\
  \begin{eqnproof}
    \eline{\semfun{\tau}{\interp{\judgeA[\Theta, \beta:\kappa'; \Gamma; \Delta]{[e''/y]p}{\upsilon}}\;(\theta,\tau)\;\gamma\;\delta}}
          {Semantics}
    \eline{\semfun{\tau}{\interp{\judgeA[\Theta, \beta:\kappa'; \Gamma, y:B; \Delta]{p}{\upsilon}}\;(\theta, \tau)\;(\gamma, \interp{\judgeE[\Theta, \beta:\kappa']{\Gamma}{e''}{B}}\;(\theta, \tau)\;\gamma)\;\delta}}
          {Induction}
    \eline{\semfun{\tau}{\interp{\judgeA[\Theta, \beta:\kappa'; \Gamma, y:B; \Delta]{p}{\upsilon}}\;(\theta, \tau)\;(\gamma, \interp{\judgeE{\Gamma}{e''}{B}}\;\theta\;\gamma)\;\delta}}
          {Strengthening}
    \eline{\interp{\judgeA[\Theta; \Gamma, y:B; \Delta]{\pfun{\beta}{\kappa'}{p}}{\kappa' \To \upsilon}}\;\theta\;(\gamma, \interp{\judgeE{\Gamma}{e''}{B}}\;\theta\;\gamma)\;\delta}
          {Semantics}
  \end{eqnproof}
  This case relies upon the fact that $\Gamma$ and $\Delta$ do not have $\beta$ free and the 
  equality of sorts under substitution. 

\item Case \textsc{TAbsAll}: $\judgeA[\Theta; \Gamma, y:B; \Delta]{\pfunall{\beta}{\kappa'}{p}}{\forallsort{\beta}{\kappa'}{\upsilon}}$
  
  First, the syntax:
  \begin{tabbedproof}
    \oo By inversion, $\judgeA[\Theta, \beta:\kappa'; \Gamma, y:B; \Delta]{p}{\upsilon}$ \\
    \oo By induction, $\judgeA[\Theta, \beta:\kappa'; \Gamma; \Delta]{[e''/y]p}{\upsilon}$ \\
    \oo By rule, $\judgeA{\pfunall{\beta}{\kappa'}{[e''/y]p}}{\forallsort{\beta}{\kappa'}{\upsilon}}$ \\
    \oo By def of subst, $\judgeA{[e''/y](\pfunall{\beta}{\kappa'}{p})}{\forallsort{\beta}{\kappa'}{\upsilon}}$ \\
  \end{tabbedproof}
  %
  For semantics, consider $\interp{\judgeA{[e''/y](\pfun{\beta}{\kappa'}{p})}{\forallsort{\beta}{\kappa'}{\upsilon}}}\;\theta\;\gamma\;\delta$ \\
  \begin{eqnproof}
    \eline{\semfun{\tau}{\interp{\judgeA[\Theta, \beta:\kappa'; \Gamma; \Delta]{[e''/y]p}{\upsilon}}\;(\theta,\tau)\;\gamma\;\delta}}
          {Semantics}
    \eline{\semfun{\tau}{\interp{\judgeA[\Theta, \beta:\kappa'; \Gamma, y:B; \Delta, \beta:\kappa']{p}{\upsilon}}\;(\theta, \tau)\;(\gamma, \interp{\judgeE[\Theta, \beta:\kappa']{\Gamma}{e''}{B}}\;(\theta, \tau)\;\gamma)\;\delta}}
          {Induction}
    \eline{\semfun{\tau}{\interp{\judgeA[\Theta, \beta:\kappa'; \Gamma, y:B; \Delta, \beta:\kappa']{p}{\upsilon}}\;(\theta, \tau)\;(\gamma, \interp{\judgeE{\Gamma}{e''}{B}}\;\theta\;\gamma)\;\delta}}
          {Strengthening}
    \eline{\interp{\judgeA[\Theta; \Gamma, y:B; \Delta]{\pfun{\beta}{\kappa'}{p}}{\forallsort{\beta}{\kappa'}{\upsilon}}}\;\theta\;(\gamma, \interp{\judgeE{\Gamma}{e''}{B}}\;\theta\;\gamma)\;\delta}
          {Semantics}
  \end{eqnproof}
  This case relies upon the fact that $\Gamma$ and $\Delta$ do not have $\beta$ free and the 
  equality of sorts under substitution. 



\item Case \textsc{TApp}: $\judgeA[\Theta; \Gamma, y:B; \Delta]{p\;q}{\upsilon}$
  
  First, the syntax:
  \begin{tabbedproof}
    \oo By inversion, $\judgeA[\Theta; \Gamma, y:B; \Delta]{p}{\omega \To \upsilon}$\\
    \oo By inversion, $\judgeA[\Theta; \Gamma, y:B; \Delta]{q}{\omega}$\\
    \oo By induction, $\judgeA[\Theta; \Gamma; \Delta]{[e''/y]p}{(\omega \To \upsilon)}$\\
    \oo By induction, $\judgeA[\Theta; \Gamma; \Delta]{[e''/y]q}{\omega}$\\
    \oo By rule, $\judgeA[\Theta; \Gamma; \Delta]{[e''/y](p\;q)}{\omega \To \upsilon}$\\
  \end{tabbedproof}

  For semantics, consider $\interp{\judgeA[\Theta; \Gamma; \Delta]{[e''/y](p\;q)}{\upsilon}}\;\theta\;\gamma\;\delta$
  \begin{eqnproof}
    \eline{\begin{array}{l}
             (\interp{\judgeA[\Theta; \Gamma; \Delta]{[e''/y]p}{(\omega \To \upsilon)}}\;\theta\;\gamma\;\delta) \\
             \;\;(\interp{\judgeA[\Theta; \Gamma; \Delta]{[e''/y]q}{\omega}}\;\theta\;\gamma\;\delta) 
      \end{array}}
    {Semantics}
    \eline{\begin{array}{l}
             (\interp{\judgeA[\Theta; \Gamma, y:B; \Delta]{p}{\omega \To \upsilon}}\;\theta\;\gamma\;\delta) \\
             \;\;(\interp{\judgeA[\Theta; \Gamma, y:B; \Delta]{q}{\omega}}\;\theta\;(\gamma, \interp{\judgeE{\Gamma}{e''}{B}}\;\theta\;\gamma)\;\delta) 
      \end{array}}
    {Induction}
    \eline{\interp{\judgeA[\Theta; \Gamma, y:B; \Delta]{p\;q}{\upsilon}}\;\theta\;(\gamma, \interp{\judgeE{\Gamma}{e''}{B}}\;\theta\;\gamma)\;\delta}
          {Semantics}
  \end{eqnproof}


\item Case \textsc{TAppAll}: $\judgeA[\Theta; \Gamma, y:B; \Delta]{p\;[\tau']}{[\tau'/\beta]\upsilon}$
  
  First, the syntax:
  \begin{tabbedproof}
    \oo By inversion, $\judgeA[\Theta; \Gamma, y:B; \Delta]{p}{\forallsort{\beta}{\kappa'}{\upsilon}}$\\
    \oo By inversion, $\judgeA[\Theta; \Gamma, y:B; \Delta]{\tau'}{\kappa'}$\\
    \oo By induction, $\judgeA[\Theta; \Gamma; \Delta]{[e''/y]p}{(\forallsort{\beta}{\kappa'}{\upsilon})}$\\
    \oo By induction, $\judgeA[\Theta; \Gamma; \Delta]{\tau'}{\kappa'}$\\
    \oo By rule, $\judgeA[\Theta; \Gamma; \Delta]{[e''/y](p\;[\tau'])}{{[\tau/\alpha, [\tau/\alpha]\tau'/\beta]}\upsilon}$\\
  \end{tabbedproof}

  For semantics, consider $\interp{\judgeA[\Theta; \Gamma; \Delta]{[e''/y](p\;[\tau'])}{\upsilon}}\;\theta\;\gamma\;\delta$
  \begin{eqnproof}
    \eline{\begin{array}{l}
             (\interp{\judgeA[\Theta; \Gamma; \Delta]{[e''/y]p}{(\forallsort{\beta}{\kappa'}{\upsilon})}}\;\theta\;\gamma\;\delta) \\
             \;\;(\interp{\judgeA[\Theta; \Gamma; \Delta]{\tau'}{\kappa'}}\;\theta\;\gamma\;\delta) 
      \end{array}}
    {Semantics}
    \eline{\begin{array}{l}
             (\interp{\judgeA[\Theta; \Gamma, y:B; \Delta]{p}{\forallsort{\beta}{\kappa'}{\upsilon}}}\;\theta\;(\gamma, \interp{\judgeE{\Gamma}{e''}{B}}\;\theta\;\gamma)\;\delta) \\
             \;\;(\interp{\judgeA[\Theta; \Gamma, y:B; \Delta]{\tau'}{\kappa'}}\;\theta\;(\gamma, \interp{\judgeE{\Gamma}{e''}{B}}\;\theta\;\gamma)\;\delta) 
      \end{array}}
    {Induction}
    \eline{\interp{\judgeA[\Theta; \Gamma, y:B; \Delta]{p\;[\tau']}{\upsilon}}\;\theta\;(\gamma, \interp{\judgeE{\Gamma}{e''}{B}}\;\theta\;\gamma)\;\delta}
          {Semantics}
  \end{eqnproof}


\item Case \textsc{TConst}:

  First, the syntax: 
  \begin{tabbedproof}
    \oo By inversion, $\judgectx{\Theta}{\Gamma, y:B}$ \\
    \oo By inversion, $\judgectx{\Theta}{\Delta}$ \\
    \oo By inversion, $\judgectx{\Theta}{\Gamma}$ \\
    \oo By rule, $\judgeA[\Theta; \Gamma; \Delta]{c}{\assert}$
  \end{tabbedproof}

  For semantics consider $\interp{\judgeA[\Theta; \Gamma; \Delta]{[e''/y]c}{\assert}}\;\theta;\gamma\;\delta$ 
  \begin{eqnproof}
    \eline{\interp{c}^0}{Semantics}
    \eline{\interp{\judgeA[\Theta; \Gamma, y:B; \Delta]{c}{\assert}}\;\theta;(\gamma, \interp{\judgeE{\Gamma}{e''}{B}}\;\theta\;\gamma)\;\delta}
          {Semantics}
  \end{eqnproof}

\item Case \textsc{TBinary}: $\judgeA[\Theta; \Gamma, y:B; \Delta]{p \oplus q}{\assert}$
  
  First, the syntax:
  \begin{tabbedproof}
    \oo By inversion, $\judgeA[\Theta; \Gamma, y:B; \Delta]{p}{\assert}$\\
    \oo By inversion, $\judgeA[\Theta; \Gamma, y:B; \Delta]{q}{\assert}$\\
    \oo By induction, $\judgeA[\Theta; \Gamma; \Delta]{[e''/y]p}{\assert}$\\
    \oo By induction, $\judgeA[\Theta; \Gamma; \Delta]{[e''/y]q}{\assert}$\\
    \oo By rule, $\judgeA[\Theta; \Gamma; \Delta]{[e''/y]p \oplus [e''/y]q}{\assert}$\\
    \oo By subst def, $\judgeA[\Theta; \Gamma; \Delta]{[e''/y](p \oplus q)}{\assert}$\\
  \end{tabbedproof}

  For semantics, consider $\interp{\judgeA[\Theta; \Gamma; \Delta]{[e''/y](p \oplus q)}{\assert}}\;\theta\;\gamma\;\delta$
  \begin{eqnproof}
    \eline{\begin{array}{l}
             (\interp{\judgeA[\Theta; \Gamma; \Delta]{[e''/y]p}{\assert}}\;\theta\;\gamma\;\delta) \;\interp{\oplus}^2 \\
             \;\;(\interp{\judgeA[\Theta; \Gamma; \Delta]{[e''/y]q}{\assert}}\;\theta\;\gamma\;\delta) 
      \end{array}}
    {Semantics}
    \eline{\begin{array}{l}
             (\interp{\judgeA[\Theta; \Gamma, y:B; \Delta]{p}{\assert}}\;\theta\;(\gamma, \interp{\judgeE{\Gamma}{e''}{B}}\;\theta\;\gamma)\;\delta) \interp{\oplus}^2 \\
             \;\;(\interp{\judgeA[\Theta; \Gamma, y:B; \Delta]{q}{\assert}}\;\theta\;(\gamma, \interp{\judgeE{\Gamma}{e''}{B}}\;\theta\;\gamma)\;\delta) 
      \end{array}}
    {Induction}
    \eline{\interp{\judgeA[\Theta; \Gamma, y:B; \Delta]{p \oplus q}{\assert}}\;\theta\;(\gamma, \interp{\judgeE{\Gamma}{e''}{B}}\;\theta\;\gamma)\;\delta}
          {Semantics}
  \end{eqnproof}

\item Case \textsc{TQuantify1}: $\judgeA[\Theta, \alpha; \Gamma; \Delta]{Q u:\upsilon.\;p}{\assert}$
  
  First, the syntax:
  \begin{tabbedproof}
    \oo By inversion, $\judgeA[\Theta; \Gamma, y:B; \Delta, u:\upsilon]{p}{\assert}$ \\
    \oo By induction, $\judgeA[\Theta; \Gamma; \Delta, u:\upsilon]{[e''/y]p}{\assert}$ \\
    \oo By rule, $\judgeA[\Theta; \Gamma; \Delta]
                         {Q u:\upsilon.\;[e''/y]p}{\assert}$ \\
    \oo By def of subst, $\judgeA[\Theta; \Gamma; \Delta]
                                 {[e''/y](Q u:\upsilon.\;p)}{\assert}$ 
  \end{tabbedproof}

  For semantics, consider
  $\interp{\judgeA[\Theta; \Gamma; \Delta]
                  {[e''/y](Q u:\upsilon.\;p)}{\assert}}\;\theta\;\gamma\;\delta$ 
  \begin{eqnproof}
    \eline{\begin{array}{l}
            \interp{Q}_{v \in \interp{\judgeSort{\upsilon}}\;\theta} \\
            \;\;\interp{\judgeA[\Theta;\Gamma;\Delta, u:\upsilon]{[e''/y]p}{\assert}}\;\theta\,\gamma\,(\delta,v)
           \end{array}}
          {Semantics}
    \eline{\begin{array}{l}
            \interp{Q}_{v \in \interp{\judgeSort[\Theta, \alpha:\kappa]{\upsilon}}\;\theta} \\
            \;\;\interp{\judgeA[\Theta;\Gamma, y:B;\Delta, u:\upsilon]{p}{\assert}}\;\theta\,(\gamma, \interp{\judgeE{\Gamma}{e''}{B}}\;\theta\;\gamma)\,(\delta,v)
           \end{array}}
          {Induction}
    \eline{\interp{\judgeA[\Theta, \alpha:\kappa;\Gamma;\Delta]{Q u:\upsilon.\;p}{\assert}}\;\theta\,(\gamma, \interp{\judgeE{\Gamma}{e''}{B}}\;\theta\;\gamma)\,\delta}
          {Semantics}
  \end{eqnproof}

\item Case \textsc{TQuantify2}: $\judgeA[\Theta; \Gamma, y:B; \Delta]{Q x:A.\;p}{\assert}$
  
  First, the syntax:
  \begin{tabbedproof}
    \oo By inversion, $\judgeA[\Theta; \Gamma, y:B, x:A; \Delta]{p}{\assert}$ \\
    \oo By induction, $\judgeA[\Theta; \Gamma, x:A; \Delta]{[e''/y]p}{\assert}$ \\
    \oo By rule, $\judgeA[\Theta; \Gamma; \Delta]
                         {Q x:A.\;[e''/y]p}{\assert}$ \\
    \oo By def of subst, $\judgeA[\Theta; \Gamma; \Delta]
                                 {[e''/y](Q x:A.\;p)}{\assert}$ 
  \end{tabbedproof}

  For semantics, consider
  $\interp{\judgeA[\Theta; \Gamma; \Delta]
                  {[e''/y](Q x:A.\;p)}{\assert}}\;\theta\;\gamma\;\delta$ 
  \begin{eqnproof}
    \eline{\begin{array}{l}
            \interp{Q}_{v \in \interp{\judgeSort{A}}\;\theta} \\
            \;\;\interp{\judgeA[\Theta;\Gamma, x:A;\Delta]{[e''/y]p}{\assert}}\;\theta\,(\gamma,v)\,\delta
           \end{array}}
          {Semantics}
    \eline{\begin{array}{l}
            \interp{Q}_{v \in \interp{\judgeSort[\Theta, \alpha:\kappa]{A}}\;\theta} \\
            \;\;\interp{\judgeA[\Theta;\Gamma, y:B, x:A;\Delta]{p}{\assert}}\;\theta\,(\gamma, \interp{\judgeE{\Gamma}{e''}{B}}\;\theta\;\gamma, v)\,\delta
           \end{array}}
          {Induction}
    \eline{\interp{\judgeA[\Theta;\Gamma, y:B;\Delta]{Q x:A.\;p}{\assert}}\;\theta\,(\gamma, \interp{\judgeE{\Gamma}{e''}{B}}\;\theta\;\gamma)\,\delta}
          {Semantics}
  \end{eqnproof}
  Here, we make use of the fact that $x$ is not free in $e''$, and we silently permute the context as 
  needed. 

\item Case \textsc{TQuantify3}: $\judgeA[\Theta; \Gamma, y:B; \Delta]{Q \beta:\kappa'.\;p}{\assert}$
  
  First, the syntax:
  \begin{tabbedproof}
    \oo By inversion, $\judgeA[\Theta, \beta:\kappa'; \Gamma, y:B; \Delta]{p}{\assert}$ \\
    \oo By induction, $\judgeA[\Theta, \beta:\kappa'; \Gamma; \Delta]{[e''/y]p}{\assert}$ \\
    \oo By rule, $\judgeA[\Theta; \Gamma; \Delta]
                         {Q \beta:\kappa'.\;[e''/y]p}{\assert}$ \\
    \oo By def of subst, $\judgeA[\Theta; \Gamma; \Delta]
                                 {[e''/y](Q \beta:\kappa'.\;p)}{\assert}$ 
  \end{tabbedproof}

  For semantics, consider
  $\interp{\judgeA[\Theta; \Gamma; \Delta]
                  {[e''/y](Q \beta:\kappa'.\;p)}{\assert}}\;\theta\;\gamma\;\delta$ 
  \begin{eqnproof}
    \eline{\begin{array}{l}
            \interp{Q}_{\tau' \in \interp{\judgeSort{\kappa'}}\;\theta} \\
            \;\;\interp{\judgeA[\Theta, \beta:\kappa';\Gamma;\Delta]{[e''/y]p}{\assert}}\;(\theta,\tau')\,\gamma\,\delta
           \end{array}}
          {Semantics}
    \eline{\begin{array}{l}
            \interp{Q}_{\tau' \in \interp{\judgeSort[\Theta, \alpha:\kappa]{\kappa'}}\;\theta} \\
            \;\;\interp{\judgeA[\Theta, \beta:\kappa';\Gamma, y:B;\Delta]{p}{\assert}}\;(\theta, \tau)\,(\gamma, \interp{\judgeE{\Gamma}{e''}{B}}\;\theta\;\gamma)\,\delta
           \end{array}}
          {Induction}
    \eline{\interp{\judgeA[\Theta;\Gamma;\Delta]{Q \beta:\kappa'.\;p}{\assert}}\;\theta\,(\gamma, \interp{\judgeE{\Gamma}{e''}{B}}\;\theta\;\gamma)\,\delta}
          {Semantics}
  \end{eqnproof}
  In this case we silently use the fact that $\beta$ does not occur free in $e''$ or $B$.

\item Case \textsc{TEqual}: $\judgeA[\Theta; \Gamma, y:B; \Delta]{p =_\omega q}{\assert}$
  
  First, the syntax:
  \begin{tabbedproof}
    \oo By inversion, $\judgeA[\Theta; \Gamma, y:B; \Delta]{p}{\omega}$ \\
    \oo By inversion, $\judgeA[\Theta; \Gamma, y:B; \Delta]{q}{\omega}$ \\
    \oo By inversion, $\judgeSort[\Theta]{\omega}$ \\
    \oo By induction, $\judgeA[\Theta; \Gamma; \Delta]{[e''/y]p}{\omega}$ \\
    \oo By induction, $\judgeA[\Theta; \Gamma; \Delta]{[e''/y]q}{\omega}$ \\
    \oo By rule, $\judgeA[\Theta; \Gamma; \Delta]{[e''/y](p =_\omega q)}{\assert}$
  \end{tabbedproof}

  For the semantics, consider $\interp{\judgeA[\Theta; \Gamma; \Delta]{[e''/y](p =_\omega q)}{\assert}}\;\theta\;\gamma\;\delta$
  \begin{eqnproof}
    \eline{\mbox{if }\interp{[e''/y]p}\theta\;\gamma\;\delta = \interp{[e''/y]q}\theta\;\gamma\;\delta\mbox{ then }\top \mbox{ else } \bot}
          {Semantics}
    \eline{\mbox{if }\interp{p}\;\theta\;(\gamma, \interp{e''}\theta\;\gamma)\;\delta = \interp{q}\;\theta\;(\gamma, \interp{e''}\theta\;\gamma)\;\delta\mbox{ then }\top \mbox{ else } \bot}
          {Induction}
    \eline{\interp{\judgeA[\Theta; \Gamma, y:B; \Delta]{(p =_\omega q)}{\assert}}\;\theta\;(\gamma, \interp{\judgeE{\Gamma}{e''}{B}}\;\theta\;\gamma)\;\delta}
          {Semantics}
  \end{eqnproof}

\item Case \textsc{TPointsto}: $\judgeA[\Theta; \Gamma, y:B; \Delta]{e \pointsto_A e'}{\assert}$

  First, the syntax:
  \begin{tabbedproof}
    \oo By inversion, $\judgeA[\Theta; \Gamma, y:B; \Delta]{e}{\reftype{A}}$ \\
    \oo By inversion, $\judgeA[\Theta; \Gamma, y:B; \Delta]{e'}{A}$ \\
    \oo By induction, $\judgeA[\Theta; \Gamma; \Delta]{[e''/y]e}{\reftype{A}}$ \\
    \oo By induction, $\judgeA[\Theta; \Gamma; \Delta]{[e''/y]e'}{A}$ \\
    \oo By rule, $\judgeA[\Theta; \Gamma; \Delta]{[e''/y](e \pointsto_A e')}{\assert}$
  \end{tabbedproof}

  For the semantics, consider $\interp{\judgeA[\Theta; \Gamma; \Delta]{[e''/y](e \pointsto_A e')}{\assert}}\;\theta\;\gamma\;\delta$
  \begin{eqnproof}
    \eline{\begin{array}{l}
           \interp{\judgeE[\Theta]{\Gamma}{[e''/y]e}{\reftype{A}}}\;\theta\;\gamma \\ 
           \pointsto \\
           \interp{\judgeE[\Theta]{\Gamma}{[e''/y]e'}{A}}\;\theta\;\gamma
           \end{array}}
          {Semantics}
    \eline{\begin{array}{l}
           \interp{\judgeE{\Gamma, y:B}{e}{\reftype{A}}}\;\theta\;(\gamma, \interp{\judgeE{\Gamma}{e''}{B}}\;\theta\;\gamma) \\ 
           \pointsto \\
           \interp{\judgeE{\Gamma, y:B}{e'}{A}}\;\theta\;(\gamma, \interp{\judgeE{\Gamma}{e''}{B}}\;\theta\;\gamma)
           \end{array}}
          {Induction}
    \eline{\interp{\judgeA[\Theta; \Gamma, y:B; \Delta]{e \pointsto_A e'}{\assert}}\;\theta\;(\gamma, \interp{\judgeE{\Gamma}{e''}{B}}\;\theta\;\gamma)\;\delta}
          {Semantics}
  \end{eqnproof}

\item Case \textsc{TEqSort}: $\judgeA[\Theta; \Gamma, y:B; \Delta]{p}{\omega}$
  
  First, the syntax:
  \begin{tabbedproof}
    \oo By inversion, $\judgeSortEq{\omega}{\omega'}$ \\
    \oo By inversion, $\judgeA[\Theta; \Gamma, y:B; \Delta]{p}{\omega'}$ \\
    \oo By induction, $\judgeA[\Theta; \Gamma; \Delta]{[e''/y]p}{\omega'}$ \\
    \oo By rule, $\judgeA[\Theta; \Gamma; \Delta]{[e''/y]p}{\omega}$ \\
  \end{tabbedproof}

  For the semantics, consider $\interp{\judgeA[\Theta; \Gamma; \Delta]{[e''/y]p}{\omega}}\;\theta\;\gamma\;\delta$ \\
  \begin{eqnproof}
    \eline{\interp{\judgeA[\Theta; \Gamma; \Delta]{[e''/y]p}{\omega'}}\;\theta\;\gamma\;\delta}
          {Semantics}
    \eline{\interp{\judgeA[\Theta; \Gamma, y:B; \Delta]{p}{\omega'}}\;\theta\;(\gamma, \interp{\judgeE{\Gamma}{e''}{B}}\;\theta\;\gamma)\;\delta}
          {Induction}
    \eline{\interp{\judgeA[\Theta; \Gamma, y:B; \Delta]{p}{\omega}}\;\theta\;(\gamma, \interp{\judgeE{\Gamma}{e''}{B}}\;\theta\;\gamma)\;\delta}
          {Semantics}
  \end{eqnproof}

\item Case \textsc{TSpec}: $\judgeA[\Theta; \Gamma, y:B; \Delta]{\validprop{S}}{\assert}$:
  
  First, the syntax:
  \begin{tabbedproof}
    \oo By inversion, $\judgeS[\Theta; \Gamma, y:B; \Delta]{S}$ \\
    \oo By mutual induction $\judgeS[\Theta; \Gamma; \Delta]{[e''/y]S}$\\
    \oo By rule, $\judgeA[\Theta; \Gamma; \Delta]{[e''/y]\validprop{S}}{\assert}$
  \end{tabbedproof}

  For the semantics, consider $\interp{\judgeA[\Theta; \Gamma; \Delta]{[e''/y]\validprop{S}}{\assert}}\;\theta\;\gamma\;\delta$ 
  \begin{eqnproof}
    \eline{\mbox{if } \interp{[e''/y]S}\;\theta\;\gamma\;\delta = \top \mbox{ then } \top \mbox{ else } \bot}
          {Semantics}
    \eline{\mbox{if } \interp{S}\;\theta\;(\gamma, \interp{\judgeE{\Gamma}{e''}{B}}\;\theta\;\gamma)\;\delta = \top \mbox{ then } \top \mbox{ else } \bot}
          {Induction}
    \eline{\interp{\judgeA[\Theta; \Gamma, y:B; \Delta]{\validprop{S}}{\assert}}\; \theta\;(\gamma, \interp{\judgeE{\Gamma}{e''}{B}}\;\theta\;\gamma)\;\delta}
          {Semantics}
  \end{eqnproof}

\item Case \textsc{SpecTriple}: $\judgeS[\Theta; \Gamma, y:B; \Delta]{\spec{p}{c}{a:A}{q}}$
  
  First, the syntax:
  \begin{tabbedproof}
    \oo By inversion, $\judgeA[\Theta; \Gamma, y:B; \Delta]{p}{\assert}$ \\
    \oo By inversion, $\judgeA[\Theta; \Gamma, y:B; \Delta]{\comp{c}}{\monad{A}}$ \\
    \oo By inversion, $\judgeA[\Theta; \Gamma, y:B; \Delta, a:A]{q}{\assert}$ \\
    \oo By induction, $\judgeA[\Theta; \Gamma; \Delta]{[e''/y]p}{\assert}$ \\
    \oo By induction, $\judgeA[\Theta; \Gamma; \Delta]{\comp{[e''/y]c}}{\monad{A}}$ \\
    \oo By induction, $\judgeA[\Theta; \Gamma; \Delta, a:A]{[e''/y]q}{\assert}$ \\
    \oo By rule, $\judgeS[\Theta; \Gamma; \Delta]{[e''/y](\spec{p}{c}{a:A}{q})}$
  \end{tabbedproof}

  For the semantics, consider $\interp{\judgeS[\Theta; \Gamma; \Delta]{[e''/y](\spec{p}{c}{a:A}{q})}}\;\theta;\gamma\;\delta$
  \begin{eqnproof}
    \eline{\begin{array}{l}
           \{\interp{\judgeA[\Theta; \Gamma; \Delta]{[e''/y]p}{\assert}}\;\theta\;\gamma\;\delta\} \\
           \interp{\judgeC{\Gamma}{[e''/y]c}{A}}\;\theta\;\gamma\;\delta \\
           \{v.\;\interp{\judgeA[\Theta; \Gamma, a:A; \Delta]{[e''/y]q}{\assert}}\;\theta\;(\gamma,v)\;\delta\} 
           \end{array}}
          {Semantics}
    \eline{\begin{array}{l}
           \{\interp{\judgeA[\Theta, \Gamma; \Delta]{p}{\assert}}\;\theta\;(\gamma, \interp{\judgeE{\Gamma}{e''}{B}}\;\theta\;\gamma)\;\delta\} \\
           \interp{\judgeC[\Theta]{\Gamma, y:B}{c}{A}}\;\theta\;(\gamma, \interp{\judgeE{\Gamma}{e''}{B}}\;\theta\;\gamma)\;\delta \\
           \{v.\;\interp{\judgeA[\Theta; \Gamma, y:B, a:A; \Delta]{q}{\assert}}\;\theta\;(\gamma, \interp{\judgeE{\Gamma}{e''}{B}}\;\theta\;\gamma, v)\;\delta\} 
           \end{array}}
          {Induction}
    \eline{\interp{\judgeS[\Theta; \Gamma, y:B; \Delta]{(\spec{p}{c}{a:A}{q})}}\;\theta;(\gamma, \interp{\judgeE{\Gamma}{e''}{B}}\;\theta\;\gamma)\;\delta}
          {Semantics}
  \end{eqnproof}
  The correctness of the application of $\gamma$ and $\delta$ follows from the equations for
  contexts under substitution. 

\item Case \textsc{SpecMTriple}: $\judgeS[\Theta; \Gamma, y:B; \Delta]{\mspec{p}{e}{a:A}{q}}$
  
  First, the syntax:
  \begin{tabbedproof}
    \oo By inversion, $\judgeA[\Theta; \Gamma, y:B; \Delta]{p}{\assert}$ \\
    \oo By inversion, $\judgeA[\Theta; \Gamma, y:B; \Delta]{e}{\monad{A}}$ \\
    \oo By inversion, $\judgeA[\Theta; \Gamma, y:B; \Delta, a:A]{q}{\assert}$ \\
    \oo By induction, $\judgeA[\Theta; \Gamma; \Delta]{[e''/y]p}{\assert}$ \\
    \oo By induction, $\judgeA[\Theta; \Gamma; \Delta]{[e''/y]e}{\monad{A}}$ \\
    \oo By induction, $\judgeA[\Theta; \Gamma; \Delta, a:A]{[e''/y]q}{\assert}$ \\
    \oo By rule, $\judgeS[\Theta; \Gamma; \Delta]{[e''/y](\mspec{p}{e}{a:A}{q})}$
  \end{tabbedproof}

  For the semantics, consider $\interp{\judgeS[\Theta; \Gamma; \Delta]{[e''/y](\mspec{p}{e}{a:A}{q})}}\;\theta;\gamma\;\delta$
  \begin{eqnproof}
    \eline{\begin{array}{l}
           \{\interp{\judgeA[\Theta; \Gamma; \Delta]{[e''/y]p}{\assert}}\;\theta\;\gamma\;\delta\} \\
           \interp{\judgeE{\Gamma}{[e''/y]e}{\monad{A}}}\;\theta\;\gamma\;\delta \\
           \{v.\;\interp{\judgeA[\Theta; \Gamma, a:A; \Delta]{[e''/y]q}{\assert}}\;\theta\;(\gamma,v)\;\delta\} 
           \end{array}}
          {Semantics}
    \eline{\begin{array}{l}
           \{\interp{\judgeA[\Theta; \Gamma, y:B; \Delta]{p}{\assert}}\;\theta\;(\gamma, \interp{\judgeE{\Gamma}{e''}{B}}\;\theta\;\gamma)\;\delta\} \\
           \interp{\judgeE{\Gamma, y:B}{e}{\monad{A}}}\;\theta\;(\gamma, \interp{\judgeE{\Gamma}{e''}{B}}\;\theta\;\gamma)\;\delta \\
           \{v.\;\interp{\judgeA[\Theta; \Gamma, y:B, a:A; \Delta]{q}{\assert}}\;\theta\;(\gamma, \interp{\judgeE{\Gamma}{e''}{B}}\;\theta\;\gamma, v)\;\delta\} 
           \end{array}}
          {Induction}
    \eline{\interp{\judgeS[\Theta; \Gamma, y:B; \Delta]{(\mspec{p}{e}{a:A}{q})}}\;\theta;(\gamma, \interp{\judgeE{\Gamma}{e''}{B}}\;\theta\;\gamma)\;\delta}
          {Semantics}
  \end{eqnproof}

\item Case \textsc{SpecQuantify1}: $\judgeS[\Theta; \Gamma, y:B; \Delta]{Q u:\upsilon.\;S}$
  
  First, the syntax:
  \begin{tabbedproof}
    \oo By inversion, $\judgeS[\Theta; \Gamma, y:B; \Delta, u:\upsilon]{S}$ \\
    \oo By induction, $\judgeS[\Theta; \Gamma; \Delta, u:\upsilon]{[e''/y]S}$ \\
    \oo By rule, $\judgeS[\Theta; \Gamma; \Delta]
                         {Q u:\upsilon.\;[e''/y]S}$ \\
    \oo By def of subst, $\judgeS[\Theta; \Gamma; \Delta]
                                 {[e''/y](Q u:\upsilon.\;S)}$ 
  \end{tabbedproof}

  For semantics, consider
  $\interp{\judgeS[\Theta; \Gamma; \Delta]
                  {[e''/y](Q u:\upsilon.\;S)}}\;\theta\;\gamma\;\delta$ 
  \begin{eqnproof}
    \eline{\begin{array}{l}
            \interp{Q}_{v \in \interp{\judgeSort{\upsilon}}\;\theta} \\
            \;\;\interp{\judgeS[\Theta;\Gamma;\Delta, u:\upsilon]{[e''/y]S}}\;\theta\,\gamma\,(\delta,v)
           \end{array}}
          {Semantics}
    \eline{\begin{array}{l}
            \interp{Q}_{v \in \interp{\judgeSort[\Theta]{\upsilon}}\;\theta} \\
            \;\;\interp{\judgeS[\Theta;\Gamma, y:B;\Delta, u:\upsilon]{S}}\;\theta\,(\gamma, \interp{\judgeE{\Gamma}{e''}{B}}\;\theta\;\gamma)\,(\delta,v)
           \end{array}}
          {Induction}
    \eline{\interp{\judgeS[\Theta;\Gamma, y:B;\Delta]{Q u:\upsilon.\;S}}\;\theta\,(\gamma, \interp{\judgeE{\Gamma}{e''}{B}}\;\theta\;\gamma)\,\delta}
          {Semantics}
  \end{eqnproof}

\item Case \textsc{SpecQuantify2}: $\judgeS[\Theta; \Gamma, y:B; \Delta]{Q x:A.\;S}$
  
  First, the syntax:
  \begin{tabbedproof}
    \oo By inversion, $\judgeS[\Theta; \Gamma, y:B, x:A; \Delta]{S}$ \\
    \oo By induction, $\judgeS[\Theta; \Gamma, x:A; \Delta]{[e''/y]S}$ \\
    \oo By rule, $\judgeS[\Theta; \Gamma; \Delta]
                         {Q x:A.\;[e''/y]S}$ \\
    \oo By def of subst, $\judgeS[\Theta; \Gamma; \Delta]
                                 {[e''/y](Q x:A.\;S)}$ 
  \end{tabbedproof}

  For semantics, consider
  $\interp{\judgeS[\Theta; \Gamma; \Delta]
                  {[e''/y](Q x:A.\;S)}}\;\theta\;\gamma\;\delta$ 
  \begin{eqnproof}
    \eline{\begin{array}{l}
            \interp{Q}_{v \in \interp{\judgeSort{A}}\;\theta} \\
            \;\;\interp{\judgeS[\Theta;\Gamma, x:A;\Delta]{[e''/y]S}}\;\theta\,(\gamma,v)\,\delta
           \end{array}}
          {Semantics}
    \eline{\begin{array}{l}
            \interp{Q}_{v \in \interp{\judgeSort[\Theta]{A}}\;\theta} \\
            \;\;\interp{\judgeS[\Theta;\Gamma, y:A, x:A;\Delta]{S}}\;\theta\,(\gamma, \interp{\judgeE{\Gamma}{e''}{B}}\;\theta\;\gamma,v)\,\delta
           \end{array}}
          {Induction}
    \eline{\interp{\judgeS[\Theta;\Gamma, y:A;\Delta]{Q x:A.\;S}}\;\theta\,(\gamma, \interp{\judgeE{\Gamma}{e''}{B}}\;\theta\;\gamma)\,\delta}
          {Semantics}
  \end{eqnproof}

\item Case \textsc{SpecQuantify3}: $\judgeS[\Theta; \Gamma, y:A; \Delta]{Q \beta:\kappa'.\;S}$
  
  First, the syntax:
  \begin{tabbedproof}
    \oo By inversion, $\judgeS[\Theta, \beta:\kappa'; \Gamma, y:A; \Delta]{S}$ \\
    \oo By induction, $\judgeS[\Theta, \beta:\kappa'; \Gamma; \Delta]{[e''/y]S}$ \\
    \oo By rule, $\judgeS[\Theta; \Gamma; \Delta]
                         {Q \beta:\kappa'.\;[e''/y]S}$ \\
    \oo By def of subst, $\judgeS[\Theta; \Gamma; \Delta]
                                 {[e''/y](Q \beta:\kappa'.\;S)}$ 
  \end{tabbedproof}

  For semantics, consider
  $\interp{\judgeS[\Theta; \Gamma; \Delta]
                  {[e''/y](Q \beta:\kappa'.\;S)}}\;\theta\;\gamma\;\delta$ 
  \begin{eqnproof}
    \eline{\begin{array}{l}
            \interp{Q}_{\tau' \in \interp{\judgeSort{\kappa'}}\;\theta} \\
            \;\;\interp{\judgeS[\Theta, \beta:\kappa';\Gamma;\Delta]{[e''/y]S}}\;(\theta,\tau')\,\gamma\,\delta
           \end{array}}
          {Semantics}
    \eline{\begin{array}{l}
            \interp{Q}_{\tau' \in \interp{\judgeSort[\Theta, \alpha:\kappa]{\kappa'}}\;\theta} \\
            \;\;\interp{\judgeS[\Theta, \alpha:\kappa, \beta:\kappa';\Gamma;\Delta]{S}}\;(\theta, \tau')\,(\gamma, \interp{\judgeE{\Gamma}{e''}{B}}\;\theta\;\gamma)\,(\delta,v)
           \end{array}}
          {Induction}
    \eline{\interp{\judgeS[\Theta;\Gamma;\Delta]{Q \beta:\kappa'.\;S}}\;\theta\,(\gamma, \interp{\judgeE{\Gamma}{e''}{B}}\;\theta\;\gamma)\,\delta}
          {Semantics}
  \end{eqnproof}


\item Case \textsc{SpecBinary}: $\judgeS[\Theta; \Gamma, y:B; \Delta]{S \oplus S'}$
  
  First, the syntax:
  \begin{tabbedproof}
    \oo By inversion, $\judgeS[\Theta; \Gamma, y:B; \Delta]{S}$\\
    \oo By inversion, $\judgeS[\Theta; \Gamma, y:B; \Delta]{S'}$\\
    \oo By induction, $\judgeS[\Theta; \Gamma; \Delta]{[e''/y]S}$\\
    \oo By induction, $\judgeS[\Theta; \Gamma; \Delta]{[e''/y]S'}$\\
    \oo By rule, $\judgeS[\Theta; \Gamma; \Delta]{[e''/y]S \oplus [e''/y]S'}$\\
    \oo By subst def, $\judgeS[\Theta; \Gamma; \Delta]{[e''/y](S \oplus S')}$\\
  \end{tabbedproof}

  For semantics, consider $\interp{\judgeS[\Theta; \Gamma; \Delta]{[e''/y](S \oplus S')}}\;\theta\;\gamma\;\delta$
  \begin{eqnproof}
    \eline{\begin{array}{l}
             (\interp{\judgeS[\Theta; \Gamma; \Delta]{[e''/y]S}}\;\theta\;\gamma\;\delta) \;\interp{\oplus} \\
             \;\;(\interp{\judgeS[\Theta; \Gamma; \Delta]{[e''/y]S'}}\;\theta\;\gamma\;\delta) 
      \end{array}}
    {Semantics}
    \eline{\begin{array}{l}
             (\interp{\judgeS[\Theta; \Gamma, y:B; \Delta]{S}}\;\theta\;(\gamma, \interp{\judgeE{\Gamma}{e''}{B}}\;\theta\;\gamma)\;\delta) \interp{\oplus} \\
             \;\;(\interp{\judgeS[\Theta; \Gamma, y:B; \Delta]{S'}}\;\theta\;(\gamma, \interp{\judgeE{\Gamma}{e''}{B}}\;\theta\;\gamma)\;\delta) 
      \end{array}}
    {Induction}
    \eline{\interp{\judgeS[\Theta; \Gamma, y:B; \Delta]{S \oplus S'}}\;\theta\;(\gamma, \interp{\judgeE{\Gamma}{e''}{B}}\;\theta\;\gamma)\;\delta}
          {Semantics}
  \end{eqnproof}

\item Case \textsc{TSpec}: $\judgeS[\Theta; \Gamma, y:B; \Delta]{\setof{p}}$:
  
  First, the syntax:
  \begin{tabbedproof}
    \oo By inversion, $\judgeA[\Theta; \Gamma, y:B; \Delta]{p}{\assert}$ \\
    \oo By mutual induction $\judgeA[\Theta; \Gamma; \Delta]{[e''/y]p}{\assert}$\\
    \oo By rule, $\judgeS[\Theta; \Gamma; \Delta]{[e''/y]\setof{p}}$
  \end{tabbedproof}

  For the semantics, consider $\interp{\judgeS[\Theta; \Gamma; \Delta]{[e''/y]\setof{p}}}\;\theta\;\gamma\;\delta$ 
  \begin{eqnproof}
    \eline{\mbox{if } \interp{[e''/y]p}\;\theta\;\gamma\;\delta = \top \mbox{ then } \top \mbox{ else } \bot}
          {Semantics}
    \eline{\mbox{if } \interp{p}\;\theta\;(\gamma, \interp{\judgeE{\Gamma}{e''}{B}}\;\theta\;\gamma)\;\delta = \top \mbox{ then } \top \mbox{ else } \bot}
          {Induction}
    \eline{\interp{\judgeS[\Theta; \Gamma, y:B; \Delta]{\setof{p}}}\; \theta\;(\gamma, \interp{\judgeE{\Gamma}{e''}{B}}\;\theta\;\gamma)\;\delta}
          {Semantics}
  \end{eqnproof}
\end{enumerate}

\ \\

First, assume $\judgeA{r}{\upsilon''}$, and $\judgeA[\Theta; \Gamma;\Delta, b:\upsilon'']{p}{\omega}$, and
$\judgeS[\Theta; \Gamma, y:B; \Delta, b:\upsilon'']{S}$. 

Now, we proceed by mutual induction on the derivation of $p$ and $S$: 
\begin{enumerate}

\item Case \textsc{TType}: $\judgeA[\Theta; \Gamma; \Delta, b:\upsilon'']{\tau'}{\kappa'}$

  First, the syntax:
  \begin{tabbedproof}
    \oo By inversion, we know $\judgectx{\Theta}{\Delta, b:\upsilon''}$ \\
    \oo By inversion, we know $\judgectx{\Theta}{\Gamma}$ \\
    \oo By inversion, we know $\judgectx{\Theta}{\Delta}$ \\
    \oo By rule, $\judgeA[\Theta; \Gamma;\Delta]{\tau'}{\kappa'}$
  \end{tabbedproof}

  For semantics,  $\interp{\judgeA[\Theta; \Gamma; \Delta]{[r/b]\tau'}{\kappa'}}\;\theta\;\gamma\;\delta$
  \begin{eqnproof}
    \eline{\interp{\judgeWK[\Theta]{\tau'}{\kappa'}}\;\theta}
          {Semantics}
    \eline{\interp{\judgeA[\Theta;\Gamma;\Delta, b:\upsilon'']{\tau'}{\kappa'}}\;\theta\;\gamma\;(\delta, \interp{\judgeA{r}{\upsilon''}}\;\theta\;\gamma\;\delta)}
          {Semantics}
  \end{eqnproof}

  \item Case \textsc{TExpr}: $\judgeA[\Theta; \Gamma; \Delta, b:\upsilon'']{e}{A}$

    First, the syntax:
    \begin{tabbedproof}
      \oo By inversion, we know $\judgectx{\Theta}{\Delta, b:\upsilon''}$ \\
      \oo By inversion, we know $\judgectx{\Theta}{\Delta}$ \\
      \oo By inversion, we know $\judgeE{\Gamma}{e}{A}$ \\
      \oo By substitution, we know $\judgeE{\Gamma}{[r/b]e}{A}$ \\
      \oo By rule, we know $\judgeA[\Theta; \Gamma; \Delta]{[r/b]e}{A}$
    \end{tabbedproof}

    For semantics, consider $\interp{\judgeA[\Theta; \Gamma; \Delta]{[r/b]e}{A}}\;\theta\;\gamma\;\delta$
    \begin{eqnproof}
      \eline{\interp{\judgeE{\Gamma}{e}{A}}\;\theta\;\gamma}
            {Semantics}
      \eline{\interp{\judgeA[\Theta; \Gamma; \Delta, b:\upsilon'']{e}{A}}\;\theta\;\gamma\;(\delta, \interp{\judgeA{r}{\upsilon''}}\;\theta\;\gamma\;\delta)}
            {Semantics}
    \end{eqnproof}

\item Case \textsc{THyp}: $\judgeA[\Theta; \Gamma; \Delta, b:\upsilon'']{u}{\upsilon}$

  There are two cases. 
  \begin{itemize}
  \item Case $u = b$: (hence $\upsilon'' = \upsilon$)

    In this case, the syntax follows since $\judgeA{r}{\upsilon''}$

    For semantics, consider that $\interp{\judgeA[\Theta; \Gamma; \Delta]{r}{\upsilon}}\;\theta;\gamma\;\delta$
    \begin{eqnproof}
    \eline{\interp{\judgeA[\Theta; \Gamma; \Delta, b:\upsilon]{b}{\upsilon}}\;\theta\;\gamma\;(\delta, \interp{\judgeA{r}{\upsilon}}\;\theta\;\gamma\;\delta)}
          {Definition}
    \end{eqnproof}

  \item Case $u \not= b$: 

  First, the syntax:
  \begin{tabbedproof}
  \oo By strengthening, $\judgeA[\Theta; \Gamma; \Delta]{[r/b]u}{\upsilon}$ \\
  \end{tabbedproof}

  For semantics, consider that $\interp{\judgeA[\Theta; \Gamma; \Delta]{[r/b]u}{\upsilon}}\;\theta;\gamma\;delta$ \\
  \begin{eqnproof}
    \eline{\pi_u(delta)}
          {Semantics} 
    \eline{\interp{\judgeA[\Theta; \Gamma; \Delta, b:\upsilon'']{u}{\upsilon}}\;\theta;\gamma\;(\delta, \interp{\judgeA{r}{\upsilon''}}\;\theta\;\gamma\;\delta)}
          {Semantics}
  \end{eqnproof}

  \end{itemize}




\item Case \textsc{TAbs1}: $\judgeA[\Theta; \Gamma; \Delta, b:\upsilon'']{\pfun{u}{\upsilon'}{p}}{\upsilon' \To \upsilon}$
  
  First, the syntax:
  \begin{tabbedproof}
    \oo By inversion, $\judgeA[\Theta; \Gamma; \Delta, b:\upsilon'', u:\upsilon']{p}{\upsilon}$ \\
    \oo By induction, $\judgeA[\Theta; \Gamma; \Delta, u:\upsilon']{[r/b]p}{\upsilon}$ \\
    \oo By rule, $\judgeA[\Theta; \Gamma; \Delta]{\pfun{u}{\upsilon'}{[r/b]p}}{\upsilon' \To \upsilon}$ \\
    \oo By def of subst, $\judgeA[\Theta; \Gamma; \Delta]{[r/b](\pfun{u}{\upsilon'}{p})}{\upsilon' \To \upsilon}$ \\
  \end{tabbedproof}
  %
  For semantics, consider $\interp{\judgeA[\Theta; \Gamma; \Delta]{(\pfun{u}{\upsilon'}{[r/b]p})}{\upsilon' \To \upsilon}}\;\theta\;\gamma\;\delta$ \\
  \begin{eqnproof}
    \eline{\semfun{v}{\interp{\judgeA[\Theta; \Gamma; \Delta, u:\upsilon']{[r/b]p}{\upsilon}}\;\theta\;\gamma\;(\delta,v)}}
          {Semantics}
    \eline{\semfun{v}{\interp{\judgeA[\Theta; \Gamma; \Delta, b:\upsilon'', u:\upsilon']{p}{\upsilon}}\;\theta\;\gamma\;(\delta, \interp{\judgeA{r}{\upsilon''}}\;\theta\;\gamma\;\delta,v)}}
          {Induction}
    \eline{\interp{\judgeA[\Theta; \Gamma; \Delta, b:\upsilon'']{\pfun{u}{\upsilon'}{p}}{\upsilon' \To \upsilon}}\;\theta\;\gamma\;(\delta, \interp{\judgeA{r}{\upsilon''}}\;\theta\;\gamma\;\delta)}
          {Semantics}
  \end{eqnproof}

\item Case \textsc{TAbs2}: $\judgeA[\Theta; \Gamma; \Delta, b:\upsilon'']{\pfun{x}{A}{p}}{A \To \upsilon}$
  
  First, the syntax:
  \begin{tabbedproof}
    \oo By inversion, $\judgeA[\Theta; \Gamma, x:A; \Delta, b:\upsilon'']{p}{\upsilon}$ \\
    \oo By induction, $\judgeA[\Theta; \Gamma, x:A; \Delta]{[r/b]p}{\upsilon}$ \\
    \oo By rule, $\judgeA[\Theta; \Gamma; \Delta]{\pfun{x}{A}{[r/b]p}}{A \To \upsilon}$ \\
    \oo By def of subst, $\judgeA[\Theta; \Gamma; \Delta]{[r/b](\pfun{x}{A}{p})}{(A \To \upsilon)}$ \\
  \end{tabbedproof}
  %
  For semantics, consider $\interp{\judgeA[\Theta; \Gamma; \Delta]{[r/b](\pfun{x}{A}{p})}{(A \To \upsilon)}}\;\theta\;\gamma\;\delta$ \\
  \begin{eqnproof}
    \eline{\semfun{v}{\interp{\judgeA[\Theta; \Gamma, x:A; \Delta]{[r/b]p}{\upsilon}}\;\theta\;(\gamma,v)\;\delta}}
          {Semantics}
    \eline{\semfun{v}{\interp{\judgeA[\Theta; \Gamma, x:A; \Delta, b:\upsilon'']{p}{\upsilon}}\;\theta\;(\gamma, v)\;(\delta, \interp{\judgeA{r}{\upsilon''}}\;\theta\;\gamma\;\delta)}}
          {Induction}
    \eline{\interp{\judgeA[\Theta; \Gamma; \Delta, b:\upsilon'']{\pfun{x}{A}{p}}{A \To \upsilon}}\;\theta\;\gamma\;(\delta, \interp{\judgeA{r}{\upsilon''}}\;\theta\;\gamma\;\delta)}
          {Semantics}
  \end{eqnproof}
  We silently permute arguments in the second line. 

\item Case \textsc{TAbs3}: $\judgeA[\Theta; \Gamma; \Delta, b:\upsilon'']{\pfun{\beta}{\kappa'}{p}}{\kappa' \To \upsilon}$
  
  First, the syntax:
  \begin{tabbedproof}
    \oo By inversion, $\judgeA[\Theta, \beta:\kappa'; \Gamma; \Delta, b:\upsilon'']{p}{\upsilon}$ \\
    \oo By weakening, $\judgeA[\Theta, \beta:\kappa']{\Gamma}{r}{\upsilon''}$ \\
    \oo By induction, $\judgeA[\Theta, \beta:\kappa'; \Gamma; \Delta]{{[r/b]}p}{\upsilon}$ \\
    \oo By rule, $\judgeA[\Theta; \Gamma; \Delta]{\pfun{\beta}{\kappa'}{{[r/b]}p}}{\kappa' \To \upsilon}$ \\
    \oo By def of subst, $\judgeA{{[r/b]}(\pfun{\beta}{\kappa'}{p})}{\kappa' \To \upsilon}$ \\
  \end{tabbedproof}
  %
  For semantics, consider $\interp{\judgeA{[r/b](\pfun{\beta}{\kappa'}{p})}{\kappa' \To \upsilon}}\;\theta\;\gamma\;\delta$ \\
  \begin{eqnproof}
    \eline{\semfun{\tau}{\interp{\judgeA[\Theta, \beta:\kappa'; \Gamma; \Delta]{[r/b]p}{\upsilon}}\;(\theta,\tau)\;\gamma\;\delta}}
          {Semantics}
    \eline{\semfun{\tau}{\interp{\judgeA[\Theta, \beta:\kappa'; \Gamma; \Delta, b:\upsilon'']{p}{\upsilon}}\;(\theta, \tau)\;\gamma\;(\delta, \interp{\judgeA{r}{\upsilon''}}\;\theta\;\gamma\;\delta)}}
          {Induction}
    \eline{\interp{\judgeA[\Theta; \Gamma; \Delta, b:\upsilon'']{\pfun{\beta}{\kappa'}{p}}{\kappa' \To \upsilon}}\;\theta\;\gamma\;(\delta, \interp{\judgeA{r}{\upsilon''}}\;\theta\;\gamma\;\delta)}
          {Semantics}
  \end{eqnproof}
  This case relies upon the fact that $\Gamma$ and $\Delta$ do not have $\beta$ free and the 
  equality of sorts under substitution. 

\item Case \textsc{TAbsAll}: $\judgeA[\Theta; \Gamma; \Delta, b:\upsilon'']{\pfunall{\beta}{\kappa'}{p}}{\forallsort{\beta}{\kappa'}{\upsilon}}$
  
  First, the syntax:
  \begin{tabbedproof}
    \oo By inversion, $\judgeA[\Theta, \beta:\kappa'; \Gamma; \Delta, b:\upsilon'']{p}{\upsilon}$ \\
    \oo By induction, $\judgeA[\Theta, \beta:\kappa'; \Gamma; \Delta]{[r/b]p}{\upsilon}$ \\
    \oo By rule, $\judgeA{\pfunall{\beta}{\kappa'}{[r/b]p}}{\forallsort{\beta}{\kappa'}{\upsilon}}$ \\
    \oo By def of subst, $\judgeA{[r/b](\pfunall{\beta}{\kappa'}{p})}{\forallsort{\beta}{\kappa'}{\upsilon}}$ \\
  \end{tabbedproof}
  %
  For semantics, consider $\interp{\judgeA{[r/b](\pfun{\beta}{\kappa'}{p})}{\forallsort{\beta}{\kappa'}{\upsilon}}}\;\theta\;\gamma\;\delta$ \\
  \begin{eqnproof}
    \eline{\semfun{\tau}{\interp{\judgeA[\Theta, \beta:\kappa'; \Gamma; \Delta]{[r/b]p}{\upsilon}}\;(\theta,\tau)\;\gamma\;\delta}}
          {Semantics}
    \eline{\semfun{\tau}{\interp{\judgeA[\Theta, \beta:\kappa'; \Gamma; \Delta, b:\upsilon'', \beta:\kappa']{p}{\upsilon}}\;(\theta, \tau)\;\gamma\;(\delta, \interp{\judgeA{r}{\upsilon''}}\;\theta\;\gamma\;\delta)}}
          {Induction}
    \eline{\interp{\judgeA[\Theta; \Gamma; \Delta, b:\upsilon'']{\pfun{\beta}{\kappa'}{p}}{\forallsort{\beta}{\kappa'}{\upsilon}}}\;\theta\;\gamma\;(\delta, \interp{\judgeA{r}{\upsilon''}}\;\theta\;\gamma\;\delta)}
          {Semantics}
  \end{eqnproof}
  This case relies upon the fact that $\Gamma$ and $\Delta$ do not have $\beta$ free and the 
  equality of sorts under substitution. 



\item Case \textsc{TApp}: $\judgeA[\Theta; \Gamma; \Delta, b:\upsilon'']{p\;q}{\upsilon}$
  
  First, the syntax:
  \begin{tabbedproof}
    \oo By inversion, $\judgeA[\Theta; \Gamma; \Delta, b:\upsilon'']{p}{\omega \To \upsilon}$\\
    \oo By inversion, $\judgeA[\Theta; \Gamma; \Delta, b:\upsilon'']{q}{\omega}$\\
    \oo By induction, $\judgeA[\Theta; \Gamma; \Delta]{[r/b]p}{(\omega \To \upsilon)}$\\
    \oo By induction, $\judgeA[\Theta; \Gamma; \Delta]{[r/b]q}{\omega}$\\
    \oo By rule, $\judgeA[\Theta; \Gamma; \Delta]{[r/b](p\;q)}{\omega \To \upsilon}$\\
  \end{tabbedproof}

  For semantics, consider $\interp{\judgeA[\Theta; \Gamma; \Delta]{[r/b](p\;q)}{\upsilon}}\;\theta\;\gamma\;\delta$
  \begin{eqnproof}
    \eline{\begin{array}{l}
             (\interp{\judgeA[\Theta; \Gamma; \Delta]{[r/b]p}{(\omega \To \upsilon)}}\;\theta\;\gamma\;\delta) \\
             \;\;(\interp{\judgeA[\Theta; \Gamma; \Delta]{[r/b]q}{\omega}}\;\theta\;\gamma\;\delta) 
      \end{array}}
    {Semantics}
    \eline{\begin{array}{l}
             (\interp{\judgeA[\Theta; \Gamma; \Delta, b:\upsilon'']{p}{\omega \To \upsilon}}\;\theta\;\gamma\;\delta) \\
             \;\;(\interp{\judgeA[\Theta; \Gamma; \Delta, b:\upsilon'']{q}{\omega}}\;\theta\;\gamma\;(\delta, \interp{\judgeA{r}{\upsilon''}}\;\theta\;\gamma\;\delta)) 
      \end{array}}
    {Induction}
    \eline{\interp{\judgeA[\Theta; \Gamma; \Delta, b:\upsilon'']{p\;q}{\upsilon}}\;\theta\;\gamma\;(\delta, \interp{\judgeA{r}{\upsilon''}}\;\theta\;\gamma\;\delta)}
          {Semantics}
  \end{eqnproof}


\item Case \textsc{TAppAll}: $\judgeA[\Theta; \Gamma; \Delta, b:\upsilon'']{p\;[\tau']}{[\tau'/\beta]\upsilon}$
  
  First, the syntax:
  \begin{tabbedproof}
    \oo By inversion, $\judgeA[\Theta; \Gamma; \Delta, b:\upsilon'']{p}{\forallsort{\beta}{\kappa'}{\upsilon}}$\\
    \oo By inversion, $\judgeA[\Theta; \Gamma; \Delta, b:\upsilon'']{\tau'}{\kappa'}$\\
    \oo By induction, $\judgeA[\Theta; \Gamma; \Delta]{[r/b]p}{(\forallsort{\beta}{\kappa'}{\upsilon})}$\\
    \oo By induction, $\judgeA[\Theta; \Gamma; \Delta]{\tau'}{\kappa'}$\\
    \oo By rule, $\judgeA[\Theta; \Gamma; \Delta]{[r/b](p\;[\tau'])}{{[\tau/\alpha, [\tau/\alpha]\tau'/\beta]}\upsilon}$\\
  \end{tabbedproof}

  For semantics, consider $\interp{\judgeA[\Theta; \Gamma; \Delta]{[r/b](p\;[\tau'])}{\upsilon}}\;\theta\;\gamma\;\delta$
  \begin{eqnproof}
    \eline{\begin{array}{l}
             (\interp{\judgeA[\Theta; \Gamma; \Delta]{[r/b]p}{(\forallsort{\beta}{\kappa'}{\upsilon})}}\;\theta\;\gamma\;\delta) \\
             \;\;(\interp{\judgeA[\Theta; \Gamma; \Delta]{\tau'}{\kappa'}}\;\theta\;\gamma\;\delta) 
      \end{array}}
    {Semantics}
    \eline{\begin{array}{l}
             (\interp{\judgeA[\Theta; \Gamma; \Delta, b:\upsilon'']{p}{\forallsort{\beta}{\kappa'}{\upsilon}}}\;\theta\;\gamma\;(\delta, \interp{\judgeA{r}{\upsilon''}}\;\theta\;\gamma\;\delta)) \\
             \;\;(\interp{\judgeA[\Theta; \Gamma; \Delta, b:\upsilon'']{\tau'}{\kappa'}}\;\theta\;\gamma\;(\delta, \interp{\judgeA{r}{\upsilon''}}\;\theta\;\gamma\;\delta)) 
      \end{array}}
    {Induction}
    \eline{\interp{\judgeA[\Theta; \Gamma; \Delta, b:\upsilon'']{p\;[\tau']}{\upsilon}}\;\theta\;\gamma\;(\delta, \interp{\judgeA{r}{\upsilon''}}\;\theta\;\gamma\;\delta)}
          {Semantics}
  \end{eqnproof}


\item Case \textsc{TConst}:

  First, the syntax: 
  \begin{tabbedproof}
    \oo By inversion, $\judgectx{\Theta}{\Gamma}$ \\
    \oo By inversion, $\judgectx{\Theta}{\Delta, b:\upsilon''}$ \\
    \oo By inversion, $\judgectx{\Theta}{\Delta}$ \\
    \oo By rule, $\judgeA[\Theta; \Gamma; \Delta]{c}{\assert}$
  \end{tabbedproof}

  For semantics consider $\interp{\judgeA[\Theta; \Gamma; \Delta]{[r/b]c}{\assert}}\;\theta;\gamma\;\delta$ 
  \begin{eqnproof}
    \eline{\interp{c}^0}{Semantics}
    \eline{\interp{\judgeA[\Theta; \Gamma; \Delta, b:\upsilon'']{c}{\assert}}\;\theta;\gamma\;(\delta, \interp{\judgeA{r}{\upsilon''}}\;\theta\;\gamma\;\delta)}
          {Semantics}
  \end{eqnproof}

\item Case \textsc{TBinary}: $\judgeA[\Theta; \Gamma; \Delta, b:\upsilon'']{p \oplus q}{\assert}$
  
  First, the syntax:
  \begin{tabbedproof}
    \oo By inversion, $\judgeA[\Theta; \Gamma; \Delta, b:\upsilon'']{p}{\assert}$\\
    \oo By inversion, $\judgeA[\Theta; \Gamma; \Delta, b:\upsilon'']{q}{\assert}$\\
    \oo By induction, $\judgeA[\Theta; \Gamma; \Delta]{[r/b]p}{\assert}$\\
    \oo By induction, $\judgeA[\Theta; \Gamma; \Delta]{[r/b]q}{\assert}$\\
    \oo By rule, $\judgeA[\Theta; \Gamma; \Delta]{[r/b]p \oplus [r/b]q}{\assert}$\\
    \oo By subst def, $\judgeA[\Theta; \Gamma; \Delta]{[r/b](p \oplus q)}{\assert}$\\
  \end{tabbedproof}

  For semantics, consider $\interp{\judgeA[\Theta; \Gamma; \Delta]{[r/b](p \oplus q)}{\assert}}\;\theta\;\gamma\;\delta$
  \begin{eqnproof}
    \eline{\begin{array}{l}
             (\interp{\judgeA[\Theta; \Gamma; \Delta]{[r/b]p}{\assert}}\;\theta\;\gamma\;\delta) \;\interp{\oplus}^2 \\
             \;\;(\interp{\judgeA[\Theta; \Gamma; \Delta]{[r/b]q}{\assert}}\;\theta\;\gamma\;\delta) 
      \end{array}}
    {Semantics}
    \eline{\begin{array}{l}
             (\interp{\judgeA[\Theta; \Gamma; \Delta, b:\upsilon'']{p}{\assert}}\;\theta\;\gamma\;(\delta, \interp{\judgeA{r}{\upsilon''}}\;\theta\;\gamma\;\delta)) \interp{\oplus}^2 \\
             \;\;(\interp{\judgeA[\Theta; \Gamma; \Delta, b:\upsilon'']{q}{\assert}}\;\theta\;\gamma\;(\delta, \interp{\judgeA{r}{\upsilon''}}\;\theta\;\gamma\;\delta)) 
      \end{array}}
    {Induction}
    \eline{\interp{\judgeA[\Theta; \Gamma; \Delta, b:\upsilon'']{p \oplus q}{\assert}}\;\theta\;\gamma\;(\delta, \interp{\judgeA{r}{\upsilon''}}\;\theta\;\gamma\;\delta)}
          {Semantics}
  \end{eqnproof}

\item Case \textsc{TQuantify1}: $\judgeA[\Theta; \Gamma; \Delta, b:\upsilon'']{Q u:\upsilon.\;p}{\assert}$
  
  First, the syntax:
  \begin{tabbedproof}
    \oo By inversion, $\judgeA[\Theta; \Gamma; \Delta, b:\upsilon'', u:\upsilon]{p}{\assert}$ \\
    \oo By induction, $\judgeA[\Theta; \Gamma; \Delta, u:\upsilon]{[r/b]p}{\assert}$ \\
    \oo By rule, $\judgeA[\Theta; \Gamma; \Delta]
                         {Q u:\upsilon.\;[r/b]p}{\assert}$ \\
    \oo By def of subst, $\judgeA[\Theta; \Gamma; \Delta]
                                 {[r/b](Q u:\upsilon.\;p)}{\assert}$ 
  \end{tabbedproof}

  For semantics, consider
  $\interp{\judgeA[\Theta; \Gamma; \Delta]
                  {[r/b](Q u:\upsilon.\;p)}{\assert}}\;\theta\;\gamma\;\delta$ 
  \begin{eqnproof}
    \eline{\begin{array}{l}
            \interp{Q}_{v \in \interp{\judgeSort{\upsilon}}\;\theta} \\
            \;\;\interp{\judgeA[\Theta;\Gamma;\Delta, u:\upsilon]{[r/b]p}{\assert}}\;\theta\,\gamma\,(\delta,v)
           \end{array}}
          {Semantics}
    \eline{\begin{array}{l}
            \interp{Q}_{v \in \interp{\judgeSort[\Theta]{\upsilon}}\;\theta} \\
            \;\;\interp{\judgeA[\Theta;\Gamma; \Delta, b:\upsilon'', u:\upsilon]{p}{\assert}}\;\theta\,\gamma\,(\delta,\interp{\judgeA{r}{\upsilon''}}\;\theta\;\gamma\;\delta, v)
           \end{array}}
          {Induction}
    \eline{\interp{\judgeA[\Theta, \alpha:\kappa;\Gamma;\Delta]{Q u:\upsilon.\;p}{\assert}}\;\theta\,\gamma\,(\delta, \interp{\judgeA{r}{\upsilon''}}\;\theta\;\gamma\;\delta)}
          {Semantics}
  \end{eqnproof}

\item Case \textsc{TQuantify2}: $\judgeA[\Theta; \Gamma; \Delta, b:\upsilon'']{Q x:A.\;p}{\assert}$
  
  First, the syntax:
  \begin{tabbedproof}
    \oo By inversion, $\judgeA[\Theta; \Gamma, x:A; \Delta, b:\upsilon'']{p}{\assert}$ \\
    \oo By induction, $\judgeA[\Theta; \Gamma, x:A; \Delta]{[r/b]p}{\assert}$ \\
    \oo By rule, $\judgeA[\Theta; \Gamma; \Delta]
                         {Q x:A.\;[r/b]p}{\assert}$ \\
    \oo By def of subst, $\judgeA[\Theta; \Gamma; \Delta]
                                 {[r/b](Q x:A.\;p)}{\assert}$ 
  \end{tabbedproof}

  For semantics, consider
  $\interp{\judgeA[\Theta; \Gamma; \Delta]
                  {[r/b](Q x:A.\;p)}{\assert}}\;\theta\;\gamma\;\delta$ 
  \begin{eqnproof}
    \eline{\begin{array}{l}
            \interp{Q}_{v \in \interp{\judgeSort{A}}\;\theta} \\
            \;\;\interp{\judgeA[\Theta;\Gamma, x:A;\Delta]{[r/b]p}{\assert}}\;\theta\,(\gamma,v)\,\delta
           \end{array}}
          {Semantics}
    \eline{\begin{array}{l}
            \interp{Q}_{v \in \interp{\judgeSort[\Theta]{A}}\;\theta} \\
            \;\;\interp{\judgeA[\Theta;\Gamma, x:A;\Delta, u:\upsilon'']{p}{\assert}}\;\theta\,(\gamma, v)\,(\delta, \interp{\judgeA{r}{\upsilon''}}\;\theta\;\gamma\;\delta)
           \end{array}}
          {Induction}
    \eline{\interp{\judgeA[\Theta;\Gamma; \Delta, b:\upsilon'']{Q x:A.\;p}{\assert}}\;\theta\,\gamma\,(\delta, \interp{\judgeA{r}{\upsilon''}}\;\theta\;\gamma\;\delta)}
          {Semantics}
  \end{eqnproof}
  Here, we make use of the fact that $x$ is not free in $e''$, and we silently permute the context as 
  needed. 

\item Case \textsc{TQuantify3}: $\judgeA[\Theta; \Gamma; \Delta, b:\upsilon'']{Q \beta:\kappa'.\;p}{\assert}$
  
  First, the syntax:
  \begin{tabbedproof}
    \oo By inversion, $\judgeA[\Theta, \beta:\kappa'; \Gamma; \Delta, b:\upsilon'']{p}{\assert}$ \\
    \oo By induction, $\judgeA[\Theta, \beta:\kappa'; \Gamma; \Delta]{[r/b]p}{\assert}$ \\
    \oo By rule, $\judgeA[\Theta; \Gamma; \Delta]
                         {Q \beta:\kappa'.\;[r/b]p}{\assert}$ \\
    \oo By def of subst, $\judgeA[\Theta; \Gamma; \Delta]
                                 {[r/b](Q \beta:\kappa'.\;p)}{\assert}$ 
  \end{tabbedproof}

  For semantics, consider
  $\interp{\judgeA[\Theta; \Gamma; \Delta]
                  {[r/b](Q \beta:\kappa'.\;p)}{\assert}}\;\theta\;\gamma\;\delta$ 
  \begin{eqnproof}
    \eline{\begin{array}{l}
            \interp{Q}_{\tau' \in \interp{\judgeSort{\kappa'}}\;\theta} \\
            \;\;\interp{\judgeA[\Theta, \beta:\kappa';\Gamma;\Delta]{[r/b]p}{\assert}}\;(\theta,\tau')\,\gamma\,\delta
           \end{array}}
          {Semantics}
    \eline{\begin{array}{l}
            \interp{Q}_{\tau' \in \interp{\judgeSort[\Theta, \alpha:\kappa]{\kappa'}}\;\theta} \\
            \;\;\interp{\judgeA[\Theta, \beta:\kappa';\Gamma; \Delta, b:\upsilon'']{p}{\assert}}\;(\theta, \tau)\,\gamma\,(\delta, \interp{\judgeA{r}{\upsilon''}}\;\theta\;\gamma\;\delta)
           \end{array}}
          {Induction}
    \eline{\interp{\judgeA[\Theta;\Gamma;\Delta, b:\upsilon'']{Q \beta:\kappa'.\;p}{\assert}}\;\theta\;\gamma\;(\delta, \interp{\judgeA{r}{\upsilon''}}\;\theta\;\gamma\;\delta) }
          {Semantics}
  \end{eqnproof}
  In this case we silently use the fact that $\beta$ does not occur free in $e''$ or $B$.

\item Case \textsc{TEqual}: $\judgeA[\Theta; \Gamma; \Delta, b:\upsilon'']{p =_\omega q}{\assert}$
  
  First, the syntax:
  \begin{tabbedproof}
    \oo By inversion, $\judgeA[\Theta; \Gamma; \Delta, b:\upsilon'']{p}{\omega}$ \\
    \oo By inversion, $\judgeA[\Theta; \Gamma; \Delta, b:\upsilon'']{q}{\omega}$ \\
    \oo By inversion, $\judgeSort[\Theta]{\omega}$ \\
    \oo By induction, $\judgeA[\Theta; \Gamma; \Delta]{[r/b]p}{\omega}$ \\
    \oo By induction, $\judgeA[\Theta; \Gamma; \Delta]{[r/b]q}{\omega}$ \\
    \oo By rule, $\judgeA[\Theta; \Gamma; \Delta]{[r/b](p =_\omega q)}{\assert}$
  \end{tabbedproof}

  For the semantics, consider $\interp{\judgeA[\Theta; \Gamma; \Delta]{[r/b](p =_\omega q)}{\assert}}\;\theta\;\gamma\;\delta$
  \begin{eqnproof}
    \eline{\mbox{if }\interp{[r/b]p}\theta\;\gamma\;\delta = \interp{[r/b]q}\theta\;\gamma\;\delta\mbox{ then }\top \mbox{ else } \bot}
          {Semantics}
    \eline{\mbox{if }\interp{p}\;\theta\;(\gamma, \interp{e''}\theta\;\gamma)\;\delta = \interp{q}\;\theta\;\gamma\;(\delta,  \interp{r}\theta\;\gamma\;\delta)\mbox{ then }\top \mbox{ else } \bot}
          {Induction}
    \eline{\interp{\judgeA[\Theta; \Gamma; \Delta, b:\upsilon'']{(p =_\omega q)}{\assert}}\;\theta\;\gamma\;(\delta, \interp{\judgeA{r}{\upsilon''}}\;\theta\;\gamma\;\delta)}
          {Semantics}
  \end{eqnproof}

\item Case \textsc{TPointsto}: $\judgeA[\Theta; \Gamma; \Delta, b:\upsilon'']{e \pointsto_A e'}{\assert}$

  First, the syntax:
  \begin{tabbedproof}
    \oo By inversion, $\judgeA[\Theta; \Gamma; \Delta, b:\upsilon'']{e}{\reftype{A}}$ \\
    \oo By inversion, $\judgeA[\Theta; \Gamma; \Delta, b:\upsilon'']{e'}{A}$ \\
    \oo By induction, $\judgeA[\Theta; \Gamma; \Delta]{[r/b]e}{\reftype{A}}$ \\
    \oo By induction, $\judgeA[\Theta; \Gamma; \Delta]{[r/b]e'}{A}$ \\
    \oo By rule, $\judgeA[\Theta; \Gamma; \Delta]{[r/b](e \pointsto_A e')}{\assert}$
  \end{tabbedproof}

  For the semantics, consider $\interp{\judgeA[\Theta; \Gamma; \Delta]{[r/b](e \pointsto_A e')}{\assert}}\;\theta\;\gamma\;\delta$
  \begin{eqnproof}
    \eline{\begin{array}{l}
           \interp{\judgeE[\Theta]{\Gamma}{[r/b]e}{\reftype{A}}}\;\theta\;\gamma \\ 
           \pointsto \\
           \interp{\judgeE[\Theta]{\Gamma}{[r/b]e'}{A}}\;\theta\;\gamma
           \end{array}}
          {Semantics}
    \eline{\interp{\judgeA[\Theta; \Gamma; \Delta, b:\upsilon'']{e \pointsto_A e'}{\assert}}\;\theta\;\gamma\;(\delta, \interp{\judgeA{r}{\upsilon''}}\;\theta\;\gamma\;\delta)}
          {Semantics}
  \end{eqnproof}

\item Case \textsc{TEqSort}: $\judgeA[\Theta; \Gamma; \Delta, b:\upsilon'']{p}{\omega}$
  
  First, the syntax:
  \begin{tabbedproof}
    \oo By inversion, $\judgeSortEq{\omega}{\omega'}$ \\
    \oo By inversion, $\judgeA[\Theta; \Gamma; \Delta, b:\upsilon'']{p}{\omega'}$ \\
    \oo By induction, $\judgeA[\Theta; \Gamma; \Delta]{[r/b]p}{\omega'}$ \\
    \oo By rule, $\judgeA[\Theta; \Gamma; \Delta]{[r/b]p}{\omega}$ \\
  \end{tabbedproof}

  For the semantics, consider $\interp{\judgeA[\Theta; \Gamma; \Delta]{[r/b]p}{\omega}}\;\theta\;\gamma\;\delta$ \\
  \begin{eqnproof}
    \eline{\interp{\judgeA[\Theta; \Gamma; \Delta]{[r/b]p}{\omega'}}\;\theta\;\gamma\;\delta}
          {Semantics}
    \eline{\interp{\judgeA[\Theta; \Gamma; \Delta, b:\upsilon'']{p}{\omega'}}\;\theta\;\gamma\;(\delta, \interp{\judgeA{r}{\upsilon''}}\;\theta\;\gamma\;\delta)}
          {Induction}
    \eline{\interp{\judgeA[\Theta; \Gamma; \Delta, b:\upsilon'']{p}{\omega}}\;\theta\;\gamma\;(\delta, \interp{\judgeA{r}{\upsilon''}}\;\theta\;\gamma\;\delta)}
          {Semantics}
  \end{eqnproof}

\item Case \textsc{TSpec}: $\judgeA[\Theta; \Gamma; \Delta, b:\upsilon'']{\validprop{S}}{\assert}$:
  
  First, the syntax:
  \begin{tabbedproof}
    \oo By inversion, $\judgeS[\Theta; \Gamma; \Delta, b:\upsilon'']{S}$ \\
    \oo By mutual induction $\judgeS[\Theta; \Gamma; \Delta]{[r/b]S}$\\
    \oo By rule, $\judgeA[\Theta; \Gamma; \Delta]{[r/b]\validprop{S}}{\assert}$
  \end{tabbedproof}

  For the semantics, consider $\interp{\judgeA[\Theta; \Gamma; \Delta]{[r/b]\validprop{S}}{\assert}}\;\theta\;\gamma\;\delta$ 
  \begin{eqnproof}
    \eline{\mbox{if } \interp{[r/b]S}\;\theta\;\gamma\;\delta = \top \mbox{ then } \top \mbox{ else } \bot}
          {Semantics}
    \eline{\mbox{if } \interp{S}\;\theta\;\gamma\;(\delta, \interp{\judgeA{r}{\upsilon''}}\;\theta\;\gamma\;\delta) = \top \mbox{ then } \top \mbox{ else } \bot}
          {Induction}
    \eline{\interp{\judgeA[\Theta; \Gamma; \Delta, b:\upsilon'']{\validprop{S}}{\assert}}\; \theta\;\gamma\;(\delta, \interp{\judgeA{r}{\upsilon''}}\;\theta\;\gamma\;\delta)}
          {Semantics}
  \end{eqnproof}

\item Case \textsc{SpecTriple}: $\judgeS[\Theta; \Gamma; \Delta, b:\upsilon'']{\spec{p}{c}{a:A}{q}}$
  
  First, the syntax:
  \begin{tabbedproof}
    \oo By inversion, $\judgeA[\Theta; \Gamma; \Delta, b:\upsilon'']{p}{\assert}$ \\
    \oo By inversion, $\judgeA[\Theta; \Gamma; \Delta, b:\upsilon'']{\comp{c}}{\monad{A}}$ \\
    \oo By inversion, $\judgeA[\Theta; \Gamma; \Delta, b:\upsilon'', a:A]{q}{\assert}$ \\
    \oo By induction, $\judgeA[\Theta; \Gamma; \Delta]{[r/b]p}{\assert}$ \\
    \oo By induction, $\judgeA[\Theta; \Gamma; \Delta]{\comp{[r/b]c}}{\monad{A}}$ \\
    \oo By induction, $\judgeA[\Theta; \Gamma; \Delta, a:A]{[r/b]q}{\assert}$ \\
    \oo By rule, $\judgeS[\Theta; \Gamma; \Delta]{[r/b](\spec{p}{c}{a:A}{q})}$
  \end{tabbedproof}

  For the semantics, consider $\interp{\judgeS[\Theta; \Gamma; \Delta]{[r/b](\spec{p}{c}{a:A}{q})}}\;\theta;\gamma\;\delta$
  \begin{eqnproof}
    \eline{\begin{array}{l}
           \{\interp{\judgeA[\Theta; \Gamma; \Delta]{[r/b]p}{\assert}}\;\theta\;\gamma\;\delta\} \\
           \interp{\judgeC{\Gamma}{[r/b]c}{A}}\;\theta\;\gamma\;\delta \\
           \{v.\;\interp{\judgeA[\Theta; \Gamma, a:A; \Delta]{[r/b]q}{\assert}}\;\theta\;(\gamma,v)\;\delta\} 
           \end{array}}
          {Semantics}
    \eline{\begin{array}{l}
           \{\interp{\judgeA[\Theta, \Gamma; \Delta]{p}{\assert}}\;\theta\;\gamma\;(\delta, \interp{\judgeA{r}{\upsilon''}}\;\theta\;\gamma\;\delta)\} \\
           \interp{\judgeC[\Theta]{\Gamma}{c}{A}}\;\theta\;\gamma \\
           \{v.\;\interp{\judgeA[\Theta; \Gamma, a:A; \Delta, b:\upsilon'']{q}{\assert}}\;\theta\;(\gamma, v)\;(\delta, \interp{\judgeA{r}{\upsilon''}}\;\theta\;\gamma\;\delta)\} 
           \end{array}}
          {Induction}
    \eline{\interp{\judgeS[\Theta; \Gamma; \Delta, b:\upsilon'']{(\spec{p}{c}{a:A}{q})}}\;\theta;\gamma\;(\delta, \interp{\judgeA{r}{\upsilon''}}\;\theta\;\gamma\;\delta)}
          {Semantics}
  \end{eqnproof}
  The correctness of the application of $\gamma$ and $\delta$ follows from the equations for
  contexts under substitution. 

\item Case \textsc{SpecMTriple}: $\judgeS[\Theta; \Gamma; \Delta, b:\upsilon'']{\mspec{p}{e}{a:A}{q}}$
  
  First, the syntax:
  \begin{tabbedproof}
    \oo By inversion, $\judgeA[\Theta; \Gamma; \Delta, b:\upsilon'']{p}{\assert}$ \\
    \oo By inversion, $\judgeA[\Theta; \Gamma; \Delta, b:\upsilon'']{e}{\monad{A}}$ \\
    \oo By inversion, $\judgeA[\Theta; \Gamma; \Delta, b:\upsilon'', a:A]{q}{\assert}$ \\
    \oo By induction, $\judgeA[\Theta; \Gamma; \Delta]{[r/b]p}{\assert}$ \\
    \oo By induction, $\judgeA[\Theta; \Gamma; \Delta]{[r/b]e}{\monad{A}}$ \\
    \oo By induction, $\judgeA[\Theta; \Gamma; \Delta, a:A]{[r/b]q}{\assert}$ \\
    \oo By rule, $\judgeS[\Theta; \Gamma; \Delta]{[r/b](\mspec{p}{e}{a:A}{q})}$
  \end{tabbedproof}

  For the semantics, consider $\interp{\judgeS[\Theta; \Gamma; \Delta]{[r/b](\mspec{p}{e}{a:A}{q})}}\;\theta;\gamma\;\delta$
  \begin{eqnproof}
    \eline{\begin{array}{l}
           \{\interp{\judgeA[\Theta; \Gamma; \Delta]{[r/b]p}{\assert}}\;\theta\;\gamma\;\delta\} \\
           \interp{\judgeE{\Gamma}{[r/b]e}{\monad{A}}}\;\theta\;\gamma \\
           \{v.\;\interp{\judgeA[\Theta; \Gamma, a:A; \Delta]{[r/b]q}{\assert}}\;\theta\;(\gamma,v)\;\delta\} 
           \end{array}}
          {Semantics}
    \eline{\begin{array}{l}
           \{\interp{\judgeA[\Theta; \Gamma; \Delta, b:\upsilon'']{p}{\assert}}\;\theta\;\gamma\;(\delta, \interp{\judgeA{r}{\upsilon''}}\;\theta\;\gamma\;\delta)\} \\
           \interp{\judgeE{\Gamma}{e}{\monad{A}}}\;\theta\;\gamma \\
           \{v.\;\interp{\judgeA[\Theta; \Gamma, a:A; \Delta, b:\upsilon'']{q}{\assert}}\;\theta\;(\gamma, v)\;(\delta, \interp{\judgeA{r}{\upsilon''}}\;\theta\;\gamma\;\delta)\} 
           \end{array}}
          {Induction}
    \eline{\interp{\judgeS[\Theta; \Gamma; \Delta, b:\upsilon'']{(\mspec{p}{e}{a:A}{q})}}\;\theta;\gamma\;(\delta, \interp{\judgeA{r}{\upsilon''}}\;\theta\;\gamma\;\delta)}
          {Semantics}
  \end{eqnproof}

\item Case \textsc{SpecQuantify1}: $\judgeS[\Theta; \Gamma; \Delta, b:\upsilon'']{Q u:\upsilon.\;S}$
  
  First, the syntax:
  \begin{tabbedproof}
    \oo By inversion, $\judgeS[\Theta; \Gamma; \Delta, b:\upsilon'', u:\upsilon]{S}$ \\
    \oo By induction, $\judgeS[\Theta; \Gamma; \Delta, u:\upsilon]{[r/b]S}$ \\
    \oo By rule, $\judgeS[\Theta; \Gamma; \Delta]
                         {Q u:\upsilon.\;[r/b]S}$ \\
    \oo By def of subst, $\judgeS[\Theta; \Gamma; \Delta]
                                 {[r/b](Q u:\upsilon.\;S)}$ 
  \end{tabbedproof}

  For semantics, consider
  $\interp{\judgeS[\Theta; \Gamma; \Delta]
                  {[r/b](Q u:\upsilon.\;S)}}\;\theta\;\gamma\;\delta$ 
  \begin{eqnproof}
    \eline{\begin{array}{l}
            \interp{Q}_{v \in \interp{\judgeSort{\upsilon}}\;\theta} \\
            \;\;\interp{\judgeS[\Theta;\Gamma;\Delta, u:\upsilon]{[r/b]S}}\;\theta\,\gamma\,(\delta,v)
           \end{array}}
          {Semantics}
    \eline{\begin{array}{l}
            \interp{Q}_{v \in \interp{\judgeSort[\Theta]{\upsilon}}\;\theta} \\
            \;\;\interp{\judgeS[\Theta;\Gamma; \Delta, b:\upsilon'', u:\upsilon]{S}}\;\theta\,\gamma\;(\delta, \interp{\judgeA{r}{\upsilon''}}\;\theta\;\gamma\;\delta,v)
           \end{array}}
          {Induction}
    \eline{\interp{\judgeS[\Theta;\Gamma; \Delta, b:\upsilon'']{Q u:\upsilon.\;S}}\;\theta\,\gamma\,(\delta, \interp{\judgeA{r}{\upsilon''}}\;\theta\;\gamma\;\delta)}
          {Semantics}
  \end{eqnproof}

\item Case \textsc{SpecQuantify2}: $\judgeS[\Theta; \Gamma; \Delta, b:\upsilon'']{Q x:A.\;S}$
  
  First, the syntax:
  \begin{tabbedproof}
    \oo By inversion, $\judgeS[\Theta; \Gamma, x:A; \Delta, b:\upsilon'']{S}$ \\
    \oo By induction, $\judgeS[\Theta; \Gamma, x:A; \Delta]{[r/b]S}$ \\
    \oo By rule, $\judgeS[\Theta; \Gamma; \Delta]
                         {Q x:A.\;[r/b]S}$ \\
    \oo By def of subst, $\judgeS[\Theta; \Gamma; \Delta]
                                 {[r/b](Q x:A.\;S)}$ 
  \end{tabbedproof}

  For semantics, consider
  $\interp{\judgeS[\Theta; \Gamma; \Delta]
                  {[r/b](Q x:A.\;S)}}\;\theta\;\gamma\;\delta$ 
  \begin{eqnproof}
    \eline{\begin{array}{l}
            \interp{Q}_{v \in \interp{\judgeSort{A}}\;\theta} \\
            \;\;\interp{\judgeS[\Theta;\Gamma, x:A;\Delta]{[r/b]S}}\;\theta\,(\gamma,v)\,\delta
           \end{array}}
          {Semantics}
    \eline{\begin{array}{l}
            \interp{Q}_{v \in \interp{\judgeSort[\Theta]{A}}\;\theta} \\
            \;\;\interp{\judgeS[\Theta;\Gamma, x:A;\Delta, b:\upsilon'']{S}}\;\theta\,(\gamma,v)\,(\delta, \interp{\judgeA{r}{\upsilon''}}\;\theta\;\gamma\;\delta)
           \end{array}}
          {Induction}
    \eline{\interp{\judgeS[\Theta;\Gamma;\Delta, b:\upsilon'']{Q x:A.\;S}}\;\theta\,\gamma\,(\delta, \interp{\judgeA{r}{\upsilon''}}\;\theta\;\gamma\;\delta)}
          {Semantics}
  \end{eqnproof}

\item Case \textsc{SpecQuantify3}: $\judgeS[\Theta; \Gamma; \Delta, b:\upsilon'']{Q \beta:\kappa'.\;S}$
  
  First, the syntax:
  \begin{tabbedproof}
    \oo By inversion, $\judgeS[\Theta, \beta:\kappa'; \Gamma; \Delta, b:\upsilon'']{S}$ \\
    \oo By induction, $\judgeS[\Theta, \beta:\kappa'; \Gamma; \Delta]{[r/b]S}$ \\
    \oo By rule, $\judgeS[\Theta; \Gamma; \Delta]
                         {Q \beta:\kappa'.\;[r/b]S}$ \\
    \oo By def of subst, $\judgeS[\Theta; \Gamma; \Delta]
                                 {[r/b](Q \beta:\kappa'.\;S)}$ 
  \end{tabbedproof}

  For semantics, consider
  $\interp{\judgeS[\Theta; \Gamma; \Delta]
                  {[r/b](Q \beta:\kappa'.\;S)}}\;\theta\;\gamma\;\delta$ 
  \begin{eqnproof}
    \eline{\begin{array}{l}
            \interp{Q}_{\tau' \in \interp{\judgeSort{\kappa'}}\;\theta} \\
            \;\;\interp{\judgeS[\Theta, \beta:\kappa';\Gamma;\Delta]{[r/b]S}}\;(\theta,\tau')\,\gamma\,\delta
           \end{array}}
          {Semantics}
    \eline{\begin{array}{l}
            \interp{Q}_{\tau' \in \interp{\judgeSort[\Theta, \alpha:\kappa]{\kappa'}}\;\theta} \\
            \;\;\interp{\judgeS[\Theta, \beta:\kappa';\Gamma;\Delta, b:\upsilon'']{S}}\;(\theta, \tau')\,\gamma\;(\delta, \interp{\judgeA{r}{\upsilon''}}\;\theta\;\gamma\;\delta,v)
           \end{array}}
          {Induction}
    \eline{\interp{\judgeS[\Theta;\Gamma;\Delta, b:\upsilon'']{Q \beta:\kappa'.\;S}}\;\theta\,\gamma\,(\delta, \interp{\judgeA{r}{\upsilon''}}\;\theta\;\gamma\;\delta)}
          {Semantics}
  \end{eqnproof}


\item Case \textsc{SpecBinary}: $\judgeS[\Theta; \Gamma; \Delta, b:\upsilon'']{S \oplus S'}$
  
  First, the syntax:
  \begin{tabbedproof}
    \oo By inversion, $\judgeS[\Theta; \Gamma; \Delta, b:\upsilon'']{S}$\\
    \oo By inversion, $\judgeS[\Theta; \Gamma; \Delta, b:\upsilon'']{S'}$\\
    \oo By induction, $\judgeS[\Theta; \Gamma; \Delta]{[r/b]S}$\\
    \oo By induction, $\judgeS[\Theta; \Gamma; \Delta]{[r/b]S'}$\\
    \oo By rule, $\judgeS[\Theta; \Gamma; \Delta]{[r/b]S \oplus [r/b]S'}$\\
    \oo By subst def, $\judgeS[\Theta; \Gamma; \Delta]{[r/b](S \oplus S')}$\\
  \end{tabbedproof}

  For semantics, consider $\interp{\judgeS[\Theta; \Gamma; \Delta]{[r/b](S \oplus S')}}\;\theta\;\gamma\;\delta$
  \begin{eqnproof}
    \eline{\begin{array}{l}
             (\interp{\judgeS[\Theta; \Gamma; \Delta]{[r/b]S}}\;\theta\;\gamma\;\delta) \;\interp{\oplus} \\
             \;\;(\interp{\judgeS[\Theta; \Gamma; \Delta]{[r/b]S'}}\;\theta\;\gamma\;\delta) 
      \end{array}}
    {Semantics}
    \eline{\begin{array}{l}
             (\interp{\judgeS[\Theta; \Gamma; \Delta, b:\upsilon'']{S}}\;\theta\;\gamma\;(\delta, \interp{\judgeA{r}{\upsilon''}}\;\theta\;\gamma\;\delta)) \interp{\oplus} \\
             \;\;(\interp{\judgeS[\Theta; \Gamma; \Delta, b:\upsilon'']{S'}}\;\theta\;\gamma\;(\delta, \interp{\judgeA{r}{\upsilon''}}\;\theta\;\gamma\;\delta)) 
      \end{array}}
    {Induction}
    \eline{\interp{\judgeS[\Theta; \Gamma; \Delta, b:\upsilon'']{S \oplus S'}}\;\theta\;\gamma\;(\delta, \interp{\judgeA{r}{\upsilon''}}\;\theta\;\gamma\;\delta)}
          {Semantics}
  \end{eqnproof}

\item Case \textsc{TSpec}: $\judgeS[\Theta; \Gamma; \Delta, b:\upsilon'']{\setof{p}}$:
  
  First, the syntax:
  \begin{tabbedproof}
    \oo By inversion, $\judgeA[\Theta; \Gamma; \Delta, b:\upsilon'']{p}{\assert}$ \\
    \oo By mutual induction $\judgeA[\Theta; \Gamma; \Delta]{[r/b]p}{\assert}$\\
    \oo By rule, $\judgeS[\Theta; \Gamma; \Delta]{[r/b]\setof{p}}$
  \end{tabbedproof}

  For the semantics, consider $\interp{\judgeS[\Theta; \Gamma; \Delta]{[r/b]\setof{p}}}\;\theta\;\gamma\;\delta$ 
  \begin{eqnproof}
    \eline{\mbox{if } \interp{[r/b]p}\;\theta\;\gamma\;\delta = \top \mbox{ then } \top \mbox{ else } \bot}
          {Semantics}
    \eline{\mbox{if } \interp{p}\;\theta\;\gamma\;(\delta, \interp{\judgeA{r}{\upsilon''}}\;\theta\;\gamma\;\delta) = \top \mbox{ then } \top \mbox{ else } \bot}
          {Induction}
    \eline{\interp{\judgeS[\Theta; \Gamma; \Delta, b:\upsilon'']{\setof{p}}}\; \theta\;\gamma\;(\delta, \interp{\judgeA{r}{\upsilon''}}\;\theta\;\gamma\;\delta)}
          {Semantics}
  \end{eqnproof}

\end{enumerate}

\end{proof}




\subsection{Soundness of Assertion Logic Axioms}

\begin{lemma}{(Getting Specs Into Assertions)} 
  If $\judgeS{S}$ is valid, then $\judgeA{\validprop{S}}{\assert}$ is valid. 
\end{lemma}
\begin{proof}
  Assume we have a suitable environment $\delta$. By hypothesis we know
$\interp{\judgeS{S}}\;\delta$ $=$ $\top$ in the specification lattice. However,
we know the interpretation of $\interp{\judgeA{\validprop{S}}{\assert}}\;\delta$ is 
equal to $\IfThenElse{\interp{\judgeS{S}}\;\delta = \top}{\top}{\bot}$, so it 
follows that $\interp{\judgeA{\validprop{S}}{\assert}}\;\delta = \top$ in the assertion
lattice. Hence it is true for all substitutions, and is therefore valid. 
\end{proof}\\

\begin{lemma}{(Getting Assertions out of Specs)}
If $\judgeA{p}{\assert}$, then $\judgeA{\validprop{\setof{p}} \implies p}{\assert}$ is valid. 
\end{lemma}
\begin{proof}
\begin{tabbedproof}
\oo Assume we have $\judgeA{p}{\assert}$ and a suitable $\delta$. \\ 
\oo We want to show that $\interp{\judgeA{\validprop{\setof{p}} \implies
    p}{\assert}}\;\delta$ is equal to $\top$.  \\
\oo This is equivalent to every $h \in H$ being in 
    $\interp{\judgeA{\validprop{\setof{p}} \implies p}{\assert}}\;\delta$ \\
\oo From the semantics of implication, we want to show that \\ 
\ooo if $h \in \interp{\judgeA{\validprop{\setof{p}}}{\assert}}\;\delta$, 
     then  $h \in \interp{\judgeA{p}{\assert}}\;\delta$. \\
\oo Assume we have an $h \in \interp{\judgeA{\validprop{\setof{p}}}{\assert}}\;\delta$. \\
\ooo We know  $\IfThenElse{\interp{\judgeS{\setof{p}}}\;\delta = \top}{\top}{\bot}$ \\
\ooo Since we have an $h$ in this set,  $\interp{\judgeS{\setof{p}}}\;\delta = \top$. \\
\ooo However,  $\interp{\judgeS{\setof{p}}}\;\delta = \IfThenElse{\interp{\judgeA{p}{\assert}}\;\delta = \top}{\top}{\bot}$ \\
\ooo So $\interp{\judgeA{p}{\assert}}\;\delta = \top$. \\ 
\ooo Since this is the full set of heaps $H$, it follows that $h \in \interp{\judgeA{p}{\assert}}\;\delta$. 
\end{tabbedproof}
\end{proof}


\begin{lemma}{(Soundness of Equality)}
  If $\judgeEqA{p}{q}{\omega}$ is a valid equality, then $\judgeA{p =_\omega q}{\assert}$ is valid.
\end{lemma}
\begin{proof}
\begin{tabbedproof}
\oo Assume $\judgeEqA{p}{q}{\omega}$ is a valid equality \\
\oo Assume we have a suitable $\delta$. \\ 
\ooo Then we know that $\interp{\judgeA{p}{\assert}}\;\delta = \interp{\judgeA{q}{\omega}}\;\delta$. \\
\ooo By semantics $\interp{\judgeA{p =_\omega q}{\assert}}\;\delta = \IfThenElse{\interp{\judgeA{p}{\omega}}\;\delta = \interp{\judgeA{q}{\omega}}\;\delta}{\top}{\bot}$ \\
\ooo Therefore $\interp{\judgeA{p =_\omega q}{\assert}}\;\delta = \top$ \\
\oo Therefore $\judgeA{p =_\omega q}{\assert}$ is valid.
\end{tabbedproof}
\end{proof}

\subsection{Soundness of Program Logic Axioms}

\begin{lemma}{(Validity and Classical and Intuitionistic Implication)}
The statement that $\judgeS{S_1 \specimp S_2}$ is valid is logically equivalent to: if
for all $\delta$ and $r$, if $r \in \interp{\judgeS{S_1}}\;\delta$ then $r \in \interp{\judgeS{S_2}}\;\delta$  
\end{lemma}

\begin{proof}
  \begin{tabbedproof}
    \oo $\To$ direction:\\
    \ooo Assume $\judgeS{S_1 \specimp S_2}$ is valid\\
    \oooo We want to show for all $\delta$ and $r$, if $r \in \interp{\judgeS{S_1}}\;\delta$, then 
          $r \in \interp{\judgeS{S_2}}\;\delta$ \\
    \oooo Assume $\delta$ is an appropriate substitution, and $r$ such that $r \in \interp{\judgeS{S_1}}\;\delta$ \\
    \ooooo We want to show $r \in \interp{\judgeS{S_2}}\;\delta$ \\
    \oooooo From the hypothesis we know that $\interp{\judgeS{S_1 \specimp S_2}}\;\delta = \top$ \\
    \oooooo So we know for all $r$ and $s \worldgeq r$, that \\
    \oooooox if $s \in \interp{\judgeS{S_1}}\;\delta$ then $s \in \interp{\judgeS{S_2}}\;\delta$ \\
    \oooooo Since $r \worldgeq r$, we know if $r \in \interp{\judgeS{S_1}}\;\delta$ then $s \in \interp{\judgeS{S_2}}\;\delta$ \\
    \oooooo Since $r \in \interp{\judgeS{S_1}}\;\delta$, we know that $r \in \interp{\judgeS{S_2}}\;\delta$ \\
    \oo $\From$ direction:\\
    \ooo Assume for all $\delta$ and $r$, if $r \in \interp{\judgeS{S_1}}\;\delta$, then 
          $r \in \interp{\judgeS{S_2}}\;\delta$ \\
    \oooo We want to show for all $\delta$ that $\interp{\judgeS{S_1 \specimp S_2}}\;\delta = \top$ \\
    \oooo Assume $\delta$ is a suitable substitution \\
    \ooooo So we want to show $\interp{\judgeS{S_1 \specimp S_2}}\;\delta = \top$ \\
    \ooooo So we want to show for all $r$ and $s \worldgeq r$, that \\
    \ooooox if $s \in \interp{\judgeS{S_1}}\;\delta$ then $s \in \interp{\judgeS{S_2}}\;\delta$ \\
    \ooooo Assume $r$ and $s$ such that $s \worldgeq r$ and $s \in \interp{\judgeS{S_1}}\;\delta$ \\
    \oooooo Instantiate the quantifier in the hypothesis with $\delta$ and $s$, so we learn \\
    \oooooox if $s \in \interp{\judgeS{S_1}}\;\delta$ then $s \in \interp{\judgeS{S_2}}\;\delta$ \\
    \oooooo Since $s \in \interp{\judgeS{S_1}}\;\delta$ we can conclude $s \in \interp{\judgeS{S_2}}\;\delta$ \\
  \end{tabbedproof}
\end{proof}

\begin{lemma}{(Equivalence of the Two Forms of Triples)}
We have that $\judgeS{\spec{p}{c}{a:A}{q}}$ is valid if and only if $\judgeS{\mspec{p}{\comp{c}}{a:A}{q}}$ is valid.
\end{lemma}

\begin{proof}
\begin{tabbedproof}
\oo We want to show that for all suitable $\delta$, $\judgeS{\spec{p}{c}{a:A}{q}} = \top$  \\
\ox iff $\judgeS{\mspec{p}{\comp{c}}{a:A}{q}} = \top$  \\
\oo Now, let \\
\oox $P = \interp{\judgeA{p}{\assert}}\;\delta$ \\
\oox $E = \interp{\judgeE[\restrictkind{\Delta}]{\restricttype{\Delta}}
                  {[c]}{A}} \;(\restricttyenv{\Delta}{\delta})
                            \;(\restrictvals{\Delta}{\delta})$ \\
\oox $C = \interp{\judgeC[\restrictkind{\Delta}]{\restricttype{\Delta}}
                  {c}{A}} \;(\restricttyenv{\Delta}{\delta})
                          \;(\restrictvals{\Delta}{\delta})$ \\
\oox $Q = \semfun{v}{\interp{\judgeA[\Delta, a:A]{q}{\assert}}\;(\delta,v)}$ \\
\oo Observe that $E = C$, from the semantics of $\comp{c}$ \\
\oo $\Rightarrow$: Assume $\judgeS{\spec{p}{c}{a:A}{q}}$ is valid \\
\ooo Hence the semantic spec $\spec{P}{C}{a}{Q(a)} = \top$ \\
\ooo Hence $\spec{P}{E}{a}{Q(a)} = \top$ \\
\oo $\Leftarrow$: Assume $\judgeS{\mspec{p}{\comp{c}}{a:A}{q}}$ is valid \\
\ooo Hence the semantic spec $\spec{P}{E}{a}{Q(a)} = \top$ \\
\ooo Hence $\spec{P}{C}{a}{Q(a)} = \top$
\end{tabbedproof}
\end{proof}


\begin{lemma}{(Return Value Axiom)}
The schema $\judgeS{\spec{P}{e}{a:A}{P \land a = e}}$ is valid.
\end{lemma}

\begin{proof}
\begin{tabbedproof}
\oo We want to show that for all suitable $\delta$, $\interp{\spec{P}{e}{a:A}{P \land a = e}}\;\delta = \top$ \\
\oo Now let \\
\oox $P = \interp{\judgeA{p}{\assert}}\;\delta$ \\
\oox $C = \interp{\judgeC[\restrictkind{\Delta}]{\restricttype{\Delta}}
                  {e}{A}} \;(\restricttyenv{\Delta}{\delta})
                          \;(\restrictvals{\Delta}{\delta})$ \\
\oox $E = \interp{\judgeE[\restrictkind{\Delta}]{\restricttype{\Delta}}
                  {e}{A}} \;(\restricttyenv{\Delta}{\delta})
                          \;(\restrictvals{\Delta}{\delta})$ \\
\oox $Q = \semfun{v}{\interp{\judgeA[\Delta, a:A]{P \land a = e}{\assert}}\;(\delta,v)}$ \\
\oo We want to show that $\spec{P}{C}{a}{Q(a)} = \top$ \\
\oo So we want to show that for all $R$, we have $R \in \spec{P}{C}{a}{Q(a)}$ \\
\oo Hence it suffices to show that $\basicspec{P * R}{C}{v}{P \land \interp{a = e}\;(\delta,v) * R}$ \\
\oo So we want to show that for all $h \in P * R$, we have $C\;\mathit{Best}(\semfun{v}{P \land \interp{a = e * R}\;(\delta,v)})\;h = \bot$ \\
\oo Assume we have $h \in P * R$ \\
\ooo Next, observe that $C\;k\;h = k\;E\;h$ \\
\ooo To show that $\mathit{Best}(\semfun{a}{P \land a = e * R})\;E\;h = \bot$, \\
\oox we need to show that $k\;\;E\;h = \bot$, for every $k \in \mathit{Approx}(\semfun{a}{P \land a = e *R})$ \\
\ooo So assume $k \in \mathit{Approx}(\semfun{v}{P \land \interp{a = e}\;(\delta,v) * R})$ \\
\oooo Therefore for all $v \in \interp{A}$, and $h \in Q(v) * R$, we know $k\;v\;h = \bot$ \\
\oooo We know $E \in \interp{A}$ \\
\oooo So we need to show that given $h \in P * R$, we have $h \in Q(E) * R$ \\
\ooooo Assume we have $h \in P * R$ \\
\ooooo So we need to show $h \in (P \land \interp{a = e}\;(\delta,E)) * R$ \\
\ooooo Note that $\interp{a = e}\;(\delta,E) = \top$ if $E = E$, which is true \\
\ooooo Hence $P = (P \land \interp{a = e}\;(\delta,E))$ \\
\ooooo Hence we need to show $h \in P * R$ \\
\ooooo This is a hypothesis, so we are done \\
\end{tabbedproof}
\end{proof}


\begin{lemma}{(Assignment Axiom)}
$\judgeS{\spec{e \pointsto_A -}{e := e'}{a:\unittype}{e \pointsto_A e'}}$ is valid  
\end{lemma}
\begin{proof}
  \begin{tabbedproof}
    \oo Assume $\delta \in \interp{\judgeACtx{\Delta}}$ \\
    \ooo We want to show $\interp{\judgeS{\spec{e \pointsto_A -}{e := e'}{a:\unittype}{e \pointsto_A e'}}}\;\delta = \top$ \\
    \ooo So we want to show $\forall r.\; r \in \interp{\judgeS{\spec{e \pointsto_A -}{e := e'}{a:\unittype}{e \pointsto_A e'}}}\;\delta$ \\
    \ooo Assume $r$ \\
    \oooo We want to show $r \in \interp{\judgeS{\spec{e \pointsto_A -}{e := e'}{a:\unittype}{e \pointsto_A e'}}}\;\delta$ \\
    \oooo Let $\theta = \restricttyenv{\Delta}{\delta}$ \\
    \oooo Let $\gamma = \restrictvals{\Delta}{\delta}$ \\
    \oooo Let $P = \interp{\judgeA{e \pointsto_A -}{\assert}}\;\delta$ \\
    \oooo Let $C = \interp{\judgeE[\restrictkind{\Delta}]{\restricttype{\Delta}}{\comp{e := e'}}{\monad{A}}}\;\theta\;\gamma$ \\
    \oooo Let $E = \interp{\judgeE[\restrictkind{\Delta}]{\restricttype{\Delta}}{e}{\reftype{A}}}\;\theta\;\gamma$ \\
    \oooo Let $E' = \interp{\judgeE[\restrictkind{\Delta}]{\restricttype{\Delta}}{e'}{A}}\;\theta\;\gamma$ \\
    \oooo let $Q = \semfun{v}{\interp{\judgeA{e \pointsto_A e'}{\assert}}\;(\delta, v)}$ \\
    \oooo We want to show $r \in \spec{P}{C}{a:A}{Q(a)}$ \\
    \oooo We want to show $\forall s \sqsupseteq r.\; \basicspec{P * s}{C}{a:A}{Q(a) * s}$ \\
    \oooo Assume $s \sqsupseteq r$ \\
    \ooooo We want to show $\basicspec{P * s}{C}{a:A}{Q(a) * s}$ \\
    \ooooo We want to show $\forall h \in (P * s).\; C\;Best(\semfun{a}{Q(a) * s})\;h = \bot$ \\
    \ooooo Assume $h \in P * s$ \\
    \oooooo By semantics $h \in P * s$ means $\exists h_1, h_2.\; h_1 \in P$, $h_2 \in s$, and
            $h_1 \# h_2$ \\
    \oooooo By semantics of assertions, $h_1 \in P$ means 
            $\exists v \in \interp{\judgeWK[\restrictkind{\Delta}]{A}{\bigstar}}\;\theta.\;
               h_1 = [E:v]$ \\
    \oooooo Expanding definitions, 
             $C = \semfun{k}{\semfun{h}{k\;\unit\;\mbox{if } E\in\domain{h} \mbox{ then } [h|E:E'] \mbox{ else }\top}}$ \\
    \oooooo Therefore $C\;Best(\semfun{a}{Q(a) * s})\;h = $ \\
    \oooooox $\mbox{if }E\in\domain{h_1 \cdot h_2} \mbox{ then }Best(\semfun{a}{Q(a) * s})\;\unit\;[h_1 \cdot h_2|E:E'] \mbox{ else } \top$ \\
    \oooooo Since $E \in \domain{h_1}$, we know $C\;Best(\semfun{a}{Q(a) * s})\;h = $ \\
    \oooooox $Best(\semfun{a}{Q(a) * s})\;\unit\;[h_1 \cdot h_2|E:E']$ \\
    \oooooo Therefore we want to show $Best(\semfun{a}{Q(a) * s})\;\unit\;[h_1 \cdot h_2|E:E'] = \bot$ \\
    \oooooo So we must show $\lnot \exists k \in Approx(\semfun{a}{Q(a) * s}).\; k\;\unit\;[h_1 \cdot h_2|E'] = \top$ \\
    \oooooo So we must show $\forall k \in Approx(\semfun{a}{Q(a) * s}).\; k\unit\;[h_1\cdot h_2|E:E'] = \bot$ \\
    \oooooo Assume $k \in Approx(\semfun{a}{Q(a) * s})$ \\
    \ooooooo Since $[h_1\cdot h_2|E:E'] = [E:E']\cdot h_2$, we want 
             $k\unit\;([E:E']\cdot h_2) = \bot$ \\
    \ooooooo Now, $k \in Approx(\semfun{a}{Q(a) * s})$ means $\forall v, h \in (Q(v) * s)\; k\;v\;h = \bot$ \\
    \ooooooo Now, instantiate the quantifier with $\unit$ and $[E:E']\cdot h_2$ \\
    \ooooooo Now we must check $[E:E']\cdot h_2 \in Q(\unit) * s$ \\
    \ooooooo We will check that $[E:E'] \in Q(\unit)$ and $h_2 \in s$ \\
    \ooooooo By semantics of assertions $Q(\unit) = \setof{[E:E']}$, so $[E:E'] \in Q(\unit)$ \\
    \ooooooo From line 19, $h_2 \in s$ \\
    \ooooooo Since $[E:E']$ has the same domain as $h_1$, we know $[E:E'] \# h_2$ \\
    \ooooooo Therefore $[E:E']\cdot h_2 \in Q(\unit) * s$ \\
    \ooooooo Therefore $k\unit\;[h_1\cdot h_2|E:E'] = \bot$ \\
    \oooooo Therefore $\forall k \in Approx(\semfun{a}{Q(a) * s}).\; k\unit\;[h_1\cdot h_2|E:E'] = \bot$ \\
    \oooooo Therefore $\lnot \exists k \in Approx(\semfun{a}{Q(a) * s}).\; k\;\unit\;[h_1 \cdot h_2|E'] = \top$ \\
    \oooooo Therefore $Best(\semfun{a}{Q(a) * s})\;\unit\;[h_1 \cdot h_2|E:E'] = \bot$ \\
    \ooooo Therefore $\forall h \in (P * s).\; C\;Best(\semfun{a}{Q(a) * s})\;h = \bot$ \\
    \ooooo Therefore $\basicspec{P * s}{C}{a:A}{Q(a) * s}$ \\
    \oooo Therefore $\forall s \sqsupseteq r.\; \basicspec{P * s}{C}{a:A}{Q(a) * s}$ \\
    \oooo Therefore $r \in \spec{P}{C}{a:A}{Q(a)}$ \\
    \ooo Therefore $\forall r.\; r \in \interp{\judgeS{\spec{e \pointsto_A -}{e := e'}{a:\unittype}{e \pointsto_A e'}}}\;\delta$ \\
    \ooo Therefore $\interp{\judgeS{\spec{e \pointsto_A -}{e := e'}{a:\unittype}{e \pointsto_A e'}}}\;\delta = \top$ \\
  \end{tabbedproof}
\end{proof}



\begin{lemma}{(Allocation Axiom)}
If $\judgeS{\spec{\emp}{\newref{A}{e}}{a:\reftype{A}}{a \pointsto e}}$ is valid
\end{lemma}
\begin{proof}
  \begin{tabbedproof}
    \oo Assume $\delta \in \interp{\judgeACtx{\Delta}}$ \\
    \ooo We want to show $\interp{\judgeS{\spec{\emp}{\newref{A}{e}}{a:\reftype{A}}{a \pointsto e}}}\delta = \top$ \\
    \ooo So we want to show $\forall r.\; r \in \interp{\judgeS{\spec{\emp}{\newref{A}{e}}{a:\reftype{A}}{a \pointsto e}}}\delta$ \\
    \ooo Assume $r$, \\
    \oooo So we want to show $r \in \interp{\judgeS{\spec{\emp}{\newref{A}{e}}{a:\reftype{A}}{a \pointsto e}}}\delta$ \\
    \oooo Let $\Theta = \restrictkind{\Delta}$ \\
    \oooo Let $\theta = \restricttyenv{\Delta}{\delta}$ \\
    \oooo Let $\Gamma = \restricttype{\Delta}$ \\
    \oooo Let $\gamma = \restrictvals{\Delta}{\delta}$ \\
    \oooo Let $C = \interp{\judgeE{\Gamma}{\comp{\newref{A}{e}}}{\monad{\reftype{A}}}\;\theta}\;\gamma$ \\
    \oooo Let $E = \interp{\judgeE{\Gamma}{e}{A}\;\theta}\;\gamma$ \\
    \oooo Let $Q = \semfun{v}{\interp{\judgeA[\Delta, a:\reftype{A}]{a \pointsto_A e}{\assert}}\;(\delta, v)}$ \\
    \oooo So we want to show $r \in \spec{I}{C}{a:A}{Q(a)}$ \\
    \oooo So we want to show $\forall s \sqsupseteq r.\; \basicspec{s}{C}{a:A}{Q(a)*s}$ \\
    \oooo Assume $s \sqsupseteq r$ \\
    \ooooo We want to show $\basicspec{s}{C}{a:A}{Q(a)*s}$ \\
    \ooooo We want to show $\forall h \in s.\; C\;Best(\semfun{a}{Q(a)*s})\;h = \bot$ \\
    \ooooo Assume $h \in s$ \\
    \oooooo Expanding definitions, $C\;Best((\semfun{a}{Q(a)*s})\;h = $ \\
    \oooooox let $l = (\max(\domain{h})+1, [\theta(A)])$ in $Best(\semfun{a}{Q(a)*s})\;l\;[h|l:E]$ \\
    \oooooo So let $l = (\max(\domain{h})+1, [\theta(A)])$ \\
    \oooooo We want to show $Best(\semfun{a}{Q(a)*s})\;l\;[h|l:E] = \bot$ \\
    \oooooo To show this, we must show $\lnot (\exists k \in Approx(\semfun{a}{Q(a)*s}).\;k\;l\;[h|l:E] = \top)$ \\
    \oooooo So we must show $\forall k \in Approx(\semfun{a}{Q(l)*s}).\;k\;l\;[h|l:E] = \bot)$ \\
    \oooooo Assume $k \in Approx(\semfun{a}{Q(a)*s})$ \\
    \ooooooo This means $\forall v, h \in Q(v)*s.\; k\;v\;h = \bot$ \\
    \ooooooo Instantiate $v$ with $l$, and $h$ with $[l:E]\cdot h$ \\
    \ooooooo Now we must check $[l:E]\cdot h \in Q(l) * s$ \\
    \ooooooo So we will check $[l:E] \in Q(l)$ and $h \in s$ \\
    \ooooooo Expanding definitions, $Q(l) = \setof{[l:E]}$, so $[l:E] \in Q(l)$ \\
    \ooooooo From line 18, $h \in s$ \\
    \ooooooo Since $l$ is bigger than anything in $\domain{h}$, it follows $[l:E] \# h$ \\
    \ooooooo Therefore $[l:E]\cdot h \in Q(l) * s$ \\
    \ooooooo Therefore $k\;l\;([l:E]\cdot h) = \bot$ \\
    \oooooo Therefore $\forall k \in Approx(\semfun{a}{Q(l)*s}).\;k\;l\;[h|l:E] = \bot)$ \\
    \ooooo Therefore $\forall h \in s.\; C\;Best(\semfun{a}{Q(a)*s})\;h = \bot$ \\
    \ooooo Therefore $\basicspec{s}{C}{a:A}{Q(a)*s}$ \\
    \oooo Therefore $\forall s \sqsupseteq r.\; \basicspec{s}{C}{a:A}{Q(a)*s}$ \\
    \oooo Therefore $r \in \spec{I}{C}{a:A}{Q(a)}$ \\
    \ooo Therefore $\forall r.\; r \in \interp{\judgeS{\spec{\emp}{\newref{A}{e}}{a:\reftype{A}}{a \pointsto e}}}\delta$ \\
    \ooo Therefore $\interp{\judgeS{\spec{\emp}{\newref{A}{e}}{a:\reftype{A}}{a \pointsto e}}}\delta = \top$ \\
  \end{tabbedproof}
\end{proof}

\begin{lemma}{(Read Axiom)}
We have that $\judgeS{\spec{e \pointsto_A e'}{!e}{a:A}{e \pointsto_A e' \land a = e'}}$ is valid.\end{lemma}
\begin{proof}
  \begin{tabbedproof}
    \oo Assume $\delta \in \interp{\judgeACtx{\Delta}}$ \\
    \ooo We want to show $\interp{\judgeS{\spec{e \pointsto_A e'}{!e}{a:A}{e \pointsto e' \land a = e'}}}\delta = \top$ \\
    \ooo So we want to show $\forall r.\; r \in \interp{\judgeS{\spec{e \pointsto_A e'}{!e}{a:A}{e \pointsto e' \land a = e'}}}\delta$ \\
    \ooo Assume $r$, \\
    \oooo So we want to show $r \in r \in \interp{\judgeS{\spec{e \pointsto_A e'}{!e}{a:A}{e \pointsto e' \land a = e'}}}\delta$ \\
    \oooo Let $\Theta = \restrictkind{\Delta}$ \\
    \oooo Let $\theta = \restricttyenv{\Delta}{\delta}$ \\
    \oooo Let $\Gamma = \restricttype{\Delta}$ \\
    \oooo Let $\gamma = \restrictvals{\Delta}{\delta}$ \\
    \oooo Let $P = \interp{\judgeA{e \pointsto_A e'}{\assert}}\;\delta$ \\
    \oooo Let $C = \interp{\judgeE{\Gamma}{\comp{!e}}{\monad{A}}\;\theta}\;\gamma$ \\
    \oooo Let $E = \interp{\judgeE{\Gamma}{e}{\reftype{A}}\;\theta}\;\gamma$ \\
    \oooo Let $E' = \interp{\judgeE{\Gamma}{e'}{A}\;\theta}\;\gamma$ \\
    \oooo Let $Q = \semfun{v}{\interp{\judgeA[\Delta, a:\reftype{A}]{e \pointsto_A e' \land a = e'}{\assert}}\;(\delta, v)}$ \\
    \oooo So we want to show $r \in \spec{P}{C}{a:A}{Q(a)}$ \\
    \oooo So we want to show $\forall s \sqsupseteq r.\; \basicspec{P*s}{C}{a:A}{Q(a)*s}$ \\
    \oooo Assume $s \sqsupseteq r$ \\
    \ooooo We want to show $\basicspec{P*s}{C}{a:A}{Q(a)*s}$ \\
    \ooooo We want to show $\forall h \in P*s.\; C\;Best(\semfun{a}{Q(a)*s})\;h = \bot$ \\
    \ooooo Assume $h \in P*s$ \\
    \oooooo There are $h_1$ and $h_2$ such that $h_1 \in P$, and $h_2 \in s$, and $h_1 \# h_2$ \\
    \oooooo Expanding definitions, $C\;Best(\semfun{a}{Q(a)*s})\;(h_1 \cdot h_2) = $ \\
    \oooooox if $E \in \domain{h}$ then $Best(\;(h\;E)\;h$ else $\top$ \\
    \oooooo From definition of $P$, $h_1 = [E:E']$ \\
    \oooooo Hence $E \in \domain{h_1}$, so $E \in \domain{h_1 \cdot h_2}$ \\
    \oooooo Hence $C\;Best((\semfun{a}{Q(a)*s})\;(h_1 \cdot h_2) = 
                   Best(\semfun{a}{Q(a)*s})\;(h\;E)\;h$ \\
    \oooooo So we want to show $Best(\semfun{a}{Q(a)*s})\;(h\;E)\;h = \bot$ \\
    \oooooo To show this, we must show $\lnot(\exists k \in Approx(\semfun{a}{Q(a)*s}).\; (h\;E)\;h = \top)$ \\
    \oooooo So we want $\forall k \in Approx(\semfun{a}{Q(a)*s}).\; (h\;E)\;h = \bot)$ \\
    \oooooo Assume $k \in Approx(\semfun{a}{Q(a)*s})$ \\
    \ooooooo So we know $\forall v, h \in Q(v)*s.\; k\;v\;h = \bot$ \\
    \ooooooo Instantiate with $v$ with $(h\;E)$, and $h$ with $h$ \\
    \ooooooo So we must show $h \in Q(h\;E)*s$ \\
    \ooooooo So we will check $h_1 \in Q(h\;E)$ and $h_2 \in s$, since $h_1 \cdot h_2 = h$ \\
    \ooooooo So we want to check $h_1 \in \interp{\judgeA{e \pointsto_A e'}{\assert}}\;\delta = P$ \\
    \ooooooox and also $h_1 \in \interp{\judgeA[\Gamma,a:A]{a = e'}{\assert}}(\delta,h\;E)$ \\
    \ooooooo We know $h_1 \in P$ by assumption \\
    \ooooooo Since $h\;E = E' = \interp{\judgeA{e'}{A}}\delta$, we know 
              $\interp{\judgeA[\Gamma,a:A]{a = e'}{\assert}}(\delta,h\;E) = H$ \\
    \ooooooo Therefore $h_1 \in \interp{\judgeA[\Gamma,a:A]{a = e'}{\assert}}(\delta,h\;E)$ \\
    \ooooooo We know $h_2 \in s$ by assumption\\
    \ooooooo Therefore $h \in Q(h\;E)*s$ \\
    \ooooooo Therefore we know $k\;(h\;E)\;h = \bot$ \\
    \oooooo Therefore $\forall k \in Approx((\semfun{a}{Q(a)*s}).\; (h\;E)\;h = \bot$ \\
    \oooooo Therefore $Best(\semfun{a}{Q(a)*s})\;(h\;E)\;h = \bot$ \\
    \ooooo Therefore $\forall h \in P*s.\; C\;Best(\semfun{a}{Q(a)*s})\;h = \bot$ \\
    \ooooo Therefore $\basicspec{P*s}{C}{a:A}{Q(a)*s}$ \\
    \oooo Therefore $\forall s \sqsupseteq r.\; \basicspec{P*s}{C}{a:A}{Q(a)*s}$ \\
    \oooo Therefore $r \in \spec{P}{C}{a:A}{Q(a)}$ \\
    \ooo Therefore $\forall r.\; r \in \interp{\judgeS{\spec{e \pointsto_A e'}{!e}{a:A}{e \pointsto e' \land a = e'}}}\delta$ \\
    \ooo Therefore $\interp{\judgeS{\spec{e \pointsto_A e'}{!e}{a:A}{e \pointsto e' \land a = e'}}}\delta = \top$ \\
  \end{tabbedproof}
\end{proof}

\begin{lemma}{(Sequential Composition Axiom)}
Suppose $\judgeS{\mspec{p}{e}{x:A}{q}}$ is valid, and $\judgeS[\Delta,x:A]{\spec{q}{c}{a:B}{r}}$ is valid, and $x \not\in \FV{r}$. Then $\judgeS{\spec{p}{\letv{x}{e}{c}}{a:B}{r}}$ is valid. 
\end{lemma}

\begin{proof}
  \begin{tabbedproof}
    \oo Assume $\judgeS{\mspec{p}{e}{x:A}{q}}$ is valid \\
    \oo Assume $\judgeS[\Delta,x:A]{\spec{q}{c}{a:B}{r}}$ is valid\\
    \ooo Assume $\delta \in \interp{\judgeACtx{\Delta}}$ \\
    \oooo We want to show $\interp{\judgeS{\spec{p}{\letv{x}{e}{c}}{a:B}{r}}}\delta = \top$\\
    \oooo So we want $\forall t.\; t \in \interp{\judgeS{\spec{p}{\letv{x}{e}{c}}{a:B}{r}}}\delta$ \\
    \oooo Assume $t$ \\
    \ooooo Let $\Theta$ = $\restrictkind{\Delta}$ \\
    \ooooo Let $\theta$ = $\restricttyenv{\Delta}{\delta}$ \\
    \ooooo Let $\Gamma$ = $\restricttype{\Delta}$ \\
    \ooooo Let $\gamma$ = $\restrictvals{\Delta}{\delta}$ \\
    \ooooo Let $E = \interp{\judgeE{\Gamma}{e}{\monad{A}}}\;\theta\;\gamma$ \\
    \ooooo Let $F = \semfun{v}{\interp{\judgeE{\Gamma,x:A}{\comp{c}}{\monad{B}}}\;\theta\;(\gamma,v)}$ \\
    \ooooo Let $P = \interp{\judgeA{p}{\assert}}\delta$ \\
    \ooooo Let $Q = \semfun{v}{\interp{\judgeA[\Delta,x:A]{q}{\assert}}\;\delta}$ \\
    \ooooo Let $R = \semfun{v}{\semfun{v'}{\interp{\judgeA[\Delta,x:A,a:B]{q}{\assert}}\;(\delta, v, v')}}$ \\
    \ooooo Let $R' = \semfun{v'}{\interp{\judgeA[\Delta,a:B]{q}{\assert}}\;(\delta, v')}$ \\
    \ooooo Note $\forall v.\; R\;v = R'$ since $x \not\in \FV{r}$ \\
    \ooooo Let $C = \interp{\judgeE{\Gamma}{\comp{\letv{x}{e}{c}}}{\monad{B}}}\;\delta$ \\
    \ooooo By semantics, $C = F^*(E)$ \\
    \ooooo By definition of monadic lift, $C = \semfun{k}{E\;(\semfun{v}{F\;v\;k})}$ \\
    \ooooo So we want to show $t \in \spec{P}{C}{a:A}{R'(a)}$ \\
    \ooooo So we want $\forall u \sqsupseteq t.\; \basicspec{P * u}{C}{a:A}{(R'(a) * u)}$ \\
    \ooooo Assume $u \sqsupseteq t$ \\
    \oooooo So we want to show $\forall h \in P * u.\; C\;Best(\semfun{a}{R'(a) * u})\;h = \bot$\\
    \oooooo Assume $h \in P * u$ \\
    \ooooooo So we want to show $E\;(\semfun{v}{F\;v\;Best(\semfun{a}{R'(a)*u})})\;h = \bot$ \\
    \ooooooo We know $\judgeS{\mspec{p}{e}{x:A}{q}}$ is valid \\
    \ooooooo Instantiating with the environment $\delta$, we get $\forall t.\; t \in \spec{P}{E}{a:A}{Q(a)}$ \\
    \ooooooo Thus $\forall t, u \sqsupseteq t.\; \basicspec{P * u}{E}{a:A}{Q(a) * u}$ \\
    \ooooooo Thus $\forall t, u \sqsupseteq t, h \in P * u, E\;Best(\semfun{a}{Q(a) * u})\;h = \bot$ \\
    \ooooooo Instantiating, we get $E\;Best(\semfun{a}{Q(a) * u})\;h = \bot$ \\
    \ooooooo So it suffices to show $(\semfun{v}{F\;v\;Best(\semfun{a}{R'(a)*u})}) \sqsubseteq 
                                Best(\semfun{a}{Q(a) * u})$ \\
    \ooooooo To do this, we can show $(\semfun{v}{F\;v\;Best(\semfun{a}{R'(a)*u})}) \in Approx(\semfun{a}{Q(a) * u})$ \\
    \ooooooo To do this, we must show $\forall v, h \in Q(v) * u, F\;v\;Best(\semfun{a}{R'(a)*u})\;h = \bot$ \\
    \ooooooo Assume $v, h \in Q(v) * u$ \\
    \oooooooo We know $\judgeS[\Delta,x:A]{\spec{q}{c}{a:B}{r}}$ is valid\\
    \oooooooo Instantiating with the environment $(\delta, v)$, we get \\
    \oooooooox $\forall v, t, u \sqsupseteq t, h \in Q(v) * u, (F\;v)\;Best(\semfun{a}{R\;v\;a * u})\;h = \bot$ \\
    \oooooooo Instantiating, we get $(F\;v)\;Best(\semfun{a}{R\;v\;a * u})\;h = \bot$ \\
    \oooooooo By equality, we get $(F\;v)\;Best(\semfun{a}{R'(a) * u})\;h = \bot$ \\
    \ooooooo Therefore $\forall v, h \in Q(v) * u, F\;v\;Best(\semfun{a}{R'(a)*u})\;h = \bot$ \\
    \ooooooo Therefore $(\semfun{v}{F\;v\;Best(\semfun{a}{R'(a)*u})}) \sqsubseteq 
                                Best(\semfun{a}{Q(a) * u})$ \\
    \ooooooo Therefore $(\semfun{v}{F\;v\;Best(\semfun{a}{R'(a)*u})}) \in Approx(\semfun{a}{Q(a) * u})$ \\                        
    \ooooooo Therefore $E\;(\semfun{v}{F\;v\;Best(\semfun{a}{R'(a)*u})})\;h = \bot$ \\
    \oooooo Therefore $\forall h \in P * u.\; C\;Best(\semfun{a}{R'(a) * u})\;h = \bot$\\
    \ooooo Therefore $\forall u \sqsupseteq t.\; \basicspec{P * u}{C}{a:A}{(R'(a) * u)}$ \\
    \ooooo Therefore $t \in \spec{P}{C}{a:A}{R'(a)}$ \\
    \oooo Therefore $\forall t.\; t \in \interp{\judgeS{\spec{p}{\letv{x}{e}{c}}{a:B}{r}}}\delta$ \\
    \oooo Therefore $\interp{\judgeS{\spec{p}{\letv{x}{e}{c}}{a:B}{r}}}\delta = \top$\\
  \end{tabbedproof}
\end{proof}

\begin{lemma}{(Fixed Point Induction)}
We have that if 
\begin{displaymath}
   S \triangleq 
   \judgeS{(\forall x:\monad{A}.\; \mspec{p}x{a:A}{q(a)} \specimp \mspec{p}{e}{a:A}{q})}
\end{displaymath}
\noindent is valid, then 
\begin{displaymath}
  S' \triangleq \judgeS{\mspec{p}{\fix{x:\monad{A}}{e}}{a:A}{q}}
\end{displaymath}
is valid.
\end{lemma}

\begin{proof}
  \begin{tabbedproof}
    \oo Let $S = \forall x:\monad{A}.\; \mspec{p}x{a:A}{q(a)} \specimp \mspec{p}{e}{a:A}{q}$\\
    \oo Let $S' = \mspec{p}{\fix{x:\monad{A}}{e}}{a:A}{q}$ \\
    \oo Assume $\delta \in \interp{\judgeACtx{\Delta}}$ \\
    \ooo So we want to show $\interp{S'}\;\delta = \top$ \\
    \ooo Let $P = \interp{\judgeA{P}{\assert}}\;\delta$ \\
    \ooo Let $C = \interp{\judgeA{\fix{x:\monad{A}}{e}}{\monad{A}}}\;\delta$ \\
    \ooo Let $Q = \semfun{v}{\interp{\judgeA[\Delta,a:A]{q}{\assert}}\;(\delta,v)}$ \\
    \ooo Let $F(v) = \interp{\judgeA[\Delta,a:A]{e}{\monad{A}}}\;(\delta,v)$ \\
    \ooo So we want to show $\spec{P}{C}{a:A}{Q(a)} = \top$ \\
    \ooo We know $\interp{S}\;\delta = \top$ \\
    \ooo Thus for all $v$, we know $\interp{\mspec{p}x{a:A}{q(a)} \specimp \mspec{p}{e}{a:A}{q}}\;(\delta,v) = \top$ \\
    \ooo Thus we know for all $v$, $r$, and $s \sqsupseteq r$, \\
    \ooox if $s \in \interp{\mspec{p}x{a:A}{q(a)}}\;(\delta,v)$ then $s \in \interp{\mspec{p}{e}{a:A}{q}}\;(\delta,v)$ \\
    \ooo Since $x \not\in \FV{p}$, we know for all $v$, 
         $\interp{\judgeA[\Delta,x:\monad{A}]{p}{\assert}}\;(\delta,v) = P$ \\
    \ooo Since $x \not\in \FV{q}$, we know for all $v$, 
         $\semfun{v'}{\interp{\judgeA[\Delta,x:\monad{A}, a:A]{q}{\assert}}\;(\delta,v,v')} = Q$ \\
    \ooo Thus we know for all $v$, $r$, and $s \sqsupseteq r$, \\
    \ooox if $s \in  \spec{P}{v}{a:A}{Q(a)}$ then 
             $s \in \spec{P}{F(v)}{a:A}{Q(a)}$ \\
    \ooo This implies that for all $v$, if $\spec{P}{F(v)}{a:A}{Q(a)} = \top $ 
                                        then $\spec{P}{F(v)}{a:A}{Q(a)} = \top$ \\
    \ooo Then by the semantic fixed point theorem, 
          $\spec{P}{fix(F)}{a:A}{Q(a)} = \top$ \\
    \ooo So $\spec{P}{C}{a:A}{Q(a)} = \top$ \\
    \ooo Therefore $\interp{S'}\;\delta = \top$ \\
  \end{tabbedproof}
\end{proof}

\begin{lemma}{(Assumptions From Preconditions)}
If we know that $\judgeS{\setof{r} \specimp \spec{p}{c}{a:A}{q}}$ is
valid, $r$ is a pure formula of separation logic, and we know that $p
\implies r$ is a valid truth of separation logic, then
$\judgeS{\spec{p}{c}{a:A}{q}}$ is valid.
\end{lemma}

\begin{proof}
  \begin{tabbedproof}
    \oo Assume $\judgeS{\setof{r} \specimp \spec{p}{c}{a:A}{q}}$ is valid \\
    \oo Assume $p \implies r$ is a valid truth of separation logic \\
    \ooo Now, we know that for all $\delta$, $\interp{\judgeS{\setof{r} \specimp \spec{p}{c}{a:A}{q}}}\;\delta$ is $\top$\\
    \ooo Now, we want to show for all $\delta$, that $\interp{\judgeS{\spec{p}{c}{a:A}{q}}}\;\delta = \top$ \\
    \ooo Assume $\delta$ is an environment \\
    \oooo So we want to show for all $r$, and all $s \worldgeq r$, that $\basicspec{P*s}{C}{a}{Q(a) * s}$ holds\\
    \ooox where \\
    \oooox $R = \interp{\judgeA{r}{\assert}}\;\delta$ \\
    \oooox $P = \interp{\judgeA{p}{\assert}}\;\delta$ \\
    \oooox $C = \interp{\judgeC[\restrictkind{\Delta}]{\restricttype{\Delta}}{c}{A}}\;\theta\;\gamma$ \\
    \oooox $Q = \semfun{v}{\interp{\judgeA[\Delta,a:A]{q}{\assert}}\;(\delta,v)}$ \\
    \oooox $\theta = \restricttyenv{\Delta}{\delta}$ \\
    \oooox $\gamma = \restrictvals{\Delta}{\delta}$ \\
    \oooo To show $\basicspec{P*s}{C}{a}{Q(a)*s}$, we must show 
          $\forall h \in P * s.\; c\; (Best(\semfun{a}{Q(a)*s}))\;h = \bot$ \\
    \oooo Assume $h \in P * s$ \\
    \ooooo Since $p \implies r$ is a valid truth, we know $P \implies R$ and so $h \in R * s$ \\
    \ooooo Since $R$ is pure, we know $h \in R \land s$ \\
    \ooooo So we know $h \in R$ \\
    \ooooo Since $R$ is pure it is either $\emptyset$ or $H$, and since we know $h \in R$, $R = H$ \\
    \ooooo Therefore we know that $\interp{\judgeS{\setof{r}}}\;\delta = \top$ \\
    \ooooo Therefore we know that $\interp{\judgeS{\spec{p}{c}{a:A}{q}}}\;\delta = \top$ \\
  \end{tabbedproof}
\end{proof}

\begin{lemma}{(Assumptions Into Preconditions)}
If we know that $\judgeS{\setof{r} \specimp \spec{p \land r}{c}{a:A}{q}}$ is
valid, then $\judgeS{\setof{r} \specimp \spec{p}{c}{a:A}{q}}$ is valid.
\end{lemma}

\begin{proof}
\begin{tabbedproof}
  \oo Assume $\judgeS{\setof{r} \specimp \spec{p \land r}{c}{a:A}{q}}$ is valid. \\
  \ooo This is equivalent to for all $\delta$ and $s$ if $s \in \interp{\setof{r}}\;\delta$, then $s \in \interp{\spec{p \land r}{c}{a:A}{q}}\;\delta$. \\
  \ooo We want to show that $\judgeS{\setof{r} \specimp \spec{p}{c}{a:A}{q}}$ is valid \\
  \ooo This is equivalent to showing for all $\delta$ and $s$, if $s \in \interp{\setof{r}}\;\delta$, then $s \in \interp{\spec{p}{c}{a:A}{q}}\;\delta$. \\
  \ooo Assume we have $\delta$ and $s$ such that $s \in \interp{\setof{r}}\;\delta$. \\
  \oooo We want to show $s \in \interp{\spec{p}{c}{a:A}{q}}\;\delta$ is valid. \\
  \oooo So we want to show for all $t \worldgeq s$, that $\basicspec{P*t}{C}{a}{Q(a) * t}$ holds\\
  \ooox where \\
  \oooox $R = \interp{\judgeA{r}{\assert}}\;\delta$ \\
  \oooox $P = \interp{\judgeA{p}{\assert}}\;\delta$ \\
  \oooox $C = \interp{\judgeC[\restrictkind{\Delta}]{\restricttype{\Delta}}{c}{A}}\;\theta\;\gamma$ \\
  \oooox $Q = \semfun{v}{\interp{\judgeA[\Delta,a:A]{q}{\assert}}\;(\delta,v)}$ \\
  \oooox $\theta = \restricttyenv{\Delta}{\delta}$ \\
  \oooox $\gamma = \restrictvals{\Delta}{\delta}$ \\
  \oooo Assume $t \worldgeq s$, and $h \in P * t$. \\
  \ooooo We want to show that $C\;\mathit{Best}(\semfun{a}{Q(a) * s})\;h = \bot$ \\
  \ooooo Since $s \in \interp{\spec{p \land r}{c}{a:A}{q}}\;\delta$, we know $C\;\mathit{Best}(\semfun{a}{Q(a) * s})\;h = \bot$ if  $h \in (P \land R) * t$. \\
  \ooooo Since we assumed that $\interp{\setof{r}}\;\delta$ was non-empty, this means that $\interp{r}\;\delta = \top = H$. \\
  \ooooo Hence $h \in (P \land R) * t$. \\
  \ooooo Hence $C\;\mathit{Best}(\semfun{a}{Q(a) * s})\;h = \bot$ \\ 
\end{tabbedproof}
\end{proof}

\begin{lemma}{(Getting Specs Out of Assertions)}
We have that $\judgeS{\setof{\validprop{S}} \specimp S}$ is valid.
\end{lemma}
\begin{proof}
  \begin{tabbedproof}
    \oo We want to show $\judgeS{\setof{\validprop{S}} \specimp S}$ is valid \\
    \oo Equivalently, we need for all $\delta$ and $r$, if $r \in \interp{\judgeS{\setof{\validprop{S}}}}\;\delta$\\
    \ox    then $r \in \interp{\judgeS{S}}\;\delta$ \\
    \oo Assume we have a suitable $\delta$ and $r$ such that $r \in \interp{\judgeS{\setof{\validprop{S}}}}\;\delta$ \\
    \ooo From the semantics, we know $\interp{\judgeS{\setof{\validprop{S}}}}\;\delta$ is either $\top$ or $\bot$ \\
    \ooo Since $r \in \interp{\judgeS{\setof{\validprop{S}}}}\;\delta$, we know it is non-empty, hence \\
    \ooox $\interp{\judgeS{\setof{\validprop{S}}}}\;\delta = \top$ \\
    \ooo Therefore, we know that $\interp{\judgeA{\validprop{S}}{\assert}}\;\delta = H$ \\
    \ooo Therefore, we know that $\interp{\judgeS{S}}\;\delta = \top$ \\
  \end{tabbedproof}
\end{proof}

\begin{lemma}{(Existential Dropping)}
If $\judgeS{\spec{\exists u:\omega.\;p}{c}{a:A}{q}}$ is valid, then 
   $\judgeS[\Delta, u:\omega]{\spec{p}{c}{a:A}{q}}$ is valid. 
\end{lemma}

\begin{proof}
  \begin{tabbedproof}
    \oo Assume $\judgeS{\spec{\exists u:\omega.\;p}{c}{a:A}{q}}$ is valid \\
    \ooo We want to show $\judgeS[\Delta, u:\omega]{\spec{p}{c}{a:A}{q}}$ is valid \\
    \ooo So we want to show for all $\delta' = (\delta, v)$, that $\interp{\judgeS[\Delta, u:\omega]{\spec{p}{c}{a:A}{q}}}\;\delta' = \top$ \\
    \ooo Assume we have a suitable $\delta' = (\delta, v)$ \\ 
    \oooo We want to show for all $r$, that $r \in \spec{P}{C}{a}{Q(a)}$ \\
    \oooo where \\
    \oooox $P = \interp{\judgeA[\Delta, u:\omega]{p}{\assert}}\;\delta'$ \\
    \oooox $C = \interp{\judgeC[\restrictkind{\Delta, u:\omega}]{\restricttype{\Delta, u:\omega}}{c}{A}}\;\theta'\;\gamma'$ \\
    \oooox $Q = \semfun{v'}{\interp{\judgeA[\Delta, u:\omega]{q}{\assert}}\;(\delta',v')}$\\
    \oooox $\theta' = \restricttyenv{\Delta, u:\omega}{\delta'}$ \\
    \oooox $\gamma' = \restricttyenv{\Delta, u:\omega}{\delta'}$ \\
    \oooox $\theta = \restricttyenv{\Delta}{\delta}$ \\
    \oooox $\gamma = \restrictvals{\Delta}{\delta}$ \\
    \oooo We also know that $C = \interp{\judgeC[\restrictkind{\Delta}]{\restricttype{\Delta}}{c}{A}}\;\delta$ \\
    \oooo and that $Q = \semfun{v'}{\interp{\judgeA{q}{\assert}}\;(\delta,v')}$\\
    \oooo So now we must show for all $s \worldgeq r$, that $\basicspec{P * s}{C}{a}{Q(a) * s}$ holds\\
    \oooo Now assume $s$ such that $s \worldgeq r$ \\
    \ooooo We need for all $h \in \basicspec{P * s}{C}{a}{Q(a) * s}$, that 
             $C\;Best(\semfun{a}{Q(a) * s})\;h = \bot$ \\
    \ooooo We know $\spec{\exists u:\omega.\;p}{c}{a:A}{q}$ is valid \\
    \ooooo So we know for all $r$, that $r \in \spec{P'}{C}{a}{Q(a)}$ \\
    \oooox where $P' = \bigvee_{v \in \interp{\judgeSort{\omega}}\;\delta} \interp{\judgeA[\Delta, u:\omega]{p}{\assert}}\;(\delta, v)$ \\
    \ooooo Since $P = \interp{\judgeA[\Delta,u:\omega]{p}{\assert}}\;(\delta, v)$, we know $P \subseteq P'$ \\
    \ooooo Therefore $h \in P' * s$, and so $C\;Best(\semfun{a}{Q(a) * s})\;h = \bot$ \\
    
  \end{tabbedproof}
\end{proof}

\begin{lemma}{(Disjunction Rule)}
If $\judgeS{\spec{p}{c}{a:A}{q}}$ is valid and 
   $\judgeS{\spec{p'}{c}{a:A}{q'}}$ is valid, 
then $\judgeS{\spec{p \vee p'}{c}{a:A}{q \vee q'}}$ is valid. 
\end{lemma}

\begin{proof}
  \begin{tabbedproof}
    \oo Assume $\judgeS{\spec{p}{c}{a:A}{q}}$ is valid \\
    \oo Assume $\judgeS{\spec{p'}{c}{a:A}{q'}}$ is valid \\

    \oo Assume $\delta \in \interp{\judgeACtx{\Delta}}$ \\
    \ooo We want to show $\interp{\judgeS{\spec{p \vee p'}{c}{a:A}{q \vee q'}}}\delta = \top$ \\
    \ooo So we want to show $\forall r.\; r \in \interp{\judgeS{\spec{p \vee p'}{c}{a:A}{q \vee q'}}}\delta$ \\
    \ooo Assume $r$, \\
    \oooo So we want to show $r \in\interp{\judgeS{\spec{p \vee p'}{c}{a:A}{q \vee q'}}}\delta$ \\
    \oooo Let $\Theta = \restrictkind{\Delta}$ \\
    \oooo Let $\theta = \restricttyenv{\Delta}{\delta}$ \\
    \oooo Let $\Gamma = \restricttype{\Delta}$ \\
    \oooo Let $\gamma = \restrictvals{\Delta}{\delta}$ \\
    \oooo Let $P'' = \interp{\judgeA{p \vee p'}{\assert}}\;\delta$ \\
    \oooo Let $C = \interp{\judgeE{\Gamma}{[c]}{\monad{A}}}\;\theta\;\gamma$ \\
    \oooo Let $Q'' = \semfun{v}{\interp{\judgeA[\Delta,a:A]{q \vee q'}{\assert}}\;(\delta, v)}$ \\
    \oooo So we want to show $r \in \spec{P''}{C}{a:A}{Q''(a)}$ \\
    \oooo So we want to show $\forall s \sqsupseteq r.\; \basicspec{P'' * s}{C}{a:A}{Q''(a) * s}$ \\
    \oooo Assume $s \sqsupseteq r$ \\
    \ooooo We want to show $\basicspec{P'' * s}{C}{a:A}{Q''(a) * s}$ \\
    \ooooo We want to show $\forall h \in P'' * s.\; C\;Best(\semfun{a}{Q''(a) * s})\;h = \bot$ \\
    \ooooo Assume $h \in P'' * s$ \\
    \oooooo Therefore there are $h_1 \in P''$ and $h_2 \in s$ such that $h = h_1 \cdot h_2$ \\
    \oooooo We know $P'' = \interp{\judgeA{p \vee p'}{\assert}}\;\delta = 
                           \interp{\judgeA{p}{\assert}}\;\delta \vee 
                           \interp{\judgeA{p'}{\assert}}\;\delta$ \\ 
    \oooooo Call $P = \interp{\judgeA{p}{\assert}}\;\delta$ and $P' = \interp{\judgeA{p'}{\assert}}\;\delta$ \\
    \oooooo Call $Q = \interp{\judgeA{q}{\assert}}\;\delta$ and $Q' = \interp{\judgeA{q'}{\assert}}\;\delta$ \\
    \oooooo By the definition of $\vee$, we know that $h_1 \in P$ or $h_1 \in P'$ \\
    \oooooo Suppose $h_1 \in P$: \\
    \ooooooo We know by assumption that $\basicspec{P * s}{C}{a:A}{Q(a) * s}$ \\
    \ooooooo We know that $h = h_1 \cdot h_2 \in P * s$ \\
    \ooooooo Hence $C\;\mathit{Best}(\semfun{a}{Q(a) * s})\;h = \bot$ \\
    \ooooooo For any $a$, $Q(a) \subseteq Q''(a)$, \\
    \ooooooo Hence $\mathit{Best}(\semfun{a}{Q''(a) * s}) \sqsubseteq \mathit{Best}(\semfun{a}{Q(a) * s})$ \\ 
    \ooooooo Since $C$ is continuous, $C\;\mathit{Best}(\semfun{a}{Q''(a) * s})\;h = \bot$ \\
    \oooooo Suppose $h_1 \in P'$: \\
    \ooooooo We know by assumption that $\basicspec{P' * s}{C}{a:A}{Q(a) * s}$ \\
    \ooooooo We know that $h = h_1 \cdot h_2 \in P' * s$ \\
    \ooooooo Hence $C\;\mathit{Best}(\semfun{a}{Q'(a) * s})\;h = \bot$ \\
    \ooooooo For any $a$, $Q'(a) \subseteq Q''(a)$ \\
    \ooooooo Hence $\mathit{Best}(\semfun{a}{Q''(a) * s}) \sqsubseteq \mathit{Best}(\semfun{a}{Q'(a) * s})$ \\
    \ooooooo Since $C$ is continuous, $C\;\mathit{Best}(\semfun{a}{Q''(a) * s})\;h = \bot$ \\
  \end{tabbedproof}
\end{proof}

\begin{lemma}{(Equality Substitution)}
If $\judgeS{\setof{r} \specimp \spec{p}{c[e/x]}{a:A}{q}}$ is valid and
$\validP{r \implies e =_A e'}$ is valid, then $\judgeS{\setof{r} \specimp \spec{p}{c[e'/x]}{a:A}{q}}$ is valid.
\end{lemma}

\begin{proof}
This rule follows as a consequence of the restriction lemmas we proved earlier. 

\begin{tabbedproof}
\oo Assume $\judgeS{\setof{r} \specimp \spec{p}{c[e/x]}{a:A}{q}}$ is valid. \\
\ooo This is equivalent to assuming for all $\delta$ and $s$, if $s \in \interp{\setof{r}}\;\delta$, then $s \in \interp{\spec{p}{c[e/x]}{a:A}{q}}\;\delta$. \\
\ooo Assume $\delta$, $r$, and $s \in \interp{\setof{r}}\;\delta$ \\
\oooo Let \\
\oooox $C = \interp{c[e/x]}\;\restricttyenv{\Delta}{\delta}\;\restrictvals{\Delta}{\delta}$ \\
\oooox $C' = \interp{c[e'/x]}\;\restricttyenv{\Delta}{\delta}\;\restrictvals{\Delta}{\delta}$ \\
\oooox $P = \interp{\judgeA{p}{\assert}}\;\delta$ \\
\oooox $Q = \semfun{v}{\interp{\judgeA[\Delta,a:A]{q}{\assert}}\;(\delta,v)}$ \\
\oooo Therefore we know that $R = \interp{r}\;\delta = \top = H$. \\
\oooo Therefore it follows that $\interp{\judgeA{e =_A e'}{\assert}}\;\delta = H$ \\
\oooo Therefore we know that $U(\interp{\judgeE[\restrictkind{\Delta}]{\restricttype{\Delta}}{e}{A}}\;\restricttyenv{\Delta}{\delta})\;\restrictvals{\Delta}{\delta}$ equals \\
\ooox $U(\interp{\judgeE[\restrictkind{\Delta}]{\restricttype{\Delta}}{e'}{A}}\;\restricttyenv{\Delta}{\delta})\;\restrictvals{\Delta}{\delta}$ \\
\oooo By Lemma~\ref{value-restriction}, we know that 
$(\interp{\judgeE[\restrictkind{\Delta}]{\restricttype{\Delta}}{e}{A}}\;\restricttyenv{\Delta}{\delta})\;\restrictvals{\Delta}{\delta}$ equals \\
\ooox $(\interp{\judgeE[\restrictkind{\Delta}]{\restricttype{\Delta}}{e'}{A}}\;\restricttyenv{\Delta}{\delta})\;\restrictvals{\Delta}{\delta}$, using lambda-notation for CPO \\
\oooo Since substitution is sound, $\interp{c[e/x]}\;\restricttyenv{\Delta}{\delta}\;\restrictvals{\Delta}{\delta}  = \interp{c[e'/x]}\;\restricttyenv{\Delta}{\delta}\;\restrictvals{\Delta}{\delta}$, or $C = C'$\\
\oooo By hypothesis, $s \in \interp{\spec{p}{c[e/x]}{a:A}{q}}\;\delta$. \\
\oooo So for all $t \worldgeq s, h \in P * t$, we have $C\;\mathit{Best}(\semfun{v}{Q(v) * t})\;h = \bot$ \\
\oooo Since $C = C'$, we know for all $t \worldgeq s, h \in P * t$, we have $C'\;\mathit{Best}(\semfun{v}{Q(v) * t})\;h = \bot$ \\
\oooo Therefore, $s \in \interp{\spec{p}{c[e'/x]}{a:A}{q}}\;\delta$. \\
\end{tabbedproof}
\end{proof}

\begin{lemma}{(Case Analysis)}
  Suppose $\judgeS[\Delta,x:A]{\mspec{p \land e =_{A+B} \inl(x)}{e_A}{a:C}{q}}$ is valid,
      and $\judgeS[\Delta,y:B]{\mspec{p \land e =_{A+B} \inr(y)}{e_B}{a:C}{q}}$ is valid.

Then $\judgeS{\mspec{p}{\Case{e}{x}{e_A}{y}{e_B}}{a:C}{q}}$ is valid. 
\end{lemma}

\begin{proof}
\begin{tabbedproof}
\oo Assume $\judgeS[\Delta,x:A]{\mspec{p \land e =_{A+B} \inl(x)}{e_A}{a:C}{q}}$ is valid. \\
\oo Assume $\judgeS[\Delta,y:B]{\mspec{p \land e =_{A+B} \inr(y)}{e_B}{a:C}{q}}$ is valid. \\
\ooo We want to show $\judgeS{\mspec{p}{\Case{e}{x}{e_A}{y}{e_B}}{a:C}{q}}$ is valid. \\
\ooo So we want to show that for all $\delta, s$, $s \in \interp{\judgeS{\mspec{p}{\Case{e}{x}{e_A}{y}{e_B}}{a:C}{q}}}\;\delta$ \\
\ooo Assume $\delta$ and $s$, and $h \in P * s$, letting \\
\ooox $P = \interp{\judgeA{p}{\assert}}\;\delta$ \\
\ooox $Q = \semfun{v}{\interp{\judgeA[\Delta, a:C]{p}{\assert}}\;(\delta,v)}$ \\
\ooox $C_A = \semfun{v}{\interp{\judgeE[\restrictkind{\Delta, x:A}]{\restricttype{\Delta, x:A}}{e_A}{\monad{C}}}\;\restricttyenv{\Delta,x:A}{(\delta,v)}\;\restrictvals{\Delta, x:A}{(\delta,v)}}$ \\
\ooox $C_B = \semfun{v}{\interp{\judgeE[\restrictkind{\Delta, y:B}]{\restricttype{\Delta, y:B}}{e_B}{\monad{C}}}\;\restricttyenv{\Delta,y:B}{(\delta,v)}\;\restrictvals{\Delta, y:B}{(\delta,v)}}$ \\
\ooox $E = \interp{\judgeE[\restrictkind{\Delta}]{\restricttype{\Delta}}{e}{A+B}}\;\restricttyenv{\Delta}{\delta}\;\restrictvals{\Delta}{\delta}$ \\
\oooo We want to show $[C_A,C_B](E)\;\mathit{Best}(\semfun{v}{Q(v)*s})\;h = \bot$\\
\oooo Suppose $E = \inl(u)$ for some $u$: \\
\ooooo Then we want to show that $C_A\;u\;\mathit{Best}(\semfun{v}{Q(v)*s})\;h = \bot$\\
\ooooo We know $\judgeS[\Delta,x:A]{\mspec{p \land e =_{A+B} \inl(x)}{e_A}{a:C}{q}}$ is valid. \\
\ooooo So for all $h \in (P \land \interp{e =_{A+B} \inl(x)}\;(\delta,\inl(u))) * s$, $C_A\;u\;\mathit{Best}(\semfun{v}{Q(v)*s})\;h = \bot$\\
\ooooo Since $E=\inl(u)$, we know that $\top = \interp{\judgeA[\Delta,x:A]{e=_{A+B}\inl(x)}{\assert}}\;(\delta,\inl(u))$\\
\ooooo Since equality is pure, $h \in \interp{p \land e =_{A+B} \inl(x)}\;(\delta,\inl(u)) * s$ \\
\ooooo Therefore $C_A\;u\;\mathit{Best}(\semfun{v}{Q(v)*s})\;h = \bot$\\
\oooo Suppose $E = \inr(u)$ for some $u$:  \\
\ooooo Then we want to show that $C_B\;u\;\mathit{Best}(\semfun{v}{Q(v)*s})\;h = \bot$\\
\ooooo We know $\judgeS[\Delta,y:B]{\mspec{p \land e =_{A+B} \inr(x)}{e_A}{a:C}{q}}$ is valid. \\
\ooooo So for all $h \in (P \land \interp{e =_{A+B} \inr(x)}\;(\delta,\inr(u))) * s$, $C_B\;u\;\mathit{Best}(\semfun{v}{Q(v)*s})\;h = \bot$\\
\ooooo Since $E=\inr(u)$, we know that $\top = \interp{\judgeA[\Delta,y:B]{e=_{A+B}\inr(x)}{\assert}}\;(\delta,\inr(u))$\\
\ooooo Since equality is pure, $h \in \interp{p \land e =_{A+B} \inr(x)}\;(\delta,\inr(u)) * s$ \\
\ooooo Therefore $C_B\;u\;\mathit{Best}(\semfun{v}{Q(v)*s})\;h = \bot$\\

\end{tabbedproof}
\end{proof}

\begin{lemma}{(The Rule of Consequence)}
If $\judgeS{\spec{p}{c}{a:A}{q}}$ is valid and $\judgeA{p' \implies p}{\assert}$ is valid
and $\judgeA{q \implies q'}{\assert}$ is valid, then $\judgeS{\spec{p'}{c}{a:A}{q'}}$ is valid.
\end{lemma}
\begin{proof}
\begin{tabbedproof}
\oo Assume $\judgeS{\spec{p}{c}{a:A}{q}}$ is valid \\
\oo Assume $\judgeA{p' \implies p}{\assert}$ is valid \\
\oo Assume $\judgeA{q \implies q'}{\assert}$ is valid \\
\oo Assume $\delta \in \interp{\judgeACtx{\Delta}}$ \\
\ooo We want to show $\forall r.\; r \in \judgeS{\spec{p'}{c}{a:A}{q'}}$ \\
\ooo Assume $r$, and let \\
\ooox $\Theta = \restrictkind{\Delta}$ \\
\ooox $\theta = \restricttyenv{\Delta}{\delta}$ \\
\ooox $\Gamma = \restricttype{\Delta}$ \\
\ooox $\gamma = \restrictvals{\Delta}{\delta}$ \\
\ooox $P = \interp{\judgeA{p}{\assert}}\delta$ \\
\ooox $P' = \interp{\judgeA{p'}{\assert}}\delta$ \\
\ooox $C = \interp{\judgeC{\Gamma}{c}{A}}\;\theta\;\gamma$ \\
\ooox $Q = \semfun{v}{\interp{\judgeA[\Delta,a:A]{q}{\assert}}\;(\delta,v)}$ \\
\ooox $Q' = \semfun{v}{\interp{\judgeA[\Delta,a:A]{q}{\assert}}\;(\delta,v)}$ \\
\ooo We know $\interp{\judgeA{p' \implies p}{\assert}}\;\delta = P' \implies P$. \\
\ooo We know $\interp{\judgeA{q \implies q'}{\assert}}\;\delta = Q \implies Q'$. \\
\ooo So we want to show that $\forall s \sqsupseteq r.\; \basicspec{P' * s}{C}{a:A}{Q'(a) * s}$ \\
\ooo Assume $s \sqsupseteq r$, and $h \in P' * s$ \\
\oooo Hence $\exists$ $h_1$ and $h_2$ such that $h_1 \# h_2$ and $h = h_1 \cdot h_2$ 
     and $h_1 \in P'$ and $h_2 \in s$ \\
\oooo Since $P'$ is a subset of $P$, $h_1 \in P$. So $h \in P * s$ \\
\oooo Therefore $C\;\mathit{Best}(\semfun{a}{Q(a) * s})\;h = \bot$ \\
\oooo For any $a$, we know that $Q(a) \subseteq Q'(a)$ \\
\oooo Hence  we know that $\mathit{Best}(\semfun{a}{Q'(a) * s}) \sqsupseteq \mathit{Best}(\semfun{a}{Q(a) * s})$ \\
\oooo Hence by continuity, $C\;\mathit{Best}(\semfun{a}{Q'(a) * s})\;h = \bot$ \\
\ooo Hence $\forall s \sqsupseteq r.\; \basicspec{P' * s}{C}{a:A}{Q'(a) * s}$ \\
\ooo Hence $\forall r.\; r \in \interp{\judgeS{\spec{p'}{c}{a:A}{q'}}}\;\delta$ \\



\end{tabbedproof}
\end{proof}


\subsection{The Syntactic Frame Property}

We have defined a \emph{syntactic} frame operator $\Frame{S}{r}$ on
specifications.

\begin{displaymath}
  \begin{array}{lcl}
    \Frame{\spec{p}{c}{a:A}{q}}{r}    & = & \spec{p * r}{c}{a:A}{q * r} \\
    \Frame{\mspec{p}{c}{a:A}{q}}{r}   & = & \mspec{p * r}{c}{a:A}{q * r} \\
    \Frame{\setof{p}}{r}              & = & \setof{p} \\
    \Frame{(S_1 \specand S_2)}{r}      & = & \Frame{S_1}{r} \specand \Frame{S_2}{r} \\
    \Frame{(S_1 \specor S_2)}{r}       & = & \Frame{S_1}{r} \specor \Frame{S_2}{r} \\
    \Frame{(S_1 \specimp S_2)}{r}      & = & \Frame{S_1}{r} \specimp \Frame{S_2}{r} \\
    \Frame{(\forall x:\omega.\; S)}{r} & = & \forall x:\omega.\; (\Frame{S}{r}) \\
    \Frame{(\exists x:\omega.\; S)}{r} & = & \exists x:\omega.\; (\Frame{S}{r}) \\
  \end{array}
\end{displaymath}

\begin{prop*}{(Syntactic Well-Formedness of Frame Operator)}
If $\judgeS{S}$ and $\judgeA{p}{\assert}$, then
we define $\judgeS{\Frame{S}{p}}$.  
\end{prop*}
\begin{proof}
  By structural induction on specifications.
\end{proof}

More interesting than the syntactic well-formedness of the frame
operator is its semantic well-formedness. 

\begin{lemma*}{(Syntactic Framing is Semantic Framing)}
If $\judgeS{S}$ and $\judgeA{r}{\assert}$, then for all $\delta \in \interp{\Delta}$, 
$\interp{\judgeS{\Frame{S}{r}}}\;\delta = 
\Frame{\interp{\judgeS{S}}\;\delta}{\interp{\judgeA{r}{\assert}}\;\delta}$ \\
\end{lemma*}

\begin{proof}
  This proof proceeds by induction on the derivation of $S$. 

  \begin{itemize}
  \item Case \textsc{SpecTriple}: 
    \begin{tabbedproof}
      \oo We know $\judgeS{\spec{p}{c}{a:A}{q}}$ \\
      \oo By the semantics of triples, we know that 
          $\interp{\judgeS{\spec{p}{c}{a:A}{q}}}\;\delta = \spec{P}{C}{a}{Q(a)}$ \\ 
          % $\spec{p}[c}{a:A}{q}\;\delta = \spec{P}{C}{a}{Q(a)}$ \\
      \ox where \\
      \oox $P = \interp{\judgeA{p}{\assert}}\;\delta$ \\
      \oox $C = \interp{\judgeC[\restrictkind{\Delta}]{\restricttype{\Delta}}{c}{A}}\;\theta\;\gamma$ \\
      \oox $Q = \semfun{v}{(\interp{\judgeA[\Delta, a:A]{q}{\assert}}\;(\delta, v))}$ \\
      \oox $\theta = \restricttyenv{\Delta}{\delta}$ \\
      \oox $\gamma = \restrictvals{\Delta}{\delta}$ \\
      \oo Now from the definition of syntactic framing, we know that $\Frame{S}{r} = \spec{p * r}{c}{a:A}{q(a) * r}$ \\
      \oo By semantics of triples, we know that \\
      \oox $\interp{\judgeS{\spec{p * r}{c}{a:A}{q * r}}}\;\delta = \spec{P * R}{C}{a}{Q(a) * R}$ \\ 
      \ox where $R = \interp{\judgeA{r}{\assert}}\;\delta$ \\
      \oo By lemma we know that $\spec{P * R}{C}{a}{Q(a) * R} = \Frame{\spec{P}{C}{a}{Q(a)}}{R}$ 
          semantically \\
      \oo Therefore $\interp{\judgeS{\Frame{\spec{p}{c}{a:A}{q}}{r}}}\;\delta = $ \\
      \oox $\Frame{\interp{\judgeS{\spec{p}{c}{a:A}{q}}}\;\delta}{(\interp{\judgeA{r}{\assert}}\;\delta)}$
    \end{tabbedproof}

  \item Case \textsc{SpecMTriple}
    \begin{tabbedproof}
      \oo We know $\judgeS{\mspec{p}{e}{a:A}{q}}$ \\
      \oo By the semantics of triples, we know that 
          $\interp{\judgeS{\mspec{p}{e}{a:A}{q}}}\;\delta = \mspec{P}{E}{a}{Q(a)}$ \\ 
      \ox where \\
      \oox $P = \interp{\judgeA{p}{\assert}}\;\delta$ \\
      \oox $E = \interp{\judgeE[\restrictkind{\Delta}]{\restricttype{\Delta}}{e}{\monad{A}}}\;\theta\;\gamma$ \\
      \oox $Q = \semfun{v}{(\interp{\judgeA[\Delta, a:A]{q}{\assert}}\;(\delta, v))}$ \\
      \oox $\theta = \restricttyenv{\Delta}{\delta}$ \\
      \oox $\gamma = \restrictvals{\Delta}{\delta}$ \\
      \oo Now from the definition of syntactic framing, we know that $\Frame{S}{r} = \mspec{p * r}{e}{a:A}{q(a) * r}$ \\
      \oo By semantics of triples, we know that \\
      \oox $\interp{\judgeS{\mspec{p * r}{e}{a:A}{q * r}}}\;\delta = \mspec{P * R}{e}{a}{Q(a) * R}$ \\ 
      \ox where $R = \interp{\judgeA{r}{\assert}}\;\delta$ \\
      \oo By lemma we know that $\mspec{P * R}{E}{a}{Q(a) * R} = \Frame{\mspec{P}{E}{a}{Q(a)}}{R}$ 
          semantically \\
      \oo Therefore $\interp{\judgeS{\Frame{\mspec{p}{c}{a:A}{q}}{r}}}\;\delta = 
                     \Frame{\interp{\judgeS{\mspec{p}{c}{a:A}{q}}}\;\delta}{(\interp{\judgeA{r}{\assert}}\;\delta)}$
    \end{tabbedproof}

    \item Case \textsc{SpecAssert}
      \begin{tabbedproof}
        \oo We know $\judgeS{\setof{p}}$ \\
        \oo So we also know that $\Frame{\setof{p}}{r} = \setof{p}$ \\
        \oo Therefore, $\interp{\judgeS{\setof{p}}}\;\delta$ is either $\top$ or $\bot$, depending on
            whether $\interp{\judgeA{p}{\assert}}\;\delta$ is $H$ or $\emptyset$ \\
        \oo Now, we will do a case analysis on the meaning of $\setof{p}$ \\
        \ooo If $\interp{\judgeS{\setof{p}}}\;\delta = \top$ \\
        \oooo Then, since $S \subseteq \Frame{S}{R}$ for all $S$ and $R$, and $\top$ is maximal, we know that \\
        \oooox $\top = \Frame{\top}{(\interp{\judgeA{r}{\assert}}\;\delta)}$ \\
        \oooo So $\interp{\judgeS{\Frame{\setof{p}}{r}}}\;\delta = \Frame{\interp{\judgeS{\setof{p}}}\;\delta}{(\interp{\judgeA{r}{\assert}}\;\delta)}$ \\
        \ooo If $\interp{\judgeS{\setof{p}}}\;\delta = \bot$ \\
        \oooo Then, since $\bot$ is the empty set, and since $\Frame{S}{r} = \comprehend{p}{p * r \in S}$, \\
        \ooooo we know $\bot = \Frame{\bot}{\interp{\judgeA{r}{\assert}}\;\delta}$ \\
        \oooo  So we know $\interp{\judgeS{\Frame{\setof{p}}{r}}}\;\delta = \Frame{\interp{\judgeS{\setof{p}}}\;\delta}{(\interp{\judgeA{r}{\assert}}\;\delta)}$ \\
      \end{tabbedproof}

    \item Case \textsc{SpecBinary}
      \begin{tabbedproof}
        \oo We know $\judgeS{S_1 \otimes S_2}$, where $\oplus \in \setof{\specand, \specor, \specimp}$ \\
        \oo We also know that $\Frame{(S_1 \oplus S_2)}{r} = (\Frame{S_1}{r}) \oplus (\Frame{S_2}{r})$ \\
        \oo By the semantics, we know \\
        \oox $\interp{\judgeS{S_1 \oplus S_2}}\;\delta = \interp{\judgeS{S_1}}\;\delta \interp{\oplus} \; \interp{\judgeS{S_2}}\;\delta$ \\
        \oo By the semantics, we know that \\
        \oox $\interp{\judgeS{\Frame{(S_1 \oplus S_2)}{r}}}\;\delta = 
                                            (\interp{\judgeS{\Frame{S_1}{r}}}\;\delta) \interp{\oplus}
                                            (\interp{\judgeS{\Frame{S_2}{r}}}\;\delta)$ \\
        \oo By induction, we know $\interp{\judgeS{\Frame{S_1}{r}}}\;\delta = 
                                   \Frame{\interp{\judgeS{S_1}}\;\delta}{(\interp{\judgeA{r}{\assert}}\;\delta)}$ \\
        \oo By induction, we know $\interp{\judgeS{\Frame{S_2}{r}}}\;\delta = 
                                   \Frame{\interp{\judgeS{S_2}}\;\delta}{(\interp{\judgeA{r}{\assert}}\;\delta)}$ \\

        \oo Therefore, we know $\interp{\judgeS{\Frame{(S_1 \oplus S_2)}{r}}}\;\delta = $ \\
        \oox                   $(\Frame{\interp{\judgeS{S_1}}\;\delta}{(\interp{\judgeA{r}{\assert}}\;\delta)}) 
                                 \interp{\oplus} 
                                (\Frame{\interp{\judgeS{S_2}}\;\delta}{(\interp{\judgeA{r}{\assert}}\;\delta)})$ \\
        \oo Therefore we know $\interp{\judgeS{\Frame{(S_1 \oplus S_2)}{r}}}\;\delta = $ \\
        \oox $\Frame{(\interp{\judgeS{S_1}}\;\delta \;\interp{\oplus}\; \interp{\judgeS{S_2}}\;\delta)}
                    {(\interp{\judgeA{r}{\assert}}\;\delta)}$ \\
        \oo Therefore we know $\interp{\judgeS{\Frame{(S_1 \oplus S_2)}{r}}}\;\delta = $ \\
        \oox $\Frame{\interp{\judgeS{S_1 \oplus S_2}}\;\delta}{(\interp{\judgeA{r}{\assert}}\;\delta)}$ \\
      \end{tabbedproof}

    \item Case \textsc{SpecQuantify}:
      \begin{tabbedproof}
        \oo We know $\judgeS{Q u:\omega.\; S}$, where $Q \in \setof{\forall, \exists}$ \\
        \oo We also know that $\Frame{(Q u:\omega.\; S)}{r} = Q u:\omega.\; (\Frame{S}{r})$ \\
        \oo By the semantics, we know that $\interp{\judgeS{Q u:\omega.\; (\Frame{S}{r)}}}\;\delta = $ \\
        \oox $\interp{Q}_{v \in \interp{\judgeSort{\omega}}\;\delta} \interp{\judgeS[\Delta, u:\omega]{\Frame{S}{r}}}\;(\delta, v)$\\
        \oo By induction, we know that for all appropriate $\delta'$, 
               $\interp{\judgeS[\Delta, u:\omega]{\Frame{S}{r}}}\;\delta' = $ \\
        \oox $\Frame{\interp{\judgeS[\Delta, u:\omega]{S}}\;\delta'}{\interp{\judgeA[\Delta, u:\omega]{r}{\assert}}\;\delta'}$ \\
        \oo Now, if $\delta' = (\delta, v)$, then $\interp{\judgeA[\Delta, u:\omega]{r}{\assert}}\;\delta' = 
                                                   \interp{\judgeA{r}{\assert}}\;\delta$ since $u \not\in \FV{r}$ \\
        \oo So we know $\interp{\judgeS{Q u:\omega.\; (\Frame{S}{r)}}}\;\delta = $ \\
        \oox $\interp{Q}_{v \in \interp{\judgeSort{\omega}}\;\delta} (\Frame{\interp{\judgeS[\Delta, u:\omega]{S}}\;(\delta,v)}{\interp{\judgeA{r}{\assert}}\;\delta})$ \\
        \oo Since framing distributes through meets, we know $\interp{\judgeS{Q u:\omega.\; (\Frame{S}{r)}}}\;\delta = $ \\
        \oox $\Frame{(\interp{Q}_{v \in \interp{\judgeSort{\omega}}\;\delta} \interp{\judgeS[\Delta, u:\omega]{S}}\;(\delta,v))}{\interp{\judgeA{r}{\assert}}\;\delta})$ \\
        \oo So we know $\interp{\judgeS{Q u:\omega.\; (\Frame{S}{r)}}}\;\delta = $ \\
        \oox $\Frame{\interp{\judgeS{Q u:\omega.\;S}}\;\delta}{(\interp{\judgeA{r}{\assert}}\;\delta)}$ \\
        \oo So we know $\interp{\judgeS{\Frame{(Q u:\omega.\; S)}{r}}}\;\delta = $ \\
        \oox $\Frame{\interp{\judgeS{Q u:\omega.\;S}}\;\delta}{(\interp{\judgeA{r}{\assert}}\;\delta)}$ \\
        
      \end{tabbedproof}
  \end{itemize}
\end{proof}





\chapter{Proving the Correctness of Design Patterns}

\section{Introduction}

The widespread use of object-oriented languages creates an opportunity
for designers of formal verification systems, above and beyond a
potential ``target market''. Object-oriented languages have been used
for almost forty years, and in that time practitioners have developed
a large body of informal techniques for structuring object-oriented
programs called \emph{design patterns}\cite{gof}.  Design patterns
were developed to both take best advantage of the flexibility
object-oriented languages permit, and to control the potential
complexities arising from the unstructured use of these features.

This pair of characteristics make design patterns an excellent set of
benchmarks for a program logic.  First, design patterns use higher
order programs to manipulate aliased, mutable state. This is a
difficult combination for program verification systems to handle, and
attempting to verify these programs will readily reveal weaknesses or
lacunae in the program logic. Second, the fact that patterns are
intended to structure and modularize programs means that we can use
them to evaluate whether the proofs in a program logic respect the
conceptual structure of the program -- we can check to see if we need
to propagate conceptually irrelevant information out of program
modules in order to meet our proof obligations. Third, we have the
confidence that these programs, though small, actually reflect
realistic patterns of usage.

In this chapter, we give good specifications for and verify the
following programs:

\begin{itemize}
\item We prove a collection and iterator implementation, which builds
  the Java aliasing rules for iterators into its specification, and
  which allows the construction of new iterators from old ones via the
  composite and decorator patterns.  

\item We prove a general version of the flyweight pattern (also known as
  hash-consing in the functional programming community), which is a
  strategy for aggressively creating aliased objects to save memory
  and permit fast equality tests. This also illustrates the use of the
  factory pattern.

\item We prove a general version of the subject-observer pattern in a
  way that supports a strong form of information hiding between the
  subject and the observers.
\end{itemize}

% Finally, we give machine-verified proofs of the correctness of the
% iterator and flyweight patterns in the Ynot extension of Coq, and
% compare them with the paper proofs. We also see that proper treatment
% of the subject-observer pattern seems to call for the use of an
% impredicative type theory.

\section{Iterators and Composites}

The iterator pattern is a design pattern for uniformly enumerating the
elements of a collection. The idea is that in addition to a
collection, we have an auxiliary data structure called the iterator,
which has an operation $\nextiter$. Each time $\nextiter$ is
called, it produces one more element of the collection, with some
signal when all of the elements have been produced. The iterators are
mutable data structures whose invariants depend on the collection,
itself another mutable data structure. Therefore, most object oriented
libraries state that while an iterator is active, a client is only
permitted to call methods on a collection that do not change the
collection state (for example, querying the size of a collection). If
destructive methods are invoked (for example, adding or removing an
element), it is no longer valid to query the iterator again.

We also support operations to create new iterators from old ones, and
to aggregate them into composite iterators. For example, given an
iterator and a predicate, we can construct a new iterator that only
returns those elements for which the predicate returns true. This sort
of decorator takes an iterator object, and \emph{decorates} it to
yield an iterator with different behavior. Likewise, we can take two
iterators and a function, and combine them into a new,
\emph{composite} iterator that returns the result of a parallel
iteration over them both.  These sorts of synthesized iterators are found
in the \texttt{itertools} library in the Python programming language,
the Google Java collections library, or the C5 library~\cite{C5} for
C\#.

Aliasing enters into the picture, above and beyond the restrictions on
the underlying collections, because iterators are stateful
objects. For example, if we create a filtering iterator, and advance
the underlying iterator, then what the filtering iterator will return
may change. Even more strikingly, we cannot pass the same iterator
twice to a parallel iteration constructor -- the iterators must be
disjoint in order to correctly generate the two sequences of elements
to combine.

Below, we give a specification of an iterator pattern. We'll begin 
by describing the interface informally, in English, and then move on 
to giving formal specifications and explaining them. 

The interface consists of two types, one for collections, and one for
iterators. The operations the collection type supports are 1) creating
new mutable collections, 2) adding new elements to an existing
collection, and 3) querying a collection for its size. Adding new
elements to a collection is a destructive operation which modifies the
existing collection, whereas getting a collection's size does not
modify the collection.

The interface that the iterator type supports includes:
\begin{enumerate}
\item creating a new iterator on a collection,
\item destructively getting the next element from an iterator
  (returning an error value if the iterator is exhausted), and
\item operations producing new iterators from old. The 
  operations we support are:
  \begin{enumerate}
  \item a filter operation, which takes an iterator along with a
    boolean predicate, and returns an iterator which enumerates the
    elements satisfying the predicate, and
  \item a parallel map operation, which takes two iterators and a
    two-argument function, and returns an iterator which returns the
    result of enumerating the elements of the two iterators in
    parallel, and applying the function to each pair of elements.
  \end{enumerate}
\end{enumerate}

The aliasing protocol that our iterator protocol will satisfy is
essentially the same as the one the Java standard libraries specify in
their documentation.

\begin{itemize}
\item Any number of iterators can be created from a given collection.
  Each of these iterators depends on the state of the collection when
  the iterator is created, and is only valid as long as the collection
  state does not change.

  Each iterator also has its own traversal state, which tracks how 
  many of the collection's elements it has yet to yield. 

\item Iterator can be constructed from other iterators, and these
  composite iterators depend on the traversal states of all of the
  iterators they are constructed from. As a result, they also depend
  on the state of each collection each constituent iterator depends
  upon. 

\item An iterator is valid only as long as none of the collections it
  depends on have been modified, and it is the only thing that has the
  right to modify the traversal state of the iterators it depends on.

  It is legal to call functions on an iterator only when it is in a
  valid state. Performing a destructive operation on any collection an
  iterator depends upon invalidates it. For example, adding an element
  to any collection an iterator depends on will invalidate the
  iterator, as will enumerating the elements from any of the iterators
  that it depends on.

  Likewise, only the iterator itself may modify the traversal state of
  any iterator it depends upon. Any other modification may invalidate
  it.
\end{itemize}

\subsection{The Iterator Specification}

Now, we will describe the specification, given in
Figure~\ref{iterator-interface}, in detail. This whole specification
follows the usual pattern of introducing abstract types, predicates,
and operations with existential quantification, and then specifying
the behavior of the operations with a conjunction of Hoare triples.

In lines \ref{decl:colltype} and \ref{decl:itertype}, we introduce two
abstract type constructors, $\colltype$ and $\itertype$. These are
both type constructors of kind $\star \to \star$, which take an argument
that describes their element type. So $\colltype(\N)$ represents a
collection of natural numbers, and $\itertype(\bool)$ represents an
iterator which will produce a sequence of boolean values.

In lines \ref{decl:collpred} and \ref{decl:iterpred}, we give two
abstract predicates $\collpred$ and $\iterpred$, to represent the
state associated with a collection and iterator, respectively. The
sort of the collection predicate is $\Pi \alpha:\star.\;
\colltype(\alpha) \To \seqsort{\alpha} \To \assert \To \assert$, which
we will write using an expression like $\collpred_\alpha(c, xs, P)$.

The first argument is a type argument (e.g., $\alpha$), indexing the
whole predicate by the element type of the collection. (We will often
suppress this type argument when it is obvious from context.) The
second argument is an argument of collection type (in our example,
$c$, of type $\colltype(\alpha)$) which indicates the value with which
our state is associated. The third argument (here, $xs$) is the purely
mathematical sequence (i.e., an element of the free monoid over
$\alpha$) the collection $c$ represents.  The fourth, and final,
argument is a proposition-valued \emph{abstract state} of the
collection, which we use to track whether or not the collection has
been modified or not.

The appearance of this argument might be a little bit surprising:
naively, we might suppose the mathematical sequence that collection
represents constitutes a sufficient description of the collection, and
so we might expect our predicates to take on the form
$\collpred_\alpha(c, xs)$, with no state argument. However, this is
\emph{not} sufficient. The iterator contract forbids modifying the
collection at all, while iterators are active upon it, and the
mathematical sequence a collection represents is not enough
information to decide whether a collection has been modified or not.

For example, suppose we have a collection $c$ representing the
mathematical sequence $\left<2, 3, 4, 5\right>$, which might have the
predicate $\collpred_\N(c, \left<2, 3, 4, 5\right>, P)$.  Now, suppose
we first add the element $1$ to the front of the sequence (so that the
collection $c$ now represents $\left<1, 2, 3, 4, 5\right>$), and then
immediately remove the first element. Then, the collection $c$ will
still represent the sequence $\left<2, 3, 4, 5\right>$, but it will
have suffered an intervening modification.

This kind of change can be catastrophic for iterator implementations.
As a concrete example, suppose that we had represented our collection
with a balanced binary tree, and represent the iterator as a pointer
into the middle of that tree. Adding and removing elements from the
collection can cause a tree rebalancing, which can potentially leave
the iterator pointer with a stale or dangling pointer.

As a result, our specification must track whether a collection has
been modified or not, and this is what the abstract state field on the
predicate does. Operations on a collection which do not change its
underlying state will leave the abstract state unchanged from pre- to
post-condition, whereas destructive operations (such as adding or
removing elements) do change the abstract state.

On line \ref{decl:iterpred}, we assert the existence of the iterator
predicate $\iterpred$.  It is also a four-place predicate, and has
sort $\Pi \alpha:\star.\; \itertype(\alpha) \To \seqsort{\alpha} \To
\powerset{\Sigma \beta:\star.\;\colltype(\beta) \times \seqsort{\beta}
  \times \;\assert} \To \assert$, which we will write using an
expression like $\iterpred_\alpha(i, xs, S)$.

The first argument (in our example, $\alpha$) is a type argument
describing the type of elements the iterator will produce. The second
argument (here, $i$) is the concrete iterator value to which the
predicate is associated. Then, we have a sequence argument (here,
$xs$) which describes the elements yet to be produced from this
iterator -- if $xs = \left<5, 7, 9\right>$, then the next three
elements the iterator will yield are 5, 7, and 9, in that order.
Subsequently the iterator will be empty and unable to produce any
more elements.

Finally, we have a state argument for iterators, as well. In contrast
to the case for collections, this argument is a set of propositions,
representing an entire set of collection states. Since our interface
supports operations which allow building new iterators from old, an
iterator may access many different collections to produce a single new
element. As a result, we have to track the state of each collection
the iterator depends on, so that we can verify that we do not ever
need to read a modified collection.

\begin{figure}
\mbox{}
\begin{specification}
\nextlinelabel{decl:colltype}
$\exists \colltype : \star \to \star$  
\nextlinelabel{decl:itertype}
$\exists \itertype : \star \to \star$  
\nextlinelabel[0.5em]{decl:collpred}
$\exists \collpred : 
        \Pi \alpha:\star.\; \colltype(\alpha) \To \seqsort{\alpha} \To \assert \To \assert.$ 
\nextlinelabel{decl:iterpred}
$\exists \iterpred : 
        \Pi \alpha:\star.\; \itertype(\alpha) \To \seqsort{\alpha} \To \powerset{\Sigma \beta:\star.\; \colltype(\beta) \times \seqsort{\beta} \times \assert} \To \assert.$ 
\nextlinelabel[0.5em]{decl:newcoll-type}
$\exists \newcoll : 
         \forall \alpha:\star.\; \monad{(\colltype(\alpha))}.$
\nextlinelabel{decl:size-type}
 $\exists \sizecoll : 
         \forall \alpha:\star.\; \colltype(\alpha) \to \monad{\N}.$ 
\nextlinelabel{decl:add-type}
 $\exists \addcoll : 
         \forall \alpha:\star.\; \colltype(\alpha) \times \alpha \to \monad{\unittype}.$
\nextlinelabel{decl:remove-type}
 $\exists \removecoll :
         \forall \alpha:\star.\; \colltype(\alpha) \to \monad{(\opttype{(\alpha)})}$. 
\nextlinelabel[0.5em]{decl:newiter-type}
 $\exists \newiter : 
         \forall \alpha:\star.\; \colltype(\alpha) \to \monad{(\itertype(\alpha))}.$ 
\nextlinelabel{decl:filter-type}
 $\exists \filteriter : 
         \forall \alpha:\star.\; (\alpha \to \ctext{bool}) \times \itertype(\alpha) 
                                 \to \monad{(\itertype(\alpha))}.$ 
\nextlinelabel{decl:zip-type}
$\exists$\=$ \mergeiter : 
         \forall \alpha,\beta,\gamma:\star.\; (\alpha \to \beta \to \gamma) \times
                                 \itertype(\alpha) \times \itertype(\beta) 
                                   \to \monad{(\itertype(\gamma))}.$
\nextlinelabel{decl:next-type}
$\exists$\=$\nextiter : 
         \forall \alpha:\star.\; \itertype(\alpha) \to \monad{(\opttype{(\alpha)})}.$  
\nextlinelabel[0.5em]{decl:newcoll-spec}

\> $\forall \alpha.\; \mspec{\emp}{{\newcoll_\alpha}}
                                 {a:\colltype(\alpha)}{\exists P.\; \collpred_\alpha(a, \epsilon,P)}$ $\specand$ 
\nextlinelabel[0.5em]{decl:size-spec}

\> $\forall \alpha, c, P, xs.\;$\=
         $\angles{\collpred_\alpha(c, xs, P)}$
\nextline
\> \>  ${\sizecoll_\alpha(c)}$ 
\nextline
\> \>  $\angles{a:\N.\; \collpred_\alpha(c, xs, P) \land a = |xs|}$  $\specand$ 
\nextlinelabel[0.5em]{decl:add-spec}

 \> $\forall \alpha, c, P, x, xs.\;$\=
               $\angles{\collpred_\alpha(c, xs, P)}$ 
\nextline
\>\>         ${\addcoll_\alpha(c, x)}$
\nextline
\>\>         $\angles{a:1.\; \exists Q.\; \collpred_\alpha(c, x\cdot xs, Q)}$ $\specand$ 

\nextlinelabel[0.5em]{decl:remove-spec-1}

\> $\forall \alpha, c, P.\; \mspec{\collpred_\alpha(c, \epsilon, P)}
                                        {{\removecoll_\alpha(c)}}
                                        {a:\opttype{(\alpha)}}
                                        {\collpred_\alpha(c, \epsilon, P) \land a = \None}$
$\specand$ 
\nextlinelabel{decl:remove-spec-2}
\> $\forall \alpha, c, x, xs, P.$\=
          $\angles{\collpred_\alpha(c, x\cdot xs, P)}$ 
\nextline
\>\> ${\removecoll_\alpha(c)}$ 
\nextline
\>\> $\angles{a:\opttype{(\alpha)}.\;
             \exists Q.\;\collpred_\alpha(c, xs, Q) \land a = \Some(x)}$ 
\nextlinelabel[0.5em]{decl:newiter-spec}

\> $\specand \forall \alpha, c, P, xs.\;$\=
            $\angles{\collpred_\alpha(c, xs, P)}$ 
\nextline
\>\>${\newiter_\alpha(c)}$
\nextline
\>\>$\angles{a:\itertype(\alpha).\; \collpred(c, xs, P) * \iterpred(a, xs, \setof{(\alpha, c, xs, P)})}$ $\specand$ 
\nextlinelabel[0.5em]{decl:filter-spec}

 \> $\forall \alpha, p, i, S, xs.\;$\=
         $\angles{\iterpred_\alpha(i, xs, S)}$ 
\nextline
\>\>   ${\filteriter_\alpha(p, i)}$
\nextline
\>\>   $\angles{a:\itertype(\alpha).\; \iterpred_\alpha(a, \mathit{filter}\; p\;xs, S)}$ $\specand$ 
\nextlinelabel[0.5em]{decl:zip-spec}

 \> $\forall \alpha, \beta, \gamma, f, i, S,$\=$ xs, i', S', xs'.\;$ 
\nextline
 \> \> 
     $\angles{\iterpred_\alpha(i, xs, S) * \iterpred_\beta(i', xs', S')}$ % -- don't need this condition \land S \cap S' = \emptyset}$ 
\nextline
 \> \> ${\mergeiter_{\alpha\;\beta\;\gamma}(f, i, i')}$ 
\nextline
 \> \> $\angles{a:\itertype(\gamma).\; \iterpred_{\gamma}(a, \mathit{map2}\;f\;xs\;xs', S \cup S')}$ $\specand$ 
\nextlinelabel[0.5em]{decl:next-spec-1}

 \> $\forall \alpha, i, S.\;$\=
      $\angles{\mathit{colls}(S) * \iterpred_\alpha(i, \epsilon, S)}$
\nextline  
\>\>${\nextiter_\alpha(i)}$ 
\nextline
\>\>$\angles{a:\opttype{(\alpha)}.\; \mathit{colls}(S) * \iterpred_\alpha(i, \epsilon, S) \land a = \None}$ $\specand$ 
\nextlinelabel[0.5em]{decl:next-spec-2}

 \> $\forall \alpha, i, $\=$ S, x, xs.\;$ \= 
      $\angles{\iterpred_\alpha(i, x \cdot xs, S) * \mathit{colls}(S)}$
\nextline
\>\>\>${\nextiter(i)}$
\nextline
\>\>\>$\angles{a:\opttype{(\alpha)}.\; 
              \iterpred_\alpha(i, xs, S) \land a = (\Some\;x) * \mathit{colls}(S)}$ 

\end{specification}
\caption{Interface to the Iterator Library}
\label{iterator-interface}
\end{figure}


%%%%%%%%%%%%%%%%%%%%%%%%%%%%%%%%%%%%%%%%%%%%%%%%%%%%%%%%%%%%%%%%


\begin{figure}
\mbox{}
\begin{specification}
\nextline
$\mathit{colls}(\emptyset) \qquad\qquad\qquad\qquad\;\;$ \=$\equiv\;$\= $\emp$ 
\nextline
$\mathit{colls}(\setof{(\alpha, c,xs , P)} \uplus S)$ \> $\equiv$ \> $\collpred_\alpha(c, xs, P) * \mathit{colls}(S)$ 

\nextlinelabel[0.5em]{decl:mathfilter-impl}
$\mathit{filter}\;p\;\epsilon \qquad\quad$\=$\equiv \epsilon$
\nextline
$\mathit{filter}\;p\;(x\cdot xs)$ \> $\equiv \ctext{if}\;p\;x = \ctext{true}\;\ctext{then}\;
                                         x\cdot(\mathit{filter}\;p\;xs)\;
                                      \ctext{else}\; \mathit{filter}\;p\;xs$ 

\nextlinelabel[0.5em]{decl:map2-impl}
$\mathit{map2}\;f\;\epsilon\;ys \qquad\qquad\qquad $\=$= \epsilon$ 
\nextline
$\mathit{map2}\;f\;xs\;\epsilon $\>$= \epsilon$ 
\nextline
$\mathit{map2}\;f\;(x\cdot xs)\;(y\cdot ys)$ \>$= (f\;x\;y)\cdot(\mathit{map2}\;f\;xs\;ys)$
    
\end{specification}
\caption{Auxilliary Functions Used in the Iterator Specification}
\label{iterator-interface-aux}
\end{figure}

The operation $\newcoll$ is declared on line \ref{decl:newcoll-type},
and specified on line \ref{decl:newcoll-spec}. The specification
asserts that the call $\newcoll_\alpha$ may happen from any
precondition state, and that it creates and adds a new, empty
collection to the program state. The postcondition assertion $\exists
P.\; coll_\alpha(a, \epsilon, P)$ says that the return value $a$ is a
collection representing the empty sequence $\epsilon$, and that the
collection begins its life in some arbitrary abstract state $P$.

The $\sizecoll_\alpha(c)$ function, which is declared on line
\ref{decl:size-type} and specified on line \ref{decl:size-spec}, takes
a type argument $\alpha$ and a collection $c$, and returns the number
of elements in $c$. To call this function, we must have access to the
collection $\collpred_\alpha(c, xs, P)$ in our precondition, and it is
returned to us unchanged in the postcondition, with the return value
$a$ equal to the length of $xs$, the sequence $c$ represents. In
particular, note that the abstract state $P$ of the $coll(c, xs, P)$
predicate remains unchanged in the pre- and post-conditions,
indicating that this function does not change the abstract state.

The function call $\addcoll_\alpha(c, x)$, which adds an element $x$
to a collection $c$, is declared on line \ref{decl:add-type} and is
specified on line \ref{decl:add-spec}. We start with a precondition
$\collpred_\alpha(c, xs, P)$ and move to a postcondition state
$\exists Q.\; \collpred_\alpha(c, x\cdot xs, Q)$. Because we
destructively modify the collection $c$ when we add $x$ to it, we also
specify that the abstract state in the postcondition is existentially
quantified. This ensures that clients cannot assume that the abstract
state remains the same after a call to $\addcoll$ has been made. In
this way, the abstract state behaves a bit like a time stamp, changing
to some new state whenever a modification is made to the collection. 

Similarly, the function call $\removecoll_\alpha(c)$ (declared on
line \ref{decl:remove-type}) removes an element from the collection
$c$. We give this procedure two specifications, on lines
\ref{decl:remove-spec-1} and \ref{decl:remove-spec-2}, corresponding
to when the collection is empty, or not.  In the first specification
(on line \ref{decl:remove-spec-1}), we begin with the precondition
$\collpred_\alpha(c, \epsilon, P)$, and end in the postcondition
$\collpred_\alpha(c, \epsilon, P) \land a = \None$. The fact that the
abstract state remains $P$ means the collection is unchanged, and the
return value $a$ equals $\None$, an element of option type, indicating
that there was no element to remove.  In the second specification (on
line \ref{decl:remove-spec-2}), we begin with the precondition
$\collpred_\alpha(c, x\cdot xs, P)$, from which we can see that the
collection is nonempty. Then, the postcondition is $\exists Q.\;
\collpred_\alpha(c, xs, Q) \land a = \Some(x)$.  The value $\Some(x)$
is returned as the return value of the function, and the state of the
collection changes to reflect that the element $x$ has been removed
--- including a change to the abstract state of the collection.

As an aside, in practice it is usually more convenient to specify a
procedure with a single Hoare triple, rather than multiple Hoare
triples. However, in this example, I choose to give multiple
specifications of the same procedure in order to illustrate that it is
indeed possible within specification logic.

The $\newiter_\alpha(c)$ function is declared on line
\ref{decl:newiter-type}.  Its type is $\forall \alpha:\star.\;
\colltype(\alpha) \to \monad{(\itertype(\alpha))}$.  This means that
it is given a type and a collection of that type, and then it returns
an iterator over that type, possibly creating auxilliary data structures.

A call $\newiter_\alpha(c)$ is specified on line
\ref{decl:newiter-spec}, and beginning from a precondition state
$\collpred_\alpha(c, xs, P)$, it goes to a postcondition state
$\collpred_\alpha(c, xs, P) * \iterpred_\alpha(a, xs, \setof{(c, xs, P)})$.
This means that given access to a collection $c$, our function will
return an iterator object (bound to $a$), which will enumerate the
elements of $c$ (that is, it will produce the elements
$xs$). Furthermore, the abstract state that it depends on is just the
singleton set $\setof{P}$, since this iterator will read only $c$.
Finally, the fact that $\collpred_\alpha(c, xs, P)$ occurs in both
the pre- and the post-condition means that this function needs 
access to $c$'s state, but does not modify its abstract state. 

The $\filteriter_\alpha(p, i)$ (declared on line \ref{decl:filter-type})
takes a boolean function $p$ and an iterator $i$, and returns a new
iterator which will enumerate only those elements which for which $p$
returns true. This function is specified on line \ref{decl:filter-spec}, 
and it takes a precondition $\iterpred_\alpha(i, xs, S)$ to a postcondition
$\iterpred_\alpha(a, \mathit{filter}\;p\;xs, S)$. 

First, note that we use a mathematical function $\mathit{filter}$ to
explain the filtering behavior in terms of sequences. Second, note
that the original iterator state $\iterpred_\alpha(i, xs, S)$ vanishes
from the postcondition -- it is consumed by the call the
$\filteriter$.  This reflects the fact the filtered iterator takes
ownership of the underlying iterator, in order to prevent third
parties from making calls to $\nextiter(i)$ and possibly changing the
state of the filtered iterator.

This is also why the support set $S$ for an iterator only needs to
track the abstract states of the collections, rather than tracking the
state of both collections and iterators. When we take ownership of the
argument's iterator state, we prevent third parties from being able to
call functions on the argument after creating the new iterator. This
takes advantage of the resource-conscious nature of separation logic:
a specification must have access to its footprint, and so we can hide
state inside a predicate to control which operations are allowed.

The $\mergeiter$ function is declared on line $\ref{decl:zip-type}$, and
has type $\forall \alpha,\beta,\gamma:\star.\; (\alpha \to \beta \to \gamma) \times \itertype(\alpha) \times
\itertype(\beta) \to \monad{(\itertype(\gamma))}$.  Thus,
a call $\mergeiter_{\alpha\;\beta\;\gamma}(f, i, i')$ takes a function and two 
iterators, and constructs a new iterator which steps over the two inputs in parallel,
returning the result of $f$ applied to each pair of elements of $i$ and $i'$. 

We specify calls $\mergeiter_{\alpha\;\beta\;\gamma}(f, i_1, i_2)$ on line
$\ref{decl:zip-spec}$, and it takes a precondition
$\iterpred_\alpha(i_1, xs, S_1) * \iterpred_\beta(i_2, ys, S_2)$.
This means that we have state associated with two separate iterators,
which we take to the postcondition $\iterpred_{\gamma}(a,
\mathit{map2}\;f\;xs\;ys, S_1 \cup S_2)$. As with the $\filteriter$ function, 
we consume the two input iterators to produce the return value iterator. 
And also as with $\filteriter$, we use a mathematical function $\mathit{map2}$
to specify the action on mathematical sequences. 

One point worth noting is that it is important that the two argument
iterators have separate state from one another. In a functional
program, there is no difficulty with a program
$\mathit{map2}\;f\;xs\;xs$, because we are free to re-traverse a list
multiple times. However, since traversing an iterator is a destructive
operation, a call like $\mergeiter_{\alpha,\alpha,\beta}\;f\;i\;i$
could (if it were allowed) give the wrong answer, for example by
pairing consecutive elements of the iterator.

The final operation in our interface is the $\nextiter$ function,
declared on line \ref{decl:next-type}. The type of this function
is $\forall \alpha:\star.\; \itertype(\alpha) \to
\monad{(\opttype{(\alpha)})}$.  When invoked, it will return an
option, with the $\None$ value if the iterator is exhausted,
and $\Some$ of an element if the iterator still has elements to
produce.

As with $\removecoll$, we specify this procedure with two
specifications, one for the case when the iterator is empty and
another for when it is non-empty. On line \ref{decl:next-spec-1}, we
give the specification for when the iterator is exhausted, and
on line \ref{decl:next-spec-2}, we give the specification for when
the iterator is not exhausted.  

In either case, the precondition for the function contains as one part
the predicate $\mathit{colls}(S)$.  The assertion-level function
$\mathit{colls}(S)$ is a function that iterates over a set of abstract
states, and re-associates them with collection predicates (coming from
the argument $S$) in the precondition, to form a predicate
$\collpred_{\tau_1}(c_1, xs_1, S_1) * \ldots * \collpred_{\tau_n}(c_n,
xs_n, S_n)$.  This expresses the requirement that we need access to
\emph{all} of the collections $i$ depends on, all in the correct
abstract state, in order to use it. 
This function is defined in
Figure~\ref{iterator-interface-aux}.\footnote{Technically, this is an
  abuse of notation, since primitive recursion is not well defined on
  sets.  The proper way to do this would be to introduce a binary
  relation $\mathit{colls(S, R)}$ between state sets and predicates,
  and to put $R$ in the precondition state. However, since the
  separating conjunction is commutative, no confusion is possible and
  I will retain the functional form.}

In line \ref{decl:next-spec-1}, $\mathit{colls}(C, S)$ is joined with 
the specification of the iterator, $\iterpred_\alpha(i, \epsilon, S)$. Note
that the same $S$ is used, so that we are referring only to the collections
the iterator may need to read.  As expected, $\nextiter_\alpha(i)$ returns 
$\None$.  On the other hand, if the iterator still has elements (i.e., is in a state
$iter_\alpha(i, x\cdot xs, S)$), we use the specification on line \ref{decl:next-spec-2}, 
and see it returns the first element as
$\Some\;x$, and sets the state to $iter(i, xs, S)$ in the
postcondition (line 15). 

\subsubsection{Example Client}

Below, we give an example use of this module in annotated program
style. (Here, and in what follows, we suppress explicit type
applications when they are obvious in context.)

\begin{specification}
\nextline
 $\setof{\emp}$  \nextlinelabel{label:iterclient:makec1}
 $\letv{c_1}{\newcoll()}{}$  \nextline
 $\setof{\exists P'_1.\; coll(c_1, \epsilon, P'_1)}$  \nextlinelabel{label:iterclient:dropp1ex}
 $\setof{coll(c_1, \epsilon, P_1)}$  \nextlinelabel{label:iterclient:addc1}
 $\letv{()}{\addcoll(c_1, 4)}{}$  \nextline
 $\setof{\exists P_2.\; coll(c_1, 4\cdot\epsilon, P_2)}$  \nextlinelabel{label:iterclient:dropp2ex}
  $\setof{coll(c_1, 4\cdot\epsilon, P_2)}$  \nextlinelabel{label:iterclient:addc1again}
 $\letv{()}{\addcoll(c_1, 3)}{}$  \nextline
 $\letv{()}{\addcoll(c_1, 2)}{}$  \nextline
 $\setof{coll(c_1, 2\cdot3\cdot4\cdot\epsilon, P_4)}$  \nextlinelabel{label:iterclient:makec2}
 $\letv{c_2}{\newcoll()}{}$  \nextline
 $\letv{()}{\addcoll(c_2, 3)}{}$  \nextlinelabel{label:iterclient:finishc2}
 $\letv{()}{\addcoll(c_2, 5)}{}$  \nextlinelabel{label:iterclient:c2state}
 $\setof{coll(c_2, 5\cdot3\cdot\epsilon, Q_2) * coll(c_1, 2\cdot3\cdot4\cdot\epsilon, P_4)}$  \nextlinelabel{label:iterclient:makei1}
 $\letv{i_1}{\newiter(c_1)}{}$  \nextlinelabel{label:iterclient:i1pred}
 $\{$\=$iter(i_1, 2\cdot3\cdot4\cdot\epsilon, \setof{(c_1, 2\cdot3\cdot4\cdot\epsilon, P_4)})$
 \nextline 
 \>$*\; coll(c_2, 5\cdot3\cdot\epsilon, Q_2) * coll(c_1, 2\cdot3\cdot4\cdot\epsilon, P_4)\}$  \nextlinelabel{label:iterclient:makeevens}
 $\letv{i'_1}{\filteriter(even?, i_1)}{}$  \nextlinelabel{label:iterclient:evenspred}
 $\{$\=$iter(i'_1, 2\cdot4\cdot\epsilon, \setof{(c_1, 2\cdot3\cdot4\cdot\epsilon, P_4)})$  \nextline
  \>$*\; coll(c_2, 5\cdot3\cdot\epsilon, Q_2) * coll(c_1, 2\cdot3\cdot4\cdot\epsilon, P_4)\}$ \nextlinelabel{label:iterclient:makei2}
 $\letv{i_2}{\newiter(c_2)}{}$  \nextlinelabel{label:iterclient:i2state}
 $\{$\=$iter(i'_1, 2\cdot4\cdot\epsilon, \setof{(c_1, 2\cdot3\cdot4\cdot\epsilon, P_4)})$ \nextline
  \>$*\;iter(i_2, 5\cdot3\cdot\epsilon, \setof{(c_2, 5\cdot3\cdot\epsilon, Q_2)})$  \nextline
 \>$*\;coll(c_2, 5\cdot3\cdot\epsilon, Q_2) * coll(c_1, 2\cdot3\cdot4\cdot\epsilon, P_4)\}$  \nextlinelabel{label:iterclient:mapi}
 $\letv{i}{\ctext{merge}(plus, i'_1, i_2)}{}$  \nextline
 $\{$\=$iter(i, 7\cdot7\cdot\epsilon, \setof{(c_1, 2\cdot3\cdot4\cdot\epsilon, P_4), (c_2, 5\cdot3\cdot\epsilon, Q_2)})$  \nextline
 \> $*\; coll(c_2, 5\cdot3\cdot\epsilon, Q_2) * coll(c_1, 2\cdot3\cdot4\cdot\epsilon, P_4)\}$  \nextlinelabel{label:iterclient:sizecoll}
 $\letv{n}{\sizecoll(c_2)}{}$  \nextline
 $\{$\=$n = 2 \;\land
iter(i, 7\cdot7\cdot\epsilon, \setof{(c_1, 2\cdot3\cdot4\cdot\epsilon, P_4), (c_2, 5\cdot3\cdot\epsilon, Q_2)})$ \nextline
\>$*\;coll(c_2, 5\cdot3\cdot\epsilon, Q_2) * coll(c_1, 2\cdot3\cdot4\cdot\epsilon, P_4)\}$  \nextlinelabel{label:iterclient:nexti}

 $\letv{x}{\nextiter(i)}{}$  \nextline
 $\{$\= $n = 2 \land x = \ctext{Some\; }7 \;\land $ \nextlinelabel{label:iterclient:istate}
 \> $iter(i, 7\cdot\epsilon, \setof{(c_1, 2\cdot3\cdot4\cdot\epsilon, P_4), (c_2, 5\cdot3\cdot\epsilon, Q_2)})$  \nextline
 \> $*\; coll(c_2, 5\cdot3\cdot\epsilon, Q_2) * coll(c_1, 2\cdot3\cdot4\cdot\epsilon, P_4)\}$  \nextlinelabel{label:iterclient:add}
 $\addcoll(c_2, 17)$  \nextline
 $\{$\= $n = 2 \land x = \ctext{Some\; }7 \;\land $ \nextline
 \> $iter(i, 7\cdot\epsilon, \setof{(c_1, 2\cdot3\cdot4\cdot\epsilon, P_4), (c_2, 5\cdot3\cdot\epsilon, Q_2)})$  \nextlinelabel{label:iterclient:q3state}
 \> $* \; (\exists Q_3.\; coll(c_2, 17\cdot5\cdot3\cdot\epsilon, Q_3)) * coll(c_1, 2\cdot3\cdot4\cdot\epsilon, P_4)\}$  \nextline
 $\{$\= $\exists Q_2, Q_3, P_4. n = 2 \land x = \ctext{Some\; }7 \;\land $ \nextline
 \> $iter(i, 7\cdot\epsilon, \setof{(c_1, 2\cdot3\cdot4\cdot\epsilon, P_4), (c_2, 5\cdot3\cdot\epsilon, Q_2)})$  \nextline 
 \> $* \; coll(c_2, 17\cdot5\cdot3\cdot\epsilon, Q_3) * coll(c_1, 2\cdot3\cdot4\cdot\epsilon, P_4)\}$  
\end{specification}


In line 1 of this example, we begin in an empty heap. In
line~\ref{label:iterclient:makec1}, we create a new collection $c_1$,
which yields us the state $\exists P'_1.\; coll(c_1, \epsilon, P'_1)$,
with an existentially quantified abstract state.

Because $P'_1$ is existentially quantified, we do not know what value
it actually takes on. However, we can drop the existential using the
\textsc{AxForget} axiom, which says that if we prove the rest of the
program using a freshly-introduced variable $P_1$, then we know that
the rest of the program will work for \emph{any} value of $P_1$,
because free variables are implicitly universally quantified.  So it
will work with whatever value $P'_1$ had. So we drop the quantifier on
line~\ref{label:iterclient:dropp1ex}, and try to prove this program
with the universally-quantified $P_1$.\footnote{A useful analogy is
  the existential elimination rule in the polymorphic lambda calculus:
  we prove that we can use an existential by showing that our program
  is well-typed no matter what the contents of the existential are.}

This permits us to $\addcoll$ the element 4 to $c_1$ on
line~\ref{label:iterclient:addc1}. Its specification puts the
predicate $coll()$ on line 6 again into an existentially quantified
state $P_2$. So we again replace $P_2$ with a fresh variable $P_2$ on
line~\ref{label:iterclient:dropp2ex}, and will elide these existential introductions and unpackings
henceforth.

Starting on line~\ref{label:iterclient:addc1again}, we add two more
elements to $c_1$, and on
lines~\ref{label:iterclient:makec2}-\ref{label:iterclient:finishc2},
we create another collection $c_2$, and add $3$ and $5$ to it, as can
be seen in the state predicate on
line~\ref{label:iterclient:c2state}. On
line~\ref{label:iterclient:makei1}, we create the iterator $i_1$ on
the collection $c_1$. The $iter$ predicate on
line~\ref{label:iterclient:i1pred} names $i_1$ as its value, and lists
$c_1$ in state $P_4$ as its support, and promises to enumerate the
elements 2, 3, and 4.

On line~\ref{label:iterclient:makeevens}, $\filteriter(even?, i_1)$
creates the new iterator $i'_1$. This iterator yields only the even
elements of $i_1$, and so will only yield 2 and 4. On
line~\ref{label:iterclient:evenspred}, $i_1$'s iterator state has been
consumed to make $i'_1$'s state. We can no longer call
$\nextiter(i_1)$, since we do not have the resource invariant needed
to prove anything about that call. Thus, we cannot write a program
that would break $i'_1$'s representation invariant.

On line~\ref{label:iterclient:makei2}, we create a second iterator
$i_2$ enumerating the elements of $c_2$. The state on line~\ref{label:iterclient:i2state} now has
predicates for $i'_1$, $i_2$, $c_1$ and $c_2$. On line~\ref{label:iterclient:mapi},
$\ctext{merge}(plus, i'_1, i_2)$ creates a new iterator $i$, which
produces the pairwise sum of the elements of $i'_1$ and $i_2$, and
consumes the iterator states for $i'_1$ and $i_2$ to yield the state
for the new iterator $i$. Note that the invariant for $i$ does not
make any mention of what it was constructed from, naming only the
collections it needs as support.
%Furthermore, the support of $i$ is the union of the
%supports of $i'_1$ and $i_2$ -- namely, the two collections $c_1$ and
%$c_2$.

On line~\ref{label:iterclient:sizecoll}, the $\sizecoll$ call on $c_2$
illustrates that we can call non-destructive methods while iterators
are active. The call to $\nextiter(i)$ on
line~\ref{label:iterclient:nexti} binds $\ctext{Some\; }7$ to $x$, and
the the iterator's sequence argument
(line~\ref{label:iterclient:istate}) shrinks by one element. On
line~\ref{label:iterclient:add}, we call $\addcoll(c_2, 17)$ the state
of $c_2$ changes to $\exists Q_3.\; coll(c, 17\cdot 5 \cdot
3\cdot\epsilon, Q_3)$ (line~\ref{label:iterclient:q3state}). So we can
no longer call $\nextiter(i)$, since it needs $c_2$ to be in the state
$Q_2$. (On the following line, we repack all of the existentials, in
accordance with the \textsc{AxForgetEx} rule of the program logic.)

\textbf{Discussion.} This example shows a pleasant synergy between
higher-order quantification and separation logic. We can give a
relatively simple specification to the clients of the collection
library, even though the internal invariant is quite subtle (as we
will see in the next section, it will use the magic wand). Higher-order
logic also lets us freely define new data types, and so our
specifications can take advantage of the pure, non-imperative nature
of the mathematical world, as can be seen in the specifications of the
$\filteriter$ and $\ctext{merge}$ functions -- we can use equational
reasoning on purely functional lists in our specifications, even
though our algorithms are imperative.


\subsection{Example Implementation}

In this subsection, we describe one particular implementation of
iterators, based on a simple linked list implementation. The type and
predicate definitions are given in Figure~\ref{iterator-pred-impl},
and the implementation of the procedures is given in
Figure~\ref{iterator-implementation}. 


\subsubsection{Definitions of the Types and Predicates}

We'll begin by giving an intuitive explanation of the predicates,
before giving correctness proofs for the operations.

Starting on line \ref{decl:colltype-impl} of
Figure~\ref{iterator-pred-impl}, we define the type of
collections. Technically, these are recursive type definitions, which
we did not define in our semantics in Chapter 1. Fortunately, there is
no great difficulty in these definitions --- we are giving polynomial
data types, and we can justify these definition via the fact that for
every polynomial functor $F$, the category of $F$-algebras over CPO
has an initial object (which means that the data type and primitive
iteration over it are well-defined).

The type of collections $\colltype(\tau)$ is a mutable linked list,
consisting of a reference to a value of type $\Listcontent(\tau)$. A
list content is either a $\Nil$ value, or a cons cell $\Cons(x, tl)$
consisting of a value of type $\tau$ and a tail list of type
$\colltype(\tau)$. Unlike the typical definition of purely functional
linked lists in ML, the tails of a list are mutable references, rather
than list values.

The type of iterators, given in line \ref{decl:itertype-impl}, is an
inductive datatype, with one clause for each of the possible ways to
construct a new iterator from an old one. This type arises as
follows. When we filter an iterator (or merge two iterators), we
simply store a function together with the iterator (or two iterators)
that are given as inputs. For an iterator into a single collection, we
store an interior pointer into the argument list, giving us the type
of a pointer to a list as the type of the $\One$ constructor.

Then, on line \ref{decl:listpred-impl}, we define the auxilliary
predicate $\listpred(\tau, c, xs)$, which asserts that $c$ is a list
value representing the mathematical sequence $xs$, with element type
$\tau$. This is defined by recursion over the sequence $xs$, with the
empty sequence represented by a pointer to $\Nil$, and a sequence 
$y\cdot ys$ represented by a pointer to a cons cell whose head is $x$
and whose tail is a collection representing $ys$. 

\begin{figure}
\mbox{}
\begin{specification}
\nextlinelabel{decl:colltype-impl}
$\Listcontent(\alpha) = \Nil \bnfalt \Cons\; \alpha \times \reftype{(\Listcontent(\alpha))}$ \\
\> $\listtype(\alpha) = \reftype{(\Listcontent{(\alpha)})}$
\nextline
$\ctext{colltype}(\alpha) = \listtype{\alpha}$ 
\nextlinelabel[0.5em]{decl:itertype-impl}
$\itertype{(\alpha)} $\=$=  \One\;\reftype{(\colltype{(\alpha)})}$ \nextline
                       \>$ \bnfalt 
                          \Filter\;(\alpha \to bool) \times \itertype{(\alpha)}$ \nextline
                       \>$\bnfalt \Merge\; \exists \beta, \gamma:\star.\; (\beta \to \gamma \to \alpha) \times \itertype(\beta) \times \itertype(\gamma)$
\nextlinelabel[0.5em]{decl:listpred-impl}
$\listpred(\tau, c, xs)$ \qquad\qquad\qquad\= $\equiv \exists v.\; c \pointsto_\tau v * \listcontentpred(\tau, v, xs)$ 
\nextline
$\listcontentpred(\tau, \Nil, xs)$ \> $\equiv$ \= $xs = \epsilon$ 
\nextline
$\listcontentpred(\tau, \Cons(y,c), xs)$ \> $\equiv$ \> 
   $\exists ys.\; xs = y\cdot ys \land \listpred(\tau, c, ys)$ 
\nextlinelabel[0.5em]{decl:collpred-impl}
$\collpred_\tau$\=$(c, xs, P) \equiv list(\tau, c, xs) \land P \land \mbox{exact}(P)$ 

\nextlinelabel[0.5em]{decl:iterpred-impl}
$\iterpred_\tau(\One\;i, xs, \setof{(\tau, c, ys, P)}) \qquad\qquad$\=$\equiv \exists c'.\;$\=$i \pointsto c' \;*$ \nextline
                                                                   \>                      \>$(\collpred(c, ys, P) \wand $\=$(\collpred(c, ys, P) \;\land $\nextline
                                                                   \>                      \>                             \>$(\top * \listpred(c', xs))))$ 
\nextline
$\iterpred_\tau(\One\;i, xs, S) \qquad\qquad$\>$\equiv \bot$ when $|S| \not= 1$
\nextline

$\iterpred_\tau(\Filter(p, i), xs, S)$ \>$\equiv 
  \exists ys.\; \iterpred_\tau(i, ys, S) \land xs = \mathit{filter}\;p\;ys$ 
\nextline
$\iterpred_{\tau}(\Merge(\tau_1, \tau_2, f, i_1, i_2), xs, S)$ \>$\equiv 
  \exists S_1, ys, S_2, zs.$ 
\nextline
  \> \qquad \= $\iterpred_{\tau_1}(i_1, ys, S_1) * \iterpred_{\tau_2}(i_2, zs, S_2) \;\land$
\nextline \> \>$S = S_1 \cup S_2 \land xs = \mathit{map2}\;f\;ys\;zs$ 
\end{specification}
\caption{Type and Predicate Definitions of the Iterator Implementation}
\label{iterator-pred-impl}
\end{figure}

\begin{figure}
\mbox{}
\begin{specification}
\nextlinelabel[0.5em]{decl:newcoll-impl}
$\newcoll_\alpha \equiv \comp{\newref{\alpha}{\Nil}}$ 
\nextlinelabel[0.5em]{decl:sizecoll-impl}
$\sizecoll_\alpha(c) \equiv $\=
         $[$\=$\letv{p}{\comp{!c}}{}$ 
\nextline\> \>$\;
               \Run\; \Listcase($\=$p,$ 
\nextline\> \>                   \>$\Nil \to \comp{0},$
\nextline\> \>                   \>$\Cons(\_, tl) \to 
                                     \comp{\letv{n}{\sizecoll_\alpha(tl)}{n+1}})]$

\nextlinelabel[0.5em]{decl:addcoll-impl}
$\addcoll_\alpha(c, x) \equiv [$
          \=$\letv{p}{\comp{!c}}{}$ 
\nextline \>$\letv{c'}{\newref{\Listcontent(\alpha)}{p}}{}$ 
\nextline \>$c := \Cons(x, c')]$ 

\nextlinelabel[0.5em]{decl:removecoll-impl}
$\removecoll_\alpha(c) \equiv [$\=
            $\letv{p}{\comp{!c}}{}$ 
\nextline \>$\Run\;\Listcase($\=$p,$ 
\nextline \>                                \>$\Nil \to \comp{\None},$ 
\nextline \>                                \>$\Cons(x, c') \to [$\=$\letv{p'}{\comp{!c'}}{}$ 
\nextline \>                                \> \>$\letv{\_}{\comp{c := p'}}{}$ 
\nextline \>                                \> \>$\Some(x)])]$

\nextlinelabel[0.5em]{decl:newiter-impl}
$\newiter_\alpha(c) \equiv \comp{\letv{i}{\comp{\newref{\colltype(\alpha)}{c}}}{\One(i)}}$

\nextlinelabel[0.5em]{decl:filteriter-impl}
$\filteriter_\alpha(p, i) \equiv \comp{\Filter(p, i)}$

\nextlinelabel[0.5em]{decl:mergeiter-impl}
$\mergeiter_\alpha(f, i, i') \equiv \comp{\Merge(f, i, i')}$ 

\nextlinelabel[0.5em]{decl:next-one-impl}
$\nextiter_\alpha(\One\;i) \equiv [$\=$\letv{c}{\comp{!i}}{}$ 
\nextline \> $\letv{p}{\comp{!c}}{}$ 
\nextline \> $\Run\;\Listcase($\=$p,$ 
\nextline \>                                  \>$\Nil \to \comp{\None},$ 
\nextline \>                                  \>$\Cons(x, c') \to [$\=$\letv{\_}{\comp{i := c'}}{\Some(x)}])]$ 
\nextlinelabel{decl:next-filter-impl}
$\nextiter_\alpha(\Filter(p, i)) \equiv [$\= 
           $\letv{v}{\nextiter_\alpha(i)}$ 
\nextline\>$\Run\;\Listcase($\=$v,$ 
\nextline\>\> $\None \to \comp{\None},$ 
\nextline\>\> $\Some\;x \to \IfThenElse{p\;x}{\comp{v}}{\nextiter_\alpha(\Filter(p,i))})]$ 
\nextlinelabel{decl:next-merge-impl}
$\nextiter_\alpha(\Merge(\beta, \gamma, f, i_1, i_2)) \equiv [$\=
            $\letv{x_1}{\nextiter_\beta(i_1)}{}$ 
\nextline\> $\letv{x_2}{\nextiter_\gamma(i_2)}{}$ 
\nextline\> $\Optcase($\=$x_1,$ 
\nextline\> \>           $\None \to \None,$ 
\nextline\> \>           $\Some\;v_1 \to \Optcase($\=$x_2,$ 
\nextline\> \> \>                                 $\None \to \None,$
\nextlinelabel{decl:next-impl-end}\> \> \>                                 $\Some\;v_2 \to \Some(f\;v_1\;v_2)))]$
                                                   
\end{specification}
\caption{Implementation of Collections and Iterators}
\label{iterator-implementation}
\end{figure}

The collection predicate $\collpred_\tau(c, xs, P)$, defined on line
\ref{decl:collpred-impl}, makes use of the list predicate. In addition
to asserting that the value $c$ represents the sequence $xs$, it
asserts two further things. First, it says that this program state is
also described by the abstract predicate $P$, and that this predicate
is an \emph{exact} predicate.

Exact predicates are predicates that hold of exactly one heap: that
is, they are the atomic elements of the lattice of assertions. This
means that they uniquely identify a heap data structure. This property
lets us track modifications to the collection: any change to the actual
heap structure will result in the falsification of $P$.  

As a modal operator on separation logic assertions, exactness can be
\emph{defined} in higher-order separation logic with the following
definition:
\begin{displaymath}
\mathrm{Exact}(P) \triangleq \forall Q:\assert.\; (P\land Q) \to (P * (\emp \land (P \wand P \land Q))
\end{displaymath}
This definition has the benefit of concisely demonstrating why we are
interested in exactness: when $P$ is exact, we can ``subtract'' it
from any proposition $P$ and $Q$, leaving a magic wand $P \wand P
\land Q$ behind, which is ordinarily not legitimate. (There is
possibly a connection to Parkinson's ``septraction''
operator~\cite{parkinson-septraction}.)


The iterator predicate, defined on line \ref{decl:iterpred-impl} is 
given as a recursive definition. The base case is when we have an 
iterator over a single collection, in the $\One(i)$ case.  Here, 
we have the following assertion: 
\begin{displaymath}
  \iterpred_\tau(\One(i), xs, \setof{(\tau, c, ys, P)}) \equiv 
    \exists c'.\; i \pointsto c' * (\collpred(c,ys,P) \wand (\collpred(c, ys, P) \land (\top * \listpred(c', xs))))
\end{displaymath}

The $i \pointsto c'$ clause says that $i$ is a pointer to a linked
list.  The second clause of this invariant is more complex, and its
purpose is to say that $c$ is an interior pointer into a collection.

It says if ownership of the collection state $\collpred(c, ys, P)$ is
transferred to us, then we will get it back, along with the additional
information that $\listpred(c', xs)$ is within the same state. In this
way, we express the fact that the iterator's invariant depends on
having controlled access to the state of the collection. However, we
are also able to avoid giving the iterator state direct ownership of
the collection; the magic wand lets us say that we don't own the
collection currently, but rather that if we get it, then we can follow
the pointer $c$ when we are given the collection state of the
collection.

The reason we go to this effort is to simplify the specification and
proof of client programs --- we could eliminate the use of the magic
wand in the base case if iterators owned their collections, but this
would complicate verifying programs that use multiple iterators over
the same collection, or which want to call pure methods on the
underlying collection. In those cases, the alternative would require
us to explicitly transfer ownership of the collection in the proofs of
client programs, which is quite cumbersome, and forces clients to
reason using the magic wand. The current approach isolates that
reasoning within the proof of the implementation.

Also, I should note that exactness plays a role in making this use of
the magic wand work correctly. The semantics of the magic wand $p
\wand q$ quantify over all heaps in $p$, but if $p$ is exact, then
there is at most one satisfying heap. 

On the next line, we have a catch-all case saying that the predicate
is false whenever a $\One$ iterator has other than a single collection
it depends upon. 

One the next line, we give the case for $\iterpred_\tau(\Filter(p, i),
xs, S)$.  This is a very simple formula; we simply assert that $i$ is
an iterator yielding some other sequence $ys$, which when filtered with
the predicate $p$ is $xs$. There are no changes to the set of abstract
states. 

On the line after that, we give the case for
$\iterpred_\tau(\Merge(\tau_1, \tau_2, f, i_1, i_2), xs, S)$.  This
case says we can divide the abstract state into two parts (which may
overlap), one of which is used by $i_1$, and the other of which is
used by $i_2$, which yield sequences $ys$ and $zs$ respectively, and
which can be merged using $f$ to yield $xs$.

In both of these cases, we define the behavior of the imperative
linked list in terms of purely functional sequences. This is a very
common strategy in many verification efforts, but here we see that we
can use it in a local way -- in the base case, we are forced to
consider issues of aliasing and ownership, but in the inductive cases
we can largely avoid that effort.


\subsubsection{Correctness Proofs of the Iterator Implementation}

Now that we know the definitions of the types and predicates, we can
give the correctness proofs for the operations defined in
Figure~\ref{iterator-implementation}.\footnote{These definitions make
use of the $\Run$ abbreviation defined in Section~\ref{chap2:sec:proglang}.}

\begin{lemma}{(Correctness of $\newcoll$)}
  We have that 
  \begin{displaymath}
    \forall \alpha.\; \mspec{\emp}
                           {\newcoll_\alpha}
                           {a:\colltype(\alpha)}{\exists P.\; \collpred_\alpha(a, \epsilon,P)}
  \end{displaymath}
  is valid. 
\end{lemma}

\begin{proof}
All $\newcoll_\alpha$ does is allocate a list. To prove the
specification, we will assume that $\alpha:\star$, and then prove the
program in annotated program style.

\begin{specification}
\nextline $\setof{\emp}$ 
\nextline $\newref{\alpha}{\Nil}$ 
\nextline $\setof{a \pointsto_\alpha \Nil}$ 
\nextline $\setof{\listpred(\alpha, a, \epsilon)}$ 
\nextline $\setof{\listpred(\alpha, a, \epsilon) \land \exists Q.\; Q \land \exact{Q}}$ 
\nextline $\setof{\exists Q.\;\listpred(\alpha, a, \epsilon) \land Q \land \exact{Q}}$ 
\nextline $\setof{\exists Q.\; \collpred_\alpha(a, \epsilon, Q)}$ 
\end{specification}

Line 4 follows from 3, because of the definition of the list predicate. Line 5 
follows from 4, because this is an axiomatic property of all predicates -- if
a predicate holds in a heap, there is an exact predicate describing that heap. Line 6 
is a quantifier manipulation, and line 7 follows from the definition of the predicate.
\end{proof}

\begin{lemma}{(Correctness of $\sizecoll$)}
We have that 
\begin{displaymath}
\forall \alpha, c, P, xs.\; \mspec{\collpred_\alpha(c, xs, P)}{\sizecoll_\alpha(c)}{a:\N}
                                  {\collpred_\alpha(c, xs, P) \land a = |xs|}
\end{displaymath}
is valid.   
\end{lemma}

\begin{proof}
This function, defined on line \ref{decl:sizecoll-impl} of Figure~\ref{iterator-implementation},
is a recursively defined function. So we will prove it using the fixed point rule, but with the
altered specification:
\begin{displaymath}
\forall \alpha, c, P, Q, xs.\; \mspec{(P * \listpred(\alpha, c, xs)) \land Q}{\sizecoll_\alpha(c)}{a:\N}
                                     {(P * \listpred(\alpha, c, xs)) \land Q \land a = |xs|}
\end{displaymath}
The reason this specification works is that $\sizecoll$ never modifies
its argument, and so will preserve any conjoined invariant. 
To show this, assume we have $\alpha, c, xs,$ and $P$, and proceed:
\begin{specification}
\nextline $\setof{(P * \listpred(\alpha, c, xs)) \land Q}$ 
\nextline $\setof{(P * \exists p.\; c \pointsto p \land \listcontentpred(\alpha, p, xs)) \land Q}$ 
\nextline $\letv{p}{\comp{!c}}{}$
\nextline $\setof{(P * c \pointsto p \land \listcontentpred(\alpha, p, xs)) \land Q}$ 
\nextline $\Run\;\Listcase($\=$p,$ 
\nextline \> $\Nil \to $
\nextline \> \qquad \=$\setof{(P * c \pointsto p \land p = \Nil) \land xs = \epsilon \land Q}$ 
\nextline \> \> \comp{0}
\nextline \> \> $\setof{(P * c \pointsto \Nil \land p = \Nil) \land xs = \epsilon \land a = 0 \land Q}$ 
\nextline \> \> $\setof{(P * \listpred(\alpha, c, xs)) \land Q \land a = |xs|}$ 
\nextline \> $\Cons(y, c') \to $ 
\nextline \> \qquad \= $\setof{(P * c \pointsto \Cons(y, c') * \listpred(\alpha, c', ys)) \land xs = y\cdot ys \land Q}$ 
\nextline \> \> $[\letv{n}{\sizecoll_\alpha(c')}{}$
\nextline \> \> \,$\setof{(P * c \pointsto \Cons(y, c') * \listpred(\alpha, c', ys)) \land xs = y\cdot ys \land Q \land n = |ys|}$ 
\nextline \> \> \,$\setof{(P * \listpred(\alpha, c, xs)) \land xs = y\cdot ys \land n = |ys| \land Q}$ 
\nextline \> \> \,$n+1]$
\nextline \> \> $\setof{(P * \listpred(\alpha, c, xs)) \land xs = y\cdot ys \land a = |ys| + 1 \land Q}$ 
\nextline \> \> $\setof{(P * \listpred(\alpha, c, xs)) \land xs = y\cdot ys \land a = |y\cdot ys| \land Q}$ 
\nextline \> \> $\setof{(P * \listpred(\alpha, c, xs)) \land Q \land a = |xs|}$ 
\nextline $\setof{(P * \listpred(\alpha, c, xs)) \land Q \land a = |xs|}$ \\
\end{specification}

\noindent Now, we can specialize this proof to get the specification we originally sought:
\begin{specification}
\nextline $\setof{\collpred_\alpha(c, xs, P)}$ 
\nextline $\setof{\listpred(\alpha, c, xs) \land P \land \exact{P}}$
\nextline $\setof{(\emp * \listpred(\alpha, c, xs)) \land P \land \exact{P}}$
\nextline $\sizecoll_\alpha(c)$ 
\nextline $\setof{a:\N.\; (\emp * \listpred(\alpha, c, xs)) \land P \land \exact{P} \land a = |xs|}$
\nextline $\setof{a:\N.\; \collpred_\alpha(c, xs, P) \land a = |xs|}$
\end{specification}
\end{proof}

\begin{lemma}{(Specification of $\addcoll$)}
We have that for 
\begin{displaymath}
  \forall \alpha, c, P, x, xs.\; \mspec{\collpred_\alpha(c, xs, P)}{\addcoll_\alpha(c, x)}
                                      {a:\unittype}{\exists Q.\; \collpred_\alpha(c, x\cdot xs, Q)}
\end{displaymath}
\end{lemma}

\begin{proof}
  This function, defined on line \ref{decl:addcoll-impl}, just conses on an element. 
Assume we have $\alpha, x, c, xs, P$, and then proceed with the following proof, in
annotated specification style. 

\begin{specification}
\nextline $\setof{\collpred_\alpha(c, xs, P)}$ 
\nextline $\setof{\listpred(\alpha, c, xs) \land P \land \exact{P}}$ 
\nextline $\setof{\listpred(\alpha, c, xs)}$ 
\nextline $\setof{\exists p.\; c \pointsto p * \listcontentpred(\alpha, p, xs)}$ 
\nextline $\letv{p}{\comp{!c}}{}$ 
\nextline $\setof{c \pointsto p * \listcontentpred(\alpha, p, xs)}$ 
\nextline $\letv{c'}{\comp{\newref{\Listcontent(\alpha)}{p}}}{}$
\nextline $\setof{c \pointsto p * c' \pointsto p * \listcontentpred(\alpha, p, xs)}$ 
\nextline $\setof{c \pointsto p * \listpred(\alpha, c', xs)}$ 
\nextline $c := \Cons(x, c')$ 
\nextline $\setof{c \pointsto \Cons(x, c') * \listpred(\alpha, c', xs)}$ 
\nextline $\setof{\listpred(\alpha, c, x\cdot xs)}$ 
\nextline $\setof{\exists Q.\; \listpred(\alpha, c, x\cdot xs) \land Q \land \exact{Q}}$ 
\nextline $\setof{\exists Q.\; \collpred_\alpha(c, x\cdot xs, Q)}$ 
\end{specification}
\end{proof}

\begin{lemma}{(Correctness of $\removecoll$, part 1)}
We have that 
\begin{displaymath}
  \forall \alpha, c, P.\; \mspec{\collpred_\alpha(c, \epsilon, P)}
                               {\removecoll_\alpha(c)}{a:\opttype{(\alpha)}}
                               {\collpred_\alpha(c, \epsilon, P) \land a = \None}
\end{displaymath}
\end{lemma}

\begin{proof}
  This is one of the two specifications about $\removecoll$, for the case
  when the list is empty. Assume we have $\alpha, c, P$ and give an annotated
  specification as follows: 

\begin{specification}
\nextline $\setof{\collpred_\alpha(c, \epsilon, P)}$ 
\nextline $\setof{\exists p.\; c \pointsto p \land p = \Nil \land P \land \exact{P}}$ 
\nextline $\letv{p}{\comp{!c}}{}$ 
\nextline $\setof{c \pointsto p \land p = \Nil \land P \land \exact{P}}$ 
\nextline $\Run\;\Listcase($\=$p,$ 
\nextline \> $\Nil \to $\=$\comp{\None}$ 
\nextline \> \> $\setof{c \pointsto p \land p = \Nil \land P \land \exact{P} \land a = \None}$ 
\nextline \> \> $\setof{\collpred_\alpha(c, \epsilon, P) \land a = \None}$ 
\nextline \> $\Cons(y, c') \to $ 
\nextline \> \> $\setof{c \pointsto p \land p = \Nil \land p = \Cons(y, c') \land P \land \exact{P}}$ 
\nextline \> \> $\setof{\bot}$ 
\nextline \> \> $[\letv{p'}{\comp{!c'}}{}$ 
\nextline \> \> $\letv{\_}{\comp{c := p'}}{}$ 
\nextline \> \> $\Some(y)])$
\nextline \> \> $\setof{\collpred_\alpha(c, \epsilon, P) \land a = \None}$ 
\nextline $\setof{\collpred_\alpha(c, \epsilon, P) \land a = \None}$ 
\end{specification}
\end{proof}

\begin{lemma}{(Correctness of $\removecoll$, part 2)}
We have that 
\begin{displaymath}
  \forall \alpha, c, x, xs, P.\; \mspec{\collpred_\alpha(c, x\cdot xs, P)}
                                      {\removecoll_\alpha(c)}
                                      {a:\opttype{(\alpha)}}
                                      {\exists Q.\; \collpred_\alpha(c, xs, Q) 
                                       \land a = \Some(x)}
\end{displaymath}
\end{lemma}

\begin{proof}
This is the other case of $\removecoll$, for when the iterator is not
yet exhausted. Assume that we have $\alpha, c, x, xs$, and $P$ of the
appropriate type. We can give an annotated-specification style proof
as follows:

\begin{specification}
\nextline $\setof{\collpred_\alpha(c, x\cdot xs, P)}$ 
\nextline $\setof{\exists p.\; c \pointsto p \land \exists c'.\; p = \Cons(x, c') * \listpred(\alpha, c', xs) \land P \land \exact{P}}$ 
\nextline $\letv{p}{\comp{!c}}{}$ 
\nextline $\setof{c \pointsto p \land \exists c'.\; p = \Cons(x, c') * \listpred(\alpha, c', xs) \land P \land \exact{P}}$ 
\nextline $\Run\;\Listcase($\=$p,$ 
\nextline \> $\Nil \to $\= $\setof{p = \Nil \land c \pointsto p \land \exists c'.\; p = \Cons(x, c') * \listpred(\alpha, c', xs) \land P \land \exact{P}}$ 
\nextline \> \> $\setof{\bot}$
\nextline \> \> $\comp{\None}$ 
\nextline \> \> $\setof{\exists Q.\; \collpred_\alpha(c, xs, Q) \land a = \Some(x)}$
\nextline \> $\Cons(y, c') \to $ 
\nextline \> \> $\setof{p = \Cons(y, c') \land c \pointsto p \land \exists c'.\; p = \Cons(x, c') * \listpred(\alpha, c', xs) \land P \land \exact{P}}$ 
\nextline \> \> $\setof{x = y \land c \pointsto p \land p = \Cons(x, c') * \listpred(\alpha, c', xs) \land P \land \exact{P}}$ 
\nextline \> \> $\setof{x = y \land c \pointsto p \land p = \Cons(x, c') * \listpred(\alpha, c', xs)}$
\nextline \> \> $\setof{x = y \land c \pointsto p \land p = \Cons(x, c') * \exists p'.\; c' \pointsto p' * \listcontentpred(\alpha, p', xs)}$
\nextline \> \> $[\letv{p'}{\comp{!c'}}{}$ 
\nextline \> \> $\setof{x = y \land c \pointsto p \land p = \Cons(x, c') * c' \pointsto p' * \listcontentpred(\alpha, p', xs)}$
\nextline \> \> $\letv{\_}{\comp{c := p'}}{}$ 
\nextline \> \> $\setof{x = y \land c \pointsto p' * c' \pointsto p' * \listcontentpred(\alpha, p', xs)}$
\nextline \> \> $\setof{x = y \land c' \pointsto p' * \listpred(\alpha, c, xs)}$
\nextline \> \> $\setof{x = y \land \listpred(\alpha, c, xs)}$
\nextline \> \> $\Some(y)])$
\nextline \> \> $\setof{\listpred(\alpha, c, xs) \land a = \Some(y) \land x = y}$ 
\nextline \> \> $\setof{\listpred(\alpha,c, xs) \land a = \Some(x)}$
\nextline \> \> $\setof{(\exists Q.\; \listpred(\alpha,c, xs) \land Q \land \exact{Q}) \land a = \Some(x)}$
\nextline \> \> $\setof{\exists Q.\; \collpred_\alpha(c, xs, Q) \land a = \Some(x)}$
\nextline $\setof{\exists Q.\; \collpred_\alpha(c, xs, Q) \land a = \Some(x)}$
\end{specification}
\end{proof}

\begin{lemma}{(Correctness of $\newiter$)}
We have that
\begin{displaymath}
\forall \alpha, c, P, xs.\; \mspec{\collpred_\alpha(c, xs, P)}
                                 {\newiter_\alpha(c)}
                                 {a:\itertype(\alpha)}
                                 {\collpred_\alpha(c, xs, P) * 
                                  \iterpred_\alpha(a, xs, \setof{\alpha, c, xs, P})}
\end{displaymath}
is valid.
\end{lemma}

\begin{proof}
The definition of $\newiter$ is given on line \ref{decl:newiter-impl} of
Figure~\ref{iterator-implementation}. To prove the correctness of this
implementation, we assume we have $\alpha, c, xs$, and $P$, and then
give the following proof: 

\begin{specification}
\nextline $\setof{\collpred_\alpha(c, xs, P)}$ 
\nextline $\setof{\listpred(c, xs, P) \land P \land \exact{P}}$ 
\nextline $\setof{\listpred(c, xs, P) \land P \land \exact{P}  \land \exact{\listpred(c, xs, P) \land P \land \exact{P}}}$ 
\nextline $\setof{\collpred(c, xs, P) \land \exact{\collpred(c, xs, P)}}$ 
\nextline $\letv{i}{\comp{\newref{\colltype(\alpha)}{c}}}{}$ 
\nextline $\setof{i \pointsto c * \collpred(c, xs, P) \land \exact{\collpred(c, xs, P)}}$ 
\nextline $\setof{i \pointsto c * ((\top * \collpred(c, xs, P)) \land \collpred(c, xs, P)) \land \exact{\collpred(c, xs, P)}}$ 
\nextline $\setof{i \pointsto c * (\collpred(c, xs, P) \wand (\top * \collpred(c, xs, P)) \land \collpred(c, xs, P)) * \collpred(c, xs, P)}$ 
\nextline $\One(i)$ 
\nextline $\setof{a.\; \exists i.\; a = \One(i) \land \iterpred(\One(i), xs, \setof{(\alpha, c, xs, P)})}$ 
\nextline $\setof{a.\; \iterpred(a, xs, \setof{(\alpha, c, xs, P)})}$ 

\end{specification}

The key idea in this proof is that if $P$ is exact, then $P \land Q$
is also exact, no matter what $Q$ is.  This lets us use the exactness
of the abstract state to ``subtract'' the predicate $\collpred(c, xs, P)$
and keep the iterator and collection state separate. 

\end{proof}

\begin{lemma}{(Correctness of $\filteriter$)}
We have that
\begin{displaymath}
  \forall \alpha, p, i, S, xs.\; \mspec{\iterpred_\alpha(i, xs, S)}
                                      {\filteriter_\alpha(p, i)}
                                      {a:\itertype(\alpha)}
                                      {\iterpred_\alpha(a, \filtermath\;p\;xs, S)}
\end{displaymath}
\end{lemma}

\begin{proof}
The definition of this procedure is given on line \ref{decl:filteriter-impl}. Assume
$\alpha, p, i, xs$, and $S$. 

\begin{specification}
\nextline $\setof{\iterpred_\alpha(i, xs, S)}$ 
\nextline $\Filter(p, i)$ 
\nextline $\setof{\iterpred_\alpha(i, xs, S) \land a = \Filter(p, i)}$ 
\nextline $\setof{\iterpred_\alpha(i, xs, S) \land \filtermath\;p\;xs = \filtermath\;p\;xs \land a = \Filter(p, i)}$ 
\nextline $\setof{\exists i, ys.\; \iterpred_\alpha(i, ys, S) \land \filtermath\;p\;xs = \filtermath\;p\;ys \land a = \Filter(p, i)}$
\nextline $\setof{\iterpred_\alpha(a, \filtermath\;p\;xs, S)}$ 
\end{specification}
\end{proof}

\begin{lemma}{(Correctness of $\mergeiter$)}
We have that 
\begin{displaymath}
  \forall \alpha, \beta, \gamma, f, i, S, xs, i', S', xs'.\; 
  \begin{array}{l}
    \setof{\iterpred_\alpha(i, xs, S) * \iterpred_\beta(i',xs',S')} \\
    \mergeiter_{\alpha\;\beta\;\gamma}(f, i, i') \\
    \setof{a:\itertype(\gamma).\; \iterpred_\gamma(a, \mergemath\;f\;xs\;xs', S \cup S')}\\
  \end{array}
\end{displaymath}
\end{lemma}

\begin{proof}
The definition of $\mergeiter$ is given on line \ref{decl:mergeiter-impl} of
Figure~\ref{iterator-implementation}. Assume $f, i, xs, S, i', xs', S'$ as 
hypotheses, and proceed with the proof as follows: 

\begin{specification}
\nextline $\setof{\iterpred_\alpha(i, xs, S) * \iterpred_\beta(i',xs',S')}$ 
\nextline $\Merge(\alpha, \beta, f, i, i')$
\nextline $\setof{\iterpred_\alpha(i, xs, S) * \iterpred_\beta(i',xs',S')  \land a = \Merge(\alpha, \beta, f, i, i')}$ 
\nextline $\{$\=$\iterpred_\alpha(i, xs, S) * \iterpred_\beta(i',xs',S')  \land \mergemath\;f\;xs\;xs' = \mergemath\;f\;xs\;xs'$
\nextline \> $\land\; a = \Merge(f, i, i')\}$ 
\nextline $\{\exists i, i', S_1, S_2, xs_1, xs_2.\; \iterpred_\alpha(i, xs_1, S_1) * \iterpred_\beta(i', xs_2, S_2) \land 
                                                    S \cup S' = S_1 \cup S_2 \;\land $
\nextline $\;\;\mergemath\;f\;xs\;xs' = \mergemath\;f\;xs_1\;xs_2 \land 
                                                    a = \Merge(f, i, i')\}$ 
\nextline $\setof{\exists i,i'.\; \iterpred_\gamma(\Merge(f, i, i'), \mergemath\;f\;xs\;xs', S \cup S')}$ 
\nextline $\setof{\iterpred_\alpha(a, \mergemath\;f\;xs\;xs', S \cup S')}$

\end{specification}
\end{proof}

\begin{lemma}{(Correctness of $\nextiter$)}
We have that 
\begin{displaymath}
  \forall \alpha, i, S, xs.\; 
  \begin{array}{l}
    \setof{\iterpred_\alpha(i, xs, S) * \mathit{colls}(S)} \\
    \nextiter_\alpha(i) \\
    \{a:\opttype{(\alpha)}.\; \\
      \qquad [(a = \None \land xs = \epsilon \land \iterpred_\alpha(i, xs, S)) \vee \\
      \qquad (\exists y, ys.\; a = \Some(y) \land xs = y\cdot ys \land \iterpred_\alpha(i, ys, S))]
       \\
      \qquad * \mathit{colls}(S)\} \\
  \end{array}
\end{displaymath}
\end{lemma}

\begin{proof}
This function is defined on lines \ref{decl:next-one-impl} to \ref{decl:next-impl-end} 
of Figure~\ref{iterator-implementation}. To prove this, we will use fixed point induction,
and assume that the specification above holds for an identifier named $\nextiter$, and
use it to prove the correctness for the body of the function. 

Assuming $\alpha, i, S, xs$, this proof will proceed by cases, on
the structure of $i$. Then we can exploit the fact that we can unroll fixed
points and beta-reduce expression to avoid proving the impossible branches (which
in typical Hoare logic proofs are ruled out by getting false as a precondition). 

\begin{itemize}
\item Suppose $i = \One(i')$. Then, we proceed with an annotated proof as 
follows.  
\begin{specification}
\nextline $\setof{\iterpred_\alpha(\One(i'), xs, S) * \mathit{colls}(S)}$ 
\nextline When $S$ has other than one element, the precondition is false, so we only need to \nextline
          consider the one-element case:
\nextline $\setof{\exists c, ys, P.\; S = \setof{(c, ys, P)} \land \iterpred_\alpha(\One(i'), xs, \setof{(\alpha, c, ys, P)}) * \mathit{colls}_\alpha(\setof{(\alpha, c, ys, P)})}$ 
\nextline We will use the existential rule and then frame off the definition of $S$:
\nextline $\setof{\iterpred(\One(i'), xs, \setof{(c, ys, P)}) * \collpred(c, ys, P)}$ 
\nextline $\setof{\exists c'.\; i' \pointsto c' * (\top * list(c', xs)) \land \collpred(c, ys, P)}$
\nextline $\letv{c'}{\comp{!i'}}{}$ 
\nextline $\setof{i' \pointsto c' * (\top * list(c', xs)) \land \collpred(c, ys, P)}$
\nextline $\setof{\exists p.\; i' \pointsto c' * (\top * c' \pointsto p * \listcontentpred(p, xs)) \land \collpred(c, ys, P)}$
\nextline $\letv{p}{\comp{!c'}}{}$ 
\nextline $\setof{i' \pointsto c' * (\top * c' \pointsto p * \listcontentpred(p, xs)) \land \collpred(c, ys, P)}$
\nextline $\Run\;$\=$\Listcase(p,$ 
\nextline \> $\Nil $\=$\to$ 
\nextline \> \> $\setof{p = \Nil \land i' \pointsto c' * (\top * c' \pointsto p * \listcontentpred(p, xs)) \land \collpred(c, ys, P)}$
\nextline \> \> $\setof{xs = \epsilon \land i' \pointsto c' * (\top * c' \pointsto p * \listcontentpred(p, xs)) \land \collpred(c, ys, P)}$
\nextline \> \> $\setof{xs = \epsilon \land \iterpred(i, xs, \setof{(c, ys, P)}) * \collpred(c, ys, P)}$
\nextline \> \> Restoring the definition of $S$:
\nextline \> \> $\setof{xs = \epsilon \land \iterpred(i, xs, S) * \mathit{colls}(S)}$
\nextline \> \> $\comp{\None}$ 
\nextline \> \> $\setof{a.\; a = \None \land xs = \epsilon \land \iterpred(i, xs, S) * \mathit{colls}(S)}$
\nextline \> $\Cons(z, c'') \to$ 
\nextline \> \> $\setof{p = \Cons(z, c'') \land i' \pointsto c' * (\top * c' \pointsto p * \listcontentpred(p, xs)) \land \collpred(c, ys, P)}$
\nextline \> \> $\{$\=$\exists zs.\; xs = z\cdot zs \land p = \Cons(z, c'') \land i' \pointsto c' \;*$ 
\nextline \> \>     \>$(\top * c' \pointsto \Cons(z, c'') * \listpred(c'', zs)) \land \collpred(c, ys, P)\}$
\nextline \> \> $\setof{\exists zs.\; xs = z\cdot zs \land i' \pointsto c' \;*
                       (\top * \listpred(c'', zs)) \land \collpred(c, ys, P)}$
\nextline \> \> $[\letv{\_}{i' := c''}{}$ 
\nextline \> \> $\setof{\exists zs.\; xs = z\cdot zs \land i' \pointsto c'' \;*
                       (\top * \listpred(c'', zs)) \land \collpred(c, ys, P)}$
\nextline \> \> $\setof{\exists zs.\; xs = z\cdot zs \land
                        \iterpred(\One(i'), zs, \setof{(c, ys, P)}) * \collpred(c, ys, P)}$ 
\nextline \> \> Restoring the definition of $S$:
\nextline \> \> $\setof{\exists zs.\; xs = z\cdot zs \land
                        \iterpred(\One(i'), zs, S) * \mathit{colls}(S)}$ 
\nextline \> \> $\Some(z)])$ 
\nextline \> \> $\setof{\exists z, zs.\; a = \Some(z) \land xs = z\cdot zs \land
                        \iterpred(i, zs, S) * \mathit{colls}(S)}$ 
\nextline $\{a.\;$\=$((a = \None \land xs = \epsilon \land \iterpred(i, xs, S)) \; \vee$ 
\nextline         \>$(\exists z, zs.\; a = \Some(z) \land xs = z\cdot zs \land \iterpred(i, zs, S))) \;*$ 
\nextline         \>$\mathit{colls}(S)\}$
\end{specification}

\item Suppose $i = \Filter(p, i')$. Then, we proceed with an annotated proof as 
follows: 
\begin{specification}
\nextline $\setof{\iterpred_\alpha(\Filter(p, i'), xs, S) * \mathit{colls}(S)}$ 
\nextline $\setof{\exists ys. (\iterpred_\alpha(i', ys, S) \land 
                              xs = \mathit{filter}\;p\;ys) * 
                              \mathit{colls}(S)}$   
\nextline $\setof{xs = \mathit{filter}\;p\;ys \land
                  \iterpred_\alpha(i', ys, S) *
                  \mathit{colls}(S)}$
\nextline Since it is pure, we can pull $xs = \mathit{filter}\;p\;ys$ out of the precondition. 
\nextline That is, using the axiom \textsc{AxExtract} we can pull pure consequences 
\nextline of a precondition into the ambient specification context, and then 
\nextline restore them whenever we need using the \textsc{AxEmbed} axiom. 
\nextline $\setof{\iterpred_\alpha(i', ys, S) * \mathit{colls}(S)}$
\nextline $[\letv{v}{\nextiter_\alpha(i')}{}$ 
\nextline \{\=$[(ys = \epsilon \land v = \None \land \iterpred_\alpha(i', ys, S)) \vee 
                (\exists z, zs.\; ys = z\cdot zs \land v = \Some(z) \land \iterpred_\alpha(i', zs, S))]$
\nextline \> $ * \mathit{colls}(S)$\}
\nextline \;$\Run$\=$\;\Listcase(v,$ 
\nextline \> $\None$\=$ \to $ 
\nextline \> \> $\{$\=$[(ys = \epsilon \land v = \None \land \iterpred_\alpha(i', ys, S)) \vee $
\nextline \> \>     \>$(\exists z, zs.\; ys = z\cdot zs \land v = \Some(z) \land \iterpred_\alpha(i', zs, S))] * \mathit{colls}(S) \land v = \None\}$
\nextline \> \> $\{$\=$ys = \epsilon \land \iterpred_\alpha(i', ys, S) * \mathit{colls}(S)\}$
\nextline \> \> Since $xs = \mathit{filter}\;p\;ys = \epsilon$, we know
\nextline \> \> $\{$\=$xs = \epsilon \land \iterpred_\alpha(i, xs, S) * \mathit{colls}(S)\}$
\nextline \> \> $[\None]$ 
\nextline \> \> $\{a.\; a = \None \land xs = \epsilon \land \iterpred_\alpha(i, xs, S) * \mathit{colls}(S)\}$
\nextline \> $\Some(z)\to$
\nextline \> \> $\{$\=$[(ys = \epsilon \land v = \None \land \iterpred_\alpha(i', ys, S)) \vee $
\nextline \> \> \> $(\exists z, zs.\; ys = z\cdot zs \land v = \Some(z) \land \iterpred_\alpha(i', zs, S))] * \mathit{colls}(S) \land v = \Some(z)\}$
\nextline \> \> $\{\exists zs.\; ys = z\cdot zs \land v = \Some(z) \land \iterpred_\alpha(i', zs, S) * \mathit{colls}(S)\}$
\nextline \> \> $\{ys = z\cdot zs \land v = \Some(z) \land \iterpred_\alpha(i', zs, S) * \mathit{colls}(S)\}$
\nextline \> \> $\Run\;\ctext{if}($\=$p\;z,$ 
\nextline \> \> \> $\{p\;z = \True \land ys = z\cdot zs \land \iterpred_\alpha(i', zs, S) * \mathit{colls}(S) \land v = \Some(z)\}$
\nextline \> \> \> Hence $xs = \mathit{filter}\;p\;ys = z\cdot(\mathit{filter}\;p\;zs)$. So 
\nextline \> \> \> $\{xs = z\cdot(\mathit{filter}\;p\;zs) \land \iterpred_\alpha(i', zs, S) * \mathit{colls}(S)\}$
\nextline \> \> \> $\{\exists xs'.\; xs = z\cdot xs' \land \iterpred_\alpha(\Filter(p, i'), xs', S) * \mathit{colls}(S) \land v = \Some(z)\}$
\nextline \> \> \> $\{\exists xs'.\; xs = z\cdot xs' \land \iterpred_\alpha(i, xs', S) * \mathit{colls}(S) \land v = \Some(z)\}$
\nextline \> \> \> $[v],$
\nextline \> \> \> $\{a.\; \exists z, zs.\; xs = z\cdot zs \land a = \Some(z) \land \iterpred_\alpha(i, zs, S) * \mathit{colls}(S) \land v = \Some(z)\}$
\nextline \> \> \> $\{a.\;$\=$((a = \None \land xs = \epsilon \land \iterpred(i, xs, S)) \; \vee$ 
\nextline \> \> \>         \>$(\exists z, zs.\; a = \Some(z) \land xs = z\cdot zs \land \iterpred(i, zs, S))) \;*$ 
\nextline \> \> \>         \>$\mathit{colls}(S)\}$
\nextline \> \> \> Now, the else-branch:
\nextline \> \> \> $\{p\;z = \False \land ys = z\cdot zs \land \iterpred_\alpha(i', zs, S) * \mathit{colls}(S)\}$
\nextline \> \> \> Hence $xs = \mathit{filter}\;p\;ys = \mathit{filter}\;p\;zs$. So 
\nextline \> \> \> $\{xs = \mathit{filter}\;p\;zs \land \iterpred_\alpha(i', zs, S) * \mathit{colls}(S)\}$
\nextline \> \> \> $\{\iterpred_\alpha(\Filter(p, i'), xs, S) * \mathit{colls}(S)\}$
\nextline \> \> \> $\{\iterpred_\alpha(i, xs, S) * \mathit{colls}(S)\}$
\nextline \> \> \> $\nextiter_\alpha(i))))]$ 
\nextline \> \> \> $\{a.\;$\=$((a = \None \land xs = \epsilon \land \iterpred(i, xs, S)) \; \vee$ 
\nextline \> \> \>         \>$(\exists z, zs.\; a = \Some(z) \land xs = z\cdot zs \land \iterpred(i, zs, S))) \;*$ 
\nextline \> \> \>         \>$\mathit{colls}(S)\}$
\nextline $\{a.\;$\=$((a = \None \land xs = \epsilon \land \iterpred(i, xs, S)) \; \vee$ 
\nextline         \>$(\exists z, zs.\; a = \Some(z) \land xs = z\cdot zs \land \iterpred(i, zs, S))) \;*$ 
\nextline         \>$\mathit{colls}(S)\}$
\end{specification}

Note that this is not a structural induction on $i$; we make a
non-structural recursive call when the test of $p$ fails. Here, we
make use of fixed-point induction in our proof of $\nextiter$.

\item Suppose $i = \Merge(f, i_1, i_2)$. Then, we proceed with an
annotated proof as follows:

\begin{specification}
\nextline $\setof{\iterpred_\alpha(\Merge(f, i_1, i_2), xs, S) * \mathit{colls}(S)}$ 
\nextline $\{$\=$\exists \beta, \gamma, S_1, ys, S_2, zs.\; 
                  xs = \mathit{map2}\;f\;ys\;zs \land 
                    S = S_1 \cup S_2 \land$ 
\\\>     \>$   \iterpred_\beta(i_1, ys, S_1) * 
                    \iterpred_\gamma(i_2, zs, S_2) * 
                    \mathit{colls}(S)\}$ 
\nextline $\{xs = \mathit{map2}\;f\;ys\;zs \land 
                  S = S_1 \cup S_2 \land$ 
\\ \>\;     $\iterpred_\beta(i_1, ys, S_1) * 
                  \iterpred_\gamma(i_2, zs, S_2) * 
                  \mathit{colls}(S)$\}
\nextline \{$xs = \mathit{map2}\;f\;ys\;zs \land 
                  S = S_1 \cup S_2 \land$ 
\\ \>\;     $\iterpred_\beta(i_1, ys, S_1) * 
                  \iterpred_\gamma(i_2, zs, S_2) * 
                  \mathit{colls}(S_1) * \mathit{colls}(S - S_1)$\}
\nextline $\letv{v_1}{\nextiter_\beta(i_1)}{}$
\nextline \{$xs = \mathit{map2}\;f\;ys\;zs \land 
                  S = S_1 \cup S_2 \land$ 
\\ \>\;     $[(ys = \epsilon \land v_1 = \None \land \iterpred_\beta(i_1, ys, S_1))$ 
\\ \>\;\;    $\vee\; (\exists b, bs.\; ys = b\cdot bs \land v_1 = \Some(b) \land 
                                \iterpred_\alpha(i, bs, S_1))] \;*$ 
\\ \>\;     $\iterpred_\gamma(i_2, zs, S_2) * 
             \mathit{colls}(S_1) * \mathit{colls}(S - S_1)$\}
\nextline \{$xs = \mathit{map2}\;f\;ys\;zs \land 
                  S = S_1 \cup S_2 \land$ 
\\ \>\;     $[(ys = \epsilon \land v_1 = \None \land \iterpred_\beta(i_1, ys, S_1))$ 
\\ \>\;\;    $\vee\; (\exists b, bs.\; ys = b\cdot bs \land v_1 = \Some(b) \land 
                                \iterpred_\alpha(i, bs, S_1))] \;*$ 
\\ \>\;     $\iterpred_\gamma(i_2, zs, S_2) * 
             \mathit{colls}(S_2) * \mathit{colls}(S - S_2)$\}
\nextline $\letv{v_2}{\nextiter_\gamma(i_2)}{}$ 
\nextline \{$xs = \mathit{map2}\;f\;ys\;zs \land 
                  S = S_1 \cup S_2 \land$ 
\\ \>\;     $[(ys = \epsilon \land v_1 = \None \land \iterpred_\beta(i_1, ys, S_1))$ 
\\ \>\;\;    $\vee\; (\exists b, bs.\; ys = b\cdot bs \land v_1 = \Some(b) \land 
                                \iterpred_\beta(i_1, bs, S_1))] \;*$ 
\\ \>\;     $[(zs = \epsilon \land v_2 = \None \land \iterpred_\gamma(i_2, zs, S_2))$ 
\\ \>\;\;    $\vee\; (\exists c, cs.\; zs = c\cdot cs \land v_2 = \Some(c) \land 
                                \iterpred_\gamma(i_2, cs, S_2))] \;*$ 
\\ \>\;     $\mathit{colls}(S_2) * \mathit{colls}(S - S_2)$\}
\nextline \{$xs = \mathit{map2}\;f\;ys\;zs \land 
                  S = S_1 \cup S_2 \land$ 
\\ \>\;     $[(ys = \epsilon \land v_1 = \None \land \iterpred_\beta(i_1, ys, S_1))$ 
\\ \>\;\;    $\vee\; (\exists b, bs.\; ys = b\cdot bs \land v_1 = \Some(b) \land 
                                \iterpred_\beta(i_1, bs, S_1))] \;*$ 
\\ \>\;     $[(zs = \epsilon \land v_2 = \None \land \iterpred_\gamma(i_2, zs, S_2))$ 
\\ \>\;\;    $\vee\; (\exists c, cs.\; zs = c\cdot cs \land v_2 = \Some(c) \land 
                                \iterpred_\gamma(i_2, cs, S_2))] \;*$ 
\\ \>\;     $\mathit{colls}(S)$\}
\nextline $\Optcase(v_1$ 
\nextline \;\;\; $\None$\=$ \to $ 
\nextline \> \{$xs = \mathit{map2}\;f\;ys\;zs \land 
                  S = S_1 \cup S_2 \land$ 
\\ \> \>\;     $(ys = \epsilon \land v_1 = \None \land \iterpred_\beta(i_1, ys, S_1)) * $ 
\\ \> \>\;     $[(zs = \epsilon \land v_2 = \None \land \iterpred_\gamma(i_2, zs, S_2))$ 
\\ \> \>\;\;    $\vee\; (\exists c, cs.\; zs = c\cdot cs \land v_2 = \Some(c) \land 
                                \iterpred_\gamma(i_2, cs, S_2))] \;*$ 
\\ \> \>\;     $\mathit{colls}(S)$\}
\nextline \> \{$xs = \epsilon \land xs = \mathit{map2}\;f\;ys\;zs \land 
                  S = S_1 \cup S_2 \land$ 
\\ \> \>\;     $(ys = \epsilon \land v_1 = \None \land \iterpred_\beta(i_1, ys, S_1)) * $ 
\\ \> \>\;     $[(zs = \epsilon \land v_2 = \None \land \iterpred_\gamma(i_2, zs, S_2))$ 
\\ \> \>\;\;    $\vee\; (\exists c, cs.\; zs = c\cdot cs \land v_2 = \Some(c) \land 
                                \iterpred_\gamma(i_2, cs, S_2))] \;*$ 
\\ \> \>\;     $\mathit{colls}(S)$\}
\\ \> \>     The next line follows because it holds in either branch of the disjunction 
\nextline \> \{$xs = \epsilon \land \iterpred_\alpha(\Merge(f, i_1, i_2), xs, S)
               * \; \mathit{colls}(S)$\}
\nextline \> $\None$ 
\nextline \> \{$a = \None \land xs = \epsilon \land \iterpred_\alpha(\Merge(f, i_1, i_2), xs, S)
               * \mathit{colls}(S) $\}
\nextline \;\;\;$\Some(b)$\=$ \to $ 
\nextline \> \{$xs = \mathit{map2}\;f\;ys\;zs \land 
                  S = S_1 \cup S_2 \land$ 
\\ \> \>\;\;    $(\exists bs.\; ys = b\cdot bs \land v_1 = \Some(b) \land 
                                \iterpred_\beta(i_1, bs, S_1)) \;*$ 
\\ \> \>\;     $[(zs = \epsilon \land v_2 = \None \land \iterpred_\gamma(i_2, zs, S_2))$ 
\\ \> \>\;\;    $\vee\; (\exists c, cs.\; zs = c\cdot cs \land v_2 = \Some(c) \land 
                                \iterpred_\gamma(i_2, cs, S_2))] \;*$ 
\\ \> \>\;     $\mathit{colls}(S)$\}
\nextline \> $\Optcase(v_2,$ 
\nextline \> \;\;\; $\None$\=$\to$ 
\nextline \> \> \{$xs = \mathit{map2}\;f\;ys\;zs \land 
                  S = S_1 \cup S_2 \land$ 
\\ \> \> \>\;\;    $(\exists bs.\; ys = b\cdot bs \land v_1 = \Some(b) \land 
                                \iterpred_\beta(i_1, bs, S_1)) \;*$
\\ \> \> \>\;     $(zs = \epsilon \land v_2 = \None \land \iterpred_\gamma(i_2, zs, S_2)) *$ 
\\ \> \> \>\;     $\mathit{colls}(S)$\}

\nextline \> \> \{$xs = \epsilon \land xs = \mathit{map2}\;f\;ys\;zs \land 
                  S = S_1 \cup S_2 \land zs = \epsilon \;\land$ 
\\ \> \> \>\;\;    $(\exists bs.\; ys = b\cdot bs \land v_1 = \Some(b) \land 
                                \iterpred_\beta(i_1, bs, S_1)) \;*$
\\ \> \> \>\;     $(zs = \epsilon \land v_2 = \None \land \iterpred_\gamma(i_2, zs, S_2))$ 
\\ \> \> \>\;     $\mathit{colls}(S)$\}

\nextline \> \> \{$xs = \epsilon \land \iterpred_\alpha(\Merge(f, i_1, i_2), xs, S) * 
                   \mathit{colls}(S)$\}
\nextline \> \> $\None$,
\nextline \> \> \{$a = \None \land xs = \epsilon \land \iterpred_\alpha(\Merge(f, i_1, i_2), xs, S) * 
                   \mathit{colls}(S)$\}

\nextline \> \;\;\; $\Some(c)$\=$ \to $ 
\nextline \> \> \{$xs = \mathit{map2}\;f\;ys\;zs \land 
                  S = S_1 \cup S_2 \land$ 
\\ \> \> \>\;\;    $(\exists bs.\; ys = b\cdot bs \land v_1 = \Some(b) \land 
                                \iterpred_\beta(i_1, bs, S_1)) \;*$
\\ \> \> \>\;\;    $(\exists cs.\; zs = c\cdot cs \land v_2 = \Some(c) \land 
                                \iterpred_\gamma(i_2, cs, S_2)) \;*$

\\ \> \> \>\;     $\mathit{colls}(S)$\}
\nextline \> \> \{$xs = \mathit{map2}\;f\;ys\;zs \land 
                  S = S_1 \cup S_2 \land$ 
\\ \> \> \>\;    $(ys = b\cdot bs \land v_1 = \Some(b) \land 
                                \iterpred_\beta(i_1, bs, S_1)) \;*$
\\ \> \> \>\;    $(zs = c\cdot cs \land v_2 = \Some(c) \land 
                                \iterpred_\gamma(i_2, cs, S_2)) \;*$

\\ \> \> \>\;     $\mathit{colls}(S)$\}

\nextline \> \> \{$xs = (f\;b\;c) \cdot \mathit{map2}\;f\;bs\;cs \land 
                  S = S_1 \cup S_2 \land$ 
\\ \> \> \>\;    $\iterpred_\beta(i_1, bs, S_1)) * \iterpred_\gamma(i_2, cs, S_2)) \;*$

\\ \> \> \>\;     $\mathit{colls}(S)$\}

\nextline \> \> \{$xs = (f\;b\;c) \cdot \mathit{map2}\;f\;bs\;cs \; \land$
\\ \> \> \>\;    $\iterpred_\alpha(\Merge(f, i_1, i_2), \mathit{map2}\;f\;bs\;cs, S) \;*$
\\ \> \> \>\;     $\mathit{colls}(S)$\}

\nextline \> \> $\Some(f\;b\;c)))$ 
\nextline \> \> \{$a = \Some(f\;b\;c) \land xs = (f\;b\;c) \cdot \mathit{map2}\;f\;bs\;cs \; \land$
\\ \> \> \>\;    $\iterpred_\alpha(\Merge(f, i_1, i_2), \mathit{map2}\;f\;bs\;cs, S) \;*$
\\ \> \> \>\;     $\mathit{colls}(S)$\}

\nextline \> \> \{$\exists v, vs.\; a = \Some(v) \land xs = v \cdot vs \; \land$
\\ \> \> \>\;    $\iterpred_\alpha(\Merge(f, i_1, i_2), vs, S) \;*$
\\ \> \> \>\;     $\mathit{colls}(S)$\}


\end{specification}

\end{itemize}

\end{proof}


$\nextiter$ (lines 18-33 of Figure~\ref{iterator-implementation})
recursively walks down the structure of the iterator tree, and
combines the results from the leaves upwards.  The base case is the
$\One(ii)$ case (lines 18-22). The iterator pointer is
doubly-dereferenced, and then the contents examined. If the end of the
list has been reached and the contents are $\Nil$, then $\None$ is
returned to indicate there are no more elements. Otherwise, the
pointer $r$ is advanced, and the head returned as the observed
value. The $\ctext{Filter(p,i)}$ case (lines 23-26) will return
$\None$ if $i$ is exhausted, and if it is not, it will pull elements
from $i$ until it finds one that satisfies $p$, calling itself
recursively until it succeeds or $i$ is exhausted.  Finally, in the
$\ctext{Map2}(f, i_1, i_2)$ case (lines 27-33), $\nextiter$ will draw
a value from both $i_1$ and $i_2$, and will return $\None$ if either
is exhausted, and otherwise it will return $f$ applied to the pair of
values.


\section{The Flyweight and Factory Patterns}

The flyweight pattern is a style of cached object creation. Whenever a
constructor method is called, it first consults a table to see if an
object corresponding to those arguments has been created. If it has,
then the preexisting object is returned.  Otherwise, it allocates a
new object, and updates the table to ensure that future calls with the
same arguments will return this object. Because objects are re-used,
they become pervasively aliased, and must be used in an immutable
style to avoid surprising updates. (Functional programmers call this
style of value creation ``hash-consing''.)

This is an interesting design pattern to verify, for two reasons.
First, the constructor has a memo table to cache the result of
constructor calls, which needs to be hidden from clients. Second, this
pattern makes pervasive use of aliasing, in a programmer-visible
way. In particular, programmers can test two references for identity
in order to establish whether two values are equal or not. This allows
constant-time equality testing, and is a common reason for using this
pattern. Therefore, our specification has to be able to justify this
reasoning.

In Figure~\ref{flyweight-spec}, we give a specification which uses the
flyweight pattern to create and access glyphs (i.e., refs of pairs of
characters and fonts) of a particular font $f$. We have a function
$\ctext{newglyph}$ to create new glyphs, which does the caching
described above, using a predicate variable $I$ to refer to the table
invariant; and a function $\ctext{getdata}$ to get the character and
font information from a glyph.

Furthermore, these functions will be created by a call to another
function, $\ctext{make\_flyweight}$, which receives a font as an
argument and will return appropriate $\ctext{newglyph}$ and
$\ctext{getdata}$ functions.

\begin{figure}
\begin{tabbing}
\\
Flyweight$($\=$I : (\ctext{char} \rightharpoonup \ctext{glyph}) \to \assert,\;\; $\\
\> $\ctext{newglyph} : \chartp \to \monad{\ctext{glyph}},$ \\
\> $\ctext{getdata} : \ctext{glyph} \to \monad{(\chartp \times \fonttp)},$ \\
\> $f:\fonttp) \equiv$ \\
% 1 \qquad \=$\exists $\=$glyph : \ctext{glyph} \times \chartp \times \fonttp \To \assert.$ 
% \\[0.5em]

1 \qquad\=  $\forall$\=$c, h.\;$\=
         $\angles{I(h)}$ \\
  \>\>\>${\ctext{newglyph}(c)}$ \\
  \>\>\>$\angles{a:\ctext{glyph}.\; I([h|c:a]) \land (c \in \domain{h} \implies a = h(c))}$ \\
  \> \!$\specand$ \\
2 \> $\forall l, h, c.\;$\=
     $\angles{I(h) \land l = h(c)}$ \\
\>\> ${\ctext{getdata}(l)}$ \\
\>\> $\angles{a:\chartp \times \fonttp.\;  I(h) \land a = (c,f)}$ \\[1em]

where $\ctext{glyph} \triangleq \reftype{(\ctext{char} \times \ctext{font})}$

%   \> \!$\specand$ \\
% 4 \> $\{\forall l, l', c, c'.\;$\=$I \land glyph(l,c,f) \land glyph(l',c',f')
%  \implies $ \\
% \>\>  $\;\;\;\left(l = l' \iff (c = c'\land f=f')\right)\}$ \\[0.5em]
% 
% $chars(\emptyset)$ \qquad\qquad\qquad \;\;\= $\equiv$ \= $\top$ \\
% $chars(\setof{(l,(c,f))} \cup S)$ \> $\equiv$ \> $glyph(l,c,f) \land chars(S)$ \\
\end{tabbing}
\caption{Flyweight Specification}
\label{flyweight-spec}
\end{figure}

In the opening, we informally parametrize our specification over the
predicate variable $I$, the function variable $\ctext{newglyph}$, the
function variable $\ctext{getdata}$, and the variable $f$ of $\fonttp$
type. The reason we do this instead of existentially quantifying over
them will become clear shortly, once we see the factory function that
creates flyweight constructors.

On line 1, we specify the effect of a call $\ctext{newglyph}(c)$
procedure. Its precondition is the predicate $I(h)$, which asserts
that the abstract state contains glyphs for all of the characters in
the domain of the partial function $h$. Its postcondition changes to a
state $I([h|c:a])$, which is a partial function extended to include
the character $c$, with the additional condition that if $c$ was
already in $h$'s domain, the same glyph value is returned. 

The intuition for this specification is that $I(h)$ abstractly represents
a memo table (i.e., the function $h$), which characterizes all the 
glyphs allocated so far. 

On line 2, we specify the $\ctext{getdata}$ function. This just says
that if we call $\ctext{getdata}(l)$ on a glyph $l$, then we are
returned the data (the alphabetic character and font) for this glyph.

\begin{figure}
\begin{tabbing}
1 \qquad \= $\exists \ctext{make\_flyweight} :
\fonttp \to \bigcirc($\=$(\chartp \to \monad{\ctext{glyph}}) \times$ \\
\> \> $(\ctext{glyph} \to \monad{(\chartp \times \fonttp)})).$\\
2 \> \;\;\= $\forall f.\;$\=$\setof{\emp}$ \\
  \>\> \> $\run{\ctext{make\_flyweight}(f)}$ \\
  \>\> \> $\setof{a.\; \exists I.\; I([]) \land \validprop{\mbox{Flyweight}(I, \fst{a}, \snd{a}, f)}}$ 
\end{tabbing}
\caption{Flyweight Factory Specification}
\label{flyweight-factory-spec}
\end{figure}


The specification of the flyweight factory is given in
Figure~\ref{flyweight-factory-spec}.  Here, we assert the existence of
a function $\ctext{make\_flyweight}$, which takes a font $f$ as an
input argument, and returns two functions to serve as the
$\ctext{getchar}$ and $\ctext{getdata}$ functions of the flyweight. In
the postcondition, we assert the existence of some private state $I$,
which contains the table used to cache glyph creations.

This pattern commonly arises when encoding aggressively
object-oriented designs in a higher-order style --- we call a
function, which creates a hidden state, and returns other procedures
which are the only way to access that state. This style of
specification resembles the existential encodings of objects into type
theory. The difference is that instead of making the fields of an
object an existentially quantified \emph{value}~\cite{pierce-turner}, we
make use of existentially-quantified \emph{state}.

\begin{figure}
\begin{specification}
\nextline $\ctext{m}$\=$\ctext{ake\_flyweight} \equiv$ 
\nextline            \> $\lambda f:$\=$\fonttp.\;$
\nextline \> $[$\=$\letv{t}{\ctext{newtable}()}{}$ 
\nextline \> \> $\ctext{letv\;}$\=$\mathit{newglyph} =$ 
\nextline \> \> \> \!\!$[\lambda c.[$\=$\ctext{letv}\; x = \ctext{lookup}(t, c)\;\ctext{in}$
\nextline \> \>\>\> $\ctext{run}\;\ctext{case}(x,$\=
$\None \to [$\=$\ctext{letv}\; r = [\newref{\ctext{char} \times \ctext{font}}{c,f}] \;\ctext{in}$
\nextline \> \>\>\>\>\> $\ctext{letv } \_ = \ctext{update}(t, c, r) \;\ctext{ in }\;r],$ 
\nextline \> \>\>\>\> $\Some(r) \to [r])] \;\ctext{in}$ 
\nextline \> \> $\ctext{letv\;}\mathit{getdata} = [\lambda r.\; [!r]] \;\ctext{in}$ 
\nextline \> \> $(\mathit{newglyph}, \mathit{getdata})]]$
\\

\nextline $J : \ctext{font} \times \ctext{table} \times (\ctext{char} \rightharpoonup \ctext{glyph}) \to \assert$ 
\nextline $J(f, t, h) \equiv table(t,h) * \mathit{refs}(f, h)$ \\
\nextline $\mathit{refs}(f, h) = \forall^{*} c \in \domain{h}.\; h(c) \pointsto (c, f)$  \\

\nextline $\forall^* x \in \emptyset.\; P(x) \triangleq \emp $ 
\nextline $\forall^* x \in \setof{y} \uplus Y.\; P(x) \triangleq P(y) * \forall^* x \in Y.\; P(x)$
\end{specification}
\caption{Flyweight Implementation}
\label{flyweight-impl}
\end{figure}

In Figure~\ref{flyweight-impl}, we define $\ctext{make\_flyweight}$
and its predicates. In this implementation we have assumed the
existence of a hash table implementation with operations
$\ctext{newtable}$, $\ctext{lookup}$, and $\ctext{update}$, whose
specifications we give in Figure~\ref{hash-table-spec}. The
$\ctext{make\_flyweight}$ function definition takes a font argument
$f$, and then in its body it creates a new table $t$. It then
constructs two functions as closures which capture this state (and the
argument $f$) and operate on it. In lines 4-8, we define $newglyph$,
which takes a character and checks to see (line 6) if it is already in
the table. If it is not (lines 6-7), it allocates a new glyph
reference, stores it in the table, and returns the
reference. Otherwise (line 8), it returns the existing reference from
the table.  On line 9, we define $\mathit{getdata}$, which dereferences its
pointer argument and returns the result. This implementation does no
writes, fulfilling the promise made in the specification. The
definition of the invariant state $I$ describes the state of the table
$t$ (as the partial function $h$), which are hidden from clients.

Observe how the post-condition to $\ctext{make\_flyweight}$ nests the
existential state $I$ within the validity assertion to specialize the
flyweight spec to the \emph{dynamically} created table. Each created
flyweight factory receives its own private state, and we can reuse
specifications and proofs with no possibility that the wrong
$\ctext{getdata}$ will be called on the wrong reference, even though
they have compatible types.

\begin{figure}
\begin{specification}
\nextline $\mspec{\emp}{\ctext{newtable}}{a:\ctext{table}}{\mathit{table}(a, [])}$  \\

\nextline $\{\mathit{table}(t, h)\}$
\nextline $\ctext{lookup}(t, c)$ 
\nextline $\{a:\opttype{(\ctext{glyph})}.\;$\=$\mathit{table}(t, h) $ 
\nextline \> $\land ((c \not\in \domain{h} \land a = \None)) \vee (c \in \domain{h} \land a = \Some(h(c))) \}$ \\

\nextline $\{\mathit{table}(t, h)\}$
\nextline $\ctext{update}(t, c, g)$ 
\nextline $\{a:1.\; \mathit{table}(t, [h|c:g])\}$
\end{specification}
\caption{Hash Table Specification}
\label{hash-table-spec}
\end{figure}

\subsection{Verification of the Implementation}

To prove the correctness of the implementation, we will work
inside-out, first giving specifications and correctness proofs for the
``methods'' of the factory, and then using these to prove the correctness
of the $\mathsf{make\_flyweight}$ function. 

\begin{lemma}{(Correctness of \textit{newglyph})}
Define the function $\mathit{newglyph}$ as follows:
\begin{tabbing}
  $\mathit{newglyph} \equiv \lambda c.\; [$\=
     $\letv{x}{\ctext{lookup}(t,c)}{}$ \\
\>   ${run}\;\ctext{case}(x, $\=$\None \to [$\=$\letv{r}{[\newref{\ctext{glyph}}{c,f}]}{}$ \\
\>                                  \>            \>$\letv{\_}{\ctext{update}(t, c, r)}{}$ \\
\>                                  \>            \>$r]$\\
\>                                  \>$\Some(r) \to [r])]$ 
\end{tabbing}
\noindent This function satisfies the following specification:
\begin{displaymath}
\spec{J(f, t, h)}{\mathit{newglyph}(c)}{a:\ctext{glyph}}{\exists h' \supseteq h. J(f, t, h') \land h'(c) = a}
\end{displaymath}
\end{lemma}

\begin{proof}
Note that in this definition, the $t$ and $f$ variables occur free. We are going to 
prove our specification valid with respect to these variables, so that we can substitute
them with whatever actual terms in context that we need to use. 

\begin{tabbedproof}
\oo Assume we are in the precondition state $J(f, t, h)$ \\
\oo Hence we are in the precondition state $\mathit{table}(t, h) * \forall^{*}c \in \domain{h}.\;h(c) \pointsto (c,f)$ \\
\oo $\letv{x}{\ctext{lookup}(t,c)}{}$ \\
\oo We now have $\mathit{table}(t, h) * \forall^{*}c \in \domain{h}.\;h(c) \pointsto (c,f)$  \\
\ox $\land\; ((c \not \in \domain{h} \land x = \None) \vee (c \in \domain{h} \land x = \Some(h(c))))$ \\
\oo $\mathsf{case}(x,$ \\
\ooo $\None \to$ \\
\oooo Now it follows that $x = \None$, \\
\oooo Hence $\mathit{table}(t, h) * \forall^{*}c' \in \domain{h}.\;h(c') \pointsto (c',f) \land c \not \in \domain{h} \land x = \None$\\
\oooo $[\letv{r}{[\newref{\ctext{glyph}}{(c,f)}]}{}$ \\
\oooo So $r \pointsto (c,f) * \forall^{*}c' \in \domain{h}.\;h(c') \pointsto (c',f) * \mathit{table}(t, h) \land c \not \in \domain{h}$\\
\oooo $\letv{\_}{\ctext{update}(t, c, r)}{}$ \\
\oooo So $r \pointsto (c,f) * \forall^{*}c' \in \domain{h}.\;h(c') \pointsto (c',f) * \mathit{table}(t, [h|c:r]) \land c \not \in \domain{h}$\\
\oooo So $\exists h'.\; \forall^{*}c' \in \domain{h'}.\;h(c') \pointsto (c',f) * \mathit{table}(t, h') \land h' = [h|c:r]$\\
\oooo $r]$ \\
\oooo We can choose the witness to the existential to be $h'$, yielding \\
\ooox $\exists h' \supseteq h.\; \mathit{table}(t, h') \land h'(c) = a * \forall^{*}c' \in \domain{h}.\;h(c') \pointsto (c',f)$ \\
\oooo Hence $\exists h' \supseteq h.\; J(f, t, h') \land h'(c) = a$ \\
\ooo $\Some(r) \to $ \\
\oooo We know $x = \Some(r)$, \\
\oooo Hence $\mathit{table}(t, h) * \forall^{*}c' \in \domain{h}.\;h(c') \pointsto (c',f) \land x = \Some(h(c))$\\
\oooo So $J(f, t, h) \land h(c) = r$
\oooo $[r])$ \\
\oooo Hence $a.\; J(f, t, h) \land h(c) = a$ \\ 
\oooo Choose the witness to the existential to be $h$, yielding \\
\ooox $\exists h' \supseteq h. J(f, t, h') \land h'(c) = a$ \\ 
\oo Hence $\exists h' \supseteq h. J(f, t, h') \land h'(c) =a $\\

\end{tabbedproof}
\end{proof}

\begin{lemma}{(Correctness of $\mathit{getdata}$)}
Define the $\mathit{getdata}$ function as follows:
\begin{tabbing}
\qquad $\mathit{getdata} \equiv \semfun{r}{[!r]}$  
\end{tabbing}

\noindent This function satisfies the following specification:
\begin{displaymath}
  \mspec{J(f, t, h) \land r = h(c)}
        {\mathit{getdata}(r)}
        {a:\chartp \times \fonttp}
        {J(f, t, h) \land a = (c, f)}
\end{displaymath}
\end{lemma}

\begin{proof}
The correctness proof for this function is very simple, amounting to a single application
of the frame rule, together with the dereference rule.
\begin{tabbedproof}
\oo Assume our state is $\mathit{table}(t, h) * \mathit{refs}(f, h) \land r = h(c)$ \\
\oo Hence $c \in \domain{h}$ \\ 
\oo This is $\mathit{table}(t, h) * \mathit{refs}(f, h - [c:r]) * r \pointsto (c,f)$ \\
\oo $[!r]$ \\
\oo So $a.\; \mathit{table}(t, h) * (\mathit{refs}(f, h - [c:r]) * r \pointsto (c,f) \land a = (c,f)$ \\
\oo So $a.\; J(f, t, h) \land a = (c,f)$
\end{tabbedproof}
\end{proof}

\begin{lemma}{(Correctness of $\mathsf{make\_flyweight}$)}
  The $\mathsf{make\_flyweight}$ function meets the specification in Figure~\ref{flyweight-factory-spec}. 
\end{lemma}

Now, we can prove the correctness of the $\mathsf{make\_flyweight}$ function. To do
this, we will assume a font argument $f$, and then prove the correctness of the body
of the function.

\begin{proof}
\begin{tabbedproof}
\oo We begin with an assumed $\emp$ precondition. \\
\oo $[\letv{t}{\mathsf{newtable}()}{}$ \\
\oo Now our state is $\mathit{table}(t, [])$ \\
\oo $\letv{\mathit{newglyph}}{[N]}{}$ (where $N$ is the definition of $\mathit{newglyph}$ in Lemma 60) \\
\oo $\letv{\mathit{getdata}}{[G]}{}$ (where $G$ is the  definition of $\mathit{getdata}$ in Lemma 61)\\
\oo Now our state is $\mathit{table}(t, []) \land \mathit{newglyph} = N \land \mathit{getdata} = G$ \\
\oo Since the two equalities are pure, we may assume them ambiently. \\
\oo Then we can instantiate the specs of $\mathit{newglyph}$ and $\mathit{getdata}$ to know that\\
\oo $\spec{J(f, t, \sigma)}{\Run\;N(c)}{a:\ctext{glyph}}{\exists \sigma' \supseteq \sigma. J(f, t, \sigma') \land \sigma'(c) = a}$\\
\oo and \\
\oo $\spec{J(f, t, \sigma) \land r = \sigma(c)}
          {\Run\;G(r)}
          {a:\chartp \times \fonttp}
          {J(f, t, \sigma) \land a = (c, f)}$\\
\oo Then since valid specs allow us to introduce validity assertions, we know  \\
\ox $J(f, t, []) \land \validprop{\mbox{Flyweight}((\semfun{\sigma}{J(f,t,\sigma)}, \mathit{newglyph}, \mathit{getdata}, f)}$  \\
\oo We can existentially quantify, choosing a witness $I' = \semfun{\sigma}{J(f,t,\sigma)}$ to get \\
\ox $\exists I'.\; I'([]) \land \validprop{\mbox{Flyweight}(I', \mathit{newglyph}, \mathit{getdata}, f)}$  \\
\oo $(\mathit{newglyph}, \mathit{getdata})]$\\
\oo Now we know $\exists I.\;I([]) \land \validprop{\mbox{Flyweight}(I, \fst{a}, \snd{a}, f)}$\\
\end{tabbedproof}
\end{proof}

\section{Subject-Observer}

The subject-observer pattern is one of the most characteristic
patterns of object-oriented programming, and is extensively used in
GUI toolkits. This pattern features a mutable data structure called
the \emph{subject}, and a collection of data structures called
\emph{observers} whose invariants depend on the state of the
subject. Each observer registers a callback function with the subject
to ensure it remains in sync with the subject. Then, whenever the
subject changes state, it iterates over its list of callback
functions, notifying each observer of its changed state. While
conceptually simple, this is a lovely problem for verification, since
every observer can have a different invariant from all of the others,
and the implementation relies on maintaining lists of callback
functions in the heap.  

In our example, we will model this pattern with one type of subjects,
and three functions. A subject is simply a pair, consisting of a
pointer to a number, the subject state; and a list of observer
actions, which are imperative procedures to be called with the new
value of the subject whenever it changes. There is a function
$\ctext{newsub}$ to create new subjects; a function
$\ctext{register}$, which attaches observer actions to the subject;
and finally a function $\ctext{broadcast}$, which updates a subject
and notifies all of its observers of the change. 


We give a specification for the subject-observer pattern below:
%

\begin{tabbing}
1 \qquad \= $\exists sub : A_s \times \N \times \seqsort{((\N \To \assert) \times (\N \to \monad{1}))}.$ \\
2 \> $\exists \ctext{newsub} : \N \to \monad{A_s},$ \\ 
3 \> $\exists \ctext{register} : A_s \times (\N \to \monad{1}) \to \monad{1},$ \\
4 \> $\exists \ctext{broadcast} : A_s \times \N \to \monad{1}.$ \\
\\[0.5em]
5 \>$\forall n.\; \mspec{\emp}{{\ctext{newsub}(n)}}{a:A_s}{sub(a, n, \epsilon)}$ \\
\> $\specand$ \\
6 \> $\forall f, O, s, n, os. $\=$(\forall i, k. \mspec{O(i)}{{f(k)}}{a:1}{O(k)})$ \\
7\> \>$\specimp$\=$\angles{sub(s, n, os)}$ \\
8\> \>          \>${\ctext{register}(s, f)}$ \\
9 \> \>          \>$\angles{a:1.\; sub(s, n, (O,f)\cdot os)}$ \\
\> $\specand$ \\
10 \> $\forall s,i,os,k.\; $\=$\angles{sub(s, i, os) * obs(os)}$ \\
  \>                       \>${\ctext{broadcast}}(s,k)$ \\
  \>                       \>$\angles{a:1.\; sub(s, k, os) * obs\_at(os, k)}$ 
\\[0.5em]
$obs(\epsilon) \;\qquad\qquad $\=$\equiv \emp$ \\
$obs((O,f)\cdot os) $\>$\equiv (\exists i.\; O(i)) * obs(os)$ 
\\[0.5em]
$obs\_at(\epsilon, k) \;\qquad\qquad $\=$\equiv \emp$ \\
$obs\_at((O,f)\cdot os, k) $\>$\equiv O(k) * obs\_at(os, k)$ 
\\
\end{tabbing}

%
On line 1 we assert the existence of a three-place predicate $sub(s,
n, os)$. The first argument is the subject $s$, whose state this
predicate represents. The second argument $n$ is the data the
observers depend on, and the field $os$ is a sequence of callbacks
paired with their invariants. That is, $os$ is a sequence of pairs,
each consisting of the observer functions which act on a state,
preceeded by the predicate describing what that state should be.

On lines 2-4, we assert the existence of $\ctext{newsub}$,
$\ctext{register}$ and $\ctext{broadcast}$, which create a new
subject, register a callback, and broadcast a change, respectively.

$\ctext{register}$ is a higher order function, which takes a subject
and an observer action as its two arguments. The observer action is a
function of type $\N \to \monad{1}$, which can be read as saying it
takes the new value of the subject and performs a side-effect. Because
$\ctext{register}$ take a procedure as an argument, its specification
must say how this observer action should behave. $\ctext{register}$'s
specification on lines 6-9 accomplishes this via an implication over
Hoare triples. It says that \emph{if} the function $f$ is a good
observer callback, \emph{then} it can be safely registered with the
subject. Here, a ``good callback'' $f$ is one that takes an argument
$k$ and sends an observer state to $O(k)$. If this condition is
satisfied, then $\ctext{register}(s, f)$ will add the pair $(O,f)$ to
the sequence of observers in the $sub$ predicate.

One point worth focusing on is that a given subject can have many
different observers $o_i$, each with a \emph{different} invariant
$O_i$ and callback $f_i$. This means that the $\mathit{sub}(s, n, os)$
predicate is a genuinely higher-order one, since it contains a list of
arbitrary predicate values. This is natural, given that the core of
the subject-observer pattern is to call a list of unknown higher-order
imperative procedures, but it is still worth pointing out explicitly. 

$\ctext{broadcast}$ updates a subject and all its interested
observers.  The precondition state of $\ctext{broadcast}(s,k)$
requires the subject state $sub(s,n,os)$, and all of the observer
states $obs(os)$. The definition $obs(os)$ takes the list of observers
and yields the separated conjunction of the observer states. So when
$\ctext{broadcast}$ is invoked, it can modify the subject and any of
its observers. Then, after the call, the postcondition puts the $sub$
predicate and all of the observers in the same state $k$. The
$obs\_at(os,k)$ function generates the separated conjunction of all
the $O$ predicates, all in the same state $k$.

The implementation follows:

\begin{tabbing}
1 \qquad \= $A_s \equiv \reftype{\N} \times \listtype{(\N \to \monad{1})}$
\\[0.5em]
2 \> $sub(s, n, os) \equiv$\=$ \fst{s} \pointsto n * 
              list(\snd{s}, map\; \snd{} os) \land Good(os)$ 
\\[0.5em]
3 \> $Good(\epsilon) \!\qquad\qquad \equiv \top$ \\
4 \> $Good((O,f)\cdot os) \equiv\; $\=
   $\validprop{(\forall i,k.\; \spec{O(i)}{\run{f(k)}}{a:1}{O(k)})} \land Good(os)$ 
\\[0.5em]

5 \> $\ctext{register}(s, f) \equiv$ \=
         $[$\= $\letv{cell}{\comp{!(\snd{s})}}{}$ \\
6 \> \> \> $\letv{r}{\comp{\newref{\Listcontent{(\N \to \monad{1})}}{cell}}}{}$ \\
7 \> \> \> $\snd{s} := \Cons(f, r)]$
\\[0.5em]

8  \> $\ctext{broad}$\=$\ctext{cast}(s, k) \equiv$ \\
9  \>  \> $[$\=$\letv{\_}{[\fst{s} := k]}{\ctext{loop}(k, \snd{s})}]$ 
\\[0.5em]


10 \> $\ctext{loop}$\=$(k, list) \equiv $\\
   \>         \>$[$\=$\letv{cell}{[!list]}{}$ \\
11 \>\>\> $\run{}\ctext{case}(cell,$\= 
            $\Nil \to [()],$ \\
12 \>\>\>\> $\Cons(f, tl) \to [$\=$\letv{\_}{f(k)}{}$ \\
13  \>\>\>\> \> $\run{\ctext{loop}(k,tl)}])$ \\[0.5em]

14 \> $\ctext{new}\ctext{sub}(n) \equiv$ \=
          $[$\=$\letv{data}{\newref{\N}{n}}{}$ \\
15 \> \> \> $\letv{callbacks}{\newref{\Listcontent{(\N \to \monad{1})}}{\Nil}}{}$ \\
16 \> \> \> $(data, callbacks)]$
\end{tabbing}


In line 1, we define the concrete type of the subject $A_s$ to be a
pair of a pointer to a reference, and a mutable list of callback
functions. On line 2, we define the three-place subject predicate,
$sub(s,n,os)$. The first two subclauses of the predicate's body
describe the physical layout of the subject, and assert that the first
component of $s$ should point to $n$, and that the second component of
$s$ should be a linked list containing the function pointers in
$os$. (The $list$ predicate is described in
Figure~\ref{iterator-pred-impl}, when we give the definition of the
iterator predicates.)

Then we require that $os$ be ``Good''. $Good$-ness is defined on lines
3 and 4, and says a sequence of predicates and functions is good when
every $(O,f)$ pair in the sequence satisfies the same validity
requirement the specification of $\ctext{register}$ demanded -- that
is, that each observer function $f$ update $O$ properly.  (Note that
we make use of our ability to nest specifications within assertions,
in order to constrain the behavior of code stored in the heap.)

Next, we give the implementations of $\ctext{register}$ and
$\ctext{broadcast}$. $\ctext{register}$, on lines 5-7, adds its
argument to the list of callbacks. Though the code is trivial, its
correctness depends on the fact the $Good$ predicate holds for the
extended sequence --- we use the fact that the argument $f$ updates 
$O$ properly to establish that the extended list remains $Good$. 

$\ctext{broadcast}$, on lines 8-9, updates the subject's data field
(the first component), and then calls $\ctext{loop}$ (on lines 10-13)
to invoke all the callbacks. $\ctext{loop}(k, \snd{s})$ just recurs
over the list and calls each callback with argument $k$. The
correctness of this function also relies on the $Good$ predicate --
each time we call one of the functions in the observer list, we use
the hypothesis of its behavior given in $Good(os)$ to be able to make
progress in the proof.


Below, we give a simple piece of client code using this interface.


\begin{tabbing}
1 \qquad \= 
$\setof{\emp}$ \\
2 \> 
$\letv{s}{\ctext{newsub}(0)}{}$ \\
3 \> $\setof{sub(s, 0, \epsilon)}$ \\
4 \> $\letv{d}{\newref{\N}{(0)}}{}$ \\
5 \> $\letv{b}{\newref{\ctext{bool}}{(\ctext{true})}}{}$ \\
6 \> $\setof{sub(s, 0, \epsilon) * d \pointsto 0 * b \pointsto \ctext{true}}$\\
7 \> $\letv{()}{\ctext{register}(s, f)}{}$\\
8 \> $\setof{sub(s, 0, (double, f)\cdot\epsilon) * double(0) * b \pointsto \ctext{true}}$ \\
9 \> $\letv{()}{\ctext{register}(s, g)}{}$\\
10 \> $\setof{sub(s, 0, (even, g)\cdot(double, f)\cdot\epsilon) * double(0) * even(0)}$ \\
11 \> $\ctext{broadcast}(s, 5)$ \\
12 \> $\setof{sub(s, 5, (even, g)\cdot(double, f)\cdot\epsilon) * double(5) * even(5)}$ \\
13 \> $\setof{sub(s, 5, (even, g)\cdot(double, f)\cdot\epsilon) * d \pointsto 10 * b \pointsto \ctext{false}}$ 
\\[0.5em]
14 \> $f \qquad \qquad $\=$\equiv \lambda n:\N.\; [d := 2 \times n]$ \\
15 \> $double(n)$ \> $\equiv d \pointsto (2 \times n)$ \\
16 \> $g$ \> $\equiv \lambda x:\N.\; [b := even?(x)]$ \\
17 \> $even(n)$ \> $\equiv b \pointsto even?(n)$ \\
\end{tabbing}

% \vspace{-1em}
We start in the empty heap, and create a new subject $s$ on line 2.
On line 4, we create a new reference to $0$, and on line 5, we create
a reference to $\ctext{true}$. So on line 6, the state consists of a
subject state, and two references.  On line 7, we call
$\ctext{register}$ on the function $f$ (defined on line 14), which
sets $d$ to twice its argument. To the observer list in sub, we add
$f$ and the predicate $double$ (defined on line 15), which asserts
that indeed, $d$ points to two times the predicate argument. On line
8, we call $\ctext{register}$ once more, this time with the function
$g$ (defined on line 16) as its argument, which stores a boolean
indicating whether its argument was even into the pointer $b$. Again,
the state of $sub$ changes, and we equip $g$ with the $even$ predicate
(defined on line 17) indicating that $b$ points to a boolean
indicating whether the predicate argument was even or not. Since $d
\pointsto 0$ and $b \pointsto \ctext{true}$ are the same as
$double(0)$ and $even(0)$, so we can write them in this form on line
10.  We can now invoke $\ctext{broadcast}(s, 5)$ on line 11, and
correspondingly the states of all three components of the state shift
in line 12.  In line 13, we expand $double$ and $even$ to see $d$
points to 10 (twice 5), and $b$ points to $\ctext{false}$ (since 5
is odd).

\textbf{Discussion.} One nice feature of the proof of the
subject-observer implementation is that the proofs are totally
oblivious to the concrete implementations of the notification
callbacks, or to any details of the observer invariants. Just as
existential quantification hides the details of a module
implementation from the clients, the universal quantification in the
specification of $\ctext{register}$ and $\ctext{broadcast}$ hides all
details of the client callbacks from the proof of the implementation
-- since they are free variables, we are unable to make any
assumptions about the code or predicates beyond the ones explicitly
laid out in the spec. Another benefit of the passage to higher-order
logic is the smooth treatment of observers with differing invariants;
higher-order quantification lets us store and pass formulas around,
making it easy to allow each callback to have a totally different
invariant. 

\subsection{Correctness Proofs for Subject-Observer}

\subsubsection{Proof of the $\ctext{register}$ Function}
\begin{proof}
\begin{tabbedproof}
\oo Assume we have $f, O, s, cs, n, os$ and the specification $\forall i,k.\;\mspec{O(i)}{{f(k)}}{a:1}{O(k)}$ \\
\ooo Assume we are in the prestate $\setof{\mathit{sub}((s,cs), n, os)}$ \\
\ooo This state is equivalent to $s \pointsto n * \mathit{list}(cs, \mathit{map}\;\snd{}\;os) \land \mathit{Good}(os)$ \\
\ooo By the definition of $\mathit{list}$, we know \\
\ooox $\mathit{list}(cs, \mathit{map}\;\snd{}\;os) = \exists cell.\;cs \pointsto cell * \mathit{listcell}(cell, \mathit{map}\;\snd{}\;os)$ \\
\ooo $[\letv{\mathit{cell}}{[!cs]}{}$ \\
\ooo We can drop the existential now, giving a state of \\
\ooox $s \pointsto n * cs \pointsto cell * \mathit{listcell}(cell, \mathit{map}\;\snd{}\;os) \land \mathit{Good}(os)$ \\
\ooo $\letv{r}{[\newref{\reftype{\listtype{(\N \to \monad{\unittype})}}}{\mathit{cell}}]}{}$ \\
\ooo We add $r \pointsto \mathit{cell}$ to the state, and fold the definition of list, to get \\
\ooox $s \pointsto n * cs \pointsto - * \mathit{list}(r, \mathit{map}\;\snd{}\;os) \land \mathit{Good}(os)$ \\ 
\ooo $cs := \Cons(f, r)]$ \\
\ooo Therefore $s \pointsto n * cs \pointsto \Cons(f,r) * \mathit{list}(r, \mathit{map}\;\snd{}\;os) \land \mathit{Good}(os)$ \\ 
\ooo Therefore $s \pointsto n * \mathit{list}(cs, \mathit{map}\;\snd{}\;((O,f) \cdot os)) \land \mathit{Good}(os)$ \\ 
\ooo Since we assume $\forall i,k.\;\spec{O(i)}{\run{f(k)}}{a:1}{O(k)}$, we can conjoin it to the state \\
\ooox to get $s \pointsto n * \mathit{list}(cs, \mathit{map}\;\snd{}\;((O,f) \cdot os)) \land \mathit{Good}((O,f) \cdot os)$ \\ 
\ooo This is $\mathit{sub}((s,cs), n, (O,f)\cdot os)$
\end{tabbedproof}
\end{proof}

\subsubsection{Proof of the $\ctext{newsub}$ Function}
\begin{proof}
\begin{tabbedproof}
\oo Assume we have a variable $n$ and the initial state $\emp$\\
\ooo Now consider the body of the $\ctext{newsub}$ function: \\
\ooo $[\letv{\mathit{data}}{\newref{\N}{n}}{}$ \\
\ooo The state is $\mathit{data} \pointsto n$ \\
\ooo $\letv{\mathit{callbacks}}{\newref{\Listcontent{\N\to\monad{1}}}{\Nil}}{}$ \\
\ooo The state is $\mathit{data} \pointsto n * \mathit{callbacks} \pointsto \Nil$ \\
\ooo This is equivalent to $\mathit{data} \pointsto n * \mathit{list}(\mathit{callbacks}, \epsilon)$ \\
\ooo This is equivalent to $\mathit{data} \pointsto n * \mathit{list}(\mathit{callbacks}, \epsilon) \land \mathit{Good}(\epsilon)$ \\
\ooo This is equivalent to $\mathit{sub}(\mathit{(data,callbacks)}, n, \epsilon)$ \\
\ooo $(data, callbacks)]$ \\
\ooo Therefore $\mathit{sub}(a, n, \epsilon)$ with return value $a$ 
\end{tabbedproof}
\end{proof}

\subsubsection{Proof of the $\ctext{broadcast}$ Function}

To prove this function, we first need to prove an auxilliary lemma about the $\ctext{loop}$ function:

\begin{lemma}{($\ctext{loop}$ Invariant)}
For all $k$, $fs$  and $os$ we have
\begin{displaymath}
\spec{\mathit{Good}(os) \land \mathit{list}(fs, \mathit{map}\;\snd{}os) * \mathit{obs}(os)}{\ctext{loop}(k, fs)}{a:1}{\mathit{list}(fs, \mathit{map}\;\snd{}\;os) * \mathit{obs\_at}(os, k)}
\end{displaymath}
\end{lemma}
\begin{proof}
\begin{tabbedproof}
\oo Assume we have a suitable $k$ and $fs$, and then proceed by induction on $os$.  \\
\oo Pull $\mathit{Good}(os)$ into the context as a pure assertion, and then frame it away. \\
\oo Suppose $os = \epsilon$:  \\
\ooo Then our precondition is $\mathit{list}(fs, \epsilon) * \mathit{obs}(\epsilon)$ \\
\ooo This is equivalent to $fs \pointsto \Nil$ \\
\ooo $[\letv{\mathit{cell}}{[!fs]}{}$  \\
\ooo So we know $fs \pointsto \Nil \land \mathit{cell} = \Nil$ \\
\ooo Since $\mathit{cell} = \Nil$ and equational reasoning, we know the remainder of the program is $\unit$\\
\ooo $\unit]$ \\
\ooo So the state is still $\mathit{list}(fs, \epsilon)$ \\
\ooo This is equivalent to $\mathit{list}(fs, \epsilon) * \emp$ \\
\ooo This is equivalent to $\mathit{list}(fs, \epsilon) * \mathit{obs\_at}(\epsilon, k)$ \\
\ooo This is equivalent to $\mathit{Good}(\epsilon) \land \mathit{list}(fs, \epsilon) * \mathit{obs\_at}(\epsilon, k)$ \\
\ooo This is equivalent to $\mathit{Good}(os) \land \mathit{list}(fs, \epsilon) * \mathit{obs\_at}(\epsilon, k)$ \\
\oo Suppose $os = (O,f) \cdot os'$:  \\
\ooo Then our precondition is $\mathit{list}(fs, f \cdot \mathit{map}\;\snd{}\;os') * \mathit{obs}((O,f)\cdot os')$ \\
\ooo This is equivalent to $\exists fs'.\; fs \pointsto \Cons(f, fs') * \mathit{list}(fs',  \mathit{map}\;\snd{}\; os') * \exists j.\;O(j) * \mathit{obs}( os')$ \\
\ooo By dropping existentials, $fs \pointsto \Cons(f, fs') * \mathit{list}(fs',  \mathit{map}\;\snd{}\;os') * O(j) * \mathit{obs}( os')$ \\
\ooo $[\letv{\mathit{cell}}{[!fs]}{}$ \\
\ooo We know $fs \pointsto \mathit{cell}  * \mathit{list}(fs',  \mathit{map}\;\snd{}\;os') * O(j) * \mathit{obs}( os') \land \mathit{cell} = \Cons(f, fs')$ \\
\ooo Since $\mathit{cell} = \Cons(f, fs')$, we can use equational reasoning to eliminate the case \\
\ooo Since we assumed $\mathit{Good}(os)$, we know that $\mathit{Good}{(O,f)\cdot os'}$ \\
\ooo Hence we know that $\validprop{(\forall i,k.\; \spec{O(i)}{\run{f(k)}}{a:1}{O(k)})} \land \mathit{Good}(os')$ \\
\ooo Therefore we know that $(\forall i,k.\; \spec{O(i)}{\run{f(k)}}{a:1}{O(k)})$ \\
\ooo $\letv{\mathit{\_}}{f(k)}{}$ \\
\ooo We know $fs \pointsto \mathit{cell}  * \mathit{list}(fs',  \mathit{map}\;\snd{}\;os') * O(k) * \mathit{obs}( os') \land \mathit{cell} = \Cons(f, fs')$ \\
\ooo Add $\mathit{Good}(os')$ to the precondition, since it is a consequence of $\mathit{Good}(os)$, before\\
\ooo $\mathsf{loop}(k, fs')]$\\
\ooo Then by induction, we can conclude \\
\ooo $Good(os') \land fs \pointsto \mathit{cell}  * \mathit{list}(fs',  \mathit{map}\;\snd{}\;os') * O(k) * \mathit{obs\_at}(os', k) \land \mathit{cell} = \Cons(f, fs')$ \\
\ooo This is equivalent to $\mathit{Good}(os') \land \mathit{list}(fs, \mathit{map}\;\snd{}\;os) * \mathit{obs\_at}(os, k)$ \\
\ooo Since we know $\mathit{Good}(os)$, we can strengthen this to  \\
\ooo  $\mathit{Good}(os) \land \mathit{list}(fs, \mathit{map}\;\snd{}\;os) * \mathit{obs\_at}(os, k)$ \\
\oo Hence $\mathit{Good} \land \mathit{list}(fs, \mathit{map}\;\snd{}\;os) * \mathit{obs\_at}(os, k)$
\end{tabbedproof}
\end{proof}

There is an interesting feature of this proof which distinguishes it
from the proofs we gave for the iterator implementation. In this
proof, we reason about branches by doing a case analysis in the
program logic, and then using the specification-level equality to
justify \emph{simplifying} the program we are proving. That is, when
we consider an empty input list, we can use the equational theory of
the lambda calculus to simplify the program we are proving to
eliminate that case altogether.


Now, the proof of the $\mathsf{broadcast}$ function is quite easy: 

\begin{tabbedproof}
\oo Assume we have $s, i, os, k$ with a precondition of $\mathit{sub}(s, i, os) * \mathit{obs}(os)$ \\
\oo This is equivalent to $\mathit{Good}(os) \land \fst{s}\pointsto i * \mathit{list}(\snd{s}, \mathit{map}\;\snd{}\;os) * \mathit{obs}(os)$ \\
\oo $[\letv{\_}{[\fst{s} := k]}{}$ \\
\oo This yields $\mathit{Good}(os) \land \fst{s}\pointsto k * \mathit{list}(\snd{s}, \mathit{map}\;\snd{}\;os) * \mathit{obs}(os)$ \\
\oo $\mathsf{loop}(k, \snd{s})]$ \\
\oo This yields $\mathit{Good}(os) \land \fst{s}\pointsto k * \mathit{list}(\snd{s}, \mathit{map}\;\snd{}\;os) * \mathit{obs\_at}(os, k)$ \\
\oo This is equivalent to $\mathit{sub}(s, k, os) * \mathit{obs\_at}(os,k)$ 
\end{tabbedproof}

% 
%
% In the tech report: include a link to the Ynot sources!
%


\section{Ynot Experiments}
\label{sec:ynot-experiments}

In this section we give a brief description of our experiments with
translating the design pattern specifications and implementations from
the earlier sections into Hoare Type Theory (HTT) and verifying them in the
Ynot implementation of HTT. More details can be found in the 
technical report~\cite{svendsen08}.

These experiments serve to (1) increase our confidence in the earlier given
specifications and implementations and their associated paper proofs; (2)
provide a starting point for a comparison of the specification logic of the
present paper and the Ynot type theory; (3) exercise the Ynot
implementation.

Ynot is an axiomatic extension to the Coq proof assistant, that supports
writing, reasoning about, and extracting higher-order, dependently-typed
programs with side-effects~\cite{nanevski08}.  Coq already includes a
powerful functional language that supports dependent types, but that
language is limited to pure, total functions.  Ynot extends Coq with
support for computations that may have effects such as non-termination,
accessing a mutable store, and throwing/catching exceptions.  The axioms of
Ynot form a small trusted computing base which has been formally justified
in previous work on Hoare Type Theory (HTT)~\cite{nanevski06separation,
  nanevski07esop, petersen08}.

As in the specification logic described in the earlier sections, Ynot
also makes use a monads, to ensure a monadic separation of effects and
pure Coq.  In Ynot specifications are types and one of the types is
the monadic type of computations $\{ P \} x:\tau \{ Q \}$; if a
computation has this type and it is run in a heap $i$ satisfying $P$
and it terminates, then it will produce a value $x$ of type $\tau$ and
result in a new heap $j$ such that the predicate $Q(i,j)$
holds. Loosely speaking, we may thus think of Ynot as a type theory
corresponding to the specification logic for Idealized ML presented
earlier under a Curry-Howard style correspondence.  Following this
intuition, we have experimented with translating the earlier described
design pattern specifications and implementations into Ynot and
formally verified them in Ynot.  We now describe the translations and
the lessons learned.

Note first, however, that in Ynot post-conditions are expressed in terms of
both the initial and the final heap. This is an alternative to the use of
logical variables as expressed by universally quantifying over variables
whose scope extends to both the pre- and post-condition.  We can thus
translate an Idealized ML specification,
{\small\begin{align*}
\forall x : \tau.\ \{ P(x) \}\ comp\ \{ a : 1.\ Q(x) \}
\end{align*}}%
into the following Ynot type
{\small\begin{align*}
&\{ \lambda i : \HEAP.\ \exists x : \tau.\ P\ x\ i \}\\
&\quad a : 1\\
&\{ \lambda i : \HEAP.\ \lambda j : \HEAP.\ \forall x : \tau.\ P\ x\ i
\rightarrow Q\ x\ j \}
\end{align*}}%
where $i$ is the initial heap and $j$ is the terminal heap. We will usually
abbreviate this type as follows: 
{\small\begin{align*}
\{ i.\ \exists x.\ P\ x\ i \}\ a : 1\ \{ i\ j.\ \forall x.\ P\ x\ i \rightarrow Q\ x\ j \}
\end{align*}}

\subsection{Flyweight in Ynot}
\label{sec:ynot-flyweight}

Besides Hoare triples, Idealized ML's specification language contains
specifications of the form $\{ P \}$, for asserting that $P$ is true. In the
flyweight specification this is used to express that calling $getdata$ with the
same character multiple times, produces the same reference. In HTT we can
express that a proposition $P$ is true by returning an element of
the subset type, $\{ x : 1 \mid P \}$, where $x$ is not free in $P$. 

The assertion language of Idealized ML also contains an expression for
asserting that a given specification holds. In the Flyweight specification
this is used in the post-condition of $make\_flyweight$, to assert that the
code returned implements a Flyweight. In HTT, we can express the same by
simply giving a more precise type for the return value of the
$make\_flyweight$ computation.

In the Ynot implementation, we have generalized the specification, such that
the computation can generate a flyweight for values of an arbitrary monotype.
The flyweight factory computation therefore also has to take as an argument, a
function, $\alpha_{eq}$, for deciding equality between $\alpha$ values. 

The rest of the specification can be translated almost directly into HTT,
however, we have made a few changes, to simplify the formal verification of the
implementation in Ynot. 
\begin{itemize}
\item In the specification of $newchar$, instead of using a set to associate
arguments with objects, we have used a partial function (i.e., a total
function from $\alpha$ to $\opttype\ \LOC$).
\item In the above specification the predicate $I$ has to specify the
representation of both the object table and the objects. We have
split $I$ into two predicates, $table$ and $refs$, and changed the precondition
of $newchar$ to the HTT equivalent of $table(...) * (refs(...) \land
chars(S))$, to make it explicit that the object table and the objects
are in separate subheaps, to simplify verification.
\end{itemize}
The final HTT type of the Flyweight factory thus looks as follows:
{\small\begin{align*}
&\Pi \alpha : \MONO.\ \Pi \alpha_{eq} : (\Pi x : \alpha.\ \Pi y : \alpha.\ \{ z : 1 \mid x = y \} + \{ z : 1 \mid x \neq y \}).\\
&\{ i.\ emp\ i \}\\
&\quad r : \Sigma table : (\alpha \rightarrow \opttype\ \LOC) \rightarrow \HEAP \rightarrow \PROP.\\
&\quad\quad \Sigma refs : (\alpha \rightarrow \opttype\ \LOC) \rightarrow \HEAP \rightarrow \PROP.\\
&\quad\quad \Sigma objat : \LOC \rightarrow \alpha \rightarrow \HEAP \rightarrow \PROP.\\
&\quad\quad\Sigma prf_1 : \{ x : 1 \mid \forall h, l, l', a, a', f.\ objat\ l\ a\ h \land objat\ l'\ a'\ h\ \land\\
&\quad\quad\quad\quad\quad\quad\quad\quad\quad\quad refs\ f\ h \rightarrow (l = l' \leftrightarrow a = a') \}.\\
&\quad\quad \Pi a : \alpha.\\
&\quad\quad\quad \{ i.\ \exists f.\ (table\ f * (\lambda h.\ allobjat(\alpha, objat, f, h) \land refs\ f\ h))\ i \}\\
&\quad\quad\quad\quad l : \LOC\\
&\quad\quad\quad \{ i\ j.\ \forall f.\ (table\ f * (\lambda h.\ allobjat(\alpha, objat, f, h) \land refs\ f\ h))\ i \rightarrow\\
&\quad\quad\quad\quad\quad ((\forall l'.\ f\ a = Some\ l' \rightarrow l = l')\ \land\\
&\quad\quad\quad\quad\quad (table\ f[a \mapsto l] * (\lambda h.\ allobjat(\alpha, objat, f[a \mapsto l], h)\ \land\\
&\quad\quad\quad\quad\quad\quad\quad\quad\quad\quad\quad\quad\quad\quad\quad\quad refs\ f[a \mapsto l]\ h))\ j) \}\ \times\\
&\quad\quad \Pi l : loc.\\
&\quad\quad\quad \{ i.\ \exists a : \alpha,\ objat\ l\ a\ i \}\\
&\quad\quad\quad\quad r : \alpha\\
&\quad\quad\quad \{ i\ j.\ \forall a : \alpha,\ objat\ l\ a\ i \rightarrow (i = j \land r = a) \}\\\
&\{ i\ j.\ ((fst\ r)\ []\ * (\lambda h.\ allobjat(\alpha, fst\ (snd\ (snd\ r)), [], h)\ \land\\
&\quad\quad\quad\quad\quad\quad\quad\quad\quad\quad\quad (fst\ (snd\ r))\ []\ h))\ j \}
\end{align*}}
where
\begin{align*}
allobjat(\alpha, objat, f, h) &\equiv \forall l : \LOC, o : \alpha.\ f\ o = Some\ l\ \rightarrow\\&\quad\quad\quad (objat\ l\ o * (\lambda h.\ \TRUE))\ h
\end{align*}
and $[] \equiv (\lambda x.\ None)$.

We were able to formally verify that the earlier given implementation of
the Flyweight pattern has the above type in Ynot.\footnote{The Coq script was 
780 lines long.}

\subsection{Iterators in Ynot}
\label{sec:iterators-in-ynot}

The translation of the Iterator specification and implemenation and the
formal verification in Ynot was straightforward, except for the
verification of {\tt next}, when the iterator is a {\tt Map2}.\footnote{The
Coq script was 2122 lines long.}  In that case, the implementation makes
two recursive calls to {\tt next} that each work on two subheaps of the
initial heap and the current Ynot implementation based on the $nextvc$
tactic (see~\cite{nanevski08}) for simplifying proof obligations forces one
to prove some preciseness properties because of the use of binary
post-conditions. It is unclear whether the preciseness problem encountered
is a limitation of binary post-conditions in general or the current Ynot
implementation; we think it is the latter. We did not finish the proof for
{\tt next} in this case, either with or without $nextvc$ (without $nextvc$
the proof became too long for us to finish by hand).  New versions of Ynot
should provide better tactic support for such examples. 
We did succeed in completing the formal verification of the
iterator pattern  without the {\tt Map2} iterator.

\subsection{Subject-Observers in Ynot}
\label{sec:subject-observer-in-ynot}

The Idealized ML implementation of the subject-observer pattern uses an
assertion $S\valid$ in the predicate $good$ to express that callback functions
are "good". HTT does not support this form of assertion. However, since
specifications are types, we can express the type of pairs of $(O, f)$ such
that $f$ is a good call function with respect to the predicate $O$ with the
following type: 
{\small\begin{align*}
T &\equiv \Sigma O : \N \rightarrow \HPROP.\\
&\quad\quad (\Pi m : \N.\ \{ i.\ \exists k : \N.\ O\ k\ i\}\ a : 1\ \{ i\ j.\ \forall k : \N.\ O\ k\ i \rightarrow O\ m\ j \})
\end{align*}}
Hence, we can restrict the quantification of $os$ in the specification of
$broadcast$ and $register$ to lists of good callback functions, $\listtype T$.
We can thus express the subject-observer pattern with the following HTT type:
{\small
\begin{align*}
&\Sigma \alpha : \TYPE.\ \Sigma sub : \alpha \times \N \times \listtype T
\rightarrow \HPROP.\\
&\quad \Pi n : \N.\ \{ i.\ emp\ i \} a : \alpha \{ i\ j.\ sub\ (a, n, [])\ j \}\ \times\\
&\quad \Pi a : \alpha.\ \Pi t : T.\\
&\quad\quad\{ i.\ \exists n : \N, os : list\ T. sub\ (a, n, os)\ i \}\\
&\quad\quad\quad r : 1\\
&\quad\quad \{ i\ j.\ \forall n : \N, os : list\ T.\ sub\ (a, n, os)\ i \rightarrow sub\ (a, n, t::os)\ j
\}\ \times\\
&\quad \Pi a : \alpha.\ \Pi m : \N.\\
&\quad\quad\{ i.\ \exists n : \N, os : list\ T.\ (sub\ (a, n, os) * obs\ os)\ i \}\\
&\quad\quad\quad r : 1\\
&\quad\quad \{ i\ j.\ \forall n : \N, os : list\ T.\ (sub\ (a, n, os) * obs\ os)\ i\\
&\quad\quad\quad\quad\quad \rightarrow (sub\ (a, m, os) * obs\_at\ (os, k))\ j \}
\end{align*}} In the Idealized ML implementation of the subject-observer,
the registered callback functions are stored in the heap. Since types and
specifications are separate in Idealized ML, the type of these computations
can be very weak, i.e., $\N \rightarrow \monad 1$, because the
specification language allows us to express that if these are "good"
callback functions then the broadcast computation will do a broadcast when
performed. In HTT there is no separate specification language, so these
callback functions have to be stored with a much stronger type, so that it
is possible to infer from their type that they are "good" callback
functions when they are retrieved from the heap.\footnote{In the technical
  report~\cite{svendsen08} we discuss an alternative translation into HTT
  based on the idea that implications between specifications in Idealized
  ML should be translated into function types in HTT, but that leads to
  an implementation that does not capture subject-observer pattern because
  it essentially results in a functional implementation.}

Ynot is based on the predicative version of HTT\cite{nanevski06separation} in
which dependent sums are predicative, i.e., for $\Sigma x : A. B$ to be a
monotype, both $A$ and $B$ have to be monotypes. Since the type of heaps in Ynot is defined as a subset of the type $\N \times \Sigma \alpha : \MONO.
\alpha$ and $\MONO$ is not a monotype, it follows that the $T$ type above is not a monotype either and that values of type $T$ cannot be stored in
the heap. It is thus unclear whether it is possible to give an implementation
of the above type in Ynot; the obvious attempt leads to a universe
inconsistency error in Coq, reflecting the predicativity issues just
discussed. 

The impredicative version of HTT \cite{petersen08} has an impredicative sum
type, $\Sigma^T x : A. B$, which is a monotype if $B$ is. Hence, in the
impredicative version of HTT, we can store values of type $T$, by using
impredicative sums. We conjecture that the implementation derived from
translating the Idealized ML implementation has the above 
type in impredicative HTT.



%%% Local Variables: 
%%% mode: latex
%%% TeX-master: patterns
%%% End: 


% \section{Related Work}
% % \vspace{-0.5em}
% 
% In his dissertation~\cite{parkinson-thesis}, Parkinson gave as an
% example a simple iterator protocol, lacking the integration with
% composites we have exhibited.  Subsequently, we formalized a similar
% account of iterators~\cite{iterator}, again lacking the integration
% with composites. Jacobs, Meijer, Piessens and
% Schulte~\cite{iterators-revisited} extend Boogie with new rules for
% the coroutine constructs C\# uses to define iterators. Their solution
% typifies the difficulties ownership-based approaches face with
% iterators, which arise from the fact that iterators must have access
% to the private state of a collection but may have differing
% lifetimes. This work builds on Barnett and Naumann's generalization of
% ownership to friendship~\cite{friends}, which allows object invariants
% to have some dependency on non-owned objects.
% 
% The subject-observer pattern has been the focus of a great deal of effort,
% given its prominence in important applications. Simultaneously with our own
% initial formulation, Parkinson gave an example of verifying the
% subject-observer protocol~\cite{parkinson-iwaco-07}. Recently, Parkinson and
% Distefano~\cite{jstar-parkinson-distefano} have implemented a tool to verify
% these programs, and have demonstrated several examples including a verification
% of a subject-observer pattern specified along these lines. The tool includes
% automatic generation of loop invariants. 
% 
% The style of invariant in our work and Parkinson's is very similar,
% and subject to similar limitations. Since each subject needs to know
% what its observers are, verifying programs with chains of
% subject-observers is extremely cumbersome. This is especially
% problematic given that GUI programs --- which are one of the primary
% uses of the subject-observer pattern --- rely upon chains of subjects
% and observers. 
% 
% The work of Barnett and Naumann is also capable of reasoning about the
% subject-observer pattern, but only if all of the possible observers
% are known at verification.  Leino and Schulte~\cite{boogie-sub-obs}
% made use of Liskov and Wing's concept of history invariants or
% monotonic predicates~\cite{liskov-wing} to give a more modular
% solution. The idea in this work is to require the changes that a
% subject makes to ``increase the truth'' of the observer's predicate
% along some partial order. This is less flexible than the approach we
% took, though perhaps a little easier to use when it is
% applicable. Unfortunately, their work is not applicable for
% event-driven programs such as GUIs, since there is no natural partial
% order on user actions.
% 
% More recently, Shaner, Naumann and Leavens~\cite{ShanerLN07} gave a
% ``gray-box'' treatment of the subject-observer pattern.  Instead of
% tracking the specifications of the observers in the predicate, they
% give a model program that should approximates the behavior of any
% actual notification method. This works as long as the runtime
% dependencies are known statically enough to include them in the model
% programs --- again, a limitation which is problematic in the case of
% GUIs.
% 
% Lest the reader get too depressed, in the following two chapters, I
% will give a solution to the problem of verifying invariants across
% dynamic chains of dependencies.
% 
% Pierik, Clarke and de Boer~\cite{creational-invariants} formalize another
% extension to the Boogie framework which they name \emph{creation
%   guards}, specifically to handle flyweights. They consider flyweights
% an instance of a case where object invariants can be invalidated by
% the allocation of new objects, and add guards to their specifications
% to control allocation to the permitted cases. 




\chapter{Proving the Union-Find Disjoint Set Algorithm}

\section{Introduction}

In this chapter, we introduce the technique of ``ramification'', as a
way of recovering local reasoning in the face of imperative programs
with global invariants.

The union-find disjoint set data structure~\cite{union-find} is a
technique for efficiently computing canonical representives for
equivalence classes of values. The basic technique for doing so is to
represent each value in the equivalence class as a node in a tree ---
but unlike the usual implementation of trees, each node does not
contain a pointer to its children, but rather the children each
maintain a pointer to the parent. The root of the tree has no parent
pointer, and represents the canonical representative for an
equivalence class.

The canonical representative can be found (the $\find$ operation) by
following the parent pointers to the root of the tree. Similarly, two
disjoint sets can be merged (the $\union$ operation), by finding their
canonical representatives and setting one to point to the other.

As described, this data structure is no better than using a linked
list. However, two optimizations give rise to an extremely efficient
implementation~\cite{galler-fischer-union-find}. First, the root node
can be modified to keep track of a bound on the maximum height, so
that whenever two sets are merged, the shorter tree can be made a
subtree of the deeper one. Second, the algorithm can make use of
\emph{path compression} -- whenever the $\find$ operation is called,
it can set all of the nodes on the path to the root to point directly
at the root. Together, these optimizations permit performing a
sequence of $n$ $\union$ and $\find$ operations in $O(n \cdot
\alpha(n))$ time, where $\alpha$ is the inverse Ackermann
function~\cite{tarjan-union-find-bound}.

This permitted Huet~\cite{huet-unification} to give a simple
implementation of near-linear-time unification algorithms, and
variants of this idea are used in the proofs of congruence closure
algorithms~\cite{congruence-closure}.

However, path compression is an idiom difficult to accomodate within
the framework of separation logic. In informal reasoning about the
union-find data structure, we do not explicitly track all the elements
of a union-find data structure in our reasoning --- instead, we rely
on the fact that path compression only makes changes to the heap which
our global program invariant should be insensitive to. However,
separation logic is a resource-aware logic, which demands to know the
footprint of any command. So we cannot simply leave the other elements
of the equivalence class out of the invariant, since $\union$ may read
and modify them.

The solution I propose in this chapter is to use a global invariant
structured in a way which preserves modular reasoning, both in the
style of separation logic, and in the interference-insensitive style
of the usual informal proofs. 

However, we do not only want to hide interference! One of the features
which makes union-find so elegant is that the $\union$ operation
features a well-structured use of aliased mutable state.  When merging
two equivalence classes, a single update allows efficiently
\emph{broadcasting} the change to every element of both classes in
almost constant time.  So we need a specification technique that
should also let us specify global interference in a clean,
well-structured way. We achieve this by introducing a
\emph{ramification operator} in our specification, which gives us an
abstract way of characterizing the information propagated globally.

To understand the idea of ramifications, we look back to McCarthy's
original paper introducing the frame problem~\cite{mccarthy}. There,
he described the frame problem as the problem of how to specify what
parts of a state were \emph{unaffected} by the action of a command,
which inspired the name of the frame rule in separation logic. In that
paper, he also described the \emph{qualification problem}. He observed
that many commands (such as a flipping a light switch turning on a
light bulb) have numerous implicit preconditions (such as there being
a bulb in the light socket), and dubbed the problem of identifying
these implicit preconditions the qualification problem.

Some years later, \citet{finger} observed that the qualification
problem has a dual: actions can have indirect effects that are not
explicitly stated in their specification (e.g., turning on the light can
startle the cat). He called the problem of deducing these implicit
consequences the ``ramification problem'' --- is there a simple way to
represent all of the indirect consequences of an action? If so, then 
we have a way of modularly specifying \emph{global} effects. This is
the idea we will adopt to deal with the broadcast nature of \textsf{union}
in the union-find algorithm. 

\section{The Specification}

\begin{figure}
  \begin{mathpar}
    \begin{array}{lcl}
      \formula & = & I \bnfalt \mathsf{Elt\;of\;}\tau \times \tau \bnfalt \mathsf{Tensor\;of\;}\formula \times \formula \\
      \phi, \psi & ::= & I \bnfalt \elt{x}{y} \bnfalt \phi \otimes \psi
    \end{array}
  \end{mathpar}
  \begin{mathpar}
    \begin{array}{lcl}
      R & : & (\tau \to \tau) \times \formula \to \formula \\ 
      R(\rho, I) & = & I \\
      R(\rho, \psi \otimes \phi) & = & R(\rho, \psi) \otimes R(\rho, \phi) \\
      R(\rho, \elt{x}{y}) & = & \elt{x}{\rho(y)} \\
    \end{array}
  \end{mathpar}

  \begin{mathpar}
    \boxed{\ufcontains{\phi}{x}{y}}
    \\
    \inferrule*[]
              { }
              {\ufcontains{\elt{x}{y}}{x}{y}}
    \and
    \inferrule*[]
              {\ufcontains{\phi}{x}{y}}
              {\ufcontains{\phi \otimes \psi}{x}{y}}
    \and
    \inferrule*[]
              {\ufcontains{\psi}{x}{y}}
              {\ufcontains{\phi \otimes \psi}{x}{y}}
  \end{mathpar}
  \begin{specification}
    \nextline
    $\exists \tau : \star$ 
    \nextline 
    $\exists H : \formula \to \assert$ 
    \nextline
    $\exists \newset : \monad{\tau}$ 
    \nextline
    $\exists \find : \tau \to \monad{\tau}$ 
    \nextline
    $\exists \union : \tau \times \tau \to \monad{\unittype}$
    \nextline 
      $\spec{H(\phi)}{\newset}{a:\tau}{H(\phi \otimes \elt{a}{a})}$
    \\ \> $\specand$
    \nextline 
      $\spec{H(\phi) \land (\ufcontains{\phi}{x}{y})}
            {\find(x)}
            {a:\tau}
            {H(\phi) \land a = y}$
    \\ \> $\specand$
    \nextline 
      $\spec{H(\phi) \land (\ufcontains{\phi}{x}{y})
                     \land (\ufcontains{\phi}{u}{v})}
            {\union(x, u)}
            {a:1}
            {H(R([y/v], \phi))}$ 
    \\ \> $\specand$
    \nextline
      $\setof{\forall \phi, a,b,x,y.\;H(\phi \otimes \elt{a}{b} \otimes \elt{x}{y}) \implies a \not= x}$
    \\ \> $\specand$
    \nextline
      $\setof{\forall \phi.\;H(\phi) \implies H(I)}$
    \\ \> $\specand$
    \nextline
      $\setof{\forall \phi, \psi.\;H(\phi \otimes \psi) \iff H(\psi \otimes \phi)}$
    \\ \> $\specand$
    \nextline
      $\setof{\forall \phi.\;H(I \otimes \phi) \iff H(\phi)}$
    \\ \> $\specand$
    \nextline
      $\setof{\forall \phi, \psi, \theta.\;H(\phi \otimes (\psi \otimes \theta)) \iff H((\phi \otimes \psi) \otimes \theta)}$
  \end{specification}

\caption{Specification of Union Find Algorithm}
\label{union-find:spec}
\end{figure}

In Figure~\ref{union-find:spec}, we give our specification of the
union-find algorithm. 

On lines 1-5 of the specification, we specify that there is an
abstract type $\tau$ of nodes of the disjoint set forest, and three
operations $\newset$, $\find$, and $\union$, which create new nodes,
find canonical representatives, and merge equivalence classes,
respectively. Furthermore, there is a monolithic abstract predicate
$H(\phi)$, which describes the entire disjoint-set forest all at once.

This monolithic predicate represents of one of the two tricks of this
specification. Our first trick is to replay the key idea of separation
logic, only ``one level up''. Even though we have a single abstract
predicate describing the whole forest, we can recover separation-style
modularity by indexing the abstract predicate with a formula $\phi$,
which gives a small (in fact, nearly degenerate) ``separation logic''
for describing elements of these equivalence classes.  

The datatype of formulas is given in the display above the
specification. We give it both as an inductive type, to illustrate why
it is definable in higher-order, and as a grammar (which is more
readable, and what we use in the specifications). Formulas have three
grammatical productions. First, we have the forms $I$ and $\phi
\otimes \psi$, which are an (intuitionistic) unit and separating
conjunction for the elements of this little logic. We also have an
atomic formula $\elt{x}{y}$, which says that $x$ is a term whose
canonical representative is $y$. 

These formulas have the usual resource-aware interpretation of
separation logic, so that $\elt{x}{y} \otimes \elt{a}{b}$ implies that
$x \not= a$. % \todo{Add these entailments as axioms}

We also give a simple judgement on formulas $\ufcontains{\phi}{x}{y}$,
which lets us say that $\phi$ entails drawing the conclusion that
$x$'s representative is $y$. This is of course a trivial judgement,
since the language of formulas is so simple.

In addition to this separation logic, our second trick is embodied in
the modal operator $R(\rho, \phi)$, which we call a ``ramification
operator''. Here, $\rho$ is a substitution, and $\phi$ is a
formula. $R$ is defined to operate homomorphically on the structure of
formulas, with its action on atomic formulas being $R(\rho,
\elt{x}{y}) = \elt{x}{\rho(y)}$. Intutively, a ramification $R(\rho,
\phi)$ says to replace the canonical witnesses in the domain of the
substitution with the result of the substitution. This lets us specify
the aliasing effects of global updates in a modular fashion.

So we have the tools to reason both locally and globally in our
specifications.  An example of local reasoning can be seen on line 6
of Figure~\ref{union-find:spec}.  Here, we say that given a state
$H(\phi)$, calling $\newset$ will result in a new state $H(\phi
\otimes \elt{a}{a})$. This functions a bit like a global axiom for
creating new equivalence classes, since we explicitly quantify over
the frame.

By quantifying over $\phi$, we can \emph{implement} the frame rule for
our library. This is similar to the interpretation of the frame rule
in our underlying semantics -- there, we interpret Hoare triples to mean all
the assertions that can be safely framed on, and here, we quantify over
all possible frames. 

On line 7, we see the necessity of this kind of interpretation. Our
specification for $\find$ is very simple -- it says that in any state
which entails $\elt{x}{y}$, calling $\find(x)$ will return $y$. The
$\phi$ is unchanged from precondition to postcondition, and so the
user of this library does not need to know anything about any elements
other than $x$. However, due to path compression, we can modify 
many other nodes in the forest, a fact which our domain-specific
logic conceals. 

On line 8, we give the specification of $\union$. Here, we say that if
$x$'s witness is $y$, and $u$'s witness is $v$, then calling
$\union(x,u)$ will equate the two equivalence classes, setting $u$'s
witness to $y$. Furthermore, since this is globally visible, we need
to push this ramification over the entire set of known nodes
$\phi$. Observe that unlike in the previous function we do not want
local reasoning, since the purpose the $\union$ operation is to
globally broadcast the update. But the use of a ramification operator
does \emph{structure} this update.

On lines 9-13, we add axioms corresponding to our domain-specific
logic.  On line 9, we say that disjoint elements are disjoint, and on
line 10 we say that we can forget the existence of elements (i.e., that
this logic is like intuitionistic separation logic). On lines 11-13,
we simply say that formulas are commutative, unital and associative. 

\begin{figure}
\mbox{}
\begin{specification}
\nextline $\sigma = \Child \of \reftype{\sigma} \bnfalt \Root \of \opttype\;{(\reftype{\sigma})} \times \N$ 
\nextline $\tau = \reftype{\sigma}$ 

\nextline
   $H(\phi) \triangleq \exists D \subseteq \tau, p \in D \rightharpoonup D, w \in (D-\domain{p}) \to D.\;G(D,p,w,\phi)$
\nextline[1em] $G(D, p, w, \phi) =\;$\=$p^{+} \mbox{ strict partial order} \;\land$ 
\nextline \> $w \mbox{ injective} \;\land$ 
\nextline \> $(D, p, w) \models \phi \;\land$
\nextline \> $\heap(D, p, w)$ 

\nextline[1em] $D, p, w \models I \qquad\qquad$\=$\mbox{ iff}\;\;$\=$\mbox{always}$ 
\nextline      $D, p, w \models \phi \otimes \psi$\>$\mbox{ iff}$\>$
                     \exists D_1, D_2.\; D = D_1 \uplus D_2 \mbox{ and } D_1, p, w \models \phi \mbox{ and } 
                     D_2, p, w \models \psi$
\nextline      $D, p, w \models \elt{x}{y}$ \> $\mbox{ iff }$\>$x \in D \land \exists z.\;(x,z) \in p^* \land z\mbox{ maximal} \land w(z) = y$

\nextline[1em] $\heap(D, p, w) = \;\;$\=
    $\forall^{*} l \in \domain{p}.\; l \pointsto \Child(p(l))$ 
\nextline\>  $* \; \forall^{*} l \in (D - \domain{p}).\; \exists n.\;l \pointsto \Root(\Some(w(l)), n)$ 
\end{specification}
\caption{Concrete Definition of Union Find Invariants}
\label{union-find:invariant}
\end{figure}
 
In Figure~\ref{union-find:invariant} we give the invariant for the
union-find data structure. The node type $\tau$ is a pointer to a 
term of type $\sigma$, which is either a $\Child$ value containing
the parent of the current node, or a $\Root(w, n)$ value containing
the witness plus a number to maintain balance.\footnote{We do not track the
ranks of subtrees in our invariant to avoid obscuring the essential
simplicity of the techniques underpinning ramification, though it is
straightforward to add.}

Then on line 3, we say that $H(\phi)$ holds when there are $D$, $p$, and
$w$ such that $G(D, p, w, \phi)$ holds. We require $D$ to be a finite set of
nodes $D$, $p : D \rightharpoonup D$ to be a partial map of nodes to parents, and a 
map $w : (D - \domain{p}) \to D$. The set of nodes $D$ represent all the elements that have
been allocated, and the parent map $p$ maps each node to its
parent. The function $p$ is partial since some nodes are root nodes of
the disjoint-set forest. The map $w : (D - \domain{p}) \to D$ sends
those roots to the appropriate canonical witness.

Then, on lines 4-7, we define exactly how $G$ works. In order to
ensure that the graph structure actually has no cycles, on line 4 we
impose the condition that the transitive closure of $p$ is a strict
partial order. That is, we require that $p^+$ is a transitive,
irreflexive relation. Since the domain $D$ is finite, this also 
insures that $p^+$ has no infinite ascending chains. 

In terms of graph structure, this requirement means that the nodes
form a directed acyclic graph, and the fact that it arises as the
closure of a function means that no node has more than one parent.
Together, these conditions ensure that the nodes form a forest. On
line 5, we assert that $w$ is an injective function, which ensures
that the canonical representatives for different equivalence classes
are all distinct from one another. On line 6, we assert that the
triple $(D, p, w)$ models our formulas.

These formulas are a small subset of separation logic. We have the
formula $I$, which is always satisfied. (So this logic is like an
intuitionistic rather than classical separation logic.) Then we have
the formula $\phi \otimes \psi$, which corresponds to the usual
separating conjunction, in that the resource $n$ (the collection of
disjoint-set nodes) is split into two parts, one of which must support
$\phi$ and the other of which must support $\psi$. Note that the whole
of the parent function $p$ and the canonical witness function $w$
functions are passed to both branches. This is the information that
will let us ensure that global constraints are maintained in local
invariants. Finally, the atomic proposition $\elt{x}{y}$ asserts that
$x$ is a node whose canonical witness is $y$, by saying that $x$ is in
$D$, and that there is some maximal $z$ above $x$ in the reflexive
transitive closure of $p$ (viewed as a partial order), such that $w(z)
= y$. (Note that $z$ is maximal precisely when it is not in the domain
of $p$.)

Then, the predicate $\mathit{heap}(D, p, w)$ asserts that every node
in $D$ is correctly realized by some physical node in the heap. Every
child node must point to its parent, and every root node must point 
to its witness and some count of its tree rank. 

\begin{figure}
\mbox{}
\begin{specification}
\nextline $\newset = [$\=$\letv{r}{[\newref{\sigma}{\Root(\None, 0)}]}{}$
\nextline              \>$\letv{\unit}{[r := \Root(\Some(r), 0)]}{}$
\nextline              \>$r]$

\nextline[1em] $\ctext{findroot}(x) = [$\=$
                  \letv{i}{\comp{!x}}{}$ 
\nextline \>     $\run{}\ctext{case}($\=$i,$ 
\nextline \> \>     $\Root(w, n) \to [(x, w, n)],$
\nextline \> \>     $\Child(p) \to [$\=$\letv{(r, w, n)}{\findroot(p)}{}$ 
\nextline \> \>                      \>$\letv{\_}{[x := \Child(r)]}{}$ 
\nextline \> \>                      \>$(r,w,n)]$

\nextline[1em] $\find(x) = [\letv{(\_, w, \_)}{\ctext{findroot}(x)}{w}]$ 

\nextline[1em] $\union(x, y) = [$\= 
             $\letv{(r, u, m)}{\findroot\;x}{}$ 
\nextline \> $\letv{(s, v, n)}{\findroot\;y}{}$ 
\nextline \> $\run{}$\=$\ctext{if}\;r \not= s\;\ctext{then}$ 
\nextline \>\>\;\;\=   $\ctext{if}\;m < n\;\ctext{then}$ 
\nextline \>\> \>   \;\;$[$\=$\letv{\unit}{[s := \Root(u, n)]}{}$ 
\nextline \>\> \>         \>$r := \Child(s)]$ 
\nextline \>\> \> $\ctext{else}\;\ctext{if}\;n < m\;\ctext{then}$ 
\nextline \>\> \>   \;\;$[s := \Child(r)]$
\nextline \>\> \> $\ctext{else}$ 
\nextline \>\> \>   \;\;$[\letv{\unit}{[r := \Root(u, n+1)]}{}$
\nextline \>\> \>      \>$s := \Child(r)]$
\nextline \>\> $\ctext{else}$ 
\nextline \>\> \> $[\unit]]$
\end{specification}
\caption{Implementation of Union-Find Algorithm}
\label{union-find:impl}  
\end{figure}

In Figure~\ref{union-find:impl}, we give the implementation of these functions. 
On lines 1-3, we give the code for creating a new element. This works by allocating
a pointer of type $\tau$, and then updating it so that it points to itself as its
canonical witness. 

The $\find$ function is defined on line 10, but it works by deferring
almost all of its work to an auxiliary function $\findroot$. This
function is defined on lines 4-9, and it works by recursively
following the parent pointers of each node. When it reaches a root, it
returns a triple $(r, w, n)$, containing all three of the physical
root $r$, the witness value $w$, and the tree rank $n$. On a
recursive call (i.e., when the argument is a child node), we take the
return triple, and before returning, we implement path compression, by
updating the child's parent to be the root node $r$.  The $\find$
function simply calls $\findroot$, and ignores its return values
except for the witness.

The reason we have this auxilliary function becomes evident in the
$\union$ function, on lines 11-23. Given arguments $(x,y)$, what
$\union$ does is to first call $\findroot$ on both $x$ and $y$ (on
lines 11 and 12). Then, on line 13, we compare the two physical roots
for inequality. If they are equal, then there is no work to be done
(lines 22-23). Otherwise, we compare the two returned tree ranks, and
add the smaller root as a child of the larger one (lines 14-19), and
increment the size counter if they are the same rank (lines 20-21).

\section{Correctness Proofs}

All of these proofs have a similar structure. First, we prove a lemma about how 
the properties of the parent order $p$ can change. Second, we prove how this changes
(or doesn't change) the satisfaction of a formula $\phi$. Third, we use these
two properties to verify the program itself, in an annotated specification style.




\begin{lemma}{(Soundness of Entailment)}
  If $D, p, w \models \phi$ and $\ufcontains{\phi}{u}{v}$, then $D, p, w \models \elt{u}{v}$. 
\end{lemma}

\begin{proof}
  This proof follows from an easy induction on the derivation of $\ufcontains{\phi}{u}{v}$. 
\end{proof}

\begin{lemma}{(Disjointness of Elements)}
  The formula $\forall \phi,a,b,x,y.\;H(\phi \otimes \elt{a}{b} \otimes \elt{x}{y}) \implies a \not= x$
  is valid.   
\end{lemma}
\begin{proof}
  \begin{tabbedproof}
    \oo Assume $\phi,a,b,x,y$ and $H(\phi \otimes \elt{a}{b} \otimes \elt{x}{y})$ \\
    \ooo So we know that there are $D, p,$ and $w$ such that $D, p, w \models \phi \otimes \elt{a}{b} \otimes \elt{x}{y}$ \\
    \ooo By the definition of $\otimes$, we know that there are disjoint $D_1, D_2$ and $D_3$, such that\\
    \ooox $D_1, p, w \models \phi$ \\
    \ooox $D_2, p, w \models \elt{a}{b}$ \\
    \ooox $D_3, p, w \models \elt{x}{y}$ \\
    \ooo So we know $a \in D_2$ and $x \in D_3$ \\
    \ooo Since $D_2$ and $D_3$ are disjoint, $a \not= x$ 
  \end{tabbedproof}
\end{proof}

\begin{lemma}{(Structural Properties)}
The following properties are valid:
\begin{itemize}
\item $\forall \phi.\;H(\phi) \implies H(I)$
\item $\forall \phi,\psi.\;H(\phi \otimes \psi) \iff H(\psi \otimes \phi)$
\item $\forall \phi.\;H(I \otimes \phi) \iff H(\phi)$
\item $\forall \phi,\psi,\theta.\;H(\phi \otimes (\psi \otimes \theta)) \iff H((\phi \otimes \psi) \otimes \theta)$
\end{itemize}
\end{lemma}
\begin{proof}
These properties follow immediately from the semantics of assertions. 
\end{proof}

\subsection{Proof of $\ctext{find}$}

Suppose $R \subseteq D \times D$ is a strict partial order. We say $x
\in D$ is \emph{maximal} when there is no $y$ such that $(x,y) \in R$.
Note that when we have a partial function $f : D \rightharpoonup D$
such that $f^+$ is a strict partial order, $x \in D$ is maximal
precisely when $f(x)$ is not defined. (If $f(x)$ were defined, then
$(x, f(x)) \in f^+$. Hence it cannot be defined for any maximal
element.)

\begin{lemma}{(Path Compression Lemma)}
  Suppose $D$ is a finite set and $f : D \rightharpoonup D$ is a
  partial function on $D$, such that $f^{+}$ is a strict partial
  order. Now suppose $(x,y) \in f^{+}$ with $y$ maximal.

Then, $[f|x:y]$ has the same domain as $f$, and $[f|x:y]^+$ is a
strict partial order such that for all $(u,v) \in D$, $(u,v) \in
[f|x:y]^{+}$ with $v$ maximal if and only if $(u,v) \in f^{+}$ with
$v$ maximal.
\end{lemma}

\begin{proof}
\begin{enumerate}
\item 
Since $(x,y) \in f^+$, we know that $x \in \domain{f}$. Hence $[f|x:y]$
has the same domain as $f$. 

\item 
For $[f|x:y]^+$ to be a strict partial order, it must be an
irreflexive transitive relation.  Since it is the transitive closure
of a function, it is immediately transitive. 

To show that it is irreflexive, assume that $(u, u) \in
[f|x:y]^+$. Therefore there is some sequence $(u, [f|x:y](u), \ldots,
[f|x:y]^{n+1}(u) = u)$.

Now consider whether $x$ is in the sequence. If it is not, then this
sequence is equal to $(u, f(u), \ldots, f^n(u) = u)$, which is a
contradiction, since $f^+$ is irreflexive. If $x$ is in the sequence,
then we know that either $x$ is the last element, or the
second-to-last element (since $f(x) = y$, and $y$ is maximal).
Suppose $x$ is the last element. Then we know that $u = x$, and hence
$[f|x:y](u) = y$, and so we have a contradiction. Suppose $y$ is the
last element. Then the first element is $y$, and there is no $f(y)$,
since $y$ is a maximal element, and this is also a
contradiction. 

Therefore there is no $(u, u) \in [f|x:y]^+$, and so $[f|x:y]^+$ is
irreflexive.

\item
\begin{itemize}
\item[$\Leftarrow$] Suppose that $(u, v)$ in $f^{+}$ with $v$
  maximal. Then there is some $k$ such that $f^k(u) = v$. So we have a
  sequence $u, f(u), \ldots, f^k(u)$.

If any of the $f^i(u) = x$ for $i < k$, then we know that $v = y$,
since both $v$ and $y$ are maximal. Therefore, it follows that the
sequence $u, [f|x:y](u), \ldots, [f|x:y]^i(u), y$ shows that $(u,v)
\in [f|x:y]^+$ with $v$ maximal.

If none of the $f^i(u) = x$, then it follows that the sequence $u,
f(u), \ldots, f^k(u)$ is exactly the same as $u, [f|x:y](u), \ldots,
[f|x:y]^k(u)$, and so $(u, v) \in [f|x:y]^+$ with $v$ maximal.

\item[$\Rightarrow$] Suppose $(u,v) \in [f|x:y]^{+}$ with $v$
  maximal. Then we know that there is some sequence $u, [f|x:y](u),
  \ldots, [f|x:y]^{k+1}(u)$ with $[f|x:y]^{k+1}(u) = v$.

  Now, either $x$ occurs in this sequence, or not. Suppose that 
  $x$ does not occur in the sequence -- that is, for every $i \leq k+1$,
  $[f|x:y]^i(u) \not= y$. Then it follows that this sequence is 
  the same as $(u, \ldots, f^{k+1}(u) = v)$. Since $[f|x:y]$ and $f$
  have the same domain, it follows that $v$ is maximal in $f$, 
  as well. Hence $(u, v) \in f^+$ with $v$ maximal. 

  Now, suppose that $x$ occurs in the sequence at position $i$. It
  cannot occur at $i = k+1$, since $f^{k+1}(u) = v$ is the last
  element, and we know $v$ is maximal. But since $[f|x:y](x) = y$, we
  know that $x$ is not maximal in $[f|x:y]$. Therefore it occurs at $i
  \leq k$.

  However, it also cannot occur at any $i < k$, since $y$ is maximal
  in $[f|x:y]$, and if $x$ occured at $i < k$, then the sequence would
  end at $i+1$ with $y$, and we know the sequence is $k+1$ elements
  long. Therefore, $x$ occurs at position $i = k$, and $(u, \ldots,
  [f|x:y]^k(u), [f|x:y]^{k+1}(u) = v) = (u, \ldots, f^k(u) = x, [f|x:y]^{k+1}(u) = y = v)$.

  Now, we know that $(x,y) \in f^+$ with $y$ maximal in $f$. Therefore,
  there is a sequence $(x, \ldots, f^{n+1}(x) = y)$. Therefore, 
  the sequence $(u, \ldots, f^k(u) = x, \ldots, f^{n+k+1}(u) = y = v)$ is 
  in $f^+$, with $v$ maximal in $f$. 

\end{itemize}
\end{enumerate}
\end{proof}

\begin{lemma}{(Satisfaction Depends on Maximality)}
Suppose $D$ is a finite set and $f,g : D \rightharpoonup D$ are
partial functions on $D$, such that $f^+$ and $g^+$ are strict partial
orders such that for all $x,y \in D$, we have that $(x,y) \in f^+$
with $y$ maximal in $f$ if and only if $(x,y) \in g^+$ with $y$
maximal in $g$.

Then for all $\phi, w$ and $D' \subseteq D$, if $D', f, w \models \phi$,
then $D', g, w \models \phi$. 
\end{lemma}

% \begin{lemma}{(Path Compression Preserves Satisfaction)}
% Suppose $D$ is a finite set and $f : D \rightharpoonup D$ is a partial function on $D$,
% such that $f^{+}$ is a strict partial order, and that $(x,y) \in f^{+}$ with $y$ maximal.
% 
% For all $\phi$, $w$ and $D' \subseteq D$, if $D', f, w \models \phi$, then $D', [f|x:y], w \models \phi$.  
% \end{lemma}


\begin{proof}
This lemma follows by induction on $\phi$. 
\begin{itemize}
\item Case $\phi = I$. This follows immediately from the definition of satisfaction. 
\item Case $\phi = \psi \otimes \theta$. 
  \begin{tabbedproof}
    \oo By assumption, we have $D', f, w \models \psi \otimes \theta$. \\
    \ooo From the definition of satisfaction, we get  $D_1, D_2$ such that $D' = D_1 \uplus D_2$ \\
    \oox and $D_1, f, w \models \psi$ and $D_2, f, w \models \theta$.  \\
    \ooo By induction, we get $D_1, g, w \models \psi$ \\
    \ooo By induction, we get $D_2, g, w \models \theta$ \\
    \ooo By definition, we get $D_1 \cup D_2, g, w \models \psi \otimes \theta$ \\
  \end{tabbedproof}
\item Case $\phi = \elt{x}{y}$.
  \begin{tabbedproof}
    \oo By assumption, $D', f, w \models \elt{x}{y}$ \\
    \oo So we know that $x \in D'$ and there exists an $z$ such that \\
    \ox $(x,z) \in f^*$ with $z$ maximal and $y = w(z)$ \\
    \oo By definition of reflexive transitive closure, either $(x, z) \in f^+$ or $x = z$ \\
    \oo Suppose $(x,z) \in f^+$:\\
    \ooo Then by hypothesis, $(x,z) \in g^+$ with $m$ maximal.  \\
    \ooo Hence $D', g, w \models \elt{x}{z}$  \\
    \oo Suppose $x = z$: \\
    \ooo Since $x$ is maximal, we know $x \not\in \domain{f}$. \\
    \ooo Suppose $x \in \domain{g}$. \\
    \oooo Then there is a sequence $(x, g(x), \ldots, u) \in g^+$ with $u$ maximal \\
    \oooo Then $(x, f(x), \ldots, u) \in f^+$ with $u$ maximal \\
    \oooo This is a contradiction, since $x$ is maximal in $f$ \\
    \ooo So $x \not\in \domain{g}$ \\
    \ooo Hence $(x, x) \in g^*$ and $x$ maximal in $g$ \\
    \ooo Hence $(x, z) \in g^*$ and $z$ maximal in $g$ \\
    \ooo Hence $D', g, w \models \elt{x}{y}$
  \end{tabbedproof}
\end{itemize}
\end{proof}


Now, we can specify and prove the correctness of $\findroot$. 

\begin{lemma}{(Correctness of $\findroot$)}
The $\findroot$ function satisfies the following specification:
\begin{specification}
\nextline $\{G(D, f, w, \phi) \land (D,f,w) \models \elt{u}{v}\}$
\nextline $\findroot(u)$
\nextline $\{(r,c,n):\tau \times \tau \times \N.$ 
\nextline\;\;\= $\exists f'.\; G(D, f', w, \phi) \land c = v \land (u,r) \in f'^{*} \land r \mbox{ maximal} \land w(r) = c \land \domain{f} = \domain{f'}$
\nextline\> $\land\; (f,f')\mbox{ have same maximal relationships}\}$
\end{specification}
\end{lemma}

Here, $f$ and $f'$ ``having the same maximal relationships'' means that $(u,v) \in f^+$ with $v$ maximal
if and only if $(u, v) \in (f')^+$ with $v$ maximal. 

\begin{proof}
\begin{tabbedproof}
\oo Assume our precondition state is $G(D, f, w, \phi) \land (D,f,w) \models \elt{u}{v}$ \\
\ooo Then we have a sequence $(u, \ldots, f^k(u) = x)$ with $x$ maximal and $w(x) = v$. \\
\ooo Then we know that $f^+$ is a strict partial order, and $D,f,w \models \phi$ and \\
\oox $\mathit{heap}(D,f,w)$ and $D,f,w \models \elt{u}{v}$ \\
\ooo Now, do a case split on whether or not $u \in \domain{f}$ (i.e., is maximal). \\
\ooo If $u \in \domain{f}$: \\
\oooo Then we have a sequence $(u, f(u), \ldots, f^k(u) = x)$ with $x$ maximal and $w(x) = v$ \\
\oooo Hence $(D, f, w) \models \elt{f(u)}{v}$ \\ 
\oooo Next, $\mathit{heap}(D,f,w) \implies (u \hookrightarrow \Child(f(u))$ \\
\oooo $[\letv{i}{[!u]}{}$ \\
\oooo So we add $i = \Child(f(u))$ to the state, and simplify the case statement \\
\oooo $\letv{(r,c,n)}{\findroot(f(u))}{}$ \\
\oooo Then $G(D, f', w, \phi) \land (f(u), r) \in f'^* \land r\mbox{ maximal} \land c = w(r) \land c = v$ for some $f'$ \\
\oooo with the same domain and maximal relationships as $f$\\
\oooo So we know that $r = x$, and that $(u, r) \in f'^*$ \\ 
\oooo So we know $D, f', w \models \phi$ and $(f')^+$ is a strict partial order and $w$ injective and $\mathit{heap}(D, f', w)$, \\
\oooo Since we know $r$ remains maximal in $f'$, we know that $(D, f', w) \models \elt{u}{v}$ \\ 
\oooo Therefore, due to path compression, $[f'|u:r]^+$ is a strict partial order \\
\oooox with the same maximal relationships and domain as $f'$\\
\oooo Hence $D, [f'|u:r], w \models \phi$ and $\domain{[f|u:r]} = \domain{f}$ and $(u, r) \in [f'|u:r]^*$ \\
\oooo Now we will perform updates to make the physical heap match this logical heap \\ 
\oooo $\letv{\_}{[u := \Child(r)]}{}$ \\
\oooo Hence $\mathit{heap}(D, [f'|u:r], w)$, together with the pure formulas above \\
\oooo $(r,c,n)]$ \\
\oooo Hence $\exists f''.\;G(D, f'', \phi) \land c = v \land w(r) = c \;\land$ \\
\oooox $\domain{f} = \domain{f''} \land (u,r) \in f''^{*} \land r\mbox{ maximal} \;\land$ \\
\oooox $(f,f'')$ have the same maximal relationships. \\
\ooox (with the choice of $[f'|x:y]$ as the witness for $f''$) \\
\ooo If $u \not\in \domain{f}$:  \\
\oooo Then $\mathit{heap}(D,f,w) \implies (u \hookrightarrow \Root(\Some(p), n))$ for some $n$ and $p = w(u)$\\
\oooo $[\letv{i}{[!u]}{}$ \\
\oooo So we add $i = \Root(\Some(w(u)), n)$ to the state, and simplify the case statement \\
\oooo $(u, p, n)]$\\
\oooo Note $(u, u) \in f^*$ and $u$ maximal, and that $p = w(u)$, and that $\domain{f} = \domain{f}$ \\
\oooo The fact that $D, f, w \models \elt{u}{v}$ implies $v = w(u)$\\ 
\oooo And obviously $f$ has the same maximal relationships as $f$ \\
\oooo Hence $\exists f''.\;G(D, f'', \phi) \land c = v \land w(r) = c \;\land$ \\
\oooox $\domain{f} = \domain{f'} \land (u,r) \in f''^{*} \land r\mbox{ maximal} \;\land$ \\
\oooox $(f,f')$ have the same maximal relationships. \\
\oooo (with the choice of $f$ as witness for $f''$) \\
\end{tabbedproof}
\end{proof}

\begin{lemma}{(The $\find$ Function is Correct)}
  The $\find$ function meets the specification in Figure~\ref{union-find:spec}. 
\end{lemma}

\begin{proof}
This proof is easy, since $\findroot$ does almost all the work.
\begin{tabbedproof}
\oo Assume a precondition of $H(\phi)$ and $\ufcontains{\phi}{u}{v}$ \\
\ooo This means we have $G(D, f, w, \phi)$ for some $D$, $f$, and $ w$. \\
\ooo Furthermore, we also know that $D, f, w \models \elt{u}{v}$. \\
\ooo Now, expand the call $\find(u)$: \\
\ooo $[\letv{(\_, v', \_)}{\findroot(u)}{}$ \\
\ooo Now we know $G(D, f', w, \phi) \land \domain{f} = \domain{f'} \land v' = v$ \\
\ooo $v']$ \\
\ooo Now we know $G(D, f', w, \phi) \land \domain{f} = \domain{f'} \land a = v$ 
\end{tabbedproof}
\end{proof}

\subsection{Proof of $\newset$}

\begin{lemma}{(Satisfaction and Allocation)}
Suppose $D, f, w \models \phi$ and $x \not \in D$. 

Then $D \cup \setof{x}, f, [w|x:x] \models \phi \otimes \elt{x}{x}$
\end{lemma}

\begin{proof}
\begin{tabbedproof}
\oo To prove this, we want to exhibit $D_1, D_2$ such that $D \cup \setof{x} = D_1 \uplus D_2$, \\
\ox and $D_1, f, [w|x:x] \models \phi$ and $D_2, f, [w|x:x] \models \elt{x}{x}$. \\ 
\oo Take $D_1 = D$ and $D_2 = \setof{x}$, which are disjoint since $x \not\in D$. \\ 
\oo Note that $\setof{x}, f, [w|x:x] \models \elt{x}{x}$, since:  \\
\oox $x \in \setof{x}$  \\
\oox $(x, x) \in f^*$ (since this is a reflexive relation) \\
\oox $x$ is maximal (since it is not in the domain of $f$) \\
\oox $[w|x:x](x) = x$. \\
\oo Now it remains to be shown that $D, f, [w|x:x] \models \phi$ \\
\oo We show this by an induction on $\phi$:  \\
\ooo Case $\phi = I$:\\ 
\oooo This case is immediate, since $D, f, [w|x:x] \models I$ by definition \\ 
\ooo Case $\phi = \psi \otimes \theta$: \\    
\oooo Since $D, f, w, \models \psi \otimes \theta$, there are $D_1, D_2$ so $D_1, f, w \models \psi$ and $D_2, f, w \models \theta$ \\
\ooox and $D = D_1 \uplus D_2$ \\
\oooo By induction, we know $D_1, f, [w|x:x] \models \psi$ \\ 
\oooo By induction, we know $D_2, f, [w|x:x] \models \theta$ \\
\oooo Hence $D, f, [w|x:x] \models \phi$ \\
\ooo Case $\phi = \elt{u}{v}$ \\
\oooo So we know $u \in D$ and that $(u,z) \in f^{*}$ and $z$ maximal and $w(z) = v$ for some $z$ \\
\oooo Since $x \not \in D$, $x \not\in D - \domain{f}$, and so it follows that $[w|x:x](z) = v$ \\
\oooo Hence $D, f, [w|x:x] \models \elt{u}{v}$ 
\end{tabbedproof}
\end{proof}

\begin{lemma}
The $\newset$ procedure meets the specification in Figure~\ref{union-find:spec}. 
\end{lemma}

\begin{proof}
\begin{tabbedproof}
\oo Assume we have a precondition state $H(\phi)$, and consider the body of $\newset$. \\
\ooo Then we know that $f^+$ is a strict partial order, and $D,f,w \models \phi$ and \\
\oox $\mathit{heap}(D,f,w)$ and $D,f,w \models \elt{u}{v}$ \\
\ooo $[\letv{r}{[\newref{\sigma}{\Root(\None, 0)}]}{}$ \\
\ooo Now the state is $\mathit{heap}(D,f,w) * r \pointsto \Root(\None, 0)$, plus the pure predicates. \\
\ooo $\letv{\_}{[r := \Root(\Some(r), 0)]}{}$ \\
\ooo Now the state is $\mathit{heap}(D,f,w) * r \pointsto \Root(\Some(r), 0)$, plus the pure predicates. \\
\ooo Since $\mathit{heap}(D,f,w)$ has a pointer for each $l \in D$, it follows that $r \not \in D$. \\
\ooo Thus, we know that $D' = D \uplus \setof{r}, f, [w|r:r] \models \phi \otimes \elt{r}{r}$ \\ 
\ooo $f^+$ is still a strict partial order which is a subset of $D' \times D'$ \\
\ooo $f$ has the domain $D' \rightharpoonup D'$ \\
\ooo $[w|x:x] \in (D' - \domain{f}) \to D'$ \\
\ooo It is clear that $\mathit{heap}(D,f,w) * r \pointsto \Root(\Some(r), 0)$ is \\
\oox equivalent to $\mathit{heap}(D', f, [w|x:x])$ \\
\ooo Hence $G(D', f, [w|x:x], \phi \otimes \elt{r}{r})$ \\
\ooo $r]$ \\
\ooo Hence $G(D', f, [w|x:x], \phi \otimes \elt{a}{a})$ \\
\ooo Hence $H(\phi \otimes \elt{a}{a})$ 
\end{tabbedproof}
\end{proof}

\subsection{Proof of $\mathsf{union}$}

\begin{lemma}
If $f^+$ is a strict partial order, $(u, v) \in f^+$ and $v$ maximal,
and $(x, y) \in f^+$ and $y$ maximal, and $v \not= y$, then it follows that for $g = [f|v:y]$,

\begin{enumerate}
\item $\domain{g} = \domain{f} \uplus \setof{v}$
\item $g^+$ is a strict partial order 
\item If $(a,b) \in f^*$ with $b$ maximal, either $b \not= v$ and $(a,b) \in g^*$ with $b$ maximal,
  or $b = v$ and $(a,y) \in g^*$ with $y$ maximal. 
\end{enumerate}
\end{lemma}

\begin{proof}
\begin{enumerate}
\item Since $v$ is maximal in $f$, it follows that $v \not\in \domain{f}$, and hence $\domain{[f|v:y]} = \domain{f} \uplus \setof{v}$

\item To be a strict partial order, it suffices that there is no $(x,x) \in g^+$. 
\begin{tabbedproof}
\oo Assume $(x, x) \in g^+$ \\
\oo Then there is a $k \geq 1$ such that $g^k(x) = x$ \\
\oo Now, we'll show that $v$ does not occur in the sequence $x, g(x), \ldots, g^k(x)$ \\
\ooo Suppose there is an $i \leq k$ such that $g^i(x) = v$ \\
\ooo This is a contradiction, because $v$ is maximal, and we know that $g^{i+1}(x)$ must be defined \\
\oo Therefore for $i \leq k$, we have $g^i(x) \not= v$.  \\
\oo Hence we have $g^i(x) = f^i(x)$ \\
\oo Therefore $x, f(x), \ldots, f^k(x) = x$ shows that $(x, x) \in f^+$ \\
\oo But $f^+$ is a strict partial order, which is a contradiction. \\
\end{tabbedproof}

\item If $(a,b) \in f^*$ with $b$ maximal, either $b \not= v$ and $(a,b) \in g^*$ with $b$ maximal,
  or $b = v$ and $(a,y) \in g^*$. 
\begin{tabbedproof}
\oo Assume $(a,b) \in f^*$ with $b$ maximal. \\
\ooo Then either $a = b$, or $(a,b) \in f^+$ \\
\ooo Suppose $a = b$: \\
\oooo Then either $b = v$ or not \\
\oooo Suppose $b = v$: \\
\ooooo Then $(a,y) = (v,y) \in g \subseteq g^*$, and  \\
\ooooo $y$ is maximal since $y \not\in \domain{g} = \domain{f} \cup \setof{v}$ \\
\oooo Suppose $b \not= v$: \\
\ooooo Then $(a,y) = (b, b) \in g^*$, since $b$ is maximal since $b \not\in \domain{g} = \domain{f} \cup \setof{v}$ \\ 
\ooo Suppose $(a,b) \in f^+$: \\
\oooo Then we have a $k > 0$ such that $f^k(a) = b$ \\
\oooo Now, either there is an $i \leq k$ such that $f^i(a) = v$, or not. \\
\oooo Suppose $f^i(a) = v$: \\
\ooooo Then we know that for all $j < i$, $f^j(a) \not= v$, since $v$ is maximal in $f$ \\
\ooooo Therefore for all $j \leq i$, $f^j(a) = g^j(a)$ \\
\ooooo Therefore $g^{i+1}(a) = y$ \\
\ooooo Hence $(a,y) \in g^*$ and $y$ is maximal, since $y \not\in \domain{g}$ \\
\oooo Suppose there is no $i$ such that $f^i(a) = v$: \\
\ooooo Then we know that for all $j \leq k$, $f^j(a) = g^j(a)$, and $b \not= v$ \\
\ooooo Hence $(a,b) \in g^*$ and $b$ is maximal since $b \not\in \domain{f} \cup \setof{v} = \domain{g}$\\
\end{tabbedproof}
\end{enumerate}
\end{proof}

\begin{lemma}{(Ramification)}
Suppose $D, f, w \models \phi$ and $(x,y) \in f^*$ with $y$ maximal and $(u, v) \in f^*$ with $v$ maximal,
and $y \not= v$, and $w$ injective. Let $w'$ be the restriction of $[w|v:z]$ to exclude $y$, and let $g = [f|y:v]$. 

Then, $D, g,  w' \models R([z/w(v), z/w(y)], \phi)$.
\end{lemma}

\begin{proof}
This proof follows by induction on the structure of $\phi$. 

\begin{tabbedproof}
\oo Case $\phi = I$: \\
\ooo This case is immediate since $D, g, w' \models I$  and $R([z/w(v), z/w(y)],I) = I$ \\
\oo Case $\phi = \psi \otimes \theta$: \\
\ooo By satisfaction of $\phi$, we have $D_1, D_2$ such that $D = D_1 \uplus D_2$ and \\
\ooox $D_1, f, w \models \psi$ and\\
\ooox $D_2, f, w \models \theta$ \\
\ooo By induction, we have $D_1, g, w' \models R([z/w(v), z/w(y)], \psi)$ \\ 
\ooo By induction, we have $D_2, g, w' \models R([z/w(v), z/w(y)], \theta)$ \\ 
\ooo By the definition of satisfaction, $D, g, w' \models  R([z/w(v), z/w(y)], \psi)\otimes R([w(v)/w(y)], \theta)$ \\ 
\ooo By the definition of $R$, we have $D, g, w' \models  R([z/w(v), z/w(y)], \phi)$  \\
\oo Case $\phi = \elt{a}{b}$: \\
\ooo We know that $a \in D$, $(a,c) \in f^*$ with $c$ maximal, and $w(c) = b$ \\
\ooo Therefore either $c \not= y$ and $(a,c) \in g^*$ with $c$ maximal or $c = y$ and $(a, v) \in g^*$ \\
\ooo Suppose $c \not= y$: \\
\oooo Since $w$ is injective, we know that $w(c) \not= w(y)$. Hence $[z/w(y)]w(c) = w(c)$ \\
\oooo Consider whether $c$ is $v$.  \\
\oooo If $c \not= v$:  \\
\ooooo Since $w$ is injective, we know that $w(c) \not= w(v)$. Hence $[z/w(v)]w(c) = w(c)$ \\
\ooooo Hence $[z/w(v), z/w(y)]w(c) = w(c)$ \\
\ooooo So $\elt{a}{b} = R([z/w(v),z/w(y)], \elt{a}{b})$ \\
\ooooo Furthermore $w'(c) = w(c)$ \\
\ooooo Hence $D, g, w' \models \elt{a}{b}$ \\
\ooooo Hence $D, g, w' \models R([z/w(v), z/w(y)], \elt{a}{b})$ \\
\oooo If $c = v$: \\
\ooooo Since $w(c) = w(v)$, we have $[z/w(v)]w(c) = z$ \\
\ooooo Hence $[z/w(v), z/w(y)]w(c) = z$ \\
\ooooo Note $w'(v) = z$ \\
\ooooo Hence $D, g, w' \models \elt{a}{z}$ \\
\ooooo Hence $D, g, w' \models R([z/w(v), z/w(y)], \elt{a}{b})$ \\
\ooo Suppose $c = y$: \\
\oooo Then, $b = w(c) = w(y)$, so that $[z/w(v), z/w(y)]b = z$\\ 
\oooo So $R([z/w(v), z/w(y)], \elt{a}{b}) = \elt{a}{z}$ \\
\oooo Furthermore, we know that $(a, v) \in g^*$ and and $v$ maximal and $w'(v) = z$ \\
\oooo So $D, g, w' \models \elt{a}{z}$ \\
\oooo So $D, g, w' \models R([z/w(v),z/w(y)], \elt{a}{b})$ 
\end{tabbedproof}
\end{proof}

\begin{lemma}{(Correctness of $\union$)}
The $\union$ function meets the specification in Figure~\ref{union-find:spec}. 
\end{lemma}

\begin{proof}
\begin{tabbedproof}
\oo Assume we have a precondition $H(\phi)$ and $\ufcontains{\phi}{x}{y}$ and $\ufcontains{\phi}{u}{v}$\\
\ooo Then we have $D$, $f$, $w$, such that $G(D, f, w, \phi)$ \\
\ooo Furthermore $D, f, w \models \elt{x}{y}$ \\
\ooo Now consider the body of $\union(x,u)$ \\
\ooo $[\letv{(r,y',m)}{\findroot(x)}{}$ \\
\ooo So we have $f'$ such that $G(D, f', w, \phi)$ and \\
\ooox  $(x, r) \in f'^*$ and $r$ maximal \\
\ooox  $f$ and $f'$ have the same maximal relationships \\
\ooox  $y = w(r)$ \\
\ooox  $y = y'$ \\
\ooo Furthermore $D, f', w \models \elt{u}{v}$, so \\
\ooo $\letv{(s,v',n)}{\findroot(u)}{}$ \\
\ooo So we have $f''$ such that $G(D, f'', w, \phi)$ and \\
\ooox  $(u, s) \in f''^*$ and $s$ maximal \\
\ooox  $f'$ and $f''$ have the same maximal relationships \\
\ooox  $v = w(s)$ \\
\ooox  $v = v'$ \\
\ooo So we can substitute $y$ for $y'$ and $v$ for $v'$ \\
\ooo Since $f$ and $f'$ have the same maximal relationships, and \\
\oox since $f'$ and $f''$ have the same maximal relationships,  \\
\oox we know that $f$ and $f''$ have the same maximal relationships \\
\ooo Since $(x,r) \in f^*$ and $r$ maximal in $f$, we know $(x, r) \in f''^*$ and $r$ maximal in $f''$ \\
\ooo Now case analyze on whether $r = s$: \\
\ooo If $r = s$: \\
\oooo Then we can simplify the remaining program to $\unit$ \\
\oooo Now, note that since $r = s$, $w(r) = w(s)$, and so $y = v$ and $R([y/v], \phi) = \phi$ \\
\oooo Hence we can hide $D, f'', w$ to get $H(\phi)$ \\
\ooo If $r \not= s$: \\
\oooo Now, case analyze on the ranks $m$ and $n$: \\
\oooo If $m < n$: \\
\ooooo We can simplify the if-then-else, and continue \\
\ooooo $\letv{\unit}{[s := \Root(y, n)]}{}$ \\
\ooooo $r := \Child(s)$ \\
\ooooo Now take $w'$ to be the restriction of $[w|s:y]$ to exclude $r$ \\
\ooooo Now take $g = [f''|r:s]$ \\
\ooooo Since $w(r) = y$, and $w$ is injective, $w'$ is still injective \\
\ooooo From the ramification lemma, we know $D, g, w' \models R([y/y, y/v], \phi)$ \\
\ooooo This is the same as $R([y/v], \phi)$ \\
\ooooo The two updates ensure that $\mathit{heap}(D, g, w')$ hold \\
\ooooo Hiding $D, g, w'$, we get $H(R([y/v], \phi)$ \\
\oooo If $n < m$: \\
\ooooo We can simplify the if-then-else, and continue \\
\ooooo $s := \Child(r)$ \\
\ooooo Now take $w'$ to be the restriction of $[w|r:y]$ to exclude $s$ \\
\ooooo Now take $g = [f''|s:r]$ \\
\ooooo By the ramification lemma, $D, g, w' \models R([y/y, y/v], \phi)$ \\
\ooooo This is the same as $R([y/v], \phi)$ \\
\ooooo The update ensures that $\mathit{heap}(D, g, w')$ holds \\
\ooooo Hiding $D, g, w'$, we get $H(R([y/v], \phi)$ \\
\oooo If $m = n$: \\
\ooooo $\letv{\unit}{[r := \Root(y, m+1)]}{}$ \\
\ooooo $s := \Child(r)$ \\
\ooooo Now take $w'$ to be the restriction of $[w|r:y]$ to exclude $s$ \\
\ooooo Now take $g = [f''|s:r]$ \\
\ooooo By the ramification lemma, $D, g, w' \models R([y/y, y/v], \phi)$ \\
\ooooo This is the same as $R([y/v], \phi)$ \\
\ooooo The update ensures that $\mathit{heap}(D, g, w')$ holds \\
\ooooo Hiding $D, g, w'$, we get $H(R([y/v], \phi)$ \\

\end{tabbedproof}
\end{proof}

\paragraph{Acknowledgements}

I would like to thank Peter O'Hearn for pointing out the connection
of our work with the ramification problem of AI.


% \begin{lemma}{(Existence of Upper Fixed Points)}
% Suppose $A$ is a finite set, and $f : A \to A$ is a function whose
% transitive closure is a partial order, and let $\mu(f)$ be the
% set of fixed points of $f$. Then, there is a function 
% $\mathit{root}_f : A \to \mu(f)$ such that for each $x$ there is a 
% $k$ such that $f^{k} = \mathit{root}_f(x)$.
% \end{lemma}
% 
% \begin{proof}
% Observe that the transitive closure gives an order $\sqsubseteq$ such
% that for each $i \leq j$ and $x$ in $A$, $f^i(x) \sqsubseteq
% f^j(x)$. Since $A$ is finite, it follows that the size of the longest
% chain of distinct elements is at most $|A|$ -- so any sequence longer
% than that must repeat elements of $A$.
% 
% Suppose there is a repeated element, occuring at the $i$-th and
% $j$-th iterations of applying $f$ to $x$, where $i < j$. Since the
% transitive closure is a partial order, we know that if $a \sqsubseteq
% b$ and $a \sqsubseteq b$, then $a = b$. Therefore every element 
% of the sequence from $i$ to $j$ is equal to $f^i(x)$. Since $i$ is
% strictly less than $j$, this means that $f(f^i(x)) = f^i(x)$, and 
% is hence a fixed point of $f$. 
% \end{proof}
% 
% This property justifies the recursion in the implementation of
% $\find$: it says that the root of a node's parent is still the root of
% the node. It is worth contrasting the style of specification here with
% the usual inductive specification of child-pointing trees in
% separation logic. When proceeding from parent to child, we use that
% fact that list or tree predicates are inductively defined, and try to
% work directly with the inductive structure of the predicate.  Here, we
% specify the invariant on the whole heap as a relation property of the
% heap, and then have to prove that the measures we use are suitably
% well-founded.
% 
% 
% However, $\find$ also updates parents as it goes, and so at each
% iteration it works on a \emph{different} global order structure.
% We need to ensure that the answers we return are preserved under
% this update, which is what the following lemma will ensure. 
% 
% \begin{lemma}{(Fixed Point Update)}
% Suppose $A$ is a finite set, and $f : A \to A$ is a function whose
% transitive closure is a partial order, and let $\mu(f)$ be the set of
% fixed points of $f$. For a given $z$ in $A$, if we define $f' =
% [f|z:\mathit{root}_f(z)]$, then the transitive closure of $f'$ is also
% a partial order and $\mathit{root}_f = \mathit{root}_{f'}$.
% \end{lemma}
% 
% \begin{proof}
% Suppose that we have an arbitrary $x \in A$. Now, consider the sets of
% elements $U = \comprehend{y}{y \sqsupseteq_f x}$, and $U' =
% \comprehend{y}{y \sqsupseteq_{f'} x}$. (Note that $\sqsupseteq_{f'}$
% is a well-defined relation, but that we do not yet know that it is a
% partial order --- we only know that it defines a preorder.)
% 
% First, observe that $U$ is totally ordered with respect to
% $\sqsubseteq_f$. We know that every $u \in U$ is equal to $u
% = f^k(x)$ for some $k$, and so each pair of $u$ and $v$ is $f^j(x)$
% and $f^k(x)$ for some $j$ and $k$. Since the natural numbers are
% totally ordered, and since $j \leq k$ implies that $f^j(x)
% \sqsubseteq_f f^k(x)$, it follows that $U$ is totally ordered with
% respect to $\sqsubseteq_f$.
% 
% Now, $U$ either contains $z$ or not.
% 
% Suppose it does not contain $z$. Then $U = \comprehend{y}{y
%   \sqsupseteq_{f'} x} = U'$, since $f = f'$ for all arguments not
% equal to $z$. Therefore, $\mathit{root}_{f'}(x) = \mathit{root}_f(x)$.
% 
% On the other hand, suppose $U$ does contain $z$. Then we know that $U$
% contains $\mathit{root}_f(z)$, and that this is an upper bound of
% every element of $U$. 
% 
% Each distinct element of $U$ is $f^i(x)$ for some $0 \leq i <
% |U|$. Therefore, we know that $z = f^k(x)$ for some $k$. Furthermore,
% for each $j < k$, we know that $z \not= f^j(x)$. Therefore, these
% elements are the $j$ smallest elements for both $U$ and $U'$. However,
% we know that $f'(z) = \mathit{root}_f(z)$, and so at this point the
% ordered enumeration of the elements of $U'$ ends.
% 
% Therefore, $U'$ is a subset of $U$ , whose maximum element (relative
% to $f'$) is the same as the maximum element of $U$ --- in both cases,
% it is $\mathit{root}_f(x)$. Since $U'$ is a subset of $U$, it further
% follows that the transitive closure of $f'$ remains a partial
% order: there can be no cycles since we have only removed paths from
% the graph, and there were no cycles to begin with.
% \end{proof}
% 
% Besides $\find$, our API also includes the $\union$ and $\newset$ operations. 
% To create new disjoint sets, we need to that the order structure of two distinct
% set forests can be merged sensibly. 
% 
% \begin{lemma}{(Order Extension)}
% Suppose $A$ and $B$ are disjoint finite sets, and $f : A \to A$ and $g
% : B \to B$ are functions whose transitive closures are partial
% orders. Let $\mu(f)$ and $\mu(g)$ be the set of fixed points of $f$
% and $g$, respectively. 
% 
% Then $h : A \cup B \to A \cup B = f \cup g$ is also a function whose 
% transitive closure is a partial order, with $\mu(h) = \mu(f) \cup \mu(g)$, 
% and $\mathit{root}_h = \mathit{root}_f \cup \mathit{root}_g$. 
% \end{lemma}
% 
% 
% \begin{proof}
% \begin{enumerate}
% \item We need to show that the transitive closure of $h$ is a partial
% order. To do this, we need to show that the antisymmetry axiom holds, 
% since reflexivity and transitivity arise immediately from taking the
% transitive closure. 
% 
% So, suppose that $(x,y) \in h^{*}$, and likewise $(y, x) \in
% h^{*}$. From the definition of the transitive closure, we know there
% is some $m$ and $n$ such that $y = h^m(x)$, and $x = h^n{y}$.  Since
% $A$ and $B$ are disjoint, we know that either $x \in A$ or $x \in
% B$. Furthermore, from the definition of $h$ we know that $h$ takes
% elements of $A$ to elements of $A$ according to $f$, and takes
% elements of $B$ to elements of $B$ according to $g$. So if $x \in A$,
% then $y = f^m(x)$, and so $y \in A$, and so $x = f^n(y)$.  Then by the
% antisymmetry of $f^{*}$, we know that $x = y$. So if $x \in B$, then
% $y = g^m(x)$, and so $y \in B$, and so $x = g^n(y)$.  Then by the
% antisymmetry of $g^{*}$, we know that $x = y$.
% 
% \item Next, we need to show that the set of fixed points of $h$ is the
%   union of the fixed points of $f$ and $g$. This follows immediately
%   from the fact that $A$ and $B$ are disjoint, and $f$ and $g$ are
%   endofunctions.
% 
% \item Finally, we need to show that $\mathit{root}_h = \mathit{root}_f
%   \cup \mathit{root}_g$. Again, since $A$ and $B$ are disjoint, there
%   are no increasing chains of elements of $A \cup B$ that include
%   elements of both $A$ and $B$.  Therefore, the upper fixed points of
%   $h$ on its $A$ sub-domain will be given by $\mathit{root}_f$ and the
%   upper fixed points of $h$ on its $B$ sub-domain will be given by
%   $\mathit{root}_g$.
% \end{enumerate}
% \end{proof}
% 
% Finally, we need a lemma to let us prove that when we take the union
% of two sets, we will again have a graph which has the desired properties.
% 
% \begin{lemma}{(Abstract Union)}
% Suppose $A$ is a finite set, and $f : A \to A$ is a function whose
% transitive closure is a partial order, and let $\mu(f)$ be the set of
% fixed points of $f$. For any $y, z$ in $\mu(A)$, if we define $f' =
% [f|y:z]$, then the transitive closure of $f'$ is a partial order, and 
% $\mathit{root}_{f'} = \semfun{a}{(\IfTE{\mathit{root}_f(a)=y}{z}{\mathit{root}_f(a)})}$.
% \end{lemma}
% 
% \begin{proof}
% Suppose that $x \in A$. Now, consider the sets of elements $U =
% \comprehend{y}{y \sqsupseteq_f x}$, and $U' = \comprehend{y}{y
%   \sqsupseteq_{f'} x}$. We know that $U$ can be ordered into elements
% $x, \ldots, f^k(x)$, with $k = |U|$ and with $f^k(x) = \mu(x)$. 
% 
% Now, consider whether $y \in A$ or not. If it is not, then we know
% that $f'^{i}(x) = f^{i}(x)$ for all $i \leq k$. Therefore $U'$ is
% equal to this set, and is also totally ordered and has the maximal
% element $\mathit{root}_f(x)$. 
% 
% On the other hand, if $y$ is in $A$, then we know that $f'^i(x) =
% f^i(x)$ until index $k$, at which point $f'^k(x) = z$. However, we can
% easily see that $f'^{k+1}(x) = f'(f'^k(x)) = z$, which means that $U'$
% has a unique totally order with respect to $\sqsubseteq_{f'}$, and has
% maximum $z$. 
% 
% Furthermore, since each set of upper bounds of $x$ is uniquely totally
% ordered by $\sqsubseteq_{f'}$, we know that the transitive closure of
% $f'$ is a partial order, and not just a preorder. 
% \end{proof}
% 
% 
% \begin{lemma}{(Monotonicity of $\ufcontains{\phi}{x}{y}$)}
% Suppose $\ufcontains{\phi}{x}{y}$. Then $\ufcontains{\phi \otimes \psi}{x}{y}$. 
% \end{lemma}
% 
% \begin{proof}
% Take the derivation of $\ufcontains{\phi}{x}{y}$. Apply the tensor rule 
% to it, and derive $\ufcontains{\phi \otimes \psi}{x}{y}$.
% \end{proof}
% 
% \begin{lemma}
%   Suppose $\ufcontains{\phi}{x}{y}$. Then for all $\rho$, $\ufcontains{R(\rho, \phi)}{x}{\rho(y)}$. 
% \end{lemma}
% 
% \begin{proof}
% This follows by induction on the derivation of $\ufcontains{\phi}{x}{y}$. 
% \begin{itemize}
% \item Suppose $\ufcontains{\elt{x}{y}}{x}{y}$. 
% 
% In this case $R(\rho, \elt{x}{y}) = \elt{x}{\rho(y)}$. Then apply the 
% unit rule to derive $\ufcontains{R(\rho, \phi)}{x}{\rho(y)}$. 
% 
% \item Suppose that we use the tensor-left rule, so that by inversion we have
% $\phi = \psi \otimes \theta$, and $\ufcontains{\psi}{x}{y}$. 
% 
% Then, by induction we have $\ufcontains{R(\rho, \psi)}{x}{\rho(y)}$. 
% By the tensor-left rule, we have $\ufcontains{R(\rho, \psi) \otimes R(\rho, \theta)}{x}{\rho(y)}$. Then, by the definition of $R$, we have $\ufcontains{R(\rho, \psi \otimes \theta)}{x}{\rho(y)}$.
% 
% \item The tensor-right case is symmetric. 
% \end{itemize}
% \end{proof}
% 
% \begin{lemma}{(Model Extension)}
%   Suppose $(n, p, w) \models \phi$, and $n'$ is disjoint from $n$, and $p' \in n' \to n'$ such that $p'^*$ is a partial order, and $w' \in n' \to \mu(n')$. Then $(n, p \cup p', w \cup w') \models \phi$. 
% \end{lemma}
% 
% \begin{proof}
%  We proceed by induction on $\phi$. 
% 
%  \begin{itemize}
%    \item Case $\phi = I$
% 
%      This case is immediate. 
% 
%    \item Case $\phi = \elt{x}{y}$. 
% 
%      In this case, we know that $p'$ is completely disjoint from
%      $p$. As a result, the transitive closure of $p \cup p'$ is just
%      the unions of the transitive closures of $p$ and $p'$, and any
%      upper fixed point of an element of $n \cup n'$ is just the upper
%      fixed point of $p$ or of $p'$. Therefore $\mathit{root}(x)$
%      remains unchanged, and since $w'$ is an extension of $w$,
%      $w'(\mathit{root}(x)) = y$.
%      
%      \item $\phi = \psi \otimes \theta$
% 
%        This follows by induction. Suppose that $(n_1, p, w) \models
%        \psi$ and $(n_2, p, w) \models \theta$. Then, by induction
%        hypothesis, we know that $(n_1, p \cup p', w \cup w') \models
%        \psi$ and $(n_2, p \cup p', w \cup w') \models \theta$. Then,
%        by the definition of satisfaction, we know that $(n, p \cup p',
%        w \cup w') \models \psi \otimes \theta$.
%  \end{itemize}
% \end{proof}
% 
% \begin{lemma}{(Shrinking)}
% For suitably-typed $n, p, w$, suppose $(n, p, w) \models \phi \otimes \psi$. Then 
% we know that $(n, p, w) \models \phi$ and $(n, p, w) \models \psi$. 
% \end{lemma}
% \begin{proof}
%   From the definition of the model, we know that there are smaller $n_1$ and $n_2$ 
% such that $(n_1, p, w) \models \phi$ and $(n_2, p, w) \models \psi$. Then, by 
% the model extension lemma, we know that $(n, p, w) \models \phi$ and $(n, p, w) \models \psi$. 
% \end{proof}
% 
% \begin{lemma}{(Soundness of Witness Values)}
% For suitably-typed $n, p, w$, suppose $(n, p, w) \models \phi$, and that $\ufcontains{\phi}{x}{y}$. Then $w(\mathit{root}(x)) = y$.  
% \end{lemma}
% \begin{proof}
% This follows by induction on the derivation of
% $\ufcontains{\phi}{x}{y}$, with the two inductive cases following
% trivially using shrinking. The only interesting case is the base case,
% when $\ufcontains{\elt{x}{y}}{x}{y}$.  In this case, we know that $(n,
% p, w) \models \elt{x}{y}$, so we know $w(\mathit{root}(x)) = y$. 
% 
% 
% \end{proof}
% 
% \section{Correctness Proofs}
% 
% \subsection{Proof of $\newset$}
% 
% \begin{specification}
% \nextline $\{H(\phi)\}$
% \nextline $\{G(n, p, w, \phi)\}$
% \nextline $\letv{r}{[\newref{\sigma}{\Root(\None, 0)}]}{}$
% \nextline $\{G(n, p, w, \phi) * r \pointsto \Root(\None, 0)\}$
% \nextline $\letv{\unit}{[r := \Root(\Some(r), 0)]}{}$
% \nextline $\{G(n, p, w, \phi) * r \pointsto \Root(\Some(r), 0)\}$
% \nextline $r]$
% \nextline $\{G(n, p, w, \phi) * a \pointsto \Root(\Some(a), 0)\}$
% \nextline $\{p^{*} \mbox{ partial order} \land (n, p, w) \models \phi \land
%              \heap(n, p, w) * a \pointsto \Root(\Some(a), 0)\}$
% \nextline $\{p^{*} \mbox{ partial order} \land (n, p, w) \models \phi \land
%              \heap(n, p, w) * a \pointsto \Root(\Some(a), 0) \land a \not\in n\}$
% \nextline Let $n' = n \cup \setof{a}$, and let $p' = [p|a:a]$, and let $w' = [w|a:a]$. 
% \nextline By the order extension lemma,
% \nextline $\{p'^{*} \mbox{ partial order} \land (n, p', w') \models \phi \land
%              \heap(n', p', w')\}$
% \nextline By definition of modelling, $(\setof{a}, p', w') \models \elt{a}{a}$
% \nextline So $(n', p', w') \models \phi \otimes \elt{a}{a}$. Therefore
% \nextline $\{p'^{*} \mbox{ partial order} \land (n', p', w') \models \phi \otimes \elt{a}{a} \land \heap(n', p', w')\}$
% \nextline $\{G(n', p', w', \phi \otimes \elt{a}{a})\}$
% \nextline $\{H(\phi \otimes \elt{a}{a})\}$
% \end{specification}
% 
% \subsection{Proof of $\find$}
% 
% In order to prove this function correct, we will need to appeal to the
% correctness of the $\findroot$, function, which we will specifiy as
% follows:
% 
% \begin{prop}{(Specification of $\findroot$)}
%   The specification of the $\findroot$ function is 
% 
%   \begin{specification}
%     \nextline $\{G(n, p, w, \phi) \land \ufcontains{\phi}{x}{y}\}$
%     \nextline $\findroot(x)$ 
%     \nextline $\{(u, y', n).\; \exists p'.\;G(n, p', w, \phi) \;\land
%                        u = \mathit{root}(x) \land y' = \Some(w(u)) 
%                        \land p'(x) = u\}$
%   \end{specification}
% \end{prop}
% 
% \begin{proof}
% Assume we begin in the precondition $\ufcontains{\phi}{x}{y} \land
% G(n, p, w, \phi)$.  By definition, we know that $G(n, p, w, \phi) =
% p^{*}\mbox{ partial order} \land (n,p,w) \models \phi \land
% \heap(n,p,w)$, by the soundness of the models relation, we
% additionally know that $w(\mathit{root}_p(w(x))) = y$.
% 
% 
% \end{proof}
% 
% 
% 
% \begin{specification}
% \nextline $\{\ufcontains{\phi}{x}{y} \land G(n, p, w, \phi)\}$
% \nextline $\{p^{*} \mbox{ partial order} \land (n, p, w) \models \phi \land
%              \heap(n, p, w) \land \ufcontains{\phi}{x}{y}\}$
% \nextline From the soundness of the models relationship, we know 
% \nextline $\{p^{*} \mbox{ partial order} \land (n, p, w) \models \phi \land
%              \heap(n, p, w) \land x \in n \land w(\mathit{root}(x)) = y\}$
% \nextline $\{p^{*} \mbox{ partial order} \land (n, p, w) \models \phi \land
%              x \in n \land w(\mathit{root}(x)) = y $ 
% \nextline $\;\;\land\; \heap(n - \setof{x}, p, w) * \IfTE{x = p(x)}{\exists k.\; x \pointsto \Root(\Some(w(x)), k)}{x \pointsto \Child(p(x))}\}$
% 
% \nextline $\{p^{*} \mbox{ partial order} \land (n, p, w) \models \phi \land
%              x \in n \land w(\mathit{root}(x)) = y $ 
% \nextline $\;\;\land\; \heap(n - \setof{x}, p, w) *  x \pointsto (\IfTE{x = p(x)}{\Root(\Some(w(x)), k)}{\Child(p(x))})\}$
% 
% \nextline $[$\=$\letv{v}{[!x]}{}$ 
% \nextline $\{p^{*} \mbox{ partial order} \land (n, p, w) \models \phi \land
%              x \in n \land w(\mathit{root}(x)) = y $ 
% \nextline $\;\;\land\; \heap(n, p, w) \land v = (\IfTE{x = p(x)}{\Root(\Some(w(x)), k)}{\Child(p(x))})\}$
% \nextline\> $\run{}\ctext{case}(v,$\=
% \nextline\> \> $\Root(z, k) \to$
% 
% \nextline\> \> $\{p^{*} \mbox{ partial order} \land (n, p, w) \models \phi \land
%              x \in n \land w(\mathit{root}(x)) = y $ 
% \nextline\> \> $\;\;\land\; \heap(n, p, w) \land v = \Root(z, k) \land x = p(x) \land z = \Some(w(x))\}$
% 
% \nextline\> \> $\{p^{*} \mbox{ partial order} \land (n, p, w) \models \phi \land
%              x \in n \land w(\mathit{root}(x)) = y $ 
% \nextline\> \> $\;\;\land\; \heap(n, p, w) x = p(x) \land \mathit{root}(x) = x \land z = \Some(w(x))\}$
% 
% \nextline\> \> $\{p^{*} \mbox{ partial order} \land (n, p, w) \models \phi \land
%              x \in n \land w(x) = y $ 
% \nextline\> \> $\;\;\land\; \heap(n, p, w) \land x = p(x) \land \mathit{root}(x) = x \land z = \Some(w(x))\}$
% 
% \nextline\> \> $\{p^{*} \mbox{ partial order} \land (n, p, w) \models \phi \land
%              x \in n $
% \nextline\> \> $\;\;\land\; \heap(n, p, w) \land x = p(x) \land \mathit{root}(x) = x \land z = \Some(y)\}$
% \nextline \> \> $[(w, x, k)]$
% 
% \nextline\> \> $\{a.\;p^{*} \mbox{ partial order} \land (n, p, w) \models \phi \land
%              x \in n $
% \nextline\> \> $\;\;\land\; \heap(n, p, w) \land x = p(x) \land \mathit{root}(x) = x \land z = \Some(y) land a = (x, \Some(y), k)\}$
% 
% \nextline\> \> $\{a.\;G(n, p, w) \land a = (x, \Some(y), k)\}$
% \nextline\> \> $\{a.\;\exists n, p, w.\; G(n, p, w) \land land a = (x, \Some(y), k)\}$
% 
% \nextline\> \> $\Child(m) \to$
% \nextline\> \> $\{p^{*} \mbox{ partial order} \land (n, p, w) \models \phi \land
%              x \in n \land w(\mathit{root}(x)) = y $ 
% \nextline\> \> $\;\;\land\; \heap(n, p, w) \land v = \Child(p(x)) \land m = p(x) \}$
% 
% \nextline\> \> $\{p^{*} \mbox{ partial order} \land (n, p, w) \models \phi\bullet\elt{p(x)}{y} \land
%              x \in n \land w(\mathit{root}(x)) = y $ 
% \nextline\> \> $\;\;\land\; \heap(n, p, w) \land v = \Child(p(x)) \land m = p(x) \}$
% 
% 
% \nextline\> \> $[$\=$\letv{(u, y, k)}{\ctext{findroot}(m)}{}$
% 
% \nextline\> \> $\{p'^{*} \mbox{ partial order} \land (n, p', w) \models \phi\bullet\elt{p(x)}{y} \land
%              x \in n \land w(\mathit{root}(x)) = y $ 
% \nextline\> \> $\;\;\land\; \heap(n, p, w) \land v = \Child(p(x)) \land u = \mathit{root}(p(x)) \}$
% 
% 
% 
% \end{specification}
% 
% 
% 
% 
% This specification asserts that calling $\findroot$ will give us a new
% heap which (a) models the same formula as before, and (b) will return
% the upper fixed point of $x$, as well as its witness. If the witness
% is the root itself, we return $\None$, and otherwise return the
% witness wrapped in a $\Some$. (We also return the height, whose value
% we are not tracking in our invariant.)
% 
% The key difficulty in this proof is that $\findroot$ implements
% \emph{path compression} --- each recursive call modifies the heap in
% nonlocal ways. This nonlocality is evident in the fact that the parent
% and witness functions $p$ changes from the pre- to the post-condition.
% 
% The correctness of this program relies on the path compression lemma
% we proved earlier. 
% 
% 
% 
% 
% 
% \begin{lemma}{(External Membership and Modelling)}
% Suppose $n \subseteq \tau, p \in n \to n, w \in \mu(n) \to n$. Furthermore,
% suppose $(p, n, w) \models \phi$, and $\ufcontains{\phi}{x}{y}$. 
% Then $(n, p, w) \models \elt{x}{y}$. 
% \end{lemma}
% 
% \begin{proof}
% We proceed by induction on $\phi$: 
% \begin{itemize}
%   \item Case $\phi = \elt{x'}{y'}$
%     By inversion on $\ufcontains{\phi}{x}{y}$, we know that the only 
%     rule which can apply is the element rule, and so $x = x'$ and $y = y'$. 
% 
%     Then by hypothesis we know that $(n, p, w) \models \elt{x}{y}$. 
% 
%   \item Case $\phi = I$
% 
%     This case is impossible, since there is no rule from which we can
%     conclude $\ufcontains{I}{x}{y}$. 
% 
%   \item Case $\phi = \psi \otimes \theta$. 
% 
%     By inversion, we know that either $\ufcontains{\psi}{x}{y}$, or 
%     that $\ufcontains{\theta}{x}{y}$. 
% 
%     Then, since $(n, p, w) \models \psi \otimes \theta$, we know 
%     that $(n_1, p, w) \models \psi$ and $(n_2, p, w) \models \theta$,
%     where $n = n_1 \uplus n_2$. Without loss of generality, 
%     suppose that $\ufcontains{\psi}{x}{y}$. Then by induction we know
%     that $(n_1, p, w) \models \elt{x}{y}$, and then by monotonicity 
%     we know that $(n, p, w) \models \elt{x}{y}$. 
% \end{itemize}
% \end{proof}
% 
% \begin{lemma}{(Relating Membership and Heaps)}
% Suppose $n = n_1 \uplus n_2$. Then $\heap(n, p, w) \iff \heap(n_1, p, w) * \heap(n_2, p, w)$. 
% \end{lemma}
% 
% \begin{proof}
%   \begin{specification}
%     \nextline By definition, $\heap(n, p, w) =  \forall^{*} l \in n. F(l)$, where
%     \nextline  $F(l) = \IfTE{p(l) = l}{\exists n.\; l \pointsto \Root(\IfTE{w(x) = x}{\None}{\Some(w(x))}, n)}{l \pointsto \Child(p(l))}$.
%     \nextline So $\heap(n, p, w) =  \forall^{*} l \in n_1 \uplus n_2. F(l)$. 
%     \nextline So $\heap(n, p, w) =  \forall^{*} l \in n_1 F(l) * \forall^{*} l \in n_2 F(l)$. 
%     \nextline So $\heap(n, p, w) = \heap(n_1, p, w) * \heap(n_2, p, w)$. 
%   \end{specification}
% \end{proof}
% 
% \begin{specification}
% \nextline $\setof{H(\phi) \land \ufcontains{\phi}{x}{y}}$
% \nextline Since $\ufcontains{\phi}{x}{y}$ is pure, we assume it outside
%   the precondition. 
% \nextline 
%   $\{$\=$\exists n \subseteq \tau, p \in n \to n, w \in \mu(n) \to n.\;
%       p^{*} \mbox{ partial order} \land 
%              (n, p, w) \models \phi \;\land \heap(n, p, w) \}$
% \nextline 
%   $\{$\=$n \subseteq \tau, p \in n \to n, w \in \mu(n) \to n \land
%    p^{*} \mbox{ partial order} \land 
%              (n, p, w) \models \phi \;\land \heap(n, p, w) \}$
% \nextline 
%   Since $\ufcontains{\phi}{x}{y}$ implies $(n, p, w) \models \elt{x}{y}$, it 
%   follows that:
% \nextline 
%     $\{$\=$n \subseteq \tau, p \in n \to n, w \in \mu(n) \to n \land
%       p^{*} \mbox{ partial order} \land 
%              (n, p, w) \models \phi \;\land \heap(n - {x}, p, w) \;*$ 
% \nextline 
%     \> $\IfTE{p(x) = x}{\exists n.\; x \pointsto \Root(\IfTE{w(x) = x}{\None}{\Some(w(x))}, n)}{l \pointsto \Child(p(x))}
% \}$
% \nextline $\letv{v}{[!x]}{}$
% \nextline
%     $\{$\=$n \subseteq \tau, p \in n \to n, w \in \mu(n) \to n \land
%       p^{*} \mbox{ partial order} \land 
%              (n, p, w) \models \phi \;\land \heap(n - {x}, p, w)\;*$ 
% \nextline
%     \> $x \pointsto v\;\land$ 
% \nextline
%     \> $v = \IfTE{p(x) = x}{\exists n.\; \Root(\IfTE{w(x) = x}{\None}{\Some(w(x))}, n)}{\Child(p(x))} \}$
% \nextline $\run{}\ctext{case}(v,$
% \nextline \;\;$\Root(u,k) \to\;$
% \nextline \;\;
%     $\{$\=$n \subseteq \tau, p \in n \to n, w \in \mu(n) \to n \land
%       p^{*} \mbox{ partial order} \land 
%              (n, p, w) \models \phi \;\land \heap(n - {x}, p, w)\;*$ 
% \nextline
%     \> $x \pointsto v \;\land$
% \nextline
%     \> $v = \IfTE{p(x) = x}{\exists n.\; \Root(\IfTE{w(x) = x}{\None}{\Some(w(x))}, n)}{\Child(p(x))} \;\land$ 
% \nextline \> $v = \Root(u, k)\}$
% \nextline \;\;
%     $\{$\=$n \subseteq \tau, p \in n \to n, w \in \mu(n) \to n \land
%       p^{*} \mbox{ partial order} \land 
%              (n, p, w) \models \phi \;\land \heap(n - {x}, p, w)\;*$ 
% \nextline
%     \> $x \pointsto v \;\land$
% \nextline
%     \> $u = \IfTE{w(x) = x}{\None}{\Some(w(x))} \;\land$ 
% \nextline
%     \> $p(x) = x \;\land$
% \nextline \;\;\=$\ctext{case}(u,$ 
% \nextline \>\;\;\=$\None \to [(x, x, k)]$ 
% 
%   
% 
% \end{specification}

% \chapter{Verifying Interactive Programs}

\section{Introduction}

In many interactive programs, there are mutable data structures which
change over time, and which must maintain some relationships with one
another. For example, in a web browser, we need to present a web page
both as a tree data structure for scripts to manipulate, and display a
graphical image for the human user to view. Any change made to the
tree by a script must be reflected in a change to the image that the
human sees --- the two structures must remain synchronized.

Likewise, in a spreadsheet, each cell contains a formula, which may
refer to other cells, and whenever the user changes a cell, all of the
cells which transitively depend upon it must be updated. Since
spreadsheets can get very large, this should ideally be done in a lazy
way, so that only the cells visible on the screen, and the cells
necessary to compute them, are themselves recomputed.

Typically, these dependencies are written using what is called the
\emph{subject-observer} pattern. A mutable data structure (the
subject) maintains a list of all of the data structures whose
invariants depend upon it (the observers), and whenever it changes, it
calls a function on each of those observers to update them in response
to the change. (And in turn, the observers of the subject may be
subjects of still other observers, ultimately forming DAGs of
notifications.)

While natural, these programs are very challenging to verify in a
modular way. The reason is that there are two directions of
dependency, both of which matter for program proof. First, our program
invariant must have ownership over the subject's data (its
\emph{footprint}) in order to prove the correctness of code modifying
the subject. This direction of ownership is natural to verify with
separation logic.

In addition, these programs must also explicitly maintain the
\emph{other} direction of dependency as well --- we track everything
which depends upon the subject, and modify them appropriately whenever
the subject changes.  The natural program invariant now becomes a
global property: we need to know the full dependency graph covering
all subjects and observers, so that we can say that the reads and
is-read-by relations are relational transposes of one another. The
global nature of this invariant means that a naive correctness proof
will not respect the modular structure of the program --- if we modify
the dependency graph in any way, we now have to re-verify the entire
program!

However, the intention of the subject-observer pattern is precisely to
allow the program to remain oblivious to the exact number and nature
of the observers, which allows the programmer to add new observers
without disturbing the behavior of the rest of the program.  Our goal,
then, is to find a way of taking this piece of practical software
engineering wisdom, and casting it into formal terms amenable to
proof.

Concretely, this chapter's contributions are as follows: 

\begin{itemize}
  \item I define a library with a monadic API for writing
    demand-driven computations with dynamic dependencies and local
    state, and which is implemented as higher-order functions
    dynamically creating networks of imperative callbacks.

    I give an ``abstract semantics'' for this library, structured as a
    set of separation logic lemmas about our dataflow library. These
    lemmas permit \emph{modular} correctness proofs about programs
    using this API, even in the face of the fact that the program
    invariants must be defined globally upon the whole callback
    network.

    The key idea is to distinguish between the direct footprint of a
    command, and the program state which can depends upon that
    footprint. The lemmas are then phrased so that they refer only to
    the direct footprint of each command in the API. In addition, we
    structure our lemmas to justify a ramified frame property for our
    abstract semantics, which we can use to verify different parts of
    an imperative dataflow network separately.

    To do this, I follow the recipe laid out in the previous chapter,
    only now in a more complicated setting. I introduce a
    domain-specific separation logic of dataflow cells, and show how
    to specify cells in such a way that they only mention their
    footprint, and use a ramification operator to specify their effect
    on the rest of the heap. In addition to the ramification operator
    itself, we also need to specify several predicates and operators
    to let us reason about the state of the dataflow heap.

  \item To show that this is actually a good technique for specifying
    libraries, I use this specification to verify an imperative
    implementation of combinators implementing stream transducers in
    the style of functional reactive programming. This proof does not
    need to know about the implementation of the dataflow library; it
    works entirely in terms of the specification of dataflow cells.

    Ultimately, clients can reason about the behavior of the
    imperative implementation ``as if'' it were purely functional,
    even though it is implemented using local state and imperative
    callback procedures.
\end{itemize}

\section{Demand-Driven Notification Networks}

A simple intuition for a ``demand-driven notification network'' is to
think of it as a generalized spreadsheet. We have a collection of
cells, each of which contain program expressions whose evaluation may
refer to other cells. When a cell is read, the expression within the
cell is evaluated, possibly triggering the evaluation of other cells
in a cascade. Furthermore, each cell memoizes its expression, so that
repeated reads won't trigger re-evaluation. Each cell also maintains a
set of dependencies, so that if its code is changed, it and its
transitively reachable dependencies all invalidate their memoized
values.

\subsection{Implementing Notification Networks}

\begin{figure}
\mbox{}
\begin{specification}
\nextline $\codetype{\alpha} = \monad{(\alpha \times \cellset)}$ \nextline[1em]
$\celltype{\alpha} = \{$\=$code: \reftype{(\codetype{\alpha})};$ \nextline
                   \>$value: \reftype{(\opttype{\;\alpha})};$ \nextline
                   \>$reads: \reftype{\cellset};$ \nextline
                   \>$obs:   \reftype{\cellset};$ \nextline
                   \>$id:    \N\}$ \nextline[1em]

$\ecell = \exists \alpha:\star.\; \celltype{\alpha}$ \nextline[1em]

$\ctext{return} : \forall \alpha:\star.\; \alpha \to \codetype{\alpha}$ \nextline
$\ctext{return}\;x = \comp{\pair{x}{\ctext{emptyset}}}$ \nextline[1em]

$\ctext{bind} : \forall \alpha,\beta:\star.\; \codetype{\alpha} \to (\alpha \to \codetype{\beta}) \to \codetype{\beta}$ \nextline
$\ctext{bind}\;\cmd\;f = [$\=$\letv{(v,r_1)}{\cmd}{}$ \nextline
                        \>$\letv{(v',r_2)}{f\;v}{}$ \nextline
                        \>$\;\pair{v'}{\ctext{union}\;r_1\;r_2}]$\nextline[1em]

$\ctext{read} : \forall \alpha:\star.\; \celltype{\alpha} \to \codetype{\alpha}$ \nextline
$\ctext{read}\;a = [$\=$\letv{o}{\comp{!a.value}}{}$ \nextline
                     \>$\ctext{run}\;\ctext{case}(o,$ \nextline
                     \>\qquad\= $\ctext{Some}\;v \to \comp{\pair{v}{\ctext{singleton}\;(\pack{\alpha}{a})}},$ \nextline
                     \>      \> $\ctext{None} \to [$\=$\letv{\cmd}{\comp{!a.code}}{}$ \nextline
                     \>      \>                     \>$\letv{(v,r)}{\cmd}{}$ \nextline
                     \>      \>                     \>$\letv{\_}{\comp{a.value := \Some(v)}}{}$\nextline
                     \>      \>                     \>$\letv{\_}{\comp{a.reads := r}}{}$ \nextline
                     \>      \>                     \>$\letv{\_}{\ctext{iterset}\;(\ctext{add\_observer}\;(\pack{\alpha}{a}))\;r}{}$ \nextline
                     \>      \>                     \>$\;\pair{v}{\ctext{singleton}\;(\pack{\alpha}{a})}])]$ \nextline[1em]

$\getref : \forall \alpha : \reftype{\alpha} \to \codetype{\alpha}$ \nextline
$\getref r = \comp{\letv{v}{\comp{!r}}{\pair{v}{\ctext{emptyset}}}}$ \nextline[1em]

$\setref : \forall \alpha : \reftype{\alpha} \to \alpha \to \codetype{\unittype}$ \nextline
$\setref r\;v = \comp{\letv{\_}{\comp{r := v}}{\pair{\unit}{\ctext{emptyset}}}}$ \nextline[1em]

$\newcell : \forall \alpha:\star.\; \codetype{\alpha} \to \monad{\celltype{\alpha}}$ \nextline
$\newcell\;\cmd = [$\=$\letv{id}{[!counter]}{}$ \nextline
                    \>$\letv{\_}{\comp{counter := id + 1}}{}$ \nextline
                    \>$\letv{code}{\newref{\codetype{\alpha}}{\cmd}}{}$ \nextline
                    \>$\letv{value}{\newref{\opttype{\;\alpha}}{\ctext{None}}}{}$ \nextline
                    \>$\letv{reads}{\newref{\cellset}{\ctext{emptyset}}}{}$ \nextline
                    \>$\letv{obs}{\newref{\cellset}{\ctext{emptyset}}}{}$ \nextline
                    \>$\; (code, value, reads, obs, id)]$ 
\end{specification}
\caption{Implementation of Notification Networks}
\label{notification-implementation}
\end{figure}

\begin{figure}
\mbox{}
\begin{specification}
\nextline
$\updatecell : \forall \alpha:\star.\; \celltype{\alpha} \to \codetype{\alpha} \to \monad{\unittype}$\nextline
$\updatecell\;\alpha\;\mathit{cell}\;\cmd = 
     [$\=$\letv{\_}{\ctext{mark\_unready}\;\pack{\alpha}{\mathit{cell}}}{}$ \nextline
       \>$\mathit{cell}.code := \cmd]$ \nextline[1em]

$\ctext{mark\_unready} : \ecell \to \monad{\unittype}$ \nextline
$\ctext{mark\_unready}\;cell =\; $\=$\ctext{unpack}(\alpha, a) = cell\; \ctext{in}$\nextline
\>  $[$\=$\letv{os}{\comp{!a.obs}}{}$ \nextline
\>     \>$\letv{rs}{\comp{!a.reads}}{}$ \nextline
\>     \>$\letv{\_}{\ctext{iterset}\;\ctext{mark\_unready}\;os}{}$ \nextline
\>     \>$\letv{\_}{\ctext{iterset}\;(\ctext{remove\_obs}\;cell)\;rs}{}$ \nextline
\>     \>$\letv{\_}{\comp{a.value := \ctext{None}}}{}$ \nextline
\>     \>$\letv{\_}{\comp{a.reads := \ctext{emptyset}}}{}$ \nextline
\>     \>$a.obs   := \ctext{emptyset}]$ 
\nextline[1em]
$\ctext{add\_observer} : \ecell \to \ecell \to \monad{1}$ \nextline
$\ctext{add\_observer} \;o \;(\pack{\beta}{c}) = [$\=$\letv{os}{[!(c.\mathit{obs})]}{}$ \nextline
                                                  \>$\letv{os}{[\ctext{addset}\;os\;o]}{}$ \nextline
                                                  \>$c.\mathit{obs} := os]$
\nextline[1em]
$\ctext{remove\_obs} : \ecell \to \ecell \to \monad{1}$ \nextline
$\ctext{remove\_obs} \;o \;(\pack{\beta}{c}) = [$\=$\letv{os}{[!(c.\mathit{obs})]}{}$ \nextline
                                             \>$\letv{os}{[\ctext{removeset}\;os\;o]}{}$ \nextline
                                             \>$c.\mathit{obs} := os]$

\end{specification}
\caption{Implementation of Notification Networks}
\label{notification-implementation-update}
\end{figure}

Our API for creating notification networks is given in
Figure~\ref{notification-implementation} and
Figure~\ref{notification-implementation-update}. First, we'll describe
the interface, and then discuss its implementation.

The interface exposes two basic abstract data type constructors, $\ctext{cell}$
and $\ctext{code}$.

The type $\celltype{\alpha}$ is the type of dynamic data values. A
cell contains a reference to a piece of code, a possible memoized
value, and enough information to correctly invalidate its memoized
value when the cell's dependencies change. We can create a new cell by
calling $\newcell \cmd$, which returns a brand new cell with the code
expression $\cmd$ inside it. We can also modify a cell with the
command $\updatecell\;cell\;\cmd$, which modifies the cell $cell$ by
installing the new expression $\cmd$ in it.

The type $\codetype{\alpha}$ is a monadic type, representing the type of
computations that can read cells. It supports the usual operations
$\return e$ and $\bind \cmd\; (\semfun{x}{e'})$, which embed a pure value
into the $\ctext{code}$ type and implement sequential composition,
respectively. In addition, the primitive operations on this monad
include reading a cell with the $\readcell cell$ function call, and
reading and modifying local state with the $\getref r$ and $\setref
r\;v$ operations.

The difference between $\readcell$ and $\getref$ or $\setref$ is that
the first operation reads a cell, possibly further executing the code
in that cell. In contrast, $\getref$ and $\setref$ are just the
operations to update primitive pointers, lifted to live within the
$\codetype{\alpha}$ monad.
 
The actual implementation is also given in
Figures~\ref{notification-implementation} and~\ref{notification-implementation-update}. 
The abstract type of code is
implemented using the underlying monad of imperative commands, so that
$\codetype{\alpha}$ is implemented with the type $\monad{(\alpha
  \times \cellset)}$.  The intuition is that when we evaluate a term
we are allowed to read some cells along the way, and so must return a
set of all the cells that we read in order to do proper dependency
management. So, $\cellset$ is a type representing sets of
(existentially-quantified) cells.  (The precise specification of
$\cellset$ is given in Appendix A, since describing it is a
distraction from the main development.)

Cells are represented with a 5-tuple. For readability, we use a
record-style notation for them, even though our formal language has
only binary tuples, since records can be translated into nested
tuples.

There is a reference pointing to the code expression, as well as a
value field pointing to an optional value. The value field will be set
to $\ctext{None}$ if the cell is in an unready, un-memoized state, and
will be $\ctext{Some}\;v$ if the cell's code has already been
evaluated to a value $v$. In addition there are two fields
representing the dependencies. If the code expression has been
evaluated and a memoized value generated, then the $reads$ field will
point to the set of cells that the computation directly read while
computing its value. Otherwise it will point to the empty
set. Conversely, the field $obs$ contains the cell's observers --- the
set of cells that have read the current cell as part of their own
computations. Obviously, this is only non-empty when the cell has been
evaluated. Finally, each cell also has an integer id field, which is a
unique numeric identifier for every cell that is created by the
library. This permits us to compare cells (even of different type) for
equality, which is needed for dependency management.

The $\return$ operation for the library simply returns its argument
value and the empty set, since it doesn't read any cells. Likewise,
$\bind \cmd\;f$ will evaluate the first argument and pass the returned
value to the function $f$. It will return the function's return value,
together with the union of the two read sets. $\getref r$ and $\setref
r\;v$ simply read and update the reference, and return empty read
sets, since neither of them read any cells. The only reason we need 
these operations in our interface is for typing reasons; we are defining
a user-defined monad, and $\getref$ and $\setref$ lift the read and
write operations of the underlying language into this monad, adding 
the extra plumbing associated with it.

Interesting things first happen with the $\readcell e$ operation. This
function will first check to see if the cell has a memoized value
ready. If it does, we return that immediately. Otherwise, we evaluate
the cell's code, and update the current cell's value and read set. In
addition, each cell that was read in the evaluation of the code (i.e.,
the set returned as the second component of the monadic type's return
value) also has its observer set updated with the newly-ready current
cell. Now, if any of the dependencies change, they will be able to
invalidate the current cell, which observes them. Note that the
dependencies between cells are all dynamic --- we cannot examine the
inside of a code expression to find its ``free cells'', and so we rely
upon the invariant that a code expression will return every cell it
read, in addition to its return value.

Further interesting things happen with the $\newcell\;\cmd$ operation.
It creates a new cell value, setting the code field with the argument
$\cmd$, and generating an id by dereferencing and incrementing the
contents of the variable $counter$. The $counter$ variable occurs
freely in this definition, because it is a variable of type
$\reftype{\N}$ referring to a piece of state global to the module.
The reference must be initialized by whatever initialization routine
first constructs the whole module as an existential package.  Since
$counter$ is otherwise private, we can generate unique identifiers by
incrementing it as we create new cells.

Finally, the $\updatecell\;cell\;\cmd$ operation updates a cell
$cell$ with a new code expression $\cmd$. (As an aside, it's worth
noting that this is a genuine, unavoidable, use of higher-order store:
we make use of pointers to code, including the ability to dynamically
modify them.) Once we modify a cell, any memoized value it has is no
longer necessarily correct.

Therefore, we have to drop the memoized value of the cell, and any
cell that transitively observes the cell. This is what the
$\ctext{mark\_unready}$ function does. Given a cell, it takes all of
the observers of that cell and recursively makes all of them
unready. Then it removes that cell from the observer sets of all the
cells it reads, and nulls out that cell's memoized value, as well as
setting its read and observer sets to empty. Notice that there is no
explicit base case to the recursive call; if there are any cycles in
the dependency graph, invalidation could go into an infinite loop.

So far, we have described the implementation invariants incrementally
and informally. Before proceeding to the formal description, we will
state them again informally, all in one place:

\begin{itemize}
  \item Every cell must have a unique numeric identifier.
  \item Every cell is either ready, or unready. 
  \item Every ready cell has a memoized value, and maintains 
    two sets, one containing every cell that it reads, and the
    other containing every cell it is observed by. 
  \item Every unready cell has no memoized value, and has 
    both an empty read set and an empty observer set. 
  \item The overall dependency graph among the valid cells must form a
    directed acyclic graph. 
  \item The reads and the observers must be the same, only 
    pointing in opposite directions.
\end{itemize}

Formalizing these constraints is relatively straightforward, but we
have the problem that these constraints are global in nature:
we can't be sure that the dependency graph is acyclic without having
it all available to examine, and likewise we can't in general know
that a cell is in the read set of everything in its observed set
without knowing the whole graph. Handling this difficultly is one of
the primary contributions of this work. 

\section{The Abstract Semantics of Notifications}

\subsection{The Structure of the Global Invariant}

The key to getting around our difficulties lies in the difference
between the implementation of $\updatecell$, and of $\ctext{read}$.
$\updatecell$ calls $\ctext{mark\_unready}$, which recursively follows
the observers. The $\ctext{read}$ function, on the other hand,
proceeds in the opposite direction --- it evaluates code expressions,
recursively descending into the footprint of its command. The opposite
direction these two functions look is why we end up needing a global
invariant: we need to know that these two directions are in harmony
with one another.

Now, note that the type of calls to $\ctext{update}$ is simply the
monadic type $\monad{\unittype}$, which precludes clients from
constructing expressions of type $\codetype{\alpha}$ which invoke
it. The reason for this is that the $\codetype{\alpha}$ is an abstract
type, and clients can construct terms of this type only by using the
$\ctext{return}, \ctext{bind}, \ctext{read}, \ctext{getref}$ and
$\ctext{setref}$ operations. Since none of these operations themselves
invoke $\ctext{update}$, no client code can put a code expression in a
cell which can invoke it. (Making this argument completely airtight
would require a parametric model of our language, which we do not
have, but the intuition is sufficient to guide the design of the
program invariant.)

As a result, when we evaluate a code expression, we will never
actually follow the observer fields -- we'll only patch them as
needed, whenever we evaluate a cell and change it from unready to
ready.  As a result, an abstract description of the heap which
\emph{does not explicitly mention the observer sets} will prove
sufficient for reasoning about the behavior of $\codetype{\alpha}$
expressions.

With this plan, we introduce \emph{abstract heap formulas}, which are
syntactic descriptions of the state of part of the cell heap. These
syntactic expressions are given by the following grammar:
\begin{displaymath}
  \begin{array}{lcl}
    \phi & ::= & I \bnfalt \phi \otimes \psi \bnfalt \cellpos{a}{\cmd}{v}{r} \bnfalt \cellneg{a}{\cmd}\\
         &  |  & \localref{r}{v} 
  \end{array}
\end{displaymath}

Informally, a formula $I$ represents an empty abstract heap, and a
formula $\phi \otimes \psi$ represents an abstract heap that can be
broken into two disjoint parts $\phi$ and $\psi$. We will only
consider formulas modulo the associativity and commutativity of
$\otimes$, and take $I$ to be the unit of this binary operator.

The atomic form $\localref{r}{v}$ says that $r$ is a piece of local
state owned by the network, currently with value $v$. There are two
atomic forms representing cells. $\cellneg{a}{\cmd}$ says that $a$ is
a cell with code $\cmd$ that is unready to deliver a value --- it
needs to be re-evaluated before it can yield a
value. $\cellpos{a}{\cmd}{v}{r}$ says that $a$ is a cell with code
$\cmd$ that will deliver the value $v$, under the condition that all
the cells in its read set $r$ are themselves ready. However, if
anything it reads is unready, then it itself is unready. (We will
sometimes write $\celleither{a}{\cmd}$ when we do not care whether $a$
is ready or not.)

First, notice the must/may flavor of this reading. The formula
$\cellneg{a}{\cmd}$ says that $a$ \emph{must} be unready.  The formula
$\cellpos{a}{\cmd}{v}{r}$ says that $a$ \emph{may} be ready,
conditional on the readiness of the elements of its read set
$r$. Second, notice that the backwards dependencies are entirely
missing from these formulas. We have simply left out the other half of
the dependency graph from this description. Forgetting this
information will be critical for local reasoning.

We have emphasized that the straightforward invariant is not obviously
modular. Before we can elaborate any further on this point, we will
need to look at the formal statement of the heap invariant. We
introduce the predicate $G(\phi)$, which describes the entire heap of
cells allocated by our library, and which satisfy both the conditions
described in the previous section, and the additional constraint that
the cell heap agree with $\phi$.\footnote{This is why we insisted that
  the abstract heap formulas are syntactic objects --- this permits us
  to define predicates on them by induction over the structure of the
formula.}


\begin{tabbing}
    $G(\phi) \triangleq \exists H \in CellHeap.\; Inv(H, \phi)$ \\ [1em]

$Inv(H, \phi) \triangleq $ \\
\;\;\= $$\=$ R_H^\dagger = O_H \land R_H^+ \mbox{ strict partial order }$ \\
    \> \> $\land\; R_H \subseteq V_H \times V_H \land unique(H)$ \\
    \> \> $\land\; \satisfies{H}{\phi} \land heap(H) * \mathit{refs}(\phi)$ \\
\end{tabbing}

The auxiliary definitions we used in this definition are all given in
Figure~\ref{heap-invariant}. 

We first assert the existence of a cell heap $H = (D,h)$, which is a
collection of cells $D$, together with a function $h$ mapping each
cell to a code expression, a possible value, a read set, an observed
set, and an integer identifier, which satisfies the overall invariant
$Inv(H, \phi)$.



In the first two lines of $Inv(H, \phi)$, we assert all of the global 
conditions in terms of the mathematical cell heap $H$. First, we assert that 
the relational transpose $(\cdot)^\dagger$ of the reads relation $R_H$ is the
observes relation $O_H$, which enforces the condition that the reads
and observe relations be the same, only pointing in opposite
directions (i.e., if $a$ reads $b$, then $b$ is observed by
$a$). Then, we require that the transitive (but not reflexive) closure
of the reads relation, $R^+_H$ form a strict partial order. Strictness
enforces the condition that there be no cycles in the dependence graph
(because otherwise there would be elements $a$, such that $(a, a) \in
R^+_H$).  Next, we require that the reads relation $R_H$ is a subset of
the Cartesian product $V_H \times V_H$ of the set of cells with values
$V_H$. This ensures that a) there are no dependencies on unready
cells, and b) all unready cells have empty read and observe
sets. Finally, we ask that all of the cells in $H$ have unique
identifiers --- $unique(H)$ asserts that the identifier field is one
half of an isomorphism between the cells in $H$, and the natural
numbers from 0 to the size of $H$.

In the third line, we begin by requiring that the cell heap $H$ satisfy
the abstract heap formula $\phi$, which formalizes the informal
reading of the abstract heap formulas given earlier. This satisfaction
relation follows the general style of separation logic quite closely; 
in a sense, our abstract heap formulas are giving a domain-specific
version of separation logic.

The last clause $heap(H)$ finally connects the cell heap, which is a
purely mathematical object, to the actual low-level heap the
implementation uses. We ask that the global counter reference
$counter$ point to an integer field equal to the size of the cell
heap, and then use the iterated separating conjunction $\forall^*$ to
require that for each cell in the cell heap, we have pointers to the
appropriate code, value, read and observer fields. The read and
observer fields point to values of type $\cellset$, each of which
represent the appropriate sets of cells.  Since none of the prior
clauses say anything about the local state, we also use the
$\mathit{refs}(\phi)$ function to take a formula, and assemble a
collection of points-to assertions $r \pointsto v$ from each
$\localref{r}{v}$ in our formula. 

As a reminder, the specifications of the set operations are given in
Appendix A of this chapter. One unusual feature of the specification
is that we have a three-place relation $\mathit{set}(W, v, xs)$, which
says that $v$ is a value representing a set of $\mathit{ecell}$s $xs$.
Since we cannot directly compare cells of different type for equality,
we include a unique integer id field in each cell. Furthermore, since
these id fields are dynamically-generated, we need to track the
association of id fields with the natural equality of the
$\mathit{ecell}$ type.

Note that the separation requirements in $\mathit{heap}(H)$ ensure
that every cell value $c$ in set of cells $D$ has a disjoint set of
pointer values. As a result, we do not have to specify this condition
explicitly, since its failure would falsify the precondition.

\begin{figure}
\mbox{}
\begin{tabbing}
$CellHeap = \Sigma$\=$ D \in \powersetfin{\ecell}.$ \\
               \>$(\Pi (\pack{\alpha}{\_}) \in D. ($\=${\codetype{\alpha}} \;\times$
                                         ${\opttype{\;\alpha}} \;\times$ 
                                         ${\powersetfin{\ecell}} \;\times$ 
                                         ${\powersetfin{\ecell}} \;\times$ 
                                         $\N)$ \\[1em]
       

$code = \pi_1$ \\
$value = \pi_2$ \\
$reads = \pi_3$ \\
$obs = \pi_4$ \\
$id = \pi_5$ \\[1em]

$V_{(D, h)} = \comprehend{c \in D}{\exists v.\; value(h(c)) = \ctext{Some}(v)}$ \\
$R_{(D, h)} = \comprehend{(c,c') \in D \times D}{ c' \in reads(h(c)) }$ \\
$O_{(D, h)} = \comprehend{(c,c') \in D \times D}{ c' \in obs(h(c)) }$ \\[1em]

$\mathit{Fin}(n) = \comprehend{ i \in \N }{ i < n }$ \\[1em]

$unique(D,h) = \exists$\=$i : Fin(|D|) \to D.\; i \circ (id \circ h) = I_D \; \land (id \circ h) \circ i = I_{|D|}$  \\
where $I_X : X \to X = \fun{x}{x}$ \\[1em]

$\satisfies{(D,h)}{\phi} = (D,h), D \models \phi$ \\[1em]

$\models \;\subseteq\; \Sigma (D,h):CellHeap.\; D' \subseteq D$ \\[0.5em]

$(D,h), D' \models \localref{r}{v} \qquad \qquad$\=$\iff \top$ \\
$(D,h), D' \models I$                            \>$\iff \top$ \\
$(D,h), D' \models \phi \otimes \psi$            \>$\iff \exists D_1, D_2.\;$\=$D' = D_1 \uplus D_2 
                                                      \land (D,h), D_1 \models \phi
                                                      \land (D,h), D_2 \models \psi$ \\
$(D,h), D' \models \cellneg{a}{\cmd}$               \>$\iff a \in D' \land code(h(a)) = \cmd \land a \not \in V_{(D,h)}$ \\
$(D,h), D' \models \cellpos{a}{\cmd}{v}{r}$         \>$\iff $\=$ a \in D' \land code(h(a)) = \cmd \;\land$ \\
\>      \>$\mathrm{if}\;r \subseteq V_{(D,h)}$ \\
\>      \>$\mathrm{then}\;value(h(a)) = \ctext{Some}\;v \land reads(h(a)) = r$ \\
\>      \>$\mathrm{else}\;a \not\in V_{(D,h)}$\\[1em]
 

$heap(D,h) = $ \\
\;\;$counter \pointsto |D| \;* $ \\
\;\;$\forall^* (\pack{\tau}{c}\mbox{ as }cell) \in D.\;$\=$\exists v_r, v_o : \cellset.\;$ \\
                         \>$c.code \pointsto code(h(cell))   \;* $ \\
                         \>$c.value \pointsto value(h(cell)) \;* $ \\
                         \>$c.reads \pointsto v_r \;* $ \\
                         \>$c.obs   \pointsto v_o \;* $ \\
                         \>$c.id    = id(h(cell)) \;\land$ \\
                         \>$set(D, v_r, reads(h(cell))) \;\land$ \\
                         \>$set(D, v_o, obs(h(cell)))$ \\[1em]

$\mathit{refs}(I) \qquad\qquad\qquad\qquad$\=$= \emp$ \\
$\mathit{refs}(\phi \otimes \psi) $\>$= \mathit{refs}(\phi) * \mathit{refs}(\psi)$ \\
$\mathit{refs}(\cellneg{c}{\cmd}) $\>$= \emp$ \\
$\mathit{refs}(\cellpos{c}{\cmd}{v}{r}) $\>$= \emp$ \\
$\mathit{refs}(\localref{r}{v}) $\>$= r \pointsto v$ \\
\end{tabbing}

\caption{Definitions for Heap Invariant}
\label{heap-invariant}
\end{figure}

The global character of this invariant should be evident; we need to
talk about \emph{all} of the cells in the heap in order to state our
invariants. So it is not immediately clear that we have made much
progress towards a modular proof technique. However, we are actually
very close: with just two more ideas, we will be able to give a
solution to this problem.

\subsection{Frame Properties via Polymorphism}

As we mentioned earlier, our abstract heap formulas essentially give
us a small domain-specific separation logic. This means that in order
to reason locally over cell heaps, we need to find an
application-specific version of the frame rule for our library. 

To do this, we will adapt some ideas proposed by
Benton~\cite{benton}, and Birkedal and Yang~\cite{birkedal-yang}. They
suggested interpreting the frame rule of separation logic as a form of
quantification --- instead of having a separate frame rule that allows
adding a frame to any triple, they proposed that all of the atomic
rules of the program logic be replaced with rules possessing an extra
quantifier ranging over ``the rest of the heap'':

\begin{mathpar}
  \inferrule*[right=Example]
          { }
          { \forall R.\; \setof{ (e \pointsto v) * R } \;e := v'\; \setof{ (e \pointsto v') * R}}
\end{mathpar}

This quantifier is propagated through the proof, and any use of the
frame rule can be interpreted as instantiating the universal
quantifier appropriately. The reason this idea is fruitful for us is
that it will allow us to give a frame rule, even though the underlying
semantics of our library does not actually satisfy any analogues of
the traditional safety monotonicity and frame lemmas. For example, the
$\updatecell$ operation certainly does not act locally --- it
recursively traverses the observers set, possibly mutating a very
large part of the cell graph.

\subsection{Correctness of $\newcell$ and $\updatecell$}

In this subsection, I will prove the correctness of the two operations
in our interface which lie outside the $\codetype{\alpha}$
monad. Specifications and proofs of the operations in the monad will
follow, afterwards.

\subsubsection{Correctness of $\updatecell$}

We will now prove the soundness of the following triple specifying
$\updatecell$.

\begin{prop}{(Update Rule)}
For all cells $o$ and code expressions $\cmd$ and $\cmd'$, the following
triple is provable: 

\begin{tabbing}
$\forall \psi:\formula.\; $\=$\setof{G(\celleither{o}{\cmd'} \otimes \psi)}$ \\
                           \>$\run {\updatecell\;o\;\cmd}$ \\
                           \>$\setof{a:1.\; G(\cellneg{o}{\cmd} \otimes \psi)}$
\end{tabbing}
\end{prop}

\textbf{Proof Strategy.} The key to this proof is the conditional
interpretation of the $\cellpos{c}{\cmd}{v}{r}$ formula. When we evaluate
the $\updatecell$ command, we recursively find every cell which
depends on $o$, and modify it to be unready.

Now consider any positive cell formula in $\psi$ which reads $o$,
directly or indirectly. The satisfaction relation for $\phi$ tells us
that a positive formula must have everything in its read set be ready,
in order to represent a ready cell. So by changing $o$'s formula to
the unready state, we no longer require that positive formula
represent a ready cell. As a result, we can leave the entire frame
$\psi$ untouched, even though the physical heap it represents may have
been (quite drastically) modified.

However, before we can give a proper proof of this theorem, we need to
specify and prove correct the $\ctext{mark\_unready(c)}$ procedure,
which does the bulk of the work of the update function. The purpose of
this function is to set every cell transitively reachable from $c$ to
the unready state.

As with the proofs in the previous chapter, these follow the pattern 
of a lemma about partial orders, a lemma connecting it to the satisfaction
relation, and finally a correctness proof of the procedure. 

First, we will prove the fact that the order in which we delete
reachable subtrees doesn't matter. This is necessary since the 
algorithm will sequentially delete reachable sub-DAGs, and we
need to know that the effect of sequential deletion is equivalent
to a parallel one.  

\begin{lemma}{(Reachability in Deleted Graphs)}
Suppose we have $H = (D,h)$ such that $R^+_H$ is a strict partial
order, $R_H \subseteq V_H \times V_H$, $R^\dagger_H = O_H$, and
$\mathit{unique}(H)$.  Now, for any $X \subseteq D$, define $U^H_X =
\comprehend{d \in D}{(d, c) \in R^*_H \land c \in X}$ to be the set
of cells which transitively read any cell in $X$, and also define
$h|_X$ so that:
\begin{itemize}
\item $\mathit{code}(h|_X(c')) = \mathit{code}(h(c'))$
\item $\mathit{value}(h|_X(c')) = \IfThenElse{c' \in U^H_X}{\None}{\mathit{value}(h(c'))}$
\item $\mathit{reads}(h|_X(c')) = \IfThenElse{c' \in U^H_X}{\emptyset}{\mathit{reads}(h(c'))}$
\item $\mathit{obs}(h|_X(c')) = \IfThenElse{c' \in U^H_X}{\emptyset}{\mathit{obs}(h(c')) - U^H_X}$
\item $\mathit{identity}(h|_X(c')) = \mathit{identity}(h(c'))$
\end{itemize}

It is the case that $(d,c) \in R^*_{H|_X}$ if and only if $(d,c) \in R^*_H \land c \not\in U^H_X \land d \not\in U^H_X$. 
\end{lemma}

\begin{proof}

\begin{tabbedproof}
\oo $\Rightarrow$: \\
\ooo Assume $(d,c) \in R^*_{H|_X}$ \\
\ooo Therefore we have a sequence $(d, \ldots, c)$ in $R_{H|_X}$ \\
\ooo Now, we proceed by induction on the number of steps from $d$ to $c$. \\
\ooo Case $n = 0$: \\
\oooo Then $c = d \in V_{H|_X}$.\\
\oooo Since $\mathit{value}(h|_{X}(c')) = \IfThenElse{c' \in U^H_X}{\None}{\mathit{value}(h(c'))}$, we know $c \not\in U^H_X$ \\
\ooo Case $n = k+1$: \\
\oooo Hence we have a sequence $(d, \ldots, c_k, c) \in R_{H|_X}$, such that \\
\oooo $(d, c_k) \in R^*_H$ and $d \not \in U^H_X$ and $c_k \not \in U^H_x$ \\ 
\oooo We know $c \in \mathit{reads}(h|_X(c_k)) = \IfThenElse{c_k \in U^H_X}{\emptyset}{\mathit{reads}(h(c_k))}$ \\
\oooo Since $c_k \not \in U^H_X$, we know $c \in \mathit{reads}(h(c_k))$, so $(c_k, c) \in R_H$ \\
\oooo Hence $(d, c) \in R^*_H$ \\
\oooo We know $c \not \in U^H_X$, since if it were then $c_k \in U^H_X$, which is contradictory \\
% 
% \ooo Each consecutive pair $(c_i, c_{i+1})$ satisfies $c_{i+1} \in \mathit{reads}(h|_X(c_i))$ \\
% \ooo We know $\mathit{reads}(h|_X(c_i) = \IfThenElse{c' \in U^H_X}{\emptyset}{\mathit{reads}(h(c_i))}$ \\
% \ooo Hence each $c_{i+1} \not\in U^H_X$ and $(c_i, c_{i+1}) \in R_H$ \\ 
% \ooo Hence $(d,c) \in R^*_H$ and $d \not \in U^H_X$ and $c \not \in U^H_X$ \\ 
\oo $\Leftarrow$: \\
\ooo Asssume $(d,c) \in R^*_H \land c \not\in U^H_X \land d \not\in U^H_X$. \\
\ooo So we have a sequence $(d, \ldots, c)$ with each pair in $R_H$ \\
% \ooo Suppose an intermediate element $c_i$ between $d$ and $c$ is in $U^H_X$ \\
% \oooo Then we have a sequence $(c, \ldots, c_i) \in R^*_H$ and a sequence $(c_i,x) \in R^*_X$ for some $x \in X$ \\ 
% \oooo Concatenating, we have a sequence $(c, \ldots, x) \in R^*_H$ with $x \in X$ \\
% \oooo This implies that $c \in U^H_X$, which is a contradiction \\
% \ooo Hence no element in the sequence $(d, \ldots, c)$ is in $U^H_X$ \\ 
\ooo Now, we will show $(d, \ldots, c) \in R^*_X$ by induction on the length \\
\ooo Case $n = 0$: \\
\oooo So $d = c \in V_{H|_X}$. \\
\oooo Since $d \not\in U^H_X$, we know $d \in V_H$ \\
\oooo Hence $(d,c) \in R^*_H$ \\
\ooo Case $n = k+1$: \\
\oooo We have $(d, \ldots, c_k, c) \in R^*_H \land c \not\in U^H_X \land d \not\in U^H_X$. \\
\oooo If $c_k \in U^H_X$ then $d \in U^H_X$, which is a contradiction \\
\oooo Hence $c_k \not\in U^H_X$ \\
\oooo By induction, $(d, \ldots, c_k) \in R^*_H$ \\
\oooo We know $c \in \mathit{reads}(h|_X(c_k))$ \\ 
\oooo Hence $c \in \IfThenElse{c_k \in U^H_X}{\emptyset}{\mathit{reads}(h(c_k))}$ \\
\oooo Hence $c \in \mathit{reads }(h(c_k))$ \\
\oooo Hence $(d, c) \in R^*_H$ \\ 
% \ooo For every $(c_k, c_{k+1}) \in (d, \ldots, c)$, we know $c_{k+1} \in \mathit{reads}(h(c_k))$ \\
% \ooo So we know $c_{k+1} \in \IfThenElse{c_k \in U^H_X}{\emptyset}{\mathit{reads}(h(c_k))}$ \\
% \ooo Hence for every $(c_k, c_{k+1}) \in (d, \ldots, c)$, we know $c_{k+1} \in \mathit{reads}(h|_X(c_k))$ \\
% \ooo Hence $(d,c) \in R^*_{H|_X}$ \\
\end{tabbedproof}
\end{proof}

\begin{lemma}{(Path Union)}
Suppose we have $H = (D,h)$ such that $R^+_H$ is a strict partial
order, $R_H \subseteq V_H \times V_H$, $R^\dagger_H = O_H$, and
$\mathit{unique}(H)$. Then $c \in U^{H|_{X}}_{\setof{z}} \vee
(c \not\in U^{H|_{X}}_{\setof{z}} \land c \in U^H_{X})$
is logically equivalent to $c \in U^H_{\setof{z}} \vee c \in U^H_{X}$.
\end{lemma}


\begin{proof}
\begin{tabbedproof}
\oo Note $c \in U^{H|_{X}}_{\setof{z}}$ is equivalent to $(c,z) \in R^*_{H|_{X}}$, \\
\oo which is equivalent to  $(c, z) \in R^*_H \land c \not\in U^H_{X} \land z \not\in U^H_{X}$, \\
\oo which is equivalent to  $c \in U^H_{\setof{z}} \land c \not\in U^H_{X} \land z \not\in U^H_{X}$, \\
\oo Expanding $c \in U^{H|_X}_{\setof{z}}$ in the formula $c \in U^{H|_{X}}_{\setof{z}} \vee (c \not\in U^{H|_{X}}_{\setof{z}} \land c \in U^H_{X})$, we get \\
\oox $(c \in U^H_{\setof{z}} \land c \not\in U^H_{X} \land z \not\in U^H_{X}) \;\vee$ \\
\oox $(\lnot(c \in U^H_{\setof{z}} \land c \not\in U^H_{X} \land z \not\in U^H_{X}) \land c \in U^H_{X})$   \\
\oo We can simplify the second clause as follows: \\
\oo $\lnot(c \in U^H_{\setof{z}} \land c \not\in U^H_{X} \land z \not\in U^H_{X}) \land c \in U^H_{X}$   \\
\ooo $\simeq (c \not\in U^H_{\setof{z}} \vee c \in U^H_X \vee z \in U^H_X) \land c \in U^H_X$ \\
\ooo $\simeq (c \not\in U^H_{\setof{z}} \land c \in U^H_X) \vee 
             (z \in U^H_X \land c \in U^H_X) \vee 
             c \in U^H_X$ \\
\ooo $\simeq c \in U^H_X$ \\
\oo Hence the whole condition is equivalent to:\\
\oox $(c \in U^H_{\setof{z}} \land c \not\in U^H_{X} \land z \not\in U^H_{X}) \vee c \in U^H_{X}$ \\
\oo We want to show this is equivalent to $c \in U^H_{\setof{z}} \vee c \in U^H_{X}$ \\
\oo The forward direction of the equivalence trivial. \\
\oo To prove the backwards direction, assume $c \in U^H_{\setof{z}} \vee c \in U^H_{X}$ \\
\ooo Now case on this assumption \\
\ooo Case $c \in U^H_{X}$: \\
\oooo Then $c \in U^H_{X}$ \\
\ooo Case $c \in U^H_{\setof{z}}$: \\ 
\oooo Case on whether $c \in U^H_{X}$ \\
\oooo Case $c \in U^H_{X}$: \\
\ooooo Then $c \in U^H_{X}$ \\
\oooo Case $c \not\in U^H_{X}$: \\
\ooooo Branch on whether $z \in U^H_{X}$ \\
\ooooo Case $z \in U^H_{X}$: \\
\oooooo Since $c \in U^H_{\setof{z}}$, there is a sequence $(c, \ldots, z) \in R^*_H$ \\
\oooooo Since $z \in U^H_{X}$, there is a sequence $(z, \ldots, x) \in R^*_H$ for some $x \in X$ \\
\oooooo By concatenation, there is a sequence $(c, \ldots, x) \in R^*_H$ for some  $x \in X$ \\
\oooooo Hence $c \in U^H_{X}$, which is a contradiction \\
\ooooo Case $z \not\in U^H_{X}$: \\
\oooooo Hence $c \in U^H_{\setof{z}} \land c \not\in U^H_{X} \land z \not\in U^H_{X}$ \\
\end{tabbedproof}
\end{proof}

\begin{lemma}{(Sequentialization)}
Suppose we have $H = (D,h)$ such that $R^+_H$ is a strict partial
order, $R_H \subseteq V_H \times V_H$, $R^\dagger_H = O_H$, and
$\mathit{unique}(H)$.

Then $H|_X|_Y = H|_{X \cup Y}$.
\end{lemma}

\begin{proof}
We prove this by induction on the size of $Y$. 
\begin{tabbedproof}
\oo Case $Y = \emptyset$: \\
\ooo Immediate \\
\oo Case $Y = Z \uplus \setof{z}$: \\
\ooo By induction, we know that $H|_X|_Z = H|_{X \cup Z}$ \\
\ooo Consider whether $z \in X \cup Z$ \\
\ooo Case $z \in X \cup Z$: \\
\oooo Then $U^{H|_{X \cup Z}}_{\setof{z}} = \emptyset$ \\
\oooo Therefore $H|_{X \cup Z}|_{\setof{z}} = H|_{X \cup Z} = H|_{X \cup Y}$  \\ 
\ooo Case $z \not\in X \cup Z$: \\
\oooo To show that $H|_{X \cup Z}|_{\setof{z}} = H|_{\setof X \cup Z \cup \setof{z}} $, we need to show $h|_{X \cup Z}|_{\setof{z}} = h|_{\setof X \cup Z \cup \setof{z}}$ \\
\oooo 1. Consider $\mathit{reads}(h|_{X \cup Z}|_{\setof{z}}(c))$ \\
\ooooo $\mathit{reads}(h|_{X \cup Z}|_{\setof{z}}(c)) = \IfThenElse{c \in U^{H|_{X \cup Z}}_{\setof{z}}}{\emptyset}{\mathit{reads}(h|_{X \cup Z}(c))}$ \\
\ooooo $\mathit{reads}(h|_{X \cup Z}|_{\setof{z}}(c)) = \IfThenElse{c \in U^{H|_{X \cup Z}}_{\setof{z}}}{\emptyset}{\IfThenElse{c \in U^H_{X \cup Z}}{\emptyset}{\mathit{reads}(h(c))}}$ \\
\ooooo $\mathit{reads}(h|_{X \cup Z}|_{\setof{z}}(c)) = \IfThenElse{c \in U^{H|_{X \cup Z}}_{\setof{z}} \vee (c \not\in U^{H|_{X \cup Z}}_{\setof{z}} \land c \in U^H_{X \cup Z})}{\emptyset}{\mathit{reads}(h(c))}$ \\
\ooooo By the path union lemma, \\
\ooooox $\mathit{reads}(h|_{X \cup Z}|_{\setof{z}}(c)) = \IfThenElse{c \in U^H_{X \cup Z \setof{z}}}{\emptyset}{\mathit{reads}(h(c))}$ \\
\ooooo Hence $\mathit{reads}(h|_{X \cup Z}|_{\setof{z}}(c)) = \mathit{reads}(h|_{X \cup Z \cup \setof{z}})$ \\
\oooo 2. Consider $\mathit{value}(h|_{X \cup Z}|_{\setof{z}}(c))$ \\
\ooooo $\mathit{value}(h|_{X \cup Z}|_{\setof{z}}(c)) = \IfThenElse{c \in U^{H|_{X \cup Z}}_{\setof{z}}}{\None}{\mathit{value}(h|_{X \cup Z}(c))}$ \\
\ooooo $\mathit{value}(h|_{X \cup Z}|_{\setof{z}}(c)) = \IfThenElse{c \in U^{H|_{X \cup Z}}_{\setof{z}}}{\None}{\IfThenElse{c \in U^H_{X \cup Z}}{\None}{\mathit{value}(h(c))}}$ \\
\ooooo $\mathit{value}(h|_{X \cup Z}|_{\setof{z}}(c)) = \IfThenElse{c \in U^{H|_{X \cup Z}}_{\setof{z}} \vee (c \not\in U^{H|_{X \cup Z}}_{\setof{z}} \land c \in U^H_{X \cup Z})}{\None}{\mathit{value}(h(c))}$ \\
\ooooo By the path union lemma, \\
\ooooox $\mathit{value}(h|_{X \cup Z}|_{\setof{z}}(c)) = \IfThenElse{c \in U^H_{X \cup Z \setof{z}}}{\None}{\mathit{value}(h(c))}$ \\
\ooooo Hence $\mathit{value}(h|_{X \cup Z}|_{\setof{z}}(c)) = \mathit{value}(h|_{X \cup Z \cup \setof{z}})$ \\
\oooo 3. Consider $\mathit{obs}(h|_{X \cup Z}|_{\setof{z}}(c))$ \\
\ooooo $\mathit{obs}(h|_{X \cup Z}|_{\setof{z}}(c)) = \IfThenElse{c \in U^{H|_{X \cup Z}}_{\setof{z}}}{\emptyset}{\mathit{obs}(h|_{X \cup Z}(c)) - U^{H|_{X \cup Z}}_{\setof{z}}}$ \\
\ooooo $\mathit{obs}(h|_{X \cup Z}|_{\setof{z}}(c)) = \IfThenElse{c \in U^{H|_{X \cup Z}}_{\setof{z}}}{\emptyset}{\IfThenElse{c \in U^H_{X \cup Z}}{\emptyset}{\mathit{obs}(h|_{X \cup Z}(c)) - U^{H|_{X \cup Z}}_{\setof{z}}}}$ \\
\ooooo $\mathit{obs}(h|_{X \cup Z}|_{\setof{z}}(c)) = \IfThenElse{c \in U^{H|_{X \cup Z}}_{\setof{z}} \vee (c \not\in U^{H|_{X \cup Z}}_{\setof{z}} \land c \in U^H_{X \cup Z})}{\emptyset}{\mathit{obs}(h|_{X \cup Z}(c)) - U^{H|_{X \cup Z}}_{\setof{z}}}$ \\
\ooooo By the path union lemma, \\
\ooooox $\mathit{obs}(h|_{X \cup Z}|_{\setof{z}}(c)) = \IfThenElse{c \in U^H_{X \cup Z \setof{z}}}{\emptyset}{\mathit{obs}(h|_{X \cup Z}(c)) - U^{H|_{X \cup Z}}_{\setof{z}}}$ \\
\ooooo So $\mathit{obs}(h|_{X \cup Z}|_{\setof{z}}(c)) = \IfThenElse{c \in U^H_{X \cup Z \setof{z}}}{\emptyset}{\mathit{obs}(h(c)) - U^H_{X \cup Z} - U^{H|_{X \cup Z}}_{\setof{z}}}$ \\
\ooooo So $\mathit{obs}(h|_{X \cup Z}|_{\setof{z}}(c)) = \IfThenElse{c \in U^H_{X \cup Z \setof{z}}}{\emptyset}{\mathit{obs}(h(c)) - (U^H_{X \cup Z} \cup U^{H|_{X \cup Z}}_{\setof{z}})}$ \\
\ooooo Now, suppose $y \in U^H_{X \cup Z} \cup U^{H|_{X \cup Z}}_{\setof{z}}$ \\
\ooooo This is equivalent to $y \in U^H_{X \cup Z} \vee y \in U^{H|_{X \cup Z}}_{\setof{z}}$ \\
\ooooo This is equivalent to $(y \not\in U^{H|_{X \cup Z}}_{\setof{z}} \land y \in U^H_{X \cup Z}) \vee (y \in U^{H|_{X \cup Z}}_{\setof{z}} \land y \in U^H_{X \cup Z}) \vee y \in U^H_{X \cup Z}$ \\
\ooooo This is equivalent to $(y \not\in U^{H|_{X \cup Z}}_{\setof{z}} \land y \in U^H_{X \cup Z}) \vee y \in U^H_{X \cup Z}$ \\
\ooooo By path union, this is equivalent to $y \in U^H_{X \cup Z} \vee y \in U^H_{X \cup Z}$ \\
\ooooo Hence $U^H_{X \cup Z} \cup U^{H|_{X \cup Z}}_{\setof{z}} = U^H_{X \cup Z \cup \setof{z}}$ \\
\ooooo Hence we know that \\
\ooooo $\mathit{obs}(h|_{X \cup Z}|_{\setof{z}}(c)) = \IfThenElse{c \in U^H_{X \cup Z \setof{z}}}{\emptyset}{\mathit{obs}(h(c)) - U^H_{X \cup Z \cup \setof{z}}}$\\
\ooooo Hence $\mathit{obs}(h|_{X \cup Z}|_{\setof{z}}(c)) = \mathit{obs}(h|_{X \cup Z \cup \setof{z}})$ \\
\end{tabbedproof}
\end{proof}

\begin{lemma}{(Transitive Invalidation Preserves Order Structure)}
Suppose we have $H = (D,h)$ such that 
\begin{enumerate}
\item $R^+_H$ is a strict partial order 
\item $R_H \subseteq V_H \times V_H$
\item $R^\dagger_H = O_H$ 
\item $\mathit{unique}(H)$
\end{enumerate}
\noindent For any cell $c \in D$, it is the case that for $H' = (D, h') = H|_{\setof{c}}$, we have
\begin{enumerate}
\item $R^+_{H'}$ is a strict partial order 
\item $V_{H'} = V_H - U^H_{\setof{c}}$
\item $R_{H'} \subseteq V_{H'} \times V_{H'}$
\item $R^\dagger_{H'} = O_{H'}$ 
\item $\mathit{unique}(H')$
\end{enumerate}
\end{lemma}

\begin{proof}
\begin{enumerate}
\item $R^+_{H'}$ is a strict partial order 
  \begin{tabbedproof}
    \oo This follows if $R^+_{H'}$ is transitive relation without any $(a,a) \in R^+_H$ \\
    \oo Since it is a transitive closure, it is a transitive relation \\
    \oo To show that there is no $a \in D$ such that $(a,a) \in R^+_{H'}$, \\
    \ox we will show that $R^+_{H'} \subseteq R^+_H$ \\
    \oo Assume that $(a_0,b) \in R^+_{H'}$ \\
    \oo Then there is a sequence $a_0, \ldots, a_k$ such that $b = a_k$ and each $a_{i+1} \in \mathit{reads}(h'(a_i))$ \\
    \oo Suppose we have an arbitrary $a_i, a_{i+1}$, such that $a_{i+1} \in \mathit{reads}(h'(a_i))$ \\
    \oo Hence it follows that $a_i \not\in U^H_{\setof{c}}$, since otherwise $\mathit{reads}(h'(a_i)) = \emptyset$ \\
    \oo Hence it follows that $\mathit{reads}(h'(a_i)) = \mathit{reads}(h(a_i))$ for $i < k$ \\
    \oo Hence $a_0, \ldots, a_k$ such that $b = a_k$ and each $a_{i+1} \in \mathit{reads}(h(a_i))$ \\
    \oo Hence $(a_0, b) \in R^+_H$ \\
    \oo Hence $a_0 \not= b$, since $R^+_H$ strict partial order 
  \end{tabbedproof}

\item $V_{H'} = V_H - U^H_{\setof{c}}$
  \begin{tabbedproof}
    \oo Suppose $a \in V_{H'}$ \\
    \ooo Then $\mathit{value}(h'(a)) = \Some(v)$ for some $v$ \\
    \ooo Hence $a \not\in U^H_{\setof{c}}$ \\
    \ooo Hence $\mathit{value}(h'(a)) = \mathit{value}(h(a))$  \\
    \ooo Hence $a \in V_H$ \\
    \ooo Hence $a \in V_{H} - U^H_{\setof{c}}$ \\
    \oo Conversely, suppose $a \in V_{H} - U^H_{\setof{c}}$ \\
    \ooo Then $\mathit{value}(h(a)) = \Some(v)$ for some $v$ \\
    \ooo But since $a \not\in U^H_{\setof{c}}$, we know $\mathit{value}(h'(a)) = \mathit{value}(h(a)) = \Some(v)$ \\
    \ooo Hence $a \in V_{H'}$ 
  \end{tabbedproof}

\item $R_{H'} \subseteq V_{H'} \times V_{H'}$
  \begin{tabbedproof}
    \oo Assume $(a,b) \in R_{H'}$ \\
    \oo Hence $b \in \mathit{reads}(h'(a))$ \\
    \oo Hence we know that $a \not\in U^H_{\setof{c}}$ \\
    \oo Hence we know that $b \in \mathit{reads}(h(a))$ and $(a,b) \in R_H$\\
    \oo Hence we know that $a \in V_H$ and $b \in V_H$\\
    \oo Hence we know that $a \in V_H - U^H_{\setof{c}}$ \\
    \oo Hence we know that $a \in V_{H'}$ \\
    \oo Since $U^H_{\setof{c}}$ is transitively-closed, we know that if $b \in U^H_{\setof{c}}$, then $a \in U^H_{\setof{c}}$ \\
    \oo Since $a \not\in U^H_{\setof{c}}$, we know that $b \not\in U^H_{\setof{c}}$ \\
    \oo Hence we know that $b \in V_H - U^H_{\setof{c}}$ \\
    \oo Hence we know that $b \in V_{H'}$ 
  \end{tabbedproof}

\item $R^\dagger_{H'} = O_{H'}$ 
  \begin{tabbedproof}
    \oo Suppose that $(a, b) \in R^\dagger_{H'}$. \\
    \ooo We want to show that $(a, b) \in O_{H'}$ \\
    \ooo We know that $(b, a) \in R_{H'}$ \\
    \ooo We know that $a \in \mathit{reads}(h'(b))$ \\
    \ooo Since $R_{H'} \subseteq V_{H'} \times V_{H'}$, we know $a \not\in U^H_{\setof{c}}$ and $b \not \in U^H_{\setof{c}}$ \\
    \ooo Hence $a \in \mathit{reads}(h(b))$ \\ 
    \ooo Hence $b \in \mathit{obs}(h(a))$ \\
    \ooo Hence $b \in \mathit{obs}(h'(a))$ \\
    \ooo Hence $(a,b) \in O_{H'}$ \\
    \oo Suppose that $(a, b) \in O_{H'}$. \\
    \ooo We want to show that $(a,b) \in R^\dagger_{H'}$, so we want $(b,a) \in R_{H'}$ \\
    \ooo Since $(a,b) \in O_{H'}$, we know $b \in \mathit{obs}(h'(a))$ \\
    \ooo Hence $a \not\in U^H_{\setof{c}}$ and $b \in \mathit{obs}(h(a)) - U^H_{\setof{c}}$ \\
    \ooo So $b \in \mathit{obs}(h(a))$ and $b \not \in U^H_{\setof{c}}$ \\
    \ooo So $(a,b) \in O_H$ \\
    \ooo So $(b,a) \in R_H$, and so $a \in \mathit{reads}(h(b))$ \\
    \ooo Since $b \not\in U^H_{\setof{c}}$, $\mathit{reads}(h'(b)) = \mathit{reads}(h(b))$ \\
    \ooo Therefore $b \in \mathit{reads}(h'(a))$, and so $(b,a) \in R_{H'}$ \\
    \ooo Therefore $(a,b) \in R^\dagger_{H'}$ 
  \end{tabbedproof}

\item $\mathit{unique}(h) = \mathit{unique}(h')$.

This is immediate, since $\mathit{identity}(h) = \mathit{identity}(h')$ 
\end{enumerate}
\end{proof}


\begin{lemma}{(Updating the Abstract Heap)}
Suppose we have $H = (D,h)$ and $H, D', \models \phi \otimes
\celleither{c}{\cmd}$ and $c \in D$. Then for $H' = (D, h') = H|_{\setof{c}}$, we
have $H', D' \models \phi \otimes \cellneg{c}{\cmd}$.
\end{lemma}

\begin{proof}
\begin{tabbedproof}
\oo Assume we have $H, D' \models \phi \otimes \celleither{c}{\cmd}$ \\
\oo Then we have $D_1, D_2$ such that $D_1 \uplus D_2 = D'$ and $H, D_1 \models \phi$ and $H, D_2 \models \celleither{c}{\cmd}$ \\
\oo Note that this means $c$ is not in $D_1$ \\
\oo First, consider $(D,h), D_1 \models \phi$ \\
\ooo We will show by induction that $(D,h'), D_1 \models \phi$ \\
\ooo Case $\phi = I$ or $\phi = \localref{r}{v}$: \\
\oooo In this case, by the definition of satisfaction, $(D,h), D_1 \models I$ or $(D,h), D_1 \models \localref{r}{v}$\\
\ooo Case $\phi = \psi \otimes \theta$: \\
\oooo In this case, we have $D'_1, D'_2$ such that $D_1 = D'_1 \uplus D'_2$ and \\
\oooox $(D,h), D'_1 \models \psi$ \\
\oooox $(D,h), D'_2 \models \theta$ \\
\oooo By induction, we know that \\
\oooox $(D,h'), D'_1 \models \psi$ \\
\oooox $(D,h'), D'_2 \models \theta$ \\
\oooo Hence by the definition of satisfaction, $(D,h'), D_1 \models \psi \otimes \theta$ \\
\ooo Case $\phi = \cellneg{a}{\cmd}$: \\
\oooo In this case, we know $a \in D_1$ and $\mathit{code}(h(a)) = \cmd$ and $a \not\in V_{(D,h)}$ \\
\oooo Hence $\mathit{value}(h(a)) = \None$, and so $\mathit{value}(h'(a)) = \None$ \\
% \oooo Since $R_H \subseteq V_H \times V_H$, it follows  $U = \emptyset$ \\
% \oooo Hence $h' = h$ \\
\oooo Hence $(D,h'), D_1 \models \cellneg{a}{\cmd}$ \\
\ooo Case $\phi = \cellpos{a}{\cmd}{v}{r}$: \\
\oooo Now, consider whether $r \subseteq V_H$ \\
\oooo If $r \subseteq V_H$: \\
\ooooo We know $a \in D_1$ and $\mathit{code}(h(a)) = \cmd$ and 
       $\mathit{value}(h(a)) = \Some(v)$ and $\mathit{reads}(h(a)) = r$ \\
\ooooo So $a \in V_H$. \\
\ooooo Now consider whether $a \in U^H_{\setof{c}}$ \\
\ooooo If $a \in U^H_{\setof{c}}$: \\
\oooooo First, note $a \not\in V_{(D,h')}$ \\
\oooooo Then $(a, c) \in R^*_H$ \\
\oooooo Since $c \not= a$ (since $c$ is not in the domain of $\phi$), we know $(a, c) \in R^+_H$ \\
\oooooo Hence there is a $d \in r$ such that $(a,d) \in R_H$ and $(d, c) \in R^*_H$  \\
\oooooo Hence $d \in r$ and $d \in U^H_{\setof{c}}$ and $\mathit{value}(h'(d)) = \None$, and so $d \not\in V_{(D,h')}$\\
\oooooo Hence $r \not\subseteq V_{(D, h')}$ \\
\oooooo Hence $(D,h'), D_1 \models \cellpos{a}{\cmd}{v}{r}$ \\
\ooooo If $a \not\in U^H_{\setof{c}}$: \\
\oooooo Then note $a \in V_{(D,h')}$ \\
\oooooo Furthermore, $(a, c) \not\in R^*_H$ \\
\oooooo Hence there is no $d \in r$ such that $(c, d) \in R^*_H$  \\
\oooooo Hence $r \cap U^H_{\setof{c}} = \emptyset$ \\
\oooooo Hence $r \subseteq V_{(D,h')}$ \\
\oooooo Hence $(D,h'), D_1 \models \cellpos{a}{\cmd}{v}{r}$ \\
\oooo If $r \not\subseteq V_H$: \\
\ooooo Then we know $a \in D_1$ and $\mathit{code}(h(a)) = \cmd$ and $a \not\in V_H$  \\
\ooooo Since $V_{H'} \subseteq V_H$, we know $r \not\subseteq V_{H'}$ too \\
% \ooooo Since $R_H \subseteq V_H \times V_H$, it follows that $\comprehend{c'\in D}{(a,c') \in R} = \emptyset$. \\
% \ooooo Hence $U = \emptyset$ \\
% \ooooo Hence $h' = h$ \\
\ooooo Hence $(D,h'), D_1 \models \cellpos{a}{\cmd}{v}{r}$ \\
\oo Next, consider $(D,h), D_2 \models \celleither{c}{\cmd}$: \\
\ooo Since $(c, c) \in R^*_H$,  we know $c \in U$ \\
\ooo Hence it follows $c \not\in V_{(D, h')}$ \\
\ooo Hence $(D,h'), D_2 \models \cellneg{c}{\cmd}$ \\
\oo Therefore, it follows that $(D,h'), D' \models \phi \otimes \cellneg{c}{\cmd}$ \\ 
\end{tabbedproof}
\end{proof}


\begin{lemma}{(The $\ctext{mark\_unready}$ procedure preserves order structure)}
Suppose $H = (D,h)$ where $R^+_H$ is a strict partial order,
$O^\dagger_H = R_H$, $R_H \subseteq V_H \times V_H$ and
$\mathit{unique}(H)$. Then for any $c \in D$ and $H' = (D, h') =
H|_{\setof{c}}$, the $\ctext{mark\_unready}$ function satisfies the following specification:

\begin{displaymath}
\mspec{\mathit{heap}(H)}{\ctext{mark\_unready}(c)}{a:\unittype}{\mathit{heap}(H')}
\end{displaymath}
  
\end{lemma}

\begin{proof}
We will be to prove this function using the fixed point induction
rule, using precisely this specification. (In fact, the function is
totally correct, since it is always called on arguments higher up in
the partial ordering on cells, but tracking the termination metric in
detail obscures the key ideas.)

Now we'll proceed line-by-line through the function body of $\ctext{mark\_unready}(c)$: 

\begin{tabbedproof}
\oo We know \\
\oox $R^\dagger_H = O_H$ \\
\oox $R^+_H$ strict partial order \\
\oox $R_H \subseteq V_H \times V_H$ \\
\oox $\mathit{unique}(H)$ \\
\oox $\mathit{heap}(H)$ \\
\ooo Consider whether $c$ is in $V_H$ \\
\ooo Suppose $c \not\in V_H$: \\
\oooo Then $U = \emptyset$, since $R_H \subseteq V_H \times V_H$ \\ 
\oooo Therefore $H'$ is equal to $H$ \\
\oooo Since $R_H \subseteq V_H \times V_H$, we know $\mathit{reads}(h'(c)) = \mathit{obs}(h'(c)) = \emptyset$ \\
\oooo $[\unpack{\alpha}{a}{c}{}$ \\
\oooo Now, we know that $c = \pack{\alpha}{a}$ \\
\oooo $\letv{os}{[!(a.obs)]}{}$ \\
\oooo $\letv{rs}{[!(a.reads)]}{}$ \\
\oooo From the definition of $\mathit{heap}(H)$, we know that  \\
\oooo $\mathit{set}(D, os, \emptyset)$ and $\mathit{set}(D, rs, \emptyset)$ \\
\oooo Simplifying using the axioms for sets, lines 8 and 9 of Figure~\ref{notification-implementation-update} are no-ops \\
\oooo $\letv{\_}{[a.value := \None]}{}$ \\
\oooo $\letv{\_}{[a.reads := \ctext{emptyset}]}{}$ \\
\oooo $a.obs := \ctext{emptyset}]$ \\
\oooo None of these assignments have any effect on the validity of $\mathit{heap}(H')$ \\
\ooo Suppose $c \in V_H$: \\
\oooo $[\unpack{\alpha}{a}{c}{}$ \\
\oooo Now, we know that $c = \pack{\alpha}{a}$ \\
\oooo $\letv{os}{[!(a.obs)]}{}$ \\
\oooo $\letv{rs}{[!(a.reads)]}{}$ \\
\oooo From the definition of $\mathit{heap}(H)$, we know that  \\
\oooo $\mathit{set}(D, os, \mathit{obs}(h(c)))$ and $\mathit{set}(D, rs, \mathit{reads}(h(c)))$ \\
\oooo Now, by the equational axioms in the cellset interface, $\ctext{iterseq\;mark\_unready}\;os$ \\
\ooox is equal to a sequence of calls to $\ctext{mark\_unready}$, once for each element of $\mathit{obs}(h(c))$\\
\oooo By induction on the size of $\mathit{obs}(h(c))$, we end in a logical state $(D, h|_{os})$, \\
\ooox and a physical state $\mathit{heap}(D, h|_{\setof{os}})$, which (using lemma 79)  \\
\ooox satisfies the properties $R^\dagger_{(D,h|_{os})} = O_{(D,h|_{os})}$, $R^+_{(D,h|_{os})}$ is a strict partial order, \\
\ooox $R_{(D,h|_{os})} \subseteq V_{(D,h|_{os})} \times V_{(D,h|_{os})}$ and $\mathit{unique}(D,h|_{os})$ \\
\oooo Hence in $H' = (D,h|_{\setof{os}})$, every $c'$ which read $c$ in $H$ is no longer in $V_{H'}$ \\ 
\oooo As a result $U^{H'}_{\setof{c}} = \setof{c}$ \\ 
\oooo Since $R_H$ was a strict partial order, $c$ is still in $V_{H'}$ \\
\oooo Therefore consider $h'' = h'|_{\setof{c}}$, which is equivalent to   \\
\oooox $\mathit{value}(h'') = \semfun{c'}{\IfThenElse{c' = c}{\None}{\mathit{value}(h'(c'))}}$ \\
\oooox $\mathit{reads}(h'') = \semfun{c'}{\IfThenElse{c' = c}{\emptyset}{\mathit{reads}(h'(c'))}}$ \\
\oooox $\mathit{obs}(h'') = \semfun{c'}{\IfThenElse{c'\in \mathit{obs}(h(c))}{\emptyset}{\mathit{obs}(h'(c')) - \setof{c}}}$ \\
\oooo Likewise, by the equational axiom in the cellset interface, $\ctext{iterseq}\;(\ctext{remove\_obs}\;cell)\;rs$ \\
\ooox is equal to a sequence of calls to $\ctext{remove\_obs}$, once for each element of $\mathit{reads}(h'(c))$\\
\oooo $\letv{\_}{[a.value := \None]}{}$ \\
\oooo $\letv{\_}{[a.reads := \emptyset]}{}$ \\
\oooo $a.obs := \emptyset]$ \\
\oooo These updates then ensure that $\mathit{heap}(D, h'')$ is satisfied. \\
\end{tabbedproof}
\end{proof}

\begin{lemma}{(Correctness of $\ctext{mark\_unready}$)}
The $\ctext{mark\_unready}$ function satisfies the following specification:   
\begin{displaymath}
\mspec{G(H, \phi \otimes \celleither{c}{\cmd})}{\ctext{mark\_unready}}{a:\unittype}{\exists H'.\;G(H', \phi \otimes \cellneg{c}{\cmd})}
\end{displaymath}
\end{lemma}

\begin{proof}
  This follows immediately from the previous lemma, together with the lemma about updating the 
abstract heap. 
\end{proof}


\begin{lemma}{(Updating Negative Formulas)}
Suppose we have  $H = (D,h)$ and $H, D' \models \phi \otimes \cellneg{c}{\cmd}$. Then, if
we define $h'$ such that $\mathit{code}(h'(c)) = \cmd'$ and is equal to $h$ otherwise, 
it follows that with $H' = (D,h')$, we have $H', D' \models \phi \otimes \cellneg{c}{\cmd'}$. 
\end{lemma}

\begin{proof}
  \begin{tabbedproof}
    \oo Assume $H, D' \models \phi \otimes \cellneg{c}{\cmd}$. \\
    \ooo So we have $D_1, D_2$ such that $D' = D_1 \uplus D_2$ and $H, D_1 \models \phi$ and $H, D_2 \models \cellneg{c}{\cmd}$ \\
    \ooo Since $H, D_2 \models \cellneg{c}{\cmd}$, we know that $c \in D_2$ and $c \not\in D_1$ \\
    \ooo Now proceed by induction on $\phi$: \\
    \ooo Explicitly, our induction hypothesis is that if $H, \hat{D} \models \phi$ and $c \not \in D$, then $H', \hat{D} \models \phi$ \\
    \ooo Case $\phi = I$ or $\phi = \localref{r}{v}$: \\
    \oooo This case is immediate \\
    \ooo Case $\phi = \psi \otimes \theta$: \\
    \oooo In this case, we have $D_3, D_4$ such that $D_1 = D_3 \uplus D_4$ and  \\
    \oooox $H, D_3 \models \psi$ and $H, D_4 \models \theta$ \\
    \oooo So by induction we have $H', D_3 \models \psi$ and $H', D_4 \models \theta$ \\
    \oooo Hence $H', D_1 \models \phi$\\
    \ooo Case $\phi = \cellneg{c'}{\cmd}$: \\
    \oooo We know $c' \in D_1$ \\
    \oooo We know that $c' \not= c$. Hence $\mathit{code}(h'(c')) = \mathit{code}(h(c))$ \\
    \oooo Hence $H', D_1 \models \cellneg{c'}{\cmd}$ \\
    \ooo Case $\phi = \cellpos{c'}{\cmd}{v}{r}$: \\
    \oooo We know that $c' \in D_1$ and hence $c \not= c'$ \\
    \oooo Hence $\mathit{code}(h'(c')) = \mathit{code}(h(c))$ \\
    \oooo Furthermore, $V_{H'} = V_H$ since $\mathit{value} \circ h' = \mathit{value} \circ h$\\
    \oooo Also $\mathit{reads} \circ h' = \mathit{reads} \circ h$ \\
    \oooo Hence $H', D_1 \models \cellpos{c'}{\cmd}{v}{r}$: \\
    \ooo Therefore $H', D_1 \models \phi$ \\
    \ooo Now, we want to show $H', D_2 \models \cellneg{c}{\cmd'}$ \\
    \oooo We know $c \in D_2$ \\
    \oooo Furthermore, $\mathit{code}(h'(c)) = \cmd'$ \\
    \oooo Hence $H', D_2 \models \cellneg{c}{\cmd'}$ \\
    \ooo Hence $H', D' \models \phi \otimes \cellneg{c}{\cmd'}$ 
  \end{tabbedproof}
\end{proof}

\begin{lemma}{(Update Rule)}
For all cells $o$ and code expressions $\cmd$ and $\cmd'$, the following
triple is provable: 

\begin{tabbing}
$\forall \psi:\formula.\; $\=$\setof{G(\celleither{o}{\cmd'} \otimes \psi)}$ \\
                           \>$\run {\updatecell\;o\;\cmd}$ \\
                           \>$\setof{a:1.\; G(\cellneg{o}{\cmd} \otimes \psi)}$
\end{tabbing}
\end{lemma}

\begin{proof}
This follows immediately from first applying the specification of $\ctext{mark\_unready}$, 
and then using the lemma that updating negative lemmas works as expected. 
\end{proof}


\subsubsection{Correctness of $\newcell$}

We can prove the soundness of a similar specification for $\newcell$ as
well:

\begin{prop}{(New Cell Rule)}
For all code expressions $\cmd$ of type $\alpha$, the following specification is provable: 
\begin{tabbing}
$\forall \psi:\formula.\; $\=$\setof{G(\psi)}$ \\
                           \>$\run {\newcell\;\cmd}$ \\
                           \>$\setof{a:\celltype{\alpha}.\; G(\cellneg{a}{\cmd} \otimes \psi)}$
\end{tabbing}
\end{prop}

\begin{proof}
This is much easier than $\updatecell$: we just need to allocate a new
numeric id for the new cell, and show that the extended cell heap
continues to satisfy all of the expected properties.

Concretely, suppose that $G(\psi) = \mathit{Inv}{(H, \psi)}$ for some
$H = (D,h)$. Now, by the invariant, we know that each cell in $D$ has
a unique numeric id somewhere in the range from $0$ to $|D|-1$. Then,
on lines 29-30 of Figure~\ref{notification-implementation}, we
increment the global counter by 1, and let $id = |D|$.  Then, on lines
31-34, we allocate a pointer $\mathit{code}$ to the code $\cmd$, a
pointer $\mathit{value}$ to $\None$ (to indicate the unitialized
state), as well as pointers to $\mathit{read}$ and $\mathit{obs}$ to
empty read and observer sets.

At line 35, all of these values are bound to variables in scope, so we
can define $H' = (D',h')$, where $\mathit{ec} = \pack{\tau}{\mathit{(code, value, read, obs, id)}}$, 
and 
\begin{displaymath}
(D',h') = (D \cup \setof{\mathit{ec}}, h' = [h|\mathit{ec}: (\cmd, \None, \emptyset, \emptyset, id)])
\end{displaymath}

Then, it is easy to show that the modified heap realizes $H'$ --- the
only fact that does not carry over immediately are the predicates
establishing the the read and observer sets are implemented
correctly. This is because they held for world $D$, and we are now in
world $D'$. Happily, we required as an axiom that $(W \subseteq W')
\implies \mathit{set}(W, v, S) \implies \mathit{set}(W', v, S)$.

Finally, we need to show that $\cellneg{a}{\cmd} \otimes \psi$ is
satisfied by $H'$. This follows from an easy induction on $\psi$:
since $h'$ is just an extension of $h$ with $V_{H'} = V_H$, none
of the conditionals or function calls in the cell predicates have
different results, which means that the truth-values they define
are unchanged. 

Then we can hide $H'$ behind an existential to establish the postcondition. 
\end{proof}

\begin{figure}
\mbox{}
\begin{mathpar}
  \inferrule*[right=Ready]
            {\forall a' \in r.\; \exists v.\; \ready{\phi}{a'}{v}}
            {\ready{\phi \otimes \cellpos{a}{\cmd}{v}{r}}{a}{v}}
  \\
  \\
  \inferrule*[right=UnreadyPos]
            {\exists a' \in r.\; \unready{\phi}{a'}}
            {\unready{\phi \otimes \cellpos{a}{\cmd}{v}{r}}
                     {a}}
  \and
  \inferrule*[right=UnreadyNeg]
            { }
            {\unready{\phi \otimes \cellneg{a}{\cmd}}{a}}
\end{mathpar}
\caption{Ready and Unready Judgments}
\label{readiness}
\end{figure}

\begin{figure}
\mbox{}
  \begin{displaymath}
    \begin{array}{lcl}
      \closed{I}{s} & = & \top \\
      \closed{\phi \otimes \psi}{s} & = & \closed{\phi}{s} \land \closed{\psi}{s} \\ 
      \closed{\localref{r}{v}}{s} & = & \top \\
      \closed{\cellneg{a}{\cmd}}{s} & = & \top \\
      \closed{\cellpos{a}{\cmd}{v}{r}}{s} & = & r \subseteq s \\
    \end{array}
  \end{displaymath}
\caption{Closedness predicate}
\label{closedness}  
\end{figure}

\begin{figure}
\mbox{}
  \begin{displaymath}
    \begin{array}{lcl}
      R(s, I)                 & = & I \\
      R(s, \phi \otimes \psi) & = & R(s, \phi) \otimes R(s, \psi) \\
      R(s, \localref{r}{v})   & = & \localref{r}{v} \\
      R(s, \cellneg{a}{\cmd})    & = & \cellneg{a}{\cmd} \\
      R(s, \cellpos{a}{\cmd}{v}{r}) & = & \left\{\begin{array}{ll}
                                                \cellpos{a}{\cmd}{v}{r} 
                                              & \mbox{if } s \cap r = \emptyset \\
                                                \cellneg{a}{\cmd}
                                              & \mbox{otherwise}
                                              \end{array}
                                       \right.
    \end{array}
  \end{displaymath}
\caption{Definition of the Ramification Operator $R$}
\label{ramify-def}
\end{figure}

\section{Ramified Frame Properties}

The proof strategy of the previous section is sufficient for
$\newcell$ and $\updatecell$, but is not adequate for defining a frame
property for $\codetype{\alpha}$ expressions.

As an example, suppose that we want to evaluate the formula $\readcell
a$, in a cell heap described by $\cellneg{a}{\return 5}$.  Clearly,
this is a sufficient footprint, and we expect to a) get the return
value 5, and b) see the cell formula change to $\cellpos{a}{\return
  5}{5}{\emptyset}$.  However, the fact that we are now changing cells
from negative to positive means that the conditional character of
readiness, which worked in our favor with $\updatecell$ and
$\newcell\!\!$, now works against us.

In particular, suppose that we run this command with a framed abstract heap
formula $\psi = \cellpos{b}{\readcell a}{17}{\setof{a}}$. Now, the
whole starting heap will be:
\begin{displaymath}
\cellneg{a}{\return 5} \otimes \cellpos{b}{\readcell a}{17}{\setof{a}}  
\end{displaymath}
In any heap satisfying this formula, $b$ will be unready, because it depends 
on an unready cell. But when we execute $\readcell a$, simply copying $\psi$ 
into the post-state will give us the cell formula:
\begin{displaymath}
\cellpos{a}{\return 5}{5}{\emptyset} \otimes \cellpos{b}{\readcell a}{17}{\setof{a}}
\end{displaymath}
That is, our satisfaction relation now expects $b$ to be ready and have the 
value 17, even though $\readcell a$ never touches $b$ at all!

Clearly, we cannot simply copy the same frame formula into the pre-
and the post-condition states in the specification of commands like
$\readcell a$.

To deal with this problem, we will return to the idea of
ramifications, introduced in the previous chapter. We can understand
our difficulty as an instance of the ramification problem as
follows. 

When we evaluate a code expression, we may read some unready cells and
send them from an unready state in the precondition to a ready state
in the postcondition. However, we may have had some cell formulas in
our frame which claimed their corresponding cells were unready purely
because one of the cells in our footprint was unready. Therefore, when
we update the footprint, we must modify the frame formula to account
for the ramifications of our update in the footprint. So even though
the actual physical storage representing the frame doesn't change at
all, we need to modify our abstract formula to reflect our updated
state of knowledge.

In our case, \emph{all} of the effects on the frame will arise from
the cells we flip from unready to ready. Thus, given the set of cells
which became ready, we can repair the framing formula by taking each
positive cell formula, and setting it to a negative state if its read
set includes anything that went from unready to ready. We define the
ramification operator $R(s, \psi)$ in Figure~\ref{ramify-def}.  It is
a simple structural induction over a framing formula, whose only
action is to replace the positive cell formulas in $\psi$ whose read
sets intersect with $s$ with a corresponding negative cell formula.
The ramification operator has a number of useful properties, which are
most easily expressed after we have introduced a few auxiliary
judgments and predicates.

In Figure~\ref{readiness} we define the two judgments $\unready{\phi}{o}$
and $\ready{\phi}{o}{v}$, which establish whether a cell is ready or
unready, from the syntactic structure of $\phi$. 

\begin{prop}{(Soundness of $\ready{\phi}{o}{v}$ and $\unready{\phi}{o}$)}
For all $\phi, o,$ and $H$ such that $H = (D,h)$:

\begin{enumerate}
\item $(Inv((D,h), \phi) \land \ready{\phi}{o}{v}) \implies \mathit{value}(h(o)) = \ctext{Some}\;v$
\item $(Inv((D,h), \phi) \land \unready{\phi}{o}) \implies o \not\in V_H$
\end{enumerate}
\end{prop}

\begin{proof}
\begin{enumerate}
\item $(Inv((D,h), \phi) \land \ready{\phi}{o}{v}) \implies \mathit{value}(h(o)) = \ctext{Some}\;v$

We prove this by induction on the derivation of
$\ready{\phi}{o}{v}$. By inversion, we know that $\phi = \psi \otimes
\cellpos{o}{\cmd}{v}{r}$, and for each $c' \in r$, we have a derivation
of $\ready{\psi}{c'}{v'}$ for some $v'$. By induction, we know that
for each $c' \in r$, $c' \in V_H$. By the fact that $(D,h), D \models
\psi \otimes cellpos{c}{\cmd}{v}{r}$, we know that $c \in V_H$ and
$\mathit{value}(h(o)) = \ctext{Some}(v)$. Hence the conclusion follows. 

\item $(Inv((D,h), \phi) \land \unready{\phi}{o}) \implies o \not\in V_H$

We prove this by induction on the derivation of $\unready{\phi}{o}$.
First, consider the case where $\phi = \psi \otimes \cellneg{o}{\cmd}$. 
Then, the fact that $(D,h), D \models \psi \otimes \cellneg{o}{\cmd}$
implies that $o \not\in V_H$. Now, consider the other case, where 
$\phi = \psi \otimes \cellpos{o}{\cmd}{v}{r}$. We know that there is
some $a \in r$ such that $\unready{\psi}{a}$ holds, and by induction
we know that $a \not\in V_H$. Hence, it follows that $a \not\subseteq V_H$,
and so the fact that $(D,h), D \models \psi \otimes \cellneg{o}{\cmd}$ implies
that $o \not\in V_H$. 
\end{enumerate}
\end{proof}


In Figure~\ref{closedness}, we define the $\closed{\phi}{s}$ predicate,
which asserts that every cell formula in $\phi$ reads at most the
cells in $s$. Now, we can summarize the interactions between the 
ramification operator $R$ and abstract heap formulas as follows: 

\begin{prop}{(Interaction Properties)}
Given sets of cells $s$ and $u$, cell $o$, value $v$, and formula $\phi$, we have
that:
\begin{itemize}
\item $R(s, R(u, \phi)) = R(s \cup u, \phi)$
\item If $\unready{\phi}{o}$, then $\unready{R(u, \phi)}{o}$ 
\item If $\ready{R(u, \phi)}{o}{v}$, then $\ready{\phi}{o}{v}$ 
\item If $R(u, \phi)$ and $\closed{\phi}{s}$, then $R(u, \phi) = R(u \cap s, \phi)$ 
\end{itemize}
\end{prop}

All of these facts can be proved with simple inductive arguments. 

The first property means that if we evaluate two expressions, we can
simply combine their ramification effects without having to worry
about the order that they were evaluated in. The second and third let
us know that a ramification cannot make us forget a cell is unready,
nor can it make anything ready that was not ready before. The last
property permits us to constrain the effect of a ramification --- if we
know that two parts of the abstract heap formula do not read each
other at all, we can deduce that ramifications from one will not
affect the other.

\begin{prop}{(Entailments)}
Define $\phi \vdash \theta$ to mean that for all $H$ and $D$, that if $H, D \models \phi$ then 
$H, D \models \theta$. 

\begin{itemize}
\item For all $\phi, \theta$, we have $\phi \otimes \theta \vdash \phi$
\item For all $\phi, \phi', \psi$, if $\phi \vdash \phi'$ then $\phi \otimes \psi \vdash \phi' \otimes \psi$  
\item If $\unready{\phi}{c}$ then $\phi \dashv\vdash R(\setof{c}, \phi)$
\item If $\unready{\phi \otimes \celleither{c}{\cmd}}{c}$ then $\phi \otimes \celleither{c}{\cmd} \dashv\vdash \phi \otimes \cellneg{c}{\cmd}$
\end{itemize}
\end{prop}

\begin{proof}
\begin{itemize}
\item For all $\phi, \theta$, we have $\phi \otimes \theta \vdash \phi$
\begin{tabbedproof}
\oo Assume we have $\phi, \theta$ and $H, D \models \phi \otimes \theta$ \\
\oo Now we inductively show that for all $D' \supseteq D$, we have $H, D \models \phi$ implies $H, D' \models \phi$ \\
\ooo Case $\phi = I$: \\
\oooo Immediate \\
\ooo Case $\phi = \localref{r}{v}$: \\
\oooo Immediate\\
\ooo Case $\phi = \psi \otimes \theta$: \\
\oooo If $D' \supseteq D$, then there is a $D''$ such that $D' = D \uplus D''$ \\
\oooo By inversion, we have $D_1$ and $D_2$ with $D = D_1 \uplus D_2$ such that \\
\ooox $H, D_1 \models \psi$ and $H, D_2 \models \theta$ \\
\oooo By induction, we know that $H, D_1 \uplus D'' \models \psi$ \\
\oooo Since $D''$ is disjoint from $D$, we know it is disjoint from $D_1$ and $D_2$ \\
\oooo Hence $D_1 \uplus D''$ is disjoint from $D_2$ \\
\oooo Hence $H, D_1 \uplus D'' \uplus D_2 \models \psi \otimes \theta$ \\
\oooo Hence $H, D' \models \phi$ \\
\ooo Case $\phi = \cellneg{c}{\cmd}$: \\
\oooo Since $c \in D$, we know $c \in D'$ \\
\oooo Hence $H, D' \models \cellneg{c}{\cmd}$ \\
\ooo Case $\phi = \cellpos{c}{\cmd}{v}{r}$: \\
\oooo Since $c \in D$, we know $c \in D'$ \\
\oooo Hence $H, D' \models \cellneg{c}{\cmd}$ \\
\oo Now note that we have $D_1, D_2$ such that $D = D_1 \uplus D_2$ and \\
\ox $H, D_1 \models \phi$ and $H, D_2 \models \theta$ \\
\oo Since $D \supseteq D_1$, we know that $H, D \models \phi$ \\
\end{tabbedproof}

\item For all $\phi, \phi', \psi$, if $\phi \vdash \phi'$ then $\phi \otimes \psi \vdash \phi' \otimes \psi$  
\begin{tabbedproof}
\oo Assume we have $\phi \vdash \phi'$ \\
\oo Now, assume we have $H, D$ such that $H,D \models \phi \otimes \psi$ \\
\ooo So we know that there are $D_1, D_2$ such that $D = D_1 \uplus D_2$ and \\
\oox $H, D_1 \models \phi$ and $H, D_2 \models \psi$ \\
\ooo Since $\phi \vdash \phi'$, we know $H, D_1 \models \phi'$ \\
\ooo Hence $H, D_1 \uplus D_2 \models \phi' \otimes \psi$ \\
\ooo Hence $H, D \models \phi' \otimes \psi$ \\
\end{tabbedproof}


\item If $\unready{\phi}{c}$ then $\phi \dashv\vdash R(\setof{c}, \phi)$
\begin{tabbedproof}
\oo Assume $\unready{\phi}{c}$, and that we have $H$ and $D$. \\
\oo We want to show that $H, D \models \phi$ iff $H, D \models R(\setof{c}, \phi)$ \\
\oo $\Rightarrow$: Assume $H, D \models \phi$ \\
\ooo Now, we proceed by induction on $\phi$: \\
\ooo Case $\phi = I$: \\
\oooo Since $R(\setof{c}, I) = I$, this case is immediate \\
\ooo Case $\phi = \localref{r}{v}$: \\
\oooo Since $R(\setof{c}, \localref{r}{v}) = \localref{r}{v}$, this case is immediate \\
\ooo Case $\phi = \sigma \otimes \theta$: \\
\oooo By definition, we know there are $D_1, D_2$ such that $D = D_1 \uplus D_2$ and \\
\ooox $H, D_1 \models \sigma$ and $H, D_2 \models \theta$ \\
\oooo By induction $H, D_1 \models R(\setof{c}, \sigma)$ and $H, D_2 \models R(\setof{c}, \theta)$ \\
\oooo Hence $H, D \models R(\setof{c}, \sigma) \otimes R(\setof{c}, \theta)$ \\
\oooo Hence $H, D \models R(\setof{c}, \sigma \otimes \theta)$ \\
\ooo Case $\phi = \cellneg{c'}{\cmd}$: \\
\oooo Since $R(\setof{c}, \cellneg{c'}{\cmd}) = \cellneg{c'}{\cmd}$, this case is immediate \\
\ooo Case $\phi = \cellpos{c'}{e'}{v'}{r'}$: \\
\oooo Consider whether $c \in r'$: \\
\oooo If $c \not\in r'$: \\
\ooooo Then $R(\setof{c}, \cellpos{c'}{e'}{v'}{r'}) = \cellpos{c'}{e'}{v'}{r'}$ and this case is immediate \\
\oooo If $c \in r'$: \\
\ooooo Since $\unready{c}{\cmd}$ is sound, we know that $c \not\in V_H$. \\ 
\ooooo Hence $r' \not\subseteq V_H$, and so we know that $c' \not\in V_H$ \\
\ooooo Furthermore, we know that $c' \in D$, and $\mathit{code}(h(c')) = e'$, \\
\oooox so we know $H, D \models \cellneg{c'}{e'}$ \\
\ooooo Since $c \in r'$, we know $R(\setof{c}, \cellpos{c'}{e'}{v'}{r'}) = \cellneg{c'}{e'}$ \\
\ooooo Hence $H, D \models R(\setof{c}, \cellpos{c'}{e'}{v'}{r'})$ \\
\oo $\Leftarrow$: Assume $H, D \models R(\setof{c}, \phi)$ \\
\ooo Now proceed by induction on $\phi$: \\
\ooo Case $\phi = I$: \\
\oooo Since $R(\setof{c}, I) = I$, this case is immediate \\
\ooo Case $\phi = \localref{r}{v}$: \\
\oooo Since $R(\setof{c}, \localref{r}{v}) = \localref{r}{v}$, this case is immediate \\
\ooo Case $\phi = \sigma \otimes S\theta$: \\
\oooo We know $R(\setof{c}, \sigma \otimes \theta) = R(\setof{c}, \sigma) \otimes R(\setof{c}, \theta)$ \\
\oooo Hence we know there are $D_1, D_2$ such that $D = D_1 \uplus D_2$ and \\
\ooox $H, D_1 \models R(\setof{c}, \sigma)$ and $H, D_2 \models R(\setof{c}, \theta)$ \\
\oooo By induction $H, D_1 \models \sigma$ and $H, D_2 \models \theta$ \\
\oooo Hence $H, D \models \sigma \otimes \theta$ \\
\ooo Case $\phi = \cellneg{c'}{\cmd}$: \\
\oooo Since $R(\setof{c}, \cellneg{c'}{\cmd}) = \cellneg{c'}{\cmd}$, this case is immediate \\
\ooo Case $\phi = \cellpos{c'}{e'}{v'}{r'}$: \\
\oooo Consider whether $c \in r'$: \\
\oooo If $c \not\in r'$: \\
\ooooo Then $R(\setof{c}, \cellpos{c'}{e'}{v'}{r'}) = \cellpos{c'}{e'}{v'}{r'}$ and this case is immediate \\
\oooo If $c \in r'$: \\
\ooooo Since $\unready{c}{\cmd}$ is sound, we know that $c \not\in V_H$. \\ 
\ooooo We know $H, D \models R(\setof{c}, \cellpos{c'}{e'}{v'}{r'}) = \cellneg{c'}{e'}$ \\
\ooooo Hence $c' \not\in V_H$ \\
\ooooo But since $c \in r'$, we know $H, D \models \cellpos{c'}{e'}{v'}{r'}$ \\
\end{tabbedproof}



\item If $\unready{\phi \otimes \cellpos{c}{\cmd}{v}{r}}{c}$ then $\phi \otimes \cellpos{c}{\cmd}{v}{r} \dashv\vdash \phi \otimes \cellneg{c}{\cmd}$
\begin{tabbedproof}
\oo Assume $\unready{\phi \otimes \cellpos{c}{\cmd}{v}{r}}{c}$  and $H, D$ \\
\oo Now we want to show $H, D \models \phi \otimes \cellpos{c}{\cmd}{v}{r}$ if and only if \\
\ox $H, D \models \phi \otimes \cellneg{c}{\cmd}$ \\
\oo $\Rightarrow$: Assume $H, D \models \phi \otimes \cellpos{c}{\cmd}{v}{r}$ \\
\ooo So we have $D_1, D_2$ such that $D = D_1 \uplus D_2$ and \\
\oox $H, D_1 \models \phi$ and $H, D_2 \models \cellpos{c}{\cmd}{v}{r}$ \\
\ooo From the soundness of $\unready{\phi \otimes \cellpos{c}{\cmd}{v}{r}}{c}$, we know that $c \not \in V_H$ \\
\ooo Hence we know that $H, D_2 \models cellneg{c}{\cmd}$ \\
\ooo Hence we know that $H, D \models \phi \otimes \cellneg{c}{\cmd}$ \\
\oo $\Leftarrow$: Assume $H, D \models \phi \otimes \cellneg{c}{\cmd}$ \\
\ooo So we have $D_1, D_2$ such that $D = D_1 \uplus D_2$ and \\
\oox $H, D_1 \models \phi$ and $H, D_2 \models \cellpos{c}{\cmd}{v}{r}$ \\
\ooo We want to show $H, D \models \phi \otimes \cellpos{c}{\cmd}{v}{r}$ \\
\ooo By inversion on $\unready{\phi \otimes \cellpos{c}{\cmd}{v}{r}}{c}$ we know $\unready{\phi}{d}$ for some $d \in r$ \\
\ooo By soundness of $\unready{\phi}{d}$, we know that $d \not\in V_H$ \\
\ooo Hence we know that $H, D_2 \models \cellpos{c}{\cmd}{v}{r}$ \\
\ooo Hence we know that $H, D \models \phi \otimes \cellpos{c}{\cmd}{v}{r}$ \\
\end{tabbedproof}
\end{itemize}
\end{proof}



\subsection{The Abstract Semantics of Expressions}

Now we can finally define the abstract semantics of the code
expression monad, which we give in Figure~\ref{abs-semantics}.

As before, we describe the effect of an expression $\cmd$ with a Hoare
triple, prefixed with a quantification over all possible frames
$\psi$. Then we assert that from a state $G(\phi \otimes \psi)$,
running $\cmd$ will give us a state $G(\phi' \otimes R(u, \psi))$ ---
that is, we must update the frame with the ramification $u$. Note in
particular that the framing formulas may \emph{differ} in the pre- and
the post-conditions. For composite commands (such as $\bind \cmd\;f$) we
give their specifications as implications over the specifications of
their subcomponents.

In the specification \textsc{AUnit}, we give a specification for the
$\return v$ command, which simply returns its argument and neither
reads nor updates any cells or state. 

The \textsc{ABind} rule explains how sequential composition works for
an expression $\bind \cmd\;f$ --- as expected, we evaluate the first
monadic argument, and pass the result to the functional argument, and
evaluate that. The read and update sets are simply the union of the
two executions. Reading a cell comes in two variants, \textsc{AReady}
and \textsc{AUnready}. If a cell is ready (using the
$\ready{\phi}{c}{v}$ judgment), we simply return its memoized value
without any further computation, and do not need to use a ramification
to update the frame. On the other hand, if a cell is unready (from the
$\unready{\phi}{c}$ judgment), we need to evaluate its code body, and
then update the cell with its new value. So we need to know what the
evaluation of its body can do. Note that we have to apply the
ramification operator to $\phi'$ in the postcondition in the
consequence of \textsc{AUnready}, because the cell we are reading goes
from unready to ready itself. 

We can also read and write local state (with the \textsc{AGetRef} and
\textsc{ASetRef} specifications), which do not have any effect on the
cells. Finally, we have the \textsc{AConseq} rule, which gives us a
version of the rule of consequence from ordinary Hoare logic.

Note that the use of the ready and unready judgments, together with
the entailment relation on abstract heaps, means that we can reason
``syntactically'' about the behavior of cell-manipulating programs: we
do not need to know what the concrete model (or concrete
implementation) when we are proving the correctness of client
programs. We will see this in the next section, which illustrates that
this interface really does give us a genuinely modular way of
reasoning about dataflow networks.

\begin{prop}{(Soundness of Abstract Semantics)}

All of the rules of the abstract semantics in Figure~\ref{abs-semantics} are
provable within our specification logic. 
\end{prop}

At this point, we can now reason about the behavior of the imperative
notification library in terms of its action on the abstract heap. The
combination of quantification and ramification give us a domain-specific 
frame property, which allow us to modularly prove the correctness of programs 
that construct and produce notifications.


\begin{figure}
\mbox{}
\begin{mathpar}
\begin{array}{ll}  
\mbox{\textsc{AUnit}} & 
\forall \psi.\; \mspec{G(\psi)}
                     {\return v}
                     {a}{G(\psi) \land \exists z.\; a = (v, z) \land \mathit{set}(\emptyset, z, \emptyset)} 
\\[1em]

\mbox{\textsc{ABind}} & 
\forall \psi.\; \mspec{G(\phi \otimes \psi)}
                     {\cmd}
                     {a}{G(\phi' \otimes R(u', \psi)) \land \exists z.\;a = (v', r') \land \mathit{set}(r', z, r')} 
                \specand \\
&  \forall \psi.\; \mspec{G(\phi' \otimes \psi)}
                        {f\;v'}
                        {a}{G(\phi'' \otimes R(u'', \psi)) \land \exists z.\; a = (v'', z) \land \mathit{set}(r'', z, r'')}\\
&  \specimp \\
&  \begin{array}{ll}
     \forall \psi. & \left<G(\phi \otimes \psi)\right> \\
                   & \bind \cmd\;f \\
                   &  \left<a.\;G(\phi'' \otimes R(u' \cup u'', \psi)) \land \exists z.\;a = (v'', z) \land \mathit{set}(r' \cup r'', z, r' \cup r'')\right> \\
   \end{array}
\\[2em]

\mbox{\textsc{AReady}} & 
  \setof{\ready{\phi}{c}{v}} \specimp \\
& \forall \psi.\; \mspec{G(\phi \otimes \psi)}
                       {\readcell c}
                       {a}{G(\phi \otimes \psi) \land \exists z.\;a = (v, z) \land \mathit{set}(\setof{c}, z, \setof{c})} 
\\[1em]

\mbox{\textsc{AUnready}} & 
  \setof{\unready{\phi \otimes \celleither{c}{\cmd}}{c}} \specand \\
& \forall \psi.\; \mspec{G(\phi \otimes \psi)}
                       {\cmd}
                       {a}{G(\phi' \otimes R(u, \psi)) \land \exists z.\; a = (v, z) \land \mathit{set}(r, z, r)} \\
& \specimp \\
& \begin{array}{ll}
   \forall \psi. & \left<G(\phi \otimes \celleither{c}{\cmd} \otimes \psi)\right> \\
                 & \readcell c \\
                 & \left<a.\; 
                     \begin{array}{l}
                       G(R(\setof{c}, \phi') \otimes \cellpos{c}{\cmd}{v}{r} \otimes R(u \cup \setof{c}, \psi)) 
                        \;\land \\
                        \exists z.\;a = (v, z) \land \mathit{set}(\setof{c}, z, \setof{c})
                     \end{array}\right> 
  \end{array}
\\[3em]

\mbox{\textsc{AGetRef}}
& \begin{array}{ll}
    \forall \psi.\; 
    &  \left<G(\localref{r}{v} \otimes \psi)\right> \\
    &  \getref r \\
    &  \left<a.\; G(\localref{r}{v} \otimes \psi) 
                  \land  \exists z.\;a = (v,z) \land \mathit{set}(\emptyset, z, \emptyset)
       \right> \\
     \end{array}
\\[2em]

\mbox{\textsc{ASetRef}}
& \begin{array}{ll}
    \forall \psi.\; 
    &  \left<G(\localref{r}{v'} \otimes \psi)\right> \\
    &  \setref r\;v \\
    &  \left<a.\; G(\localref{r}{v} \otimes \psi) 
                  \land  \exists z.\;a = (v,z) \land \mathit{set}(\emptyset, z, \emptyset)
       \right> \\
     \end{array}
\\[2em]

% \mbox{\textsc{ANewCell}}
% & \forall \psi.\; \mspec{G(\psi)}{\newcell \cmd}{(c,z)}{G(\cellneg{c}{\cmd} \otimes \psi) \land \mathit{set}(\emptyset, z, \emptyset)}
% \\[1em]

\mbox{\textsc{AConseq}}
& \setof{\phi' \vdash \phi} \specand \setof{\theta \vdash \theta'} \specand \\
& \forall \psi.\; \mspec{G(\phi \otimes \psi) \land P}{\cmd}{(x,z)}{G(\theta\otimes R(u, \psi)) \land Q} \\
& \specimp \\
& \forall \psi.\; \mspec{G(\phi' \otimes \psi) \land P}{\cmd}{(x,z)}{G(\theta'\otimes R(u, \psi)) \land Q} \specand \\

\end{array}
\end{mathpar}
\caption{Abstract Semantics of Notifications}
\label{abs-semantics}
\end{figure}


\subsection{Proving the correctness of \textsc{AUnit}}
\begin{prop*}{(The \textsc{AUnit} specification is sound)}
The following triple is derivable:
\begin{displaymath}
\forall \psi.\; \mspec{G(\psi)}
                     {\return v}
                     {a}{G(\psi) \land \exists z.\; a = (v, z) \land \mathit{set}(\emptyset, z, \emptyset)} 
\end{displaymath}
\end{prop*}

\begin{proof}
\begin{tabbedproof}
\oo Assume we have $\psi$ and a prestate $G(\psi)$ \\
\oo Now consider the body of $\return v$ \\
\oo $[(v, \mathsf{emptyset})]$ \\
\oo Hence we know that $\exists z.\; a = (v, z) \land \mathit{set}(\emptyset, z, \emptyset)$ \\
\ox and $G(\psi)$ \\
\end{tabbedproof}
\end{proof}

\subsection{Proving the correctness of \textsc{ABind}}

\begin{prop*}{(The \textsc{ABind} specification is sound)}
The following specification is derivable:
\begin{displaymath}
\begin{array}{l}
\forall \psi.\; \mspec{G(\phi \otimes \psi)}
                     {\cmd}
                     {a}{G(\phi' \otimes R(u', \psi)) \land \exists z.\;a = (v', r') \land \mathit{set}(r', z, r')} 
                \specand \\
\forall \psi.\; \mspec{G(\phi' \otimes \psi)}
                        {f\;v'}
                        {a}{G(\phi'' \otimes R(u'', \psi)) \land \exists z.\; a = (v'', z) \land \mathit{set}(r'', z, r'')}\\
\specimp \\
\begin{array}{ll}
     \forall \psi. & \left<G(\phi \otimes \psi)\right> \\
                   & \bind \cmd\;f \\
                   &  \left<a.\;G(\phi'' \otimes R(u' \cup u'', \psi)) \land \exists z.\;a = (v'', z) \land \mathit{set}(r' \cup r'', z, r' \cup r'')\right> \\
   \end{array}
\end{array}
\end{displaymath}
\end{prop*}

\begin{proof}
\begin{tabbedproof}
\oo Assume $\forall \psi.\; \mspec{G(\phi \otimes \psi)}
                     {\cmd}
                     {a}{G(\phi' \otimes R(u', \psi)) \land \exists z.\;a = (v', r') \land \mathit{set}(r', z, r')}$ \\
\oo Assume $\forall \psi.\; \mspec{G(\phi' \otimes \psi)}
                        {f\;v'}
                        {a}{G(\phi'' \otimes R(u'', \psi)) \land \exists z.\; a = (v'', z) \land \mathit{set}(r'', z, r'')}$ \\
\oo Assume we have $\psi$, and a prestate $G(\phi \otimes \psi)$ \\
\ooo Now consider the body of $\bind$ \\
\ooo $[\letv{(v, z_1)}{\cmd}{}$ \\
\ooo So we know $G(\phi' \otimes R(u', \psi)) \land v = v' \land \mathit{set}(r', z_1, r')$ \\
\ooo Simplifying $v$ away, we continue with \\
\ooo $\letv{(v, z_2)}{f\;v'}{}$ \\
\ooo Now we know $G(\phi'' \otimes R(u'', R(u', \psi))) \land v = v'' \land \mathit{set}(r', z_1, r') \land \mathit{set}(r'', z_2, r'')$ \\
\ooo Hence with $z = \mathsf{union}\;z_1\;z_2$ we know $\mathit{set}(r' \cup r'', z, r' \cup r'')$ \\
\ooo $(v, \mathsf{union}\;z_1\;z_2)]$ \\
\ooo Hence $G(\phi'' \otimes R(u' \cup u'', \psi)) \land \exists z.\; a = (v'', z) \land \mathit{set}(r' \cup r'', z, r' \cup r'')$ 
\end{tabbedproof}
\end{proof}

\subsection{Proving the correctness of \textsc{AReady}}

In this subsection, we will prove the following proposition: 

\begin{prop*}{(The \textsc{AReady} specification is sound)}
The following specification holds: 
\begin{displaymath}
\begin{array}{l}
\setof{\ready{\phi}{c}{v}} \specimp \\
\;\;\;\forall \psi.\; \mspec{G(\phi \otimes \psi)}
                       {\readcell c}
                       {a}{G(\phi \otimes \psi) \land \exists z.\;a = (v, z) \land \mathit{set}(\setof{c}, z, \setof{c})} 
\end{array}
\end{displaymath}
\end{prop*}

\begin{proof}
\begin{tabbedproof}
\oo Assume $\ready{\phi}{c}{v}$ and a $\psi$ \\
\ooo Now assume we have a precondition $G(\phi \otimes \psi)$ \\
\ooo So there is an $H = (D,h)$ such that $\mathit{Inv}(H, \phi \otimes \psi)$ holds \\
\ooo So we know that $H, D \models \phi \otimes \psi$ and $\mathit{heap}(H)$ \\
\ooo So we know that $c.\mathit{value} \pointsto \mathit{value}(h(c))$ \\
\ooo From soundness of $\ready{\phi}{c}{v}$, we know that $\mathit{value}(h(c)) = \Some(v)$ \\
\ooo $[\letv{o}{[!c.value]}{}$ \\
\ooo So we know that $o = \Some(v)$ \\
\ooo Now we can simplify the case statement and continue \\
\ooo $(v, \mathsf{singleton}\;c)]$ \\
\ooo So we know that $\exists z.\;a = (v, z) \land \mathit{set}(\setof{c}, z, \setof{c})$ \\
\oox and $\mathit{Inv}(H, \phi \otimes \psi)$ holds \\
\ooo So $G(H, \phi \otimes \psi)$ and $\exists z.\;a = (v, z) \land \mathit{set}(\setof{c}, z, \setof{c})$ hold
\end{tabbedproof}
\end{proof}

\subsection{Proving the correctness of \textsc{AUnready}}

In this subsection, we will prove the following proposition:

\begin{prop*}{(Soundness of the \text{AUnready} specification)}
The following triple holds:
\begin{displaymath}
\begin{array}{l}
\setof{\unready{\phi \otimes \celleither{c}{\cmd}}{c}} \specand \\
\forall \psi.\; \mspec{G(\phi \otimes \psi)}
                     {\cmd}
                     {a}{G(\phi' \otimes R(u, \psi)) \land \exists z.\; a = (v, z) \land \mathit{set}(r, z, r)} \\
\specimp \\
\begin{array}{ll}
 \forall \psi. & \left<G(\phi \otimes \celleither{c}{\cmd} \otimes \psi)\right> \\
               & \readcell c \\
               & \left<a.\; 
                   \begin{array}{l}
                     G(R(\setof{c}, \phi') \otimes \cellpos{c}{\cmd}{v}{r} \otimes R(u \cup \setof{c}, \psi)) 
                      \;\land \\
                      \exists z.\;a = (v, z) \land \mathit{set}(\setof{c}, z, \setof{c})
                   \end{array}\right> 
\end{array}
\end{array}
\end{displaymath}
\end{prop*}

This follows the standard pattern of (1) proving that the update
preserves the relevant order structure, (2) proving the new abstract
formula continues to model the updated abstract heap, and finally (3)
showing the code actually implements this update. 


\begin{lemma}{(Reading an Unready Cell Preserves Order Structure)}
Suppose that $H = (D,h)$ and $c \in D - V_H$. Then for any subset $r
\subseteq V_H$ and value $v$, define $h'$ so that
\begin{itemize}
\item $\mathit{code} \circ h' = \mathit{code} \circ h$ 
\item $\mathit{value} \circ h' = \semfun{c'}{\IfThenElse{c' = c}{\Some(v)}{\mathit{value}(h(c'))}}$ 
\item $\mathit{reads} \circ h' = \semfun{c'}{\IfThenElse{c' = c}{r}{\mathit{reads}(h(c'))}}$ 
\item $\mathit{obs} \circ h' = \semfun{c'}{\IfThenElse{c' \in r}{\mathit{obs}(h(c')) \cup \setof{c}}{\mathit{obs}(h(c'))}}$ 
\item $\mathit{identity} \circ h' = \mathit{identity} \circ h$
\end{itemize}
\noindent Then it is the case that $H' = (D,h')$ satisfies the five properties below. 
\begin{enumerate}
\item $V_{H'} = V_H \cup \setof{c}$
\item $R^+_{H'}$ is a strict partial order 
\item $R_{H'} \subseteq V_{H'} \times V_{H'}$
\item $R^\dagger_{H'} = O_{H'}$ 
\item $\mathit{unique}(H') = \mathit{unique}(H)$
\end{enumerate}
\end{lemma}

\begin{proof}
\begin{enumerate}
\item $V_{H'} = V_H \cup \setof{c}$ 
  \begin{tabbedproof}
    \oo So we want to show that $a \in V_{H'}$ iff $a \in V_H \cup \setof{c}$ \\
    \oo Assume that $a \in V_{H'}$ \\
    \ooo So we know that $\mathit{value}(h'(a)) = \Some(u)$ for some $u$ \\
    \ooo Now consider whether $a$ is $c$ \\
    \ooo Suppose $a = c$: \\
    \oooo Then $a \in \setof{c}$, and hence $a \in V_H \cup \setof{c}$ \\
    \ooo Suppose $a \not= c$: \\
    \oooo Then we know that $\mathit{value}(h'(a)) = \mathit{value}(h(a))$ \\
    \oooo Since $\mathit{value}(h'(a)) = \Some(u)$, we know $a \in V_H$ \\
    \oooo Hence $a \in V_H \cup \setof{c}$ \\
    \oo Assume that $a \in V_H \cup \setof{c}$ \\
    \ooo Now consider whether $a$ is $c$ \\
    \ooo Suppose $a = c$: \\
    \oooo Then $\mathit{value}(h'(a)) = \Some(v)$ \\
    \oooo Therefore $a \in V_{H'}$ \\
    \ooo Suppose $a \not= c$: \\
    \oooo Then $a \in V_H$ \\
    \oooo Also, $\mathit{value}(h'(a)) = \mathit{value}(h(a))$ \\
    \oooo Therefore there is a $u$ such that $\mathit{value}(h'(a)) = \Some(u)$ \\
    \oooo Hence $a \in V_{H'}$ \\
  \end{tabbedproof}

\item $R^+_{H'}$ is a strict partial order 
  \begin{tabbedproof}
    \oo This follows if $R^+_{H'}$ is a transitive, irreflexive relation \\
    \oo Since it is a transitive closure, it is a transitive relation \\
    \oo So we want to show there is no $a \in D$ such that $(a,a) \in R^+_{H'}$ \\
    \oo Consider an arbitrary $(a_0, b) \in R^+_{H'}$ \\
    \oo Hence there is a sequence $a_0, \ldots, a_{k+1}$ such that $b = a_{k+1}$ \\
    \ox and for each $i \leq k$, $a_{i+1} \in \mathit{reads}(h'(a_i))$ \\
    \oo Now consider whether $c$ is equal to any of the $a_i$ \\
    \oo Suppose $a_i \not= c$ for all $0 \leq i \leq k+1$: \\
    \ooo Then in this case, it follows that $\mathit{reads}(h'(a_i)) = \mathit{reads}(h(a_i))$ for all $i$\\
    \ooo Then $(a_0, a_{k+1}) \in R^+_H$ \\
    \ooo Then $a_0 \not= a_{k+1}$ since $R^+_H$ is irreflexive \\
    \oo Suppose $a_i = c$ for some $0 \leq i \leq k+1$: \\
    \ooo Note $\mathit{reads}(h'(a)) \subseteq V_H$ for all $a$  \\
    \ooo Since $c \not\in V_H$, it follows $c \not\in \mathit{reads}(h'(a))$ for any $a_i$ \\
    \ooo Hence $i = 0$, and $c \not= a_j$ for any $j > 0$ \\
    \ooo Then $c \not= a_{k+1}$
  \end{tabbedproof}

\item $R_{H'} \subseteq V_{H'} \times V_{H'}$ 
  \begin{tabbedproof}
    \oo We want to show that for all $(a,b)$, if $b \in \mathit{reads}(h'(a))$, then $a \in V_{H'}$ and $b \in V_{H'}$ \\
    \oo Assume $b \in \mathit{reads}(h'(a))$ \\
    \ooo Now consider whether $a = c$  \\
    \ooo Suppose $a = c$: \\
    \oooo Then $a \in V_{H'}$ \\
    \oooo Since $\mathit{reads}(h'(c)) = r$, we know $r \subseteq V_H$ \\
    \oooo Since $V_{H} \subseteq V_{H'}$, it follows $b \in V_{H'}$ \\
    \ooo Suppose $a \not= c$: \\
    \oooo Then $\mathit{reads}(h'(a)) = \mathit{reads}(h(a))$ \\
    \oooo Then $b \in \mathit{reads}(h(a))$, and so $(a,b) \in R_H$ \\
    \oooo Since $R_H \subseteq V_H \times V_H$, we know $a \in V_H$ and $b \in V_H$ \\
    \oooo Since $V_{H} \subseteq V_{H'}$, we know $a \in V_{H'}$ and $b \in V_{H'}$ \\
  \end{tabbedproof}

\item $R^\dagger_{H'} = O_{H'}$ 
  \begin{tabbedproof}
    \oo We want to show that for all $(a,b)$, we have $(a,b) \in R^\dagger_{H'}$ iff $(a,b) \in O_{H'}$ \\
    \oo Assume $(a,b) \in R^\dagger_{H'}$ \\
    \ooo Then $(b, a) \in R_{H'}$ \\
    \ooo So $a \in \mathit{reads}(h'(b))$ \\
    \ooo We want to show that $b \in \mathit{obs}(h'(a))$ \\
    \ooo Consider whether $b = c$ \\
    \ooo Suppose $b = c$: \\
    \oooo Then $a \in r$, since $\mathit{reads}(h'(b)) = r$ \\
    \oooo Therefore $\mathit{obs}(h'(a)) = \mathit{obs}(h(a)) \cup \setof{c}$ \\
    \oooo Hence $b \in \mathit{obs}(h'(a))$ \\
    \ooo Suppose $b \not= c$: \\
    \oooo Then $\mathit{reads}(h'(b)) = \mathit{reads}(h(b))$ \\
    \oooo So $a \in \mathit{reads}(h(b))$ \\
    \oooo Hence $b \in \mathit{obs}(h(a))$ \\
    \oooo Since $\mathit{obs}(h(x)) \subseteq \mathit{obs}(h(x))$ for all $x$, $b \in \mathit{obs}(h'(a))$ \\
    \oo Assume $(a,b) \in O_{H'}$ \\
    \ooo Then $b \in \mathit{obs}(h'(a))$ \\
    \ooo We want to show that $a \in \mathit{reads}(h'(b))$ \\
    \ooo Consider whether $b = c$ \\
    \ooo Suppose $b = c$: \\
    \oooo Now consider whether $a \in r$ \\
    \oooo Suppose $a \in r$: \\
    \ooooo Then by definition $a \in \mathit{reads}(h'(b))$ \\
    \oooo Suppose $a \not\in r$: \\
    \ooooo Then $\mathit{obs}(h'(a)) = \mathit{obs}(h(a)) = \emptyset$ \\
    \ooooo This is a contradiction, since we assumed $b \in \mathit{obs}(h'(a))$ \\
    \ooo Suppose $b \not= c$: \\
    \oooo Consider whether $a = c$: \\
    \oooo Suppose $a = c$: \\
    \ooooo This case is impossible since $c \not\in V_H$ and $\mathit{obs}(h'(x)) = \emptyset$ for any $x \not\in V_H$ \\
    \oooo Suppose $a \not= c$: \\
    \ooooo Then it follows that $b \in \mathit{obs}(h(a))$ \\
    \ooooo Then it follows that $a \in \mathit{reads}(h(b))$ \\
    \ooooo Therefore $a \in \mathit{reads}(h'(b))$ 
  \end{tabbedproof}

\item $\mathit{unique}(H') = \mathit{unique}(H)$

This is immediate since $\mathit{identity} \circ h' = \mathit{identity} \circ h$.
\end{enumerate}
\end{proof}

\begin{lemma}{(Reading an Unready Cell, Semantically)}
Suppose that $H = (D,h)$ and $H, D' \models \phi \otimes
\celleither{c}{\cmd}$ and $c \in D' - V_H$. Then for any subset $r
\subseteq V_H$ and value $v$, define $h'$ so that
\begin{itemize}
\item $\mathit{code} \circ h' = \mathit{code} \circ h$ 
\item $\mathit{value} \circ h' = \semfun{c'}{\IfThenElse{c' = c}{\Some(v)}{\mathit{value}(h(c'))}}$ 
\item $\mathit{reads} \circ h' = \semfun{c'}{\IfThenElse{c' = c}{r}{\mathit{reads}(h(c'))}}$ 
\item $\mathit{obs} \circ h' = \semfun{c'}{\IfThenElse{c' \in r}{\mathit{obs}(h(c')) \cup \setof{c}}{\mathit{obs}(h(c'))}}$ 
\item $\mathit{identity} \circ h' = \mathit{identity} \circ h$
\end{itemize}
Then it follows that for $H' = (D,h')$, we have $H', D' \models R(\setof{c}, \phi) \otimes \cellpos{c}{\cmd}{v}{r}$.
\end{lemma}

\begin{proof}
  \begin{tabbedproof}
    \oo Assume $H, D' \models \phi \otimes \celleither{c}{\cmd}$ and $c \in D' - V_H$ \\
    \ooo Then there are $D_1, D_2$ with $D' = D_1 \uplus D_2$ and $H, D_1 \models \phi$ and $H, D_2 \models \celleither{c}{\cmd}$ \\
    \ooo Then $c \in D_2$ and $c \not \in D_1$, and $\mathit{value}(h(c)) = \None$ \\
    \ooo By assumption, $\mathit{value}(h'(c)) = \Some(v)$ and $\mathit{reads}(h'(c)) = r$ \\
    \ooo Hence $H', D_2 \models \cellpos{c}{\cmd}{v}{r}$ \\
    \ooo Now proceed by induction on $\phi$, letting the support $D_1$ be a parameter \\
    \ooo Case $\phi = I$: \\
    \oooo By definition $H', D_1 \models I$ \\
    \ooo Case $\phi = \psi \otimes \theta$: \\
    \oooo So we know there are $D_3, D_4$ such that $D_1 = D_3 \uplus D_4$ and \\
    \oooox $H, D_3 \models \psi$ and \\
    \oooox $H, D_4 \models \theta$ \\
    \oooo By induction, we know that \\
    \oooox $H', D_3 \models R(\setof{c}, \psi)$ and \\
    \oooox $H', D_4 \models R(\setof{c}, \theta)$ \\
    \oooo Hence $H', D_1 \models R(\setof{c}, \psi) \otimes R(\setof{c}, \theta)$ \\
    \oooo Hence $H', D_1 \models R(\setof{c}, \psi \otimes \theta)$ \\

    \ooo Case $\phi = \cellneg{c'}{e'}$: \\
    \oooo So we know that $c' \not\in V_H$ and $c' \not= c$ and $\mathit{code}(h(c')) = e'$ and $c' \in D_1$\\
    \oooo Hence $c' \not\in V_{H'}$ \\
    \oooo Hence $H', D_1 \models \cellneg{c'}{e'}$ \\
    \oooo Hence $H', D_1 \models R(\setof{c}, \cellneg{c'}{e'})$ \\

    \ooo Case $\phi = \cellpos{c'}{e'}{v'}{r'}$: \\
    \oooo Now consider whether $r' \subseteq V_H$ \\
    \oooo Suppose $r' \subseteq V_{H}$: \\
    \ooooo So $\mathit{value}(h(c')) = \Some(v')$ and $\mathit{reads}(h(c')) = r'$ and
           $\mathit{code}(h(c')) = e'$ and $a \in D$ \\
    \ooooo Since $c \not \in D_1$, we know $c' \not= c$ \\
    \ooooo Therefore $\mathit{value}(h'(c')) = \Some(v')$ \\
    \ooooo Therefore $\mathit{reads}(h'(c')) = r'$ \\
    \ooooo Since $V_{H'} \supseteq V_H$, it follows that $r' \cap V_{H'} = r'$ \\
    \ooooo Therefore $H', D_1 \models \cellpos{c'}{e'}{v'}{r'}$ \\
    \ooooo Furthermore, since $r' \subseteq V_H$, we know $c \not\in r'$ \\
    \ooooo Hence $R(\setof{c}, \cellpos{c'}{e'}{v'}{r'}) = \cellpos{c'}{e'}{v'}{r'}$ \\
    \oooo Suppose $r' \not\subseteq V_{H}$: \\
    \ooooo Then $c' \not\in V_H$ and $c' \not= c$\\
    \ooooo So $c' \not \in V_{H'}$ \\
    \ooooo Now, consider whether $c \in r'$: \\
    \ooooo Suppose $c \in r'$: \\
    \oooooo Since $c' \not \in V_{H'}$, we know $H', D_1 \models \cellneg{c'}{e'}$ \\
    \oooooo Hence $H', D_1 \models R(\setof{c}, \cellpos{c'}{e'}{v'}{r'})$ \\
    \ooooo Suppose $c \not\in r'$: \\
    \oooooo Then since $r' \not\subseteq V_H$, there is $x \in r'$ such that $x \not \in V_H$ and $x \not= c$ \\
    \oooooo Therefore $r' \not\subseteq V_{H'}$ \\
    \oooooo Therefore $H', D_1 \models \cellpos{c'}{e'}{v'}{r'}$ \\
    \oooooo Therefore $H', D_1 \models R(\setof{c}, \cellpos{c'}{e'}{v'}{r'})$ 
  \end{tabbedproof}
\end{proof}

\begin{prop*}{(Soundness of the \text{AUnready} specification)}
The following triple holds:
\begin{displaymath}
\begin{array}{l}
\setof{\unready{\phi \otimes \celleither{c}{\cmd}}{c}} \specand \\
\forall \psi.\; \mspec{G(\phi \otimes \psi)}
                     {\cmd}
                     {a}{G(\phi' \otimes R(u, \psi)) \land \exists z.\; a = (v, z) \land \mathit{set}(r, z, r)} \\
\specimp \\
\begin{array}{ll}
 \forall \psi. & \left<G(\phi \otimes \celleither{c}{\cmd} \otimes \psi)\right> \\
               & \readcell c \\
               & \left<a.\; 
                   \begin{array}{l}
                     G(R(\setof{c}, \phi') \otimes \cellpos{c}{\cmd}{v}{r} \otimes R(u \cup \setof{c}, \psi)) 
                      \;\land \\
                      \exists z.\;a = (v, z) \land \mathit{set}(\setof{c}, z, \setof{c})
                   \end{array}\right> 
\end{array}
\end{array}
\end{displaymath}
\end{prop*}

\begin{proof}
\begin{tabbedproof}
\oo Assume $\unready{\phi \otimes \celleither{c}{\cmd}}{c}$ \\
\oo Assume $\forall \psi.\; 
       \mspec{G(\phi \otimes \psi)}
            {\cmd}
            {a}{G(\phi' \otimes R(u, \psi)) \land \exists z.\; a = (v, z) \land \mathit{set}(r, z, r)} $ \\
\oo Assume we have some $\psi$ \\
\ooo Assume we are in a prestate $G(\phi \otimes \psi)$ \\
\ooo So there is an $H = (D,h)$ such that $\mathit{Inv}(H, \phi \otimes \psi)$ holds \\
\ooo Hence we know that \\
\ooox $R^\dagger_H = O_H$ \\
\ooox $R^+_H$ strict partial order \\
\ooox $R_H \subseteq V_H \times V_H$ \\
\ooox $\mathit{unique}(H)$ \\
\ooox $H, D \models \phi \otimes \psi$ \\
\ooox $\mathit{heap}(H)$ \\
\ooo Since $\mathsf{unready}$ is sound, we know that $\mathit{value}(h'(c)) = \None$ \\
\ooo Hence we also know that $H, D \models \phi \otimes \cellneg{c}{\cmd} \otimes \psi$ \\
\ooo We also know that $c.\mathit{value} \pointsto \None$ \\
\ooo $[\letv{o}{!a.\mathit{value}}{}$ \\
\ooo Now we additionally know that $o = \None$ \\
\ooo Hence we can simplify the remaining program to eliminate the case \\
\ooo $\letv{\cmd}{!a.code}{}$ \\
\ooo From the definition of $\mathit{heap}$, we know that $\cmd = \mathit{code}(h(c))$ \\
\ooo From the fact that $H, D \models \phi \otimes \cellneg{c}{\cmd} \otimes \psi$, \\
\oox we know $\mathit{code}(h(c)) = \cmd$ \\
\ooo $\letv{(v',z)}{\cmd}{}$ \\
\ooo Using our assumption about $\cmd$, and setting $\psi$ to $\cellneg{c}{\cmd} \otimes \psi$, \\
\ooo our poststate is $G(\phi' \otimes R(u, \cellneg{c}{\cmd} \otimes \psi)) \land v = v' \land \mathit{set}(r, z, r)$ \\
\ooo Simplifying away the $v'$, we get $\mathit{set}(r, z, r)$ and $G(\phi' \otimes \cellneg{c}{\cmd} \otimes R(u, \psi))$ \\
\ooo So there is a shadowing $H = (D,h)$ such that $\mathit{Inv}((D, h), \phi' \otimes R(u, \cellneg{c}{\cmd} \otimes \psi))$ \\
\ooo Now take $H' = (D,h')$ and $h'$ as in the semantic update lemma \\
\ooo We know $H', D \models R(\setof{a}, \phi' \otimes \cellpos{c}{\cmd}{v}{r} \otimes R(u, \psi))$\\
\ooo So we know $H', D \models R(\setof{a}, \phi') \otimes \cellpos{c}{\cmd}{v}{r} \otimes R(u \cup \setof{a}, \psi))$ \\
\ooo Furthermore we know that this $H'$ preserves the relevant order structure \\
\ooo $\letv{\_}{[a.value := v']}{}$ \\
\ooo $\letv{\_}{[a.reads := z]}{}$ \\
\ooo $\letv{\_}{\mathsf{iterset}\;(\mathsf{add\_observer}\;a)\;z}{}$ \\
\ooo These commands establish $\mathit{heap}(H')$ \\
\ooo The last requires an induction based on the size of $r$ \\ 
\oox and the unfolding axiom in the cellset interface \\
\ooo Now we know $\mathit{Inv}(H', R(\setof{a}, \phi') \otimes \cellpos{c}{\cmd}{v}{r} \otimes R(u \cup \setof{a}, \psi))$ \\
\ooo So we know $G(R(\setof{a}, \phi') \otimes \cellpos{c}{\cmd}{v}{r} \otimes R(u \cup \setof{a}, \psi))$ \\
\ooo $(v, z)]$ \\
\ooo So $\exists z.\;a = (v,z) \land \mathit{set}(\setof{c}, z, \setof{c})$ 
\end{tabbedproof}
\end{proof}

\subsection{Proving the correctness of \textsc{AConseq}}

\begin{prop*}{(The correctness of the \textsc{AConseq} specification)}
The following specification is valid:
\begin{displaymath}
\begin{array}{l}
\setof{\phi' \vdash \phi} \specand \setof{\theta \vdash \theta'} \specand \\
\forall \psi.\; \mspec{G(\phi \otimes \psi) \land P}{\cmd}{(x,z)}{G(\theta\otimes R(u, \psi)) \land Q} \\
\specimp \\
\forall \psi.\; \mspec{G(\phi' \otimes \psi) \land P}{\cmd}{(x,z)}{G(\theta'\otimes R(u, \psi)) \land Q}  \\
\end{array}
\end{displaymath}
\end{prop*}

\begin{proof}
\begin{tabbedproof}
\oo Assume we know $\phi' \vdash \phi$ and $\psi \vdash \psi'$ \\
\oo Furthermore assume $\forall \psi.\; \spec{G(\phi \otimes \psi) \land P}{\cmd}{(x,z)}{G(\theta\otimes R(u, \psi)) \land Q}$ \\
\ooo Now assume $\psi$, and that we are in a prestate $G(\phi' \otimes \psi) \land P$ \\
\ooo So we know there is an $H$ such that $\mathit{Inv}(H, \phi' \otimes \psi)$ \\
\ooo Since $\phi' \models \phi$, we know that $H, D \models \phi \otimes \psi$ \\
\ooo Hiding the existential again, we know that $G(\phi \otimes \psi) \land P$ \\
\ooo $\cmd$
\ooo Now by assumption we know $G(\theta \otimes R(u, \psi)) \land Q$ \\
\ooo So we know there is an $H$ such that $\mathit{Inv}(H, \theta \otimes \psi)$ \\
\ooo Since $\theta \vdash \theta'$, we know that $\mathit{Inv}(H, \theta' \otimes \psi)$ \\
\ooo Hiding the existential, we get the poststate $G(\theta'\otimes R(u, \psi)) \land Q$ \\
\end{tabbedproof}
\end{proof}

\subsection{Proving the correctness of \textsc{AGetRef}}
\begin{prop*}
The following specification is valid: 
\begin{displaymath}
\begin{array}{ll}
    \forall \psi.\; 
    &  \left<G(\localref{r}{v} \otimes \psi)\right> \\
    &  \getref r \\
    &  \left<a.\; G(\localref{r}{v} \otimes \psi) 
                  \land  \exists z.\;a = (v,z) \land \mathit{set}(\emptyset, z, \emptyset)
       \right> \\
     \end{array}
\end{displaymath}
\end{prop*}
\begin{proof}
\begin{tabbedproof}
\oo Assume we are in a prestate $G(\localref{r}{v} \otimes \psi)$ \\
\oo So there is an $H = (D,h)$ such that $Inv(H, \localref{r}{v} \otimes \psi)$ holds \\
\oo The spatial part of $Inv$ is $heap(H) * \mathit{refs}(\localref{r}{v} \otimes \psi)$ \\
\oo This is the same as $heap(H) * {r}\pointsto{v} * \mathit{refs}(\psi)$ \\
\oo $[\letv{x}{[!r]}{}$ \\
\oo This leaves us in $Inv(H, \localref{r}{v} \otimes \psi) \land x = v$ \\
\oo $\pair{v}{\ctext{emptyset}}]$ \\
\oo $G(\localref{r}{v} \otimes \psi) \land a = \pair{v}{\ctext{emptyset}}$ \\
\end{tabbedproof}
\end{proof}

\subsection{Proving the correctness of \textsc{ASetRef}}

\begin{lemma}{(Preservation of Satisfiability under Local Store Changes)}
We need $\satisfies{H}{\localref{r}{v} \otimes \psi}$ if and only if $\satisfies{H}{\localref{r}{v'} \otimes \psi}$ holds.
\end{lemma}
\begin{proof}
This follows by an easy induction on $\psi$, since $sat(H, D, \localref{r}{v}) = \top$. 
\end{proof}

\ \\

\begin{prop*}
The following specification is valid:
\begin{displaymath}
\begin{array}{ll}
\forall \psi.\; 
  &    \left<G(\localref{r}{v'} \otimes \psi)\right> \\
  &    \setref r\;v \\
  &    \left<a.\; G(\localref{r}{v} \otimes \psi) 
                  \land  \exists z.\;a = (v,z) \land \mathit{set}(\emptyset, z, \emptyset)
       \right> 
\end{array}
\end{displaymath}
\end{prop*}

\begin{proof}
\begin{tabbedproof}
\oo Assume we are in a prestate $G(\localref{r}{v'} \otimes \psi)$ \\
\oo Then there is an $H$ such that $R^\dagger_H$ and  $R^+_H$ is a strict partial order, \\
\oo and $R_H \subseteq V_H \times V_H$ and $\mathit{unique}(H)$ and $\satisfies{H}{\localref{r}{v'} \otimes \phi}$ \\
\oo and spatially $\mathit{heap}(H) * \mathit{refs}(\localref{r}{v'} \otimes \phi)$\\
\oo The spatial part is equivalent to $\mathit{heap}(H) * r \pointsto v' * \mathit{refs}(\phi)$ \\
\oo $[\letv{()}{[r := v]}{}$ \\
\oo Hence $\mathit{heap}(H) * r \pointsto v * \mathit{refs}(\phi)$ \\
\oo Hence $\mathit{heap}(H) * \mathit{refs}(\localref{r}{v} \otimes \phi)$ \\
\oo By lemma, $\satisfies{H}{\localref{r}{v} \otimes \phi}$ \\
\oo Hence $Inv(H, \localref{r}{v} \otimes \phi)$ \\
\oo Hence $G(\localref{r}{v} \otimes \phi)$ \\
\oo $\pair{\unit}{\ctext{emptyset}}]$ \\
\oo Hence $G(\localref{r}{v} \otimes \phi) \land a = \pair{\unit}{\ctext{emptyset}}$ \\
\end{tabbedproof}
\end{proof}

\section{Implementing Functional Reactive Programming}

In this section, we will see how to verify an imperative
implementation of a simple synchronous functional reactive programming
system.

\subsection{Specifying Functional Reactive Programs}

\emph{Functional Reactive Programming}~\cite{frp} is a style of
writing interactive programs based on the idea of \emph{stream
  transducers}.  The idea is to model a time-varying input signal of
type $A$ as an infinite stream of $A$'s, and to model an interactive
system as a function that takes a stream of inputs $\stream{A}$ and
yields a stream of outputs $\stream{B}$. Note that a stream can be
viewed either as an infinite sequence of values, or isomorphically as
a function from natural numbers to values (i.e., a function from times
to values). In our discussion, we'll switch freely between these two
views, using the most convenient viewpoint.\footnote{Given an infinite stream $vs$, we will use use $take\;n\;vs$ to denote
the finite list consisting of the first $n$ elements of the stream
$vs$. Correspondingly, $drop\;n\;vs$ is the infinite stream with $vs$
with its first $n$ elements cut off. With a function $f$, $map\;f\;vs$
maps $f$ over the elements of $vs$, and given another infinite stream
$us$, the call $zip\;us\;vs$ returns the infinite stream of pairs of
elements of $us$ and $vs$. If $v$ is an element, $v \cdot vs$ will 
denote consing $v$ to the front of $vs$, and if $xs$ is a finite list, then
$xs \cdot vs$ will denote appending the finite sequence $xs$ to the
front of $vs$. Finally, we will write $vs_n$ to denote the $n$-th element
of the stream $vs$.}

However, not all functions $\stream{A} \to \stream{B}$ are legitimate
stream transducers; we need to restrict our attention to \emph{causal}
stream transducers. A transducer is causal if we can compute the first
$n$ elements of the output after having read at most $n$ elements of
the input. 

\begin{tabbing}
$causal(f : \stream{A} \to \stream{B}) \equiv$ \\
\;\;\= $\exists \hat{f} : \listtype{A} \to \listtype{B}.\;\forall as:\stream{A}, n:\N.$ \\
    \> \;\;$take\;n\;(f\;as) = \hat{f}\;(take\;n\;as)$ 
\end{tabbing}

If we are given a causal transducer $p$, we will write $\hat{p}$ to
indicate the corresponding list function which computes its finite
approximations. Then, we can define a family of combinators acting on
causal transducers, which we give in Figure~\ref{transducer-semantics}.

\begin{figure}
\mbox{}
\begin{tabbing}
$\ST{A}{B} = \comprehend{f \in \stream{A} \to \stream{B}}{causal(f)}$\\[1em]

$lift : (A \to B) \to \ST{A}{B}$ \\
$lift\;f\;as = map\;f\;as$ \\[1em]

$seq  : \ST{A}{B} \to \ST{B}{C} \to \ST{A}{C}$ \\
$seq\;p\;q = q \circ p$ \\[1em]

$par  : \ST{A}{B} \to \ST{C}{D} \to \ST{A \times C}{B \times D}$ \\
$par\; p\;q\;abs = zip\; (p\;(map\;\pi_1\;abs))\;(q\;(map\;\pi_2\;abs))$\\[1em]

$switch : \N \to \ST{A}{B} \to \ST{A}{B} \to \ST{A}{B}$ \\
$switch\;k\;p\;q = \semfun{as}{(take\;k\;(p\;as))\cdot(q\;(drop\;k\;as))}$ \\[1em]

$loop : A \to \ST{A\times B}{A \times C} \to \ST{B}{C}$ \\
$loop\;a_0\;p = (map\;\pi_2) \circ (cycle\;a_0\;p)$ \\[1em]

$cycle : A \to \ST{A\times B}{A \times C} \to \ST{B}{A \times C}$ \\
$cycle\;a_0\;p = \lambda bs.\;\lambda n.\;last(gen\;a_0\;p\;v\;n)$ \\[1em]

$gen : A \to \ST{A\times B}{A \times C} \to \listtype{(A \times C)}$\\
$gen\;a_0\;p\;bs\;0 \;\;\; = \hat{p}\; [(a_0, bs_0)]$ \\
$gen\;a_0\;p\;bs\;(n+1) = $ \\
\;\;$\hat{p}\;(zip (a_0 \cdot (map\;\pi_1\;(gen\;a_0\;p\;bs\;n)))\;
                                        (take\;(n+2)\;bs))$ 
\end{tabbing}
\caption{Semantics of Stream Transducers}
\label{transducer-semantics}
\end{figure}

The operation $lift\;f$ creates a stream transducer that simply maps
the function $f$ over its input. Calls to $seq\;p\;q$ are sequential
composition: it feeds the output of $p$ into the input of $q$. The
operator $par\;p\;q$ defines parallel composition --- it takes a
stream of pairs, and feeds each component to its arguments,
respectively, and then merges the two output streams to produce the
combined output stream. The function $switch\;k\;p\;q$ is a very
simple ``switching combinator''.  It behaves as if it were $p$ for the
first $k$ time steps, and then behaves as if it were $q$, only
starting with the input stream beginning at time $k$.

The combinator $loop\;a_0\;p$ is a feedback operation. It acts
upon a transducer $p$ which takes pairs of $A$s and $B$s, and yields
pairs of $A$s and $C$s. It turns it into a combinator that takes $B$s
to $C$s, by giving $p$ the value $a_0$ (and its $B$-input) on the
first time step, and uses the output $A$ at time $n$ as the input $A$
at time $n+1$. This is useful for constructing transducers that do
things like sum their inputs over time, and other stateful operations. 

Because this function involves feedback, it should not be surprising
that it makes use of the causal nature of its argument operation. The
$loop$ function is defined in terms of $cycle$, which also returns the
sequence of output $A$s, and $cycle$ is defined in terms of $gen$,
which is a function that given an argument $n$ returns a list of
outputs for the time steps from $0$ to $n$. Notice that
$gen\;a_0\;p\;bs\;n$ will always return $n+1$ elements (e.g., at
argument 0, it will return a 1 element list containing the output at
time step 0), which means that the call to $last$ in $cycle$ is
actually safe. In order to calculate $gen$, we need to recursively
calculate the outputs for all smaller time steps, and this is what
$\hat{p}$ is needed for --- it is what lets us know that $p$ has a good
finite approximation.

All of these definitions are familiar to functional programmers, and
there are many techniques to prove properties of these functions ---
coinductive proofs, the $take$-lemma of \citet{bird-wadler}, arguments based on
the isomorphism between streams and functions from natural
numbers. All of these serve to make proving properties about stream
transducers very pleasant. For example, one property we will need in
the next section is the following:

\begin{lemma}{(Loop Unrolling)} We have that 
  \begin{displaymath}
    cycle\;a_0\;p\;bs = f\;(zip\;(a_0\cdot(map\;\pi_1\;(cycle\;a_0\;p\;bs)))\;bs)
  \end{displaymath}
\end{lemma}

\begin{proof}
  This is easily proved using Bird and Wadler's $take$-lemma, which
  says that two streams are equal if all their finite prefixes are
  equal.
\end{proof}


\subsection{Realizing Stream Transducers with Notifications}

\begin{figure}
\mbox{}
\begin{tabbing}
$\ST{\alpha}{\beta} \equiv \celltype{\alpha} \to \monad{\celltype{\beta}}$ \\[1em]

$\liftop : \forall \alpha,\beta:\star.\; (\alpha \to \beta) \to \ST{\alpha}{\beta}$ \\
$\liftop\;f\;input = $ \\
\;\; $\newcell\; (\bind\;(\readcell input)\; (\fun{x}{\alpha}{\return (f\;x)}))$ \\[1em]

$\composeop : \forall \alpha,\beta,\gamma:\star.\; \ST{\alpha}{\beta} \to \ST{\beta}{\gamma} \to \ST{\alpha}{\gamma}$ \\
$\composeop p\;q\;input = [$\=$\letv{middle}{p\;input}$ \\
                            \>$\letv{output}{q\;middle}$ \\ 
                            \>$\;output]$ \\[1em]

$\parop : \forall \alpha,\beta,\gamma,\delta:\star.\; \ST{\alpha}{\beta} \to \ST{\gamma}{\delta} \to \ST{\alpha \times \gamma}{\beta \times \delta}$ \\
$\parop p \; q \; input = $ \\
\;\;$[$\=$\ctext{letv}\;a = \newcell\; (\bind$\=$(\readcell\;input)$ \\
     \>                                   \>$(\fun{x}{\alpha\times \beta}{\return (\fst{x})}))$ \\
     \>$\ctext{letv}\;b = {p\;a}{}$ \\
     \>$\ctext{letv}\;c = \newcell\; (\bind$\=$(\readcell\;input)\;$\\ 
     \>                                   \>$(\fun{x}{\alpha\times \beta}{\return (\snd{x})}))$ \\
     \>$\ctext{letv}\;d = {q\;b} = $ \\
     \>$\ctext{letv}\;output = \newcell\; ($\=$\bind (\readcell b)\; (\lambda b:\beta.$ \\
     \>                                   \>$\bind (\readcell d)\; (\lambda d:\delta.$ \\
     \>                                   \>$\;\;\return \pair{b}{d})))] \;\ctext{in}$ \\
     \>$\;output]$ \\[1em]

$\switchop : \forall \alpha,\beta:\star.\; \N \to \ST{\alpha}{\beta} \to \ST{\alpha}{\beta} \to \ST{\alpha}{\beta}$ \\
$\switchop k\;p\;q\; input =  $ \\
\;\;$[$\=$\letv{r}{\newref{\N}{0}}{}$ \\
    \>$\letv{a}{p\;input}$ \\
    \>$\letv{b}{p\;input}$ \\
    \>$\ctext{letv}\; out = \newcell\; ($\=$\bind (\getref r) \;(\lambda i:\N.\;$ \\
    \>                                 \>$\bind (\setref r\;(i+1)) \; (\lambda q:\unittype.$ \\
    \>                                 \>$\;\;\ctext{if}(i < k, \readcell a, \readcell b)))) \;\ctext{in}$ \\
    \>$\;\;out]$\\[1em]

$\loopop : \forall \alpha,\beta,\gamma:\star.\; \alpha \to \ST{\alpha\times \beta}{\alpha\times \gamma} \to \ST{\beta}{\gamma}$ \\
$\loopop a_0\; p \; input = $ \\
\;\;$[$\=$\letv{r}{\newref{\alpha}{a_0}}{}$ \\
    \>$\ctext{letv}\; ab = \newcell\; ($\=$\bind (\readcell input)\; (\lambda b:\beta.$ \\
    \>                                \>$\bind (\getref r)\;       (\lambda a:\alpha.$ \\
    \>                                \>$\;\; \return \pair{a}{b}))) \;\ctext{in}$ \\
    \>$\letv{ac}{p\;ab}{}$ \\
    \>$\ctext{letv}\;c = \newcell\; ($\=$\bind (\readcell ac) \;(\lambda v:\alpha \times \gamma.$ \\
    \>                              \>$\bind (\setref r\;(\fst{v})) \;(\lambda q:\unittype.$ \\
    \>                              \>$\;\;\return (\snd{v}))))\;\ctext{in}$ \\
    \>$\;\;c]$ 
\end{tabbing}
\caption{Imperative Stream Transducers}
\label{imperative-transducer-impl}
\end{figure}

While the definitions in the previous subsection yield very clean
proofs, they are not suitable as implementations --- for example,
$loop$ recomputes an entire history at each time step! We can derive
better implementations by thinking about how imperative, event-driven
programming works.

The intuition underlying event-driven programming is that a stream
transducer is implemented with the combination of a notification
network, and an \emph{event loop}.  The event loop is a
(possibly-infinite) loop which updates an input cell at teach time
step, to reflect the events that occurred on that time step, and then
it reads the output cell of the network. When the input cell is
updated, invalidations are propagated throughout the dependency
network, and when the outputs are read, exactly the necessary
re-computations are performed.

We will shortly formalize exactly this idea, but we will first discuss
the implementation given in
Figure~\ref{imperative-transducer-impl} in informal terms. Here,
we define the type of imperative stream transducers as a function type
$\celltype{A} \to \monad{(\celltype{B})}$. This type should be read as
saying that the implementation is a function that, given an input cell
of type $A$, will \emph{construct} a dataflow notification network
realizing the corresponding transducer, and whose return value is the
output cell of type $B$ that the event loop should read. 

The simplest example of this is $\liftop\;f$. It will take an input
cell $input$, and build a new cell which reads $input$, and return $f$
applied to that value. Likewise, given two imperative implementations $p$
and $q$, $\composeop\;p\;q$ will take an input cell, and feed the
input to $p$ to build a network whose output is named $middle$, and
will then give $middle$ to $q$ to get the final output cell. The
overall network will be network built by the calls to both $p$ and
$q$, which interact through $p$'s network putting a value in $middle$,
and $q$'s network reading it.

The operation $\switchop\;k\;p\;q$ is the first example that uses 
local state. Given an $input$ cell, we first build networks corresponding
to $p$ and to $q$ (with outputs $a$ and $b$, respectively). Then we
create a local reference $r$, initialized to $0$. Then we build a cell $out$,
whose code reads and increments $r$, and which will read $a$ or $b$ depending
on whether the reference's contents are less than or equal to $k$. Notice
that the demand-driven nature of evaluation means that we never redundantly
evaluate $p$ or $q$'s networks --- we only ever execute one of them. 

Finally, the operation $\loopop a_0\;p$ builds a feedback network by
explicitly creating a reference to hold an accumulator parameter. It
constructs a local reference initialized to $a_0$, and then constructs
a cell $ab$ which reads the input and the local reference to produce a
pair of type $A \times B$. This cell is given to $p$, to construct a
network with an output cell $ac$, yielding pairs of type $A \times
C$. Finally, we construct the overall output cell $c$, which reads
$ac$ and updates the local reference with a new value of type $A$, and
returns a value of type $C$. The use of a local reference (rather than
a cell) to store the current state of $A$ is essential, because we need
to maintain the acyclicity of the dataflow graph. 

\subsection{Formally Specifying Transducers}

In Figure~\ref{transducer-specification}, we give the predicates we
will use to specify our imperative stream transducer library. The key
predicate is the predicate $\mathit{Transduce}(vs, i, \phi, o, ws)$
which says that the dataflow network $\phi$ reads the streams $vs$ on
input cell $i$, and writes the stream $ws$ on the output cell $o$. The
only subtlety in this specification is that $\phi$ is a function in
$\N \to \mathsf{formula}$, giving the formula of the dataflow network
as a function of time.

However, the definition of the predicate itself is consderably more
complicated. The first five lines of the specification give syntactic
conditions on readiness and unreadiness. 
\begin{enumerate}
\item The first condition, that $\unready{\theta}{i} \implies
\unready{\theta \otimes \phi_n}{o}$, says that the output $o$ must
depend on the input cell $i$.

\item The second condition, that $\ready{\theta}{i}{vs_n} \implies
  \ready{\theta \otimes \phi_{n+1}}{o}{ws_n}$, says that when the
  input cell $i$ has been read for its value $vs_n$, then the output
  cell will be ready with the value $ws_n$, and furthermore the
  network $\phi_{n+1}$ will have evolved to the state it will use to
  execute at time $n + 1$.

\item The condition $\closed{\phi_n}{\domain{\phi_n} \cup \setof{i}}$
means that $\phi_n$ depends on no external cells except for the input
cell $i$. 

\item The condtion $o \in \domain{\phi_n}$ simply ensures that the
output cell is always part of the network that computes it. 

\item The condition $\unready{R(\setof{i}, \phi_n)}{o}$ ensures
that the output is not ready to read, after we update the input
cell with a new, yet-to-be-read value. 
\end{enumerate}

Then, we can give the specification for the transducer itself.  The
intuitive reading of the specification is that if reading $i$ produces
a value $vs_n$, then reading $o$ should produce the value
$ws_n$. However, this definition is complicated by the fact that we do
not wish to specify the exact state owned by $i$ --- we are only
interested in the state $\phi$, and want our specification to work
obliviously to the state read by $i$. So we state the specification of
$o$ hypothetically, quantifying over the specification of $o$. So our
specification says that for any $\theta$ and $\theta'$, for which
reading $i$ leads to value $vs_n$ with ramification $u$, then reading
$o$ will give us the desired value $ws_n$. 

The other hypothetical assumptions give syntactic representations to
some natural conditions on $\theta$ and $\theta'$. Namely, if $i$ is
ready in $\theta$, then reading $\theta$ should do nothing (i.e.,
$\theta' = \theta$. If $i$ is not ready in $\theta$, then $i$ should
be ready with value $vs_n$ in $\theta'$, and finally that $\theta'$
should be closed, meaning that it has everything needed to evaluate
$i$. 

The $\mathit{Realize}$ predicate simply says that $\mathsf{\hat{f}}$
is a realizer for the function $f$, which means that executing it 
with input cell $i$ builds a dataflow graph which can act as a 
suitable transducer.  


\begin{figure}
\begin{specification}
\\ $\mathit{Transduce}(vs, i, \phi, o, ws)  =  $
\nextline $\forall \theta, n.\; \unready{\theta}{i} \implies \unready{\theta \otimes \phi_n}{o} \specand$ 
\nextline $\forall \theta, n.\; \ready{\theta}{i}{vs_n} \implies \ready{\theta \otimes \phi_{n+1}}{o}{ws_n} \specand$
\nextline $\forall n.\; \closed{\phi_n}{\domain{\phi_n} \cup \setof{i}} \specand$ 
\nextline $\forall n.\; o \in \domain{\phi_n} \specand$
\nextline $\forall n.\; \unready{R(\setof{i}, \phi_n)}{o} \specand$
\nextline\;\; $\forall n, \theta, \theta', u \subseteq \domain{\theta}.$  
\nextline\qquad\=$(\ready{\theta}{i}{v_n} \land \theta = \theta' \land u = \emptyset \specor
                         \unready{\theta}{i} \land i \in u) \specand$ 
\nextline\> $\ready{\theta'}{i}{v_n} \specand \closed{\theta'}{\domain{\theta'}} \specand$
\nextline\> $\forall \psi.\;$\=$\left<G(\theta \otimes \psi)\right>$ 
\nextline\>\> $\readcell i$
\nextline\>\> $\left<a.\;a = (vs_n, \setof{i}) \land G(\theta' \otimes R(u,\psi)\right>$ 
\nextline\> $\specimp$ 
\nextline\> $\exists \setof{o} \subseteq u' \subseteq \domain{\phi_n}.\;  \forall \psi.\;$\=$\left<G(\theta \otimes R(\setof{i}, \phi_n) \otimes \psi)\right>$ 
\nextline\>\>$\readcell o$ 
\nextline\>\>$\left<a.\;a = (ws_n, \setof{o}) \land 
                        G(\theta' \otimes \phi_{n+1} \otimes R(u \cup u', \psi))\right>$ 
\end{specification}


\begin{specification}
\nextline $\mathit{Realize}(\mathsf{\hat{f}}, f) = $ 
\nextline\;\;\= $\forall \psi, i.\; \spec{G(\psi)}{\mathsf{\hat{f}}\;i}{a}{\exists \phi.\;G(\psi \otimes R(\setof{i}, \phi_0)) \land \forall vs.\;\mathit{Transduce}(vs, i, \phi, a, f\;vs)}$
\end{specification}
\caption{Transducer Specification}
\label{transducer-specification}
\end{figure}


\subsection{Correctness Proofs}

\begin{prop}{(FRP Correctness)}
Then, the following specifications are provable: 
\begin{enumerate}
\item $\forall f:A\to B.\; \mathit{Realize}(\liftop f, \mathit{lift}\;f)$
\item $\forall p, f, q, g.\;\mathit{Realize}(p, f) \specand \mathit{Realize}(q, g) \specimp \mathit{Realize}(\composeop p\;q, \mathit{compose}\;f\;g)$
\item $\forall p, f, q, g.\;\mathit{Realize}(p, f) \specand \mathit{Realize}(q, g)
                       \specimp \mathit{Realize}(\parop p\;q, par\;f\;g)$
\item $\forall k, p, f, q, g.\;\mathit{Realize}(p, f) \specand \mathit{Realize}(q, g)
                          \specimp \mathit{Realize}(\switchop k\;p\;q, switch\;k\;f\;g)$
\item $\forall a_0, p, f.\; \mathit{Realize}(p, f) \specimp \mathit{Realize}(\loopop\;a_0\;p, loop\;a_0\;f)$
\end{enumerate}

\end{prop}

\noindent To prove these properties, we will first prove transduction lemmas for each kind of 
network, and then show that each operator creates an appropriate network. 

\subsubsection{Correctness of $\mathsf{lift}$}

\begin{lemma}{(The $\mathsf{lift}$ network)}
It is the case that $\mathit{Transduce}(vs, i, \phi, c, \mathit{lift}\;f\;vs)$ holds, when
\begin{itemize}
\item $\phi = \semfun{n}{\cellpos{c}{e}{ws_n}{\setof{i}}}$, where $ws = x \cdot (f\;vs)$ for some dummy $x$ and
\item $e = \bind (\readcell i)\; (\semfun{v}{\return (f\;v)})$
\end{itemize}
\end{lemma}

\begin{proof}
\begin{itemize}
\item First, it is clear that for all $\theta$ and $n$, if $\unready{\theta}{i}$, then $\unready{\theta \otimes \phi_n}{c}$, since $i$ is in the read set of $\phi_n$. 
\item Second, it is clear that for all $\theta$ and $n$, if $\ready{\theta}{i}{v_n}$, then $\ready{\theta \otimes \phi_n}{c}{(f\;v)_n}$, since the read set of $\phi_n$ is $i$, which is by hypothesis ready. 
\item Third, it is immediately clear that for all $n$, $\closed{\phi_n}{\setof{i} \cup \domain{\phi_n}}$,
since $\phi_n$ has a single cell whose read set contains only $i$. 
\item Fourth, it is immediate $c \in \domain{\cellpos{c}{e}{ws_n}{\setof{i}}}$
\item Fifth, for all $n$, we know that $R(\setof{i}, \phi_n) = \cellneg{c}{e}$, which obviously satisfies $\unready{R(\setof{i}, \phi_n)}{c}$ 

\end{itemize}
Now we need to show the implication over triples. 
\begin{tabbedproof}
\oo Assume we have $n$, $\theta$, $\theta'$ and $u \subseteq \domain{\theta}$. \\
\oo Assume that either $\ready{\theta}{i}{v_n}$ and $\theta = \theta'$ and $u = \emptyset$,  \\
\oo or $\unready{\theta}{i} \land i \in u$ \\
\oo Assume that $\closed{\theta'}{\domain{\theta'}}$ and that $\ready{\theta'}{i}{v_n}$ \\
\oo Assume for all $\psi$, we have \\
\ox  $\mspec{G(\theta \otimes \psi)}{\readcell i}{a}{a = ((f\;v)_n, \setof{c}) 
      \land G(\theta' \otimes R(u,\psi))}$ \\
\oo Now assume we have a $\psi$ and a prestate $G(\theta \otimes R(\setof{i}, \phi_n) \otimes \psi)$ \\
\oo We know $R(\setof{i}, \phi_n) = \cellneg{c}{e}$ \\
\oo So we know $\unready{\theta \otimes R(\setof{i}, \phi_n) \otimes \psi}{c}$ \\
\oo By the \textsc{AUnready}, \textsc{ABind}, and \textsc{AUnit} rules, we end in a state \\
\oo $G(R(u',\theta') \otimes \cellpos{c}{e}{f\;v_n}{\setof{i}} \otimes R(u \cup u',\psi)) \land a = (f\;v_n, \setof{o})$ \\
\oo where $u' = \setof{c}$ \\
\oo Since $c$ does not occur in $\theta'$ or else false is entailed, we know $c \not\in \domain{\theta'}$ \\
\oo Since $\theta'$ is closed with respect to its domain, $R(u',\theta') = \theta'$ \\
\oo Note that $f\;v_n = (\mathit{lift}\;f\;v)_n$, hence we have the conclusion we want. \\
\end{tabbedproof}

\end{proof}

\subsubsection{Correctness of $\mathsf{compose}$}

\begin{lemma}{(The $\mathsf{compose}$ network)}
If it is the case that $\mathit{Transduce}(vs, i, \phi, m, f\;vs)$ holds, and that
$\mathit{Transduce}(f\;vs, m, \sigma, o, g(f\;vs))$ holds, then it is the case that
$\mathit{Transduce}(vs, i, \phi \otimes \sigma, o, g\;(f\;vs))$ holds.
\end{lemma}

\begin{proof}
\begin{itemize}
\item First, we want to show that for all $\theta, n$, if $\unready{\theta}{i}$ then $\unready{\theta \otimes \phi_n \otimes \sigma_n}{o}$. 

  To show this assume $\unready{\theta}{i}$. Then we know from $\mathit{Transduce}(vs, i, \phi, m, f\;vs)$
  that $\unready{\theta \otimes \phi_n}{m}$. Then we know from $\mathit{Transduce}(f\;vs, m, \sigma, o, g(f\;vs))$
  that $\unready{\theta \otimes \phi_n \otimes \sigma_n}{o}$. 

\item Second, we want to show that for all $\theta, n$, if
  $\ready{\theta}{i}{vs_n}$ then $\ready{\theta \otimes \phi_n \otimes
  \sigma_{n+1}}{o}{(f\;vs)_n}$.

  Assume $\ready{\theta}{i}{vs_n}$. 

  Then we know from $\mathit{Transduce}(vs, i, \phi, m, f\;vs)$ that
  $\ready{\theta \otimes \phi_{n+1}}{m}{(f\;vs)_n}$. Then we know from
  $\mathit{Transduce}(f\;vs, m, \sigma, o, g(f\;vs))$ that
  $\ready{\theta \otimes \phi_{n+1} \otimes
    \sigma_{n+1}}{o}{(g\;(f\;vs))_n}$.

\item Third, we want to show that $\unready{R(\setof{i}, \phi_n \otimes \sigma_n)}{o}$. 

  By hypothesis, we know that $\unready{R(\setof{i},
    \phi_n)}{m}$. Hence $\unready{R(\setof{i}, \phi_n) \otimes
    \sigma_n}{o}$, and hence $\unready{R(\setof{i}, \phi_n \otimes
    \sigma_n)}{o}$, since $\sigma$ is closed with respect to $i$. 

\item Fourth, we want to show that $\closed{\phi_n \otimes \sigma_n}{\domain{\phi_n \otimes \sigma_n} \cup \setof{i}}$. We know that  $\closed{\phi_n}{\domain{\phi_n} \cup \setof{i}}$ and 
$\closed{\sigma_n}{\domain{\sigma_n} \cup \setof{m}}$. So we know that $\closed{\phi_n \otimes \sigma_n}{\domain{\phi_n \otimes \sigma_n} \cup \setof{i,m}}$. Since $m \in \domain{\phi_n}$, we have that
$\closed{\phi_n \otimes \sigma_n}{\domain{\phi_n \otimes \sigma_n} \cup \setof{i}}$

\item Finally we want to show that $o \in \domain{\phi_n \otimes \sigma_n}$. Since $o \in \domain{\sigma_n}$,
  it follows immediately that $o \in \domain{\phi_n \otimes \sigma_n}$. 


\end{itemize}

Now, let's prove the entailment of specifications. 

\begin{tabbedproof}
\oo Assume we have $\theta, \theta', n$, and $u$. \\
\oo Assume that either $\ready{\theta}{i}{v_n}$ and $\theta = \theta'$ and $u = \emptyset$,  \\
\oo or $\unready{\theta}{i} \land i \in u$ \\
\oo Assume that $\closed{\theta'}{\domain{\theta'}}$ \\
\oo Assume that $\ready{\theta'}{i}{v_n}$ and $\closed{\theta'}{\domain{\theta'}}$ \\
\oo Assume for all $\psi$, we have \\
\ox  $\mspec{G(\theta \otimes \psi)}{\readcell i}{a}{a = ((f\;v)_n, \setof{c}) 
      \land G(\theta' \otimes R(u,\psi)) }$ \\
\oo Assume we have a $\psi$. Now, we use the hypothesis above, together \\
\oo with the fact that $\mathit{Transduce}(f\;vs, m, \sigma, o, g(f\;vs))$ holds, \\
\oo to conclude that there is a $u'$ such that $\setof{m} \subseteq u' \subseteq \domain{\phi_n}$ and \\
\ox $\left<G(\theta \otimes R(\setof{i}, \phi_n) \otimes \psi)\right>$ \\
\ox $\readcell m$ \\
\ox $\left<a.\;G(\theta' \otimes \phi_{n+1} \otimes R(u \cup u',\psi)
             \land  a = ((f\;v)_n, \setof{m})\right>$ \\
\oo Note that since $\theta'$ is closed with respect to its own domain and $\phi_n$ is closed with \\
\oo respect to $i$ and its own domain, we have $\closed{\theta' \otimes \phi_{n+1}}{\domain{\theta' \otimes \phi_{n+1}}}$ \\
\oo Note also that $m \in u \cup u'$ and $\unready{\theta \otimes R(\setof{i}, \phi_n)}{m}$ \\
\oo Furthermore, since $\ready{\theta'}{i}{v_n}$, we know $\ready{\theta' \otimes \phi_{n+1}}{m}{(f\;v)_n}$ \\
\oo Hence from $\mathit{Transduce}(f\;vs, m, \sigma, o, g(f\;vs))$, we can conclude that \\
\oo there is a $u''$ such that $\setof{o} \subseteq u'' \subseteq \domain{\sigma_n}$ and for all $\psi$ \\
\ox $\left<G(\theta \otimes R(\setof{i}, \phi_n) \otimes R(\setof{m}, \sigma_n) \otimes \psi)\right>$ \\
\ox $\readcell o$ \\
\ox $\left<a.\;G(\theta' \otimes \phi_{n+1} \otimes \sigma_{n+1} \otimes R(u \cup u' \cup u'',\psi)
             \land a = ((f\;v)_n, \setof{o})\right>$ \\
\oo Since $\unready{\theta \otimes R(\setof{i}, \phi_n)}{m}$, we know that \\
\oo $\unready{\theta \otimes R(\setof{i}, \phi_n) \otimes \sigma_n}{m}$ \\
\oo Hence $\theta \otimes R(\setof{i}, \phi_n) \otimes \sigma_n \vdash R(\setof{m}, \theta \otimes R(\setof{i}, \phi_n) \otimes \sigma_n)$ \\
\oo Hence $\theta \otimes R(\setof{i}, \phi_n) \otimes \sigma_n \vdash \theta \otimes R(\setof{i}, \phi_n) \otimes R(\setof{m}, \sigma_n)$ \\
\oo Since $i \not\in \domain{\sigma_n} \cup \setof{m}$, we know $R(\setof{i}, \sigma_n) = \sigma_n$ \\
\oo Hence $\theta \otimes R(\setof{i}, \phi_n) \otimes R(\setof{i}, \sigma_n) \vdash \theta \otimes R(\setof{i}, \phi_n) \otimes R(\setof{m}, \sigma_n)$ \\
\oo Hence $\theta \otimes R(\setof{i}, \phi_n \otimes \sigma_n) \vdash \theta \otimes R(\setof{i}, \phi_n) \otimes R(\setof{m}, \sigma_n)$ \\
\oo Hence we can conclude that \\
\ox $\left<G(\theta \otimes R(\setof{i}, \phi_n \otimes \sigma_n) \otimes R(\setof{m}, \sigma_n) \otimes \psi)\right>$ \\
\ox $\readcell o$ \\
\ox $\left<a.\;G(\theta' \otimes \phi_{n+1} \otimes \sigma_{n+1} \otimes R(u \cup u' \cup u'',\psi)
             \land a = ((f\;v)_n, \setof{o})\right>$ \\
\oo Note that $u' \cup u'' \subseteq \domain{\phi_n \otimes \sigma_n}$, which is the existential witness
\end{tabbedproof}
\end{proof}

\subsubsection{Correctness of $\mathsf{par}$}

\begin{lemma}{(The $\mathsf{par}$ network)}
If it is the case that $\mathit{Transduce}(v, i, \phi, b, f\;v)$ holds, and that
$\mathit{Transduce}(v, i, \sigma, c, g\;v)$ holds, then it is the case that
$\mathit{Transduce}(v, i, \phi \otimes \sigma \otimes \omega, o, \mathit{par}\;f\;g\;v)$ holds,
where 

\begin{itemize}
\item $\omega = \semfun{n}{\cellpos{o}{e}{ws_n}{\setof{b,c}}}$, where $ws = x \cdot (\mathit{par}\;f\;g\;v)$ for some dummy $x$ and
\item $e = \bind (\readcell b)\; (\semfun{x}{\bind (\readcell c)\; (\semfun{y}{\return (x,y)})})$
\end{itemize}
\end{lemma}

\begin{proof}
\begin{itemize}
\item We want to show that $\unready{\theta}{i}$ implies $\unready{\theta \otimes \phi_n \otimes \sigma_n \otimes \omega_n}{o}$ 

  Assume that $\unready{\theta}{i}$ holds. Then we know by hypothesis
  that $\unready{\theta \otimes \phi_n}{b}$ holds. This lets us conclude
  that $\unready{\theta \otimes \phi_n \otimes \omega_n}{o}$ holds. From this,
  it follows that $\unready{\theta \otimes  \phi_n \otimes \sigma_n \otimes \omega_n}{o}$.

\item We want to show that $\ready{\theta}{i}{v_n}$ implies $\ready{\theta \otimes \phi_{n+1} \otimes \sigma_{n+1} \otimes \omega_{n+1}}{o}{(\mathit{par}\;f\;g\;v)_n}$ 

  Assume that $\ready{\theta}{i}{v_n}$ holds. Then we know that 
  $\ready{\theta \otimes \phi_{n+1}}{b}{(f\;v)_n}$ holds, and that 
  $\ready{\theta \otimes \sigma_{n+1}}{c}{(g\;v)_n}$ holds. 

  So we know that $\ready{\theta \otimes \phi_{n+1} \otimes \sigma_{n+1} \otimes \omega_{n+1}}{b}{(f\;v)_n}$ holds.

  So we know that $\ready{\theta \otimes \phi_{n+1} \otimes \sigma_{n+1} \otimes \omega_{n+1}}{c}{(g\;v)_n}$ holds.

  Hence $\ready{\theta \otimes \phi_{n+1} \otimes \sigma_{n+1} \otimes \omega_{n+1}}{o}{(\mathit{par}\;f\;g\;v)_n}$ holds.

\item We want to show that $\closed{\phi_n \otimes \sigma_n \otimes \omega_n}{\domain{\phi_n \otimes \sigma_n \otimes \omega_n} \cup \setof{i}}$ 

  We know $\closed{\phi_n}{\domain{\phi_n} \cup \setof{i}}$. 
  We know $\closed{\sigma_n}{\domain{\sigma_n} \cup \setof{i}}$. 
  We know $\closed{\omega_n}{\domain{\omega_n} \cup \setof{b,c}}$. 

  So $\closed{\phi_n \otimes \sigma_n \otimes \omega_n}{\domain{\phi_n \otimes \sigma_n \otimes \omega_n} \cup \setof{i, b, c}}$ .

  But $b \in \domain{\phi_n}$ and $c \in \domain{\sigma_n}$, so 
   $\closed{\phi_n \otimes \sigma_n \otimes \omega_n}{\domain{\phi_n \otimes \sigma_n \otimes \omega_n} \cup \setof{i}}$. 

\item We want to show that $\unready{R(\setof{i}, \phi_n \otimes \sigma_n \otimes \omega_n)}{o}$ 

  We know by hypothesis that $\unready{R(\setof{i}, \phi_n)}{b}$
  holds. This lets us conclude that $\unready{R(\setof{i}, \phi_n
    \otimes \omega_n)}{o}$ holds. From this, it follows that
  $\unready{R(\setof{i}, \phi_n \otimes \sigma_n \otimes
    \omega_n)}{o}$.

\item We want to show that $o \in \domain{\omega_n}$. This is immediate. 
\end{itemize}

\noindent Now we will show the implication over triples holds.

\begin{tabbedproof}
\oo Assume we have $\theta, \theta', n$, and $u$. \\
\oo Assume that either $\ready{\theta}{i}{v_n}$ and $\theta = \theta'$ and $u = \emptyset$,  \\
\oo or $\unready{\theta}{i} \land i \in u$ \\
\oo Assume that $\closed{\theta'}{\domain{\theta'}}$ \\
\oo Assume that $\ready{\theta'}{i}{v_n}$ and $\closed{\theta'}{\domain{\theta'}}$ \\
\oo Assume for all $\psi$, we have \\
\ox  $\mspec{G(\theta \otimes \psi)}{\readcell i}{a}{a = ((f\;v)_n, \setof{c}) 
      \land G(\theta' \otimes R(u,\psi)) }$ \\
\oo Since $\mathit{Transduce}(v, i, \phi, b, f\;v)$, we can conclude that \\
\oo there is a $u'$ such that $\setof{b} \subseteq u' \subseteq \domain{\phi_n}$, \\
\oo $\forall \psi.\;\mspec{G(\theta \otimes R(\setof{i},\phi_n) \otimes \psi)}{\readcell b}{a}{a = ((f\;v)_n, \setof{b}) \land G(\theta' \otimes \phi_{n+1} \otimes R(u \cup u', \psi))}$ \\
\oo Now, note that $\ready{\theta'}{i}{v_n}$, so by rule \textsc{AReady} we can conclude that \\
\oo $\forall \psi.\;\mspec{G(\theta' \otimes \psi)}{\readcell i}{a}{a = (v_n, \setof{i}) \land G(\theta' \otimes \psi)}$ \\
\oo So from $\mathit{Transduce}(v, i, \sigma, c, f\;v)$, we can conclude that\\
\oo there is a $u''$ such that $\setof{c} \subseteq u'' \subseteq \domain{\sigma_n}$ and \\
\oo $\forall \psi.\;\mspec{G(\theta' \otimes R(\setof{i},\sigma_n) \otimes \psi)}{\readcell c}{a}{a = ((g\;v)_n, \setof{c}) \land G(\theta' \otimes \phi_{n+1} \otimes R(u'', \psi))}$ \\
\oo Now let $u''' = u' \cup u'' \cup \setof{o}$ \\
\oo So we want to show that \\
\oo $\forall \psi.\;$\=$\left<G(\theta \otimes R(\setof{i},\phi_n \otimes \sigma_n \otimes \omega_n) \otimes \psi)\right>$ \\
\oo \> ${\readcell c}$ \\
\oo \> $\left<{a.\;a = ((\mathit{par}\;f\;g\;v)_n, \setof{c}) \land G(\theta' \otimes \phi_{n+1} \otimes \sigma_{n+1} \otimes \omega_{n+1} \otimes R(u'', \psi))}\right>$ \\
\oo Since $c$ is unready in the precondition, we need to use the \textsc{AUnready} rule. \\
\oo So consider the expression $e$, in a state \\
\oo $G(\theta \otimes R(\setof{i},\phi_n \otimes \sigma_n) \otimes \psi)$ \\
\oo Now, to use the \textsc{ABind} rule, consider  $\readcell b$. By hypothesis, we end up in a state \\
\oo $G(\theta' \otimes \phi_{n+1} \otimes R(u' \cup \setof{i},\sigma_n) \otimes R(u', \psi)) \land (x,r_1) = ((f\;v)_n, \setof{b})$ \\
\oo Since $\phi$ and $\sigma$ have disjoint domains, this implies (pulling out $(x,r_1) = ((f\;v)_n, \setof{b})$) \\
\oo Also, since $i \in u'$, \\
\oo $G(\theta' \otimes \phi_{n+1} \otimes R(\setof{i},\sigma_n) \otimes R(u', \psi))$ \\
\oo Now, using the \textsc{ABind} rule again, consider $\readcell c$. By hypothesis, we end up in a state \\
\oo $G(\theta' \otimes R(u'',\phi_{n+1}) \otimes \sigma_{n+1} \otimes R(u' \cup u'', \psi)) \land (y,r_2) = ((g\;v)_n, \setof{b, c})$ \\
\oo Since $\phi$ and $\sigma$ have disjoint domains, this implies (pulling out $(y,r_2) = ((g\;v)_n, \setof{b, c})$) \\
\oo $G(\theta' \otimes \phi_{n+1} \otimes \sigma_{n+1} \otimes R(u' \cup u'', \psi)) \land (y, r_2) = ((g\;v)_n, \setof{b,c})$ \\
\oo Now, by the \textsc{AReturn} rule and the definition of $\mathit{par}$, we get \\
\oo $G(\theta' \otimes \phi_{n+1} \otimes \sigma_{n+1} \otimes R(u' \cup u'', \psi)) \land a = ((\mathit{par}\;f\;g\;v)_n, \setof{b,c})$ \\
\oo So the final state is \\
\oo $G(R(\setof{c}, \theta' \otimes \phi_{n+1} \otimes \sigma_{n+1}) \otimes \omega_{n+1} \otimes R(\setof{c} \cup u' \cup u'', \psi)) \land a = ((\mathit{par}\;f\;g\;v)_n, \setof{c})$ \\
\oo Since $c$ is not in the domain of $\theta', \phi_{n+1},$ or $\sigma_{n+1}$, and these are all suitably closed\\
\oo $G(\theta' \otimes \phi_{n+1} \otimes \sigma_{n+1} \otimes \omega_{n+1} \otimes R(u''', \psi)) \land a = ((\mathit{par}\;f\;g\;v)_n, \setof{c})$ \\
\end{tabbedproof}
\end{proof}

\subsubsection{Correctness of $\mathsf{switch}$}

\begin{lemma}{(The $\mathsf{par}$ network)}
If it is the case that $\mathit{Transduce}(v, i, \phi, b,f\;v)$ holds, and that 
$\mathit{Transduce}(\mathit{drop}\;k\;v, i, \sigma, c, g\;v)$ holds,
then it is the case that $\mathit{Transduce}(v, i, \omega, o,
\mathit{switch}\;k\;f\;g\;v)$ holds, where

\begin{itemize}
\item $\omega = \semfun{n}{\left\{\begin{array}{ll}
                                    \localref{r}{n} \otimes \phi_n \otimes R(\setof{i}, \sigma_0) \otimes \delta_n 
                                    & \mbox{if } n < k \\
                                    \localref{r}{n} \otimes R(\setof{i}, \phi_k) \otimes \sigma_{n-k} \otimes \delta_n 
                                    & \mbox{otherwise} \\
                                   \end{array}\right.}$
\item $\delta_n = \left\{\begin{array}{ll}
                          \cellpos{o}{e}{(\mathit{switch}\;f\;g\;v)_n}{\setof{b}} & \mbox{if } n < k\\
                          \cellpos{o}{e}{(\mathit{switch}\;f\;g\;v)_n}{\setof{c}} & \mbox{otherwise} \\
                         \end{array}
                 \right.$
\item $e = \bind (\getref r)\; (\semfun{n}{\bind (\setref r\;(n+1))\; (\semfun{\unit}{\IfThenElse{n < k}{\readcell b}{\readcell c}})})$
           

\end{itemize}
\end{lemma}

\begin{proof}
\begin{itemize}
\item First, we need to show that $\unready{\theta}{i}$ implies $\unready{\theta \otimes \omega_n}{o}$. 

  \begin{tabbedproof}
    \oo To prove this, assume $\unready{\theta}{i}$ and consider whether $n < k$:\\
    \oo If $n < k$:\\
    \ooo Then $\omega_n = \localref{r}{n} \otimes \phi_n \otimes R(\setof{i}, \sigma_0) \otimes \delta_n$ \\
    \ooo So we know that $\unready{\theta \otimes \phi_n}{b}$ holds \\
    \ooo Since $\delta_n = \cellpos{o}{e}{(\mathit{switch}\;f\;g\;v)_n}{\setof{b}}$, we can see that \\
    \ooo $\unready{\theta \otimes \omega_n}{o}$ \\
    \oo If $n \geq k$: \\
    \ooo Then $\omega_n = \localref{r}{n} \otimes R(\setof{i}, \phi_k) \otimes \sigma_{n-k} \otimes \delta_n$ \\
    \ooo So we know that $\unready{\theta \otimes \sigma_{n-k}}{c}$ holds \\
    \ooo Since $\delta_n = \cellpos{o}{e}{(\mathit{switch}\;f\;g\;v)_n}{\setof{c}}$, we can see that \\
    \ooo $\unready{\theta \otimes \omega_n}{o}$ \\
  \end{tabbedproof}

\item We want to show that $\ready{\theta}{i}{v_n}$ implies $\ready{\theta \otimes \omega_{n+1}}{o}{(\mathit{par}\;f\;g\;v)_n}$. 

  \begin{tabbedproof}
    \oo To prove this, assume $\ready{\theta}{i}{v_n}$ and consider whether $n < k$:\\
    \oo If $n < k$:\\
    \ooo Then $\omega_n = \localref{r}{n} \otimes \phi_n \otimes R(\setof{i}, \sigma_0) \otimes \delta_n$ \\
    \ooo So we know that $\ready{\theta \otimes \phi_n}{b}{(f\;v)_n}$ holds \\
    \ooo Since $\delta_n = \cellpos{o}{e}{(\mathit{switch}\;f\;g\;v)_n}{\setof{b}}$, we can see that \\
    \ooo $\ready{\theta \otimes \omega_n}{o}{(\mathit{switch}\;f\;g\;v)_n}$ \\
    \oo If $n \geq k$: \\
    \ooo Then $\omega_n = \localref{r}{n} \otimes R(\setof{i}, \phi_k) \otimes \sigma_{n-k} \otimes \delta_n$ \\
    \ooo So we know that $\ready{\theta}{i}{(\mathit{drop}\;k\;v)_{n-k}}$ \\
    \ooo So we know that $\ready{\theta \otimes \sigma_{n-k}}{c}{(g\;v)_{n-k}}$ holds \\
    \ooo Since $\delta_n = \cellpos{o}{e}{(\mathit{switch}\;f\;g\;v)_n}{\setof{c}}$, we can see that \\
    \ooo $\ready{\theta \otimes \omega_n}{o}{(\mathit{switch}\;f\;g\;v)_n}$ \\
  \end{tabbedproof}

\item Now we want to show that $\closed{\omega_n}{\domain{\omega_n} \cup \setof{i}}$. 

  \begin{tabbedproof}
    \oo Suppose $n < k$:\\
    \ooo Then $\omega_n = \localref{r}{n} \otimes \phi_n \otimes R(\setof{i}, \sigma_0) \otimes \delta_n$ \\
    \ooo So $\closed{\phi_n}{\domain{\phi_n} \cup \setof{i}}$ and \\
    \ooox $\closed{\sigma_0}{\domain{\sigma_n} \cup \setof{i}}$ so $\closed{R(\setof{i}, \sigma_0)}{\domain{R(\setof{i}, \sigma_n)} \cup \setof{i}}$ and \\
    \ooox $\closed{\delta_n}{\setof{b}}$ \\
    \ooo Since $b \in \domain{\phi_n}$, it follows that $\closed{\omega_n}{\domain{\omega_n} \cup \setof{i}}$ \\
    \oo Suppose $n \geq k$:\\
    \ooo Then $\omega_n = \localref{r}{n} \otimes R(\setof{i}, \phi_k) \otimes \sigma_{n-k} \otimes \delta_n$ \\
    \ooo So $\closed{\sigma_{n-k}}{\domain{\sigma_{n-k} \cup \setof{i}}}$ and \\
    \ooox $\closed{\phi_k}{\domain{\phi_k} \cup \setof{i}}$ so $\closed{R(\setof{i}, \phi_k)}{\domain{R(\setof{i}, \phi_k)} \cup \setof{i}}$ and \\
    \ooox $\closed{\delta_n}{\setof{c}}$ \\
    \ooo Since $c \in \domain{\sigma_{n-k}}$, it follows that $\closed{\omega_n}{\domain{\omega_n} \cup \setof{i}}$  \end{tabbedproof}

\item Next, we want to show that $\unready{R(\setof{i}, \omega_n)}{o}$

  \begin{tabbedproof}
    \oo To prove this, consider whether $n < k$:\\
    \oo If $n < k$:\\
    \ooo Then $\omega_n = \localref{r}{n} \otimes \phi_n \otimes R(\setof{i}, \sigma_0) \otimes \delta_n$ \\
    \ooo So we know that $\unready{\theta \otimes \phi_n}{b}$ holds \\
    \ooo Since $\delta_n = \cellpos{o}{e}{(\mathit{switch}\;f\;g\;v)_n}{\setof{b}}$, we can see that \\
    \ooo $\unready{\theta \otimes \omega_n}{o}$ \\
    \oo If $n \geq k$: \\
    \ooo Then $\omega_n = \localref{r}{n} \otimes R(\setof{i}, \phi_k) \otimes \sigma_{n-k} \otimes \delta_n$ \\
    \ooo So we know that $\unready{\theta \otimes \sigma_{n-k}}{c}$ holds \\
    \ooo Since $\delta_n = \cellpos{o}{e}{(\mathit{switch}\;f\;g\;v)_n}{\setof{c}}$, we can see that \\
    \ooo $\unready{\theta \otimes \omega_n}{o}$ \\
  \end{tabbedproof}


\item Next, we want to show that $o \in \domain{\omega_n}$, which is immediate. 
\end{itemize}

Now we will show the implication over triples. 
\begin{tabbedproof}
\oo Assume we have $\theta, \theta', n$, and $u$. \\
\oo Assume that either $\ready{\theta}{i}{v_n}$ and $\theta = \theta'$ and $u = \emptyset$,  \\
\oo or $\unready{\theta}{i} \land i \in u$ \\
\oo Assume that $\closed{\theta'}{\domain{\theta'}}$ \\
\oo Assume that $\ready{\theta'}{i}{v_n}$ and $\closed{\theta'}{\domain{\theta'}}$ \\
\oo Assume for all $\psi$, we have \\
\ox  $\mspec{G(\theta \otimes \psi)}{\readcell i}{a}{a = ((f\;v)_n, \setof{c}) 
      \land G(\theta' \otimes R(u,\psi)) }$ \\
\oo Now suppose $n < k$: \\
\ooo Now assume we have $\psi$ and note that  \\
\oooo since we know $\mathit{Transduce}(v, i, \phi, b, (f\;v)_n)$, we can conclude that \\
\oooo there is a $u'$ such that $\setof{b} \subseteq u' \subseteq \domain{\phi_n}$ such that \\
\oooo $\left<G(\theta \otimes \localref{r}{n+1} \otimes R(\setof{i}, \phi_n) \otimes R(\setof{i}, \sigma_0) \otimes \psi)\right>$ \\
\oooo $\readcell b$\\
\oooo $\left<a.\;G(\theta' \otimes \localref{r}{n+1} \otimes \phi_{n+1} \otimes R(u' \cup \setof{i}, \sigma_0) \otimes R(u', \psi))
      \land a = ((f\;v)_n, \setof{b})\right>$ \\
\ooo Choose a witness to the existential equal to $u' \cup \setof{o}$ and consider the prestate \\
\ooo $G(\theta \otimes R(\setof{i}, \omega_n) \otimes \psi)$ \\
\ooo We know $R(\setof{i}, \omega_n) = \theta \otimes \localref{r}{n} \otimes R(\setof{i}, \phi_n) \otimes R(\setof{i}, \sigma_0) \otimes \delta_n$ \\
\ooo Now, we know that $\unready{\theta \otimes \omega_n \otimes \psi}{o}$, since $\delta_n$'s read set \\
\ooo contains $b$, which is unready \\
\ooo So to evaluate $\readcell o$, we need to use the \textsc{AUnready} rule \\
\ooo So we need to evaluate $e$ in the state \\
\ooo $G(\theta \otimes \localref{r}{n} \otimes R(\setof{i}, \phi_n) \otimes R(\setof{i}, \sigma_0) \otimes \psi)$ \\
\ooo We begin with the \textsc{ABind} rule, which asks us to evaluate $\getref r$, which gives us \\
\ooo $G(\theta \otimes \localref{r}{n} \otimes R(\setof{i}, \phi_n) \otimes R(\setof{i}, \sigma_0) \otimes \psi) \land (n_1,r_1) = (n,\emptyset)$ \\
\ooo In this case, $e$ in $\delta_n$ simplifies to \\
\oooo $e = \bind (\getref r)\; (\semfun{n}{\bind (\setref r\;(n+1))\; (\semfun{\unit}{\readcell b})})$ \\
\ooo So now we need to evaluate $\bind (\setref r\;(n+1))\; (\semfun{\unit}{\readcell b})$ in \\
\ooo $G(\theta \otimes \localref{r}{n+1} \otimes R(\setof{i}, \phi_n) \otimes R(\setof{i}, \sigma_0) \otimes \psi)$ \\
\ooo Using the \textsc{ABind} rule, we need to evaluate $\setref r\;(n+1)$, which yields  \\
\ooo $G(\theta \otimes \localref{r}{n+1} \otimes R(\setof{i}, \phi_n) \otimes R(\setof{i}, \sigma_0) \otimes \psi)$ \\
\ooo with the need to consider $\readcell b$ \\
\ooo Using the triple we derived above, we get \\
\ooo $G(\theta' \otimes \localref{r}{n+1} \otimes \phi_{n+1} \otimes R(u' \cup \setof{i}, \sigma_0) \otimes R(u', \psi))
      \land a = ((f\;v)_n, \setof{b})$ \\
\ooo So then finishing the \textsc{AUnready} rule, we have a state \\
\ooo $G(R(\setof{o}, \theta' \otimes \localref{r}{n+1} \otimes \phi_{n+1} \otimes R(u' \cup \setof{i}, \sigma_0) \otimes R(u', \psi)) \otimes \delta_{n+1}) \land a = ((f\;v)_n, \setof{o})$\\
\ooo Simplifying using the closure properties, we get \\
\ooo $G(\theta' \otimes \localref{r}{n+1} \otimes \phi_{n+1} \otimes R(u' \cup \setof{i}, \sigma_0) \otimes \delta_{n+1} \otimes R(u' \cup \setof{o}, \psi))) \land a = ((f\;v)_n, \setof{o})$\\
\ooo which is \\
\ooo $G(\theta' \otimes \localref{r}{n+1} \otimes \phi_{n+1} \otimes R(u' \cup \setof{i}, \sigma_0) \otimes \delta_{n+1} \otimes R(u' \cup \setof{o}, \psi))) \land a = ((f\;v)_n, \setof{o})$\\
\ooo Since $(f\;v)_n = (\mathit{switch}\;k\;f\;g\;v)_n$, we have the conclusion \\
\oo Now suppose $n \geq k$: \\
\ooo Now assume we have $\psi$ and note that  \\
\oooo since we know $\mathit{Transduce}(\mathit{drop}\;k\;v, i, \sigma, c, (g\;(\mathit{drop}\;k\;v))_n)$, we can conclude that \\
\oooo there is a $u'$ such that $\setof{c} \subseteq u' \subseteq \domain{\sigma_{n-k}}$ such that \\
\oooo $\left<G(\theta \otimes \localref{r}{n+1} \otimes R(\setof{i}, \phi_k) \otimes R(\setof{i}, \sigma_{n-k}) \otimes \psi)\right>$ \\
\oooo $\readcell c$\\
\oooo $\left<a.\;G(\theta \otimes \localref{r}{n+1} \otimes R(\setof{i}, \phi_k) \otimes \sigma_{n-k+1} \otimes R(u', \psi))
      \land a = ((g\;(\mathit{drop}\;k\;v))_{n-k}, \setof{b})\right>$ \\
\ooo making use of the observation that $(\mathit{drop}\;k\;v)_{n-k} = v_n$ \\
\ooo Choose a witness to the existential equal to $u' \cup \setof{o}$ and consider the prestate \\
\ooo $G(\theta \otimes R(\setof{i}, \omega_n) \otimes \psi)$ \\
\ooo We know $R(\setof{i}, \omega_n) = \theta \otimes \localref{r}{n} \otimes R(\setof{i}, \phi_k) \otimes R(\setof{i}, \sigma_{n-k}) \otimes \delta_n$ \\
\ooo Now, we know that $\unready{\theta \otimes \omega_n \otimes \psi}{o}$, since $\delta_n$'s read set \\
\ooo contains $c$, which is unready \\
\ooo So to evaluate $\readcell o$, we need to use the \textsc{AUnready} rule \\
\ooo So we need to evaluate $e$ in the state \\
\ooo $G(\theta \otimes \localref{r}{n} \otimes R(\setof{i}, \phi_k) \otimes R(\setof{i}, \sigma_{n-k}) \otimes \psi)$ \\
\ooo We begin with the \textsc{ABind} rule, which asks us to evaluate $\getref r$, which gives us \\
\ooo $G(\theta \otimes \localref{r}{n} \otimes R(\setof{i}, \phi_k) \otimes R(\setof{i}, \sigma_{n-k}) \otimes \psi) \land (n_1,r_1) = (n,\emptyset)$ \\
\ooo In this case, $e$ in $\delta_n$ simplifies to \\
\oooo $e = \bind (\getref r)\; (\semfun{n}{\bind (\setref r\;(n+1))\; (\semfun{\unit}{\readcell c})})$ \\
\ooo So now we need to evaluate $\bind (\setref r\;(n+1))\; (\semfun{\unit}{\readcell c})$ in \\
\ooo $G(\theta \otimes \localref{r}{n+1} \otimes R(\setof{i}, \phi_k) \otimes R(\setof{i}, \sigma_{n-k}) \otimes \psi)$ \\
\ooo Using the \textsc{ABind} rule, we need to evaluate $\setref r\;(n+1)$, which yields  \\
\ooo $G(\theta \otimes \localref{r}{n+1} \otimes R(\setof{i}, \phi_k) \otimes R(\setof{i}, \sigma_{n-k}) \otimes \psi)$ \\
\ooo with the need to consider $\readcell c$ \\
\ooo Using the triple we derived above, we get \\
\ooo $G(\theta' \otimes \localref{r}{n+1} \otimes R(\setof{i}, \phi_k) \otimes \sigma_{n-k+1} \otimes R(u', \psi))
      \land a = ((g\;(\mathit{drop}\;k\;v))_{n-k}, \setof{c})$ \\
\ooo Here, we have used the fact that $\phi_k$ is closed with respect to $u'$ to simplify the state\\
\ooo Hiding the equality, and then finishing the \textsc{AUnready} rule, we have a state \\
\ooo $G(R(\setof{o}, \theta' \otimes \localref{r}{n+1} \otimes R(\setof{i}, \phi_k) \otimes R(u' \cup \setof{i}, \sigma_0) \otimes R(u', \psi)) \otimes \delta_{n+1})$\\
\ooo Simplifying using the closure properties, we get \\
\ooo $G(\theta' \otimes \localref{r}{n+1} \otimes R(\setof{i}, \phi_k) \otimes R(u' \cup \setof{i}, \sigma_0) \otimes \delta_{n+1} \otimes R(u' \cup \setof{o}, \psi))$\\
\ooo which is \\
\ooo $G(\theta' \otimes \localref{r}{n+1} \otimes R(\setof{i}, \phi_k) \otimes R(u' \cup \setof{i}, \sigma_0) \otimes \delta_{n+1} \otimes R(u' \cup \setof{o}, \psi)))$\\
\ooo Note that $(\mathit{switch}\;k\;f\;g\;v)_n = (g\;(\mathit{drop}\;k\;v))_{n-k}$, so restoring the equality \\
\oox we get our goal
\end{tabbedproof}
\end{proof}

\subsubsection{Correctness of $\mathsf{loop}$}

\begin{lemma}
  Suppose that for all $xys$, we have $\mathit{Transduce}(xys, s,
  \phi, t, f\;xys)$. Then it is the case that for all $ys$ and $x$ we
  have $\mathit{Transduce}(ys, i, \sigma, o, \mathit{loop}\;x\;f\;ys)$, where

  \begin{itemize}
  \item $\sigma_n = \localref{r}{(x \cdot \mathit{map}\;\pi_1\;(\mathit{cycle}\;x\;f\;ys))_n} \otimes
                      \iota_n \otimes \phi_n \otimes \delta_n$ 
  \item $\iota_n = \cellpos{s}{e_1}{ws_n}{\setof{i}}$ where $ws_{n+1} = ((x \cdot \mathit{map}\;\pi_1\;(\mathit{cycle}\;x\;f\;ys))_n, ys_n)$ and $ws_0$ is some dummy value\\
  \item $e_1 = \bind (\readcell i)\; \semfun{y}{\bind (\getref r)\; \semfun{x}{\return (x,y)}}$ \\
  \item $\delta_n = \cellpos{o}{e_2}{(z\cdot (\mathit{loop}\;x\;f\;ys))_n}{\setof{t}}$ for some dummy $z$ \\ 
  \item $e_2 = \bind (\readcell t)\; (\semfun{(x,z)}{
               \bind (\setref r\;x)\; (\semfun{\unit}{\return z}}))$ \\
  \end{itemize}
\end{lemma}

\begin{proof}
  \begin{itemize}
  \item First, we want to prove that if $\unready{\theta}{i}$ then $\unready{\theta \otimes \sigma_n}{o}$ 

    \begin{tabbedproof}
      \oo Assume $\unready{\theta}{i}$ \\
      \oo Hence $\unready{\theta \otimes \iota_n}{s}$ \\
      \oo Hence $\unready{\theta \otimes \iota_n \otimes \phi_n}{t}$ \\
      \oo Hence $\unready{\theta \otimes \iota_n \otimes \phi_n \otimes \delta_n}{o}$ \\
    \end{tabbedproof}

  \item Now, we want to show that if $\ready{\theta}{i}{ys_n}$, then $\ready{\theta \otimes \sigma_{n+1}}{o}{(\mathit{loop}\;x\;f\;ys)_n}$ 
    \begin{tabbedproof}
      \oo Assume $\ready{\theta}{i}{ys_n}$ \\
      \oo Hence $\ready{\theta \otimes \iota_n}{s}{((\mathit{map}\;\pi_1\;(\mathit{cycle}\;x\;f\;ys))_n, ys_n)}$ \\
      \oo Hence $\ready{\theta \otimes \iota_n \otimes \phi_n}{t}{(f\;(\mathit{zip}\;(\mathit{map}\;\pi_1\;(\mathit{cycle}\;x\;f\;ys))\;ys))_n}$ \\
      \oo Hence $\ready{\theta \otimes \iota_n \otimes \phi_n \otimes \delta_n}{o}{(\mathit{loop}\;x\;f\;ys)_n}$ \\
    \end{tabbedproof}

  \item Now, we want to show that $\closed{\sigma_n}{\domain{\sigma_n} \cup \setof{i}}$ 

    \begin{tabbedproof}
      \oo First, observe that $\closed{\iota_n}{\setof{i} \cup \domain{\iota_n}}$\\
      \oo Next, by hypothesis $\closed{\phi_n}{\domain{\phi_n} \cup \setof{s}}$ \\
      \oo Note that $s \in \domain{\iota_n}$ \\
      \oo Next, note that $\closed{\delta_n}{\setof{t} \cup \domain{\delta_n}}$ \\
      \oo Note that $t \in \domain{\phi_n}$ \\
      \oo Hence $\closed{\sigma_n}{\setof{i} \cup \domain{\sigma_n}}$
    \end{tabbedproof}
  \item Next, we want to show that $o \in \domain{\sigma_n}$. This is obvious, since $\delta_n$ is part 
    of $\sigma_n$. 
  \item Next, we want to show that $\unready{R(\setof{i}, \sigma_n)}{o}$
    
    \begin{tabbedproof}
      \oo $R(\setof{i}, \iota_n) = \cellneg{s}{e_1}$, so $\unready{R(\setof{i}, \iota_n)}{s}$ \\
      \oo Hence by hypothesis, we know $\unready{R(\setof{i}, \iota_n) \otimes R(\setof{i}, \phi_n)}{t}$ \\
      \oo So we know that $\unready{R(\setof{i}, \iota_n) \otimes R(\setof{i}, \phi_n) \otimes \delta_n}{o}$ \\
      \oo But $R(\setof{i}, \delta_n) = \delta_n$ \\
      \oo Furthermore, $R(\setof{i}, \localref{r}{\ldots}) = \localref{r}{\ldots}$ \\
      \oo Hence $\unready{R(\setof{i}, \sigma_n)}{o}$ 
    \end{tabbedproof}

  \end{itemize}

Now, we'll prove the entailment of implications. 
\begin{tabbedproof}
\oo Assume we have $\theta, \theta', n$, and $u$. \\
\oo Assume that either $\ready{\theta}{i}{y_n}$ and $\theta = \theta'$ and $u = \emptyset$,  \\
\oo or $\unready{\theta}{i} \land i \in u$ \\
\oo Assume that $\closed{\theta'}{\domain{\theta'}}$ \\
\oo Assume that $\ready{\theta'}{i}{y_n}$ and $\closed{\theta'}{\domain{\theta'}}$ \\
\oo Assume for all $\psi$, we have \\
\ox  $\mspec{G(\theta \otimes \psi)}{\readcell i}{a}{a = ((f\;v)_n, \setof{c}) 
      \land G(\theta' \otimes R(u,\psi)) }$ \\
\oo Now, we'll consider the effect of reading $s$ \\
\ooo We know that $\unready{\theta \otimes \localref{r}{(x \cdot \mathit{map}\;\pi_1\;(\mathit{cycle}\;x\;f\;ys))_n} \otimes R(i, \iota_n)}{s}$ \\
\ooo Using the \textsc{AUnready} rule, we can derive the following triple: \\
\ooo $\left<G(\theta \otimes \localref{r}{(x \cdot \mathit{map}\;\pi_1\;(\mathit{cycle}\;x\;f\;ys))_n} \otimes R(\setof{i}, \iota_n) \psi)\right>$ \\
\ooo $\readcell s$ \\
\ooo $\left<a.\; G(\theta' \otimes \localref{r}{(x \cdot \mathit{map}\;\pi_1\;(\mathit{cycle}\;x\;f\;ys))_n} \otimes \iota_{n+1} \otimes R(u \cup {s},\psi))\right.$ \\
\ooo $\left.\;\;\land\;a = ((\pi_1(\mathit{cycle}\;x\;f\;ys)_n, y_n), \setof{s}) \right>$ \\
\ooo using the fact that $s \not\in \domain{\theta}$ and its closure \\
\oo Furthermore, it's clear that $u \cup \setof{s} \subseteq \domain{\theta \otimes \localref{r}{\ldots} \otimes R(\setof{i}, \iota_n)}$\\
\oo and that $\unready{\theta \otimes \localref{r}{\ldots} \otimes R(\setof{i}, \iota_n)}{s}$ \\
\oo and that $s \in \domain{\theta \otimes \localref{r}{\ldots} \otimes R(\setof{i}, \iota_n)}$\\
\oo So we can then use $\mathit{Transduce}(xys, s, \phi, t, f\;xys)$ to conclude that for all $\psi$\\
\ooo $\left<G(\theta \otimes \localref{r}{(x \cdot \mathit{map}\;\pi_1\;(\mathit{cycle}\;x\;f\;ys))_n} \otimes R(\setof{i}, \iota_n \otimes \phi_n) \otimes \psi)\right>$\\
\ooo $\readcell t$ \\
\ooo $\left<G(\theta \otimes \localref{r}{(x \cdot \mathit{map}\;\pi_1\;(\mathit{cycle}\;x\;f\;ys))_n} \otimes \iota_{n+1} \otimes \phi_{n+1} \otimes R(u \cup \setof{s} \cup u', \psi))\right.$\\
\ooo $\;\;\land\;\left.a = ((f\;(\mathit{zip}\;(x \cdot \mathit{map}\;\pi_1\;(\mathit{cycle}\;x\;f\;ys))\;y))_n, \setof{t})\right>$\\
\ooo By the loop unrolling lemma, this postcondition is equivalent to  \\
\ooo $\left<G(\theta \otimes \iota_{n+1} \otimes \phi_{n+1} \otimes R(u \cup \setof{s} \cup u', \psi))
            \land a = ((\mathit{cycle}\;x\;f\;ys)_n, \setof{t})\right>$\\
\oo Take the witness to be $u' \cup \setof{s, t}$ \\
\oo Now, assume we have $\psi$, and a prestate \\
\oo $G(\localref{r}{\pi_1(\mathit{cycle}\;x\;f\;ys)_n} \otimes R(\setof{i}, \iota_n \otimes \phi_n \otimes \delta_n) \otimes \psi)$ \\
\oo Since $\unready{\localref{r}{(x \cdot \mathit{map}\;\pi_1\;(\mathit{cycle}\;x\;f\;ys))_n} \otimes \theta \otimes R(\setof{i}, \iota_n)}{s}$, we have \\
\oox $\unready{\localref{r}{(x \cdot \mathit{map}\;\pi_1\;(\mathit{cycle}\;x\;f\;ys))_n} \otimes \theta \otimes R(\setof{i}, \iota_n \otimes \phi_n)}{t}$, so we have \\
\oox $\unready{\localref{r}{(x \cdot \mathit{map}\;\pi_1\;(\mathit{cycle}\;x\;f\;ys))_n} \otimes \theta \otimes R(\setof{i}, \iota_n \otimes \phi_n \otimes \delta_n)}{o}$, so we have \\
\oox $\unready{\localref{r}{(x \cdot \mathit{map}\;\pi_1\;(\mathit{cycle}\;x\;f\;ys))_n} \otimes \theta \otimes R(\setof{i}, \iota_n \otimes \phi_n \otimes \delta_n) \otimes \psi}{o}$ \\
\oo So to read $o$, we should use the \textsc{AUnready} rule. So we want to evaluate $e_2$ in a state\\
\oo $G(\localref{r}{(x \cdot \mathit{map}\;\pi_1\;(\mathit{cycle}\;x\;f\;ys))_n} \otimes R(\setof{i}, \iota_n \otimes \phi_n) \otimes \psi)$\\
\oo Now by \textsc{ABind} we need to consider $\readcell t$, and from our derivation above \\
\oo $G(\localref{r}{(x \cdot \mathit{map}\;\pi_1\;(\mathit{cycle}\;x\;f\;ys))_n} \otimes \iota_{n+1} \otimes \phi_{n+1} \otimes R(u \cup u' \cup \setof{s}, \psi))$\\
\oo and furthermore the return value is $((\mathit{loop}\;x\;f\;ys)_n, \setof{t})$ \\
\oo Calling $(x, z) = (\mathit{cycle}\;x\;f\;ys)_n$, we need to consider $\bind (\setref r\;x)\; (\semfun{\unit}{\return z})$ \\
\oo Using the \textsc{ABind} rule and first the \textsc{ASetRef} rule, we see that \\
\oo $G(\localref{r}{\pi_1\;(\mathit{cycle}\;x\;f\;ys)_n} \otimes \iota_{n+1} \otimes \phi_{n+1} \otimes R(u \cup u' \cup \setof{s}, \psi))$\\
\oo By equational reasoning, this is equivalent to \\
\oo $G(\localref{r}{(x \cdot \mathit{map}\;\pi_1\;(\mathit{cycle}\;x\;f\;ys))_{n+1}} \otimes \iota_{n+1} \otimes \phi_{n+1} \otimes R(u \cup u' \cup \setof{s}, \psi))$\\
\oo Finishing with $\return z$, we get the poststate: \\
\oo $G(R(\setof{t}, \localref{r}{(x \cdot \mathit{map}\;\pi_1\;(\mathit{cycle}\;x\;f\;ys))_{n+1}} \otimes \iota_{n+1} \otimes \phi_{n+1} \otimes R(u \cup u' \cup \setof{s}, \psi)) \otimes \delta_{n+1})$\\
\oox $\land\; a = (z, \setof{t})$\\
\oo Note $z = \pi_2\;(\mathit{cycle}\;x\;f\;ys)_n = (\mathit{map}\;\pi_2\;(\mathit{cycle}\;x\;f\;ys))_n = (\mathit{loop}\;x\;f\;ys)_n$ \\
\oo Furthermore, the poststate simplifies to \\
\oo $G(\localref{r}{(x \cdot \mathit{map}\;\pi_1\;(\mathit{cycle}\;x\;f\;ys))_{n+1}} \otimes \iota_{n+1} \otimes \phi_{n+1} \otimes R(u \cup u' \cup \setof{s,t}, \psi)) \otimes \delta_{n+1})$ \\
\oox $\land\; a = (z, \setof{t})$\\
\oo using the closure properties of $\phi, \theta, \delta,$ and $\iota$, which is what we want.
\end{tabbedproof}
\end{proof}

\subsubsection{Correctness of Realizability}

\begin{prop*}{(FRP Correctness)}
Then, the following specifications are provable: 
\begin{enumerate}
\item $\forall f:A\to B.\; \mathit{Realize}(\liftop f, \mathit{lift}\;f)$
\item $\forall p, f, q, g.\;\mathit{Realize}(p, f) \specand \mathit{Realize}(q, g) \specimp \mathit{Realize}(\composeop p\;q, \mathit{compose}\;f\;g)$
\item $\forall p, f, q, g.\;\mathit{Realize}(p, f) \specand \mathit{Realize}(q, g)
                       \specimp \mathit{Realize}(\parop p\;q, par\;f\;g)$
\item $\forall k, p, f, q, g.\;\mathit{Realize}(p, f) \specand \mathit{Realize}(q, g)
                          \specimp \mathit{Realize}(\switchop k\;p\;q, switch\;k\;f\;g)$
\item $\forall a_0, p, f.\; \mathit{Realize}(p, f) \specimp \mathit{Realize}(\loopop\;a_0\;p, loop\;a_0\;f)$
\end{enumerate}
\end{prop*}

\begin{proof}
  Since each of these combinators are straight line code, the
  correctness proof consists of nothing but building the dataflow
  graph and observing that the resulting graph matches the state
  posited in the transduction lemmas we just proved.

  The only even marginally interesting part of these proofs comes in
  the use of the rule of consequence to take newly-created cells (whose
  predicates are of the form $\cellneg{c}{e}$), and turn them into
  $\cellpos{c}{e}{\mathit{dummy}}{rs}$.
\end{proof}

\section{Future Work}

Now that we have introduced the idea of ramifications and seen two
different applications for them we see many possible further uses.  It
would be interesting to investigate the relationship between
ramification operators and methods based on
rely-guarantee~\cite{rely-guarantee-jones}. Rely-guarantee works by
imposing a mutual contract between a piece of code and the rest of the
world, which is at least conceptually similar to the idea of a
ramification, though we see no obvious direct relationship.

Third, so far we have presented ramifications as a style of
specification useful for verifying a particular library.  Might it be
sensible or useful to bake ramification operators into the basic
logical framework?  If so, what are their logical properties --- $R(u,
\phi)$ looks like a family of modal operators on the formula $\phi$.
However, our first two examples exhibited a number of different
auxiliary interaction lemmas, which is what made them truly useful, 
and the common features are still unclear. 

\section{Related Work}

Prior work on verifying callbacks using separation logic includes both
\citet{tldi09,ftfjp07} work as well as by
\citet{parkinson-iwaco-07}. The approach in these papers was similar;
each subject predicate was equipped with a list of observers, rather
than maintaining them implicitly in the invariant. These approaches
were not modular, because the observers had to be explicitly
named. Worse still, this approach broke down when faced with chains of
subjects and observers, because separation logic is a resource-based
logic, and the existence of multiple paths to an object meant that
ownership of observer state became ambiguous.

\citet{shaner-leavens-naumann} studied using gray-box model programs
to model higher-order method calls (which can be understood as a a
variant of refinement calculus-based methods) in JML, and
\citet{history-invariants} have applied Liskov and Wing's idea of
history invariants~\cite{liskov-wing} to model observers. As before,
both of these methods are also non-modular, since they require knowing
what all of the observers are, and furthermore both of these methods
sharply restrict the kinds of invariants that can be used, making it
very difficult to model the code-update-based protocol seen in our FRP
example.

\citet{self-adjusting} have proposed
\emph{self-adjusting computation} as a technique for using change
propagation to write programs that incrementally
recompute answers as the inputs are adjusted. It would be worth 
studying if our techniques could help verifying implementations of this
idea.

Functional reactive programming (FRP) was proposed by \citet{frp} as a
declarative formalism for interactive programming. The API in our
paper differs from theirs in two ways. First, our interface is a
variation of the \emph{arrowized FRP} interface proposed in
\cite{afrp}, and secondly, we use a discrete model of time, rather
than a continuous model of time --- though we found the core idea of
using a declarative semantics as a specification for the interface an
inspiring one.

Furthermore, our work could also serve as a bridge between the work on
purely functional reactive programming, and more imperative
implementations, such as the work done by \citet{superglue} on
SuperGlue and by \citet{frtime} on the FrTime system.


\section{Appendix: The $\cellset$ Interface}

In this section, we'll describe the interface to the $\cellset$ type,
which is intended to be used to represent pure collections of
existentially quantified cells. Specifying this interface is not
entirely trivial, because of the way equality works for this
type. Ordinarily, we would simply give a two-place predicate $set(v,
\elts)$ relating a value $v$, and the mathematical set of elements
$\elts$ it represents.

However, this approach is not sufficient in our case. In order to
manage dependencies, we need to be able to test cells of
\emph{different} concrete type for equality, and the natural equality
test for references only permits testing references of the same
type. As a result, we can't unpack an existentially-quantified cell
and compare the elements in its tuple, because we don't know that the
two cells are of the same type (and indeed, they might not be).

To get around this problem, we generate a unique integer identifier
for each cell we create, and then compare those identifiers to
establish equality. Since these identifiers are all generated
dynamically along with the cells, this means that the precise partial
equivalence relation we need to use is determined dynamically as
well. So we add an additional index to the set predicate $set(W, v,
\elts)$. The extra parameter $W$ is the \emph{world}, the set of all
the cells allocated so far, which must collectively satisfy the
constraint that two elements of the world are equal if and only if their
identifier fields match.

All of the usual set operations exist. We have an $\ctext{emptyset}$,
representing an empty set of cells, as well as singleton
($\ctext{singleton}$) and union ($\ctext{union}$) operations.  We have
$\ctext{addset}\;v\;x$, which adds the element $x$ to the set $v$
represents, and $\ctext{removeset}\;v\;x$, which removes $x$ from the
set $v$ represents. We also have $\ctext{iterset}\;f\;v$, which
iterates over the elements of $v$'s set and applies $f$ to each
element in some sequential order. (Observe that the specification
makes use of two auxiliary predicates: $matches$, which assert that a
set and a list have the same elements; and $iterseq$, which constructs
a command representing the sequential execution of those elements.)

We have three axioms that our implementation must satisfy. First, if a
$\cellset$ value $v$ represents a set in a world $W$, it will also
represent a set in any larger world $W'$. Second, the values in a set
are always a subset of the world $W$. Finally, we require that
$set(W,v,\elts)$ is a pure predicate (i.e., is not heap-dependent),
which implies that it have a purely functional implementation (for
example, as a binary tree). 

\begin{specification}
$World = $ \nextline
\;\;\;\;\=$\{D \in \powersetfin{\ecell}\;|\;\forall (\pack{\alpha}{c_0} \; \mathrm{as}\; c),(\pack{\beta}{d_0} \;\mathrm{as}\; d) \in D.\;$\=$c_0.id = d_0.id \iff c = d\}$\nextline[1em]

$\exists \cellset : \star.$ \nextline
$\exists set : World \To \cellset \To \powersetfin{\ecell} \To \assert.$ \nextline
$\exists \ctext{emptyset}    : cellset.$ \nextline
$\exists \ctext{singleton}   : \ecell \to \cellset.$ \nextline
$\exists \ctext{union}      : \cellset \to \cellset \to \cellset.$ \nextline
$\exists \ctext{addset}      : \cellset \to \ecell \to \cellset.$ \nextline
$\exists \ctext{removeset}   : \cellset \to \ecell \to \cellset.$ \nextline
$\exists \ctext{iterset}     : (\ecell \to \monad{\unittype}) \to \cellset \to \monad{\unittype}.$\nextline[1em]

$\forall W \in World.\; set(W, \ctext{emptyset}, \emptyset)$ \nextline[1em]

$\forall W \in World, c \in W.\; set(W, \ctext{singleton}\;c, \setof{c})$ \nextline[1em]

$\forall W \in World, c_1 : \cellset, \elts_1 \in \powersetfin{W}, c_2 : \cellset, \elts_2 \in \powersetfin{W}.$ \nextline
\;\;$set(W, c_1, \elts_1) \land set(W, c_2, \elts_2) \implies set(W, \ctext{union}\;c_1\;c_2, \elts_1 \cup \elts_2)$ \nextline[1em]


$\forall W \in World, v : \cellset, x : \ecell, \elts \in \powersetfin{\ecell}.$ \nextline 
\> $set(W, v, \elts) \land x \in W \implies set(W, \ctext{addset}\;v\;x, \elts \cup \setof{x})$ \nextline[1em]


$\forall W \in World, v : \cellset, x : \ecell, \elts \in \powersetfin{\ecell}.$ \nextline 
\> $set(W, v, \elts) \land x \in W \implies set(W, \ctext{removeset}\;v\;x, \elts - \setof{x})$ \nextline[1em]


$\forall W \in World, v : \cellset, \elts \in \powersetfin{\ecell}, 
         f : (\ecell \to \monad{\unittype}).$ \nextline 
\> $set(W, v, \elts) \implies \exists L : \seqsort{\ecell}.\;$\=$matches\;\elts\;L\; \land \ctext{iterset}\;f\;v = iterseq\;f\;L$ \nextline[1em]


$\forall W, W' \in World, v, \elts.$ \nextline
\>$set(W,v,\elts) \land W \subseteq W' \implies set(W',v, \elts)$\nextline[1em]

$\forall W \in World, v, \elts.$ \nextline
\>$set(W, v, \elts) \implies \elts \subseteq W$ \nextline[1em]

$\forall W, v, \elts.\; \mathrm{Pure}(set(W,v,\elts))$ \nextline[1em]
  
$matches : \powerset{\ecell} \to seqsort{\ecell} \to \assert$ \nextline
$matches\;\elts\;[] \qquad\;\;\; = \elts = \emptyset$ \nextline
$matches\;\elts\;(v :: vs) = v \in \elts \land matches\;(\elts - \setof{v})\;vs$\nextline[1em]

$iterseq : (\ecell \to \monad{\unittype}) \to \seqsort{\ecell} \to \monad{\unittype}$ \nextline
$iterseq\; f\; [] \qquad\;\;\;\;$\=$= \comp{\unit}$ \nextline
$iterseq\; f\; (v :: vs)$\>$= \comp{\letv{\unit}{f\;v}{\ctext{run}\;iterseq\;f\;vs}}$
\end{specification}



%%% Local Variables: 
%%% mode: latex
%%% TeX-master: "thesis"
%%% End: 

\end{sloppypar}
%\appendix
%\include{appendix}

\backmatter

\bibliography{thesis-doc}
\bibliographystyle{plainnat}{}

% 
% %\renewcommand{\baselinestretch}{1.0}\normalsize
% 
% % By default \bibsection is \chapter*, but we really want this to show
% % up in the table of contents and pdf bookmarks.
% \renewcommand{\bibsection}{\chapter{\bibname}}
% %\newcommand{\bibpreamble}{This text goes between the ``Bibliography''
% %  header and the actual list of references}
% \bibliographystyle{plainnat}
% \bibliography{register} %your bib file

\end{document}
