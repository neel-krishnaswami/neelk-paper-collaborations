\chapter{Type Structure}

\section{Syntax of Types and Kinds}

In this section, I will describe the type structure of the 
programming language I will use for this thesis. The language is a
pure, total, predicatively polymorphic programming language (with
quantification over higher kinds), augmented with a monadic type
constructor that permits nontermination, higher-order state, and
continuations. Its syntax is given in figure~\ref{lang-type-syntax}. 


\begin{figure}[h]
\todo{Add support for polynomial data types}

\begin{displaymath}
  \begin{array}{lcll}
    \mbox{Kinds} & 
      \kappa & ::= & \star \bnfalt \kappa \to \kappa 
    \\[1em]
    \mbox{Monotypes} & 
      \tau & ::= & 
         \unittype \bnfalt 
         \tau \times \tau \bnfalt 
         \tau \to \tau \bnfalt 
         \tau + \tau \bnfalt
         \N \bnfalt \\
     &&& \reftype{A} \bnfalt
         \monad{\tau} \bnfalt 
         \cont{\tau} \\
     &&& \alpha \bnfalt
         \tau\;\tau \bnfalt 
         \fun{\alpha}{\kappa}{\tau} 
    \\[1em]
    \mbox{Polytypes} & 
      A & ::= & 
         \unittype \bnfalt 
         A \times B \bnfalt 
         A + B \bnfalt
         A \to B \bnfalt 
         \N \bnfalt  \\
    &&&  \reftype{A} \bnfalt
         \monad{A} \bnfalt 
         \cont{A} \\
    &&&  \alpha \bnfalt
         \tau\;\tau \bnfalt \\
    &&&  \forall \alpha:\kappa.\; A \bnfalt 
         \exists \alpha:\kappa.\; A \\[1em]
    \mbox{Type Contexts} & 
      \Theta & ::= & \cdot \bnfalt \Theta, \alpha:\kappa \\
  \end{array}
\end{displaymath}
\caption{Syntax of the Core Programming Language}
\label{lang-type-syntax}
\end{figure}

The basic kind structure of the language is given with the kinds
$\kappa$. The The kind $\star$ is the kind of monotypes, which are
written with the letter $\tau$. These are the unit type $\unittype$,
pair types $\tau \times \sigma$, sums $\tau + \sigma$, function space
$\tau \to \sigma$, natural numbers $\N$, computation types
$\monad{\tau}$, continuation types $\cont{\tau}$ and finally
(ML-style) references $\reftype{A}$. Also, within open types we may
also use type variables $\alpha$ to refer to types, and we can also
define lambda-abstractions and applications to inhabit the higher
kinds of this language.

The type $\reftype{A}$ is not merely a pointer to a value of
monomorphic type; instead, it also permits storing a pointer to a
value of polymorphic types $A$. The intuition justifying this
liberality is that a reference to a value of polymorphic type is
itself merely a location, so it is permissible to treat a reference to
a value of polymorphic type as a monomorphic value. Indeed, the
denotational semantics of references will rely upon this intuition.

The polytypes $A$ themselves extend the monotypes with universal
quantification $\forall \alpha:\kappa.\;A$ as well as existential
types $\exists \alpha:\kappa.\;A$. Each of the simple type
constructors -- sums, products, functions, computations -- also may
contain polymorphic types as subexpressions within it. Though it may
not look like it at first glance, this is actually only a fairly
modest generalization of the type schemes of ML. Because the universal
and existential quantifiers range over the kinds $\kappa$, it is
impossible to instantiate them with a polytype, thereby limiting us to
predicative polymorphism.

The kinding judgements $\judgeWK{\tau}{\kappa}$ and
$\judgeWK{A}{\bigstar}$ determine whether a monotype or polytype
(respectively). Notice that we use two judgements, because we do not
give polytypes their own first-class kind $\bigstar$ -- this means
that we can syntactically enforce the restriction that the context
$\Theta$, which is a collection of type variables and their kinds, 
can only range over kinds which can contain only monotypes. 

We give the kinding rules for the monotypes in
figure~\ref{monotype-kinding}, and the kinding rules for polytypes
in figure~\ref{polytype-kinding}. 


\begin{figure}[h]
\begin{mathpar}
\inferrule*[right=KUnit]
          { }
          {\judgeWK{\unittype}{\star}}
\and
\inferrule*[right=KProd]
          {\judgeWK{\tau}{\star} \\
           \judgeWK{\sigma}{\star}}
          {\judgeWK{\tau \times \sigma}{\star}}
\and
\inferrule*[right=KArrow]
          {\judgeWK{\tau}{\star} \\
           \judgeWK{\sigma}{\star}}
          {\judgeWK{\tau \to \sigma}{\star}}
\and
\inferrule*[right=KSum]
          {\judgeWK{\tau}{\star} \\
           \judgeWK{\sigma}{\star}}
          {\judgeWK{\tau + \sigma}{\star}}
\and
\inferrule*[right=KNat]
          { }
          {\judgeWK{\N}{\star}}
\and
\inferrule*[right=KRef]
          {\judgeWK{A}{\bigstar}}
          {\judgeWK{\reftype{A}}{\star}}
\and
\inferrule*[right=KComp]
          {\judgeWK{\tau}{\star}}
          {\judgeWK{\monad{\tau}}{\star}}
\and
\inferrule*[right=KCont]
          {\judgeWK{\tau}{\star}}
          {\judgeWK{\cont{\tau}}{\star}}
\\
\inferrule*[right=KHyp]
          { \alpha:\kappa \in \Theta }
          { \judgeWK{\alpha}{\kappa} }
\and
\inferrule*[right=KApp]
          { \judgeWK{\tau}{\kappa' \to \kappa} \\
            \judgeWK{\tau'}{\kappa'} }
          { \judgeWK{\tau\;\tau'}{\kappa} }
\and
\inferrule*[right=KLam]
          { \judgeWK[\Theta, \alpha:\kappa']{\tau}{\kappa} }
          { \judgeWK{\fun{\alpha}{\kappa'}{\tau}}{\kappa' \to \kappa} }
\end{mathpar}
\caption{Kinding Rules for Monotypes}
\label{monotype-kinding}
\end{figure}

\begin{figure}[h]
\begin{mathpar}
\inferrule*[right=KUnit]
          { }
          {\judgeWK{\unittype}{\bigstar}}
\and
\inferrule*[right=KProd]
          {\judgeWK{A}{\bigstar} \\
           \judgeWK{B}{\bigstar}}
          {\judgeWK{A \times B}{\bigstar}}
\and
\inferrule*[right=KArrow]
          {\judgeWK{A}{\bigstar} \\
           \judgeWK{B}{\bigstar}}
          {\judgeWK{A \to B}{\bigstar}}
\and
\inferrule*[right=KSum]
          {\judgeWK{A}{\bigstar} \\
           \judgeWK{B}{\bigstar}}
          {\judgeWK{A + B}{\bigstar}}
\and
\inferrule*[right=KNat]
          { }
          {\judgeWK{\N}{\bigstar}}
\and
\inferrule*[right=KRef]
          {\judgeWK{A}{\bigstar}}
          {\judgeWK{\reftype{A}}{\bigstar}}
\and
\inferrule*[right=KComp]
          {\judgeWK{A}{\bigstar}}
          {\judgeWK{\monad{A}}{\bigstar}}
\and
\inferrule*[right=KCont]
          {\judgeWK{A}{\bigstar}}
          {\judgeWK{\cont{A}}{\bigstar}}
\\
\inferrule*[right=KHyp]
          { \alpha:\star \in \Theta }
          { \judgeWK{\alpha}{\bigstar} }
\and
\inferrule*[right=KApp]
          { \judgeWK{\tau}{\kappa' \to \star} \\
            \judgeWK{\tau'}{\kappa'} }
          { \judgeWK{\tau\;\tau'}{\bigstar} }
\\
\inferrule*[right=KForall]
           { \judgeWK[\Theta, \alpha:\kappa]{A}{\bigstar} }
           { \judgeWK{\forall \alpha:\kappa.\;A}{\bigstar} }
\and
\inferrule*[right=KExists]
           { \judgeWK[\Theta, \alpha:\kappa]{A}{\bigstar} }
           { \judgeWK{\exists \alpha:\kappa.\;A}{\bigstar} }
\end{mathpar}
\caption{Kinding Rules for Polytypes}
\label{polytype-kinding}
\end{figure}


Types also support a pair of equality judgements
$\judgeKeq{\tau}{\tau'}{\kappa}$ and $\judgeKeq{A}{B}{\bigstar}$.  The
equality judgement for monotypes implements the $\beta$- and
$\eta$-equality principles of the lambda calculus, along with
congruence rules for all of the type constructors of our language.
The only rules we have for the equality judgement for polytypes are
simple congruence rules, plus a recursive call back to the other
equality judgement whenever we need to compare monotyped terms.

Because types and kinds form an instance of the simply typed lambda
calculus, we know that there is a unique $\beta$-normal, $\eta$-long
form for each well-kinded type expression, both for the monotypes and
the polytypes. This means that when we quotient the set of well-kinded
terms by the equality judgement, we know that each equivalence class
contains a single long normal term, which we can take as the canonical
representative of that class.


\begin{figure}
\begin{mathpar}
\inferrule[]
          { }
          {\judgeKeq{1}{1}{\star}}
\and
\inferrule[]
          {\judgeKeq{\tau}{\tau'}{\star} \\
           \judgeKeq{\sigma}{\sigma'}{\star}}
          {\judgeKeq{\tau \times \sigma}{\tau' \times \sigma'}{\star}}
\and
\inferrule[]
          {\judgeKeq{\tau}{\tau'}{\star} \\
           \judgeKeq{\sigma}{\sigma'}{\star}}
          {\judgeKeq{\tau \to \sigma}{\tau' \to \sigma'}{\star}}
\and
\inferrule[]
          {\judgeKeq{\tau}{\tau'}{\star} \\
           \judgeKeq{\sigma}{\sigma'}{\star}}
          {\judgeKeq{\tau + \sigma}{\tau' + \sigma'}{\star}}
\and
\inferrule[]
          { }
          {\judgeKeq{\N}{\N}{\star}}
\and
\inferrule[]
          { \judgeKeq{A}{A'}{\bigstar} }
          { \judgeKeq{\reftype{A}}{\reftype{A'}}{\star} }
\and
\inferrule[]
          { \judgeKeq{\tau}{\tau'}{\star} }
          { \judgeKeq{\monad{\tau}}{\monad{\tau'}}{\star} }
\and
\inferrule[]
          { \judgeKeq{\tau}{\tau'}{\star} }
          { \judgeKeq{\cont{\tau}}{\cont{\tau'}}{\star} }
\and
\inferrule[]
          { \alpha:\kappa \in \Theta }
          { \judgeKeq{\alpha}{\alpha}{\kappa} }
\and
\inferrule[]
          { \judgeKeq{\tau}{\sigma}{\kappa' \to \kappa} \\
            \judgeKeq{\tau'}{\sigma'}{\kappa'}}
          { \judgeKeq{\tau \; \tau'}{\sigma \; \sigma'}{\kappa} }
\and
\inferrule[]
          { \judgeWK{(\fun{\alpha}{\kappa'}{\tau})\;\tau'}{\kappa} }
          { \judgeKeq{(\fun{\alpha}{\kappa'}{\tau})\;\tau'}
                     {[\tau'/\alpha]\tau}
                     {\kappa} }
\and
\inferrule[]
          { \judgeKeq[\Theta, \alpha:\kappa']
                     {\tau\;\alpha}{\sigma\;\alpha}{\kappa} }
          { \judgeKeq{\tau}{\sigma}{\kappa' \to \kappa} }
\end{mathpar}
\caption{Equality Rules for Monotypes}
\label{monotype-equality}
\end{figure}


\begin{figure}
\begin{mathpar}
\inferrule[]
          { }
          {\judgeKeq{1}{1}{\bigstar}}
\and
\inferrule[]
          {\judgeKeq{A}{A'}{\bigstar} \\
           \judgeKeq{B}{B'}{\bigstar}}
          {\judgeKeq{A \times B}{A' \times B'}{\bigstar}}
\and
\inferrule[]
          {\judgeKeq{A}{A'}{\bigstar} \\
           \judgeKeq{B}{B'}{\bigstar}}
          {\judgeKeq{A \to B}{A' \to B'}{\bigstar}}
\and
\inferrule[]
          {\judgeKeq{A}{A'}{\bigstar} \\
           \judgeKeq{B}{B'}{\bigstar}}
          {\judgeKeq{A + B}{A' + B'}{\bigstar}}
\and
\inferrule[]
          { }
          {\judgeKeq{\N}{\N}{\bigstar}}
\and
\inferrule[]
          { \judgeKeq{A}{A'}{\bigstar} }
          { \judgeKeq{\reftype{A}}{\reftype{A'}}{\bigstar} }
\and
\inferrule[]
          { \judgeKeq{A}{A'}{\bigstar} }
          { \judgeKeq{\monad{A}}{\monad{A'}}{\bigstar} }
\and
\inferrule[]
          { \judgeKeq{A}{A'}{\bigstar} }
          { \judgeKeq{\cont{A}}{\cont{A'}}{\bigstar} }
\and
\inferrule[]
          { \alpha:\kappa \in \Theta }
          { \judgeKeq{\alpha}{\alpha}{\kappa} }
\and
\inferrule[]
          { \judgeKeq{\tau \; \sigma}{\tau' \; \sigma'}{\star} }
          { \judgeKeq{\tau \; \sigma}{\tau' \; \sigma'}{\bigstar} }
\and
\inferrule[]
          { \judgeKeq{\Theta, \alpha:\kappa}{A}{B}{\bigstar} }
          { \judgeKeq{\forall \alpha:\kappa.\;A}
                     {\forall \alpha:\kappa.\;B}
                     {\bigstar} }
\and
\inferrule[]
          { \judgeKeq{\Theta, \alpha:\kappa}{A}{B}{\bigstar} }
          { \judgeKeq{\exists \alpha:\kappa.\;A}
                     {\exists \alpha:\kappa.\;B}
                     {\bigstar} }
\end{mathpar}
\caption{Equality Rules for Polytypes}
\label{polytype-equality}
\end{figure}

\section{Semantics of Types}

Now, we want to give a semantics for these types, which we will then
use to interpret terms of our programming language. Repeating the
design criteria mentioned earlier, we need to obey the two principles
below:

\begin{itemize}
\item Our interpretation of types should make all non-monadic types 
  pure. 
\item In particular, I want to treat even nontermination as a side
  effect, in addition to the more-obvious effects of control and state. 
\end{itemize}

The purpose of this choice is twofold. First, it will give us a rich
subset of the language which is total and pure, which will be convenient
when writing assertions about programs -- we will be able to use any pure 
function directly in an assertion, without having to worry about any 
side effects it might have. Second, purity means that we will get a 
very rich equality theory for the language -- both the $\beta$ \emph{and}
$\eta$ laws will hold for all of the types of the programming language,
which will facilitate equational reasoning about the pure part of the 
programming language. 

Because we count nontermination as an effect, our denotational
semantics is in CPO, the category of complete partial orders and
continuous functions between them. In particular, we do not demand
that all domains have least elements -- that is, we only require that
the objects of this category be \emph{predomains}, rather than
domains. This permits us to model pure types as predomains lacking a
bottom element.

In figure~\ref{interp-monotypes}, we give an interpretation of the
closed, canonical monotypes (i.e., monotypes of kind $\star$, with no
free occurrences of type variables within them, and no $\beta$-redexes
within them).  This semantics is additionally parameterized with two
arguments $K^+$ and $K^-$, whose purpose I will explain shortly.

\begin{figure}
\begin{displaymath}
\interpmono{-} : \mbox{Monotype} \to CPO_\bot \times CPO^{op}_\bot \to CPO   
\end{displaymath}
\begin{displaymath}
\begin{array}{lcl}
Loc(A) & = & \N \times \setof{A} \\
Loc    & = & \bigcup \setof{X \;|\; \exists A.\; (\judgeWK[\cdot]{A}{\bigstar}) \;\land\;X = Loc(A)} \\[1em]

\interpmono{\judgeWK[\cdot]{\unittype}{\star}}(K_+, K_-) & = & \setof{*} \\

\interpmono{\judgeWK[\cdot]{\N}{\star}}(K_+, K_-) & = &  \N \\

\interpmono{\judgeWK[\cdot]{\tau \times \sigma}{\star}}(K_+, K_-) & = & 
  \interpmono{\judgeWK[\cdot]{\tau}{\star}}(K_+, K_-) \times 
  \interpmono{\judgeWK[\cdot]{\sigma}{\star}}(K_+, K_-) \\

\interpmono{\judgeWK[\cdot]{\tau + \sigma}{\star}}(K_+, K_-) & = & 
  \interpmono{\judgeWK[\cdot]{\tau}{\star}}(K_+, K_-) + 
  \interpmono{\judgeWK[\cdot]{\sigma}{\star}}(K_+, K_-) \\


\interpmono{\judgeWK[\cdot]{\tau \to \sigma}{\star}}(K_+, K_-) & = & 
  \interpmono{\judgeWK[\cdot]{\tau}{\star}}(K_-, K_+) \to
  \interpmono{\judgeWK[\cdot]{\sigma}{\star}}(K_+, K_-) \\

\interpmono{\judgeWK[\cdot]{\reftype{A}}{\star}}(K_+, K_-) & = & Loc(A) \\

\interpmono{\judgeWK[\cdot]{\cont{\tau}}{\star}}(K_+, K_-) & = & 
    \interpmono{\judgeWK[\cdot]{\tau}{\star}}(K_-,K_+) \to K_+ \\

\interpmono{\judgeWK[\cdot]{\monad{\tau}}{\star}}(K_+, K_-) & = & 
   (\interpmono{\judgeWK[\cdot]{\tau}{\star}}(K_+, K_-) \to K_-) \to K_+ \\[1em]
\end{array}
\end{displaymath}
\caption{Locally Continuous Functor Interpreting Monotypes}
\label{interp-monotypes}
\end{figure}
Before explaining the clauses in detail, I will explain why this
defines a function at all. First, because we are considering closed
terms of kind $\star$, the normalization theorem tells us that any
such term will normalize to one of the cases listed above. In
particular, we will never bottom out at a variable, because the
context is closed. We will never bottom out at a lambda, because we
are considering only the kind $\star$, and we will never bottom out an
application, because there will be room for further beta-reduction in
this case, and by hypothesis we are only considering the normal forms.

This means we cover all of the possibilities in this definition, and
furthermore we know it is well-founed, because all of the recursive
calls to $\interpmono{-}$ are always on immediate subterms of the
type.

This definition defines a functor -- because the category Cat of
categories and functors is itself cartesian closed, we can take the
liberty of writing it as a lambda-expression. Most of the clauses of
this definition should be relatively straightforward --- the main
mystery is that we have parameterized this interpretation by two
arguments $K_+$ and $K_-$, which I will explain when we reach the
continuation and monadic type constructors.  We interpret the unit
type as the one-element, discretely ordered predomain, the natural
number type as the natural numbers with a discrete order, pairs as the
categorical products of $CPO$, sums as coproducts, and functions via
the exponentials of $CPO$.

Reference types $\reftype{A}$ are interpreted as pairs consisting of
natural numbers and the syntactic object $A$. The intuition is that
references are just numbers, together with a type tag saying what
type of value the reference will point to. It's important that we do 
\emph{not} interpret the type tag $A$ in this definition --- a ref cell
is a number plus the purely syntactic object $A$, acting as a label. 
To do otherwise would make our definition circular. 

The first time we use $K$ is when we interpret the type $\cont{\tau}$.
Now, the $K_+$ and $K_-$ arguments are revealed as the positive and
negative occurences of the ``answer type'' of a continuation. That is,
we interpret $\cont{\tau}$ as the function space
$\interpmono{\tau}(K_-, K_+) \to K_+$.  Notice that the positive and
negative occurences trade places on the recursive call, because it
occurs on the left-hand-side of a function space. Likewise, the
monadic type $\monad{\tau}$ is interpreted as a sort of
double-negation: $(\interpmono{\tau}(K_+, K_-) \to K_-) \to K_+$.

However, a careful reader will have noticed that so far our
definitions do not mention state at all. What will ultimately happen
is that we will interpret continuations as functions from heaps to the
two-point Sierpinski lattice $O = \setof{\top, \bot}$. However, since
heaps can contain references to values of polymorphic type, and since
we have not yet defined how the semantics of polymorphic types work, we
have carefully parameterized our functorial semantics of monotypes so
that we do not needed to mention them explicitly. 

To fix this gap, we will give an interpretation of polymorphic types
as an indexed function from a context of closed monokinded type
constructors to a a CPO.  As before, we parameterize by the
continuation arguments, and again define a functor.

\begin{displaymath}
\begin{array}{lcl}\interp{\judgeWK{A}{\bigstar}} &\in& \mathrm{Type Context} \to (CPO_\bot \times CPO^{op}_\bot) \to CPO 
\\[1em]
\interp{\judgeWK{\alpha}{\bigstar}}\;\eta\;(K_+,K_-) & = & 
    \interpmono{\judgeWK[\cdot]{\eta(\alpha)}{\star}}(K_+, K_-) 
\\
\interp{\judgeWK{\tau\;\tau'}{\bigstar}}\;\eta\;(K_+,K_-) & = & 
    \interpmono{\judgeWK[\cdot]{\eta(\tau)\;\eta(\tau')}{\star}}(K_+, K_-) 
\\
\interp{\judgeWK{\forall \alpha:\kappa.\;A}{\bigstar}}\;\eta\;(K_+,K_-) & = & 
    \Pi \tau:\kappa.\; 
        \interp{\judgeWK[\Theta, \alpha:\kappa]{A}{\bigstar}}\;(\eta,\tau)\;(K_+,K_-) 
\\
\interp{\judgeWK{\exists \alpha:\kappa.\;A}{\bigstar}}\;\eta\;(K_+,K_-) & = & 
    \Sigma \tau:\kappa.\; 
        \interp{\judgeWK[\Theta, \alpha:\kappa]{A}{\bigstar}}\;(\eta,\tau)\;(K_+,K_-) 
\\
\interp{\judgeWK{A \times B}{\bigstar}}\;\eta\;(K_+,K_-) & = & 
   \interp{\judgeWK{A}{\bigstar}}\;\eta\;(K_+,K_-) \times
   \interp{\judgeWK{B}{\bigstar}}\;\eta\;(K_+,K_-) 
\\
\interp{\judgeWK{A + B}{\bigstar}}\;\eta\;(K_+,K_-) & = & 
   \interp{\judgeWK{A}{\bigstar}}\;\eta\;(K_+,K_-) +
   \interp{\judgeWK{B}{\bigstar}}\;\eta\;(K_+,K_-) 
\\
\interp{\judgeWK{A \to B}{\bigstar}}\;\eta\;(K_+,K_-) & = & 
   \interp{\judgeWK{A}{\bigstar}}\;\eta\;(K_-,K_+) \to
   \interp{\judgeWK{B}{\bigstar}}\;\eta\;(K_+,K_-) 
\\
\interp{\judgeWK{\unittype}{\bigstar}}\;\eta\;(K_+, K_-) & = & \setof{*} 
\\

\interp{\judgeWK{\N}{\bigstar}}\;\eta\;(K_+, K_-) & = &  \N 
\\
\interp{\judgeWK{\cont{A}}{\bigstar}}\;\eta\;(K_+,K_-) & = & 
   \interp{\judgeWK{A}{\bigstar}}\;\eta\;(K_-,K_+) \to K_+
\\
\interp{\judgeWK{\monad{A}}{\bigstar}}\;\eta\;(K_+,K_-) & = & 
   (\interp{\judgeWK{A}{\bigstar}}\;\eta\;(K_+,K_-) \to K_-) \to K_+
\\
\interp{\judgeWK{\reftype{A}}{\star}}\;\eta\;(K_+, K_-) & = & Loc(A) 
\\
\end{array}
\end{displaymath}

Here, a $\mbox{TypeContext}$ is a tuple of the interpretations of
each kind in the environment $\Theta$, which are just the closed 
canonical forms of that kind. (So $\star$ are just the closed 
monotypes, $\star \to \star$ the closed lambda-terms of that kind,
and so on.)

This definition is also well-founded, since it is defined by a
structural recursion over the canonical forms of the kinding
derivation of $\judgeWK{A}{\bigstar}$. 

Whenever we reach a variable or application case, we can apply the
substitution and invoke the intepretation function for the
monotypes. Universal types $\forall \alpha:\kappa.\;A$ are interpreted
as set-indexed predomains or dependent functions from the set of
closed objects of the kind $\kappa$ into predodmains. Existential
types $\exists \tau:\kappa.\;A$ are interpreted as pairs or dependent
sums: we pair a syntactic monotype with the predomain interpreting the
second component.

The remaining cases essentially repeat the clauses of the definitions
for monotypes, to allow for the possibility that there may be
sub-components of pairs/sums/functions/etc that contain universal or
existential types.

Now we are finally in a position to define the recursive domain equation we 
would like to solve:

\begin{displaymath}
\begin{array}{lcl}
H(K_+,K_-) & = & \Sigma L \subseteq^{fin} Loc.\; 
                    \left(\Pi (l,A) \in L.\; 
                             \interp{\judgeWK[\cdot]{A}{\bigstar}}(\cdot)\;(K_+,K_-) 
                    \right)\\

\mathcal{K}(K_+,K_-) & = & H(K_-,K_+) \to O \\    
\end{array}
\end{displaymath}

We use $H$ to define what heaps mean. A heap is a pair whose first
component is a finite set of allocated locations, together with a map
that takes each location in the set of allocated location and returns
a value of the appropriate type.

The continuation type is defined by the solution to the equation
$\mathcal{K}$, which of type $CPO^{op}_\bot \times CPO_\bot \to
CPO_\bot$. We know that this goes into $CPO_\bot$, since it defines a
map into the two point domain, and a function space into a domain is
itself a domain. So if we can solve the equation $K = \mathcal{K}(K,
K)$, then we can plug $K$ into all our other definitions to interpret
all of the types in our language.

Filling this in, we can understand how continuations of type
$\cont{\tau}$ work: they receive a value of type $\tau$ and a heap,
and then either loop or terminate. The monadic type $\monad{\tau}$ now
can be seen as the state-and-continuation monad, which combines the
effects of the continuation monad and the state monad via a domain 
which looks a bit like $(\tau \to H \to O) \to H \to O$. 

To show that this equation actually has a solution, it suffices to 
show that $F$ is a locally-continuous functor. We'll prove this by 
induction over $\interp{-}$ and $\interpmono{-}$, and then at each
case appeal to a series of lemmas showing that each construction we
use preserves local continuity. 

\section{Local Continuity and Pitts's Theorem}

\subsection{Local Continuity}

A functor $F : CPO_\bot \times CPO^{op}_\bot \to CPO$ is \emph{locally
  continuous} if it preserves the order structure on its arguments.
That is, it must be monotone --- if $f \sqsubseteq f'$ and $g
\sqsubseteq g'$, then $F(f, g) \sqsubseteq F(f', g')$ --- and it
must also preserve limits --- given a pair of chains $f_i, g_i$, 
$\bigsqcup_i F(f_i, g_i) = F(\bigsqcup_i f_i, \bigsqcup_i g_i)$. 

Now, we'll show that the functors we used earlier are all locally 
continuous. 

\begin{lemma}{Local Continuity}
\begin{enumerate}
\item If $F,G : CPO_\bot \times CPO^{op}_\bot \to CPO$ are locally continuous,
      then $\lambda A,B. F(A,B) \times G(A,B)$ is locally continuous.  
\item If $F,G : CPO_\bot \times CPO^{op}_\bot \to CPO$ are locally continuous,
      then $\lambda A,B.\; F(A,B) + G(A,B)$ is locally continuous.  
\item If $F,G : CPO_\bot \times CPO^{op}_\bot \to CPO$ are locally continuous,
      then $\lambda A,B.\;F(A,B) \to G(A,B)$ is locally continuous.  
\item The constant functor $K_C$ is locally continuous. 
\item If X is a set, and $F(x) : CPO_\bot \times CPO^{op}_\bot \to CPO$ is an
      $X$-indexed family of locally continuous functors, then 
      $\lambda (A,B).\; \Pi x:X.\;F(x)(A,B)$ is a locally continuous functor. 
\item If X is a set, and $F(x) : CPO_\bot \times CPO^{op}_\bot \to CPO$ is an
      $X$-indexed family of locally continuous functors, then 
      $\lambda (A,B).\;\Sigma x:X.\;F(x)(A,B)$ is a locally continuous functor. 
\end{enumerate}
\end{lemma}
 
\begin{proof}

\begin{enumerate}
\item Suppose $F$ and $G$ are locally continuous. Now, for $A$ and $B$, we have 
the functor which takes $(A,B)$ to $F(A,B) \times G(A,B)$
on the object part, and which takes $(f,g)$ to $F(f,g) \times G(f,g)$ on
the arrow part.

Next, suppose that we have a chain of functions $\left<f_i\right> : A \to B$ and
$\left<g_i\right> : X \to Y$. Now, we calculate:

\begin{displaymath}
\begin{array}{lcl}
  \sqcup_i (F \times G (f_i,g_i)) 
   & = & \sqcup_i (F(f_i,g_i) \times G(f_i,g_i)) \\
   & = & \sqcup_i \left<F(f_i,g_i) \circ \pi_1; 
                        G(f_i,g_i) \circ \pi_2\right>\\
   & = & \left<\sqcup_i (F(f_i,g_i) \circ \pi_1);
               \sqcup_i (G(f_i,g_i) \circ \pi_2)\right> \;\;\;\;(*)\\
   & = & \left<\sqcup_i F(f_i,g_i) \circ \sqcup_i \pi_1;
               \sqcup_i G(f_i,g_i) \circ \sqcup_i \pi_2\right>\\
   & = & \left<\sqcup_i F(f_i,g_i) \circ \pi_1);
               \sqcup_i G(f_i,g_i) \circ \pi_2)\right>\\
   & = & (\sqcup_i F(f_i,g_i)) \times (\sqcup_i G(f_i,g_i))\\
\end{array}
\end{displaymath}

The interesting step is marked with (*); it is justified by the fact
that we know that $\pi_j \circ (\sqcup_i \left<h^1_i;h^2_i\right>) = 
\sqcup_i (\pi_j \circ \left<h^1_i;h^2_i\right>) = 
\sqcup_i h^j_i$,
and that $\pi_j \circ \left<\sqcup_i h^1_i; \sqcup_i h^2_i\right> = \sqcup_i h^j_i$,
and that the mediating morphism is unique. 


\item Suppose $F$ and $G$ are locally continuous. Now, for $A$ and $B$
we have the functor which takes $(A,B)$ to $F(A,B) + G(A,B)$
on the object part, and which takes $(f,g)$ to $F(f,g) + G(f,g)$ on
the arrow part.

Next, suppose that we have a chain of functions $\left<f_i\right> : C \to A$ and
$\left<g_i\right> : B \to D$. Now, we calculate:

\begin{displaymath}
\begin{array}{lcl}
  \sqcup_i (F + G (f_i,g_i)) 
   & = & \sqcup_i (F(f_i,g_i) + G(f_i,g_i)) \\
   & = & \sqcup_i \left[\inl \circ F(f_i,g_i); 
                        \inr \circ G(f_i,g_i)\right]\\
   & = & \left[\inl \circ \sqcup_i (F(f_i,g_i));
               \inr \circ \sqcup_i (G(f_i,g_i))\right] \;\;\;\;(*)\\
   & = & \left[\sqcup_i \inl \circ F(f_i,g_i);
               \sqcup_i \inr \circ G(f_i,g_i) \right]\\
   & = & \left[\inl \circ \sqcup_i F(f_i,g_i));
               \inr \circ \sqcup_i G(f_i,g_i))\right]\\
   & = & (\sqcup_i F(f_i,g_i)) + (\sqcup_i G(f_i,g_i))\\
\end{array}
\end{displaymath}

The interesting step is marked with (*); it is justified by the fact
that we know that $(\sqcup_i \left[h^1_i;h^2_i\right]) \circ
\mathsf{in}_j = 
\sqcup_i (\left[h^1_i;h^2_i\right] \circ \mathsf{in}_j) = 
\sqcup_i h^j_i$, and that $\left<\sqcup_i h^1_i; \sqcup_i
h^2_i\right> \circ \mathsf{in}_j = \sqcup_i h^j_i$, and that the mediating
morphism is unique.

\item Suppose $F$ and $G$ are locally continuous. Now, for $A$ and $B$
we have the functor which takes $(A,B)$ to $F(B,A) \to G(A,B)$
on the object part, and which takes $(f,g)$ to $F(g,f) \to G(f,g)$ on
the arrow part.
 
Next, suppose that we have a chain of functions $\left<f_i\right> : A \to C$ 
and $\left<g_i\right> : D \to B$. Now, we calculate:
 
\begin{displaymath}
\begin{array}{lcl}
\sqcup_i [F \to G](f_i,g_i) & = & F(g_i,f_i) \to G(f_i,g_i) \\
& = & 
  \sqcup_i \lambda h.\; [G(f_i,g_i) \circ h \circ F(g_i, f_i)] \\
& = & 
  \lambda h.\; \sqcup_i [G(f_i,g_i) \circ h \circ F(g_i, f_i)]  \\
& = & 
  \lambda h.\; [(\sqcup_i G(f_i,g_i)) \circ h \circ (\sqcup_i F(g_i, f_i))] \\
& = & 
  \lambda h.\; [G(\sqcup_i f_i, \sqcup_ig_i) \circ h \circ 
                F(\sqcup_i g_i, \sqcup_i f_i)] \\
\end{array}
\end{displaymath}

% The interesting step is at (*). 



% F(g : D -> B, f : A -> C) : F(D,C) -> F(B,A)
% G(f : A -> C, g : D -> B) : G(A,B) -> G(C,D)
% 
% F(g,f) : F(D,C) -> F(B,A)
% G(f,g) : G(A,B) -> G(C,D)
% 
% F(g,f) -> G(f,g) : (F(B,A) -> G(A,B)) -> F(D,C) -> G(C,D)
%                  = \lambda h. G(f,g) \circ h \circ F(g,f)
% 





\item The constant functor is locally continuous because it maps all 
morphisms to the identity morphism, and hence trivially preserves 
limits. 

\item Suppose $X$ is a set, and $F(x)$ is an $X$-indexed family of
locally continuous functors. 

First, given objects $(A,B)$, we have the object part of this as the
dependent function space $\Pi x:X.\; F(x)(A,B)$, with elements $u
\sqsubseteq v$ if and only if for all $x \in X$, $u(x)
\sqsubseteq_{F(x)(A,B)} v(x)$. 

It's worth noting that this object is a true product over $X$. Given
$\Pi x:X.\; F(x)(A,B)$, we can define the $x$-th projection as $\pi_x
= \lambda f.\; f(x)$. Then, it's clear that given a family of 
morphisms $f_x : Y \to F(x)(A,B)$, we can define a function 
$\left<f_x\right> : Y \to \Pi x:X.\;F(x)(A,B) = \lambda y.\; \lambda x.\; f_x(y)$, 
which means that for all $x$, and that $\pi_x \circ \left<f_x\right> = f_x$. 

We show uniqueness by supposing that that there is some $g$ such
that $\pi_x \circ g = f_x$. Then, we know that 
$g = \lambda x.\; pi_x \circ g$, which means that $g = \lambda x.\; f_x$,
which is exactly $\left<f_x\right>$. 


Next, given morphisms $f \in A \to C$ and $g \in D \to B$, we have
$[\Pi x:X.\; F(x)](f,g) \in [\Pi x:X.\; F(x)](A,B) \to [\Pi x:X.\; F(x)](C,D)$
as:
  
\begin{displaymath}
  [\Pi x:X.\; F(x)](f,g) = \lambda x.\; F(x)(f,g)
\end{displaymath}

Clearly, this preserves identities and composition, and is hence a
functor. 

Now, suppose that $(f,g) \sqsubseteq (f',g')$, and that $x$ is an
arbitrary element of $X$. Then $[\Pi x:X.\; F(x)](f,g) = F(x)(f,g)$
and $[\Pi x:X.\; F(x)](f',g') = F(x)(f',g')$. Since $F(x)$ is a
locally continuous functor, we know that $F(x)(f,g) \sqsubseteq
F(x)(f',g')$, and so $[\Pi x:X.\; F(x)]$ preserves ordering.

Now, suppose that $(f_i,g_i)$ form a chain. So, we know that 

\begin{displaymath}
\begin{array}{lcl}
\sqcup_i ([\Pi x:X.\; F(x)](f_i,g_i)) & = & 
  \sqcup_i (\lambda x:X.\; (F(x) (f_i, g_i))) \\
& = & 
  \lambda x:X.\; (\sqcup_i (F(x) (f_i, g_i))) \;\;\;\;(*) \\
& = & 
  \lambda x:X.\; (F(x) (\sqcup_i (f_i, g_i))) \\
& = & 
[\Pi x:X.\; F(x)] (\sqcup_i (f_i,g_i)) \\
\end{array}
\end{displaymath}

The interesting step is (*); it is justified by the fact
that we know that $\pi_x \circ \sqcup_i \left<h^x_i\right> =  
\sqcup_i (\pi_x \circ_i \left<h^x_i\right>) = 
\sqcup_i h^x_i$, 
and that $\pi_x \circ \left<\sqcup_i h^x_i\right> 
          = \sqcup_i h^x_i$, 
and that the mediating morphism is unique.

As a result, we can conclude that this functor is locally 
continuous.

\item Suppose $X$ is a set, and $F(x)$ is an $X$-indexed family
of locally continuous functors. 

First, given objects $(A,B)$, we have the object part of the functor
yielding the dependent sum $\Sigma x:X.\; F(x)(A,B)$. Ordering is
given pairwise, equipping the set $X$ with the trivial ordering. 
That is $(x,o) \sqsubseteq (x', o')$ if and only if $x = x'$ and 
$o \sqsubseteq_{F(x)(A,B)} o'$. 

It's worth noting that this is a true coproduct over $X$. Given
$\Sigma x:X.\; F(x)(A,B)$, we can define the injections $\inj{x} \in
F(x)(A,B) \to \Sigma x:X.\; F(x)(A,B)$ as $\lambda v. (x,v)$. 
Next, suppose we have a family of functions $f_x : F(x)(A,B) \to Y$. 
We can define a function $[f_x] \in (\Sigma x:X.\; F(x)(A,B)) \to Y$ as
$\lambda (x,v).\; (f_x\;v)$. It's clear that $[f_x]\circ \inj{i} = f_i$. 

Finally, we can establish uniqueness as follows. Suppose that there 
is a $g$ such that $g \circ \inj{i} = f_i$. Next, we know that 
$g = \lambda (x,v).\;g (x, v) = \lambda (x,v).\;(g\circ\inj{x})(v)$,
which is clearly $\lambda (x,v).\; (f_x\;v)$, which is just $[f_x]$


Next, given morphisms $f \in A \to C$ and $g \in D \to B$, we have
$[\Sigma x:X.\; F(x)](f,g) \in [\Pi x:X.\; F(x)](A,B) \to [\Sigma x:X.\; F(x)](C,D)$
as:

\begin{displaymath}
  [\Sigma x:X.\; F(x)](f,g) = \lambda (x, v).\; (x, F(x)(f,g)(v))
\end{displaymath}

This clearly preserves identities and composition, and hence defines
a functor. 

Now, suppose that $(f,g) \sqsubseteq (f',g')$ and that $(x,v)$ is
an element of $\Sigma x:X.\; F(x)(A,B)$. Then we have that
$[\Sigma x:X.\;F(x)](f,g)](x,v) = (x,F(x)(f,g)(v))$ and that
$[\Sigma x:X.\;F(x)](f',g')](x,v) = (x,F(x)(f',g')(v))$. So we
know that $x=x$, and by the local continuity of $F(x)$, we know
that $F(x)(f,g)(v) \sqsubseteq F(x)(f',g')(v)$. So this functor
preserves ordering. 

Finally, suppose $(f_i, g_i)$ form a chain. 

\begin{displaymath}
\begin{array}{lcl}
\sqcup_i [\Sigma x:X.\;F(x)](f_i,g_i) 
& = & \sqcup_i \lambda (x,v).\; (F(x)(f_i,g_i)(v)) \\
& = & \lambda (x,v). \sqcup_i (F(x)(f_i,g_i)(v)) \;\;\;\; (*) \\
& = & \lambda (x,v).\; F(x)(\sqcup_i f_i, \sqcup_i g_i))(v) \\
& = & [\Sigma x:X.\;F(x)](\sqcup_i f_i, \sqcup_i g_i) \\
\end{array}
\end{displaymath}

The interesting step is (*); it is justified by the fact
that we know that 
$(\sqcup_i [h^x_i]) \circ \inj{x} =  
\sqcup_i ([h^x_i] \circ \inj{x}) =  \sqcup_i h^x_i$, 
and that $[\sqcup_i h^x_i] \circ \inj{x} = \sqcup_i h^x_i$, 
and that the mediating morphism is unique.
\end{enumerate}
\end{proof}

\subsection{Pitts's Theorem and the Inverse Limit Construction}

Once we have a locally continuous functor, we'd like to find a
solution to the fixed point equation it defines. 

\begin{prop}{(Pitts's Theorem)}
Any locally-continuous functor $F : CPO_\bot \times CPO^{op}_\bot \to
CPO_\bot$ has a solution to the equation $X \cong F(X)$. Moreover, there
is also a \emph{minimal} isomorphism solving this equation. 
\end{prop}

The existence of a solution follows from Scott's inverse limit
construction, together with Smyth and Plotkin's characterization of
such solution. We give the details in the sequel, in which we will
always take $F$ to be a locally-continuous functor of the type
mentioned above.

\subsubsection{Embeddings and Projections}

First, recall that an \emph{embedding} $e : C \to D$ between pointed
cpos is a continuous function such that there exists a function $p : D
\to C$ (called a \emph{projection}) with the properties that $p \circ
e = id_C$ and $e \circ p \sqsubseteq id_D$. 

Now, we'll introduce the category $CPO_\bot^{O}$, which is the
category whose objects are the pointed domains, and whose morphisms
from $D$ to $E$ are the embedding-projection pairs. The identity
morphism from domain $D$ to $D$ is the pair $\left<id, id\right>$, and
the composition operation on $\left<e, p\right>$ and $\left<e',
p'\right>$ is $\left<e' \circ e, p \circ p'\right>$. To verify that
this is indeed a category, we check that:


\begin{itemize}
\item The identity $\left<id, id\right> : D \to D$ is an 
embedding-projection pair because $id \circ id = id$ and 
$id \sqsubseteq id$. 

\item The composition $\left<e, p\right> \circ \left<e',
p'\right>$ is an embedding-projection pair because it 
is defined to be equal to $\left<e \circ e', p' \circ p\right>$, 
and we have that embedding followed by projection is:

\begin{displaymath}
  \begin{array}{lcl}
    (p' \circ p) \circ (e \circ e') 
       & = & p' \circ (p \circ e) \circ e' \\ 
       & = & p' \circ id \circ e' \\ 
       & = & p' \circ e' \\
       & = & id \\                 
  \end{array}
\end{displaymath}

and likewise we have for a projection followed by an embedding:

\begin{displaymath}
  \begin{array}{lcll}
    (e \circ e')  \circ (p' \circ p) 
       & = & e \circ (e' \circ p') \circ p \\
       & \sqsubseteq & e \circ id \circ p \\ 
       & \sqsubseteq & e \circ p \\
       & \sqsubseteq & id \\                 
  \end{array}
\end{displaymath}

\item Finally, it's clear that composition is associative and has
  identities as units because it inherits these properties from the
  underlying composition operations.
\end{itemize}


Now, consider the empty one-point domain $\emptyset_\bot =
\setof{\bot}$, and the sequence of domains $X_i$, defined inductively
by $X_0 = \emptyset_\bot$ and $X_{i+1} = F(X_i, X_i)$. Next, 
we will define embeddings and projections $e_i : X_i \to X_{i+1}$ 
and $p_i : X_{i+1} \to X_i$ as follows: 

\begin{displaymath}
  \begin{array}{llcl}
    e_0     & : X_0 \to X_1 & = &\lambda x.\; \bot \\
    e_{i+1} & : X_{i+1} \to X_{i+2} & = & F(e_i, p_i) \\[0.5em]
    p_0 &  : X_1 \to X_0 & = & \lambda x.\; \bot \\
    p_{i+1} & : X_{i+2} \to X_{i+1} & = & F(p_i, e_i) \\[0.5em]
  \end{array}
\end{displaymath}

\begin{lemma}{(Embeddings and Projections)} Each $\left<e_i, p_i\right>$ forms
an arrow from $X_i$ to $X_{i+1}$ in $CPO_\bot^O$.
\end{lemma}

\begin{proof}
This proof proceeds by induction on $i$. 
\begin{itemize}
\item Case $i=0$: Obviously $e_0 \circ p_0 = id$, since $X_0 = \setof{\bot}$.  
  Likewise, since $p_0(e_0(x)) = \bot$, and $\bot \sqsubseteq x$, it follows that
  $p_0 \circ e_0 \sqsubseteq id$. 

\item Case $i=n+1$: 

  First, we'll show that $e_i \circ p_i$ is the identity: 

  \begin{displaymath}
    \begin{array}{lcll}
      e_i \circ p_i & = & e_{n+1} \circ p_{n+1}            & \mbox{Def.}\\ 
                    & = & F(e_n, p_n) \circ F(p_n, e_n)   & \mbox{Def.}\\ 
                    & = & F(e_n \circ p_n, e_n \circ p_n) & \mbox{Functor property}\\ 
                    & = & F(id, id)                       & \mbox{Ind. hyp.}\\  
                    & = & id                              & \mbox{Functor property} \\
    \end{array}
  \end{displaymath}

  \noindent Now, we'll show that $p_i \circ e_i \sqsubseteq id$: 

  \begin{displaymath}
    \begin{array}{lcll}
      p_i \circ e_i & = & p_{n+1} \circ e_{n+1}            & \mbox{Def.}\\ 
                    & = & F(p_n, e_n) \circ F(e_n, p_n)   & \mbox{Def.}\\ 
                    & = & F(p_n \circ e_n, p_n \circ e_n) & \mbox{Functor property}\\ 
    \end{array}
  \end{displaymath}

  \noindent By induction, we know that $p_n \circ e_n \sqsubseteq id$, and
  because locally continuous functors are also monotone, we know that 
  $F(p_n \circ e_n, p_n \circ e_n) \sqsubseteq F(id, id) \equiv id$.
\end{itemize}
\end{proof}

\ \\

\subsubsection{Construction of the Domain}

Now, we'll define the domain $X$ to be the domain with the underlying set:
\begin{displaymath}
X \equiv \left\{ x \in (\Pi n:\N.\; X_n) \;|\; \forall m:\N.\; x_m = p_m(x_{m+1}) \right\}
\end{displaymath}
with the ordering being the usual component-wise ordering. (As a
notational convenience, we will write $x_n$ to indicate the $n$-th
component of $x$, or $x(n)$.)  To be in $CPO_\bot$, it needs a least
element, which is just $\lambda n:\N.\;\bot$.  

We claim that this pointed CPO $X$ is the colimit of the chain of
domains $X_i$ in $CPO_\bot^O$. To prove it, we must proceed in two
stages.

\begin{lemma}{($X$ is a cocone)} $X$ is a cocone of the
diagram $X_0 \longrightarrow X_1 \longrightarrow \ldots$. 
\end{lemma}


\begin{proof}
To show this, we must give morphisms $\left<\hat{e}_i,
\hat{p}_i\right> : X_i \to X$. To do so, we'll define:

\begin{displaymath}
  \hat{e}_n : X_n \to X \equiv 
    \lambda x:X_n.\; \lambda m:\N.\; 
       \left\{ 
          \begin{array}{ll}
            p_{m,n}(x) & \mbox{if } m < n \\
            x         & \mbox{if } m = n \\
            e_{n,m}(x) & \mbox{if } m > n \\
          \end{array}
       \right.
\end{displaymath}

We define $e_{i,j}$ to be the composition $e_{j-1} \circ e_{j-2} \circ \ldots \circ e_i$,
which will have the type $X_i \to X_j$. Likewise, we define $p_{i,j}$ to be the 
composition $p_i \circ \ldots \circ p_{j-1}$, which will have the type $X_j \to X_i$. 

\noindent The projection $\hat{p}_n : X \to X_n$ is much simpler. It's just

\begin{displaymath}
  \hat{p}_n : X \to X_n \equiv \lambda x:X.\; x_n
\end{displaymath}

Now, we'll verify that these do form an embedding-projection pair. 

  \begin{itemize}
  \item First, we'll show that $\hat{p}_n \circ \hat{e}_n = id$. 

    \begin{displaymath}
      \begin{array}{lcl}
        \hat{p}_n \circ \hat{e}_n & = & \lambda x:X_n.\; (\hat{p}_n \circ \hat{e}_n)\;x \\
                                  & = & \lambda x:X_n.\; \hat{p}_n(\hat{e}_n \;x) \\
                                  & = & \lambda x:X_n.\; (\hat{e}_n\;x)\;n \\
                                  & = & \lambda x:X_n.\; x \\
                                  & = & id \\
      \end{array}
    \end{displaymath}

  \item Now, we'll show that $\hat{e}_n \circ \hat{p}_n \sqsubseteq id$. 
    
    \begin{displaymath}
      \begin{array}{lcl}
        \hat{e}_n \circ \hat{p}_n & = & \lambda x:X.\; (\hat{e}_n \circ \hat{p}_n) \; x \\
                                  & = & \lambda x:X.\; \hat{e}_n (\hat{p}_n \; x) \\ 
                                  & = & \lambda x:X.\; \hat{e}_n (x_n) \\ 
      \end{array}
    \end{displaymath}

    Now, when applied to an argument $x \in X$, it's clear that the
    result element is component-wise equal to $x$ for the components
    less than or equal to $n$, and less than that for components
    bigger than $n$, which makes the result smaller than $x$.
  \end{itemize}
This establishes that there are morphisms $\left<\hat{e}_i,
\hat{p}_i\right> : X_i \to X$. Now, we need 1) to show that the 
equation $\left<\hat{e}_i, \hat{p}_i\right> : X_i \to X = 
\left<\hat{e}_{i+1}, \hat{p}_{i+1}\right> \circ \left<e_i, p_i\right>$ 
holds, and 2) that $\bigsqcup_i \hat{e}_i \circ \hat{p}_i = id$, 
which will establish that the diagram commutes appropriately. 

Expanding the definition of composition, we want to show that 
$\left<\hat{e}_i, \hat{p}_i\right> = \left<\hat{e}_{i+1} \circ e_i,
                                          p_i \circ \hat{p}_{i+1} \right>$.
So, we have that
\begin{displaymath}
  \begin{array}{lcl}
    \hat{e}_{i+1} \circ e_i
      & = &     
      \lambda x:X_i.\; \lambda m:\N.\; 
       \left\{ 
          \begin{array}{ll}
            p_{m,{i+1}}(e_i\;x) & \mbox{if } m < i+1 \\
            e_i\;x         & \mbox{if } m = i+1 \\
            e_{{i+1},m}(e_i\;x) & \mbox{if } m > i+1 \\
          \end{array}
       \right. 
   \\[1em]
      & = &     
      \lambda x:X_i.\; \lambda m:\N.\; 
       \left\{ 
          \begin{array}{ll}
            p_{m,i}(p_i(e_i\;x)) & \mbox{if } m < i+1 \\
            e_{i, i+1}\;x         & \mbox{if } m = i+1 \\
            e_{{i+1},m}(e_i\;x) & \mbox{if } m > i+1 \\
          \end{array}
       \right. 
   \\[1em]
     & = & 
      \lambda x:X_i.\; \lambda m:\N.\; 
       \left\{ 
          \begin{array}{ll}
            p_{m,i}(x) & \mbox{if } m < i+1 \\
            e_{i,m}(e_i\;x) & \mbox{if } m > i \\
          \end{array}
       \right. 
   \\[1em]
     & = & 
      \lambda x:X_i.\; \lambda m:\N.\; 
       \left\{ 
          \begin{array}{ll}
            p_{m,i}(x) & \mbox{if } m < i \\
            x             & \mbox{if } m = i \\
            e_{i,m}(e_i\;x) & \mbox{if } m > i \\
          \end{array}
       \right. 
   \\[1em]
     & = & \hat{e}_i \\
  \end{array}
\end{displaymath}

\noindent In the other direction, we show that 
\begin{displaymath}
  \begin{array}{lcl}
    p_i \circ \hat{p}_{i+1} & = & \lambda x:X.\; p_i(x_{i+1}) \\ 
                           & = & \lambda x:X.\; x_i \\
                           & = & \hat{p}_i \\
  \end{array}
\end{displaymath}
The second step follows from the definition of $X$. 

Now, we need to show that $\bigsqcup_i \hat{e}_i \circ \hat{p}_i = id$. 

\begin{displaymath}
  \begin{array}{lcl}
    \bigsqcup_i \hat{e}_i \circ \hat{p}_i 
       & = & 
       \bigsqcup_i \lambda x:X.\; \hat{e}_i(\hat{p}_i \;x)
    \\
       & = & 
       \bigsqcup_i \lambda x:X.\; \hat{e}_i(x_i)
    \\
       & = & 
       \bigsqcup_i \lambda x:X.\; \hat{e}_i(x_i)
    \\
       & = & 
       \bigsqcup_i \lambda x:X.\;\lambda m:\N.\; 
          \left\{ 
          \begin{array}{ll}
            p_{m,i}(x_i) & \mbox{if } m < i \\
            x_i         & \mbox{if } m = i \\
            e_{i,m}(x_i) & \mbox{if } m > i \\
          \end{array}
       \right.
    \\
       & = & 
       \bigsqcup_i \lambda x:X.\;\lambda m:\N.\; 
          \left\{ 
          \begin{array}{ll}
            x_m         & \mbox{if } m \leq i \\
            e_{i,m}(x_i) & \mbox{if } m > i \\
          \end{array}
       \right.
    \\
       & = & 
       \bigsqcup_i \lambda x:X.\;\lambda m:\N.\; 
          \left\{ 
          \begin{array}{ll}
            x_m         & \mbox{if } m \leq i \\
            e_{i,m}(p_{i,m}\;x_m) & \mbox{if } m > i \mbox{ (*)}\\
          \end{array}
       \right.
    \\
       & = & 
       \lambda x:X.\;\lambda m:\N.\; x_m
    \\
       & = & 
       id
  \end{array}
\end{displaymath}

In (*), we use the definition of $X$ to see that the components greater than
$i$ are smaller than $x$'s component at that index. So the limit is the 
identity.


\end{proof}

To show that $X$ is the colimit of this diagram, we need to show there
is a unique map from it to any other cocone. 

\begin{lemma}{(Universality of $X$)} 
Suppose that there is a $Y$ with morphisms $\left<f_n, q_n\right> : X_n
\to Y$ and $q_n : Y \to X_n$, forming a cocone over $X_0 \longrightarrow
X_1 \longrightarrow \ldots$. Then, there is a unique 
$\left<h_e,h_p\right> : X \to Y$ such that for all $n$, 
$\left<f_n,q_n\right> = \left<h_e,h_p\right> \circ \left<\hat{e}_n, \hat{p}_n\right>$.
\end{lemma}

\begin{proof}
To show this, we need to explicitly construct $h_e$ and $h_p$, and show
that they form an embedding-projection pair. We'll define $h_e : X \to Y = 
\bigsqcup_i f_i \circ \hat{p}_i$, and define 
$h_p : Y \to X = \lambda y:Y.\; \lambda i:\N.\; q_i\; y$.

Before we can proceed any further, we need to establish that $h_e$
actually defines a function --- that is, we have to establish that
$f_i \circ \hat{p}_i$ is a chain in $i$. So, assume we have some
arbitrary $i$, and some arbitrary $x : X$.

\begin{enumerate}
\item Now, by the properties of embedding-projection pairs, we know
$e_i(p_i\;x_{i+1}) \sqsubseteq x_{i+1}$, 
\item By the continuity of $f_{i+1}$, 
this means $f_{i+1}(e_i(p_i\;x_{i+1})) \sqsubseteq f_{i+1}(x_{i+1})$. 
\item By the fact that $Y$ is a cocone, this
means $f_i(p_i\;x_{i+1})
\sqsubseteq f_{i+1}(x_{i+1})$. 
\item By the
definition of $X$, this means $f_i(p_i\;x_{i+1}) 
\sqsubseteq f_{i+1}(x_{i+1})$. 
\item By the definition of $\hat{p}$, this is 
the same as showing $f_i(x_i) \sqsubseteq f_{i+1}(x_{i+1})$. 
\item By the definition of $\hat{p}$, this is the same as 
   $f_i(\hat{p}_i\;x) \sqsubseteq f_{i+1}(\hat{p}_{i+1}\;x)$. 
\item Since continuity for functions is pointwise, we have 
shown $f_i \circ \hat{p}_i \sqsubseteq f_{i+1} \circ \hat{p}_{i+1}$. 
\end{enumerate}

Next, let's establish that $h_e$ and $h_p$ form an embedding-projection
pair. To show that $h_e \circ h_p = id$, we use equational reasoning:

\begin{displaymath}
  \begin{array}{lcl}
    h_e \circ h_p  
     & = & 
      (\bigsqcup_i f_i \circ \hat{p}_i) \circ (\lambda y:Y.\; \lambda i:\N.\; (q_i\; y)) 
\\
      & = & 
      \bigsqcup_i (f_i \circ \hat{p}_i \circ (\lambda y:Y.\; \lambda i:\N.\; (q_i\; y)) 
\\      
      & = & 
      \bigsqcup_i \lambda y:Y.\;(f_i \circ \hat{p}_i \circ (\lambda y:Y.\; \lambda i:\N.\; (q_i\; y))\;y 
\\
      & = & 
      \bigsqcup_i (\lambda y:Y.\;f_i(\hat{p}_i\;(\lambda i:\N.\; (q_i\; y))))
\\
      & = & 
      \bigsqcup_i (\lambda y:Y.\;f_i(q_i\; y))
\\
      & = & 
      \bigsqcup_i \lambda y:Y.\;y
\\
      & = & 
      \lambda y:Y.\;y
\\
  \end{array}
\end{displaymath}

In other direction, we need to show $h_p \circ h_e \sqsubseteq id$.

\begin{displaymath}
  \begin{array}{lcl}
    h_p \circ h_e
     & = &  
      (\lambda y:Y.\; \lambda j:\N.\; (q_j\; y)) \circ (\bigsqcup_i f_i \circ \hat{p}_i)
\\
     & = &  
      \bigsqcup_i ((\lambda y:Y.\; \lambda j:\N.\; (q_j\; y)) \circ f_i \circ \hat{p}_i)
\\
     & = &  
      \lambda x:X.\; \bigsqcup_i ((\lambda y:Y.\; \lambda j:\N.\; (q_j\; y)) \circ f_i \circ \hat{p}_i))\;x
\\
     & = &  
      \lambda x:X.\; \bigsqcup_i ((\lambda y:Y.\; \lambda j:\N.\; (q_j\; y))\;(f_i\;(\hat{p}_i\;x)))
\\   
     & = &  
      \lambda x:X.\; \bigsqcup_i (\lambda j:\N.\; (q_j\; (f_i\;(\hat{p}_i\;x))))
\\   
     & = &  
      \lambda x:X.\; \lambda j:\N.\; \bigsqcup_i ((q_j\; (f_i\;(\hat{p}_i\;x))))
  \end{array}
\end{displaymath}

To finish this calculation, consider an arbitrary $x:X$ and $j:\N$. Now, consider
the tail of the chain, where $i > j$. Now, since we know that $q_k = p_k \circ q_{k+1}$, 
it follows that:

\begin{displaymath}
  \begin{array}{lcl}
   q_j\; (f_i\;(\hat{p}_i\;x))
     & = & p_{j,i}(q_i\; (f_i\;(\hat{p}_i\;x))) \\
     & \sqsubseteq & p_{j,i}(\hat{p}_i\;x) \\
     & \sqsubseteq & \hat{p}_j\;x \\
     & \sqsubseteq & x_j \\
  \end{array}
\end{displaymath}

Which means that the limit of the chain has to be less than 
$\lambda x:X.\;\lambda j:\N.\;x_j$ -- which means that it is less
than the identity.


So we have established that $\left<h_e,h_p\right>$ is a morphism
between $X$ and $Y$. Next, let's see whether it commutes:
$\left<f_n,q_n\right> = \left<h_e,h_p\right> \circ \left<\hat{e}_n,
\hat{p}_n\right>$. Unfolding the definition of composition, we 
get two proof obligations. First, 

\begin{displaymath}
\begin{array}{lcl}
   h_e \circ \hat{e}_n 
   & = & 
     (\bigsqcup_i f_i \circ \hat{p}_i) \circ \hat{e}_n 
\\
   & = & 
     \bigsqcup_i (f_i \circ \hat{p}_i \circ \hat{e}_n)
\\
   & = & 
     \lambda x:X_n.\; \bigsqcup_i (f_i (\hat{p}_i \; (\hat{e}_n\;x)))
\\
   & = & 
     \lambda x:X_n.\; \bigsqcup_i (f_i (\hat{e}_n\;x\;i))
\\
   & = & 
     \lambda x:X_n.\; \bigsqcup_i f_i 
       (\left\{ 
          \begin{array}{ll}
            p_{i,n}(x) & \mbox{if } i < n \\
            x         & \mbox{if } i = n \\
            e_{n,i}(x) & \mbox{if } i > n \\
          \end{array}
       \right.)
\end{array}
\end{displaymath}

To find the limit of this chain, consider any $i > n$. Because $f_{k+1} \circ e_k = f_k$, 
we can see that $f_i(e_{n,i}\;x) = f_n\;x$, which means that the limit is $f_n\;x$, and
hence $h_e \circ \hat{e}_n = f_n$. 

Next, consider $\hat{p}_n \circ h_p$: 

\begin{displaymath}
\begin{array}{lcl}
   \hat{p}_n \circ h_p
   & = & 
     \hat{p}_n \circ (\lambda y:Y.\;\lambda i:\N.\;(q_i\;y))
\\
   & = & 
     \lambda y:Y.\; \hat{p}_n (\lambda i:\N.\;(q_i\;y))
\\
   & = & 
     \lambda y:Y.\; (q_n\;y)
\\
   & = & 
     q_n 
\\
\end{array}
\end{displaymath}

At this point, we have established that $X$ is a weak colimit --
there's a morphism from it to any other cone, but we still have yet to
show that it is a unique morphism. So, suppose that we have some 
other mediating morphism $\left<h'_e, h'_p\right> : X \to Y$. 

For the embedding $h'_e$, we proceed as follows:

\begin{enumerate}
\item Now, 
it must be the case that $\left<h'_e, h'_p\right> \circ \left<\hat{e}_n, \hat{p}_n\right> = 
\left<f_n, q_n\right>$. 

\item So $h'_e \circ \hat{e}_n = f_n$. 

\item Composing both sides with $\hat{p}_n$, we get 
   $h'_e \circ \hat{e}_n \circ \hat{p}_n = f_n \circ \hat{p}_n$. 

\item Taking limits of chains on both sides, we get $h'_e \circ \bigsqcup_n \hat{e}_n \circ{p}_n = \bigsqcup_n f_n \circ \hat{p}_n$

\item Simplifying, we get $h'_e = h_e$.
\end{enumerate}

For the projection $h'_p$, we have

\begin{enumerate}
\item We have $\hat{p}_n \circ h'_p = q_n$. 
\item Composing on both sides with $\hat{e}_n$, we have $\hat{e}_n \circ \hat{p}_n \circ h'_p = \hat{e}_n \circ q_n$. 
\item Taking limits on both sides, we have $h'_p = \bigsqcup_n \hat{e}_n \circ q_n$. 
\item Simplifying the limit expression, we get $h'_p = \lambda y:Y.\;\lambda n:\N.\; q_n(y)$. 
\item So $h'_p = h_p$. 
\end{enumerate}

\end{proof}


\subsubsection{Showing $X$ is a Solution to $F(X, X) \cong X$}

\begin{lemma}{(X is a fixed point)} The equation $F(X, X) \cong X$
is valid.
\end{lemma}
\begin{proof}
First, note that applying $F$ to each of the $X_i$ and $\left<e_i,
p_i\right>$ yields $X_{i+1}$ and $\left<e_{i+1}, p_{i+1}\right>$.  In
other words, applying $F$ to our old diagram gives us the same thing
as before, only with the first element chopped off.

Therefore, $X$ is still a colimit for this diagram, because if we
replicate the colimit construction for this diagram, we can establish
an isomorphism between the ``new'' construction and $X$, since the
leading elements of the infinite product are determined by the
requirement that $x_i = p_i(x_{i+1})$.

Since $F$ is a locally continuous functor, it preserves colimits of
chains $\bot \longrightarrow F(\bot) \longrightarrow \ldots$, so $F(X,
X)$ is itself a colimiting object. 

Since colimits are unique up to isomorphism, it follows that $F(X, X)
\cong X$. 

\end{proof}


\section{The Programming Language}

We have given the semantics of types in domain-theoretic terms. Now,
we'll give the syntax and typing of the programming language. Then,
we'll use the domain-theoretic semantics just given in order to first
give a denotational semantics for the programming language, and
second, to give an interesting equality theory for it. The syntax of 
terms is given in figure~\ref{lang-syntax}. 


\begin{figure}
\begin{displaymath}
  \begin{array}{llcl}
    \mbox{Pure expressions} & 
     e & ::= & 
         \unit \bnfalt
         \pair{e}{e'} \bnfalt
         \fst{e} \bnfalt
         \snd{e} \bnfalt 
\\
     &&& \inl{e} \bnfalt
         \inr{e} \bnfalt
         \Case{e_0}{x_1}{e_1}{x_2}{e_2} \bnfalt
\\
     &&& \z \bnfalt 
         \s{e} \bnfalt 
         \iter{e}{e_0}{x}{e_1}
\\ 
     &&& x \bnfalt \fun{x}{A}{e} \bnfalt e\;e' \bnfalt
\\ 
     &&& \Fun{\alpha}{\kappa}{e} \bnfalt e\;\tau \bnfalt
\\ 
     &&& \pack{\tau}{e} \bnfalt \unpack{\alpha}{x}{e}{e'} \bnfalt
\\
     &&& \comp{c} \bnfalt \fix{x:D}{e}
\\[1em]
  \mbox{Computations} & 
    c & ::= & e \bnfalt \letv{x}{e}{c} \bnfalt
              \newref{A}{e} \bnfalt !e \bnfalt e := e'
\\[1em]
  \mbox{Contexts} & 
    \Gamma & ::= & \cdot \bnfalt \Gamma, x:A 
\\[1em]
  \mbox{Pointed Types} & 
     D & ::= & \monad{A} \bnfalt A \to D \bnfalt \forall \alpha:\kappa.\; D \bnfalt D \times D 
\\[1em] 
  \end{array}
\end{displaymath}
\caption{Syntax of the Programming Language}
\label{lang-syntax}
\end{figure}


\begin{figure}
\begin{mathpar}
\inferrule*[right=EUnit]
          { }
          {\judgeE{\Gamma}{\unit}{\unittype}}
\\
\inferrule*[right=EPair]
          {\judgeE{\Gamma}{e_1}{A} \\ 
           \judgeE{\Gamma}{e_2}{B}}
          {\judgeE{\Gamma}{\pair{e_1}{e_2}}{A \times B}}
\and
\inferrule*[right=EFst]
          {\judgeE{\Gamma}{e}{A \times B}}
          {\judgeE{\Gamma}{\fst{e}}{A}}
\and
\inferrule*[right=ESnd]
          {\judgeE{\Gamma}{e}{A \times B}}
          {\judgeE{\Gamma}{\snd{e}}{B}}
\\
\inferrule*[right=EInl]
          {\judgeE{\Gamma}{e}{A}}
          {\judgeE{\Gamma}{\inl{e}}{A + B}}
\and
\inferrule*[right=EInr]
          {\judgeE{\Gamma}{e}{B}}
          {\judgeE{\Gamma}{\inr{e}}{A + B}}
\and
\inferrule*[right=ECase]
          {\judgeE{\Gamma}{e}{A+B} \\
           \judgeE{\Gamma, x:A}{e_1}{C} \\
           \judgeE{\Gamma, y:B}{e_2}{C}}
          {\judgeE{\Gamma}{\Case{e}{x}{e_1}{y}{e_2}}{C}}
\\
\inferrule*[right=EZero]
          { }
          {\judgeE{\Gamma}{\z}{\N}}
\and
\inferrule*[right=ESucc]
          {\judgeE{\Gamma}{e}{\N}}
          {\judgeE{\Gamma}{\s{e}}{\N}}
\and
\inferrule*[right=EIter]
          {\judgeE{\Gamma}{e}{\N} \\ 
           \judgeE{\Gamma}{e_0}{A} \\
           \judgeE{\Gamma,x:A}{e_1}{A}}
          {\judgeE{\Gamma}{\iter{e}{e_0}{x}{e_1}}{A}}
\end{mathpar}
\begin{mathpar}
\inferrule*[right=EVar]
          { x:A \in \Gamma }
          {\judgeE{\Gamma}{x}{A}}
\and
\inferrule*[right=ELam]
          {\judgeE{\Gamma, x:A}{e}{B}}
          {\judgeE{\Gamma}{\fun{x}{A}{e}}{A \to B}}
\and
\inferrule*[right=EApp]
          {\judgeE{\Gamma}{e}{A \to B} \\
           \judgeE{\Gamma}{e'}{A} }
          {\judgeE{\Gamma}{e\;e'}{B}}
\\
\inferrule*[right=ETLam]
          {\judgeE[\Theta, \alpha:\kappa]{\Gamma}{e}{A}}
          {\judgeE{\Gamma}{(\Fun{\alpha}{\kappa}{e})}{(\forall \alpha:\kappa.\;A)}}
\and
\inferrule*[right=ETapp]
          {\judgeE{\Gamma}{e}{\forall \alpha:\kappa.\;A} \\
           \judgeWK{\tau}{\kappa}}
          {\judgeE{\Gamma}{e\;\tau}{[\tau/\alpha]A}}
\\
\inferrule*[right=EPack]
          {\judgeE{\Gamma}{\tau}{\kappa} \\ 
           \judgeWK[\Theta, \alpha:\kappa]{A}{\bigstar} \\
           \judgeE{\Gamma}{e}{[\tau/\alpha]A}}
          {\judgeE{\Gamma}{\pack{\tau}{e}}{\exists \alpha:\kappa.\; A}}
\and
\inferrule*[right=EUnpack]
          {\judgeE{\Gamma}{e}{\exists \alpha:\kappa.\;A} \\
           \judgeE[\Theta, \alpha:\kappa]{\Gamma, x:A}{e'}{B} \\
           \judgeWK{B}{\bigstar}}
          {\judgeE{\Gamma}{\unpack{\alpha}{x}{e}{e'}}{B}}
\\
\inferrule*[right=EMonad]
          {\judgeC{\Gamma}{c}{A}}
          {\judgeE{\Gamma}{\comp{c}}{\monad{A}}}
\and
\inferrule*[right=EFix]
          {\judgeE{\Gamma, x:D}{e}{D}}
          {\judgeE{\Gamma}{\fix{x:D}{e}}{D}}
\and
\inferrule*[right=EKeq]
          {\judgeE{\Gamma}{e}{A} \\
           \judgeKeq{A}{B}{\bigstar}}
          {\judgeE{\Gamma}{e}{B}}

\end{mathpar}
\caption{Typing of the Pure Expressions}
\label{lang-typing-pure}
\end{figure}

The pure terms of the language include the unit value $\unit$; pairs
$\pair{e}{e'}$ and projections $\fst{e}$ and $\snd{e}$; injections
into sum types $\inl{e}$ and $\inr{e}$, and a case form
$\Case{e}{x}{e'}{y}{e''}$; zero $\z$, successor $\s{e}$, and primitive
iteration $\iter{e}{e_0}{x}{e_1}$; lambda abstractions $\fun{x}{A}{e}$
and applications $e\;e'$; type abstraction $\Fun{\alpha}{\kappa}{e}$
and type application $e\;\tau$; and existential packing
$\pack{\tau}{e}$ and unpacking $\unpack{\alpha}{x}{e}{e'}$. Suspended
monadic computations $\comp{c}$ are terms of the type $\monad{A}$.

Finally, we have a term-level fixed point operator, $\fix{x:D}{e}$. It
is not defined over all types; it is only permitted to use this over
\emph{well-pointed} types. That is, fixed points are only defined over
types whose interpretations are domains with least elements. This
ensures that the semantic fixed point operation is well-defined, and
hence we can give a semantics of fixed points at this type. The typing
rules for all of these forms are given in figure~\ref{lang-typing-pure}. 

\todo{However, I'm not entirely happy with the \emph{operational}
interpretation of the fixed point construct. Each term
$\fix{x:D}{e}$ is really a \emph{value}, which means that the
introduction forms for $D$ isn't just the ordinary intro forms
for each type -- we have an extra constant for each type. This will 
make the equational theory and operational semantics all go a bit funny.}

I give the computation forms of the language in
figure~\ref{lang-typing-monadic}. 


\begin{figure}
\begin{mathpar}
\inferrule*[right=CReturn]
          {\judgeE{\Gamma}{e}{A}}
          {\judgeC{\Gamma}{e}{A}}
\and
\inferrule*[right=CLet]
          {\judgeE{\Gamma}{e}{\monad{A}} \\
           \judgeC{\Gamma, x:A}{c}{B}}
          {\judgeE{\Gamma}{\letv{x}{e}{c}}{B}}
\and
\inferrule*[right=CGet]
          {\judgeE{\Gamma}{e}{\reftype{A}}}
          {\judgeC{\Gamma}{!e}{A}}
\and
\inferrule*[right=CSet]
          {\judgeE{\Gamma}{e}{\reftype{A}} \\
           \judgeE{\Gamma}{e'}{A}}
          {\judgeE{\Gamma}{e := e'}{\unittype}}
\and
\inferrule*[right=CNew]
          {\judgeE{\Gamma}{e}{A}}
          {\judgeC{\Gamma}{\newref{A}{e}}{\reftype{A}}}
\end{mathpar}
\caption{Typing of Monadic Expressions}
\label{lang-typing-monadic}
\end{figure}

\section{Denotational Semantics}

Now that we know that the basic operations we use in our
interpretation are locally continuous, we can show that our 
interpretation function gives rise to a locally continuous
functor. 

\begin{lemma}{Functoriality of $\interpmono{-}$ and $\interp{-}$}
  \begin{enumerate}
  \item For all canonical derivations $\judgeWK[\cdot]{\tau}{\star}$, 
    $\interpmono{\judgeWK[\cdot]{\tau}{\star}}$ is locally continuous. 
  \item For all canonical derivations $\judgeWK[\cdot]{A}{\bigstar}$, 
    $\interp{\judgeWK[\cdot]{A}{\bigstar}}$ is locally continuous. 
  \item $H$ is a locally continuous functor
  \item $\mathcal{K}$ is a locally continuous functor. 
  \end{enumerate}
\end{lemma}

\begin{proof}
 The proof of the first case follows by structural
induction on the canonical derivations of monotypes. This is then used
as a lemma in the proof of the second case, which is done via a
structural induction on the canonical derivations of polytypes. This
then lets us prove the third case, that $H$ is a locally-continuous
functor, because we can work from the inside out, using the fact that
set-indexed sums and products of locally-continuous functors are
themselves locally continuous. Finally, since $\mathcal{K}$ is just 
$H \to K_O$, we know it is locally continuous also. 
\end{proof}

\ \\\noindent 
Observe that $\mathcal{K}$, applied to any arguments, yields a pointed
domain, since the Sierpinski domain is pointed, and a continuous
function space into a pointed domain is itself pointed. Hence our
functor $K$ is also a functor into $CPO_\bot$, the category of
complete \emph{pointed} partial orders and continuous functions. Now,
we can apply Pitts's theorem to solve the recursive domain equation $K
\cong \mathcal{K}(K, K)$.


Now, we give the semantics of the expression and command languages,
with functions $\interpE{\judgeE{\Gamma}{e}{A}}$ and
$\interpC{\judgeE{\Gamma}{c}{A}}$. Since we have two contexts, we will
need two environments, one a type environment (as before, a tuple of
closed type expressions), and the other, a value environment
(consisting of a tuple of values of the appropriate type). In other
words, the interpretation of a term is a type-indexed family of
morphisms in $CPO$.

We give the interpetation of contexts in
Figure~\ref{lang-context-interp}. In this definition, we take the
usual liberties in not explicitly giving the isomorphisms necessary to
implement structural rules like Exchange. There are two
mutually-recursive interpretation functions, $\interpE{\Gamma}{e}{A}$
and $\interpC{\Gamma}{c}{A}$, which interpret pure expressions and
computations, whose definitions are given in
figure~\ref{lang-pure-interp} and figure~\ref{lang-monadic-interp},
respectively.

Most of the base type constructors are interpreted in terms of the
underlying categorical constructions: pair types are categorical
products, sums are categorical sums, functions are exponentials, and
natural numbers are interpreted as a natural numbers object. This
means that we can give semantics for these types in terms of the 


\begin{lemma}{(Computations form a Monad)} 
The functor $T(A) = (A \to K) \to K$ forms a monad in CPO. 
\end{lemma}

\begin{proof}
We need to give a unit (a family of arrows $\eta_A : A \to T(A)$) and a lift
operations (given $f : A \to T(B)$, we need to give $f^* : T(A) \to T(B)$, 
such that the following equations hold:
\begin{enumerate}
\item $\eta_A^* = id_{T(A)}$
\item $f^* \circ \eta_A = f$ 
\item $f^* \circ g^* = (f^* \circ g)^*$
\end{enumerate}
(Technically, these conditions are the conditions for a Kleisli
triple, which is equivalent to a monad.)  We can now define $\eta_A$
and $f^*$ in terms of the internal language of CPO as follows:

\begin{displaymath}
  \begin{array}{lcl}
    \eta_A(a) & = & \lambda k.\; k\;a \\
    f^*       & = & \lambda a' : (A \to K) \to K. \lambda k_b : (B \to K).\;
                         a' (\lambda a.\; f\;a\;k_b)
  \end{array}
\end{displaymath}

\begin{enumerate}
\item The proof of the first equation is as follows: 
  \begin{displaymath}
    \begin{array}{lcl}
      \eta_A^* 
        & = & \lambda a'.\;\lambda k_{A}.\; 
                 a'(\lambda a.\; \eta_A\;a\;k_{A}) \\ 
        & = & \lambda a'.\;\lambda k_{A}.\; 
                 a'(\lambda a.\; k_A\;a) \\ 
        & = & \lambda a'.\;\lambda k_{A}.\; a'\;k_A \\
        & = & \lambda a'.\;a' \\
        & = & id_{T(A)} \\
    \end{array}
  \end{displaymath}

\item The proof of the second equation is as follows: 
  \begin{displaymath}
    \begin{array}{lcl}
      f^* \circ \eta_A 
        & = & \lambda a.\; f^*(\eta_A(a)) \\
        & = & \lambda a.\; f^*(\lambda k.\;k\;a) \\
        & = & \lambda a.\; \lambda k_a.\; 
                (\lambda k.\;k\;a)\;(\lambda a.\;f\;a\;k_a) \\
        & = & \lambda a.\;\lambda k_a.\;
                  (\lambda a.\;f\;a\;k_a)\;a \\
        & = & \lambda a.\;\lambda k_a.\; f\;a\;k_a \\
        & = & \lambda a.\; f\;a \\
        & = & f \\                       
    \end{array}
  \end{displaymath}

\item The proof of the third equation is as follows: 
  \begin{displaymath}
    \begin{array}{lcl}
      f^* \circ g^* 
      & = & \lambda a''.\; f^*(g^*(a'')) \\
      & = & \lambda a''.\; f^*(\lambda k.\;a''(\lambda a.\;g\;a\;k)) \\
      & = & \lambda a''.\;\lambda k.\;
              (\lambda k.\;a''(\lambda a.\;g\;a\;k))\;
              (\lambda a_1.\; f\;a\;k) \\
      & = & \lambda a''.\;\lambda k.\;
              a''(\lambda a.\;g\;a\;(\lambda a_1.\;f\;a_1\;k)) \\
      & = & \lambda a''.\;\lambda k.\;
               a''(\lambda a.\;f^*\; (g a)\; k) \\
      & = & \lambda a''.\;\lambda k.\;
               a''(\lambda a.\;(f^* \circ g)\;a\;k) \\
      & = & (f^* \circ g)^* \\       
    \end{array}
  \end{displaymath}
\end{enumerate}
\end{proof}

\begin{figure}
\begin{displaymath}
  \begin{array}{lcl}
  \interp{\Theta \vdash \cdot}\theta & = & 1 \\
  \interp{\Theta \vdash \Gamma, x:A}\theta & = & 
       \interp{\Theta \vdash \Gamma}\theta 
       \times 
       \interp{\Theta \vdash A:\bigstar}\theta \\
  \end{array}
\end{displaymath}
\caption{Interpretation of Program Contexts}
\label{lang-context-interp}
\end{figure}

\begin{figure}
\begin{displaymath}
  \begin{array}{lcl}
    \interpE{\judgeE{x_1:A_1, \ldots, x_n:A_n}{x_i}{A_i}}\;\theta\;\gamma
       & = & \pi_i(\gamma)
    \\
    \interpE{\judgeE{\Gamma}{\fun{x}{A}{e}}{A \to B}}\;\theta\;\gamma
       & = & \semfun{v}{\interpE{\judgeE{\Gamma, x:A}{e}{B}}\;\theta\;(\gamma, v)} \\
    \interpE{\judgeE{\Gamma}{e\;e'}{B}}\;\theta\;\gamma
       & = & 
       (\interpE{\judgeE{\Gamma}{e}{A\to B}}\;\theta\;\gamma)\;
       (\interpE{\judgeE{\Gamma}{e'}{A}}\theta\;\gamma)
    \\
    \interpE{\judgeE{\Gamma}{\unit}{\unittype}}\;\theta\;\gamma
       & = & 
       *
    \\
    \interpE{\judgeE{\Gamma}{\pair{e_1}{e_2}}{A_1\times A_2}}\;\theta\;\gamma
       & = & 
          \sempair{\interpE{\judgeE{\Gamma}{e_1}{A_1}}\;\theta\;\gamma}
                  {\interpE{\judgeE{\Gamma}{e_2}{A_2}}\;\theta\;\gamma}
    \\
    \interpE{\judgeE{\Gamma}{\fst{e}}{A_1}}\;\theta\;\gamma
       & = & 
       \pi_1(\interpE{\judgeE{\Gamma}{e}{A_1\times A_2}}\;\theta\;\gamma)
    \\
    \interpE{\judgeE{\Gamma}{\snd{e}}{A_2}}\;\theta\;\gamma
       & = & 
       \pi_2(\interpE{\judgeE{\Gamma}{e}{A_1\times A_2}}\;\theta\;\gamma)
    \\
    \interpE{\judgeE{\Gamma}{\inl{e}}{A_1+A_2}}\;\theta\;\gamma 
       & = & 
       \iota_1(\interpE{\judgeE{\Gamma}{e}{A_1}}\;\theta\;\gamma)
    \\
    \interpE{\judgeE{\Gamma}{\inr{e}}{A_1+A_2}}\;\theta\;\gamma 
       & = & 
       \iota_2 (\interpE{\judgeE{\Gamma}{e}{A_2}}\;\theta\;\gamma)
    \\
    \interpE{\judgeE{\Gamma}{\Case{e}{x}{e_1}{y}{e_2}}{C}}\;\theta\;\gamma
       & = & 
          \mbox{let } a = \interpE{\judgeE{\Gamma}{e}{A_1 + A_2}}\theta\;\gamma
          \mbox{ in}\\
       &   & 
          \mbox{let } f_1 = \semfun{v_1}{\interpE{\judgeE{x:A_1, \Gamma}{e_1}{C}}\theta\;(\gamma, v_1)}
          \mbox{ in}\\
       &   & 
          \mbox{let } f_2 = \semfun{v_2}{\interpE{\judgeE{y:A_2, \Gamma}{e_2}{C}}\theta\;(\gamma, v_2)} 
          \mbox{ in}\\
       &   &\;\;
           [f_1, f_2](a)
    \\
    \interpE{\judgeE{\Gamma}{\z}{\N}}\;\theta\;\gamma
       & = & 
         0 
    \\
    \interpE{\judgeE{\Gamma}{\s{e}}{\N}}\;\theta\;\gamma
       & = & 
         s (\interpE{\judgeE{\Gamma}{e}{\N}}\;\theta\;\gamma)
    \\
    \interpE{\judgeE{\Gamma}{\iter{e}{e_0}{x}{e_1}}{A}}\;\theta\;\gamma
    &=& 
       \mbox{let } 
          a = \interpE{\judgeE{\Gamma}{e}{\N}}\;\theta\;\gamma  
       \mbox{ in}\\
    & &\mbox{let } 
          i = \interpE{\judgeE{\Gamma}{e_0}{A}}\;\theta\;\gamma
       \mbox{ in}\\
    & &\mbox{let } 
          s = \semfun{v}{\interpE{\judgeE{\Gamma,x:A}{e_1}{A}}\;\theta\;(\gamma,v)}  
       \mbox{ in}\\
    & &\mbox{ }iter_A[i, s](a)
   \\
   \interpE{\judgeE{\Gamma}{\comp{c}}{\monad{A}}}\;\theta\;\gamma
   & = & 
     \interpC{\judgeC{\Gamma}{c}{A}}\;\theta\;\gamma
   \\
   \interpE{\judgeE{\Gamma}{\fix{x:D}{e}}{D}}\;\theta\;\gamma
   & = & 
     fix(\semfun{v}{(\interpE{\judgeE{\Gamma, x:D}{e}{D}}\;\theta\;(\gamma,v))})
   \\
   \interpE{\judgeE{\Gamma}{\Fun{\alpha}{\kappa}{e}}
                           {\forall \alpha:\kappa.\; A}}\;\theta\;\gamma 
   & = & 
     \lambda \tau:\interp{\kappa}.\; 
        (\interpE{\judgeE[\Theta, \alpha:\kappa]{\Gamma}{e}{A}}\;(\theta, \tau)\;\gamma
   \\
   \interpE{\judgeE{\Gamma}{e\;\tau}{A[\tau/\alpha]}}\;\theta\;\gamma
   & = & 
     \interp{\judgeE{\Gamma}{e}{\forall \alpha:\kappa.\;A}}\;\theta\;\gamma\;
            [\theta(\tau)]
   \\
   \interpE{\judgeE{\Gamma}{\pack{\tau}{e}}{\exists \alpha:\kappa.\;A}}
           \;\theta\;\gamma
   & = & 
     ([\theta(\tau)], \interpE{\judgeE{\Gamma}{e}{A[\tau/\alpha]}}\;\theta\;\gamma)
   \\
   \interpE{\judgeE{\Gamma}{\unpack{\alpha}{x}{e}{e'}}{B}}\;\theta\;\gamma
   & = & 
   (\semfun{\pair{\tau}{v}}
           {\interpE{\judgeE[\Theta, \alpha:\kappa]{\Gamma, x:A}{e'}{B}}
                    \;(\theta, \tau)\;(\gamma,v)})
   \\
   & & \;\;
   (\interpE{\judgeE{\Gamma}{e}{\exists \alpha:\kappa.\;A}}\;\theta\;\gamma)
   \\
   \interpE{\judgeE{\Gamma}{e}{B}}\;\theta\;\gamma 
   & = & 
     \interpE{\judgeE{\Gamma}{e}{A}}\;\theta\;\gamma \mbox{ when } \judgeKeq{A}{B}{\bigstar}
     \mbox{ by EKeq}
   \\
  \end{array}
\end{displaymath}
\caption{Interpretation of Pure Terms}
\label{lang-pure-interp}
\end{figure}

\begin{figure}
\begin{displaymath}
  \begin{array}{lcl}
    \interpC{\judgeC{\Gamma}{e}{A}}\;\theta
    & = & 
    \eta_{\interp{A}\;\theta} \circ \interpE{\judgeE{\Gamma}{e}{A}}\;\theta
    \\
    \interpC{\judgeC{\Gamma}{\letv{x}{e}{c}}{B}}\;\theta\;\gamma
    & = & 
       \mbox{let }c : \interp{\monad{A}}\theta = \interpE{\judgeE{\Gamma}{e}{\monad{A}}}\;\theta\;\gamma \mbox{ in}\\
    && \mbox{let }f : \interp{A}\theta \to \interp{\monad{B}}\theta = 
                  \lambda v.\;\interpE{\judgeE{\Gamma, x:A}{c}{B}}\;\theta\; (\gamma, v) \mbox{ in}\\
    && \;\; f^{*}(c)
    \\
    \interpC{\judgeC{\Gamma}{!e}{A}}\;\theta\;\gamma
    & = &
       \mbox{let }l = \interpE{\judgeE{\Gamma}{e}{\reftype{A}}}\;\theta\;\gamma 
       \mbox{ in}\\
    && \;\;\lambda k.\;\lambda (L, h).\; 
        \left\{ \begin{array}{ll}
                  k\; (h\;l)\; (L, h) & \mbox{when } l \in L \\
                  \top                & \mbox{otherwise} \\
        \end{array}
        \right.
    \\
    \interpC{\judgeC{\Gamma}{e := e'}{\unittype}}\;\theta\;\gamma 
    & = & 
       \mbox{let }l = \interpE{\judgeE{\Gamma}{e}{\reftype{A}}}\;\theta\;\gamma 
    \mbox{ in}\\
    && \mbox{let }v = \interpE{\judgeE{\Gamma}{e'}{A}}\;\theta\;\gamma \mbox{ in}\\
    && \;\; \lambda k.\;\lambda (L, h).\; 
              \left\{ \begin{array}{ll}
                       k \unit (L, [h|l:v]) & \mbox{when } l \in L \\
                      \top                & \mbox{otherwise} \\
              \end{array}
              \right.
    \\
    \interpC{\judgeC{\Gamma}{\newref{A}{e}}{\reftype{A}}}\;\theta\;\gamma 
    & = & 
      \mbox{let }v = \interpE{\judgeE{\Gamma}{e}{A}}\;\theta\;\gamma \mbox{ in}\\
    && \;\;\lambda k.\;\lambda (L, h).\; \\
    && \;\;\qquad \mbox{let }l = (max(L) + 1, A) \mbox{ in}\\
    && \;\;\qquad k\;l\;(L \cup \setof{l}, [h|l:v])
    \\
  \end{array}
\end{displaymath}
\caption{Interpretation of Computations}
\label{lang-monadic-interp}
\end{figure}



\begin{figure}
\begin{mathpar}
\inferrule*[right=EqUnit]
          {\judgeE{\Gamma}{e}{\unittype} \\
           \judgeE{\Gamma}{e'}{\unittype} }
          {\judgeEq{\Gamma}{e}{e'}{\unittype}}
\\
% \inferrule*[right=EqPairCong]
%           {\judgeEq{\Gamma}{e_1}{e'_1}{A_1} \\
%            \judgeEq{\Gamma}{e_2}{e'_2}{A_2}}
%           {\judgeEq{\Gamma}{\pair{e_1}{e_2}}{\pair{e'_1}{e'_2}}{A_1\times A_2}}
% \and
% \inferrule*[right=EqFstCong]
%           {\judgeEq{\Gamma}{e}{e'}{A_1\times A_2}}
%           {\judgeEq{\Gamma}{\fst{e}}{\fst{e'}}{A_1}}
% \and
% \inferrule*[right=EqFstCong]
%           {\judgeEq{\Gamma}{e}{e'}{A_1\times A_2}}
%           {\judgeEq{\Gamma}{\snd{e}}{\snd{e'}}{A_2}}
% \\
\inferrule*[right=EqPairFst]
          {\judgeE{\Gamma}{\pair{e_1}{e_2}}{A_1\times A_2}}
          {\judgeEq{\Gamma}{\fst{\pair{e_1}{e_2}}}{e_1}{A_1}}
\and
\inferrule*[right=EqPairSnd]
          {\judgeE{\Gamma}{\pair{e_1}{e_2}}{A_1\times A_2}}
          {\judgeEq{\Gamma}{\snd{\pair{e_1}{e_2}}}{e_2}{A_2}}
\and
\inferrule*[right=EqPairEta]
          {\judgeE{\Gamma}{e}{A_1\times A_2}}
          {\judgeEq{\Gamma}{e}{\pair{\fst{e}}{\snd{e}}}{A_1\times A_2}}
\\
% \inferrule*[right=EqFunCong]
%           {\judgeEq{\Gamma, x:A}{e}{e'}{B} \\
%            \judgeKeq{A}{A'}{\bigstar}}
%           {\judgeEq{\Gamma}{\fun{x}{A}{e}}{\fun{x}{A'}{e'}}{B}}
% \and
% \inferrule*[right=EqAppCong]
%           {\judgeEq{\Gamma}{e_1}{e'_1}{A \to B} \\
%            \judgeEq{\Gamma}{e_2}{e'_2}{A}}
%           {\judgeEq{\Gamma}{e_1\;e_2}{e'_1\;e'_2}{B}}
% \and
\inferrule*[right=EqFunEta]
          {\judgeEq{\Gamma, x:A}{e\;x}{e'\;x}{B}}
          {\judgeEq{\Gamma}{e}{e'}{A \to B}}
\and
\inferrule*[right=EqFunBeta]
          {\judgeE{\Gamma}{(\fun{x}{A}{e})\;e'}{B}}
          {\judgeEq{\Gamma}{(\fun{x}{A}{e})\;e'}{[e'/x]e}{B}}
\\
% \inferrule*[right=EqInlCong]
%           {\judgeEq{\Gamma}{e}{e'}{A}}
%           {\judgeEq{\Gamma}{\inl{e}}{\inl{e'}}{A+B}}
% \and
% \inferrule*[right=EqInrCong]
%           {\judgeEq{\Gamma}{e}{e'}{B}}
%           {\judgeEq{\Gamma}{\inr{e}}{\inr{e'}}{A+B}}
% \and
% \inferrule*[right=EqCaseCong]
%           {\judgeEq{\Gamma}{e}{e'}{A+B} \\
%            \judgeEq{\Gamma, x:A}{e_1}{e'_1}{C} \\
%            \judgeEq{\Gamma, y:B}{e_2}{e'_2}{C} }
%           {\judgeEq{\Gamma}{\Case{e}{x}{e_1}{y}{e_2}}{\Case{e'}{x}{e'_1}{y}{e'_2}}{C}}
% \and
\inferrule*[right=EqSumInlBeta]
          {\judgeE{\Gamma}{\Case{\inl{e}}{x}{e_1}{y}{e_2}}{C}}
          {\judgeEq{\Gamma}{\Case{\inl{e}}{x}{e_1}{y}{e_2}}{[e/x]e_1}{C}}
\and
\inferrule*[right=EqSumInrBeta]
          {\judgeE{\Gamma}{\Case{\inr{e}}{x}{e_1}{y}{e_2}}{C}}
          {\judgeEq{\Gamma}{\Case{\inr{e}}{x}{e_1}{y}{e_2}}{[e/x]e_1}{C}}
\and
\inferrule*[right=EqSumEta]
          {\judgeE{\Gamma}{e}{A+B} \\
           \judgeE{\Gamma, z:A+B}{e'}{C}}
          {\judgeEq{\Gamma}{[e/z]e'}{\Case{e}{x}{[\inl{x}/z]e'}
                                             {y}{[\inr{x}/z]e'}}{C}}
\\
\inferrule*[right=EqMonad]
          {\judgeEqC{\Gamma}{c}{c'}{A}}
          {\judgeEq{\Gamma}{\comp{c}}{\comp{c'}}{\monad{A}}}
\and
\inferrule*[right=EqFix]
          {\judgeE{\Gamma}{\fix{x:D}{e}}{D}}
          {\judgeEq{\Gamma}{\fix{x:D}{e}}{[\fix{x:D}{e}/x]e}{D}}
\end{mathpar}
\caption{Equality Rules for Sums, Products, Exponentials, and Suspended Computations}
\label{lang-pure-eq-1}
\end{figure}

\begin{figure}
\begin{mathpar}
\inferrule*[right=EqNatZBeta]
          {\judgeE{\Gamma}{\iter{\z}{e_0}{x}{e_1}}{A}}
          {\judgeEq{\Gamma}{\iter{\z}{e_0}{x}{e_1}}{e_0}{A}}
\and
\inferrule*[right=EqNatSBeta]
          {\judgeE{\Gamma}{\iter{\s{e}}{e_0}{x}{e_1}}{A}}
          {\judgeEq{\Gamma}{\iter{\s{e}}{e_0}{x}{e_1}}{e_1[\iter{e}{e_0}{x}{e_1}/x]}{A}}
\and
\inferrule*[right=EqNatEta]
          {\judgeE{\Gamma}{e_0}{A} \\
           \judgeE{\Gamma, x:A}{e_1}{A} \\
           \judgeE{\Gamma}{e}{A} \\
           \judgeEq{\Gamma}{[\z/n]e}{e_0}{A} \\
           \judgeEq{\Gamma, m:\N}{[\s{m}/n]e}{[[m/n]e/x]e_1}{A}}
          {\judgeEq{\Gamma, n:\N}{e}{\iter{n}{e_0}{x}{e_1}}{A}}
\and
\inferrule*[right=EqAllBeta]
          {\judgeE{\Gamma}{\Fun{\alpha}{\kappa}{e}}
                          {\forall \alpha:\kappa.\;A} \\
           \judgeWK{\tau}{\kappa}}
          {\judgeEq{\Gamma}{(\Fun{\alpha}{\kappa}{e})\;\tau}
                           {[\tau/\alpha]e}
                           {[\tau/\alpha]A}}
\and
\inferrule*[right=EqAllEta]
          {\judgeEq[\Theta, \alpha:\kappa]{\Gamma}{e\;\alpha}{e'\;\alpha}{A} \\
           \judgeE{\Gamma}{e}{\forall \alpha:\kappa.\;A} \\
           \judgeE{\Gamma}{e'}{\forall \alpha:\kappa.\;A} \\
           \Theta \vdash \Gamma}
          {\judgeEq{\Gamma}{e}{e'}{\forall \alpha:\kappa.\;A}}
\and
\inferrule*[right=EqExistsBeta]
          {\judgeE{\Gamma}{\pack{\tau}{e}}{\exists \alpha:\kappa.\;A} \\
           \judgeE[\Theta, \alpha:\kappa]{\Gamma, x:A}{e'}{B}}
          {\judgeEq{\Gamma}{\unpack{\alpha}{x}{\pack{\tau}{e}}{e'}}
                          {[\tau/\alpha][e/x]e'}
                          {B}}
\and
\inferrule*[right=EqExistsEta]
          {\judgeE{\Gamma, z:\exists \alpha:\kappa.\;A}{e'}{B} \\ 
           \judgeE{\Gamma}{e}{\exists \alpha:\kappa.\;A}}
          {\judgeEq{\Gamma}{\unpack{\alpha}{x}{e}{[\pack{\alpha}{x}/z]e'}}{[e/z]e'}{B}}
\end{mathpar}
\caption{Equality Rules for Numbers, Universals, and Existentials}
\label{lang-pure-eq-2} 
\end{figure}

\begin{figure}
\begin{mathpar}
\inferrule*[right=EqCommandEta]
          {\judgeC{\Gamma}{c}{A}}
          {\judgeEqC{\Gamma}{e}{\letv{x}{e}{x}}{A}}
\and
\inferrule*[right=EqCommandBeta]
          {\judgeC{\Gamma}{\letv{x}{\comp{e}}{c}}{A}}
          {\judgeEqC{\Gamma}{\letv{x}{\comp{e}}{c}}{[e/x]c}{A}}
\and
\inferrule*[right=EqCommandComm]
          {\judgeC{\Gamma}{\letv{x}{\comp{\letv{y}{e}{c_1}}}{c_2}}{A}}
          {\judgeEqC{\Gamma}{\letv{x}{\comp{\letv{y}{e}{c_1}}}{c_2}}
                            {\letv{y}{e}{\letv{x}{\comp{c_1}}}{c_2}}{A}}
\end{mathpar}
\caption{Equality Rules for Computations}
\label{lang-monad-eq}  
\end{figure}


\begin{figure}
\begin{mathpar}
\inferrule*[right=EqRefl]
          {\judgeE{\Gamma}{e}{A}}
          {\judgeEq{\Gamma}{e}{e}{A}}
\and
\inferrule*[right=EqSymm]
          {\judgeEq{\Gamma}{e}{e'}{A}}
          {\judgeEq{\Gamma}{e'}{e}{A}}
\and
\inferrule*[right=EqTrans]
          {\judgeEq{\Gamma}{e}{e'}{A} \\
           \judgeEq{\Gamma}{e'}{e''}{A}}
          {\judgeEq{\Gamma}{e}{e''}{A}}
\and
\inferrule*[right=EqSubst]
          {\judgeEq{\Gamma, x:A}{e_1}{e_2}{B} \\
           \judgeEq{\Gamma}{e'_1}{e'_2}{A}}
          {\judgeEq{\Gamma}{[e'_1/x]e_1}{[e'_2/x]e'_2}{B}}
\and
\inferrule*[right=EqCommandRefl]
          {\judgeC{\Gamma}{c}{A}}
          {\judgeEqC{\Gamma}{c}{c}{A}}
\and
\inferrule*[right=EqCommandSymm]
          {\judgeEqC{\Gamma}{c}{c'}{A}}
          {\judgeEqC{\Gamma}{c'}{c}{A}}
\and
\inferrule*[right=EqCommandTrans]
          {\judgeEqC{\Gamma}{c}{c'}{A} \\
           \judgeEqC{\Gamma}{c'}{c''}{A}}
          {\judgeEqC{\Gamma}{c}{c''}{A}}
\and
\inferrule*[right=EqCommandSubst]
          {\judgeEqC{\Gamma, x:A}{c_1}{c_2}{B} \\
           \judgeEq{\Gamma}{e_1}{e_2}{A}}
          {\judgeEqC{\Gamma}{[e_1/x]c_1}{[e_2/x]c_2}{B}}
\end{mathpar}
\caption{Congruence Rules for Equality}
\label{lang-cong-eq}
\end{figure}

\begin{lemma}{(Soundness of Substitution)}
\begin{enumerate}
\item If we know that $\judgeE{\Gamma, y:A, \Gamma'}{e}{B}$ and $\judgeE{\Gamma}{e'}{A}$
  and $\Theta \vdash \theta$, 
  then $\interpE{\judgeE{\Gamma}{[e'/y]e}{B}}\;\theta\;(\gamma,\gamma') = 
        \interpE{\judgeE{\Gamma, y:A}{e}{B}}\;\theta 
                \;\left(\gamma, 
                        \interpE{\judgeE{\Gamma}{e'}{A}}\;\theta\;\gamma,
                        \gamma'
                \right)$

\item If we know that $\judgeC{\Gamma, y:A, \Gamma'}{c}{B}$ and $\judgeE{\Gamma}{e'}{A}$
  and $\Theta \vdash \theta$, 
  then $\interpC{\judgeE{\Gamma}{[e'/y]c}{B}}\;\theta\;(\gamma,\gamma') = 
        \interpC{\judgeC{\Gamma, y:A}{c}{B}}\;\theta \;
                \left(\gamma,
                      \interpE{\judgeE{\Gamma}{e'}{A}}\;\theta\;\gamma,
                      \gamma'
                \right)$
\end{enumerate}
\end{lemma}

\begin{proof}
This property follows by a mutual structural induction on the derivations 
$\judgeE{\Gamma, y:A}{e}{B}$ and $\judgeC{\Gamma, y:A}{c}{C}$. 

First, we'll do the cases for pure terms. (When there's no confusion,
we'll write $\interpE{e}$ for
$\interpE{\judgeE{\Gamma}{e}{A}}\;\theta$.)

\begin{itemize}
\item case EVar: 
  There are two cases, depending on whether the $e = x_i$, or $e = y$

  \begin{enumerate}
  \item If $e = x_i$ for some $i$, then $[e'/y]x_i = x_i$, and so we have
   $\interpE{\judgeE{\Gamma, \Gamma'}{[e'/y]x_i}{B}}\;\theta\;(\gamma, \gamma')$

    \begin{eqnproof}
      \eline
            {\interpE{\judgeE{\Gamma, \Gamma'}{x_i}{B}}\;\theta\;(\gamma, \gamma')}
            { Subst.}
      \eline{\pi_i(\gamma, \gamma')}
            { Semantics}
      \eline{\pi_i(\gamma, \interpE{\judgeE{\Gamma}{e'}{A}}\;\theta\;\gamma, \gamma')}
            { Adjusting $i$}
      \eline{\interpE{\judgeE{\Gamma, y:A, \Gamma'}{x_i}{B}}\;\theta\;(\gamma, \interpE{\judgeE{\Gamma}{e'}{A}}\;\theta, \gamma')}
            { Semantics}
    \end{eqnproof}

  \item If $e = y$, then we have $\interpE{\judgeE{\Gamma, \Gamma'}{[e'/y]y}{A}\;\theta\;(\gamma, \gamma')}$

    \begin{eqnproof}
      \eline
            {\interpE{\judgeE{\Gamma, \Gamma'}{e'}{A}}\;\theta\;(\gamma,\gamma')}
            { Subst. }
      \eline{\interpE{\judgeE{\Gamma}{e'}{A}}\;\theta\;\gamma}
            { Since $FV(e') \cap \Gamma = \emptyset$}
      \eline{\interpE{\judgeE{\Gamma, y:A, \Gamma'}{y}{A}}\;\theta
              \;(\gamma, \interpE{\judgeE{\Gamma}{y}{A}}\;\theta\;\gamma, \gamma')}
            { Semantics}
    \end{eqnproof}
  \end{enumerate}

\item case ELam: We have $\interpE{\judgeE{\Gamma, \Gamma'}{[e'/y](\fun{x}{B}{e})}{B \to B'}\;\theta\;(\gamma, \gamma')}$

  \begin{eqnproof}
    \eline
          {\interpE{\judgeE{\Gamma, \Gamma'}{\fun{x}{B}{[e'/y]e}}{B \to B'}}\;\theta\;(\gamma, \gamma')}
          {Subst.}
    \eline{\semfun{v}{\interpE{\judgeE{\Gamma, \Gamma', x:B}{[e'/y]e}{B'}}\;\theta\;(\gamma, \gamma', v)}}
          {Semantics}

    \eline{\semfun{v}
            {\interpE{\judgeE{\Gamma, y:A, \Gamma', x:B}{e}{B'}}\;\theta\;
                       (\gamma, \interpE{\judgeE{\Gamma}{e'}{A}}\;\theta\;\gamma, 
                        \gamma', v)}}     
          {IH}
    \eline{\interpE{\judgeE{\Gamma, y:A, \Gamma'}{\fun{x}{B}{e}}{B \to B'}}
           \;\theta\;(\gamma, 
                      \interpE{\judgeE{\Gamma}{e'}{A}}\;\theta\;\gamma, 
                      \gamma')}
          {Semantics}
  \end{eqnproof}

\item Case EApp: We have 
   $\interpE{\judgeE{\Gamma, \Gamma'}{[e'/y](e_1\;e_2)}{B}}\;\theta\;
       (\gamma, \gamma')$

  \begin{eqnproof}
    \eline{\interpE{\judgeE{\Gamma, \Gamma'}{[e'/y]e_1\;[e'/y]e_2}{B}}\;\theta\;
           (\gamma, \gamma')}
          {Subst.}
    \eline{\interpE{\judgeE{\Gamma, \Gamma'}{[e'/y]e_1}{B_2 \to B}}\;\theta\;(\gamma, \gamma')\;\;
           \interpE{\judgeE{\Gamma, \Gamma'}{[e'/y]e_2}{B_2}}\;\theta\;(\gamma, \gamma')}
          {Semantics}
     \eline{\interpE{\judgeE{\Gamma, y:A, \Gamma'}{e_1}{B_2 \to B}}\;\theta\;
              \gamma''\;\;
            \interpE{\judgeE{\Gamma, y:A, \Gamma'}{e_2}{B_2}}\;\theta\;\gamma''}
           {IH, IH}
     \eline{\interpE{\judgeE{\Gamma, y:A, \Gamma'}{e_1\;e_2}{B}}\;\theta\;\gamma''}
           {Semantics}
  \end{eqnproof}

  (Where $\gamma'' = (\gamma, \interpE{\judgeE{\Gamma}{e'}{A}}\;\theta\;\gamma, \gamma')$)

\item case EUnit: We have $\interpE{\judgeE{\Gamma, \Gamma'}{[e'/y]\unit}{\unittype}}\;\theta\;(\gamma, \gamma')$
  \begin{eqnproof}
   \eline{\interpE{\judgeE{\Gamma, \Gamma'}{\unit}{\unittype}}\;\theta\;(\gamma, \gamma')}
         {Substitution}
   \eline{*}
         {Semantics}
   \eline{\interpE{\judgeE{\Gamma, y:A, \Gamma'}{\unit}{\unittype}}\;\theta\;
          (\gamma, \interpE{\judgeE{\Gamma}{e'}{A}}\;\theta\;\gamma, \gamma')}
         {Semantics}
  \end{eqnproof}


\item case EPair: We have $\interpE{\judgeE{\Gamma, \Gamma'}{[e'/y]\pair{e_1}{e_2}}{B}}\;\theta\;
       (\gamma, \gamma')$ 

  \begin{eqnproof}
    \eline{\interpE{\judgeE{\Gamma, \Gamma'}{\pair{[e'/y]e_1}{[e'/y]e_2}}{B_1\times B_2}}\;\theta\;(\gamma, \gamma')}
          {Subst.}
    \eline{\sempair{\interpE{\judgeE{\Gamma, \Gamma'}{[e'/y]e_1}{B_1}}\;\theta\;(\gamma, \gamma')}
                   {\interpE{\judgeE{\Gamma, \Gamma'}{[e'/y]e_2}{B_2}}\;\theta\;(\gamma, \gamma')}}
          {Semantics}
     \eline{\sempair{\interpE{\judgeE{\Gamma, y:A, \Gamma'}{e_1}{B_1}}\;\theta\;
                     \gamma''}
                    {\interpE{\judgeE{\Gamma, y:A, \Gamma'}{e_2}{B_2}}\;\theta\;\gamma''}}
           {IH, IH}
     \eline{\interpE{\judgeE{\Gamma, y:A, \Gamma'}{\pair{e_1}{e_2}}{B}}\;\theta\;\gamma''}
           {Semantics}
  \end{eqnproof}

  (Where $\gamma'' = (\gamma, \interpE{\judgeE{\Gamma}{e'}{A}}\;\theta\;\gamma, \gamma')$)


\item Case EFst: We have $\interpE{\judgeE{\Gamma, \Gamma'}{[e'/y]\fst{e}}{B_1}}\;\theta\;(\gamma, \gamma')$

  \begin{eqnproof}
    \eline{\interpE{\judgeE{\Gamma, \Gamma'}{\fst{[e'/y]e}}{B_1}}\;\theta\;(\gamma, \gamma')}
          {Substitution}
    \eline{\pi_1(\interpE{\judgeE{\Gamma, \Gamma'}{[e'/y]e}{B_1\times B_2}}\;\theta\;(\gamma, \gamma'))}
          {Semantics}
    \eline{\pi_1(\interpE{\judgeE{\Gamma, y:A, \Gamma'}{e}{B_1\times B_2}}\;\theta\;\gamma'')}
          {IH}
    \eline{\interpE{\judgeE{\Gamma, y:A, \Gamma'}{\fst{e}}{B_1\times B_2}}\;\theta\;\gamma''}
          {Semantics}
  \end{eqnproof}

  (Where $\gamma'' = (\gamma, \interpE{\judgeE{\Gamma}{e'}{A}}\;\theta\;\gamma, \gamma')$)

\item Case ESnd: We have $\interpE{\judgeE{\Gamma, \Gamma'}{[e'/y]\snd{e}}{B_1}}\;\theta\;(\gamma, \gamma')$

  \begin{eqnproof}
    \eline{\interpE{\judgeE{\Gamma, \Gamma'}{\snd{[e'/y]e}}{B_2}}\;\theta\;(\gamma, \gamma')}
          {Substitution}
    \eline{\pi_2(\interpE{\judgeE{\Gamma, \Gamma'}{[e'/y]e}{B_1\times B_2}}\;\theta\;(\gamma, \gamma'))}
          {Semantics}
    \eline{\pi_2(\interpE{\judgeE{\Gamma, y:A, \Gamma'}{e}{B_1\times B_2}}\;\theta\;\gamma'')}
          {IH}
    \eline{\interpE{\judgeE{\Gamma, y:A, \Gamma'}{\snd{e}}{B_1\times B_2}}\;\theta\;\gamma''}
          {Semantics}
  \end{eqnproof}

  (Where $\gamma'' = (\gamma, \interpE{\judgeE{\Gamma}{e'}{A}}\;\theta\;\gamma, \gamma')$)

\item Case EInl: We have $\interpE{\judgeE{\Gamma, \Gamma'}{[e'/y]\inl{e}}{B_1 + B_2}}\;\theta\;(\gamma, \gamma')$

  \begin{eqnproof}
    \eline{\interpE{\judgeE{\Gamma, \Gamma'}{\inl{([e'/y]e)}}{B_1+B_2}}\;\theta\;(\gamma, \gamma')}
          {Substitution}
    \eline{\iota_1(\interpE{\judgeE{\Gamma, \Gamma'}{[e'/y]e}{B_1}}\;\theta\;(\gamma, \gamma'))}
          {Semantics}
    \eline{\iota_1(\interpE{\judgeE{\Gamma, y:A, \Gamma'}{e}{B_1}}\;\theta\;\gamma'')}
          {IH}
    \eline{\interpE{\judgeE{\Gamma, y:A, \Gamma'}{\inl{e}}{B_1+ B_2}}\;\theta\;\gamma''}
          {Semantics}
  \end{eqnproof}

  (Where $\gamma'' = (\gamma, \interpE{\judgeE{\Gamma}{e'}{A}}\;\theta\;\gamma, \gamma')$)


\item Case EInr: We have $\interpE{\judgeE{\Gamma, \Gamma'}{[e'/y]\inl{e}}{B_1 + B_2}}\;\theta\;(\gamma, \gamma')$

  \begin{eqnproof}
    \eline{\interpE{\judgeE{\Gamma, \Gamma'}{\inr{([e'/y]e)}}{B_1+B_2}}\;\theta\;(\gamma, \gamma')}
          {Substitution}
    \eline{\iota_2(\interpE{\judgeE{\Gamma, \Gamma'}{[e'/y]e}{B_2}}\;\theta\;(\gamma, \gamma'))}
          {Semantics}
    \eline{\iota_2(\interpE{\judgeE{\Gamma, y:A, \Gamma'}{e}{B_2}}\;\theta\;\gamma'')}
          {IH}
    \eline{\interpE{\judgeE{\Gamma, y:A, \Gamma'}{\inr{e}}{B_1+ B_2}}\;\theta\;\gamma''}
          {Semantics}
  \end{eqnproof}

  (Where $\gamma'' = (\gamma, \interpE{\judgeE{\Gamma}{e'}{A}}\;\theta\;\gamma, \gamma')$)

\item Case ECase: We have $\judgeE{\Gamma,\Gamma'}{[e'/y]\Case{e}{x_1}{e_1}{x_2}{e_2}}{C}$

  \begin{eqnproof}
    \eline{\judgeE{\Gamma, \Gamma'}{\Case{[e'/y]e}{x_1}{[e'/y]e_1}{x_2}{[e'/y]e_2}}{C}}
          {Substitution}[1em]

    \eclaim[(1)]{\judgeE{\Gamma, \Gamma'}{[e'/y]e}{B_1+B_2}}
           {Inversion}
    \eclaim[(2)]{\judgeE{\Gamma, \Gamma', x_1:B_1}{[e'/y]e_1}{B_1}}
           {Inversion}
    \eclaim[(3)]{\judgeE{\Gamma, \Gamma', x_2:B_2}{[e'/y]e_2}{B_2}}
           {Inversion}
    \eline[\interpE{(1)}]
          {\lambda \theta\;(\gamma, \gamma').\; \interpE{\judgeE{\Gamma, y:A, \Gamma'}{e}{B_1+B_2}}\;\theta\;(\gamma, \interpE{e'}\theta\gamma, \gamma')}
          {IH}
    \eline[\interpE{(2)}]
          {\lambda \theta\;(\gamma, \gamma').\; \interpE{\judgeE{\Gamma, y:A, \Gamma', x_1:B_1}{e_1}{C}}\;\theta\;(\gamma, \interpE{e'}\theta\gamma, \gamma')}
          {IH}
    \eline[\interpE{(3)}]
          {\lambda \theta\;(\gamma, \gamma').\; \interpE{\judgeE{\Gamma, y:A, \Gamma', x_2:B_2}{e_2}{C}}\;\theta\;(\gamma, \interpE{e'}\theta\gamma, \gamma')}
          {IH}
  \end{eqnproof}

  Now, the interpretation $\interpE{\judgeE{\Gamma, \Gamma'}{\Case{[e'/y]e}{x_1}{[e'/y]e_1}{x_2}{[e'/y]e_2}}{C}}\;\theta\;(\gamma, \gamma')$ will be equal to 
$[f_1, f_2]\;a$, where 

  \begin{eqnproof}
    \eline[a]{\interpE{\judgeE{\Gamma, \Gamma'}{[e'/y]e}{B_1+B_2}}\;\theta\;(\gamma, \gamma')}{}
    \eline{\interpE{\judgeE{\Gamma, y:A, \Gamma'}{e}{B_1+B_2}}\;\theta\;(\gamma, \interpE{e'}\;\theta\;\gamma, \gamma')}{}
    \eline[f_1]{\lambda v.\;\interpE{\judgeE{\Gamma, \Gamma', x_1:B_1}{e_1}{C}}\;\theta\;(\gamma,\gamma', v)}{}
    \eline{\lambda v.\;\interpE{\judgeE{\Gamma, y:A, \Gamma', x_1:B_1}{e_1}{C}}\;\theta\;(\gamma, \interpE{e'}\;\theta\;\gamma, \gamma', v)}
               {}
    \eline[f_2]{\lambda v.\;\interpE{\judgeE{\Gamma, \Gamma', x_2:B_2}{e_2}{C}}\;\theta\;(\gamma,\gamma', v)}{}
    \eline{\lambda v.\;\interpE{\judgeE{\Gamma, y:A, \Gamma', x_2:B_2}{e_1}{C}}\;\theta\;(\gamma, \interpE{e'}\;\theta\;\gamma, \gamma', v)}
         {}
  \end{eqnproof}

  Which means that $[f_1, f_2]\;a = \interpE{\judgeE{\Gamma, y:A, \Gamma'}{\Case{e}{x_1}{e_1}{x_2}{y_2}}{C}}\;\theta\;(\gamma, \interpE{e'}\;\theta\;\gamma, \gamma')$

\item case EZero: We have $\interpE{\judgeE{\Gamma, \Gamma'}{[e'/y]\z}{\N}}\;\theta\;(\gamma, \gamma')$
  \begin{eqnproof}
   \eline{\interpE{\judgeE{\Gamma, \Gamma'}{\z}{\N}}\;\theta\;(\gamma, \gamma')}
         {Substitution}
   \eline{0}
         {Semantics}
   \eline{\interpE{\judgeE{\Gamma, y:A, \Gamma'}{\z}{\N}}\;\theta\;
          (\gamma, \interpE{\judgeE{\Gamma}{e'}{A}}\;\theta\;\gamma, \gamma')}
         {Semantics}
  \end{eqnproof}

\item case ESucc: We have $\interpE{\judgeE{\Gamma, \Gamma'}{[e'/y]\s{e}}{\N}}\;\theta\;(\gamma, \gamma')$

  \begin{eqnproof}
    \eline{\interpE{\judgeE{\Gamma, \Gamma'}{\s{[e'/y]e}}{\N}}\;\theta\;(\gamma, \gamma')}
          {Substitution}
    \eline{s(\interpE{\judgeE{\Gamma, \Gamma'}{[e'/y]e}{\N}}\;\theta\;(\gamma, \gamma'))}
          {Semantics}
    \eline{s(\interpE{\judgeE{\Gamma, y:A, \Gamma'}{e}{\N}}\;\theta\;\gamma'')}
          {IH}
    \eline{\interpE{\judgeE{\Gamma, y:A, \Gamma'}{\s{e}}{\N}}\;\theta\;\gamma''}
          {Semantics}
  \end{eqnproof}

  (Where $\gamma'' = (\gamma, \interpE{\judgeE{\Gamma}{e'}{A}}\;\theta\;\gamma, \gamma')$)

\item case EIter: We have $\judgeE{\Gamma, \Gamma'}{[e'/y]\iter{e}{e_0}{x}{e_1}}{C}$
  \begin{eqnproof}
    \eline{\judgeE{\Gamma, \Gamma'}{\iter{[e'/y]e}{[e'/y]e_0}{x}{[e'/y]e_1}}{C}}
          {Substitution}
          [1em]
    \eclaim[(1)]
          {\judgeE{\Gamma, \Gamma'}{[e'/y]e}{\N}}
          {Inversion}
    \eclaim[(2)]
          {\judgeE{\Gamma, \Gamma'}{[e'/y]e_0}{C}}
          {Inversion}
    \eclaim[(3)]
          {\judgeE{\Gamma, \Gamma', x:C}{[e'/y]e_1}{C}}
          {Inversion}
          [1em]
    \eline[\interpE{(1)}]
          {\lambda \theta\;(\gamma, \gamma').\;
            \interpE{\judgeE{\Gamma, y:A, \Gamma'}{e}{\N}}
              \;\theta
              \;(\gamma, \interpE{e'}\;\theta\;\gamma, \gamma')}
          {IH}
    \eline[\interpE{(2)}]
          {\lambda \theta\;(\gamma, \gamma').\;
            \interpE{\judgeE{\Gamma, y:A, \Gamma'}{e_0}{C}}
              \;\theta
              \;(\gamma, \interpE{e'}\;\theta\;\gamma, \gamma')}
          {IH}               
    \eline[\interpE{(3)}]
          {\lambda \theta\;(\gamma, \gamma'').\;
            \interpE{\judgeE{\Gamma, y:A, \Gamma', x:C}{e_0}{C}}
              \;\theta
              \;(\gamma, \interpE{e'}\;\theta\;\gamma, \gamma'')}
          {IH}               
  \end{eqnproof}
  Now, we assume suitable $\theta, \gamma, \gamma'$, and consider the 
  interpretation of 

  $$\interpE{\judgeE{\Gamma, \Gamma'}{\iter{[e'/y]e}{[e'/y]e_0}{x}{[e'/y]e_1}}{C}}\;\theta\;(\gamma, \gamma')$$ 

  This is equal to $iter_C[i,s]\;a$, where:

  \begin{eqnproof}
    \eline[a]{\interpE{(1)}\;\theta\;(\gamma,\gamma')}{}
    \eline{\interpE{\judgeE{\Gamma, y:A, \Gamma'}{e}{\N}}
              \;\theta
              \;(\gamma, \interpE{e'}\;\theta\;\gamma, \gamma')}
          {}
          [1em]
    \eline[i]{\interpE{(2)}\;\theta\;(\gamma, \gamma')}{}
    \eline{\interpE{\judgeE{\Gamma, y:A, \Gamma'}{e_0}{C}}
              \;\theta
              \;(\gamma, \interpE{e'}\;\theta\;\gamma, \gamma')}
          {}
    \eline[s]{\semfun{v}{\interpE{(3)}\;\theta\;(\gamma,\gamma',v)}}
             {}
    \eline{\semfun{v}{
           \interpE{\judgeE{\Gamma, y:A, \Gamma', x:C}{e_1}{C}}
           \;\theta
           \;(\gamma, \interpE{e'}\;\theta\;\gamma, \gamma', v)}}
          {}
  \end{eqnproof}

  Which means that
  \begin{displaymath}
    iter_C[i,s]\;a = \interpE{\judgeE{\Gamma, y:A, \Gamma'}{\iter{e}{[e'/y]e_0}{x}{[e'/y]e_1}}{C}}    \;\theta\;(\gamma, \interpE{e'}\;\theta\;\gamma, \gamma')
  \end{displaymath}

\item case EMonad: We have $\judgeE{\Gamma, \Gamma'}{[e'/y]\comp{c}}{\monad{B}}$. 

  \begin{eqnproof}
    \eline{\judgeE{\Gamma, \Gamma'}{\comp{[e'/y]c}}{\monad{B}}}
          {Substitution}
          [1em]
    \eclaim[\mbox{We have}]
           {\judgeC{\Gamma, \Gamma'}{[e'/y]c}{B}}
           {Inversion}
  \end{eqnproof}

  By mutual induction, we know that for all suitable $\theta, \gamma, \gamma'$,
  \begin{displaymath}
   \interpC{\judgeC{\Gamma,\Gamma'}{[e'/y]c}{B}}\;\theta\;(\gamma,\gamma') = 
   \interpC{\judgeE{\Gamma, y:A, \Gamma'}{c}{B}}
           \;\theta\;
           \;(\gamma, \interpE{e'}\;\theta\;\gamma, \gamma')
  \end{displaymath}

  Therefore we know that $\interpE{\judgeE{\Gamma, \Gamma'}{[e'/y]\comp{c}}{\monad{B}}}\;\theta\;(\gamma, \gamma')$ is

  \begin{eqnproof}
    \eline{\interpC{\judgeC{\Gamma,\Gamma'}{[e'/y]c}{B}}\;\theta\;(\gamma,\gamma)}
          {Semantics}
    \eline{\interpC{\judgeE{\Gamma, y:A, \Gamma'}{c}{B}}
           \;\theta\;
           \;(\gamma, \interpE{e'}\;\theta\;\gamma, \gamma')}
          {See above}
    \eline{\interpE{\judgeE{\Gamma, y:A, \Gamma'}{\comp{c}}{\monad{B}}}\;
           \;\theta\;
           \;(\gamma, \interpE{e'}\;\theta\;\gamma, \gamma')}
          {Semantics}
  \end{eqnproof}

\item case EFix: We have that $\interpE{\judgeE{\Gamma, \Gamma'}{[e'/y](\fix{x:D}{e})}{D}}\;\theta\;(\gamma,\gamma')$ is

  \begin{eqnproof}
    \eline{\interpE{\judgeE{\Gamma, \Gamma'}{\fix{x:D}{([e'/y]e)}}{D}}
           \;\theta\;(\gamma,\gamma')}
          {Substitution}
    \eline{\interpE{fix(\semfun{v}{
             (\interpE{\judgeE{\Gamma, \Gamma', x:D}{[e'/y]e}{D}}
               \;\theta\;(\gamma,\gamma',v))})}}
          {Semantics}
     \eline{\interpE{fix(\semfun{v}{
             (\interpE{\judgeE{\Gamma, y:A, \Gamma', x:D}{e}{D}}
               \;\theta\;(\gamma,\interpE{e'}\;\theta\;\gamma, \gamma',v))})}}
          {IH }
     \eline{\interpE{\judgeE{\Gamma, y:A, \Gamma'}{\fix{x:D}{e}}{D}}
             \;\theta
             \;(\gamma,\interpE{e'}\;\theta\;\gamma, \gamma',v)}
           {Semantics}
  \end{eqnproof}

\item case ETLam: We have that $\interpE{\judgeE{\Gamma,\Gamma'}{[e'/y](\Fun{\alpha}{\kappa}{e})}{\forall \alpha:\kappa.\;B}}\;\theta\;(\gamma,\gamma')$ 

  \begin{eqnproof}
    \eline{\interpE{
             \judgeE{\Gamma,\Gamma'}{\Fun{\alpha}{\kappa}{[e'/y]e}}{\forall \alpha:\kappa.\;B}}\;\theta\;(\gamma, \gamma')}
          {Substitution}
    \eline{\semfun{\tau}{\interpE{
             \;\judgeE[\Theta, \alpha:\kappa]{\Gamma,\Gamma'}{[e'/y]e}{B}}\;(\theta, \tau)\;(\gamma, \gamma')}}
          {Semantics}
    \eline{\semfun{\tau}{\interpE{
             \;\judgeE[\Theta, \alpha:\kappa]{\Gamma,\Gamma'}{e}{B}}
      \;(\theta, \tau)\;(\gamma, \interpE{e'}\;(\theta, \tau)\;\gamma, \gamma')}}
          {IH}
    \eline{\semfun{\tau}{\interpE{
             \;\judgeE[\Theta, \alpha:\kappa]{\Gamma,\Gamma'}{e}{B}}
      \;(\theta, \tau)\;(\gamma, \interpE{e'}\;\theta\;\gamma, \gamma')}}
          {Since $\alpha \not\in FTV(e')$}

    \eline{\interpE{\judgeE{\Gamma, y:A, \Gamma'}
                           {\Fun{\alpha}{\kappa}{e}}
                           {\forall \alpha:\kappa.\;B}}
           \;\theta\;(\gamma, \interpE{e'}\;\theta\;\gamma, \gamma')}
          {Semantics}                 
  \end{eqnproof}

\item case ETApp: We have that $\interpE{\judgeE{\Gamma,\Gamma'}{[e'/y](e\;\tau)}{[\tau/\alpha]B}}\;\theta\;(\gamma, \gamma')$ is 

  \begin{eqnproof}
    \eline{\interpE{
           \judgeE{\Gamma,\Gamma'}{([e'/y]e)\;\tau}{[\tau/\alpha]B}}
           \;\theta\;(\gamma,\gamma')}
          {Substitution}
    \eline{\left(\interpE{\judgeE{\Gamma,\Gamma'}{[e'/y]e}{\forall \alpha:\kappa.\;B}}\;\theta\;(\gamma,\gamma')\right)\;[\theta(\tau)]}
          {Semantics}
    \eline{\left(
             \interpE{\judgeE{\Gamma,y:A,\Gamma'}{e}{\forall \alpha:\kappa.\;B}}
                 \;\theta\;(\gamma,\interpE{e'}\;\theta\;\gamma,\gamma')
           \right)\; [\theta(\tau)]}
          {IH}
    \eline{\interpE{
           \judgeE{\Gamma,\Gamma'}{e\;\tau}{[\tau/\alpha]B}}
           \;\theta\;(\gamma,\interpE{e'}\;\theta\;\gamma, \gamma')}
          {Substitution}
  \end{eqnproof}

\item case EPack: We have that $\interpE{\judgeE{\Gamma,\Gamma'}{[e'/y]\pack{\tau}{e}}{\exists \alpha:\kappa.\;B}}\;\theta\;(\gamma, \gamma')$ is

  \begin{eqnproof}
    \eline{\interpE{
           \judgeE{\Gamma,\Gamma'}
                  {[e'/y]\pack{\tau}{e}}
                  {\exists \alpha:\kappa.\;B}}
           \;\theta\;(\gamma, \gamma')}
          {Substitution}
    \eline{([\theta(\tau)], 
            \interpE{
            \judgeE{\Gamma,\Gamma'}
                   {[e'/y]e}
                   {[\tau/\alpha]B}}
            \;\theta\;(\gamma,\gamma'))}
          {Semantics}
    \eline{([\theta(\tau)], 
            \interpE{
            \judgeE{\Gamma,y:A,\Gamma'}
                   {e}
                   {[\tau/\alpha]B}}
            \;\theta\;(\gamma,\interpE{e'}\;\theta\;\gamma, \gamma'))}
          {IH}
    \eline{\interpE{
           \judgeE{\Gamma,y:A, \Gamma'}
                  {\pack{\tau}{e}}
                  {\exists \alpha:\kappa.\;B}}
           \;\theta\;(\gamma,\interpE{e'}\;\theta\;\gamma, \gamma')}
          {Semantics}
  \end{eqnproof}
\item case EUnpack: We have $\interpE{\judgeE{\Gamma,\Gamma'}{[e'/y](\unpack{\alpha}{x}{e_1}{e_2})}{B}}\;\theta\;(\gamma,\gamma')$ as

  \begin{eqnproof}
    \eline{\interpE{
           \judgeE{\Gamma,\Gamma'}
                  {\unpack{\alpha}{x}{[e'/y]e_1}{[e'/y]e_2}}
                  {B}}
             \;\theta\;(\gamma,\gamma')}
          {Substitution}
    \eline{(\semfun{\pair{\tau}{v}}
                   {(\interpE{\judgeE[\Theta, \alpha:\kappa]
                             {\Gamma,\Gamma', x:A}
                             {[e'/y]e_2}
                             {B}}
                     \;(\theta,\tau)
                     \;(\gamma, \gamma', v))})}
           {}
    \eclaim{\;\;(\interpE{
            \judgeE{\Gamma,\Gamma'}
                   {[e'/y]e_1}
                   {\exists \alpha:\kappa.\;A}}
            \;\theta\;(\gamma,\gamma'))}
          {Semantics}
    \eline{(\semfun{\pair{\tau}{v}}
                   {(\interpE{\judgeE[\Theta, \alpha:\kappa]
                             {\Gamma,\Gamma', x:A}
                             {e_2}
                             {B}}
                     \;(\theta,\tau)
                     \;(\gamma, \interpE{e'}\;(\theta,\tau)\;\gamma, \gamma', v))})}
           {}
    \eclaim{\;\;(\interpE{
            \judgeE{\Gamma,\Gamma'}
                   {e_1}
                   {\exists \alpha:\kappa.\;A}}
            \;\theta\;(\gamma,\interpE{e'}\;\theta\;\gamma, \gamma'))}
          {IH}
    \eline{(\semfun{\pair{\tau}{v}}
                   {(\interpE{\judgeE[\Theta, \alpha:\kappa]
                             {\Gamma,\Gamma', x:A}
                             {e_2}
                             {B}}
                     \;(\theta,\tau)
                     \;(\gamma, \interpE{e'}\;\theta\;\gamma, \gamma', v))})}
           {}
    \eclaim{\;\;(\interpE{
            \judgeE{\Gamma,\Gamma'}
                   {e_1}
                   {\exists \alpha:\kappa.\;A}}
            \;\theta\;(\gamma,\interpE{e'}\;\theta\;\gamma, \gamma'))}
          {$\alpha \not\in FTV(e')$}
    \eline{\interpE{
           \judgeE{\Gamma,\Gamma'}
                  {\unpack{\alpha}{x}{e_1}{e_2}}
                  {B}}
           \;\theta
           \;(\gamma,\interpE{e'}\;\theta\;\gamma, \gamma')}
          {Semantics}
  \end{eqnproof}

\item case EKeq: We have $\interpE{\judgeE{\Gamma,\Gamma'}{[e'/y]e}{B}}\;\theta\;(\gamma,\gamma')$
  \begin{eqnproof}
    \eline{\interpE{\judgeE{\Gamma,\Gamma'}{[e'/y]e}{C}}\;\theta\;(\gamma,\gamma')}
          {Semantics, $B = C$}
    \eline{\interpE{\judgeE{\Gamma,y:A,\Gamma'}{e}{C}}
           \;\theta
           \;(\gamma,\interpE{e'}\;\theta\;\gamma,\gamma')}
          {IH}
    \eline{\interpE{\judgeE{\Gamma,y:A,\Gamma'}{e}{B}}
           \;\theta
           \;(\gamma,\interpE{e'}\;\theta\;\gamma,\gamma')}
          {Semantics}
  \end{eqnproof}
\end{itemize}

Now, the cases for the computation terms follow.

\begin{itemize}
\item case CReturn: We have that $\interpC{\judgeC{\Gamma,\Gamma'}{[e'/y]e}{B}}\;\theta\;(\gamma,\gamma')$
  \begin{eqnproof}
    \eline{\eta_{\interp{B}\theta}(
             \interpE{\judgeE{\Gamma,\Gamma'}{[e'/y]e}{B}}
               \;\theta
               \;(\gamma,\gamma'))}
          {Semantics}
    \eline{\eta_{\interp{B}\theta}(
             \interpE{\judgeE{\Gamma,y:A,\Gamma'}{e}{B}}
               \;\theta
               \;(\gamma,\interpE{e'}\;\theta\;\gamma, \gamma'))}
          {Mutual IH}
    \eline{\eta_{\interp{B}\theta}(
             \interpE{\judgeE{\Gamma,y:A,\Gamma'}{e}{B}}
               \;\theta
               \;(\gamma,\interpE{e'}\;\theta\;\gamma, \gamma'))}
          {Semantics}
  \end{eqnproof}

\item case CLet: We have that $\interpC{\judgeC{\Gamma,\Gamma'}{[e'/y](\letv{x}{e}{c})}{C}} \;\theta\;(\gamma,\gamma')$

  \begin{eqnproof}
    \eline{\interpC{
           \judgeC{\Gamma,\Gamma'}{\letv{x}{[e'/y]e}{[e'/y]c}}{C}} 
           \;\theta\;(\gamma,\gamma')}
          {Substitution}
    \eline{(\semfun{v}{\interpC{\judgeC{\Gamma,\Gamma',x:B}{[e'/y]c}{C}}
                       \;\theta
                       \;(\gamma,\gamma',v)})^*}
          {}
    \eclaim{\;\;(\interpE{\judgeE{\Gamma,\Gamma'}{[e'/y]e}{\monad{B}}}\;
                \theta\;(\gamma,\gamma'))}
           {Semantics}
    \eline{(\semfun{v}{\interpC{\judgeC{\Gamma,y:A,\Gamma',x:B}{c}{C}}
                       \;\theta
                       \;(\gamma,\interpE{e'}\;\theta\;\gamma,\gamma',v)})^*}
          {}
    \eclaim{\;\;(\interpE{\judgeE{\Gamma,y:A,\Gamma'}{e}{\monad{B}}}\;
                \theta\;(\gamma,\interpE{e'}\;\theta\;\gamma,\gamma'))}
           {IH,IH}
    \eline{\interpC{
           \judgeC{\Gamma,y:A,\Gamma'}{\letv{x}{e}{c}}{C}} 
           \;\theta\;(\gamma,\interpE{e'}\;\theta\;\gamma,\gamma')}
          {Semantics}
  \end{eqnproof}

\item case CGet: We have that $\interpC{\judgeC{\Gamma,\Gamma'}
                                               {[e'/y](!e)}{B}}
                                       \;\theta\;(\gamma,\gamma')$

  \begin{eqnproof}
    \eline{\interpC{
           \judgeC{\Gamma,\Gamma'}
                  {!([e'/y]e)}{B}}
                  \;\theta\;(\gamma,\gamma')}
          {Substitution}
    \eline{\lambda k.\;\lambda (L, h).\; 
              \left\{ \begin{array}{ll}
                        k\; (h\;l)\; (L, h) & \mbox{when } l \in L \\
                        \top                & \mbox{otherwise} \\
                      \end{array}
              \right.}
          {Semantics}
    \eclaim
           {\mbox{where }l = \interpE{\judgeE{\Gamma,\Gamma'}
                                             {[e'/y]e}{\reftype{B}}}
                  \;\theta\;(\gamma,\gamma')}
           {}[1em]

    \eline{\interpC{
           \judgeC{\Gamma,y:A,\Gamma'}
                  {!e}{B}}
                  \;\theta\;(\gamma,\interpE{e'}\;\theta\;\gamma, \gamma')}
          {}
    \eclaim[]
          {\mbox{because }l = \interpE{\judgeE{\Gamma,y:A,\Gamma'}
                                              {e}{\reftype{B}}}
                              \;\theta\;(\gamma,
                                         \interpE{e'}\;\theta\;\gamma, 
                                         \gamma')}
          {IH}
  \end{eqnproof}

\item case CSet: We have $\interpC{\judgeC{\Gamma,\Gamma'}
                                          {[e'/y](e_1 := e_2)}{\unittype}}
                                  \;\theta\;(\gamma,\gamma')$
  \begin{eqnproof}
    \eline{\interpC{\judgeC{\Gamma,\Gamma'}
                                          {[e'/y]e_1 := [e'/y]e_2}{\unittype}}
                                  \;\theta\;(\gamma,\gamma')}
          {Substitution}
    \eline{\lambda k.\;\lambda (L, h).\; 
              \left\{ \begin{array}{ll}
                       k \unit (L, [h|l:v]) & \mbox{when } l \in L \\
                      \top                & \mbox{otherwise} \\
              \end{array}
              \right.}
          {Semantics}
    \eclaim{\mbox{where }l = \interpE{\judgeE{\Gamma,\Gamma;}{[e'/y]e_1}{\reftype{B}}}
                               \;\theta\;(\gamma,\gamma')}
           {}
    \eclaim{\mbox{where }v = \interpE{\judgeE{\Gamma,\Gamma;}{[e'/y]e_2}{B}}
                               \;\theta\;(\gamma,\gamma')}
           {}
           [1em]
    \eline{\interpC{\judgeC{\Gamma,y:A,\Gamma'}{e_1 := e_2}{\unittype}}
                   \;\theta
                   \;(\gamma,\interpE{e'}\;\theta\;\gamma,\gamma')}
          {}
    \eclaim{\mbox{because }l = \interpE{\judgeE{\Gamma,y:A,\Gamma}{e_1}{\reftype{B}}}
                               \;\theta\;(\gamma,\interpE{e'}\;\theta\;\gamma,\gamma')}
           {IH}
    \eclaim{\mbox{because }v = \interpE{\judgeE{\Gamma,y:A,\Gamma}{e_2}{B}}
                               \;\theta\;(\gamma,\interpE{e'}\;\theta\;\gamma,\gamma')}
           {IH}
  \end{eqnproof}

\item case CNew: We have that $\interpC{\judgeC{\Gamma,\Gamma'}{[e'/y]\newref{B}{e}}
                                               {\reftype{B}}}
                                       \;\theta\;(\gamma,\gamma')$.
  \begin{eqnproof}
    \eline{\interpC{\judgeC{\Gamma,\Gamma'}{\newref{B}{([e'/y]e)}}{\reftype{B}}}
            \;\theta
            \;(\gamma,\gamma')}
          {Substitution}
    \eline{\lambda k.\;\lambda (L, h).\;\left(
             \begin{array}{l}
               \mbox{let }l = (max(L) + 1, A) \mbox{ in} \\
               k\;l\;(L \cup \setof{l}, [h|l:v]) \\
             \end{array}\right)}
          {}
    \eclaim{\mbox{where }v = \interpE{\judgeE{\Gamma,\Gamma'}
                                             {[e'/y]e}{A}}\;\theta\;(\gamma,\gamma')}
           {Semantics}
    \eline{\interpC{\judgeC{\Gamma,y:A,\Gamma'}{\newref{B}{e}}{\reftype{B}}}
            \;\theta
            \;(\gamma,\interpE{e'}\;\theta\;\gamma,\gamma')}
          {}
    \eclaim{\mbox{because }v = \interpE{\judgeE{\Gamma,\Gamma'}
                                             {e}{A}}\;\theta\;(\gamma,\interpE{e'}\;\theta\;\gamma,\gamma')}
           {IH}
  \end{eqnproof}

    
\end{itemize}
\end{proof}



\begin{lemma}{(Soundness of Equality Rules)}
We have that:
\begin{enumerate}
\item If $\judgeEq{\Gamma}{e}{e'}{A}$, then $\judgeE{\Gamma}{e}{A}$ and 
$\judgeE{\Gamma}{e'}{A}$ and 
$\interpE{\judgeE{\Gamma}{e}{A}} = \interpE{\judgeE{\Gamma}{e'}{A}}$.

\item If $\judgeEqC{\Gamma}{c}{c'}{A}$, then $\judgeC{\Gamma}{c}{A}$ and 
$\judgeC{\Gamma}{c'}{A}$ and 
$\interpC{\judgeE{\Gamma}{c}{A}} = \interpC{\judgeC{\Gamma}{c'}{A}}$.
\end{enumerate}
\end{lemma}


\begin{proof}
The proof of this theorem is by induction on the derivations of $\judgeEq{\Gamma}{e}{e'}{A}$
and $\judgeEqC{\Gamma}{c}{c'}{A}$. So, assuming we have suitable $\theta$ and $\gamma$, we
proceed as follows:

\begin{itemize}
\item case EqUnit: From $\judgeEq{\Gamma}{e}{e'}{\unittype}$, we have

  \begin{eqnproof}
    \eclaim{\judgeE{\Gamma}{e}{\unittype}}{By inversion}
    \eclaim{\judgeE{\Gamma}{e'}{\unittype}}{By inversion}
  \end{eqnproof}

  \begin{eqnproof}
    \eline[\interpE{\judgeE{\Gamma}{e}{\unittype}}\;\theta\;\gamma]
          {(*)}
          {Semantics}
    \eline{\interpE{\judgeE{\Gamma}{e'}{\unittype}}\;\theta\;\gamma}
          {Semantics}
  \end{eqnproof}

\item case EqPairFst: From $\judgeEq{\Gamma}{\fst{\pair{e_1}{e_2}}}{e_1}{A_1}$, we have:

  \begin{eqnproof}
    \eclaim[1]{\judgeEq{\Gamma}{\fst{\pair{e_1}{e_2}}}{e_1}{A_1}}
              {Hypothesis}
    \eclaim[2]{\judgeE{\Gamma}{\pair{e_1}{e_2}}{A_1\times A_2}}
           {By inversion on 1}
    \eclaim[3]{\judgeE{\Gamma}{e_1}{A_1}}
              {By inversion on 2}
    \eclaim[4]{\judgeE{\Gamma}{\fst{\pair{e_1}{e_2}}}{A_1}}
              {By rule EFst on 2}
  \end{eqnproof}

  \begin{eqnproof}[\interpE{\judgeE{\Gamma}{\fst{\pair{e_1}{e_2}}}{A_1}}\;\theta\;\gamma =]
    \eline{\pi_1(\interpE{\judgeE{\Gamma}{\pair{e_1}{e_2}}{A_1\times A_2}}\;\theta\;\gamma)}
          {Semantics}
    \eline{\pi_1(\sempair{\interpE{\judgeE{\Gamma}{e_1}{A_1}}\;\theta\;\gamma}
                         {\interpE{\judgeE{\Gamma}{e_1}{A_2}}\;\theta\;\gamma})}
          {Semantics}
    \eline{\interpE{\judgeE{\Gamma}{e_1}{A_1}}\;\theta\;\gamma}
          {Products}
  \end{eqnproof}

\item case EqPairSnd: From $\judgeEq{\Gamma}{\snd{\pair{e_1}{e_2}}}{e_1}{A_2}$, we have:

  \begin{eqnproof}
    \eclaim[1]{\judgeEq{\Gamma}{\snd{\pair{e_1}{e_2}}}{e_1}{A_2}}
              {Hypothesis}
    \eclaim[2]{\judgeE{\Gamma}{\pair{e_1}{e_2}}{A_1\times A_2}}
           {By inversion on 1}
    \eclaim[3]{\judgeE{\Gamma}{e_2}{A_2}}
              {By inversion on 2}
    \eclaim[4]{\judgeE{\Gamma}{\snd{\pair{e_1}{e_2}}}{A_2}}
              {By rule EFst on 2}
  \end{eqnproof}

  \begin{eqnproof}[\interpE{\judgeE{\Gamma}{\snd{\pair{e_1}{e_2}}}{A_2}}\;\theta\;\gamma =]
    \eline{\pi_2(\interpE{\judgeE{\Gamma}{\pair{e_1}{e_2}}{A_1\times A_2}}\;\theta\;\gamma)}
          {Semantics}
    \eline{\pi_2(\sempair{\interpE{\judgeE{\Gamma}{e_1}{A_1}}\;\theta\;\gamma}
                         {\interpE{\judgeE{\Gamma}{e_1}{A_2}}\;\theta\;\gamma})}
          {Semantics}
    \eline{\interpE{\judgeE{\Gamma}{e_2}{A_2}}\;\theta\;\gamma}
          {Products}
  \end{eqnproof}

\item case EqPairEta: 
  \begin{eqnproof}
     \eclaim[1]{\judgeEq{\Gamma}{e}{\pair{\fst{e}}{\snd{e}}}{A_1\times A_2}}
               {Hypothesis}
     \eclaim[2]{\judgeE{\Gamma}{e}{A_1\times A_2}}
               {Inversion}
     \eclaim[3]{\judgeE{\Gamma}{\fst{e}}{A_1}}
               {Rule EFst on 2}
     \eclaim[4]{\judgeE{\Gamma}{\snd{e}}{A_2}}
               {Rule ESnd on 2}
     \eclaim[5]{\judgeE{\Gamma}{\pair{\fst{e}}{\snd{e}}}{A_1\times A_2}}
               {Rule EPair on 3, 4}
  \end{eqnproof}

  \begin{eqnproof}[\interpE{\judgeE{\Gamma}{e}{A_1\times A_2}}\;\theta\;\gamma =]
     \eline{\sempair{\pi_1(\interpE{\judgeE{\Gamma}{e}{A_1\times A_2}}\;\theta\;\gamma)}
                     {\pi_2(\interpE{\judgeE{\Gamma}{e}{A_1\times A_2}}\;\theta\;\gamma)}}
            {Products}
     \eline{\sempair{\interpE{\judgeE{\Gamma}{\fst{e}}{A_1}}\;\theta\;\gamma}
                     {\interpE{\judgeE{\Gamma}{\snd{e}}{A_2}}\;\theta\;\gamma}}
            {Semantics $\times$ 2}
     \eline{\interpE{\judgeE{\Gamma}{\pair{\fst{e}}{\snd{e}}}{A_1\times A_2}}\;\theta\;\gamma}
            {Semantics}    
  \end{eqnproof}

\item case EqFunBeta:

  \begin{eqnproof}
    \eclaim[1]{\judgeEq{\Gamma}{(\fun{x}{A}{e})\;e'}{[e'/x]e}{B}}
              {Hypothesis}
    \eclaim[2]{\judgeE{\Gamma}{(\fun{x}{A}{e})\;e'}{B}}
              {Inversion on 1}
    \eclaim[3]{\judgeE{\Gamma}{e'}{A}}
              {Inversion on 2}
    \eclaim[4]{\judgeE{\Gamma}{\fun{x}{A}{e}}{A \to B}}
              {Inversion on 2}
    \eclaim[5]{\judgeE{\Gamma,x:A}{e}{B}}
              {Inversion on 4}
    \eclaim[6]{\judgeE{\Gamma}{[e'/x]e}{B}}
              {Substitution 3 into 5}
  \end{eqnproof}

  \begin{eqnproof}[\interpE{\judgeE{\Gamma}{(\fun{x}{A}{e})\;e'}{B}}\;\theta\;\gamma = ]
    \eline{(\interpE{\judgeE{\Gamma}{\fun{x}{A}{e}}{A \to B}}\;\theta\;\gamma)\;
           (\interpE{\judgeE{\Gamma}{e'}{B}}\;\theta\;\gamma)}
          {Semantics}
    \eline{(\semfun{v}{(\interpE{\judgeE{\Gamma,x:A}{e}{B}}\;\theta\;(\gamma,v))}\;
           \interpE{\judgeE{\Gamma}{e'}{B}\;\theta\;\gamma}}
          {Semantics}
    \eline{\interpE{\judgeE{\Gamma,x:A}{e}{B}}\;\theta\;
             (\gamma,\interpE{\judgeE{\Gamma}{e'}{B}}\;\theta\;\gamma)}
          {Functions}
    \eline{\interpE{\judgeE{\Gamma}{[e'/x]e}{B}}\;\theta\;\gamma}
          {Substitution}
  \end{eqnproof}

\item case EqFunEta: 

  \begin{eqnproof}
    \eclaim[1]{\judgeEq{\Gamma}{e}{e'}{A \to B}}
           {Hypothesis}
    \eclaim[2]{\judgeEq{\Gamma, x:A}{e\;x}{e'\;x}{B}}
              {Inversion}
    \eclaim[3]{\judgeE{\Gamma,x:A}{e}{B}}
              {Induction}
    \eclaim[4]{\judgeE{\Gamma,x:A}{e'}{B}}
              {Induction}
    \eclaim[5]{\judgeE{\Gamma}{e}{B}}
              {since $x\not \in FV(e)$}
    \eclaim[6]{\judgeE{\Gamma}{e'}{B}}
              {since $x\not \in FV(e')$}
  \end{eqnproof}
  \begin{eqnproof}[\mbox{For arbitrary }v,]
    \eline[\interpE{\judgeE{\Gamma,x:A}{e'}{B}}\;\theta\;(\gamma,v)]
          {\interpE{\judgeE{\Gamma,x:A}{e}{B}}\;\theta\;(\gamma,v)}
          {Induction}
    \eline[\interpE{\judgeE{\Gamma,x:A}{e}{B}}\;\theta\;(\gamma,v)]
          {\interpE{\judgeE{\Gamma}{e}{B}}\;\theta\;\gamma}
          {since $x\not \in FV(e)$}
    \eline[\interpE{\judgeE{\Gamma,x:A}{e'}{B}}\;\theta\;(\gamma,v)]
          {\interpE{\judgeE{\Gamma}{e'}{B}}\;\theta\;\gamma}
          {since $x\not \in FV(e')$}
    \eline[\interpE{\judgeE{\Gamma}{e'}{B}}\;\theta\;\gamma]
          {\interpE{\judgeE{\Gamma}{e}{B}}\;\theta\;\gamma}
          {Transitivity}
  \end{eqnproof}

\item case EqSumInlBeta
  \begin{eqnproof}
    \eclaim[1]{\judgeEq{\Gamma}{\Case{\inl{e}}{x}{e_1}{y}{e_2}}{[e/x]e_1}{C}}
              {Hypothesis}
    \eclaim[2]{\judgeE{\Gamma}{\Case{\inl{e}}{x}{e_1}{y}{e_2}}{C}}
              {Inversion on 1}
    \eclaim[3]{\judgeE{\Gamma}{\inl{e}}{A + B}}
              {Inversion on 2}
    \eclaim[4]{\judgeE{\Gamma, x:A}{e_1}{C}}
              {Inversion on 2}
    \eclaim[5]{\judgeE{\Gamma, y:B}{e_2}{C}}
              {Inversion on 2}
    \eclaim[6]{\judgeE{\Gamma}{e}{A}}
              {Inversion on 3}
    \eclaim[7]{\judgeE{\Gamma}{[e/x]e_1}{C}}
              {Substitute 6 into 4}
  \end{eqnproof}
  \begin{eqnproof}[\interpE{\judgeE{\Gamma}{\Case{\inl{e}}{x}{e_1}{y}{e_2}}{C}}\;\theta\;\gamma =]
    \eline{[f_1,f_2](a)}
          {Semantics}
    \eclaim[\mbox{where}]
           {\begin{array}{lcl}
               a & = & \interpE{\judgeE{\Gamma}{\inl{e}}{A + B}}\;\theta\;\gamma \\
                 & = & \iota_1(\interpE{\judgeE{\Gamma}{e}{A}}\;\theta\;\gamma) \\
               f_1 & = & \semfun{v}{\interpE{\judgeE{\Gamma,x:A}{e_1}{C}}\;\theta\;(\gamma,v)} \\
               f_2 & = & \semfun{v}{\interpE{\judgeE{\Gamma,y:B}{e_2}{C}}\;\theta\;(\gamma,v)} \\
            \end{array}}
           {}[2em]
    \eline[{[f_1,f_2](a)}]
          {[f_1,f_2](\iota_1(\interpE{\judgeE{\Gamma}{e}{A}}\;\theta\;\gamma))}
          {Sums}
    \eline{f_1(\interpE{\judgeE{\Gamma}{e}{A}}\;\theta\;\gamma)}
          {}
    \eline{\interpE{\judgeE{\Gamma,x:A}{e_1}{C}}\;\theta\;(\gamma,\interpE{\judgeE{\Gamma}{e}{A}})}
          {Def of $f_1$}
    \eline{\interpE{\judgeE{\Gamma}{[e/x]e_1}{C}}\;\theta\;\gamma}
          {Substitutition}
  \end{eqnproof}

\item case EqSumInrBeta:
  \begin{eqnproof}
    \eclaim[1]{\judgeEq{\Gamma}{\Case{\inr{e}}{x}{e_1}{y}{e_2}}{[e/y]e_2}{C}}
              {Hypothesis}
    \eclaim[2]{\judgeE{\Gamma}{\Case{\inr{e}}{x}{e_1}{y}{e_2}}{C}}
              {Inversion on 1}
    \eclaim[3]{\judgeE{\Gamma}{\inr{e}}{A + B}}
              {Inversion on 2}
    \eclaim[4]{\judgeE{\Gamma, x:A}{e_1}{C}}
              {Inversion on 2}
    \eclaim[5]{\judgeE{\Gamma, y:B}{e_2}{C}}
              {Inversion on 2}
    \eclaim[6]{\judgeE{\Gamma}{e}{B}}
              {Inversion on 3}
    \eclaim[7]{\judgeE{\Gamma}{[e/y]e_2}{C}}
              {Substitute 6 into 5}
  \end{eqnproof}
  \begin{eqnproof}[\interpE{\judgeE{\Gamma}{\Case{\inr{e}}{x}{e_1}{y}{e_2}}{C}}\;\theta\;\gamma =]
    \eline{[f_1,f_2](a)}
          {Semantics}
    \eclaim[\mbox{where}]
           {\begin{array}{lcl}
               a & = & \interpE{\judgeE{\Gamma}{\inr{e}}{A + B}}\;\theta\;\gamma \\
                 & = & \iota_2(\interpE{\judgeE{\Gamma}{e}{B}}\;\theta\;\gamma) \\
               f_1 & = & \semfun{v}{\interpE{\judgeE{\Gamma,x:A}{e_1}{C}}\;\theta\;(\gamma,v)} \\
               f_2 & = & \semfun{v}{\interpE{\judgeE{\Gamma,y:B}{e_2}{C}}\;\theta\;(\gamma,v)} \\
            \end{array}}
           {}[2em]
    \eline[{[f_1,f_2](a)}]
          {[f_1,f_2](\iota_2(\interpE{\judgeE{\Gamma}{e}{B}}\;\theta\;\gamma))}
          {Sums}
    \eline{f_2(\interpE{\judgeE{\Gamma}{e}{B}}\;\theta\;\gamma)}
          {}
    \eline{\interpE{\judgeE{\Gamma,y:B}{e_2}{C}}\;\theta\;(\gamma,\interpE{\judgeE{\Gamma}{e}{B}})}
          {Def of $f_1$}
    \eline{\interpE{\judgeE{\Gamma}{[e/y]e_2}{C}}\;\theta\;\gamma}
          {Substitution}
  \end{eqnproof}

\item case EqSumEta:
  \begin{eqnproof}
    \eclaim[1]{\judgeEq{\Gamma}{\Case{e}{x}{[\inl{x}/z]e'}{y}{[\inr{y}/z]e'}}{[e/z]e'}{C}}
              {Hypothesis}
    \eclaim[2]{\judgeE{\Gamma}{e}{A+B}}
              {Inversion on 1}
    \eclaim[3]{\judgeE{\Gamma, z:A+B}{e'}{C}}
              {Inversion on 1}
    \eclaim[4]{\judgeE{\Gamma, x:A, z:A+B}{e'}{C}}
              {Weakening on 3}
    \eclaim[5]{\judgeE{\Gamma, x:A}{\inl{x}}{A+B}}
              {By rules}
    \eclaim[6]{\judgeE{\Gamma, x:A}{[\inl{x}/z]e'}{C}}
              {Substitution of 5 into 4}
    \eclaim[7]{\judgeE{\Gamma, y:B, z:A+B}{e'}{C}}
              {Weakening on 3}
    \eclaim[8]{\judgeE{\Gamma, y:B}{\inr{y}}{A+B}}
              {By rules}
    \eclaim[9]{\judgeE{\Gamma, y:B}{[\inr{y}/z]e'}{C}}
              {Substitution of 8 into 7}
    \eclaim[10]{\judgeE{\Gamma}{\Case{e}{x}{[\inl{x}/z]e'}{y}{[\inr{y}/z]e'}}{C}}
               {By ECase on 2, 6, 9}
    \eclaim[11]{\judgeE{\Gamma}{[e/z]e'}{C}}
               {Substitution of 2 into 3}

  \end{eqnproof}

Now from the semantics, we know that $\interpE{\judgeE{\Gamma}{e}{A+B}}\;\theta\;\gamma$ is
either equal to some $\iota_i(v_A)$ or some $\iota_2(v_B)$. 

Suppose it is equal $\iota_1(v_A)$. Then,  $\interpE{\judgeE{\Gamma}{\Case{e}{x}{[\inl{x}/z]e'}{y}{[\inr{y}/z]e'}}{C}}\;\theta\;\gamma$ is equal to 
\begin{eqnproof}
  \eline{\left[
           \begin{array}{l}
            \semfun{v}{\interpE{\judgeE{\Gamma,x:A}{[\inl{x}/z]e'}{C}}\;\theta\;(\gamma,v)}, \\
            \semfun{v}{\interpE{\judgeE{\Gamma,y:B}{[\inr{y}/z]e'}{C}}\;\theta\;(\gamma,v)}  \\
           \end{array}\right]
         (\iota_1(v_A))}
        {Semantics}
  \eline{\interpE{\judgeE{\Gamma,x:A}{[\inl{x}/z]e'}{C}}\;\theta\;(\gamma,v_A)}
        {Sums}  
  \eline{\begin{array}{l}
           \interpE{\judgeE{\Gamma,x:A,z:A+B}{e'}{C}}\;\theta \\
           \qquad (\gamma, v_A, \interpE{\judgeE{\Gamma,x:A}{\inl{x}}{A+B}}\;\theta\;(\gamma, v_A)) \\
         \end{array}}
        {Substitution}
  \eline{\interpE{\judgeE{\Gamma,x:A,z:A+B}{e'}{C}}\;\theta\;(\gamma, v_A, \iota_1(v_A))}
        {Semantics}
  \eline{\interpE{\judgeE{\Gamma, z:A+B}{e'}{C}}\;\theta\;(\gamma, \iota_1(v_A))}
        {Since $x \not \in FV(e')$}
  \eline{\interpE{\judgeE{\Gamma, z:A+B}{e'}{C}}\;\theta\;(\gamma, \interpE{\judgeE{\Gamma}{e}{A+B}}\;\theta\;\gamma)}
        {Meaning of $\iota_1(v_A)$}
  \eline{\interpE{\judgeE{\Gamma}{[e/z]e'}{C}}\;\theta\;\gamma}
        {Substitution}
\end{eqnproof}

Suppose it is $\iota_2(v_B)$. Then,
$\interpE{\judgeE{\Gamma}{\Case{e}{x}{[\inl{x}/z]e'}{y}{[\inr{y}/z]e'}}{C}}\;\theta\;\gamma$
is equal to
\begin{eqnproof}
  \eline{\left[
           \begin{array}{l}
            \semfun{v}{\interpE{\judgeE{\Gamma,x:A}{[\inl{x}/z]e'}{C}}\;\theta\;(\gamma,v)}, \\
            \semfun{v}{\interpE{\judgeE{\Gamma,y:B}{[\inr{y}/z]e'}{C}}\;\theta\;(\gamma,v)}  \\
           \end{array}\right]
         (\iota_2(v_B))}
        {Semantics}
  \eline{\interpE{\judgeE{\Gamma,y:B}{[\inr{y}/z]e'}{C}}\;\theta\;(\gamma,v_B)}
        {Sums}  
  \eline{\begin{array}{l}
           \interpE{\judgeE{\Gamma,y:B,z:A+B}{e'}{C}}\;\theta \\
           \qquad (\gamma, v_B, \interpE{\judgeE{\Gamma,y:B}{\inr{y}}{A+B}}\;\theta\;(\gamma, v_B)) \\
         \end{array}}
        {Substitution}
  \eline{\interpE{\judgeE{\Gamma,y:B,z:A+B}{e'}{C}}\;\theta\;(\gamma, v_B, \iota_2(v_B))}
        {Semantics}
  \eline{\interpE{\judgeE{\Gamma, z:A+B}{e'}{C}}\;\theta\;(\gamma, \iota_2(v_B))}
        {Since $x \not \in FV(e')$}
  \eline{\interpE{\judgeE{\Gamma, z:A+B}{e'}{C}}\;\theta\;(\gamma, \interpE{\judgeE{\Gamma}{e}{A+B}}\;\theta\;\gamma)}
        {Meaning of $\iota_2(v_B)$}
  \eline{\interpE{\judgeE{\Gamma}{[e/z]e'}{C}}\;\theta\;\gamma}
        {Substitution}
\end{eqnproof}

\item case EqMonad:

  \begin{eqnproof}
    \eclaim[1]{\judgeEq{\Gamma}{\comp{c}}{\comp{c'}}{\monad{A}}}
             {Hypothesis}
    \eclaim[2]{\judgeEqC{\Gamma}{c}{c'}{A}}
             {Inversion on 1}
    \eclaim[3]{\judgeC{\Gamma}{c}{A}}
             {Mutual Induction on 2}
    \eclaim[4]{\judgeC{\Gamma}{c'}{A}}
             {Induction on 2}
    \eclaim[5]{\judgeE{\Gamma}{\comp{c}}{\monad{A}}}
             {By rule EMonad on 3}
    \eclaim[6]{\judgeE{\Gamma}{\comp{c'}}{\monad{A}}}
             {By rule EMonad on 4}
  \end{eqnproof}

  \begin{eqnproof}
    \eline[\interpE{\judgeE{\Gamma}{\comp{c}}{\monad{A}}}\;\theta\;\gamma]
          {\interpC{\judgeC{\Gamma}{c}{A}}\;\theta\;\gamma}
          {Semantics}
    \eline{\interpC{\judgeC{\Gamma}{c'}{A}}\;\theta\;\gamma}
          {Mutual Induction}
    \eline{\interpE{\judgeE{\Gamma}{\comp{c'}}{\monad{A}}}\;\theta\;\gamma}
          {Semantics}
  \end{eqnproof}

\item case EqFix
  \begin{eqnproof}
    \eclaim[1]{\judgeEq{\Gamma}{\fix{x:D}{e}}{[(\fix{x:D}{e})/x]e}{D}}
              {Hypothesis}
    \eclaim[2]{\judgeE{\Gamma}{\fix{x:D}{e}}{D}}
              {Inversion on 1}
    \eclaim[3]{\judgeE{\Gamma, x:D}{e}{D}}
              {Inversion on 2}
    \eclaim[4]{\judgeE{\Gamma}{[(\fix{x:D}{e})/x]e}{D}}
              {Substitution of 2 into 3}
  \end{eqnproof}
  \begin{eqnproof}[\interpE{\judgeE{\Gamma}{\fix{x:D}{e}}{D}}\;\theta\;\gamma]
    \eline{fix(\semfun{v}{(\interpE{\judgeE{\Gamma, x:D}{e}{D}}\;\theta\;(\gamma,v))})}
          {Semantics}
    \eline{\interpE{\judgeE{\Gamma, x:D}{e}{D}}\;\theta\;(\gamma,
           fix(\semfun{v}{(\interpE{\judgeE{\Gamma, x:D}{e}{D}}\;\theta\;(\gamma,v))}))}
          {Unroll $fix$}
    \eline{\interpE{\judgeE{\Gamma, x:D}{e}{D}}\;\theta\;(\gamma,
            \interpE{\judgeE{\Gamma}{\fix{x:D}{e}}{D}}\;\theta\;\gamma)}
          {Definition}
    \eline{\interpE{\judgeE{\Gamma}{[(\fix{x:D}{e})/x]e}{D}}\;\theta\;\gamma}
          {Substitution}
  \end{eqnproof}

\item case EqNatZBeta:
  \begin{eqnproof}
    \eclaim[1]{\judgeEq{\Gamma}{\iter{\z}{e_0}{x}{e_1}}{e_0}{A}}
              {Hypothesis}
    \eclaim[2]{\judgeE{\Gamma}{\iter{\z}{e_0}{x}{e_1}}{A}}
              {Inversion on 1}
    \eclaim[3]{\judgeE{\Gamma}{\z}{\N}}
              {Inversion on 2}
    \eclaim[4]{\judgeE{\Gamma,x:A}{e_1}{A}}
              {Inversion on 2}
    \eclaim[5]{\judgeE{\Gamma}{e_0}{A}}
              {Inversion on 2}
  \end{eqnproof}
  \begin{eqnproof}[\interpE{\judgeE{\Gamma}{\iter{\z}{e_0}{x}{e_1}}{A}}\;\theta\;\gamma =]
    \eline{\begin{array}{l}
             iter[\interpE{\judgeE{\Gamma}{e_0}{A}}\;\theta\;\gamma,
                \semfun{v}{\interpE{\judgeE{\Gamma,x:A}{e_1}{A}}\;\theta\;(\gamma,v)}] \\
            (\interpE{\judgeE{\Gamma}{\z}{\N}}\;\theta\;\gamma) \\
           \end{array}}
          {Semantics}
    \eline{iter[\interpE{\judgeE{\Gamma}{e_0}{A}}\;\theta\;\gamma,
                \semfun{v}{\interpE{\judgeE{\Gamma,x:A}{e_1}{A}}\;\theta\;(\gamma,v)}]
            (z)}
          {Semantics}
    \eline{\interpE{\judgeE{\Gamma}{e_0}{A}}\;\theta\;\gamma}
          {Iter properties}
  \end{eqnproof}

\item EqNatSBeta
  \begin{eqnproof}
    \eclaim[1]{\judgeEq{\Gamma}{\iter{\s{e}}{e_0}{x}{e_1}}{[\iter{e}{e_0}{x}{e_1}/x]e_1}{A}}
              {Hypothesis}
    \eclaim[2]{\judgeE{\Gamma}{\iter{\s{e}}{e_0}{x}{e_1}}{A}}
              {Inversion on 1}
    \eclaim[3]{\judgeE{\Gamma}{\s{e}}{\N}}
              {Inversion on 2}
    \eclaim[4]{\judgeE{\Gamma}{e_0}{A}}
              {Inversion on 2}
    \eclaim[5]{\judgeE{\Gamma,x:A}{e_1}{A}}
              {Inversion on 2}
    \eclaim[6]{\judgeE{\Gamma}{e}{\N}}
              {Inversion on 3}
    \eclaim[7]{\judgeE{\Gamma}{\iter{e}{e_0}{x}{e_1}}{A}}
              {Rule EIter on 6, 4, 5}
  \end{eqnproof}
  \begin{eqnproof}[\interpE{\judgeE{\Gamma}{\iter{\s{e}}{e_0}{x}{e_1}}{A}}\;\theta\;\gamma =]
    \eline{\begin{array}{l}
              iter[\interpE{\judgeE{\Gamma}{e_0}{A}}\;\theta\;\gamma,
                   \semfun{v}{\interpE{\judgeE{\Gamma,x:A}{e_1}{A}}\;\theta\;(\gamma,v)}] \\
              \;\;(\interpE{\judgeE{\Gamma}{\s{e}}{\N}}\;\theta\;\gamma)
           \end{array}}
          {Semantics}
    \eline{\begin{array}{l}
              iter[\interpE{\judgeE{\Gamma}{e_0}{A}}\;\theta\;\gamma,
                   \semfun{v}{\interpE{\judgeE{\Gamma,x:A}{e_1}{A}}\;\theta\;(\gamma,v)}] \\
              \;\;(s(\interpE{\judgeE{\Gamma}{e}{\N}}\;\theta\;\gamma))
           \end{array}}
          {Semantics}
    \eline{\begin{array}{l}
             \interpE{\judgeE{\Gamma,x:A}{e_1}{A}}\;\theta \\
             \left(\gamma,
              \begin{array}{l}
                iter[\interpE{\judgeE{\Gamma}{e_0}{A}}\;\theta\;\gamma,
                     \semfun{v}{\interpE{\judgeE{\Gamma,x:A}{e_1}{A}}\;\theta\;(\gamma,v)}] \\
                  \;\;(\interpE{\judgeE{\Gamma}{e}{\N}}\;\theta\;\gamma) \\
              \end{array}\right) \\
           \end{array}}
          {Iter}
     \eline{\interpE{\judgeE{\Gamma,x:A}{e_1}{A}}\;\theta
            (\gamma, \interpE{\judgeE{\Gamma}{\iter{e}{e_0}{x}{e_1}}{A}}\;\theta\;\gamma)}
           {Semantics}
     \eline{\interpE{\judgeE{\Gamma}{[\iter{e}{e_0}{x}{e_1}/x]e_1}{A}}\;\theta\;\gamma}
           {Substitution}
  \end{eqnproof}

\item case EqNatEta:
  \begin{eqnproof}
    \eclaim[1]{\judgeEq{\Gamma, n:\N}{\iter{n}{e_0}{x}{e_1}}{e}{A}}
              {Hypothesis}
    \eclaim[2]{\judgeE{\Gamma}{e}{A}}
              {Inversion on 1}
    \eclaim[3]{\judgeE{\Gamma}{e_0}{A}}
              {Inversion on 1}
    \eclaim[4]{\judgeE{\Gamma,n:\N}{e_0}{A}}
              {Weakening on 3}
    \eclaim[5]{\judgeE{\Gamma,x:A}{e_1}{A}}
              {Inversion on 1}
    \eclaim[6]{\judgeE{\Gamma,n:\N,x:A}{e_1}{A}}
              {Weakening on 5}
    \eclaim[7]{\judgeE{\Gamma,n:\N}{n}{\N}}
              {Rule Hyp}
    \eclaim[8]{\judgeE{\Gamma,n:\N}{\iter{n}{e_0}{x}{e_1}}{A}}
              {Rule EIter on 7, 4, 6}
  \end{eqnproof}

Now, assume we have some suitable environment $(\gamma, v)$. So $v$ is a natural
number, and we shall proceed by induction on it. 

\begin{itemize}
  \item case $v = 0$. 

    \begin{eqnproof}[\interpE{\judgeE{\Gamma,n:\N}{e}{A}}\;\theta\;(\gamma,0) = ]
      \eline{\interpE{\judgeE{\Gamma,n:\N}{e}{A}}\;\theta\;(\gamma, \interpE{\judgeE{\Gamma}{\z}{\N}}\;\theta\;\gamma)}
            {Semantics}
      \eline{\interpE{\judgeE{\Gamma}{[\z/n]e}{A}}\;\theta\;\gamma}
            {Substitution}
      \eline{\interpE{\judgeE{\Gamma}{e}{A}}\;\theta\;\gamma}
            {Induction Hypothesis}
      \eline{\interpE{\judgeE{\Gamma,n:\N}{e}{A}}\;\theta\;(\gamma, 0)}
            {Weakening}
    \end{eqnproof}

\item case $v = s(k)$
  By induction, we know \\
$\interpE{\judgeE{\Gamma, n:\N}{e}{A}}\;\theta\;(\gamma, k)$ $=$ $\interpE{\judgeE{\Gamma,n:\N}{\iter{n}{e_0}{x}{e_1}}{A}}\;\theta\;(\gamma, k)$

\ \\

  \begin{eqnproof}[\interpE{\judgeE{\Gamma,n:\N}{\iter{n}{e_0}{x}{e_1}}{A}}\;\theta\;(\gamma,s(k)) =]
    \eline{iter\left[
            \begin{array}{l}
              \interpE{\judgeE{\Gamma, n:\N}{e_0}{A}}\;\theta\;(\gamma,s(k)), \\
              \semfun{v}{\interpE{\judgeE{\Gamma,n:\N,x:A}{e_1}{A}}\;\theta\;(\gamma,s(k),v)} \\
            \end{array}\right](s(k))}
          {Semantics}
    \eline{iter\left[
            \begin{array}{l}
              \interpE{\judgeE{\Gamma}{e_0}{A}}\;\theta\;(\gamma), \\
              \semfun{v}{\interpE{\judgeE{\Gamma,x:A}{e_1}{A}}\;\theta\;(\gamma,v)} \\
            \end{array}\right](s(k))}
          {since $x \not \in FV(e_0), FV(e_1)$}
    \eline{iter\left[
            \begin{array}{l}
              \interpE{\judgeE{\Gamma, m:\N}{e_0}{A}}\;\theta\;(\gamma,k), \\
              \semfun{v}{\interpE{\judgeE{\Gamma,m:\N,x:A}{e_1}{A}}\;\theta\;(\gamma,k,v)} \\
            \end{array}\right](s(k))}
          {By weakening}
    \eline{\interpE{\judgeE{\Gamma,n:\N,x:A}{e_1}{A}}\;\theta\;(\gamma, iter[\ldots](k))}
          {By $iter$}
    \eline{\begin{array}{l}
             \interpE{\judgeE{\Gamma,n:\N,x:A}{e_1}{A}}\;\theta \\
             \;\;(\gamma, k,
               \interpE{\judgeE{\Gamma,n:\N}{\iter{n}{e_0}{x}{e_1}}{A}}\;\theta\;(\gamma, k)) \\
           \end{array}}
          {Semantics}
    \eline{\begin{array}{l}
             \interpE{\judgeE{\Gamma,n:\N,x:A}{e_1}{A}}\;\theta \\
             \;\;(\gamma, k,
                  \interpE{\judgeE{\Gamma, n:\N}{e}{A}}\;\theta\;(\gamma, k)) \\
           \end{array}}
          {Inner Induction}
    \eline{\interpE{\judgeE{\Gamma,n:\N}{[e/x]e_1}{A}}\;\theta\;(\gamma, k)} 
          {Substitution}
  \end{eqnproof}

  \begin{eqnproof}[\interpE{\judgeE{\Gamma, n:\N}{e}{A}}\;\theta\;(\gamma, s(k)) =]
    \eline{\interpE{\judgeE{\Gamma, m:\N, n:\N}{e}{A}}\;\theta\;(\gamma, k, s(k))}
          {Weakening}
    \eline{\interpE{\judgeE{\Gamma, m:\N, n:\N}{e}{A}}\;\theta\;(\gamma, k, \interpE{\judgeE{\Gamma, m:\N}{\s{m}}{\N}}\;\theta\;(\gamma, k))}
          {Semantics}
    \eline{\interpE{\judgeE{\Gamma, m:\N}{[\s{m}/n]e}{A}}\;\theta\;(\gamma, k)}
          {Substitution}
  \end{eqnproof}
\end{itemize}

  These two are equal by appeal to the outer induction hypothesis, which we get via 
  inversion on the original judgement. 

\item case EqAllBeta: 

  \begin{eqnproof}
    \eclaim[1]{\judgeEq{\Gamma}{(\Fun{\alpha}{\kappa}{e})\;\tau}{[\tau/\alpha]e}{[\tau/\alpha]A}}
              {Hypothesis}
    \eclaim[2]{\judgeE{\Gamma}{\Fun{\alpha}{\kappa}{e}}{\forall \alpha:\kappa.\;A}}
              {Inversion on 1}
    \eclaim[3]{\judgeWK{\tau}{\kappa}}
              {Inversion on 1}
    \eclaim[4]{\judgeE[\Theta,\alpha:\kappa]{\Gamma}{e}{A}}
              {Inversion on 2}
    \eclaim[5]{\judgeE{\Gamma}{[\tau/\alpha]e}{[\tau/\alpha]A}}
              {Substitute 3 into 4}
    \eclaim[6]{\judgeE{\Gamma}{(\Fun{\alpha}{\kappa}{e})\;\tau}{[\tau/\alpha]A}}
              {Rule ETApp on 2, 3}
  \end{eqnproof}

  \begin{eqnproof}[\interpE{\judgeE{\Gamma}{(\Fun{\alpha}{\kappa}{e})\;\tau}
                                   {[\tau/\alpha]A}}\;\theta\;\gamma = ]
    \eline{\interpE{\judgeE{\Gamma}{\Fun{\alpha}{\kappa}{e}}{\forall \alpha:\kappa.\;A}}
                   \;\theta\;\gamma
           \;
           [\theta(\tau)]}
          {Semantics}
    \eline{(\semfun{\sigma}{\interpE{\judgeE[\Theta, \alpha:\kappa]{\Gamma}{e}{A}}\;(\theta, \sigma)\;\gamma}) [\theta(\tau)]}
          {Semantics}
    \eline{\interpE{\judgeE[\Theta, \alpha:\kappa]{\Gamma}{e}{A}}\;(\theta, [\theta(\tau)])\;\gamma}
          {}
    \eline{\interpE{\judgeE{\Gamma}{[\tau/\alpha]e}{[\tau/\alpha]A}}\;\theta\;\gamma}
          {Type Substitution}
  \end{eqnproof}

\item case EqAllEta:

  \begin{eqnproof}
    \eclaim[1]{\judgeEq{\Gamma}{e}{e'}{\forall \alpha:\kappa.\;A}}
              {Hypothesis}
    \eclaim[2]{\judgeE{\Gamma}{e}{\forall \alpha:\kappa.\;A}}
              {Inversion on 1}
    \eclaim[3]{\judgeE{\Gamma}{e'}{\forall \alpha:\kappa.\;A}}
              {Inversion on 1}
  \end{eqnproof}

  \begin{eqnproof}
    \eline[\interpE{\judgeE[\Theta, \alpha:\kappa]{\Gamma}{e\;\alpha}{A}}\;(\theta, \sigma)\;\gamma]        {\interpE{\judgeE[\Theta, \alpha:\kappa]{\Gamma}{e'\;\alpha}{A}}\;(\theta, \sigma)\;\gamma}
          {Induction}[1em]

    \eline[\interpE{\judgeE[\Theta, \alpha:\kappa]{\Gamma}{e\;\alpha}{A}}\;(\theta, \sigma)\;\gamma]
          {(\interpE{\judgeE[\Theta, \alpha:\kappa]{\Gamma}{e}{\forall \alpha:\kappa.\;A}}\;(\theta, \sigma)\;\gamma)\;\sigma}
          {Semantics}
    \eline{(\interpE{\judgeE{\Gamma}{e}{\forall \alpha:\kappa.\;A}}\;\theta\;\gamma)\;\sigma}
          {Strengthening}[1em]

    \eline[\interpE{\judgeE[\Theta, \alpha:\kappa]{\Gamma}{e'\;\alpha}{A}}\;(\theta, \sigma)\;\gamma]
          {(\interpE{\judgeE[\Theta, \alpha:\kappa]{\Gamma}{e'}{\forall \alpha:\kappa.\;A}}\;(\theta, \sigma)\;\gamma)\;\sigma}
          {Semantics}
    \eline{(\interpE{\judgeE{\Gamma}{e'}{\forall \alpha:\kappa.\;A}}\;\theta\;\gamma)\;\sigma}
          {Strengthening}[1em]

    \eline[(\interpE{\judgeE{\Gamma}{e}{\forall \alpha:\kappa.\;A}}\;\theta\;\gamma)\;\sigma]
          {(\interpE{\judgeE{\Gamma}{e'}{\forall \alpha:\kappa.\;A}}\;\theta\;\gamma)\;\sigma}
          {Transitivity}[1em]

    \eline[\interpE{\judgeE{\Gamma}{e}{\forall \alpha:\kappa.\;A}}\;\theta\;\gamma]
          {\interpE{\judgeE{\Gamma}{e'}{\forall \alpha:\kappa.\;A}}\;\theta\;\gamma}
          {Extensionality}[1em]
  \end{eqnproof}

\item case EqExistsBeta: 

  \begin{eqnproof}
    \eclaim[1]{\judgeEq{\Gamma}{\unpack{\alpha}{x}{\pack{\tau}{e}}{e'}}{[\tau/\alpha, e/x]e'}{C}}
              {Hypothesis}
    \eclaim[2]{\judgeE{\Gamma}{\pack{\tau}{e}}{\exists \alpha:\kappa.\;A}}
              {Inversion on 1}
    \eclaim[3]{\judgeE[\Theta, \alpha:\kappa]{\Gamma, x:A}{e'}{C}}
              {Inversion on 1}
    \eclaim[4]{\judgeWK{\tau}{\kappa}}
              {Inversion on 2}
    \eclaim[5]{\judgeE{\Gamma}{e}{[\tau/\alpha]A}}
              {Inversion on 2}
    \eclaim[6]{\judgeE{\Gamma}{[\tau/\alpha, e/x]e'}{C}}
              {Substitution of 4,5 into 3}
    \eclaim[7]{\judgeE{\Gamma}{\unpack{\alpha}{x}{\pack{\tau}{e}}{e'}}{C}}
              {By rule EUnpack on 2, 3}
  \end{eqnproof}

  \begin{eqnproof}[\interpE{\judgeE{\Gamma}{\unpack{\alpha}{x}{\pack{\tau}{e}}{e'}}{C}}\;\theta\;\gamma = ]
    \eline{\begin{array}{l}
             (\semfun{(\sigma,v)}{\interpE{\judgeE[\Theta,\alpha:\kappa]{\Gamma,x:A}{e'}{C}}\;(\theta,\sigma)\;(\gamma,v)}) \\
             \;\;\interpE{\judgeE{\Gamma}{\pack{\tau}{e}}{\exists \alpha:\kappa.\;A}}\;\theta\;\gamma \\
           \end{array}}
          {Semantics}
    \eline{\begin{array}{l}
             (\semfun{(\sigma,v)}{\interpE{\judgeE[\Theta,\alpha:\kappa]{\Gamma,x:A}{e'}{C}}\;(\theta,\sigma)\;(\gamma,v)}) \\
             \;\;\left([\theta(\tau)], \interpE{\judgeE{\Gamma}{e}{[\tau/\alpha]A}}\;\theta\;\gamma\right) \\
           \end{array}}
          {Semantics}
    \eline{\begin{array}{l}
             \interpE{\judgeE[\Theta,\alpha:\kappa]{\Gamma,x:A}{e'}{C}} \\
             (\theta,[\theta(\tau)])\;(\gamma, \interpE{\judgeE{\Gamma}{e}{[\tau/\alpha]A}}\;\theta\;\gamma) \\
           \end{array}}
          {Simplify}
    \eline{\interpE{\judgeE{\Gamma,x:A}{[\tau/\alpha, e/x]e'}{C}} \theta\;\gamma}
          {Substitution}
  \end{eqnproof}

\item case EqExistsEta: 
  \begin{eqnproof}
    \eclaim[1]{\judgeEq{\Gamma}{\unpack{\alpha}{x}{e}{[\pack{\alpha}{x}/z]e'}}{[e/z]e'}{B}}
              {Hypothesis}
    \eclaim[2]{\judgeE{\Gamma}{e}{\exists \alpha:\kappa.\;A}}
              {Inversion on 1}
    \eclaim[3]{\judgeE{\Gamma, z:\exists \alpha:\kappa.\;A}{e'}{B}}
              {Inversion on 1}
    \eclaim[4]{\judgeE{\Gamma}{[e/z]e'}{B}}
              {Substitution of 2 into 3}
    \eclaim[5]{\judgeE[\Theta,\alpha:\kappa]{\Gamma, x:A, z:\exists \alpha:\kappa.\;A}{e'}{B}}
              {Weakening on 3}
    \eclaim[6]{\judgeE[\Theta,\alpha:\kappa]{\Gamma, x:A}{\pack{\alpha}{x}}{\exists \alpha:\kappa.\;A}}
              {Rule EPack}
    \eclaim[7]{\judgeE[\Theta,\alpha:\kappa]{\Gamma, x:A}{[\pack{\alpha}{x}/z]e'}{B}}
              {Substitution of 6 into 5}
    \eclaim[8]{\judgeE{\Gamma}{\unpack{\alpha}{x}{e}{[\pack{\alpha}{x}/z]e'}}{B}}
              {Rule EUnpack on 2, 7}
  \end{eqnproof}

  \begin{eqnproof}[\interpE{\judgeE{\Gamma}{\unpack{\alpha}{x}{e}{[\pack{\alpha}{x}/z]e'}}{B}}
                   \;\theta\;\gamma =]
    \eline{\begin{array}{l}
             (\semfun{(\sigma,v)}{\interpE{\judgeE[\Theta, \alpha:\kappa]
                                                  {\Gamma, x:A}{[\pack{\alpha}{x}/z]e'}{B}}
                                \;(\theta,\sigma)\;(\gamma,v)}) \\
           \;\interpE{\judgeE{\Gamma}{\pack{\tau}{e}}{\exists \alpha:\kappa.\;A}}
                     \;\theta\;\gamma \\
           \end{array}}
          {Semantics}
    \eline{\begin{array}{l}
             (\lambda (\sigma,v).\;\interpE{\judgeE[\Theta,\alpha:\kappa]
                                                  {\Gamma, x:A, z:\exists \alpha:\kappa.\;A}
                                                  {e'}{B}} \\
             \;\;  (\theta,\sigma) \\
             \;\;  (\gamma,v, 
                      \interpE{\judgeE[\Theta, \alpha:\kappa]
                                      {\Gamma, x:A}{\pack{\alpha}{x}}{\exists \alpha:\kappa.\;A}}
                      \;(\theta,\sigma)\;(\gamma, v)) \\
           \;\interpE{\judgeE{\Gamma}{\pack{\tau}{e}}{\exists \alpha:\kappa.\;A}}
                     \;\theta\;\gamma \\
           \end{array}}
          {Substitution}
    \eline{\begin{array}{l}
             (\lambda (\sigma,v).\;\interpE{\judgeE[\Theta,\alpha:\kappa]
                                                  {\Gamma, x:A, z:\exists \alpha:\kappa.\;A}
                                                  {e'}{B}} 
             \;(\theta,\sigma)\; (\gamma,v, (\sigma, v))) \\
           \;\interpE{\judgeE{\Gamma}{\pack{\tau}{e}}{\exists \alpha:\kappa.\;A}}
                     \;\theta\;\gamma \\
           \end{array}}
          {Semantics}
    \eline{\begin{array}{l}
             (\lambda (\sigma,v).\;\interpE{\judgeE[\Theta,\alpha:\kappa]
                                                  {\Gamma, x:A, z:\exists \alpha:\kappa.\;A}
                                                  {e'}{B}} 
             \;(\theta,\sigma)\; (\gamma,v, (\sigma, v))) \\
           \;([\theta(\tau)], \interpE{\judgeE{\Gamma}{e}{[\tau/\alpha]A}}\;\theta\;\gamma) \\
           \end{array}}
          {Semantics}
    \eline{\begin{array}{l}
             \interpE{\judgeE[\Theta,\alpha:\kappa]
                             {\Gamma, x:A, z:\exists \alpha:\kappa.\;A}
                             {e'}{B}} \\
             \; (\theta, [\theta(\tau)]) \\
             \; (\gamma, \interpE{\judgeE{\Gamma}{e}{[\tau/\alpha]A}}\;\theta\;\gamma,
                 ([\theta(\tau)], \interpE{\judgeE{\Gamma}{e}{[\tau/\alpha]A}}\;\theta\;\gamma))\\
           \end{array}}
          {Simplify}
    \eline{\begin{array}{l}
             \interpE{\judgeE[\Theta,\alpha:\kappa]
                             {\Gamma, x:A, z:\exists \alpha:\kappa.\;A}
                             {e'}{B}} \\
             \; (\theta, [\theta(\tau)]) \\
             \; (\gamma, \interpE{\judgeE{\Gamma}{e}{[\tau/\alpha]A}}\;\theta\;\gamma,
                 \interpE{\judgeE{\Gamma}{\pack{\tau}{e}}{\exists \alpha:\kappa.\;A}}
                     \;\theta\;\gamma) \\
           \end{array}}
          {Semantics}
    \eline{\begin{array}{l}
             \interpE{\judgeE[\Theta,\alpha:\kappa]{\Gamma, x:A}{[e/z]e'}{B}} \\
             \; (\theta, [\theta(\tau)]) 
             \; (\gamma, \interpE{\judgeE{\Gamma}{e}{[\tau/\alpha]A}}\;\theta\;\gamma) \\
           \end{array}}
          {Substitution}
    \eline{\interpE{\judgeE{\Gamma}{[e/z]e'}{B}}\;\theta\;\gamma}
          {$\alpha,x \not \in FV([e/z]e')$}
  \end{eqnproof}

\item case EqCommandEta: 

  \begin{eqnproof}
    \eclaim[1]{\judgeEqC{\Gamma}{c}{\letv{x}{\comp{c}}{x}}{A}}
              {Hypothesis}
    \eclaim[2]{\judgeC{\Gamma}{c}{A}}
              {Inversion on 1}
    \eclaim[3]{\judgeE{\Gamma}{\comp{c}}{\monad{A}}}
              {Rule EMonad on 2}
    \eclaim[4]{\judgeE{\Gamma, x:A}{x}{A}}
              {Rule EHyp}
    \eclaim[5]{\judgeC{\Gamma, x:A}{x}{A}}
              {Rule CReturn on 4}
    \eclaim[6]{\judgeC{\Gamma}{\letv{x}{\comp{c}}{x}}{A}}
              {By Rule CLet on 3,5}
  \end{eqnproof}

  \begin{eqnproof}[\interpC{\judgeC{\Gamma}{\letv{x}{\comp{c}}{x}}{A}}\;\theta\;\gamma =]
    \eline{\begin{array}{l}
             (\semfun{v}{\interpC{\judgeC{\Gamma,x:A}{x}{A}}\;\theta\;(\gamma,v)})^* \\
           \; \interpE{\judgeE{\Gamma}{\comp{c}}{\monad{A}}}\;\theta\;\gamma \\ 
           \end{array}}
          {Semantics}
    \eline{\begin{array}{l}
            (\semfun{v}{\eta(\interpE{\judgeE{\Gamma,x:A}{x}{A}}\;\theta\;(\gamma,v))})^* \\
            \;\interpC{\judgeC{\Gamma}{c}{A}}\;\theta\;\gamma \\
           \end{array}}
          {Semantics}
    \eline{(\semfun{v}{(\eta(v))})^*\;(\interpC{\judgeC{\Gamma}{c}{A}}\;\theta\;\gamma)}
          {Simplify}
    \eline{id(\interpC{\judgeC{\Gamma}{c}{A}}\;\theta\;\gamma)}
          {Monad law}
    \eline{\interpC{\judgeC{\Gamma}{c}{A}}\;\theta\;\gamma}
          {Simplify}
  \end{eqnproof}

\item EqCommandBeta:

  \begin{eqnproof}
    \eclaim[1]{\judgeEqC{\Gamma}{\letv{x}{\comp{e}}{c}}{[e/x]c}{B}}
              {Hypothesis}
    \eclaim[2]{\judgeC{\Gamma}{\letv{x}{\comp{e}}{c}}{B}}
              {Inversion on 1}
    \eclaim[3]{\judgeE{\Gamma}{\comp{e}}{\monad{A}}}
              {Inversion on 2}
    \eclaim[4]{\judgeC{\Gamma,x:A}{c}{B}}
              {Inversion on 2}
    \eclaim[5]{\judgeC{\Gamma}{e}{A}}
              {Inversion on 3}
    \eclaim[6]{\judgeE{\Gamma}{e}{A}}
              {Inversion on 5}
    \eclaim[7]{\judgeC{\Gamma}{[e/x]c}{B}}
              {Substitution of 6 into 4}
  \end{eqnproof}

  \begin{eqnproof}[\interpC{\judgeC{\Gamma}{\letv{x}{\comp{e}}{c}}{B}}\;\theta\;\gamma =]
    \eline{\begin{array}{l}
             (\semfun{v}{\interpC{\judgeC{\Gamma,x:A}{c}{B}}\;\theta\;(\gamma,v)})^* \\
             \;\; (\interpE{\judgeE{\Gamma}{\comp{e}}{\monad{A}}}\;\theta\;\gamma) \\
           \end{array}}
          {Semantics}
    \eline{\begin{array}{l}
             (\semfun{v}{\interpC{\judgeC{\Gamma,x:A}{c}{B}}\;\theta\;(\gamma,v)})^* \\
             \;\; (\interpC{\judgeC{\Gamma}{e}{A}}\;\theta\;\gamma) \\
           \end{array}}
          {Semantics}
    \eline{\begin{array}{l}
             (\semfun{v}{\interpC{\judgeC{\Gamma,x:A}{c}{B}}\;\theta\;(\gamma,v)})^* \\
             \;\; \eta(\interpE{\judgeE{\Gamma}{e}{A}}\;\theta\;\gamma) \\
           \end{array}}
          {Semantics}
    \eline{\begin{array}{l}
             (\semfun{v}{\interpC{\judgeC{\Gamma,x:A}{c}{B}}\;\theta\;(\gamma,v)}) \\
             \;\; (\interpE{\judgeE{\Gamma}{e}{A}}\;\theta\;\gamma) \\
           \end{array}}
          {Monad laws}
    \eline{\interpC{\judgeC{\Gamma,x:A}{c}{B}}
             \;\theta\;(\gamma, \interpE{\judgeE{\Gamma}{e}{A}}\;\theta\;\gamma)}
          {Simplify}
    \eline{\interpC{\judgeC{\Gamma}{[e/x]c}{B}}\;\theta\;\gamma}
          {Substitution}
  \end{eqnproof}

\item EqCommandComm:
  \begin{eqnproof}
    \eclaim[1]{\scriptsize \judgeEqC{\Gamma}{\letv{x}{\comp{\letv{y}{e}{c_1}}}{c_2}}
                                {\letv{y}{e}{\letv{x}{\comp{c_1}}{c_2}}}{C}}
              {Hypothesis}
    \eclaim[2]{\judgeC{\Gamma}{\letv{x}{\comp{\letv{y}{e}{c_1}}}{c_2}}{C}}
              {Inversion on 1}
    \eclaim[3]{\judgeE{\Gamma}{\comp{\letv{y}{e}{c_1}}}{\monad{B}}}
              {Inversion on 2}
    \eclaim[4]{\judgeC{\Gamma,x:B}{c_2}{C}}
              {Inversion on 2}
    \eclaim[5]{\judgeC{\Gamma}{\letv{y}{e}{c_1}}{B}}
              {Inversion on 3}
    \eclaim[6]{\judgeE{\Gamma}{e}{\monad{A}}}
              {Inversion on 5}
    \eclaim[7]{\judgeC{\Gamma,y:A}{c_1}{B}}
              {Inversion on 5}
    \eclaim[8]{\judgeC{\Gamma,y:A,x:B}{c_2}{C}}
              {Weakening on 4}
    \eclaim[9]{\judgeE{\Gamma,y:A}{\comp{c_1}}{\monad{B}}}
              {Rule EMonad on 7}
    \eclaim[10]{\judgeC{\Gamma,y:A}{\letv{x}{\comp{c_1}}{c_2}}{C}}
               {Rule CLet on 9, 8}
    \eclaim[11]{\judgeC{\Gamma}{\letv{y}{e}{\letv{x}{\comp{c_1}}{c_2}}}{C}}
               {Rule CLet on 6, 10}
  \end{eqnproof}

  \begin{eqnproof}[\interpC{\judgeC{\Gamma}{\letv{x}{\comp{\letv{y}{e}{c_1}}}{c_2}}{C}}
                   \;\theta\;\gamma =]
    \elines{(\semfun{v_2}{\interpC{\judgeC{\Gamma,x:B}{c_2}{C}}\;\theta\;(\gamma,v_2)})^* \\
            \; \interpE{\judgeE{\Gamma}{\comp{\letv{y}{e}{c_1}}}{\monad{B}}}\;\theta\;\gamma \\}
           {Semantics}
    \elines{(\semfun{v_2}{\interpC{\judgeC{\Gamma,x:B}{c_2}{C}}\;\theta\;(\gamma,v_2)})^* \\
            \; \interpC{\judgeC{\Gamma}{\letv{y}{e}{c_1}}{\monad{B}}}\;\theta\;\gamma \\}
           {Semantics}
    \elines{(\semfun{v_2}{\interpC{\judgeC{\Gamma,x:B}{c_2}{C}}\;\theta\;(\gamma,v_2)})^* \\
            \; ((\semfun{v_1}{\interpC{\judgeC{\Gamma,y:A}{c_1}{B}}\;\theta\;(\gamma,v_1)})^* \\
            \;\;\; \interpE{\judgeE{\Gamma}{e}{\monad{A}}}\;\theta\;\gamma) \\ }
           {Semantics}
    \elines{(\lambda v_1.\; (\semfun{v_2}{\interpC{\judgeC{\Gamma,x:B}{c_2}{C}}\;\theta\;(\gamma,v_2)})^* \\
            \;\;(\interpC{\judgeC{\Gamma,y:A}{c_1}{B}}\;\theta\;(\gamma,v_1)))^* \\  
            \qquad \interpE{\judgeE{\Gamma}{e}{\monad{A}}}\;\theta\;\gamma) \\}
           {Monad Laws}
    \elines{(\lambda v_1.\; (\semfun{v_2}{\interpC{\judgeC{\Gamma,y:A,x:B}{c_2}{C}}\;\theta\;(\gamma,v_1, v_2)})^* \\
            \;\;(\interpC{\judgeC{\Gamma,y:A}{c_1}{B}}\;\theta\;(\gamma,v_1)))^* \\  
            \qquad \interpE{\judgeE{\Gamma}{e}{\monad{A}}}\;\theta\;\gamma) \\}
           {Weakening}
    \elines{(\lambda v_1.\; (\semfun{v_2}{\interpC{\judgeC{\Gamma,y:A,x:B}{c_2}{C}}\;\theta\;(\gamma,v_1, v_2)})^* \\
            \;\;(\interpE{\judgeE{\Gamma,y:A}{\comp{c_1}}{\monad{B}}}\;\theta\;(\gamma,v_1)))^* \\  
            \qquad \interpE{\judgeE{\Gamma}{e}{\monad{A}}}\;\theta\;\gamma) \\}
           {Semantics}
    \elines{(\lambda v_1.\;
              \interpC{\judgeC{\Gamma,y:A}{\letv{x}{\comp{c_1}}{c_2}}{C}}
                      \;\theta\;(\gamma,v_1))^* \\
            \qquad \interpE{\judgeE{\Gamma}{e}{\monad{A}}}\;\theta\;\gamma \\}
           {Semantics}
    \eline{\interpC{\judgeC{\Gamma}{\letv{y}{e}{\letv{x}{c_1}{c_2}}}{C}}\;\theta\;\gamma}
          {Semantics}
  \end{eqnproof}

\item EqRefl
  \begin{eqnproof}
    \eclaim[1]{\judgeEq{\Gamma}{e}{e}{A}}
              {Hypothesis}
    \eclaim[2]{\judgeE{\Gamma}{e}{A}}
              {Inversion on 1}
  \end{eqnproof}

  \begin{eqnproof}
    \eline[\interpE{\judgeE{\Gamma}{e}{A}}\;\theta\;\gamma] 
          {\interpE{\judgeE{\Gamma}{e}{A}}\;\theta\;\gamma}
          {Reflexivity}
  \end{eqnproof}

\item EqSymm
   \begin{eqnproof}
     \eclaim[1]{\judgeEq{\Gamma}{e}{e'}{A}}
               {Hypothesis}
     \eclaim[2]{\judgeEq{\Gamma}{e'}{e}{A}}
               {Inversion on 1}
     \eclaim[3]{\judgeE{\Gamma}{e}{A}}
               {Induction on 2}
     \eclaim[4]{\judgeE{\Gamma}{e'}{A}}
               {Induction on 2}
   \end{eqnproof}

   \begin{eqnproof}
     \eline[\interpE{\judgeE{\Gamma}{e'}{A}}\;\theta\;\gamma]
           {\interpE{\judgeE{\Gamma}{e}{A}}\;\theta\;\gamma}
           {Induction on 2, above}
     \eline[\interpE{\judgeE{\Gamma}{e}{A}}\;\theta\;\gamma]
           {\interpE{\judgeE{\Gamma}{e'}{A}}\;\theta\;\gamma}
           {Symmetry on prev step}
   \end{eqnproof}

\item case EqTrans
  \begin{eqnproof}
    \eclaim[1]{\judgeEq{\Gamma}{e}{e''}{A}}
              {Hypothesis}
    \eclaim[2]{\judgeE{\Gamma}{e}{e'}{A}}
              {Inversion on 1}
    \eclaim[3]{\judgeE{\Gamma}{e'}{e''}{A}}
              {Inversion on 1}
    \eclaim[4]{\judgeE{\Gamma}{e}{A}}
              {Induction on 2}
    \eclaim[5]{\judgeE{\Gamma}{e''}{A}}
              {Induction on 3}
  \end{eqnproof}

  \begin{eqnproof}
    \eline[\interpE{\judgeE{\Gamma}{e}{A}}\;\theta\;\gamma]
          {\interpE{\judgeE{\Gamma}{e'}{A}}\;\theta\;\gamma}
          {Induction}
    \eline[\interpE{\judgeE{\Gamma}{e'}{A}}\;\theta\;\gamma]
          {\interpE{\judgeE{\Gamma}{e''}{A}}\;\theta\;\gamma}
          {Induction}
    \eline[\interpE{\judgeE{\Gamma}{e}{A}}\;\theta\;\gamma]
          {\interpE{\judgeE{\Gamma}{e''}{A}}\;\theta\;\gamma}
          {Transitivity}
  \end{eqnproof}

\item case EqSubst:
  \begin{eqnproof}
    \eclaim[1]{\judgeEq{\Gamma}{[e_2/x]e_1}{[e'_2/x]e'_1}{B}}
              {Hypothesis}
    \eclaim[2]{\judgeEq{\Gamma,x:A}{e_1}{e'_1}{B}}
              {Inversion on 1}
    \eclaim[3]{\judgeEq{\Gamma}{e_2}{e'_2}{A}}
              {Inversion on 1}
    \eclaim[4]{\judgeE{\Gamma,x:A}{e_1}{B}}
              {Induction on 2}
    \eclaim[5]{\judgeE{\Gamma,x:A}{e'_1}{B}}
              {Induction on 2}
    \eclaim[6]{\judgeE{\Gamma}{e_2}{A}}
              {Induction on 3}
    \eclaim[7]{\judgeE{\Gamma}{e'_2}{A}}
              {Induction on 3}
    \eclaim[8]{\judgeE{\Gamma}{[e_2/x]e_1}{B}}
              {Substitution of 6 into 4}
    \eclaim[9]{\judgeE{\Gamma}{[e'_2/x]e'_1}{B}}
              {Substitution of 7 into 5}
  \end{eqnproof}

  \begin{eqnproof}[\interpE{\judgeE{\Gamma}{[e_2/x]e_1}{B}}\;\theta\;\gamma =]
     \eline{\interpE{\judgeE{\Gamma,x:A}{e_1}{B}}\;\theta\;
              (\gamma,\interpE{\judgeE{\Gamma}{e_2}{A}}\;\theta\;\gamma)} 
           {Substitution}
     \eline{\interpE{\judgeE{\Gamma,x:A}{e_1}{B}}\;\theta\;
              (\gamma,\interpE{\judgeE{\Gamma}{e'_2}{A}}\;\theta\;\gamma)} 
           {Induction}
     \eline{\interpE{\judgeE{\Gamma,x:A}{e'_1}{B}}\;\theta\;
              (\gamma,\interpE{\judgeE{\Gamma}{e'_2}{A}}\;\theta\;\gamma)} 
           {Induction}
     \eline{\interpE{\judgeE{\Gamma}{[e'_2/x]e'_1}{B}}\;\theta\;\gamma}
           {Substitution}
  \end{eqnproof}

\item EqCommandRefl
  \begin{eqnproof}
    \eclaim[1]{\judgeEqC{\Gamma}{c}{c}{A}}
              {Hypothesis}
    \eclaim[2]{\judgeC{\Gamma}{c}{A}}
              {Inversion on 1}
  \end{eqnproof}

  \begin{eqnproof}
    \eline[\interpC{\judgeC{\Gamma}{c}{A}}\;\theta\;\gamma] 
          {\interpC{\judgeC{\Gamma}{c}{A}}\;\theta\;\gamma}
          {Reflexivity}
  \end{eqnproof}

\item EqCommandSymm
   \begin{eqnproof}
     \eclaim[1]{\judgeEqC{\Gamma}{c}{c'}{A}}
               {Hypothesis}
     \eclaim[2]{\judgeEqC{\Gamma}{c'}{c}{A}}
               {Inversion on 1}
     \eclaim[3]{\judgeC{\Gamma}{c}{A}}
               {Induction on 2}
     \eclaim[4]{\judgeC{\Gamma}{c'}{A}}
               {Induction on 2}
   \end{eqnproof}

   \begin{eqnproof}
     \eline[\interpC{\judgeC{\Gamma}{c'}{A}}\;\theta\;\gamma]
           {\interpC{\judgeC{\Gamma}{c}{A}}\;\theta\;\gamma}
           {Induction on 2, above}
     \eline[\interpC{\judgeC{\Gamma}{c}{A}}\;\theta\;\gamma]
           {\interpC{\judgeC{\Gamma}{c'}{A}}\;\theta\;\gamma}
           {Symmetry on prev step}
   \end{eqnproof}

\item case EqCommandTrans
  \begin{eqnproof}
    \eclaim[1]{\judgeEqC{\Gamma}{c}{c''}{A}}
              {Hypothesis}
    \eclaim[2]{\judgeC{\Gamma}{c}{c'}{A}}
              {Inversion on 1}
    \eclaim[3]{\judgeC{\Gamma}{c'}{c''}{A}}
              {Inversion on 1}
    \eclaim[4]{\judgeC{\Gamma}{c}{A}}
              {Induction on 2}
    \eclaim[5]{\judgeC{\Gamma}{c''}{A}}
              {Induction on 3}
  \end{eqnproof}

  \begin{eqnproof}
    \eline[\interpC{\judgeC{\Gamma}{c}{A}}\;\theta\;\gamma]
          {\interpC{\judgeC{\Gamma}{c'}{A}}\;\theta\;\gamma}
          {Induction}
    \eline[\interpC{\judgeC{\Gamma}{c'}{A}}\;\theta\;\gamma]
          {\interpC{\judgeC{\Gamma}{c''}{A}}\;\theta\;\gamma}
          {Induction}
    \eline[\interpC{\judgeC{\Gamma}{c}{A}}\;\theta\;\gamma]
          {\interpC{\judgeC{\Gamma}{c''}{A}}\;\theta\;\gamma}
          {Transitivity}
  \end{eqnproof}

\item case EqCommandSubst:
  \begin{eqnproof}
    \eclaim[1]{\judgeEqC{\Gamma}{[e_2/x]c_1}{[e'_2/x]c'_1}{B}}
              {Hypothesis}
    \eclaim[2]{\judgeEqC{\Gamma,x:A}{c_1}{c'_1}{B}}
              {Inversion on 1}
    \eclaim[3]{\judgeEq{\Gamma}{e_2}{e'_2}{A}}
              {Inversion on 1}
    \eclaim[4]{\judgeC{\Gamma,x:A}{c_1}{B}}
              {Induction on 2}
    \eclaim[5]{\judgeC{\Gamma,x:A}{c'_1}{B}}
              {Induction on 2}
    \eclaim[6]{\judgeE{\Gamma}{e_2}{A}}
              {Induction on 3}
    \eclaim[7]{\judgeE{\Gamma}{e'_2}{A}}
              {Induction on 3}
    \eclaim[8]{\judgeC{\Gamma}{[e_2/x]c_1}{B}}
              {Substitution of 6 into 4}
    \eclaim[9]{\judgeC{\Gamma}{[e'_2/x]c'_1}{B}}
              {Substitution of 7 into 5}
  \end{eqnproof}

  \begin{eqnproof}[\interpC{\judgeC{\Gamma}{[e_2/x]c_1}{B}}\;\theta\;\gamma =]
     \eline{\interpC{\judgeC{\Gamma,x:A}{c_1}{B}}\;\theta\;
              (\gamma,\interpE{\judgeE{\Gamma}{e_2}{A}}\;\theta\;\gamma)} 
           {Substitution}
     \eline{\interpC{\judgeC{\Gamma,x:A}{c_1}{B}}\;\theta\;
              (\gamma,\interpE{\judgeE{\Gamma}{e'_2}{A}}\;\theta\;\gamma)} 
           {Induction}
     \eline{\interpC{\judgeC{\Gamma,x:A}{c'_1}{B}}\;\theta\;
              (\gamma,\interpE{\judgeE{\Gamma}{e'_2}{A}}\;\theta\;\gamma)} 
           {Induction}
     \eline{\interpC{\judgeC{\Gamma}{[e'_2/x]c'_1}{B}}\;\theta\;\gamma}
           {Substitution}
  \end{eqnproof}

\end{itemize}
\end{proof}
