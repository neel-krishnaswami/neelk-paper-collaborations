\chapter{Introduction}

My thesis is that it is possible to give modular correctness proofs of
interesting higher-order imperative programs using higher-order
separation logic.


\section{Contributions}

To demonstrate my thesis, I make a number of concrete contributions.
In Chapter 2, I define a higher-order imperative programming language
based on a predicate variant of $F_\omega$~\citep{fomega}, augmented
with reference types which uses a monadic language to encapsulate its
side-effects (including both modification of the heap, and
nontermination). I give this language a domain-theoretic denotational
semantics based on the techniques of \citet{smyth-plotkin}, which lets
me validate strong equational reasoning principles --- including both
the $\beta-$ and $\eta$-rules of the lambda calculus ---

Equational reasoning is less helpful for imperative programs, and to
support reasoning about this part of the programming language, I
define and prove the soundness of a program logic in Chapter 3. The
program logic combines ideas from specification logic and higher-order
separation logic to give an expressive program logic 
