\documentclass{article}
\usepackage{mathpartir}
\usepackage{amsmath}
\usepackage{amssymb}

\newcommand{\ctext}[1]{\mathsf{#1}}

\newcommand{\normalize}[1]{\ctext{nf}({#1})}

\newcommand{\unittype}{\mathbf{1}}
\newcommand{\reftype}[1]{\ctext{ref}\;#1}
\newcommand{\monad}[1]{\bigcirc{#1}}
\newcommand{\cont}[1]{\ctext{cont}\;{#1}}
\newcommand{\opttype}[1]{\ctext{option }{#1}}
\newcommand{\seqsort}[1]{\ctext{seq}\;{#1}}
\newcommand{\listtype}[1]{\ctext{list}\;{#1}}

\newcommand{\pair}[2]{\left<{#1},{#2}\right>}
\newcommand{\fst}[1]{\ctext{fst}\; #1}
\newcommand{\snd}[1]{\ctext{snd}\; #1}

\newcommand{\inj}[1]{\iota_{#1}}
\newcommand{\inl}[1]{\ctext{inl }#1}
\newcommand{\inr}[1]{\ctext{inr }#1}
\newcommand{\Case}[5]{\ctext{case}(#1,\; {#2}.\; #3,\; {#4}.\; #5)}
\newcommand{\listcase}[5]{\ctext{case}(#1,\;\ctext{Nil} \to {#2},\;
                                       \ctext{Cons}({#3},{#4}) \to {#5})}
\newcommand{\optcase}[4]{\ctext{case}(#1,\;
                                      \ctext{None} \to {#2},\;
                                      \ctext{Some}\;{#3} \to {#4})}
\newcommand{\z}{\ctext{z}}
\newcommand{\s}[1]{\ctext{s}(#1)}
\newcommand{\iter}[4]{\ctext{iter}(#1, {#2}, {#3}.\; {#4})}
\newcommand{\iterseq}[4]{\ctext{iter}_{\mathsf{seq}}(#1, {#2}, #3.\; {#4})}
\newcommand{\comp}[1]{[#1]}
\newcommand{\fun}[3]{\lambda #1:#2.\;#3}
\newcommand{\Fun}[3]{\Lambda #1:#2.\;#3}
\newcommand{\unit}{\left<\right>}
\newcommand{\pack}[2]{\ctext{pack}({#1}, {#2})}
\newcommand{\unpack}[4]{\ctext{unpack}({#1}, {#2}) = {#3} \;\ctext{in}\;{#4}}
\newcommand{\alt}{\;|\;}
\newcommand{\letv}[3]{\ctext{letv}\;#1 = #2\;\ctext{in}\;#3}
\newcommand{\newref}[2]{\ctext{new}_{#1}(#2)}
\newcommand{\run}[1]{\ctext{run}\;{#1}}
\newcommand{\ok}{\ctext{ ok}}
\newcommand{\FV}[1]{\mathrm{FV}({#1})}

\newcommand{\fix}[2]{\ctext{fix}\;{#1}.\;{#2}}

\newcommand{\statecfg}[2]{\left<{#1};\;{#2}\right>}
\newcommand{\eval}[4]{\left<{#1};\;{#2}\right> \leadsto \left<{#3};\;{#4}\right>}
\newcommand{\evalabort}[2]{\left<{#1};\;{#2}\right> \leadsto \mathbf{abort}}

\newcommand{\domain}[1]{\mbox{dom}({#1})}
\newcommand{\upset}[1]{\mathcal{P}^{\uparrow}({#1})}

\newcommand{\pointsto}{\mapsto}
\newcommand{\disj}{\vee}
\renewcommand{\implies}{\supset}
\newcommand{\wand}{-\!\!*\,}
\newcommand{\emp}{\mathsf{emp}}
\newcommand{\validprop}[1]{{#1}\;\ctext{valid}}

\newcommand{\todo}[1]{\texttt{[TODO: {#1}]}}

\newcommand{\setof}[1]{\{{#1}\}}

\newcommand{\To}{\Rightarrow}
\newcommand{\From}{\Leftarrow}

\newcommand{\N}{\mathbb{N}}

\newcommand{\assert}{\ctext{prop}}

\newcommand{\bigstep}{\Downarrow}

\newcommand{\bigeval}[4]{\left<{#1};\;{#2}\right> \bigstep \left<{#3};\;{#4}\right>}
\newcommand{\bigevalabort}[2]{\left<{#1};\;{#2}\right> \bigstep \mathbf{abort}}

\newcommand{\spec}[4]{\{{#1}\}{#2}\{{#3}.\;{#4}\}}
\newcommand{\specX}[3]{\{{#1}\}{#2}\{{#3}\}}
\newcommand{\mspec}[4]{\langle{#1}\rangle{#2}\langle{#3}.\;{#4}\rangle}
\newcommand{\bnfalt}{\;\;|\;\;}

\newcommand{\specor}{\;||\;}
\newcommand{\specand}{\;\&\;}
\newcommand{\specimp}{\Rightarrow\!\!\!>}
% \newcommand{\specimp}{\Longrightarrow}
% \newcommand{\specor}{\ctext{ or }}
% \newcommand{\specand}{\ctext{ and }}
% \newcommand{\specimp}{\ctext{ implies }}
\newcommand{\spectype}{\ctext{spec}}
\newcommand{\valid}{\ctext{ valid}}

\newcommand{\interp}[1]{[\![{#1}]\!]}
\newcommand{\interpE}[1]{\interp{#1}^e}
\newcommand{\interpC}[1]{\interp{#1}^c}

\newcommand{\interpF}[1]{[\![{#1}]\!]_f}
\newcommand{\interpmono}[1]{\interp{#1}^{\mathrm{m}}}

\newcommand{\entails}{\models}

\newcommand{\judgeE}[4][\Theta]{{#1};\;{#2} \vdash {#3} : {#4}}
\newcommand{\judgeC}[4][\Theta]{{#1};\;{#2} \vdash {#3} \div {#4}}
\newcommand{\judgeEq}[5][\Theta]{{#1};\;{#2} \vdash {#3} \equiv {#4} : {#5}}
\newcommand{\judgeEqC}[5][\Theta]{{#1};{#2} \vdash {#3} \equiv {#4} \div {#5}}




%% semantic operations

\newcommand{\worldleq}{\preceq}
\newcommand{\worldgeq}{\succeq}

\newcommand{\semfun}[2]{\lambda #1.\;#2}
\newcommand{\sempair}[2]{\left({#1}, {#2}\right)}
\newcommand{\powerset}[1]{\mathcal{P}(#1)}
\newcommand{\powersetfin}[1]{\mathcal{P}^{\mathrm{fin}}(#1)}

\newcommand{\paircat}[2]{\left<{#1};{#2}\right>}
\newcommand{\sumcat}[2]{\left[{#1};{#2}\right]}
\newcommand{\abscat}[1]{\lambda({#1})}


\newcommand{\judgeP}[3]{{#1} \vdash {#2} : {#3}}
\newcommand{\judgeS}[2][\Delta]{{#1} \vartriangleright {#2} : \spectype}

\newcommand{\judgeSCtx}[2]{{#1} \vartriangleright {#2} : \ctext{context}}

\newcommand{\judgeEqP}[4]{{#1} \vdash {#2} \equiv {#3} : {#4}}


\newcommand{\judgeWK}[3][\Theta]{{#1} \vdash {#2} : {#3}}
\newcommand{\judgeKeq}[4][\Theta]{{#1} \vdash {#2} \equiv {#3} : {#4}}

\newcommand{\entailsP}[3]{{#1} \vartriangleright {#2} \vdash {#3}}
\newcommand{\entailsS}[3]{{#1}; {#2} \vdash {#3} \ok}

\newcommand{\chartp}{\ctext{char}}
\newcommand{\fonttp}{\ctext{font}}

\newcommand{\LOC}{loc}
\newcommand{\MONO}{\mathbf{mono}}
\newcommand{\HEAP}{heap}
\newcommand{\PROP}{Prop}
\newcommand{\TRUE}{True}
\newcommand{\HPROP}{Prop}
\newcommand{\TYPE}{Type}

\newtheorem{prop}{Proposition}
\newtheorem{lemma}{Lemma}

\newcommand{\comprehend}[2]{\setof{{#1}\;|\;{#2}}}


% \newenvironment{proof}{\begin{comment}}{\end{comment}}

\newenvironment{proof}{\noindent\textbf{Proof.}}{\noindent\ensuremath{\Box}}

\newcounter{prooflinenum}
\newenvironment{tabbedproof}
   {\setcounter{prooflinenum}{0}
    \begin{tabbing}\;\;\=\;\;\;\;\;\=\;\;\;\;\=\;\;\;\;\=\;\;\;\;\=\;\;\;\;\=\;\;\;\;\=\;\;\;\;\=\;\;\;\;\=\\[-2em]}
   {\end{tabbing}}

\newcommand{\oo}{\addtocounter{prooflinenum}{1}\arabic{prooflinenum}\>\>}
\newcommand{\ooo}{\addtocounter{prooflinenum}{1}\arabic{prooflinenum}\>\>\>}
\newcommand{\oooo}{\addtocounter{prooflinenum}{1}\arabic{prooflinenum}\>\>\>\>}
\newcommand{\ooooo}{\addtocounter{prooflinenum}{1}\arabic{prooflinenum}\>\>\>\>\>}
\newcommand{\oooooo}{\addtocounter{prooflinenum}{1}\arabic{prooflinenum}\>\>\>\>\>\>}
\newcommand{\ooooooo}{\addtocounter{prooflinenum}{1}\arabic{prooflinenum}\>\>\>\>\>\>\>}
\newcommand{\oooooooo}{\addtocounter{prooflinenum}{1}\arabic{prooflinenum}\>\>\>\>\>\>\>\>}
\newcommand{\ooooooooo}{\addtocounter{prooflinenum}{1}\arabic{prooflinenum}\>\>\>\>\>\>\>\>\>}

\newcommand{\ox}{\>\>}
\newcommand{\oox}{\>\>\>}
\newcommand{\ooox}{\>\>\>\>}
\newcommand{\oooox}{\>\>\>\>\>}
\newcommand{\ooooox}{\>\>\>\>\>\>}
\newcommand{\oooooox}{\>\>\>\>\>\>\>}
\newcommand{\ooooooox}{\>\>\>\>\>\>\>\>}
\newcommand{\oooooooox}{\>\>\>\>\>\>\>\>\>}


\newenvironment{eqnproof}[1][]{${#1}$\begin{displaymath}\begin{array}{lcll}}
                         {\end{array}\end{displaymath}}

\newcommand{\eline}[3][]{{#1} & = & {#2} & \mbox{{#3}} \\}

\newcommand{\elines}[3][]{{#1} & = & \begin{array}{l} #2 \end{array} & \mbox{{#3}} \\}

\newcommand{\eclaim}[3][]{{#1} &  & {#2} & \mbox{{#3}} \\}
\newcommand{\efact}[2]{{#1} & & & \mbox{#2} \\}

\newcommand{\basicspec}[4]{[{#1}]\;{#2}\;[{#3}.\;{#4}]}

% Macros for type checking assertions

\newcommand{\pfun}[3]{\hat{\lambda} #1:#2.\;#3}
\newcommand{\restrictkind}[1]{({#1})\Downarrow_K}
\newcommand{\restricttype}[1]{({#1})\Downarrow_T}

\newcommand{\restricttyenv}[2]{{#2}\Downarrow^{#1}_K}
\newcommand{\restrictvals}[2]{{#2}\Downarrow^{#1}_T}
\newcommand{\judgeACtx}[1]{\vartriangleright {#1} \;\ctext{ok}}
\newcommand{\judgeA}[3][\Delta]{{#1} \vartriangleright {#2} : {#3}}
\newcommand{\judgeSort}[2][\Delta]{\judgeA[{#1}]{{#2}}{\ctext{sort}}}
\newcommand{\judgeSortEq}[3][\Delta]{{#1} \vartriangleright {#2} \equiv {#3} : \ctext{sort}}

\newcommand{\ms}[1]{\mathcal{#1}}
\newcommand{\Frame}[2]{{#1} \otimes {#2}}


\begin{document}

\section{Type Structure}

In this section, I will describe the type structure of the 
programming language I will use for this thesis. The language is a
pure, total, predicatively polymorphic programming language (with
quantification over higher kinds), augmented with a monadic type
constructor that permits nontermination, higher-order state, and
continuations. 

\todo{Add support for polynomial data types}

\begin{displaymath}
  \begin{array}{lcll}
    \mbox{Kinds} & 
      \kappa & ::= & \star \bnfalt \kappa \to \kappa 
    \\[1em]
    \mbox{Monotypes} & 
      \tau & ::= & 
         \unittype \bnfalt 
         \tau \times \tau \bnfalt 
         \tau \to \tau \bnfalt 
         \tau + \tau \bnfalt
         \N \bnfalt 
         \reftype{A} \bnfalt
         \monad{\tau} \bnfalt \\
     &&& \alpha \bnfalt
         \tau\;\tau \bnfalt 
         \fun{\alpha}{\kappa}{\tau} 
    \\[1em]
    \mbox{Polytypes} & 
      A & ::= & 
         \unittype \bnfalt 
         A \times B \bnfalt 
         A + B \bnfalt
         A \to B \bnfalt 
         \N \bnfalt 
         \reftype{A} \bnfalt
         \monad{A} \\
    &&&  \alpha \bnfalt
         \tau\;\tau \bnfalt \\
    &&&  \forall \alpha:\kappa.\; A \bnfalt 
         \exists \alpha:\kappa.\; A \\[1em]
    \mbox{Type Contexts} & 
      \Theta & ::= & \cdot \bnfalt \Theta, \alpha:\kappa \\
  \end{array}
\end{displaymath}

The basic kind structure of the language is given with the kinds
$\kappa$. The The kind $\star$ is the kind of monotypes, which are
written with the letter $\tau$. These are the unit type $\unittype$,
pair types $\tau \times \sigma$, sums $\tau + \sigma$, function space
$\tau \to \sigma$, natural numbers $\N$, computation types $\monad{\tau}$, 
and finally (ML-style) references $\reftype{A}$. Also, within open types
we may also use type variables $\alpha$ to refer to types, and we can
also define lambda-abstractions and applications to inhabit the higher
kinds of this language. 

The type $\reftype{A}$ is not merely a pointer to a value of
monomorphic type; instead, it also permits storing a pointer to a
value of polymorphic types $A$. The intuition justifying our
liberality is that a reference to a value of polymorphic type is
itself merely a location, so it is permissible to treat pointers as
monomorphic objects. Indeed, the denotational semantics of 
references will be built upon precisely this intuition. 

The polytypes $A$ themselves extend the monotypes with universal
quantification $\forall \alpha:\kappa.\;A$ as well as existential
types $\exists \alpha:\kappa.\;A$. Each of the simple type
constructors -- sums, products, functions, computations -- also may
contain polymorphic types as subexpressions within it. Though it may
not look like it at first glance, this is actually only a fairly modest
generalization of the type schemes of ML. Because the universal and
existential quantifiers range over the kinds $\kappa$, it is
impossible to instantiate them with a polytype, thereby limiting us to
predicative polymorphism. 

The kinding judgements $\judgeWK{\tau}{\kappa}$ and
$\judgeWK{A}{\bigstar}$ determine whether a monotype or polytype
(respectively). Notice that we use two judgements, because we do not
give polytypes their own first-class kind $\bigstar$ -- this means
that we can syntactically enforce the restriction that the context
$\Theta$, which is a collection of type variables and their kinds, 
can only range over kinds which can contain only monotypes. 

First, we give the rules for the monotypes. 

\begin{mathpar}
\inferrule*[right=KUnit]
          { }
          {\judgeWK{\unittype}{\star}}
\and
\inferrule*[right=KNat]
          { }
          {\judgeWK{\N}{\star}}
\and
\inferrule*[right=KProd]
          {\judgeWK{\tau}{\star} \\
           \judgeWK{\sigma}{\star}}
          {\judgeWK{\tau \times \sigma}{\star}}
\and
\inferrule*[right=KSum]
          {\judgeWK{\tau}{\star} \\
           \judgeWK{\sigma}{\star}}
          {\judgeWK{\tau + \sigma}{\star}}
\and
\inferrule*[right=KArrow]
          {\judgeWK{\tau}{\star} \\
           \judgeWK{\sigma}{\star}}
          {\judgeWK{\tau \to \sigma}{\star}}
\and
\inferrule*[right=KComp]
          {\judgeWK{\tau}{\star}}
          {\judgeWK{\monad{\tau}}{\star}}
\and
\inferrule*[right=KRef]
          {\judgeWK{A}{\bigstar}}
          {\judgeWK{\reftype{A}}{\star}}
\\
\inferrule*[right=KHyp]
          { \alpha:\kappa \in \Theta }
          { \judgeWK{\alpha}{\kappa} }
\and
\inferrule*[right=KApp]
          { \judgeWK{\tau}{\kappa' \to \kappa} \\
            \judgeWK{\tau'}{\kappa'} }
          { \judgeWK{\tau\;\tau'}{\kappa} }
\and
\inferrule*[right=KLam]
          { \judgeWK[\Theta, \alpha:\kappa']{\tau}{\kappa} }
          { \judgeWK{\fun{\alpha}{\kappa'}{\tau}}{\kappa' \to \kappa} }
\end{mathpar}

Next, we can give the rules for polytypes. 

\begin{mathpar}
\inferrule*[right=KUnit]
          { }
          {\judgeWK{\unittype}{\bigstar}}
\and
\inferrule*[right=KNat]
          { }
          {\judgeWK{\N}{\bigstar}}
\and
\inferrule*[right=KProd]
          {\judgeWK{\tau}{\bigstar} \\
           \judgeWK{\sigma}{\bigstar}}
          {\judgeWK{\tau \times \sigma}{\bigstar}}
\and
\inferrule*[right=KSum]
          {\judgeWK{\tau}{\bigstar} \\
           \judgeWK{\sigma}{\bigstar}}
          {\judgeWK{\tau + \sigma}{\bigstar}}
\and
\inferrule*[right=KArrow]
          {\judgeWK{\tau}{\bigstar} \\
           \judgeWK{\sigma}{\bigstar}}
          {\judgeWK{\tau \to \sigma}{\bigstar}}
\and
\inferrule*[right=KComp]
          {\judgeWK{A}{\bigstar}}
          {\judgeWK{\monad{A}}{\bigstar}}
\and
\inferrule*[right=KRef]
          {\judgeWK{A}{\bigstar}}
          {\judgeWK{\reftype{A}}{\bigstar}}
\\
\inferrule*[right=KHyp]
          { \alpha:\star \in \Theta }
          { \judgeWK{\alpha}{\bigstar} }
\and
\inferrule*[right=KApp]
          { \judgeWK{\tau}{\kappa' \to \star} \\
            \judgeWK{\tau'}{\kappa'} }
          { \judgeWK{\tau\;\tau'}{\bigstar} }
\\
\inferrule*[right=KForall]
           { \judgeWK[\Theta, \alpha:\kappa]{A}{\bigstar} }
           { \judgeWK{\forall \alpha:\kappa.\;A}{\bigstar} }
\and
\inferrule*[right=KExists]
           { \judgeWK[\Theta, \alpha:\kappa]{A}{\bigstar} }
           { \judgeWK{\exists \alpha:\kappa.\;A}{\bigstar} }
\end{mathpar}


\end{document}
