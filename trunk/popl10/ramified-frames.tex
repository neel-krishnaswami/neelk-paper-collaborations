%-----------------------------------------------------------------------------
%
%               Template for LaTeX Class/Style File
%
% Name:         sigplanconf-template.tex
% Purpose:      A template for sigplanconf.cls, which is a LaTeX 2e class
%               file for SIGPLAN conference proceedings.
%
% Author:       Paul C. Anagnostopoulos
%               Windfall Software
%               978 371-2316
%               paul@windfall.com
%
% Created:      15 February 2005
%
%-----------------------------------------------------------------------------


\documentclass[preprint,natbib]{sigplanconf}

\usepackage{amsmath}
\usepackage{amssymb}
\usepackage{mathpartir}
\newcommand{\ctext}[1]{\mathsf{#1}}

\newcommand{\normalize}[1]{\ctext{nf}({#1})}

\newcommand{\unittype}{\mathbf{1}}
\newcommand{\reftype}[1]{\ctext{ref}\;#1}
\newcommand{\monad}[1]{\bigcirc{#1}}
\newcommand{\cont}[1]{\ctext{cont}\;{#1}}
\newcommand{\opttype}[1]{\ctext{option }{#1}}
\newcommand{\seqsort}[1]{\ctext{seq}\;{#1}}
\newcommand{\listtype}[1]{\ctext{list}\;{#1}}

\newcommand{\pair}[2]{\left<{#1},{#2}\right>}
\newcommand{\fst}[1]{\ctext{fst}\; #1}
\newcommand{\snd}[1]{\ctext{snd}\; #1}

\newcommand{\inj}[1]{\iota_{#1}}
\newcommand{\inl}[1]{\ctext{inl }#1}
\newcommand{\inr}[1]{\ctext{inr }#1}
\newcommand{\Case}[5]{\ctext{case}(#1,\; {#2}.\; #3,\; {#4}.\; #5)}
\newcommand{\listcase}[5]{\ctext{case}(#1,\;\ctext{Nil} \to {#2},\;
                                       \ctext{Cons}({#3},{#4}) \to {#5})}
\newcommand{\optcase}[4]{\ctext{case}(#1,\;
                                      \ctext{None} \to {#2},\;
                                      \ctext{Some}\;{#3} \to {#4})}
\newcommand{\z}{\ctext{z}}
\newcommand{\s}[1]{\ctext{s}(#1)}
\newcommand{\iter}[4]{\ctext{iter}(#1, {#2}, {#3}.\; {#4})}
\newcommand{\iterseq}[4]{\ctext{iter}_{\mathsf{seq}}(#1, {#2}, #3.\; {#4})}
\newcommand{\comp}[1]{[#1]}
\newcommand{\fun}[3]{\lambda #1:#2.\;#3}
\newcommand{\Fun}[3]{\Lambda #1:#2.\;#3}
\newcommand{\unit}{\left<\right>}
\newcommand{\pack}[2]{\ctext{pack}({#1}, {#2})}
\newcommand{\unpack}[4]{\ctext{unpack}({#1}, {#2}) = {#3} \;\ctext{in}\;{#4}}
\newcommand{\alt}{\;|\;}
\newcommand{\letv}[3]{\ctext{letv}\;#1 = #2\;\ctext{in}\;#3}
\newcommand{\newref}[2]{\ctext{new}_{#1}(#2)}
\newcommand{\run}[1]{\ctext{run}\;{#1}}
\newcommand{\ok}{\ctext{ ok}}
\newcommand{\FV}[1]{\mathrm{FV}({#1})}

\newcommand{\fix}[2]{\ctext{fix}\;{#1}.\;{#2}}

\newcommand{\statecfg}[2]{\left<{#1};\;{#2}\right>}
\newcommand{\eval}[4]{\left<{#1};\;{#2}\right> \leadsto \left<{#3};\;{#4}\right>}
\newcommand{\evalabort}[2]{\left<{#1};\;{#2}\right> \leadsto \mathbf{abort}}

\newcommand{\domain}[1]{\mbox{dom}({#1})}
\newcommand{\upset}[1]{\mathcal{P}^{\uparrow}({#1})}

\newcommand{\pointsto}{\mapsto}
\newcommand{\disj}{\vee}
\renewcommand{\implies}{\supset}
\newcommand{\wand}{-\!\!*\,}
\newcommand{\emp}{\mathsf{emp}}
\newcommand{\validprop}[1]{{#1}\;\ctext{valid}}

\newcommand{\todo}[1]{\texttt{[TODO: {#1}]}}

\newcommand{\setof}[1]{\{{#1}\}}

\newcommand{\To}{\Rightarrow}
\newcommand{\From}{\Leftarrow}

\newcommand{\N}{\mathbb{N}}

\newcommand{\assert}{\ctext{prop}}

\newcommand{\bigstep}{\Downarrow}

\newcommand{\bigeval}[4]{\left<{#1};\;{#2}\right> \bigstep \left<{#3};\;{#4}\right>}
\newcommand{\bigevalabort}[2]{\left<{#1};\;{#2}\right> \bigstep \mathbf{abort}}

\newcommand{\spec}[4]{\{{#1}\}{#2}\{{#3}.\;{#4}\}}
\newcommand{\specX}[3]{\{{#1}\}{#2}\{{#3}\}}
\newcommand{\mspec}[4]{\langle{#1}\rangle{#2}\langle{#3}.\;{#4}\rangle}
\newcommand{\bnfalt}{\;\;|\;\;}

\newcommand{\specor}{\;||\;}
\newcommand{\specand}{\;\&\;}
\newcommand{\specimp}{\Rightarrow\!\!\!>}
% \newcommand{\specimp}{\Longrightarrow}
% \newcommand{\specor}{\ctext{ or }}
% \newcommand{\specand}{\ctext{ and }}
% \newcommand{\specimp}{\ctext{ implies }}
\newcommand{\spectype}{\ctext{spec}}
\newcommand{\valid}{\ctext{ valid}}

\newcommand{\interp}[1]{[\![{#1}]\!]}
\newcommand{\interpE}[1]{\interp{#1}^e}
\newcommand{\interpC}[1]{\interp{#1}^c}

\newcommand{\interpF}[1]{[\![{#1}]\!]_f}
\newcommand{\interpmono}[1]{\interp{#1}^{\mathrm{m}}}

\newcommand{\entails}{\models}

\newcommand{\judgeE}[4][\Theta]{{#1};\;{#2} \vdash {#3} : {#4}}
\newcommand{\judgeC}[4][\Theta]{{#1};\;{#2} \vdash {#3} \div {#4}}
\newcommand{\judgeEq}[5][\Theta]{{#1};\;{#2} \vdash {#3} \equiv {#4} : {#5}}
\newcommand{\judgeEqC}[5][\Theta]{{#1};{#2} \vdash {#3} \equiv {#4} \div {#5}}




%% semantic operations

\newcommand{\worldleq}{\preceq}
\newcommand{\worldgeq}{\succeq}

\newcommand{\semfun}[2]{\lambda #1.\;#2}
\newcommand{\sempair}[2]{\left({#1}, {#2}\right)}
\newcommand{\powerset}[1]{\mathcal{P}(#1)}
\newcommand{\powersetfin}[1]{\mathcal{P}^{\mathrm{fin}}(#1)}

\newcommand{\paircat}[2]{\left<{#1};{#2}\right>}
\newcommand{\sumcat}[2]{\left[{#1};{#2}\right]}
\newcommand{\abscat}[1]{\lambda({#1})}


\newcommand{\judgeP}[3]{{#1} \vdash {#2} : {#3}}
\newcommand{\judgeS}[2][\Delta]{{#1} \vartriangleright {#2} : \spectype}

\newcommand{\judgeSCtx}[2]{{#1} \vartriangleright {#2} : \ctext{context}}

\newcommand{\judgeEqP}[4]{{#1} \vdash {#2} \equiv {#3} : {#4}}


\newcommand{\judgeWK}[3][\Theta]{{#1} \vdash {#2} : {#3}}
\newcommand{\judgeKeq}[4][\Theta]{{#1} \vdash {#2} \equiv {#3} : {#4}}

\newcommand{\entailsP}[3]{{#1} \vartriangleright {#2} \vdash {#3}}
\newcommand{\entailsS}[3]{{#1}; {#2} \vdash {#3} \ok}

\newcommand{\chartp}{\ctext{char}}
\newcommand{\fonttp}{\ctext{font}}

\newcommand{\LOC}{loc}
\newcommand{\MONO}{\mathbf{mono}}
\newcommand{\HEAP}{heap}
\newcommand{\PROP}{Prop}
\newcommand{\TRUE}{True}
\newcommand{\HPROP}{Prop}
\newcommand{\TYPE}{Type}

\newtheorem{prop}{Proposition}
\newtheorem{lemma}{Lemma}

\newcommand{\comprehend}[2]{\setof{{#1}\;|\;{#2}}}


% \newenvironment{proof}{\begin{comment}}{\end{comment}}

\newenvironment{proof}{\noindent\textbf{Proof.}}{\noindent\ensuremath{\Box}}

\newcounter{prooflinenum}
\newenvironment{tabbedproof}
   {\setcounter{prooflinenum}{0}
    \begin{tabbing}\;\;\=\;\;\;\;\;\=\;\;\;\;\=\;\;\;\;\=\;\;\;\;\=\;\;\;\;\=\;\;\;\;\=\;\;\;\;\=\;\;\;\;\=\\[-2em]}
   {\end{tabbing}}

\newcommand{\oo}{\addtocounter{prooflinenum}{1}\arabic{prooflinenum}\>\>}
\newcommand{\ooo}{\addtocounter{prooflinenum}{1}\arabic{prooflinenum}\>\>\>}
\newcommand{\oooo}{\addtocounter{prooflinenum}{1}\arabic{prooflinenum}\>\>\>\>}
\newcommand{\ooooo}{\addtocounter{prooflinenum}{1}\arabic{prooflinenum}\>\>\>\>\>}
\newcommand{\oooooo}{\addtocounter{prooflinenum}{1}\arabic{prooflinenum}\>\>\>\>\>\>}
\newcommand{\ooooooo}{\addtocounter{prooflinenum}{1}\arabic{prooflinenum}\>\>\>\>\>\>\>}
\newcommand{\oooooooo}{\addtocounter{prooflinenum}{1}\arabic{prooflinenum}\>\>\>\>\>\>\>\>}
\newcommand{\ooooooooo}{\addtocounter{prooflinenum}{1}\arabic{prooflinenum}\>\>\>\>\>\>\>\>\>}

\newcommand{\ox}{\>\>}
\newcommand{\oox}{\>\>\>}
\newcommand{\ooox}{\>\>\>\>}
\newcommand{\oooox}{\>\>\>\>\>}
\newcommand{\ooooox}{\>\>\>\>\>\>}
\newcommand{\oooooox}{\>\>\>\>\>\>\>}
\newcommand{\ooooooox}{\>\>\>\>\>\>\>\>}
\newcommand{\oooooooox}{\>\>\>\>\>\>\>\>\>}


\newenvironment{eqnproof}[1][]{${#1}$\begin{displaymath}\begin{array}{lcll}}
                         {\end{array}\end{displaymath}}

\newcommand{\eline}[3][]{{#1} & = & {#2} & \mbox{{#3}} \\}

\newcommand{\elines}[3][]{{#1} & = & \begin{array}{l} #2 \end{array} & \mbox{{#3}} \\}

\newcommand{\eclaim}[3][]{{#1} &  & {#2} & \mbox{{#3}} \\}
\newcommand{\efact}[2]{{#1} & & & \mbox{#2} \\}

\newcommand{\basicspec}[4]{[{#1}]\;{#2}\;[{#3}.\;{#4}]}

% Macros for type checking assertions

\newcommand{\pfun}[3]{\hat{\lambda} #1:#2.\;#3}
\newcommand{\restrictkind}[1]{({#1})\Downarrow_K}
\newcommand{\restricttype}[1]{({#1})\Downarrow_T}

\newcommand{\restricttyenv}[2]{{#2}\Downarrow^{#1}_K}
\newcommand{\restrictvals}[2]{{#2}\Downarrow^{#1}_T}
\newcommand{\judgeACtx}[1]{\vartriangleright {#1} \;\ctext{ok}}
\newcommand{\judgeA}[3][\Delta]{{#1} \vartriangleright {#2} : {#3}}
\newcommand{\judgeSort}[2][\Delta]{\judgeA[{#1}]{{#2}}{\ctext{sort}}}
\newcommand{\judgeSortEq}[3][\Delta]{{#1} \vartriangleright {#2} \equiv {#3} : \ctext{sort}}

\newcommand{\ms}[1]{\mathcal{#1}}
\newcommand{\Frame}[2]{{#1} \otimes {#2}}


\begin{document}

\conferenceinfo{POPL '10}{January 20-22, Madrid.} 
\copyrightyear{2010} 
\copyrightdata{[to be supplied]} 

\titlebanner{preprint}        % These are ignored unless
\preprintfooter{short description of paper}   % 'preprint' option specified.

\title{Verifying Event-Driven Programs using Ramified Frame Properties}
% \subtitle{Subtitle Text, if any}

\authorinfo{Neelakantan R. Krishnaswami}
           {Carnegie Mellon University}
           {neelk@cs.cmu.edu}
\authorinfo{Lars Birkedal}
           {IT University of Copenhagen}
           {birkedal@itu.dk}
\authorinfo{Jonathan Aldrich}
           {Carnegie Mellon University}
           {jonathan.aldrich@cs.cmu.edu}
% \authorinfo{John C. Reynolds}
%            {Carnegie Mellon University}
%            {jcr@cs.cmu.edu}
\maketitle

\begin{abstract}
Interactive programs, such as GUIs or spreadhseets, often maintain
dependency information over dynamically-created networks of objects.
By ``dependency information'', we mean that each imperative object
tracks not only the objects its own invariant depends on, but also all
of the objects which depend upon it.

The bidirectional linkages in this style pose a challenge to
verification, because their correctness relies upon a global invariant
over the object graph --- we cannot simply track the footprint of an
object and rely on (e.g.) the frame rule of separation logic to ignore
the rest of the heap. As a result, it seems difficult to verify
different parts of a dependency network separately and combine their
correctness proofs.

In this paper, we show how to \emph{modularly} verify programs written
using dynamically-generated bidirectional dependency information. The
critical idea is to distinguish between the footprint of a command,
and the state whose invariants depends upon the footprint. To do so,
we define an application-specific semantics of updates, and introduce
the concept of a \emph{ramification operator} to explain how a local
changes can alter our knowledge of the rest of the heap.

To illustrate the utility of this style of proof, we verify an
imperative implementation of combinators implementing stream
transducers in the style of functional reactive programming. Our
specification allows clients to reason about the behavior of the
imperative implementation as if it were purely functional, even though
it is implemented using state and imperative callback procedures.
\end{abstract}

\category{CR-number}{subcategory}{third-level}

\terms
Functional reactive programming, subject-observer, model-view-controller,
separation logic, frame properties, dataflow

\keywords
keyword1, keyword2

\section{Introduction}

In many interactive programs, there are mutable data structures which
change over time, and which must maintain some relationships with one
another. For example, in a web browser, we need to present a web page
both as a tree data structure for scripts to manipulate, and at the
display a graphical image for the human user to view. Any change made
to the tree by a script must be reflected in a change to the image
that the human sees --- the two structures must remain synchronized.

Likewise, in a spreadsheet, each cell contains a formula, which may
refer to other cells, and whenever the user changes a cell, all of the
cells which transitively depend upon it must be updated. Since
spreadsheets can get very large, this should ideally be done in a lazy
way, so that only the cells visible on the screen, and the cells
necessary to compute them, are themselves recomputed.

Typically, these dependencies are written using what is called the
\emph{subject-observer} pattern. A mutable data structure (the
subject) maintains a list of all of the data structures whose
invariants depend upon it (the observers), and whenever it changes, it
calls a function on each of those observers to update them in response
to the change. (And in turn, the observers of the subject may be
subjects of still other observers, ultimately forming DAGs of
notifications.)

While natural, these programs are very challenging to verify in a
modular way, even when using a resource-sensitive logic adapted to
reasoning about aliased mutable data, such as separation logic. The
reason is that there are two directions of dependency, both of which
matter for program proof. First, our program invariant must have
ownership over the subject's data (its \emph{footprint}) in order to
prove the correctness of code modifying the subject. This direction
of ownership is natural to verify with separation logic. 

However, we explicitly maintain the \emph{other} direction of
dependency as well --- we must track everything which depends upon the
subject, and modify them appropriately whenever the subject changes.
However, the natural program invariant now becomes a global property:
we need to know the full dependency graph covering all subjects and
observers, so that we can say that the reads and is-read-by relations
are relational transposes of one another. The global nature of this
invariant means that a naive correctness proof will not respect the
modular structure of the program --- if we modify the dependency graph
in any way, we now have to re-verify the entire program!

However, the the intention of the subject-observer pattern is
precisely to allow the program to remain oblivious to the exact number
and nature of the observers, which allows the programmer to add new
observers without disturbing the behavior of the rest of the program.
Our goal, then, is to find a way of taking this piece of practical
software engineering wisdom, and casting it into formal terms amenable
to proof.

Concretely, our contributions are as follows: 

\begin{itemize}
  \item We define a library with a monadic API for writing
    demand-driven computations with dynamic dependencies and local
    state, and which is implemented as higher-order functions
    dynamically creating networks of imperative callbacks

    We then give an ``application-specific semantics'' for this
    library, structured as a set of separation logic lemmas about our
    dataflow library. These lemmas permit \emph{modular} correctness
    proofs about programs using this API, even in the face of the fact
    that the program invariants must be defined globally upon the
    whole callback network.

    The key idea is to distinguish between the direct footprint of a
    command, and the program state which can depends upon that
    footprint. The lemmas are then phrased so that they refer only to
    the direct footprint of each command in the API. In addition, we
    structure our lemmas to justify an unusual frame property for our
    abstract semantics, which we can use to verify different parts of
    an imperative dataflow network separately.

    Unlike typical frame properties, the frame in our frame rule is
    not the same in the pre- and the post-states. Instead, the two
    sides of the frame are related by a \emph{ramification operator}
    (so named in analogy to the ``ramification problem'' in AI), which
    explains how local changes can alter our knowledge of the rest of
    the heap.

  \item To illustrate the utility of this proof technique, we verify
    an imperative implementation of combinators implementing stream
    transducers in the style of functional reactive programming. We 
    can write programs which build up 

    Ultimately, clients can reason about the behavior of the imperative
    implementation ``as if'' it were purely functional, even though it
    is implemented using local state and imperative callback
    procedures.
\end{itemize}

\section{Programming Language and Logic}

The formal system we present has three layers. First, we have a core
programming language we call Idealized ML. It is a
predicatively-polymorphic functional language which isolates all side
effects inside a monadic type~\cite{pfenning-davies}. Our notion of
side effects includes nontermination in addition to the allocation,
access, and modification of general references (including pointers to
closures).  Then, we give an assertion language based on higher-order
separation logic~\cite{hosl} to describe the state of a
heap. Separation logic allows us to give a clean treatment of issues
related to specifying and controlling aliasing, and higher-order
predicates allow us to abstract over the heap, hiding the exact layout
of a module's heap data structures and thereby enforcing
encapsulation.  Finally, we have a specification logic to describe the
effects of programs, which is a first-order logic whose atomic
propositions are Hoare triples $\spec{p}{c}{a:A}{q}$, which assert
that if the heap is in a state described by the assertion $p$, then
executing the command $c$ will result in a postcondition state $q$
(with the return value of the command bound to $a$).


\textbf{Programming Language.} The core programming language we have
formalized is an extension of the polymorphic lambda calculus with a
monadic type constructor to represent side-effecting computations.
The types of our language are the unit type $1$, the function space $A
\to B$, the natural number type $\N$, the reference type
$\reftype{A}$, as well as universal and existential types $\forall
\alpha:\kappa.\;A$ and $\exists \alpha:\kappa.\;A$.\footnote{These
  quantifiers are actually all restricted to \emph{predicative}
  quantification (ie, they can only be instantiated with terms
  lacking any quantifiers themselves) in order to keep the
  denotational semantics simple, though recent
  work~\cite{birkedal-impred} has studied how to combine store with
  impredicative polymorphism.}

In addition, we have the monadic type $\monad{A}$, which is the type
of suspended side-effecting computations producing values of type
$A$. Side effects include both heap effects (such as reading, writing,
or allocating a reference) and nontermination.

We maintain such a strong distinction between pure and impure code for
two reasons. First, it allows us to use strong equational
reasoning principles for our language: we can validate the full
$\beta$ and $\eta$ rules of the lambda calculus for each of the pure
types. This simplifies reasoning even about imperative programs,
because we can relatively freely restructure the program source to
follow the logical structure of a proof. Second, when program
expressions appear in assertions --- that is, the pre- and
post-conditions of Hoare triples --- they must be pure. However,
allowing a rich set of program expressions like function calls or
arithmetic in assertions makes it much easier to write
specifications. So we restrict which types can contain side-effects,
and thereby satisfy both requirements.

The pure terms of the language are typed with a typing judgment
$\judgeE{\Gamma}{e}{A}$, which can be read as ``In the type context
$\Theta$ and the variable context $\Gamma$, the pure expression $e$
has type $A$.'' Computations are typed with the judgment
$\judgeC{\Gamma}{c}{A}$, which can be read as ``In the type context
$\Theta$ and the variable context $\Gamma$, the computation $c$ is
well-typed at type $A$ .'' The rules for both of these judgements are
standard and omitted.

We have $\unit$ as the inhabitant of $1$, natural numbers $\z$ and
$\s{e}$, and functions $\fun{x}{A}{e}$. We also have the corresponding
eliminations for each type, including projections for products and
case statements for sum types. For the natural numbers, we add a
primitive recursion construct $\iter{e}{e_z}{x}{e_s}$. If $e = \z$,
this computes $e_z$ , and if $e = \s{e'}$, it computes
$e_s[(\iter{e'}{e_z}{x}{e_s})/x]$. This bounded iteration allows us
to implement (for example) arithmetic operations as pure expressions.
We will also freely make use of other polymonial data types (such as
lists, option types, and trees) as needed. 

Suspended computations $\comp{c}$ inhabit the monadic type
$\monad{A}$.  These computations are not immediately evaluated, which
allows us to embed them into the pure part of the programming
language. Furthermore, we can take fixed points of elements of pointed
domains, which gives us a general recursion facility. (We must
restrict $\ctext{fix}$ to these domains because they are the only ones
which admit nontermination. Also, we will write recursive functions as
syntactic sugar for $\ctext{fix}$.)  

The computations themselves include all expressions $e$, as
computations that coincidentally have no side-effects. Furthermore, we
have sequential composition $\letv{x}{e}{c}$. Intuitively, the
behavior of this command is as follows. We evaluate $e$ until we get
some $\comp{c'}$, and then evaluate $c'$, modifying the heap and
binding its return value to $x$. Then, in this augmented environment,
we run $c$. The fact that monadic commands have return values explains
why our sequential composition is also a binding construct. Finally,
we have primitive computations $\newref{A}{(e)}$, $!e$, and $e := e'$,
which let us allocate, read and write references (inhabiting type
$\reftype{A}$), respectively. To save space, we will also write
$\ctext{run}\;e$, when $e$ is a term of monadic type, as an
abbreviation for $\letv{x}{e}{x}$.

This language has been given a typed denotational semantics, which we
do not give here both for space reasons, and because it is not central
to the contributions of this paper . The details of the semantics
(including the assertion and specification levels) can be found in the
first author's forthcoming Ph.D. thesis~\cite{tech-report}.

\begin{figure}
\begin{displaymath}
  \begin{array}{lcll}
    \mbox{Kinds} & 
      \kappa & ::= & \star \bnfalt \kappa \to \kappa 
    \\[1em]
     \mbox{Monotypes} & 
      \tau & ::= & 
         \unittype \bnfalt 
         \tau \times \tau \bnfalt 
         \tau \to \tau \bnfalt 
         \tau + \tau \\
     &&& \N \bnfalt 
         \reftype{A} \bnfalt
         \monad{\tau} \bnfalt \\
     &&& \alpha \bnfalt
         \tau\;\tau \bnfalt 
         \fun{\alpha}{\kappa}{\tau} 
    \\[1em]
    \mbox{Polytypes} & 
      A & ::= & 
         \unittype \bnfalt 
         A \times B \bnfalt 
         A \to B \bnfalt
         A + B \\
    &&&  \N \bnfalt 
         \reftype{A} \bnfalt
         \monad{A} \bnfalt \\
    &&&  \alpha \bnfalt
         \tau\;\tau \bnfalt \\
    &&&  \forall \alpha:\kappa.\; A \bnfalt 
         \exists \alpha:\kappa.\; A \\[1em]
    \mbox{Type Contexts} & 
      \Theta & ::= & \cdot \bnfalt \Theta, \alpha:\kappa \\
  \end{array}
\end{displaymath}
\caption{Language Types}
\label{type-syntax}
\end{figure}


\begin{figure}
\begin{displaymath}
  \begin{array}{llcl}
    \mbox{Pure expressions} & 
     e & ::= & 
         \unit \bnfalt
         \pair{e}{e'} \bnfalt
         \fst{e} \bnfalt
         \snd{e} 
\\
     &&|& \inl{e} \bnfalt
          \inr{e}  \\
     &&|& \Case{e_0}{x_1}{e_1}{x_2}{e_2} 
\\
     &&|& \z \bnfalt 
          \s{e} \bnfalt 
          \iter{e}{e_0}{x}{e_1}
\\ 
     &&|& x \bnfalt \fun{x}{A}{e} \bnfalt e\;e' 
\\ 
     &&|& \Fun{\alpha}{\kappa}{e} \bnfalt e\;\tau 
\\ 
     &&|& \pack{\tau}{e} \bnfalt \unpack{\alpha}{x}{e}{e'} 
\\
     &&|& \comp{c} \bnfalt \fix{x:D}{e}
\\[1em]
  \mbox{Computations} & 
    c & ::= & e \bnfalt \letv{x}{e}{c} \\
   &  &  |  & \newref{A}{e} \bnfalt !e \bnfalt e := e'
\\[1em]
  \mbox{Contexts} & 
    \Gamma & ::= & \cdot \bnfalt \Gamma, x:A 
\\[1em]
  \mbox{Pointed Types} & 
     D & ::= & \unittype \bnfalt \monad{A} \bnfalt D \times D \bnfalt A \to D  \\
    &  &  |  & \forall \alpha:\kappa.\; D 
\\[1em] 
  \end{array}
\end{displaymath}
\caption{Syntax of the Programming Language}
\label{lang-syntax}
\end{figure}


\textbf{Assertion Language.} The sorts and syntax of the assertion
language are given in Figure~\ref{assert-syntax}. The assertion
language is a version of separation logic, extended to higher order.


In ordinary Hoare logic, a predicate describes a set of program states
(in our case, heaps), and a conjunction like $p \land q$ means that a
heap in $p \land q$ is in the set described by $p$ and the described
by $q$. While this is a natural approach, aliasing can become quite
difficult to treat --- if $x$ and $y$ are pointer variables, we need to
explicitly state whether they alias or not. This means that as the
number of variables in a program grows, the number of aliasing
conditions grows quadratically. 

With separation logic, we add the \emph{spatial} connectives to
address this difficulty. A separating conjunction $p * q$ means that
the state can be broken into two \emph{disjoint} parts, one of which
is in the state described by $p$, and the other of which is in the
state described by $q$. The disjointness property makes the
noninterference of $p$ and $q$ implicit. This avoids the unwanted
quadratic growth in the size of our assertions. In addition to the
separating conjunction, we have its unit $\emp$, which is true of the
empty heap, and the points-to relation $e \pointsto e'$, which holds
of the one-element heap in which the reference $e$ has contents
$e'$. A heap is described by the ``magic wand'' $p \wand q$, when we
can merge it with any disjoint heap described by $p$, and the
combination is described by $q$.


The universal and existential quantifiers $\forall x:\omega.\;p$ and
$\exists x:\omega.\;p$ are higher-order quantifiers ranging over all
sorts $\omega$. The sorts include the language types $A$, the sort of
propositions $\assert$, and function spaces over sorts $\omega \To
\omega'$. Constructors for terms of all these sorts in the syntax
given in Figure~\ref{assert-syntax}. For the function space, we
include lambda-abstraction and application. Because our assertion
language contains within it the classical higher-order logic of sets,
we will freely make use of features like subsets, indexed sums, and
indexed products, exploiting their definability.

Finally, we include the atomic formulas $\validprop{S}$, which are
\emph{assertions} that a \emph{specification} $S$ holds. This facility
is useful when we write assertions about pointers to code --- for
example, the assertion $r \pointsto e$ $\land$
$(\mspec{p}{e}{a:A}{q})$ $\valid$ says that the reference $r$ points
to a monadic expression $e$, whose behavior is described by the Hoare
triple $\mspec{p}{e}{a:A}{q}$.


\begin{figure}
\begin{displaymath}
\begin{array}{llcl}
\mbox{Assertion Sorts} & 
\omega & ::= & A \bnfalt \omega \To \omega \bnfalt \assert 
\\[0.5em]
\mbox{Assertion} & 
p & ::= & e \bnfalt x \bnfalt \fun{x}{\omega}{p} \bnfalt p\;q \\
\mbox{Constructors}
& &  |  & \top \bnfalt p \land q \bnfalt p \implies q 
          \bnfalt \bot \bnfalt p \vee q \\
& &  |  &  \emp \bnfalt p * q \bnfalt p \wand q \bnfalt e \pointsto e' \\
& &  |  & \forall x:\omega.\; p \bnfalt \exists x:\omega.\; p \bnfalt
          \validprop{S} 
\\[0.5em]
\mbox{Specifications} &
S & ::= & \spec{p}{c}{a:A}{q} \bnfalt \mspec{p}{e}{a:A}{q} \bnfalt \setof{p} \\
& &  |  & S \specand S' \bnfalt S \specimp S' \bnfalt S \specor S' \\
& &  |  & \forall x:\omega.\; S \bnfalt \exists x:\omega.\;S 
\\
\end{array}
\end{displaymath}
\caption{Syntax of Assertions and Specifications}
\label{assert-syntax}
\end{figure}

\textbf{Specification Language.} Given programs and assertions about
the heap, we need specifications to relate the two. We begin with the
Hoare triple $\spec{p}{c}{a:A}{q}$. This specification represents the
claim that if we run the computation $c$ in any heap the predicate $p$
describes, then if $c$ terminates, it will end in a heap described by
the predicate $q$. Since monadic computations can return a value in
addition to having side-effects, we add the binder $a:A$ to the third
clause of the triple to let us name and use the return value in the
postcondition.

We then treat Hoare triples as one of the atomic proposition forms of
a first-order intuitionistic logic (see
Figure~\ref{assert-syntax}). The other form of atomic proposition are
the specifications $\setof{p}$, which are $specifications$ saying that
an \emph{assertion} $p$ is true. These formulas are useful for
expressing aliasing relations between defined predicates, without
necessarily revealing the implementations. In addition, we can form
specifications with conjunction, disjunction, implication, and
universal and existential quantification over the sorts of the
assertion language. 

Having a full logic of triples also lets us express program modules as
formulas of the specification logic. We can expose a module to a
client as a collection of existentially quantified functions
variables, and provide the client with Hoare triples describing the
behavior of those functions. Furthermore, modules can existentially
quantify over predicates to grant client programs access to module
state without revealing the actual implementation. A client program
that uses an existentially quantified specification cannot depend on
the concrete implementation of this module, since the existential
quantifier hides that from it --- for example, we can expose a
$table(t, map)$ predicate that does not reveal whether a hash table is
implemented with single or double hashing.

\section{Demand-Driven Notification Network}

A simple intuition for a ``demand-driven notification network'' is to
think of it as a generalized spreadsheet. We have a collection of
cells, each of which contain program expressions whose evaluation may
refer to other cells. When a cell is read, the expression within the
cell is evaluated, possibly triggering the evaluation of other cells
in a cascade. Furthermore, each cell memoizes its expression, so that
repeated reads won't trigger re-evaluation, and maintains a set of
dependencies so that when the code within a cell is changed, it and
everything that depends on it invalidates the memoized value.

\subsection{Implementing Notification Networks}

\begin{figure}
\begin{tabbing}
$\codetype{A} = \monad{(A \times \cellset)}$ \\[1em]
$\celltype{A} = \{$\=$code: \reftype{\codetype{A}};$ \\
                   \>$value: \reftype{\opttype{A}};$ \\
                   \>$reads: \reftype{\cellset};$ \\
                   \>$obs:   \reftype{\cellset}\;$ \\
                   \>$id:    \N\}$ \\[1em]

$ecell = \exists \alpha:\star.\; \celltype{\alpha}$ \\[1em]

$\ctext{unit} : \forall \alpha:\star.\; \alpha \to \codetype{\alpha}$ \\
$\ctext{unit}\;x = \comp{\pair{x}{\ctext{emptyset}}}$ \\[1em]

$\ctext{bind} : \forall \alpha,\beta:\star.\; \codetype{\alpha} \to (\alpha \to \codetype{\beta}) \to \codetype{\beta}$ \\
$\ctext{bind}\;e\;f = [$\=$\letv{(v,r_1)}{e}{}$ \\
                        \>$\letv{(v',r_2)}{f\;v}{}$ \\
                        \>$\;\pair{v'}{\ctext{union}\;r_1\;r_2}]$\\[1em]

$\ctext{read} : \forall \alpha:\star.\; \celltype{\alpha} \to \codetype{\alpha}$ \\
$\ctext{read}\;a = [$\=$\letv{o}{\comp{!a.value}}{}$ \\
                     \>$\ctext{run}\;\ctext{case}(o,$ \\
                     \>\qquad\= $\ctext{Some}\;v \to \comp{\pair{v}{\ctext{singleton}\;a}},$ \\
                     \>      \> $\ctext{None} \to [$\=$\letv{exp}{\comp{!a.code}}{}$ \\
                     \>      \>                     \>$\letv{(v,r)}{exp}{}$ \\
                     \>      \>                     \>$\letv{\_}{\comp{a.value := v}}{}$\\
                     \>      \>                     \>$\letv{\_}{\comp{a.reads := r}}{}$ \\
                     \>      \>                     \>$\letv{\_}{\ctext{setiter}\;(\ctext{add\_observer}\;a)\;r}{}$ \\
                     \>      \>                     \>$\;\pair{v}{\ctext{singleton}\;a}])$ \\[1em]

$\getref : \forall \alpha : \reftype{\alpha} \to \codetype{\alpha}$ \\
$\getref r = \comp{\letv{v}{\comp{!r}}{\pair{v}{\ctext{emptyset}}}}$ \\[1em]

$\setref : \forall \alpha : \reftype{\alpha} \to \alpha \to \codetype{\unittype}$ \\
$\setref r\;v = \comp{\letv{dummy}{\comp{r := v}}{\pair{\unit}{\ctext{emptyset}}}}$ \\[1em]

$\ctext{newcell} : \forall \alpha:\star.\; \codetype{\alpha} \to \monad{\celltype{\alpha}}$ \\
$\ctext{newcell}\;\alpha\;code = [$\=$\letv{id}{!counter}{}$ \\
                                   \>$\letv{\_}{\comp{counter := n + 1}}{}$ \\
                                   \>$\letv{code}{\newref{\codetype{\alpha}}{code}}{}$ \\
                                   \>$\letv{value}{\newref{\opttype{\alpha}}{\ctext{None}}}{}$ \\
                                   \>$\letv{reads}{\newref{\cellset}{\ctext{emptyset}}}{}$ \\
                                   \>$\letv{obs}{\newref{\cellset}{\ctext{emptyset}}}{}$ \\
                                   \>$\; (code, value, reads, obs, id)]$ \\[1em]

$\ctext{update} : \forall \alpha:\star.\; \codetype{\alpha} \to \celltype{\alpha} \to \monad{\unittype}$\\
$\ctext{update}\;\alpha\;exp\;a = 
     [$\=$\letv{\_}{\ctext{invalidate}\;\pack{\alpha}{a}}{}$ \\
       \>$a.code := exp]$ \\[1em]

$\ctext{invalidate} : ecell \to \monad{\unittype}$ \\
$\ctext{invalidate}\;ecell =\; $\=$\ctext{unpack}(\alpha, a) = ecell\; \ctext{in}$\\
\>  $[$\=$\letv{os}{\comp{!a.obs}}{}$ \\
\>     \>$\letv{rs}{\comp{!a.reads}}{}$ \\
\>     \>$\letv{\_}{\ctext{iterset}\;\ctext{invalidate}\;os}{}$ \\
\>     \>$\letv{\_}{\ctext{iterset}\;(\ctext{remove\_obs}\;ecell)\;rs}{}$ \\
\>     \>$\letv{\_}{\comp{a.value := \ctext{None}}}{}$ \\
\>     \>$\letv{\_}{\comp{a.reads := \ctext{emptyset}}}{}$ \\
\>     \>$a.obs   := \ctext{emptyset}]$ 
\end{tabbing}
\caption{Implementation of Notification Networks}
\label{notification-implementation}
\end{figure}

Our API for creating notification networks is given in
Figure~\ref{notification-implementation}. First, we'll describe
the interface, and then discuss its implementation. 

The interface exposes two basic abstract data types, $\ctext{cell}$
and $\ctext{code}$.

The type $\celltype{A}$ is the type of dynamic data values. A cell
contains a reference to a piece of code, a possible memoized value,
plus enough information to correctly invalidate its memoized value
when the cell's dependencies change. We can create a new cell by
calling $\ctext{newcell}\;e$, which returns a brand new cell with the
code expression $e$ inside it. We can also modify a cell with 
the command $\ctext{update}\;cell\;exp$, which modifies the cell
$cell$ by installing the new expression $exp$ in it. 

The type $\codetype{A}$ is a monadic type, representing the type of
computations that can read cells. It supports the usual operations
$\return e$ and $\bind e\; (\semfun{x}{e'})$, which embed a pure value
into the $\ctext{code}$ type and implement sequential composition,
respectively. In addition, the primitive operations on this monad 
include reading a cell with the $\readcell cell$ function call, and 
reading and modifying local state with the $\getref r$ and $\setref r\;v$
operations. 
 
The actual implementation is also given in
Figure~\ref{notification-implementation}. The abstract type of code is
implemented using the underlying monad of imperative commands, so that
a $\codetype{A}$ is implemented with the type $\monad{(A \times
  \cellset)}$.  The intuition is that when we evaluate a term we are
allowed to read some cells along the way, and so must return a set of
all the cells that we read in order to do proper dependency
management, so $\cellset$ is a type representing sets of
(existentially-quantified) cells.  (The precise specification of
$\cellset$ is given in Appendix A, since describing it is a
distraction from the main development.)

Cells are represented with a 5-tuple. There is a reference pointing to
the code expression, as well as a value field pointing to an optional
value. The value field will be set to $\ctext{None}$ if the cell is in
an unready, un-memoized state, and will be $\ctext{Some}\;v$ if the
cell's code has been evalated to a value $v$. In addition there are
two fields representing the dependencies. If the code expression has
been evaluated and a memoized value generated, then the $reads$ field
will point to the set of cells that the computation directly read
while computing its value. Otherwise it will point to the empty
set. Conversely, the field $obs$ contains the cell's observers -- the
set of cells that have read the current cell as part of their own
computations. Obviously, this is only non-empty when the cell has been
evaluated. Finally, each cell also has an integer id field, which is a
unique numeric identifier for every cell that is created by the
library. It allows us to compare cells (even of different type) for
equality, which is needed for dependency management.

The $\return$ operation for the library simply returns its argument
value and the empty set, since it doesn't read any cells. Likewise,
$\bind e\;f$ will evaluate the first argument and pass the returned
value to the function $f$. It will return the function's return value,
together with the union of the two read sets. $\getref r$ and $\setref r\;v$
simply read an update the reference, and return empty read sets, since
neither of them read any cells. 

Interesting things first happen with the $\readcell e$ operation. This
function will first check to see if the cell has a memoized value. If
it does, we return that immediately. Otherwise, we evaluate the cell's
code, and update the current cell's value and read set. In addition,
each cell that was read in the evaluation of the code (i.e., the set
returned as the second component of the monadic type's return value)
also has its observer set updated with the current cell. Now, if any
of the dependencies change, they will be able to invalidate the
current cell, which depends upon them. Note that the dependencies
between cells are all dynamic -- we cannot examine the inside of a
code expression to find its ``free cells'', and so we rely upon the
invariant that a code expression will return every cell it read, in
addition to its return value.

Further interesting things happen with the $\ctext{newcell}\;cmd$
operation.  It creates a new cell value, setting the code field with
the argument $cmd$, and generating an id by dereferencing and
incrementing the variable $counter$. The $counter$ variable occurs
freely in this definition, because it is a piece of state global to
this whole module, and must be initialized by whatever initialization
routine first constructs this whole module as an existential package.
Since $counter$ is otherwise private, we can generate unique
identifiers by incrementing it as we create new cells. 

Finally, the $\ctext{update}\;cell\;cmd$ operation updates a cell
$cell$ with a new code expression $cmd$. (As an aside, it's worth
noting that this is a genuine, unavoidable, use higher-order store: we
make use of pointers to code, including the ability to dynamically
modify them.) Once we modify a cell, any memoized value it has is no
longer necessarily correct. 

Therefore, we have to drop the memoized value of the cell, and any
cell that transitively observes the cell. This is what the
$\ctext{invalidate}$ function does. Given a cell, it takes all of the
observers of the current cell and recursively invalidates all of
them. Then it removes the current cell from the observer sets of
everyone it depends on, and then it nulls out the current cell's
memoized value, as well as setting its read and observer sets to
empty. Notice that there is no explicit base case to the recursive
call; if there are any cycles in the dependency graph, invalidation
could go into an infinite loop. 

So far, we've desribed the implementation invariants incrementally.
Before we describe them formally, let's state them again, all in 
one place: 

\begin{itemize}
  \item Every cell must have a unique numeric identifer
  \item Every cell is either ready/valid, or unready/invalid. 
  \item Every valid cell has a memoized value, and maintains 
    two sets, one containing every cell that it reads, and the
    other containing every cell it is ready by. 
  \item Every invalid cell has no memoized value, and has 
    both an empty read set and an empty observer set. 
  \item The overall dependency graph among the valid cells must form a
    directed acyclic graph. 
  \item The reads and the observers must be the same, only 
    pointing in opposite directions.
\end{itemize}

Formalizing these constraints is relatively straightforward, but we
have the problem that these constraints are somewhat global in nature:
we can't be sure that the dependency graph is acyclic without having
it all available to examine, and likewise we can't in general know
that a cell is in the read set of everything in its observed set
without knowing the whole graph. Handling this difficultly is one of
the primary contributions of this work. 

\section{The Abstract Semantics of Notifications}

The key to getting around our difficulties lies in the difference
between the implementation of $\ctext{update}$, and in $\ctext{read}$.
$\ctext{update}$ calls $\ctext{invalidate}$, which recursively follows
the observers. The $\ctext{read}$ function, on the other hand,
proceeds in the opposite direction -- it evaluates code expressions,
recursively descending into the footprint of its command. The opposite
direction these two functions look is why we end up needing a global
invariant: we need to know that these two directions are in harmony
with one another.

Now, note that the type of $\ctext{invalidate}$ is simply monadic
$\monad{\unittype}$, which precludes it from being called from within
a $\codetype{A}$. This means that when we are evaluating a code
expression, we will never actually follow the observers fields --
we'll only patch them as needed to maintain the global invariant.
As a result, we can conceivably find an abstract description 


An \emph{abstract heap formula} is a syntactic description of the
state of part of the cell heap. It is given by the following grammar:
\begin{displaymath}
  \begin{array}{lcl}
    \phi & ::= & I \bnfalt \phi \otimes \psi \bnfalt \cellpos{a}{e}{v}{r} \bnfalt \cellneg{a}{e}\\
         &  |  & \localref{r}{v} 
  \end{array}
\end{displaymath}

Informally, $I$ represents an empty abstract heap, and $\phi \otimes
\psi$ represents an abstract heap that can be broken into two disjoint
parts $\phi$ and $\psi$. The atomic form $\localref{r}{v}$ says that
$r$ is a piece of local state owned by the network, currently with
value $v$.

There are two atomic forms representing cells. $\cellneg{a}{e}$ says
that $a$ is a cell with code $e$, which is unready to deliver a value
--- it needs to be re-evaluated before it can yield a
value. $\cellpos{a}{e}{v}{r}$ says that $a$ is a cell with code $e$,
and that it has value $v$, assuming that all of its reads in $r$ all
themselves have ready values. 

First, notice the must/may flavor of this reading. The formula
$\cellneg{a}{e}$ says that $a$ \emph{must} not have a current value. 
The formula $\cellpos{a}{e}{v}{r}$  says that $a$ \emph{may} have a
value, conditional on its read set $r$ having values. 

Second, notice that the backwards dependencies are entirely missing
from these formulas. We have simply left out the other half of the
dependency graph from this description. Forgetting this information is
actually critical for local reasoning. 

The invariant governing 

\begin{tabbing}
    $G(\phi) = $ \\ 
\;\;\= $\exists H \in Heap.\;$\=$ R_H^\dagger = O_H \land R_H^+ \mbox{ strict partial order }$ \\
    \> \> $\land\; R \subseteq V_H \times V_H \land unique(H)$ \\
    \> \> $\land\; \satisfies{H}{\phi} \land heap(H)$ \\
\end{tabbing}

\begin{figure}
\begin{tabbing}
$Heap = \Sigma$\=$ D \in \powersetfin{ecell}.$ \\
               \>$(\Pi (\pack{\alpha}{\_}) \in D. ($\=${\codetype{\alpha}} \;\times$ \\
               \>                          \>${\opttype{\alpha}} \;\times$ \\
               \>                          \>${\powersetfin{ecell}} \;\times$ \\
               \>                          \>${\powersetfin{ecell}} \;\times$ \\
               \>                          \>$\N)$ \\[1em]
       

$code = \pi_1$ \\
$value = \pi_2$ \\
$reads = \pi_3$ \\
$obs = \pi_4$ \\
$identity = \pi_5$ \\[1em]

$V_{(D, h)} = \comprehend{c \in D}{\exists v.\; value(h(c)) = \ctext{Some}(v)}$ \\
$R_{(D, h)} = \comprehend{(c,c') \in D}{ c' \in reads(h(c)) }$ \\
$O_{(D, h)} = \comprehend{(c,c') \in D}{ c' \in obs(h(c)) }$ \\[1em]

$unique(D,h) = \exists$\=$i : Fin(|D|) \to D.$\\
                       \>$i \circ (identity \circ h) = id \; \land$ \\
                       \>$(identity \circ h) \circ i = id$ \\[1em]

$\satisfies{(D,h)}{\phi} = sat((D,h), D, h)$ \\[1em]

$sat(H, D, \localref{r}{v}) = \top$ \\
$sat(H, D, I) = \top$ \\
$sat(H, D, \phi \otimes \psi) = \exists D_1, D_2.\;$\=$D = D_1 \uplus D_2$ \\
                                                    \>$\land\; sat(H, D_1, \phi)$ \\
                                                    \>$\land\; sat(H, D_2, \psi)$ \\
$sat(H, D, \cellneg{a}{e}) = a \in D \land code(h(a)) = e \land a \not \in V_H$ \\
$sat(H, D, \cellpos{a}{e}{v}{r}) = $ \\
\qquad\= $a \in D \land code(h(a)) = e \;\land$ \\
      \>$\mathrm{if}($\=$r \cap V_H = r,$ \\
      \>     \>$value(h(a)) = \ctext{Some}\;v \land reads(h(a)) = r,$ \\
      \>     \>$a \not\in V_H)$\\[1em]
 

$heap(D,h) = $ \\
\;\;$counter \pointsto |D| \;* $ \\
\;\;$\forall^* c \in D.\;$\=$\exists v_r, v_o : \cellset.\;$ \\
                         \>$c.code \pointsto code(h(c))   \;* $ \\
                         \>$c.value \pointsto value(h(c)) \;* $ \\
                         \>$c.reads \pointsto v_r \;* $ \\
                         \>$c.obs   \pointsto v_o \;* $ \\
                         \>$c.id    \pointsto identity(h(c)) \;\land$ \\
                         \>$set(D, v_r, reads(h(c))) \;\land$ \\
                         \>$set(D, v_o, obs(h(c)))$ \\[1em]
\end{tabbing}
\caption{Definitions for Heap Invariant}
\label{heap-invariant}
\end{figure}

\begin{figure}
\begin{mathpar}
  \inferrule*[right=AUnit]
            { }
            {\astep{\phi}{\return v}{\phi}{v}{\emptyset}{\emptyset}}
\and
  \inferrule*{\astep{\phi}{e}{\phi'}{v'}{r_1}{u_1} \\
              \astep{\phi'}{f\;v'}{\phi''}{v''}{r_2}{u_2}}
             {\astep{\phi}{\bind e\; f}{\phi''}{v''}{r_1 \cup r_2}{u_1 \cup u_2}}
\\
\\
  \inferrule*[right=AReady]
            {\ready{\phi}{a}{v}}
            {\astep{\phi}{\readcell a}{\phi}{v}{\setof{a}}{\emptyset}}
\and
  \inferrule*[right=AUnready]
            {\unready{\phi}{a} \\
             \astep{\phi}{e}{\phi'}{v}{r}{u}}
            {\aconfig{\celleither{a}{e} \otimes \phi}
                     {\readcell a} \Downarrow \\
             \aconfig{\cellpos{a}{e}{v}{r} \otimes R(\setof{a},\phi')}
                     {v}
             \aeffect{\setof{a}}
                     {u \cup \setof{a}}}
\\
\\
  \inferrule*[right=AGetRef]
            { } 
            {\astep{\phi \otimes \localref{r}{v}}{\getref r}
                   {\phi \otimes \localref{r}{v}}{v}{\emptyset}{\emptyset}}
\and
  \inferrule*[right=ASetRef]
            { } 
            {\astep{\phi \otimes \localref{r}{v}}{\setref r\;v'}
                   {\phi \otimes \localref{r}{v'}}{\unit}{\emptyset}{\emptyset}}
\\
\\
  \inferrule*[right=AbstractFrame]
            {\astep{\phi}{e}{\phi'}{v}{r}{u}}
            {\astep{\phi \otimes \psi}{e}{\phi' \otimes R(u, \psi)}{v}{r}{u}}
\end{mathpar}
\caption{Abstract Semantics of Notifications}
\label{abs-semantics}
\end{figure}

\begin{figure}
\begin{mathpar}
  \inferrule*[right=Ready]
            {\forall a' \in r.\; \exists v.\; \ready{\phi}{a'}{r}}
            {\ready{\phi \otimes \cellpos{a}{e}{v}{r}}{a}{v}}
  \\
  \\
  \inferrule*[right=UnreadyPos]
            {\exists a' \in r.\; \unready{\phi}{a}}
            {\unready{\phi \otimes \cellpos{a}{e}{v}{r}}
                     {a}}
  \and
  \inferrule*[right=UnreadyNeg]
            { }
            {\unready{\phi \otimes \cellneg{a}{e}}{a}}
\end{mathpar}
\caption{Ready and Unready Judgements}
\label{readiness}
\end{figure}

\begin{figure}
  \begin{displaymath}
    \begin{array}{lcl}
      R(s, I)                 & = & I \\
      R(s, \phi \otimes \psi) & = & R(s, \phi) \otimes R(s, \psi) \\
      R(s, \localref{r}{v})   & = & \localref{r}{v} \\
      R(s, \cellneg{a}{e})    & = & \cellneg{a}{e} \\
      R(s, \cellpos{a}{e}{v}{r}) & = & \left\{\begin{array}{ll}
                                                \cellpos{a}{e}{v}{r} 
                                              & \mbox{if } s \cap r = \emptyset \\
                                                \cellneg{a}{e}
                                              & \mbox{otherwise}
                                              \end{array}
                                       \right.
    \end{array}
  \end{displaymath}
\caption{Definition of the Ramification Operator $R$}
\label{ramify-def}
\end{figure}


\section{Verifying an Imperative Implementation of Functional Reactive Programming}

In this section, we will see how to verify an imperative
implementation of a simple synchronous functional reactive programming
system.

\subsection{Specifying Functional Reactive Programs}

\emph{Functional Reactive Programming}~\cite{frp} is a style of
writing interactive programs based on the idea of \emph{stream
  transducers}.  The idea is to model a time-varying input signal of
type $A$ as an infinite stream of $A$'s, and to model an interactive
system as a function that takes a stream of inputs $\stream{A}$ and
yields a stream of outputs $\stream{B}$. Note that a stream can be
viewed either as an infinite sequence of values, or isomorphically as
a function from natural numbers to values (i.e., a function from times
to values). In our discussion, we'll switch freely between these two
views, using whichever viewpoint is most convenient.
\footnote{Given an infinite stream $vs$, we will use use $take\;n\;vs$ to denote
the finite list consisting of the first $n$ elements of the stream
$vs$. Correspondingly, $drop\;n\;vs$ is the infinite stream with $vs$
with its first $n$ elements cut off. With a function $f$, $map\;f\;vs$
maps $f$ over the elements of $vs$, and given another infinite stream
$us$, the call $zip\;us\;vs$ returns the infinite stream of pairs of
elements of $us$ and $vs$. If $v$ is an element, $v \cdot vs$ will 
denote consing $v$ to the front of $vs$, and if $xs$ is a finite list, then
$xs \cdot vs$ will denote appending the finite sequence $xs$ to the
front of $vs$. Finally, we will write $vs_n$ to denote the $n$-th element
of the stream $vs$.}

However, not all functions $\stream{A} \to \stream{B}$ are legitimate
stream transducers; we need to restrict our attention to \emph{causal}
stream transducers. A transducer is causal if we can compute the first
$n$ elements of the output after having read at most $n$ elements of
the input. 

\begin{tabbing}
$causal(f : \stream{A} \to \stream{B}) \equiv$ \\
\;\;\= $\exists \hat{f} : \listtype{A} \to \listtype{B}.\;\forall as:\stream{A}, n:\N.$ \\
    \> \;\;$take\;n\;(f\;as) = \hat{f}\;(take\;n\;as)$ 
\end{tabbing}

If we are given a causal transducer $p$, we will write $\hat{p}$ to
indicate the corresponding list function which computes its finite
approximations. Then, we can define a family of combinators acting on
causal transducers, which we give in Figure~\ref{transducer-semantics}.

\begin{figure}
\begin{tabbing}
$\ST{A}{B} = \comprehend{f \in \stream{A} \to \stream{B}}{causal(f)}$\\[1em]

$lift : (A \to B) \to \ST{A}{B}$ \\
$lift\;f\;as = map\;f\;as$ \\[1em]

$seq  : \ST{A}{B} \to \ST{B}{C} \to \ST{A}{C}$ \\
$seq\;p\;q = q \circ p$ \\[1em]

$par  : \ST{A}{B} \to \ST{C}{D} \to \ST{A \times C}{B \times D}$ \\
$par\; p\;q\;abs = zip\; (p\;(map\;\pi_1\;abs))\;(q\;(map\;\pi_2\;abs))$\\[1em]

$switch : \N \to \ST{A}{B} \to \ST{A}{B} \to \ST{A}{B}$ \\
$switch\;k\;p\;q = \semfun{as}{(take\;k\;(p\;as))\cdot(q\;(drop\;k\;as))}$ \\[1em]

$loop : A \to \ST{A\times B}{A \times C} \to \ST{B}{C}$ \\
$loop\;a_0\;p = (map\;\pi_2) \circ (cycle\;a_0\;p)$ \\[1em]

$cycle : A \to \ST{A\times B}{A \times C} \to \ST{B}{A \times C}$ \\
$cycle\;a_0\;p = \lambda bs.\;\lambda n.\;last(gen\;a_0\;p\;v\;n)$ \\[1em]

$gen : A \to \ST{A\times B}{A \times C} \to \listtype{(A \times C)}$\\
$gen\;a_0\;p\;bs\;0 \;\;\; = \hat{p}\; [(a_0, bs_0)]$ \\
$gen\;a_0\;p\;bs\;(n+1) = $ \\
\;\;$\hat{p}\;(zip (a_0 \cdot (map\;\pi_1\;(gen\;a_0\;p\;bs\;n)))\;
                                        (take\;(n+2)\;bs))$ 
\end{tabbing}
\caption{Semantics of Stream Transducers}
\label{transducer-semantics}
\end{figure}

The operation $lift\;f$ creates a stream transducer that simply maps
the function $f$ over its input. Calls to $seq\;p\;q$ are sequential
composition: it feeds the output of $p$ into the input of $q$. The
operator $par\;p\;q$ defines parallel composition --- it takes a
stream of pairs, and feeds each component to its arguments,
respectively, and then merges the two output streams to produce the
combined output stream. The function $switch\;k\;p\;q$ is a very
simple ``switching combinator''.  It behaves as if it were $p$ for the
first $k$ time steps, and then behaves as if it were $q$, only
starting with the input stream beginning at time $k$.

The combinator $loop\;a_0\;p$ is a ``feedback'' operation. It acts
upon a transducer $p$ which takes pairs of $A$s and $B$s, and yields
pairs of $A$s and $C$s. It turns it into a combinator that takes $B$s
to $C$s, by giving $p$ the value $a_0$ (and its $B$-input) on the
first timestep, and uses the output $A$ at time $n$ as the input $A$
at time $n+1$. This is useful for constructing transducers that do
things like sum their inputs over time, and other stateful operations. 

Because this function involves feedback, it should not be surprising
that it makes use of the causal nature of its argument operation. The
$loop$ function is defined in terms of $cycle$, which also returns the
sequence of output $A$s, and $cycle$ is defined in terms of $gen$,
which is a function that given an argument $n$ returns a list of
outputs for the time steps from $0$ to $n$. Notice that
$gen\;a_0\;p\;bs\;n$ will always return $n+1$ elements (e.g., at
argument 0, it will return a 1 element list containing the output at
timestep 0), which means that the call to $last$ in $cycle$ is
actually safe. In order to calculate $gen$, we need to recursively
calculate the outputs for all smaller timesteps, and this is what
$\hat{p}$ is needed for -- it is what lets us know that $p$ has a good
finite approximation.

All of these definitions are familiar to functional programmers, and
there are many techniques to prove properties of these functions ---
coinductive proofs, Bird and Wadler's $take$-lemma, arguments based on
the isomorphism between streams and functions from natural
numbers. All of these serve to make proving properties about stream
transducers very pleasant. For example, one property we will need in
the next section is the following:

\begin{lemma}{(Loop Unrolling)} We have that 
  \begin{displaymath}
    cycle\;a_0\;p\;bs = f\;(zip\;(a_0\cdot(map\;\pi_1\;(cycle\;a_0\;p\;bs)))\;bs)
  \end{displaymath}
\end{lemma}

\begin{proof}
  This is easily proven using Bird and Wadler's $take$-lemma, which
  says that two streams are equal if all their finite prefixes are
  equal.
\end{proof}


\subsection{Realizing Stream Transducers with Notifications}

\begin{figure}
\begin{tabbing}
$\ST{A}{B} \equiv \celltype{A} \to \monad{\celltype{B}}$ \\[1em]

$\liftop : (A \to B) \to \ST{A}{B}$ \\
$\liftop\;f\;input = $ \\
\;\; $\newcell (\bind\;(\readcell input)\; (\fun{x}{A}{\return (f\;x)}))$ \\[1em]

$\composeop : \ST{A}{B} \to \ST{B}{C} \to \ST{A}{C}$ \\
$\composeop p\;q\;input = [$\=$\letv{middle}{p\;input}$ \\
                            \>$\letv{output}{q\;middle}$ \\ 
                            \>$\;output]$ \\[1em]

$\parop : \ST{A}{B} \to \ST{C}{D} \to \ST{A \times C}{B \times D}$ \\
$\parop p \; q \; input = $ \\
\;\;$[$\=$\ctext{letv}\;a = \newcell (\bind$\=$(\readcell\;input)$ \\
     \>                                   \>$(\fun{x}{A\times B}{\return (\fst{x})}))$ \\
     \>$\ctext{letv}\;b = {p\;a}{}$ \\
     \>$\ctext{letv}\;c = \newcell (\bind$\=$(\readcell\;input)\;$\\ 
     \>                                   \>$(\fun{x}{A\times B}{\return (\snd{x})}))$ \\
     \>$\ctext{letv}\;d = {q\;b} = $ \\
     \>$\ctext{letv}\;output = \newcell ($\=$\bind (\readcell b)\; (\lambda b:B.$ \\
     \>                                   \>$\bind (\readcell d)\; (\lambda d:D.$ \\
     \>                                   \>$\;\;\return \pair{b}{d})))] \;\ctext{in}$ \\
     \>$\;output]$ \\[1em]

$\switchop : \N \to \ST{A}{B} \to \ST{A}{B} \to \ST{A}{B}$ \\
$\switchop k\;p\;q\; input =  $ \\
\;\;$[$\=$\letv{r}{\newref{\N}{0}}{}$ \\
    \>$\letv{a}{p\;input}$ \\
    \>$\letv{b}{p\;input}$ \\
    \>$\ctext{letv}\; out = \newcell ($\=$\bind (\getref r) \;(\lambda i:\N.\;$ \\
    \>                                 \>$\bind (\setref r\;(i+1)) \; (\lambda q:\unittype.$ \\
    \>                                 \>$\;\;\ctext{if}(i < k, \readcell a, \readcell b)))) \;\ctext{in}$ \\
    \>$\;\;out]$\\[1em]

$\loopop : A \to \ST{A\times B}{A\times C} \to \ST{B}{C}$ \\
$\loopop a_0\; p \; input = $ \\
\;\;$[$\=$\letv{r}{\newref{A}{a_0}}{}$ \\
    \>$\ctext{letv}\; ab = \newcell ($\=$\bind (\readcell input)\; (\lambda b:B.$ \\
    \>                                \>$\bind (\getref r)\;       (\lambda a:A.$ \\
    \>                                \>$\;\; \return \pair{a}{b}))) \;\ctext{in}$ \\
    \>$\letv{ac}{p\;ab}{}$ \\
    \>$\ctext{letv}\;c = \newcell ($\=$\bind (\readcell ac) \;(\lambda v:A \times C.$ \\
    \>                              \>$\bind (\setref r\;(\fst{v})) \;(\lambda q:\unittype.$ \\
    \>                              \>$\;\;\return (\snd{v}))))\;\ctext{in}$ \\
    \>$\;\;c]$ 
\end{tabbing}
\caption{Imperative Stream Transducers}
\label{imperative-transducer-semantics}
\end{figure}

While the definitions in the previous subsection yield very clean
proofs, they are not suitable as implementations -- for example,
$loop$ recomputes an entire history at each time step! We can derive
better implementations by thinking about how imperative, event-driven
programming works.

The intuition underlying event-driven programming is that a stream
transducer is implemented with the combination of a notification
network, and an \emph{event loop}.  The event loop is a
(possibly-infinite) loop which updates an input cell at teach time
step, to reflect the events that occured on that time step, and then
it reads the output cell of the network. When the input cell is
updated, invalidations are propagated throughout the dependency
network, and when the outputs are read, exactly the necessary
recomputations are performed.

We will shortly formalize exactly this idea, but we will first discuss
the implementation given in
Figure~\ref{imperative-transducer-semantics} in informal terms. Here,
we define the type of imperative stream transducers as a function type
$\celltype{A} \to \monad{(\celltype{B})}$. This type should be read as
saying that the implementation is a function that, given an input cell
of type $A$, will \emph{construct} a dataflow notification network
realizing the corresponding transducer, and whose return value is the
output cell of type $B$ that the event loop should read. 

The simplest example of this is $\liftop\;f$. It will take an input
cell $input$, and build a new cell which reads $input$, and return $f$
applied to that value. Likewise, given two imperative implementations $p$
and $q$, $\composeop\;p\;q$ will take an input cell, and feed the
input to $p$ to build a network whose output is named $middle$, and
will then give $middle$ to $q$ to get the final output cell. The
overall network will be network built by the calls to both $p$ and
$q$, which interact through $p$'s network putting a value in $middle$,
and $q$'s network reading it.

The operation $\switchop\;k\;p\;q$ is the first example that uses 
local state. Given an $input$ cell, we first build networks corresponding
to $p$ and to $q$ (with outputs $a$ and $b$, respectively). Then we
create a local reference $r$, initialized to $0$. Then we build a cell $out$,
whose code reads an increments $r$, and which will read $a$ or $b$ depending
on whether the reference's contents are less than or equal to $k$. Notice
that the demand-driven nature of evaluation means that we never redundantly
evaluate $p$ or $q$'s networks -- we only ever execute one of them. 

Finally, the operation $\loopop\;a_0\;p$ builds a feedback network by
explicitly creating a reference to hold an accumulator parameter. It
constructs a local reference initialized to $a_0$, and then constructs
a cell $ab$ which reads the input and the local reference to produce a
pair of type $A \times B$. This cell is given to $p$, to construct a
network with an output cell $ac$, yielding pairs of type $A \times
C$. Finally, we construct the overall output cell $c$, which reads
$ac$ and updates the local reference with a new value of type $A$, and
returns a value of type $C$. The use of a local reference (rather than
a cell) to store the current state of $A$ is essential, because we need
to maintain the acyclicity of the dataflow graph. 

Now, with these ideas in mind, we come to the definition of what it
means for a dataflow network to realize a stream transducer. This
property is (unavoidably) quite large, but despite its size is quite
pleasant to work with.
\begin{tabbing}
$Realize(i, \Phi, o, f) \triangleq$ \\
\;\;\= $\forall v:$\=$\,\stream{A}.\;$ \fbox{$\exists \phi:\stream{\formula}, u:\stream{\powersetfin{ecell}}.$} \\
\>\> $[\forall n:\N.\;$\=$\closed{\phi_n}{\domain{\phi_n} \cup \setof{i}} \;\land$ \\
\>\>\>$\domain{\phi_n} = \domain{\phi_{n+1}}\; \land$ \\
\>\>\>$\forall \psi.\; \unready{\psi}{i} \implies \unready{\psi \otimes \phi_n}{o}]$ \\
\>\> $\land \; \Phi = \phi_0$ \\
\>\> $\land \;  Transduce(i, \phi, o, u, f, v)$ \\[1em]


$Transduce(i, \phi, o, u, f, v) \triangleq$ \\
\> $\forall $\=$ n:\N, \phi_i, \phi'_i, u_i.$ \\
\>\> \fbox{$\astep{\phi_i}{\readcell i}{\phi'_i}{v_n}{\setof{i}}{u_i}$} $\; \land$ \\
\>\> $\domain{\phi_i} = \domain{\phi'_i} \;\land $ \\
\>\> $\closed{\phi_i}{\domain{\phi_i}} \land 
      \closed{\phi_i}{\domain{\phi'_i}} \land $ \\
\>\> $\unready{\phi_i}{i} \land
        i \in u_i$ \\
\>\> $\Longrightarrow$ \\
\>\> \fbox{$\astep{\phi_i \otimes \phi_n}{\readcell o}
              {\phi'_i \otimes \phi_{n+1}}{(f\;v)_n}
              {\setof{o}}{u_i \cup u_n}$} \\
\>\> $\land\; u \subseteq \domain{\phi_n} \land o \in u$ \\
\end{tabbing}

We read $Realize(i, \Phi, o, f)$ as saying ``the dataflow network
$\Phi$ realizes the stream transducer $f$, when the event loop writes
inputs into $i$ and reads outputs from $o$''. We have highlighted the
key pieces of this definition with boxes. First, given some input
stream $v$, we existentially assert the existence of a \emph{stream}
of abstract heap formulas $\phi$. These streams represent the evolving
state of the network over time --- because our notification networks
contain local state, that state can potentially have a different value
at each time step. Then, in the unboxed formulas, we assert some
well-formedness properties, such as 1) the initial value of the stream
matching our $\Phi$, 2) the domain of the network remaining constant
over time, 3) that the output is unready if the input is, and 4) that
the network reads only the specified input cell. (All of these
conditions could be relaxed --- for example, we could add operations
that created new cells --- but for the basic FRP combinators we
consider there is no such need.)

We state the stream property with the $Transducer$ sub-predicate. It
asserts that if, at each time $n$ we have input state $\phi_i$ which
yields the $n$-th input when $i$ is read (the second boxed formula), then 
reading $o$ in a larger state containing both the input $phi_i$ and the
$n$-th transducer state $\phi_n$ will 1) return the $n$-th transduced
value $(f\;v)_n$, and 2) will update the state of the network to 
the next timestep $\phi_{n+1}$ (the third boxed formula). 

With this property in hand, we can prove the following specifications: 

\begin{tabbing}
$Relate_{A,B}(p, f) \triangleq$ \\
\;\; $\forall \psi:\formula, i:\celltype{A}.$ \\ 
\qquad $\spec{G(\psi)}{p\;i}{o:\celltype{B}}
             {\exists \Phi.\;G(\Phi \otimes \psi) \land Realize(i, \Phi, o, f)}$ \\[1em]

$\forall f:A\to B.\; Relate(\liftop f, lift\;f)$ \\[1em]

$\forall p, f, q, g.\;$\=$Relate(p, f) \specand Relate(q, g)$ \\
                       \>$\specimp Relate(compose\;f\;g, \composeop p\;q)$ \\[1em]

$\forall p, f, q, g.\;$\=$Relate(p, f) \specand Relate(q, g)$ \\
                       \>$\specimp Relate(\parop p\;q, par\;f\;g)$ \\[1em]

$\forall k, p, f, q, g.\;$\=$Relate(p, f) \specand Relate(q, g)$ \\
                          \>$\specimp Relate(\switchop k\;p\;q, switch\;k\;f\;g)$ \\[1em]

$\forall a_0, p, f.\; 
  Relate(p, f) \specimp Relate(\loopop\;a_0\;p, loop\;f\;a_0)$
\end{tabbing}

These lemmas are what permit us to reason about our transducer
implementation as if it were a pure implementation -- for each
combinator in the interface, we have a proof that shows the
corresponding implementation combinator lifts


\appendix

\section{Appendix: The $\cellset$ Interface}

In this section, we'll describe the interface to the $\cellset$ type,
which is intended to be used to represent pure collections of
existentially quantified cells. Specifying this interface is not
entirely trivial, because of the way equality works for this
type. Ordinarily, we would simply give a two-place predicate $set(v,
elts)$ relating a value $v$, and the mathematical set of elements
$elts$ it represents.

However, this approach is not sufficient in our case. In order to
manage dependencies, we need to be able to test cells of
\emph{different} concrete type for equality, and the natural equality
for references only permits testing references of the same type. As a
result, we can't unpack an existentially-quantified cell and compare
the elements in its tuple, because we don't know that the two cells
are of the same type (and indeed, they might not be). 

To get around this problem, we generate a unique integer identifier
for each cell we create, and then compare those identifiers to
establish equality. Since these identifiers are all generated
dynamically along with the cells, this means that the precise partial
equivalence relation we need to use is determined dynamically as
well. So we add an additional index to the set predicate $set(W, v,
elts)$. The extra parameter $W$ is the \emph{world}, the set of all
the cells allocated so far, which must collectively satisfy the
constraint that they are equal if and only if their identifier fields
match. 

All of the usual set operations exist. We have an $\ctext{emptyset}$,
representing an empty set of cells. We have $\ctext{addset}\;v\;x$,
which adds the element $x$ to the set $v$ represents, and
$\ctext{removeset}\;v\;x$, which removes $x$ from the set $v$
represents. We also have $\ctext{iterset}\;f\;v$, which iterates over
the elements of $v$'s set and applies $f$ to each element in some
sequential order. (Observe that the specification makes use of two 
auxilliary predicates, $matches$ and $iterseq$, which assert that a
set and a list have the same elements, and $iterseq$, which constructs
a command representing the sequential exection of those elements.) 

We have three axioms that our implementation must satisfy. First, if a
$\cellset$ value $v$ represents a set in a world $W$, it will also
represent a set in any larger world $W'$. Second, the values in a set
are always a subset of the world $W$. Finally, we require that
$set(W,v,elts)$ is a pure predicate (i.e., is not heap-dependent),
which implies that it have a purely functional implementation (for
example, as a binary tree). This is not strictly necessary, but eases
verification. 


\begin{tabbing}
$World = $ \\
\;\;\;\;\=$\{D \in \powersetfin{ecell}\;|\;\forall c,d \in D.\;$\=$identity(c) = identity(d)$\\
        \>                                                      \>$\iff c = d\}$\\[1em]

$\exists \cellset : \star.$ \\
$\exists set : World \To \cellset \To \powersetfin{ecell} \To \assert.$ \\
$\exists \ctext{emptyset}    : cellset.$ \\
$\exists \ctext{addset}      : \cellset \to ecell \to \cellset.$ \\
$\exists \ctext{removeset}   : \cellset \to ecell \to \cellset.$ \\
$\exists \ctext{iterset}     : \cellset \to (ecell \to \monad{\unittype}) \to \unittype.$\\[1em]

$\forall W \in World.\; set(W, \ctext{emptyset}, \emptyset)$ \\[1em]


$\forall W \in World, v : \cellset, x : ecell, elts \in \powersetfin{ecell}.$ \\ 
\> $set(W, v, elts) \land x \in W \implies set(W, \ctext{addset}\;v\;x, elts \cup \setof{x})$ \\[1em]


$\forall W \in World, v : \cellset, x : ecell, elts \in \powersetfin{ecell}.$ \\ 
\> $set(W, v, elts) \land x \in W \implies set(W, \ctext{removeset}\;v\;x, elts - \setof{x})$ \\[1em]


$\forall W \in World, v : \cellset, elts \in \powersetfin{ecell}, 
         f : (ecell \to \monad{\unittype}).$ \\ 
\> $set(W, v, elts) \implies \exists L : \seqsort{ecell}.\;$\=$matches\;elts\;L\; \land$ \\
\>                                \>$\ctext{iterset}\;f\;v = iterseq\;f\;L$ \\[1em]


$\forall W, W' \in World, v, elts.$ \\
\>$set(W,v,elts) \land W \subseteq W' \implies set(W',v, elts)$\\[1em]

$\forall W, W' \in World, v, elts.$ \\
\>$set(W, v, elts) \implies elts \subseteq W$ \\[1em]

$\forall W, v, elts.\; pure(set(W,v,elts))$ \\[1em]
  

$matches\;elts\;[] \qquad\;\;\; = elts = \emptyset$ \\
$matches\;elts\;(v :: vs) = v \in elts \land matches\;(elts - \setof{v})\;vs$\\[1em]

$iterseq\; f\; [] \qquad\;\;\;\;$\=$= \comp{\unit}$ \\
$iterseq\; f\; (v :: vs)$\>$= \comp{\letv{\unit}{f\;v}{\ctext{run}\;iterseq\;f\;vs}}$ \\
\end{tabbing}


\acks

We would like to thank Peter O'Hearn for pointing out the connection
of our work with the ramification problem of AI.

\bibliographystyle{plainnat}

\begin{thebibliography}{}

\bibitem{smith02}
Smith, P. Q. reference text

\end{thebibliography}

\end{document}

