\chapter{The Semantics of Separation Logic}

\section{BI Algebras}

A \emph{BI algebra} is a Heyting algebra with additional residuated
monoidal structure to model the separating conjunction and wand. This
means that a BI algebra is a partial order $(B, \leq)$ with
operations $(\top, \land, \implies, \bot, \vee, I, *, \wand)$
satisfying:

\begin{enumerate}
\item $\forall p \in B.\; p \leq \top$
\item $\forall p \in B.\; \bot \leq p$
\item $\forall p,q,r \in B.$ if $r \leq p$ and $r \leq q$, then
      $r \leq p \land q$ and 
      $p \land q \leq p$ and $p \land q \leq q$
\item $\forall p,q,r \in B.$ if $p \leq r$ and $q \leq r$, then
      $p \vee q \leq r$ and
      $p \vee q \leq p$ and $p \vee q \leq q$.
\item $\forall p, q, r.\; p \land q \leq r \iff p \leq q \implies r$
\item $\forall p.\; p * I = p$
\item $\forall p, q.\; p * q = q * r$
\item $\forall p, q, r.\; (p * q) * r = p * (q * r)$
\item $\forall p, q, r.\; p * q \leq r \iff p \leq q \wand r$
\end{enumerate}

The first five conditions are just the usual conditions for a Heyting
algebra (that it have greatest and least elements, greatest lower
bounds and least upper bounds, and that it have an implication). The
next three are the monoid structure axioms, that say $I$ is the unit,
and that $*$ is commutative and associative. The last axiom asserts
the existence of a wand that behaves with respect to the separating
conjunction the same way that the implication behaves with respect to
the ordinary conjunction.

\subsection{Partial Commutative Monoids}

A partial commutative monoid is a triple $(M, e, \cdot)$, where 
$e \in M$, and $(\cdot)$ is a partial operation from $M \times M$ to 
$M$. We write $m \# m'$ to mean that $m$ and $m'$ are compatible -- 
that is, $\exists m''.\; m \cdot m' = m''$. Furthermore, the following
properties hold: 

\begin{itemize}
\item $e$ is a unit, so that $e \cdot m = m$. 
\item $\cdot$ is commutative. $m \cdot m' = m' \cdot m$, if defined.
\item $\cdot$ is associative. $m \cdot (m' \cdot m'') = (m \cdot m') \cdot m''$, if defined.
\end{itemize}

Given a partial commutative monoid, we can show that the powerset $\powerset{M}$ 
forms a BI-algebra. 

\begin{lemma}{(Powersets of Partial Commutative Monoids)}
Given a partial commutative monoid $(M, e, \cdot)$, its powerset
$(\powerset{M}, \subseteq)$ forms a BI-algebra
with the following operations:

\begin{itemize}
\item $\top = M$
\item $p \land q = p \cap q$
\item $p \implies q = \setof{m \in M \;|\;\mbox{if }m \in p \mbox{ then } m \in q}$
\item $\bot = \emptyset$ 
\item $p \vee q = p \cup q$
\item $I = \setof{e}$
\item $p * q = \setof{m \in M\;|\; \exists m_1, m_2.\; 
                       m_1 \# m_2 \land m_1 \in p \land m_2 \in q \land m_1\cdot m_2 = m}$
\item $p \wand q = \{ m \in M \;|\; \forall m' \in p.\; \mbox{if } m \# m' \mbox{ then } m \cdot m' \in q \}$
\end{itemize}
\end{lemma}



\begin{proof}
\begin{enumerate}
\item  We want to show $\forall p \in \powerset{M}.\; p \leq \top$
  \begin{tabbedproof}
        Assume $p \in \powerset{M}$ \\
    \oo By definition of powerset, $p \subseteq M$ \\
    \oo By definition of $\top$, $p \subseteq \top$ \\
    \oo By definition of $\leq$, $p \leq \top$ \\
  \end{tabbedproof}

\item We want to show $\forall p \in B.\; \bot \leq p$
  \begin{tabbedproof}
        Assume $p \in \powerset{M}$ \\ 
    \oo By definition of $\emptyset$, $\emptyset \subseteq p$ \\
    \oo By definition of $\bot$, $\leq$, $\bot \leq p$ \\
  \end{tabbedproof}

\item We want to show $\forall p,q,r \in \powerset{M}.$ if $r \leq p$ and $r \leq q$, then
      $r \leq p \land q$ and 
      $p \land q \leq p$ and $p \land q \leq q$

   \begin{tabbedproof}
      Assume $p,q,r \in \powerset{M}$ \\
      Assume $r \leq p$, $r \leq q$ \\[1em]

      \oo By definition of $\leq$, $r \subseteq p$, $r \subseteq q$ \\
      \oo By properties of $\cap$, $r \cap r \subseteq p \cap q$ \\
      \oo Hence $r \subseteq p \cap q$ \\
      \oo By definition of $\leq$ and $\land$, $r \leq p \land q$ \\[1em]
      \oo By properties of $\cap$, $p \cap q \subseteq p$ \\
      \oo By definition of $\leq$ and $\land$, $p \cap q \leq p$ \\[1em]
      \oo By properties of $\cap$, $p \cap q \subseteq q$ \\
      \oo By definition of $\leq$ and $\land$, $p \cap q \leq q$ \\
   \end{tabbedproof}

\item We want to show $\forall p,q,r \in \powerset{M}.$ if $p \leq r$ and $q \leq r$, then
      $p \vee q \leq r$ and
      $p \leq p \vee q$ and $q \leq p \vee q$.

  \begin{tabbedproof}
    Assume $p, q, r \in \powerset{M}$, $p \leq r$, $q \leq $ \\[1em]
    \oo By definition of $\leq$, $p \subseteq r$, $q \subseteq r$ \\
    \oo By set properties, $p \cup q \subseteq r$ \\
    \oo By definitions, $p \vee q \leq r$ \\[1em]

    \oo By set properties $p \subseteq p \cup q$ \\
    \oo By definitions $p \leq p \vee q$ \\[1em]
  
    \oo By set properties $q \subseteq p \cup q$ \\
    \oo By definitions $q \leq p \vee q$ \\
  \end{tabbedproof}

\item We want to show $\forall p, q, r \in \powerset{M}.\; p \land q \leq r \iff p \leq q \implies r$

  \begin{tabbedproof}
    Assume $p,q,r \in \powerset{M}$ \\[1em]

    $\To$ direction: \\
    \oo Assume $p \land q \leq r$ \\
    \ooo By definitions, $p \cap q \subseteq r$ \\
    \ooo Assume $m \in p$ \\
    \oooo Want to show $m \in q \implies r$ \\
    \oooo Equivalent to showing if $m \in q$, then $m \in r$ \\
    \oooo Assume $m \in q$ \\
    \ooooo Since $m \in p$ and $m \in q$, $m \in p \cap q$. \\
    \ooooo Since $p \cap q \subseteq r$, $m \in r$ \\
    \ooo Hence $p \subseteq q \implies r$ \\
    \ooo By definition of $\leq$, $p \leq q \implies r$ \\[1em]

    $\Leftarrow$ direction:\\
    \oo Assume $p \leq q \implies r$ \\
    \ooo By definition $p \subseteq q \implies r$ \\
    \ooo Assume $m \in p \cap q$ \\
    \oooo Hence $m \in p$ and $m \in q$ \\
    \oooo Since $p \subseteq q \implies r$, if $m \in p$, then if $m \in q$, then $m \in r$ \\
    \oooo Hence $m \in r$ \\
    \ooo Hence $p \cap q \subseteq r$ \\
    \ooo By definitions, $p \land q \leq r$ \\
  \end{tabbedproof}

\item We want to show $\forall p \in \powerset{M}.\; p * I = p$

  \begin{eqnproof}[\mbox{Assume } p \in \powerset{M}]
    \eline[p * I]
          {\setof{m \in M\;|\; \exists m_1 \in p, m_2 \in \setof{e}.\;
                                    m_1 \# m_2 \land m_1\cdot m_2 = m}}
          {}
    \eline{\setof{m \in M\;|\; \exists m_1 \in p.\; m_1 \# e \land m_1 \cdot e = m}}
          {Simplification}
    \eline{\setof{m \in M\;|\; \exists m_1 \in p.\; m_1 = m}}
          {Monoid axioms}
    \eline{p}
          {}
  \end{eqnproof}

\item We want to show $\forall p, q \in \powerset{M}.\; p * q = q * r$

  \begin{eqnproof}[\mbox{Assume }p,q \in \powerset{M}]
    \eline[p * q]
          {\setof{m \in M\;|\; \exists m_1 \in p, m_2 \in q.\; 
                       m_1 \# m_2 \land m_1\cdot m_2 = m}}
          {Definition}
    \eline{\setof{m \in M\;|\; \exists m_2 \in q, m_1 \in p.\; 
                       m_1 \# m_2 \land m_1\cdot m_2 = m}}
          {Logical manipulation}
    \eline{\setof{m \in M\;|\; \exists m_2 \in q, m_1 \in p.\; 
                       m_2 \# m_1 \land m_2\cdot m_1 = m}}
          {Commutativity}
    \eline{q * p}
          {Definition}    
  \end{eqnproof}

\item We want to show $\forall p, q, r \in \powerset{M}.\; (p * q) * r = p * (q * r)$

  \begin{eqnproof}
     \eline[p * q]
           {\setof{m_1 \;|\; \exists m_p \in p, m_q \in q.\; 
                             m_p \# m_q \land m_p \cdot m_q = m_1}}
           {Definition}
     \eline[(p * q) * r]
           {\setof{m \;|\; \exists m_1 \in p * q, m_r \in r.\; 
                             m_1 \# m_r \land m_1 \cdot m_r = m}}
           {Definition}
     \elines{\{ m\;|\; \exists m_1 \in p * q, m_p \in p, m_q \in q, m_r \in r. \\ 
             \qquad m_p \# m_q \land m_p \cdot m_q = m_1 \\
             \qquad m_1 \# m_r \land m_1 \cdot m_r = m \}\\ }
            {Comprehension}
     \elines{\{ m\;|\; \exists m_p \in p, m_q \in q, m_r \in r. \\ 
             \qquad m_p \# m_q \land 
                   (m_p\cdot m_q) \# m_r \land (m_p \cdot m_q) \cdot m_r = m \}\\ }
            {Simplification}
     \elines{\{ m\;|\; \exists m_p \in p, m_q \in q, m_r \in r. \\ 
             \qquad m_p \# m_q \land 
                   m_p \# m_r \land m_q \# m_r \land (m_p \cdot m_q) \cdot m_r = m \}\\ }
            {Distribute $\#$}
     \elines{\{ m\;|\; \exists m_p \in p, m_q \in q, m_r \in r. \\ 
             \qquad m_p \# m_q \land 
                   m_p \# m_r \land m_q \# m_r \land m_p \cdot (m_q \cdot m_r) = m \}\\ }
            {Associativity}
     \elines{\{ m\;|\; \exists m_p \in p, m_q \in q, m_r \in r. \\ 
             \qquad m_p \# (m_q \cdot m_r) \land 
                    m_q \# m_r \land m_p \cdot (m_q \cdot m_r) = m \}\\ }
            {Redistributing $\#$}
     \elines{\{ m\;|\; \exists m_p \in p, m_q \in q, m_r \in r, m_2 \in q * r \\ 
             \qquad m_p \# (m_q \cdot m_r) \land m_2 = m_q \cdot m_r \\
             \qquad m_q \# m_r \land m_p \cdot (m_q \cdot m_r) = m \}\\ }
            {Introducing $m_2$}
     \elines{\{m \;|\; \exists m_p \in p, m_2 \in q * r.\;
              m_p \# m_2 \land m_p \cdot m_2 = m \}\\ }
            {Comprehension}
     \elines{p * (q * r)}
            {Definition}
  \end{eqnproof}
\end{enumerate}
\end{proof}

\subsection{Sets of Heaps form a BI algebra}


As a example, given a set of heaps $H$, we can construct a BI algebra
from it by considering the set of sets of heaps $\mathcal{P}(H)$. The
ordering is inclusion $\subseteq$, and the operations are given by:

\begin{itemize}
\item $\top = H$
\item $\bot = \emptyset$
\item $p \land q = p \cap q$
\item $p \vee q = p \cup q$
\item $p \implies q = \{h \in H \;|\; \mbox{if } h \in p \mbox{ then } h \in q\}$
\item $I = \{\emptyset\}$
\item $p * q = \{ h \in H \;|\; \exists h_1 \in p, h_2 \in q.\; h_1 \# h_2 \land h = h_1 \cdot h_2\}$
\item $p \wand q = \{ h \in H \;|\; \forall h' \in p.\; \mbox{if } h \# h' \mbox{ then } h \cdot h' \in q \}$
\end{itemize}

These definitions all satisfy the previous list of bullet points.

\section{World Preorders}

Now, we can define a world preorder $W(B)$ over a BI algebra as
follows.  The elements of $W(B)$ are the elements of $B$, and for
any two elements $p$ and $q$, the ordering $p \sqsubseteq q$ is 
defined as follows:

\begin{displaymath}
p \sqsubseteq q \iff \exists r.\; p * r = q
\end{displaymath}

To verify the relation $\sqsubseteq$ is a preorder, we need to 
show it is reflexive and transitive. 

$p \sqsubseteq p$ holds because we can take $r$ to be $I$. To show 
transitivity, we must show that $p_1 \sqsubseteq p_3$, given
that $p_1 \sqsubseteq p_2$ and $p_2 \sqsubseteq p_3$. 
\\

\begin{tabular}{ll}
$p_1 \sqsubseteq p_2$                 & By assumption  \\
$\exists r.\; p_1 * r = p_2$          & Def of $\sqsubseteq$ \\
$p_1 * r = p_2$                       & Existential elim (1)\\
$p_2 \sqsubseteq p_3$                 & By assumption \\
$\exists r'.\; p_2 * r' = p_3$        & Def of $\sqsubseteq$ \\
$p_2 * r' = p_3$                      & Existential elim \\
$(p_1 * r) * r' = p_3$                & Equality elim via (1) \\
$p_1 * (r * r') = p_3$                & Associativity of $*$ \\
$\exists r''. p_1 * r'' = p_3$        & Existential intro \\
$p_1 \sqsubseteq p_3$                 & Def of $\sqsubseteq$ \\
\end{tabular}

\section{Heyting Algebras over Preorders}

Given any preorder $(P, \sqsubseteq)$, we can construct a Heyting
algebra by considering the set of its upward-closed subsets
$\upset{P}$:

\begin{displaymath}
\upset{P} = 
  \{ S \in \mathcal{P}(P) \;|\;
     \forall p \in S, \forall q \in B. \mbox{ if } p \sqsubseteq q \mbox{ then } q \in S 
  \}
\end{displaymath}

The ordering relation for the Heyting algebra is set inclusion, and
the operations are:

\begin{itemize}
\item $\top = P$
\item $\bot = \emptyset$
\item $S \land S' = S \cap S'$
\item $S \vee S' = S \cup S'$
\item $S \implies S' = \{ r \in P \;|\; \forall r' \sqsupseteq r.\; 	
                          \mbox{if } r' \in S \mbox{ then } r' \in S' \}$
\end{itemize}

To show that these operations actually form a Heyting algebra, we need
to show that they satisfy the Heyting algebra axioms. It's immediately
evident that $\top$ and $\bot$ are the maximal and minimal elements of
the lattice. 

Now we need to show that the meet and join are the greatest lower
bounds and least upper bounds respectively.

So, first, we need to show that if $S_1, S_2 \in \upset{P}$, 
then $S_1 \land S_2 \in \upset{P}$. So we need to show that
the intersection of two upwards-closed subsets is itself an upwards-closed
subset. Suppose $x \in S_1 \land S_2$, and $y \sqsupseteq x$. Then:
\\

\begin{tabular}{ll}
$x \in S_1 \cap S_2$          & Definition of $\land$ \\
$x \in S_1$                   & Property of intersection (1) \\
$x \in S_2$                   & Property of intersection (2) \\
$y \in S_1$                   & By (1), and $S_1$ is upwards-closed \\
$y \in S_2$                   & By (2), and $S_2$ is upwards-closed \\
$y \in S_1 \cap S_2$          & Property of intersection \\
$y \in S_1 \land S_2$         & Definition of $\land$ \\
\end{tabular}
\\

Now we need to verify that that the intersection is the greatest lower
bound. Suppose we have an $R$ such that $R \subseteq S_1$ and $R
\subseteq S_2$. Then every element of $R$ must be in $S_1 \cap S_2$, which
means that $S_1 \cap S_2$ is the greatest lower bound. 

Next, we need to show that if $S_1, S_2 \in \upset{P}$, 
then $S_1 \vee S_2 \in \upset{P}$. Suppose that 
$x \in S_1 \vee S_2$ and that $y \sqsupseteq x$. Then:
\\

\begin{tabular}{ll}
$x \in S_1 \cup S_2$               & Definition of $\vee$ \\
$x \in S_1 \mbox{ or } x \in S_2$  & Property of union \\
Suppose $x \in S_1$:               & \\
\qquad $y \in S_1$                 & $S_1$ is upwards-closed \\
\qquad $y \in S_1 \cup S_2$        & property of union \\
\qquad $y \in S_1 \vee S_2$        & Def of $\vee$ \\
Suppose $x \in S_2$:               & \\
\qquad $y \in S_2$                 & $S_2$ is upwards-closed \\
\qquad $y \in S_1 \cup S_2$        & property of union \\
\qquad $y \in S_1 \vee S_2$        & Def of $\vee$ \\
$y \in S_1 \vee S_2$               & Or-elim \\
\end{tabular}
\\

Now, we verify that the union is the least upper bound. Suppose that
we have an $R$ such that $S_1 \subseteq R$ and $S_2 \subseteq R$. Then,
since each element of $S_1 \cup S_2$ is either an element of $S_1$
or $S_2$, each element of $S_1 \cup S_2$ is also an element of $R$. So
it's a least upper bound. 

Now, we have to show that if $S_1, S_2 \in \upset{P}$, then
$S_1 \implies S_2 \in \upset{P}$. So, suppose that 
$x \in S_1 \implies S_2$, and that $y \sqsupseteq x$. 
\\

\begin{tabular}{ll}
$x \in \{ r \in P \;|\; \forall r' \sqsupseteq r.\; 	
                          \mbox{if } r' \in S_1 \mbox{ then } r' \in S_2 \}$ &
Def of $S_1 \implies S_2$ \\

$\forall r' \sqsupseteq x.\; 	
    \mbox{if } r' \in S_1 \mbox{ then } r' \in S_2$ &
Comprehension elim (1)
\\

Assume $z \sqsupseteq y$. & \\

\qquad $z \sqsupseteq x$ & 
Since $z \sqsupseteq y$ and $y \sqsupseteq x$ \\

\qquad $\mbox{if } z \in S_1 \mbox{ then } z \in S_2$
& Instantiate universal quant in (1) \\

$\forall z \sqsupseteq y.\; 	
\mbox{if } z \in S_1 \mbox{ then } z \in S_2$ &
Universal Intro. \\

$y \in \{ r \in P \;|\; \forall z \sqsupseteq r.\; 	
                          \mbox{if } z \in S_1 \mbox{ then } z \in S_2 \}$ &
Comprehension intro \\

$y \in S_1 \implies S_2$ 
& Def. of $\implies$ \\
\end{tabular}

Now that we know that $\implies$ has the correct codomain, we need to
verify that it satisfies the following axiom: 

\begin{displaymath}
S_1 \land S_2 \subseteq R \iff S_1 \subseteq S_2 \implies R
\end{displaymath}

This is equivalent to showing that 

\begin{displaymath}
(\forall x.\; x \in S_1 \land S_2 \Rightarrow x \in R) \iff
(\forall x.\; x \in S_1 \Rightarrow x \in S_2 \implies R)
\end{displaymath}

First, let's show the $\Rightarrow$ direction.
\\

\begin{tabular}{ll}
Assume $\forall x.\; x \in S_1 \land S_2 \Rightarrow x \in R$ &
(1)
\\

Assume $x \in S_1$ &
(2)
\\

Assume $r' \sqsupseteq x$ &
(3)
\\

Assume $r' \in S_2$ & 
(4)
\\

$r' \in S_1$ & 
Since $x \in S_1$ and $r' \sqsupseteq x$ \\

$r' \in S_1 \cap S_2$ & 
Since $r' \in S_1$ and $r' \in S_2$ \\

$r' \in R$ &
By assumption (1) \\

$r' \in S_2 \Rightarrow r' \in R$ &
Implication intro (4) \\

$\forall r' \sqsupseteq x.\; r' \in S_2 \Rightarrow r' \in R$ &
Universal intro (3) \\

$x \in \{ r \in P \;|\; \forall r' \sqsupseteq r.\; r' \in S_2 \Rightarrow r' \in R$ &
Comprehension intro \\

$x \in S_2 \implies R$ &
Definition of $\implies$ \\

$\forall x.\; x \in S_1. \Rightarrow x \in S_2 \implies R$ &
Universal intro (2) \\

$(\forall x.\; x \in S_1 \land S_2 \Rightarrow x \in R) \Rightarrow (\forall x.\; x \in S_1 \Rightarrow x \in S_2 \implies R)$ &
Implication intro (1) \\
\end{tabular}
\\

Next, let's show the $\Leftarrow$ direction. 
\\

\begin{tabular}{ll}
Assume $\forall x.\; x \in S_1 \Rightarrow x \in S_2 \implies R$ &
(1) \\

Assume $x \in S_1 \land S_2$ & 
(2) \\

$x \in S_1$ & 
Since $x \in S_1 \land S_2$ (2)\\

$x \in S_2$ & 
Since $x \in S_1 \land S_2$ (3) \\

$x \in S_2 \implies R$ & 
By (2) and (1) \\

$x \in \{ r \in P \;|\; \forall r' \sqsupseteq r.\; \mbox{if } r' \in S_2 \mbox{ then } r' \in R \}$ &
Definition of $\implies$ \\

$\forall r' \sqsupseteq x.\; \mbox{if } r' \in S_2 \mbox{ then } r' \in R$ &
Comprehension instantiation (4) \\

$x \sqsupseteq x$ & 
Reflexivity (5) \\

$\mbox{if } x \in S_2 \mbox{ then } x \in R$ & 
Instantation of (4) with (5) \\

$x \in R$ & 
Implication elim via (3) \\

$\forall x.\; x \in S_1 \land S_2 \Rightarrow x \in R$ & 
Universal/Implication intro (2) \\

$(\forall x.\; x \in S_1 \Rightarrow x \in S_2 \implies R) \Rightarrow (\forall x.\; x \in S_1 \land S_2 \Rightarrow x \in R)$ & 
Implication intro (1) \\
\end{tabular}
\\

This establishes that $\upset{P}$ forms a Heyting algebra. 

\section{Specifications and the Higher Order Frame Rule}

Now, since we know that given a BI algebra $B$, $W(B)$ forms a preorder, and
that for any preorder its upward-closed subsets form a Heyting algebra, we
can immediately see that $\upset{W(B})$ is a Heyting algebra. 

This will give us the basic semantics of specifications. Now, define an 
operation $S \otimes p$, where $S \in \upset{W(B})$ and
$p \in W(B)$:

\begin{displaymath}
S \otimes p = \{ r \in W(B) \;|\; r * p \in S \}
\end{displaymath}

Now, we can show that $S \otimes p$ is in $\upset{W(B})$, and that 
$S \subseteq S \otimes p $. The reason to do this is to give a semantics
to the frame rule. Intuitively, if $S$ is a specification, then $S \otimes p$ 
will be that specification with $p$ framed on to it. So we need to show that
first, $S \otimes p$ actually is an element of our Heyting algebra of 
specification truth values, and second, we want the frame rule to be 
sound -- we want $S$ to always imply $S \otimes p$. 

To show that $S \otimes p \in \upset{W(B)}$, we need to show that for all 
$x,y$, if $x \in S \otimes p$ and $y \sqsupseteq x$, then $y \in S \otimes p$. 
\\

\begin{tabular}{ll}
Assume $x \in S \otimes p$                          & (1) \\
Assume $x \sqsubseteq y$                            & (2) \\
$x \in \{ r \in \upset{W(B)} \;|\; r * p \in S \}$  & Def of $\otimes$ \\
$x * p \in S$                                       & Comprehension elim \\
$\exists r' \in W(B). y = x * r'$                            & Def of $\sqsubseteq$ via (2)\\
Suppose $r' \in W(B)$.                              & Existential elim. \\
$y = x * r'$                                        & (3) \\
$x * p * r' \in S$                                  & $S$ upward-closed \\
$(x * r') * p \in S$                                & Assoc, commutativity \\
$y * p \in S$                                       & Equality via (3) \\
$y \in \{r \in \upset{W(B)} \;|\; r * p \in S \}$   & Comprehension intro \\
$y \in S \otimes p$                                 & Def. of $\otimes$ \\
\end{tabular}
\\


To show $S \subseteq S \otimes p$, we need to show that if $x \in S$, then $x \in S \otimes p$. 
\\

\begin{tabular}{ll}
Assume $x \in S$                      & (1) \\
$x \sqsubseteq x * p$                 & Property of $\sqsubseteq$ \\
$x * p \in S$                         & $S$ is upward-closed \\
$x \in \{ r \in \upset{W(B)} \;|\; r * p \in S \}$ & Comprehension intro \\
$x \in S \otimes p$                   & Definition of $\otimes$ \\ 
If $x \in S$ then $x \in S \otimes p$ & Implication introduction via (1) \\ 
\end{tabular}
\\


