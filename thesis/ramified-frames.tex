\chapter{Ramified Frame Properties}

\section{Introduction}

In many interactive programs, there are mutable data structures which
change over time, and which must maintain some relationships with one
another. For example, in a web browser, we need to present a web page
both as a tree data structure for scripts to manipulate, and 
display a graphical image for the human user to view. Any change made
to the tree by a script must be reflected in a change to the image
that the human sees --- the two structures must remain synchronized.

Likewise, in a spreadsheet, each cell contains a formula, which may
refer to other cells, and whenever the user changes a cell, all of the
cells which transitively depend upon it must be updated. Since
spreadsheets can get very large, this should ideally be done in a lazy
way, so that only the cells visible on the screen, and the cells
necessary to compute them, are themselves recomputed.

Typically, these dependencies are written using what is called the
\emph{subject-observer} pattern. A mutable data structure (the
subject) maintains a list of all of the data structures whose
invariants depend upon it (the observers), and whenever it changes, it
calls a function on each of those observers to update them in response
to the change. (And in turn, the observers of the subject may be
subjects of still other observers, ultimately forming DAGs of
notifications.)

While natural, these programs are very challenging to verify in a
modular way, even when using a resource-sensitive logic adapted to
reasoning about aliased mutable data, such as separation logic. The
reason is that there are two directions of dependency, both of which
matter for program proof. First, our program invariant must have
ownership over the subject's data (its \emph{footprint}) in order to
prove the correctness of code modifying the subject. This direction
of ownership is natural to verify with separation logic. 

However, we must explicitly maintain the \emph{other} direction of
dependency as well --- we track everything which depends upon the
subject, and modify them appropriately whenever the subject changes.
However, the natural program invariant now becomes a global property:
we need to know the full dependency graph covering all subjects and
observers, so that we can say that the reads and is-read-by relations
are relational transposes of one another. The global nature of this
invariant means that a naive correctness proof will not respect the
modular structure of the program --- if we modify the dependency graph
in any way, we now have to re-verify the entire program!

However, the intention of the subject-observer pattern is
precisely to allow the program to remain oblivious to the exact number
and nature of the observers, which allows the programmer to add new
observers without disturbing the behavior of the rest of the program.
Our goal, then, is to find a way of taking this piece of practical
software engineering wisdom, and casting it into formal terms amenable
to proof.

Concretely, our contributions are as follows: 

\begin{itemize}
  \item We define a library with a monadic API for writing
    demand-driven computations with dynamic dependencies and local
    state, and which is implemented as higher-order functions
    dynamically creating networks of imperative callbacks

    We then give an ``abstract semantics'' for this
    library, structured as a set of separation logic lemmas about our
    dataflow library. These lemmas permit \emph{modular} correctness
    proofs about programs using this API, even in the face of the fact
    that the program invariants must be defined globally upon the
    whole callback network.

    The key idea is to distinguish between the direct footprint of a
    command, and the program state which can depends upon that
    footprint. The lemmas are then phrased so that they refer only to
    the direct footprint of each command in the API. In addition, we
    structure our lemmas to justify an unusual frame property for our
    abstract semantics, which we can use to verify different parts of
    an imperative dataflow network separately.

    Unlike typical frame properties, the frame in our frame rule is
    not the same in the pre- and the post-states. Instead, the two
    sides of the frame are related by a \emph{ramification operator}
    (so named in analogy to the ``ramification problem'' in AI), which
    explains how local changes can alter our knowledge of the rest of
    the heap.

  \item To illustrate the utility of this proof technique, we verify
    an imperative implementation of combinators implementing stream
    transducers in the style of functional reactive programming. 

    Ultimately, clients can reason about the behavior of the imperative
    implementation ``as if'' it were purely functional, even though it
    is implemented using local state and imperative callback
    procedures.
\end{itemize}

\section{Demand-Driven Notification Networks}

A simple intuition for a ``demand-driven notification network'' is to
think of it as a generalized spreadsheet. We have a collection of
cells, each of which contain program expressions whose evaluation may
refer to other cells. When a cell is read, the expression within the
cell is evaluated, possibly triggering the evaluation of other cells
in a cascade. Furthermore, each cell memoizes its expression, so that
repeated reads won't trigger re-evaluation, and maintains a set of
dependencies so that when the code within a cell is changed, it and
everything that depends on it invalidates the memoized value.

\subsection{Implementing Notification Networks}

\begin{figure}
\mbox{}
\begin{specification}
\nextline $\codetype{A} = \monad{(A \times \cellset)}$ \nextline[1em]
$\celltype{A} = \{$\=$code: \reftype{\codetype{A}};$ \nextline
                   \>$value: \reftype{\opttype{A}};$ \nextline
                   \>$reads: \reftype{\cellset};$ \nextline
                   \>$obs:   \reftype{\cellset}\;$ \nextline
                   \>$id:    \N\}$ \nextline[1em]

$ecell = \exists \alpha:\star.\; \celltype{\alpha}$ \nextline[1em]

$\ctext{unit} : \forall \alpha:\star.\; \alpha \to \codetype{\alpha}$ \nextline
$\ctext{unit}\;x = \comp{\pair{x}{\ctext{emptyset}}}$ \nextline[1em]

$\ctext{bind} : \forall \alpha,\beta:\star.\; \codetype{\alpha} \to (\alpha \to \codetype{\beta}) \to \codetype{\beta}$ \nextline
$\ctext{bind}\;e\;f = [$\=$\letv{(v,r_1)}{e}{}$ \nextline
                        \>$\letv{(v',r_2)}{f\;v}{}$ \nextline
                        \>$\;\pair{v'}{\ctext{union}\;r_1\;r_2}]$\nextline[1em]

$\ctext{read} : \forall \alpha:\star.\; \celltype{\alpha} \to \codetype{\alpha}$ \nextline
$\ctext{read}\;a = [$\=$\letv{o}{\comp{!a.value}}{}$ \nextline
                     \>$\ctext{run}\;\ctext{case}(o,$ \nextline
                     \>\qquad\= $\ctext{Some}\;v \to \comp{\pair{v}{\ctext{singleton}\;a}},$ \nextline
                     \>      \> $\ctext{None} \to [$\=$\letv{exp}{\comp{!a.code}}{}$ \nextline
                     \>      \>                     \>$\letv{(v,r)}{exp}{}$ \nextline
                     \>      \>                     \>$\letv{\_}{\comp{a.value := v}}{}$\nextline
                     \>      \>                     \>$\letv{\_}{\comp{a.reads := r}}{}$ \nextline
                     \>      \>                     \>$\letv{\_}{\ctext{setiter}\;(\ctext{add\_observer}\;a)\;r}{}$ \nextline
                     \>      \>                     \>$\;\pair{v}{\ctext{singleton}\;a}])$ \nextline[1em]

$\getref : \forall \alpha : \reftype{\alpha} \to \codetype{\alpha}$ \nextline
$\getref r = \comp{\letv{v}{\comp{!r}}{\pair{v}{\ctext{emptyset}}}}$ \nextline[1em]

$\setref : \forall \alpha : \reftype{\alpha} \to \alpha \to \codetype{\unittype}$ \nextline
$\setref r\;v = \comp{\letv{dummy}{\comp{r := v}}{\pair{\unit}{\ctext{emptyset}}}}$ \nextline[1em]

$\newcell : \forall \alpha:\star.\; \codetype{\alpha} \to \monad{\celltype{\alpha}}$ \nextline
$\newcell\;\alpha\;code = [$\=$\letv{id}{!counter}{}$ \nextline
                                   \>$\letv{\_}{\comp{counter := n + 1}}{}$ \nextline
                                   \>$\letv{code}{\newref{\codetype{\alpha}}{code}}{}$ \nextline
                                   \>$\letv{value}{\newref{\opttype{\alpha}}{\ctext{None}}}{}$ \nextline
                                   \>$\letv{reads}{\newref{\cellset}{\ctext{emptyset}}}{}$ \nextline
                                   \>$\letv{obs}{\newref{\cellset}{\ctext{emptyset}}}{}$ \nextline
                                   \>$\; (code, value, reads, obs, id)]$ 
\end{specification}
\caption{Implementation of Notification Networks}
\label{notification-implementation}
\end{figure}

\begin{figure}
\mbox{}
\begin{specification}
\nextline
$\updatecell : \forall \alpha:\star.\; \codetype{\alpha} \to \celltype{\alpha} \to \monad{\unittype}$\nextline
$\updatecell\;\alpha\;exp\;a = 
     [$\=$\letv{\_}{\ctext{mark\_unready}\;\pack{\alpha}{a}}{}$ \nextline
       \>$a.code := exp]$ \nextline[1em]

$\ctext{mark\_unready} : ecell \to \monad{\unittype}$ \nextline
$\ctext{mark\_unready}\;cell =\; $\=$\ctext{unpack}(\alpha, a) = cell\; \ctext{in}$\nextline
\>  $[$\=$\letv{os}{\comp{!a.obs}}{}$ \nextline
\>     \>$\letv{rs}{\comp{!a.reads}}{}$ \nextline
\>     \>$\letv{\_}{\ctext{iterset}\;\ctext{mark\_unready}\;os}{}$ \nextline
\>     \>$\letv{\_}{\ctext{iterset}\;(\ctext{remove\_obs}\;cell)\;rs}{}$ \nextline
\>     \>$\letv{\_}{\comp{a.value := \ctext{None}}}{}$ \nextline
\>     \>$\letv{\_}{\comp{a.reads := \ctext{emptyset}}}{}$ \nextline
\>     \>$a.obs   := \ctext{emptyset}]$ 
\end{specification}
\caption{Implementation of Notification Networks}
\label{notification-implementation-update}
\end{figure}

Our API for creating notification networks is given in
Figure~\ref{notification-implementation}. First, we'll describe
the interface, and then discuss its implementation. 

The interface exposes two basic abstract data types, $\ctext{cell}$
and $\ctext{code}$.

The type $\celltype{A}$ is the type of dynamic data values. A cell
contains a reference to a piece of code, a possible memoized value,
plus enough information to correctly invalidate its memoized value
when the cell's dependencies change. We can create a new cell by
calling $\newcell e$, which returns a brand new cell with the
code expression $e$ inside it. We can also modify a cell with 
the command $\updatecell\;cell\;exp$, which modifies the cell
$cell$ by installing the new expression $exp$ in it. 

The type $\codetype{A}$ is a monadic type, representing the type of
computations that can read cells. It supports the usual operations
$\return e$ and $\bind e\; (\semfun{x}{e'})$, which embed a pure value
into the $\ctext{code}$ type and implement sequential composition,
respectively. In addition, the primitive operations on this monad 
include reading a cell with the $\readcell cell$ function call, and 
reading and modifying local state with the $\getref r$ and $\setref r\;v$
operations. 
 
The actual implementation is also given in
Figure~\ref{notification-implementation}. The abstract type of code is
implemented using the underlying monad of imperative commands, so that
$\codetype{A}$ is implemented with the type $\monad{(A \times
  \cellset)}$.  The intuition is that when we evaluate a term we are
allowed to read some cells along the way, and so must return a set of
all the cells that we read in order to do proper dependency
management. So, $\cellset$ is a type representing sets of
(existentially-quantified) cells.  (The precise specification of
$\cellset$ is given in Appendix A, since describing it is a
distraction from the main development.)

Cells are represented with a 5-tuple. There is a reference pointing to
the code expression, as well as a value field pointing to an optional
value. The value field will be set to $\ctext{None}$ if the cell is in
an unready, un-memoized state, and will be $\ctext{Some}\;v$ if the
cell's code has already been evaluated to a value $v$. In addition there are
two fields representing the dependencies. If the code expression has
been evaluated and a memoized value generated, then the $reads$ field
will point to the set of cells that the computation directly read
while computing its value. Otherwise it will point to the empty
set. Conversely, the field $obs$ contains the cell's observers --- the
set of cells that have read the current cell as part of their own
computations. Obviously, this is only non-empty when the cell has been
evaluated. Finally, each cell also has an integer id field, which is a
unique numeric identifier for every cell that is created by the
library. It allows us to compare cells (even of different type) for
equality, which is needed for dependency management.

The $\return$ operation for the library simply returns its argument
value and the empty set, since it doesn't read any cells. Likewise,
$\bind e\;f$ will evaluate the first argument and pass the returned
value to the function $f$. It will return the function's return value,
together with the union of the two read sets. $\getref r$ and $\setref r\;v$
simply read and update the reference, and return empty read sets, since
neither of them read any cells. 

Interesting things first happen with the $\readcell e$ operation. This
function will first check to see if the cell has a memoized value
ready. If it does, we return that immediately. Otherwise, we evaluate
the cell's code, and update the current cell's value and read set. In
addition, each cell that was read in the evaluation of the code (i.e.,
the set returned as the second component of the monadic type's return
value) also has its observer set updated with the newly-ready current
cell. Now, if any of the dependencies change, they will be able to
invalidate the current cell, which observes them. Note that the
dependencies between cells are all dynamic --- we cannot examine the
inside of a code expression to find its ``free cells'', and so we rely
upon the invariant that a code expression will return every cell it
read, in addition to its return value.

Further interesting things happen with the $\newcell\;cmd$
operation.  It creates a new cell value, setting the code field with
the argument $cmd$, and generating an id by dereferencing and
incrementing the variable $counter$. The $counter$ variable occurs
freely in this definition, because it is a piece of state global to
this module, and must be initialized by whatever initialization
routine first constructs the whole module as an existential package.
Since $counter$ is otherwise private, we can generate unique
identifiers by incrementing it as we create new cells. 

Finally, the $\updatecell\;cell\;cmd$ operation updates a cell
$cell$ with a new code expression $cmd$. (As an aside, it's worth
noting that this is a genuine, unavoidable, use of higher-order store:
we make use of pointers to code, including the ability to dynamically
modify them.) Once we modify a cell, any memoized value it has is no
longer necessarily correct.

Therefore, we have to drop the memoized value of the cell, and any
cell that transitively observes the cell. This is what the
$\ctext{mark\_unready}$ function does. Given a cell, it takes all of the
observers of the current cell and recursively makes all of
them unready. Then it removes the current cell from the observer sets of
all the cells it reads, and then it nulls out the current cell's
memoized value, as well as setting its read and observer sets to
empty. Notice that there is no explicit base case to the recursive
call; if there are any cycles in the dependency graph, invalidation
could go into an infinite loop. 

So far, we have described the implementation invariants incrementally.
Before proceeding to describe them formally, we will state them again
informally, all in one place:

\begin{itemize}
  \item Every cell must have a unique numeric identifier
  \item Every cell is either ready, or unready. 
  \item Every ready cell has a memoized value, and maintains 
    two sets, one containing every cell that it reads, and the
    other containing every cell it is observed by. 
  \item Every unready cell has no memoized value, and has 
    both an empty read set and an empty observer set. 
  \item The overall dependency graph among the valid cells must form a
    directed acyclic graph. 
  \item The reads and the observers must be the same, only 
    pointing in opposite directions.
\end{itemize}

Formalizing these constraints is relatively straightforward, but we
have the problem that these constraints are global in nature:
we can't be sure that the dependency graph is acyclic without having
it all available to examine, and likewise we can't in general know
that a cell is in the read set of everything in its observed set
without knowing the whole graph. Handling this difficultly is one of
the primary contributions of this work. 

\section{The Abstract Semantics of Notifications}

\subsection{The Structure of the Global Invariant}

The key to getting around our difficulties lies in the difference
between the implementation of $\updatecell$, and of $\ctext{read}$.
$\updatecell$ calls $\ctext{mark\_unready}$, which recursively follows
the observers. The $\ctext{read}$ function, on the other hand,
proceeds in the opposite direction --- it evaluates code expressions,
recursively descending into the footprint of its command. The opposite
direction these two functions look is why we end up needing a global
invariant: we need to know that these two directions are in harmony
with one another.

Now, note that the type of $\ctext{mark\_unready}$ is simply the monadic
type $\monad{\unittype}$, which precludes it from being called from within
a $\codetype{A}$. This means that when we are evaluating a code
expression, we will never actually follow the observer fields --
we'll only patch them as needed, whenever we evaluate a
cell and change it from unready to ready.  As a result, an abstract
description of the heap which \emph{does not explicitly mention
the observer sets} will prove sufficient for reasoning about
the behavior of $\codetype{A}$ expressions.

With this plan, we introduce \emph{abstract heap formulas}, which are
syntactic descriptions of the state of part of the cell heap. These
syntactic expressions are given by the following grammar:
\begin{displaymath}
  \begin{array}{lcl}
    \phi & ::= & I \bnfalt \phi \otimes \psi \bnfalt \cellpos{a}{e}{v}{r} \bnfalt \cellneg{a}{e}\\
         &  |  & \localref{r}{v} 
  \end{array}
\end{displaymath}

Informally, a formula $I$ represents an empty abstract heap, and a
formula $\phi \otimes \psi$ represents an abstract heap that can be
broken into two disjoint parts $\phi$ and $\psi$. We will only
consider formulas modulo the associativity and commutativity of
$\otimes$, and take $I$ to be the unit of this binary operator.

The atomic form $\localref{r}{v}$ says that $r$ is a piece of local
state owned by the network, currently with value $v$. There are two
atomic forms representing cells. $\cellneg{a}{e}$ says that $a$ is a
cell with code $e$, which is unready to deliver a value --- it needs
to be re-evaluated before it can yield a value. $\cellpos{a}{e}{v}{r}$
says that $a$ is a cell with code $e$, and that may be ready to deliver
the value $v$, assuming that all the cells in its read set $r$ are themselves
ready. If anything it reads is known to be unready, then it is unready
itself. (We will sometimes write $\celleither{a}{e}$ when we do not care
whether $a$ is ready or not.)

First, notice the must/may flavor of this reading. The formula
$\cellneg{a}{e}$ says that $a$ \emph{must} be unready.  The formula
$\cellpos{a}{e}{v}{r}$ says that $a$ \emph{may} have be ready,
conditional on the readiness of the elements of its read set
$r$. Second, notice that the backwards dependencies are entirely
missing from these formulas. We have simply left out the other half of
the dependency graph from this description. Forgetting this
information will be critical for local reasoning.

We have emphasized that the straightforward invariant is not obviously
modular. Before we can elaborate any further on this point, we will
need to look at the formal statement of the heap invariant. We
introduce the predicate $G(\phi)$, which describes the entire heap of
cells allocated by our library, and which satisfy both the conditions
described in the previous section, and the additional constraint that
the cell heap agree with $\phi$.\footnote{This is why we insisted that
  the abstract heap formulas are syntactic objects --- this permits us
  to define predicates on them by induction over the structure of the
formula.}


\begin{tabbing}
    $G(\phi) \triangleq \exists H \in CellHeap.\; Inv(H, \phi)$ \\ [1em]

$Inv(H, \phi) \triangleq $ \\
\;\;\= $$\=$ R_H^\dagger = O_H \land R_H^+ \mbox{ strict partial order }$ \\
    \> \> $\land\; R_H \subseteq V_H \times V_H \land unique(H)$ \\
    \> \> $\land\; \satisfies{H}{\phi} \land heap(H)$ \\
\end{tabbing}

The auxiliary definitions we used in this definition are all given in
Figure~\ref{heap-invariant}. 

We first assert the existence of a cell heap $H$, which is a
collection of cells, together with a function mapping each cell to a
code expression, a possible value, a read set, an observed set, and an
identifier, which satisfies the overall invariant $Inv$.

In the first two lines of $Inv(H, \phi)$, we assert all of the global 
conditions in terms of the mathematical cell heap $H$. First, we assert that 
the relational transpose $(\cdot)^\dagger$ of the reads relation $R_H$ is the
observes relation $O_H$, which enforces the condition that the reads
and observe relations be the same, only pointing in opposite
directions (i.e., if $a$ reads $b$, then $b$ is observed by
$a$). Then, we require that the transitive (but not reflexive) closure
of the reads relation, $R^+_H$ form a strict partial order. Strictness
enforces the condition that there be no cycles in the dependence graph
(because otherwise there would be elements $a$, such that $(a, a) \in
R^+_H$).  Next, we require that the reads relation $R_H$ is a subset of
the Cartesian product $V_H \times V_H$ of the set of cells with values
$V_H$. This ensures that a) there are no dependencies on unready
cells, and b) all unready cells have empty read and observe
sets. Finally, we ask that all of the cells in $H$ have unique
identifiers -- $unique(H)$ asserts that the identifier field is one
half of an isomorphism between the cells in $H$, and the natural
numbers from 0 to the size of $H$.

In the third line, we begin by requiring that the cell heap $H$ satisfy
the abstract heap formula $\phi$, which formalizes the informal
reading of the abstract heap formulas given earlier. This satisfaction
relation follows the general style of separation logic quite closely; 
in a sense, our abstract heap formulas are giving a domain-specific
version of separation logic.

The last clause $heap(H)$ finally connects the cell heap, which is a purely
mathematical object, to the actual low-level heap the implementation
uses. We ask that the global counter reference $counter$ point to an integer
field equal to the size of the cell heap, and then use the iterated
separating conjunction $\forall^*$ to require that for each cell in
the cell heap, we have pointers to the appropriate code, value, read
and observer fields. The read and observer fields point to values of
type $\cellset$, each of which represent the appropriate sets of cells. 


\begin{figure}

\begin{tabbing}
$CellHeap = \Sigma$\=$ D \in \powersetfin{ecell}.$ \\
               \>$(\Pi (\pack{\alpha}{\_}) \in D. ($\=${\codetype{\alpha}} \;\times$
                                         ${\opttype{\alpha}} \;\times$ 
                                         ${\powersetfin{ecell}} \;\times$ 
                                         ${\powersetfin{ecell}} \;\times$ 
                                         $\N)$ \\[1em]
       

$code = \pi_1$ \\
$value = \pi_2$ \\
$reads = \pi_3$ \\
$obs = \pi_4$ \\
$identity = \pi_5$ \\[1em]

$V_{(D, h)} = \comprehend{c \in D}{\exists v.\; value(h(c)) = \ctext{Some}(v)}$ \\
$R_{(D, h)} = \comprehend{(c,c') \in D}{ c' \in reads(h(c)) }$ \\
$O_{(D, h)} = \comprehend{(c,c') \in D}{ c' \in obs(h(c)) }$ \\[1em]

$unique(D,h) = \exists$\=$i : Fin(|D|) \to D.$\\
                       \>$i \circ (identity \circ h) = id \; \land (identity \circ h) \circ i = id$ \\[1em]

$\satisfies{(D,h)}{\phi} = sat((D,h), D, h)$ \\[1em]

$sat(H, D, \localref{r}{v}) = \top$ \\
$sat(H, D, I) = \top$ \\
$sat(H, D, \phi \otimes \psi) = \exists D_1, D_2.\;$\=$D = D_1 \uplus D_2 
                                                      \land sat(H, D_1, \phi) 
                                                      \land sat(H, D_2, \psi)$ \\
$sat(H, D, \cellneg{a}{e}) = a \in D \land code(h(a)) = e \land a \not \in V_H$ \\
$sat(H, D, \cellpos{a}{e}{v}{r}) =\; $\=$ a \in D \land code(h(a)) = e \;\land$ \\
      \>$\mathrm{if}($\=$r \cap V_H = r,$ \\
      \>     \>$value(h(a)) = \ctext{Some}\;v \land reads(h(a)) = r,$ \\
      \>     \>$a \not\in V_H)$\\[1em]
 

$heap(D,h) = $ \\
\;\;$counter \pointsto |D| \;* $ \\
\;\;$\forall^* c \in D.\;$\=$\exists v_r, v_o : \cellset.\;$ \\
                         \>$c.code \pointsto code(h(c))   \;* $ \\
                         \>$c.value \pointsto value(h(c)) \;* $ \\
                         \>$c.reads \pointsto v_r \;* $ \\
                         \>$c.obs   \pointsto v_o \;* $ \\
                         \>$c.id    = identity(h(c)) \;\land$ \\
                         \>$set(D, v_r, reads(h(c))) \;\land$ \\
                         \>$set(D, v_o, obs(h(c)))$ 
\end{tabbing}

\caption{Definitions for Heap Invariant}
\label{heap-invariant}
\end{figure}


The global character of this invariant should be evident; we need to
talk about \emph{all} of the cells in the heap in order to state our
invariants. So it is not immediately clear that we have made much
progress towards a modular proof technique. However, we are actually
very close: with just two more ideas, we will be able to give a
solution to this problem.

\subsection{Frame Properties via Polymorphism}

As we mentioned earlier, our abstract heap formulas essentially give
us a small domain-specific separation logic. This means that in order
to reason locally over cell heaps, we need to find an
application-specific version of the frame rule for our library. 

To do this, we will adapt some ideas proposed by
Benton~\cite{benton}, and Birkedal and Yang~\cite{birkedal-yang}. They
suggested interpreting the frame rule of separation logic as a form of
quantification -- instead of having a separate frame rule that allows
adding a frame to any triple, they proposed that all of the atomic
rules of the program logic be replaced with rules possessing an extra
quantifier ranging over ``the rest of the heap'':

\begin{mathpar}
  \inferrule*[right=Example]
          { }
          { \forall R.\; \setof{ (e \pointsto v) * R } \;e := v'\; \setof{ (e \pointsto v') * R}}
\end{mathpar}

This quantifier is propagated through the proof, and any use of the
frame rule can be interpreted as instantiating the universal
quantifier appropriately. The reason this idea is fruitful for us is
that it will allow us to give a frame rule, even though the underlying
semantics of our library does not actually satisfy any analogues of
the traditional safety monotonicity and frame lemmas. For example, the
$\updatecell$ operation certainly does not act locally -- it
recursively traverses the observers set, possibly mutating a very
large part of the cell graph.

Nonetheless, we can prove the soundness of the following triple
specifying $\updatecell$.

\begin{prop}{(Update Rule)}
For all cells $o$ and code expressions $e$ and $e'$, the following
triple is provable: 

\begin{tabbing}
$\forall \psi:\formula.\; $\=$\setof{G(\celleither{o}{e} \otimes \psi)}$ \\
                           \>$\runcmd {\updatecell\;o\;e}$ \\
                           \>$\setof{a:1.\; G(\cellneg{o}{e} \otimes \psi)}$
\end{tabbing}
\end{prop}

\begin{proof}
\\\textbf{Proof Strategy.} The key to this proof is the conditional
interpretation of the $\cellpos{c}{e}{v}{r}$ formula. When we evaluate
the $\updatecell$ command, we recursively find every cell which
depends on $o$, and modify it to be unready.

Now consider any positive cell formula in $\psi$ which reads $o$,
directly or indirectly. The satisfaction relation for $\phi$ tells us
that a positive formula must have everything in its read set be
ready, in order to represent a ready cell. So by changing $o$'s
formula to the unready state, we no longer require that positive
formula to represent a ready cell. As a result, we can leave the
entire frame $\psi$ untouched, even though the physical heap it
represents may have been (quite drastically) modified. 

\textbf{Proof Details.} We will not give this proof as an annotated
specification, since the invariant is so large that repeatedly copying
it from precondition to postcondition would make the annotation
unreadable.

First, let's consider the specification of $\ctext{mark\_unready(c)}$,
which does the bulk of the work of the update function. The purpose of
this function is to set every cell transitively reachable from $c$ to
the unready state. So, what we want is for the procedure to satisfyh
the specification

\begin{displaymath}
\spec{\mathit{Inv}(H, \phi \otimes \celleither{c}{e})}{\ctext{mark\_unready}(c)}{a:1}{\mathit{Inv}(H', \phi \otimes \cellneg{c}{e})}
\end{displaymath}

In the above, supposing that $H = (D,h)$, we take $H'$ to be the pair
$(D, h')$, where we let $U = \comprehend{c' \in D}{(c, c') \in R^{*}_H}$, 
and define $h'$ with the following components. 
\begin{itemize}
\item $\mathit{code}(h') = \mathit{code}(h)$
\item $\mathit{value}(h') = \semfun{c'}{\IfThenElse{c'\in U}{\None}{\mathit{value}(h(c'))}}$
\item $\mathit{reads}(h') = \semfun{c'}{\IfThenElse{c'\in U}{\emptyset}{\mathit{reads}(h(c'))}}$
\item $\mathit{obs}(h') = \semfun{c'}{\IfThenElse{c'\in U}{\emptyset}{\mathit{obs}(h(c'))}}$
\item $\mathit{identity}(h') = \mathit{identity}(h)$
\end{itemize}


Note that we do not mention the $G$ predicate in this specification --
$\ctext{mark\_unready}$ is part of the implementation of the library
and has a specification in terms of invariants invisible to the
client.

Our next step will be to prove this function using the fixed point
induction rule, and precisely this specification. (In fact, the
function is totally correct, since it is always called on arguments
higher up in the partial ordering on cells, but tracking the
termination metric in detail obscures the key ideas.) 

Now we'll proceed line-by-line through the function body. To simplify
matters, we'll split into two cases, one for when the argument is
already unready, and one for when it is ready.

First, suppose the cell $c$ is unready --- that is, it is not an
element of $V_H$.  In this case, $H'$ is just the same as $H$. Then from the
definition of $\mathit{Inv}$ we know that its reads and observers sets
are empty, since $R_H \subseteq V_H \times V_H$.  So the
$\mathit{heap}(H)$ predicates tells us that the pointer dereferences
on lines 6 and 7 of Figure~\ref{notification-implementation-update}
bind the empty set to the $os$ and $rs$ variables. Then, by the
definition of $\mathit{iterseq}$, we know that the calls to
$\ctext{iterset}$ on lines 8 and 9 are equal to $\comp{\unit}$ and
hence no-ops.  Furthermore, the definition of $V_H$ means that
$\mathit{value}{h(c)} = \None$. So the assignments to
$c.\mathit{value}$ on line 10 does not change the contents of the
heap, and likewise neither do the assignments to $c.\mathit{reads}$
and $c.\mathit{obs}$ on lines 11 and 12. Since the heap is unchanged,
we know that $\mathit{Inv}(H,\phi)$ must continue to hold.

On the other hand, suppose that $c$ is in $V_H$. Then from the
definition of $\mathit{Inv}$ we know that $c.\mathit{reads}$ and
$c.\mathit{obs}$ point to sets which implement the reads and observers
sets of $c$. Then, the $\mathit{heap}(H)$ predicates tells us that the
pointer dereferences on lines 6 and 7 of
Figure~\ref{notification-implementation-update} bind values
corresponding to those sets to the $os$ and $rs$ variables, in the
current world $D$.

Now, the axioms describing $\ctext{iterseq}$ says that
$\ctext{iterseq\;mark\_unready}\;os$ is equal to a sequence of calls
of $\ctext{mark\_unready}$ to the elements of $os$ (which we will name
$c_1, \ldots, c_k$) in some order.  The net effect of such a sequence
of calls is to give us an $H_k$ which have removed $U - \setof{c}$
from $V_{H_k}$. However, this is not an entirely trivial fact to
prove, since each call to $\ctext{mark\_unready}(c_{i+1})$ starts from
a different state $H_{i}$. 

So we need a simple observation about (finite) partial orders. 

\begin{lemma}{(Commutativity and Idempotence of Deletion)}
Suppose $R$ is a partial order on a set $X$. Now, define $R - \uparrow(x)$ to 
be the set $\comprehend{y \in X}{\lnot R(x,y)}$. Then, it follows that 
$R - \uparrow(x) - \uparrow(y) = R - \uparrow(y) - \uparrow(x)$. 
Furthermore, $R - \uparrow(x) - \uparrow(x) = R - \uparrow(y) - \uparrow(x)$. 
\end{lemma}

These facts essentially follows immediately from the definition of set
comprehension, and imply that the order we remove transitive closures
from a partial order does not matter.

Then, observe that $h'$ in the specification of
$\ctext{mark\_unready}$ is defined by setting $U$ to $V_H -
\uparrow{c}$, and that $\mathit{value}$, $\mathit{reads}$, and
$\mathit{obs}$ simply take $h$ and restrict to the domain $V_H -
\uparrow{c}$. Since we call $\ctext{mark\_unready}$ on each element
directly above $c$ in the partial order, it follows that $V_{H_k} =
V_{H} - (U - \setof{c})$, with the functions updated in the evident
way.

This in turn means that there are no longer any elements of $D$ above
$c$ in $R_{H_k}$. So after setting $c.\mathit{value}$ to $\None$ and
removing $c$ from the observer sets of the elements of $os$, we know
that we have reached a state whose heap satisfies $\mathit{heap}(D,
h')$.

Now, it becomes relatively easy to prove the correctness of
$\ctext{update}$.  We perform the invalidation, which takes the
current cell $c$ and takes us into a state $H'$ in which the the cell
is unready (i.e., $V_{H'}(c) = \None$). Now, the only remaining challenge
is to show that $\cellneg{c}{e'} \otimes \psi$ is still satisfied by
the new model. This follows from an easy induction on $\psi$ -- as the
ready cells in the new heap are a strict subset of the ready cells of
the original heap, any negative cell formulas in $\psi$ stay negative,
and we only need to consider the case of positive cell formulas. 

Concretely, suppose we have a formula $\cellpos{c'}{e}{v}{r}$. If
$c' \not\in V_H$ (i.e., was not semantically ready in the original
heap) then it will not be in $V_{H'}$, since $V_{H'} \subseteq V_H$,
since its dependencies will not have gotten any more ready.

On the other hand, if $c' \in V_H$, then consider whether $c' \in U$.
If it is in $U$, then there will be path in $H$ from $c$ to $u$, which
means that both $c'$ will be semantically invalid, \emph{and} all of
the cells in $r$ (which lie on all the paths from $c$ to $c'$) are in
$U$, and hence unready. Then in this case, the clause of the
satisfaction relation holds. If $c' \not\in U$, then it also follows
that nothing in $r$ is in $U$, either, and so both $r$ and $c'$ are
semantically ready, in which case they continue to validate the other
case of the conditional in the definition of the satisfaction
relation.

\end{proof}


We can prove the soundness of a similar specification for $\newcell$ as
well:

\begin{prop}{(New Cell Rule)}
For all code expressions $e$, the following specification is provable: 
\begin{tabbing}
$\forall \psi:\formula.\; $\=$\setof{G(\psi)}$ \\
                           \>$\run {\newcell\;e}$ \\
                           \>$\setof{a:\celltype{A}.\; G(\cellneg{a}{e} \otimes \psi)}$
\end{tabbing}
\end{prop}

\begin{proof}
This is much easier than $\updatecell$: we just need to allocate a new
numeric id for the new cell, and show that the extended cell heap
continues to satisfy all of the expected properties.

Concretely, suppose that $G(\psi) = \mathit{Inv}{(H, \psi)}$ for some
$H = (D,h)$. Now, by the invariant, we know that each cell in $D$ has 
a unique numeric id somewhere in the range from $0$ to $|D|-1$. Then,
on lines 29-30, we increment the global counter by 1, and let $id = |D|$. 
Then, on lines 31-34, we allocate a pointer $\mathit{code}$ to the code $e$, a pointer
$\mathit{value}$ to $\None$ (to indicate the unitialized state), as well as pointers to 
$\mathit{read}$ and $\mathit{obs}$ to empty read and observer sets. 

At line 35, all of these values are bound to variables in scope, so we
can define $H' = (D',h')$, where $\mathit{ec} = \pack(\tau,\mathit{(code, value, read, obs, id)})$, 
and 
\begin{displaymath}
(D',h') = (D \cup \setof{\mathit{ec}}, h' = [h|\mathit{ec}: (e, \None, \emptyset, \emptyset, id)])
\end{displaymath}

Then, it is easy to show that the modified heap realizes $H'$ -- the
only fact that does not carry over immediately are the predicates
establishing the the read and observer sets are implemented
correctly. This is because they held for world $D$, and we are now in
world $D'$. Happily, we required as an axiom that $(W \subseteq W')
\implies \mathit{set}(W, v, S) \implies \mathit{set}(W', v, S)$.

Finally, we need to show that $\cellneg{a}{e} \otimes \psi$ is
satisfied by $H'$. This follows from an easy induction on $\psi$:
since $h'$ is just an extension of $h$ with $V_{H'} = V_H$, none
of the conditionals or function calls in the cell predicates have
different results, which means that the truth-values they define
are unchanged. 

Then we can hide $H'$ behind an existential to establish the postcondition. 
\end{proof}

\begin{figure}
\begin{mathpar}
  \inferrule*[right=AUnit]
            { }
            {\astep{I}{\return v}{I}{v}{\emptyset}{\emptyset}}
\and
  \inferrule*{\astep{\phi}{e}{\phi'}{v'}{r_1}{u_1} \\
              \astep{\phi'}{f\;v'}{\phi''}{v''}{r_2}{u_2}}
             {\astep{\phi}{\bind e\; f}{\phi''}{v''}{r_1 \cup r_2}{u_1 \cup u_2}}
\\
\\
  \inferrule*[right=AReady]
            {\ready{\phi}{a}{v}}
            {\astep{\phi}{\readcell a}{\phi}{v}{\setof{a}}{\emptyset}}
\and
  \inferrule*[right=AUnready]
            {\unready{\celleither{a}{e} \otimes \phi}{a} \;\;
             \astep{\phi}{e}{\phi'}{v}{r}{u}}
            {\aconfig{\celleither{a}{e} \otimes \phi}
                     {\readcell a} \Downarrow \\
             \aconfig{\cellpos{a}{e}{v}{r} \otimes R(\setof{a},\phi')}
                     {v}
             \aeffect{\setof{a}}
                     {u \cup \setof{a}}}
\\
\\
  \inferrule*[right=AGetRef]
            { } 
            {\astep{\localref{r}{v}}{\getref r}
                   {\localref{r}{v}}{v}{\emptyset}{\emptyset}}
\and
  \inferrule*[right=ASetRef]
            { } 
            {\astep{\localref{r}{v}}{\setref r\;v'}
                   {\localref{r}{v'}}{\unit}{\emptyset}{\emptyset}}
\\
\\
  \inferrule*[right=AbstractFrame]
            {\astep{\phi}{e}{\phi'}{v}{r}{u}}
            {\astep{\phi \otimes \psi}{e}{\phi' \otimes R(u, \psi)}{v}{r}{u}}
\end{mathpar}
\caption{Abstract Semantics of Notifications}
\label{abs-semantics}
\end{figure}

\begin{figure}
\begin{mathpar}
  \inferrule*[right=Ready]
            {\forall a' \in r.\; \exists v.\; \ready{\phi}{a'}{r}}
            {\ready{\phi \otimes \cellpos{a}{e}{v}{r}}{a}{v}}
  \\
  \\
  \inferrule*[right=UnreadyPos]
            {\exists a' \in r.\; \unready{\phi}{a}}
            {\unready{\phi \otimes \cellpos{a}{e}{v}{r}}
                     {a}}
  \and
  \inferrule*[right=UnreadyNeg]
            { }
            {\unready{\phi \otimes \cellneg{a}{e}}{a}}
\end{mathpar}
\caption{Ready and Unready Judgments}
\label{readiness}
\end{figure}

\begin{figure}
  \begin{displaymath}
    \begin{array}{lcl}
      closed(I, s) & = & \top \\
      closed(\phi \otimes \psi, s) & = & closed(\phi, s) \land closed(\psi, s) \\ 
      closed(\localref{r}{v}, s) & = & \top \\
      closed(\cellneg{a}{e}, s) & = & \top \\
      closed(\cellpos{a}{e}{v}{r}, s) & = & r \subseteq s \\
    \end{array}
  \end{displaymath}
\caption{Closedness predicate}
\label{closedness}  
\end{figure}

\begin{figure}
  \begin{displaymath}
    \begin{array}{lcl}
      R(s, I)                 & = & I \\
      R(s, \phi \otimes \psi) & = & R(s, \phi) \otimes R(s, \psi) \\
      R(s, \localref{r}{v})   & = & \localref{r}{v} \\
      R(s, \cellneg{a}{e})    & = & \cellneg{a}{e} \\
      R(s, \cellpos{a}{e}{v}{r}) & = & \left\{\begin{array}{ll}
                                                \cellpos{a}{e}{v}{r} 
                                              & \mbox{if } s \cap r = \emptyset \\
                                                \cellneg{a}{e}
                                              & \mbox{otherwise}
                                              \end{array}
                                       \right.
    \end{array}
  \end{displaymath}
\caption{Definition of the Ramification Operator $R$}
\label{ramify-def}
\end{figure}

\subsection{Ramified Frame Properties}

Ironically, this strategy is sufficient for $\newcell$and
$\updatecell$, but is not adequate for defining a frame property
for $\codetype{A}$ expressions. 

As an example, suppose that we want to evaluate the formula $\readcell
a$, in a cell heap described by $\cellneg{a}{\return 5}$.  Clearly,
this is a sufficient footprint, and we expect to a) get the return
value 5, and b) see the cell formula change to $\cellpos{a}{\return
  5}{5}{\emptyset}$.  However, the fact that we are now changing cells
from negative to positive means that the conditional character of
readiness, which worked in our favor with $\updatecell$ and
$\newcell\!\!$, now works against us.

In particular, suppose that we run this command with a framed abstract heap
formula $\psi = \cellpos{b}{\readcell a}{17}{\setof{a}}$. Now, the
whole starting heap will be:
\begin{displaymath}
\cellneg{a}{\return 5} \otimes \cellpos{b}{\readcell a}{17}{\setof{a}}  
\end{displaymath}
In any heap satisfying this formula, $b$ will be unready, because it depends 
on an unready cell. But when we execute $\readcell a$, simply copying $\psi$ 
into the post-state will give us the cell formula:
\begin{displaymath}
\cellpos{a}{\return 5}{5}{\emptyset} \otimes \cellpos{b}{\readcell a}{17}{\setof{a}}
\end{displaymath}
That is, our satisfaction relation now expects $b$ to be ready and have the 
value 17, even though $\readcell a$ never touches $b$ at all!

Clearly, we cannot expect to be able simply copy the same frame formula into 
the pre- and the post-condition states in the specification of commands like 
$\readcell a$. 

To deal with this problem, we look back to McCarthy's original paper
introducing the frame problem~\cite{mccarthy}. There, he described the
frame problem as the problem of how to specify what parts of a state
were \emph{unaffected} by the action of a command, which inspired the
name of the frame rule in separation logic. In that paper, he also
described the \emph{qualification problem}. He observed that many
commands (such as a flipping a light switch turning on a light bulb)
have numerous implicit preconditions (such as there being a bulb in
the light socket), and dubbed the problem of identifying these
implicit preconditions the qualification problem. 

Some years later, \citet{finger} observed that the qualification
problem has a dual: actions can have indirect effects that are not
explicitly stated in their specification (e.g., turning on the light can
startle the cat). He called the problem of deducing these implicit
consequences the ``ramification problem'' --- is there a simple way to
represent all of the indirect consequences of an action?

We can understand our difficulty as an instance of the ramification
problem. When we evaluate a code expression, we may read some unready
cells and send them from an unready state in the precondition to a ready
state in the postcondition. However, we may have had some cell formulas in
our frame which claimed their corresponding cells were unready purely
because one of the cells in our footprint was unready. Therefore, when
we update the footprint, we must modify the frame formula to account
for the ramifications of our update in the footprint. So even though
the actual physical storage representing the frame doesn't change at all, 
we need to modify our abstract formula to reflect our updated state
of knowledge. 

In our case, \emph{all} of the effects on the frame will arise from
the cells we flip from unready to ready. Thus, given the set of cells
which became ready, we can repair the framing formula by taking each
positive cell formula, and setting it to a negative state if its read
set includes anything that went from unready to ready. We define the
ramification operator $R(s, \psi)$ in Figure~\ref{ramify-def}.  It is
a simple structural induction over a framing formula, whose only
action is to replace the positive cell formulas in $\psi$ whose read
sets intersect with $s$ with a corresponding negative cell formula.
The ramification operator has a number of useful properties, which are
most easily expressed after we have introduced a few auxiliary
judgments and predicates.

In Figure~\ref{readiness} we define the two judgments $\unready{\phi}{o}$
and $\ready{\phi}{o}{v}$, which establish whether a cell is ready or
unready, from the syntactic structure of $\phi$. 

\begin{prop}{(Soundness of $\ready{\phi}{o}{v}$ and $\unready{\phi}{o}$)}
For all $\phi, o,$ and $H$ such that $H = (D,h)$:

\begin{itemize}
\item $(Inv(H, \phi) \land \ready{\phi}{o}{v}) \implies value(h(o)) = \ctext{Some}\;v$
\item $(Inv(H, \phi) \land \unready{\phi}{o}) \implies o \not\in V_H$
\end{itemize}
\end{prop}

In Figure~\ref{closedness}, we define the $closed(\phi, s)$ predicate,
which asserts that every cell formula in $\phi$ reads at most the
cells in $s$. Now, we can summarize the interactions between the 
ramification operator $R$ and abstract heap formulas as follows: 

\begin{prop}{(Interaction Properties)}
Given sets of cells $s$ and $u$, cell $o$, value $v$, and formula $\phi$, we have
that:
\begin{itemize}
\item $R(s, R(u, \phi)) = R(s \cup u, \phi)$
\item If $\unready{\phi}{o}$, then $\unready{R(u, \phi)}{o}$ 
\item If $\ready{R(u, \phi)}{o}{v}$, then $\ready{phi}{o}{v}$ 
\item If $R(u, \phi)$ and $closed(\phi, s)$, then $R(u, \phi) = R(u \cap s, \phi)$ 
\end{itemize}
\end{prop}

All of these facts can be proved with simple inductive arguments. 

The first property means that if we evaluate two expressions, we can
simply combine their ramification effects without having to worry
about the order that they were evaluated in. The second and third let
us know that a ramification cannot make us forget a cell is unready,
nor can it make anything ready that was not ready before. The last
property permits us to constrain the effect of a ramification --- if we
know that two parts of the abstract heap formula do not read each
other at all, we can deduce that ramifications from one will not
affect the other.

We can finally define the abstract semantics of the code expression
monad. We introduce the ``judgment''
$\astep{\phi}{e}{\phi'}{v}{r}{u}$, which is read as ``from an initial
state $\phi$, evaluating the $\codetype{A}$ expression $e$ will result
in a modified state $\phi'$ and a return value $v$ of type $A$. The
expression $e$ will have directly read the cells in $r$, and will have
updated the cells in $u$ from an unready to a ready state.'' 

\begin{figure}
\begin{tabbing}
$\astep{\phi}{e}{\phi'}{v}{r}{u} \triangleq$ \\[0.5em]
\;\;\= $\forall \psi.\;$\=$\setof{G(\phi \otimes \boxed{\psi})}$ \\
    \>                  \>${\runcmd e}$ \\
    \>                  \>$\setof{a.\;
                             G(\phi' \otimes \boxed{R(u, \psi)})
                             \land \exists z.\; a = (v, z) \land set(u, z, u)}$ \\
    \>                  \>$\!\!\specand$ \\
    \>                  \>$\{\forall o.\;$\=$(\unready{\phi \otimes \psi}{o} \land o \in \domain{\psi})$ \\
    \>                  \> \> $\implies \unready{\phi' \otimes R(u, \psi)}{o}\}$ \\
\>$\!\!\specand$ \\ 
\>$\{\forall c \in r \cup u.\; \exists v.\; \ready{\phi'}{c}{v} \;\land \forall c \in u.\; \unready{\phi}{c}\}$ 
\end{tabbing}
\caption{Definition of the Abstract Semantics}
\label{abstract-semantics}
\end{figure}

We define the meaning of the abstract semantics in
Figure~\ref{abstract-semantics}, where our ``judgment'' is revealed to 
be a notational abbreviation for a formula in our logic of specifications.

As before, we describe the effect of a command with a Hoare triple,
prefixed with a quantification over all possible frames $\psi$. Then
we assert that from a state $G(\phi \otimes \psi)$, running $e$ will
give us a state $G(\phi' \otimes R(u, \psi))$ --- that is, we must
update the frame with the ramification $u$. The frames have been put
into boxes, to highlight the fact that the framing formulas are
\emph{different} in the pre- and the post-condition.

Then, there are some extra conditions which constrain how the
expression might have modified the state. We assert that anything
unready in the frame will stay unready, and that the cells in the
update set were unready before running the command and ready afterward,
and that everything the read set is also ready. 

The reason we have to explicitly maintain these conditions is in order
to prove the rules given in Figure~\ref{abs-semantics}. Here, we
present a number of implications over specifications in inference rule
format, mimicking the structure of a big-step operational semantics. 
In rule \textsc{AUnit}, we give a specification for the $\return v$ command, 
which simply returns its argument and neither reads nor updates any 
cells or state. The \textsc{ABind} rule explains how sequential composition 
works --- as expected, we evaluate the first monadic argument, and pass
the result to the functional argument, and evaluate that. The read and
update sets are simply the union of the two executions. Reading a cell
comes in two variants, \textsc{AReady} and \textsc{AUnready}. If a cell is ready, we
simply return its memoized value without any further computation. If
a cell is unready, we need to evaluate its code body, and then update
the cell with its new value. Note that we have to apply the ramification
operator in \textsc{AUnready}, because the cell we're reading goes from unready
to ready itself. We can also read and write local state, which do not have any
effect on the cells. 

Finally, we have the \textsc{AbstractFrame} rule, which allows us to
extend the abstract heap formulas in the style of the frame rule of 
separation logic --- the signal difference being that we have to apply 
the ramification operator $R$ to the frame in the post-state. 

\begin{prop}{(Soundness of Abstract Semantics)}

All of the rules of the abstract semantics in Figure~\ref{abs-semantics} are
provable within our specification logic. 
\end{prop}

At this point, we can now reason about the behavior of the imperative
notification library in terms of its action on the abstract heap. The
combination of quantification and ramification give us a domain-specific 
frame property, which allow us to modularly prove the correctness of programs 
that construct and produce notifications.

\section{Verifying an Imperative Implementation of Functional Reactive Programming}

In this section, we will see how to verify an imperative
implementation of a simple synchronous functional reactive programming
system.

\subsection{Specifying Functional Reactive Programs}

\emph{Functional Reactive Programming}~\cite{frp} is a style of
writing interactive programs based on the idea of \emph{stream
  transducers}.  The idea is to model a time-varying input signal of
type $A$ as an infinite stream of $A$'s, and to model an interactive
system as a function that takes a stream of inputs $\stream{A}$ and
yields a stream of outputs $\stream{B}$. Note that a stream can be
viewed either as an infinite sequence of values, or isomorphically as
a function from natural numbers to values (i.e., a function from times
to values). In our discussion, we'll switch freely between these two
views, using the most convenient viewpoint.\footnote{Given an infinite stream $vs$, we will use use $take\;n\;vs$ to denote
the finite list consisting of the first $n$ elements of the stream
$vs$. Correspondingly, $drop\;n\;vs$ is the infinite stream with $vs$
with its first $n$ elements cut off. With a function $f$, $map\;f\;vs$
maps $f$ over the elements of $vs$, and given another infinite stream
$us$, the call $zip\;us\;vs$ returns the infinite stream of pairs of
elements of $us$ and $vs$. If $v$ is an element, $v \cdot vs$ will 
denote consing $v$ to the front of $vs$, and if $xs$ is a finite list, then
$xs \cdot vs$ will denote appending the finite sequence $xs$ to the
front of $vs$. Finally, we will write $vs_n$ to denote the $n$-th element
of the stream $vs$.}

However, not all functions $\stream{A} \to \stream{B}$ are legitimate
stream transducers; we need to restrict our attention to \emph{causal}
stream transducers. A transducer is causal if we can compute the first
$n$ elements of the output after having read at most $n$ elements of
the input. 

\begin{tabbing}
$causal(f : \stream{A} \to \stream{B}) \equiv$ \\
\;\;\= $\exists \hat{f} : \listtype{A} \to \listtype{B}.\;\forall as:\stream{A}, n:\N.$ \\
    \> \;\;$take\;n\;(f\;as) = \hat{f}\;(take\;n\;as)$ 
\end{tabbing}

If we are given a causal transducer $p$, we will write $\hat{p}$ to
indicate the corresponding list function which computes its finite
approximations. Then, we can define a family of combinators acting on
causal transducers, which we give in Figure~\ref{transducer-semantics}.

\begin{figure}
\begin{tabbing}
$\ST{A}{B} = \comprehend{f \in \stream{A} \to \stream{B}}{causal(f)}$\\[1em]

$lift : (A \to B) \to \ST{A}{B}$ \\
$lift\;f\;as = map\;f\;as$ \\[1em]

$seq  : \ST{A}{B} \to \ST{B}{C} \to \ST{A}{C}$ \\
$seq\;p\;q = q \circ p$ \\[1em]

$par  : \ST{A}{B} \to \ST{C}{D} \to \ST{A \times C}{B \times D}$ \\
$par\; p\;q\;abs = zip\; (p\;(map\;\pi_1\;abs))\;(q\;(map\;\pi_2\;abs))$\\[1em]

$switch : \N \to \ST{A}{B} \to \ST{A}{B} \to \ST{A}{B}$ \\
$switch\;k\;p\;q = \semfun{as}{(take\;k\;(p\;as))\cdot(q\;(drop\;k\;as))}$ \\[1em]

$loop : A \to \ST{A\times B}{A \times C} \to \ST{B}{C}$ \\
$loop\;a_0\;p = (map\;\pi_2) \circ (cycle\;a_0\;p)$ \\[1em]

$cycle : A \to \ST{A\times B}{A \times C} \to \ST{B}{A \times C}$ \\
$cycle\;a_0\;p = \lambda bs.\;\lambda n.\;last(gen\;a_0\;p\;v\;n)$ \\[1em]

$gen : A \to \ST{A\times B}{A \times C} \to \listtype{(A \times C)}$\\
$gen\;a_0\;p\;bs\;0 \;\;\; = \hat{p}\; [(a_0, bs_0)]$ \\
$gen\;a_0\;p\;bs\;(n+1) = $ \\
\;\;$\hat{p}\;(zip (a_0 \cdot (map\;\pi_1\;(gen\;a_0\;p\;bs\;n)))\;
                                        (take\;(n+2)\;bs))$ 
\end{tabbing}
\caption{Semantics of Stream Transducers}
\label{transducer-semantics}
\end{figure}

The operation $lift\;f$ creates a stream transducer that simply maps
the function $f$ over its input. Calls to $seq\;p\;q$ are sequential
composition: it feeds the output of $p$ into the input of $q$. The
operator $par\;p\;q$ defines parallel composition --- it takes a
stream of pairs, and feeds each component to its arguments,
respectively, and then merges the two output streams to produce the
combined output stream. The function $switch\;k\;p\;q$ is a very
simple ``switching combinator''.  It behaves as if it were $p$ for the
first $k$ time steps, and then behaves as if it were $q$, only
starting with the input stream beginning at time $k$.

The combinator $loop\;a_0\;p$ is a feedback operation. It acts
upon a transducer $p$ which takes pairs of $A$s and $B$s, and yields
pairs of $A$s and $C$s. It turns it into a combinator that takes $B$s
to $C$s, by giving $p$ the value $a_0$ (and its $B$-input) on the
first time step, and uses the output $A$ at time $n$ as the input $A$
at time $n+1$. This is useful for constructing transducers that do
things like sum their inputs over time, and other stateful operations. 

Because this function involves feedback, it should not be surprising
that it makes use of the causal nature of its argument operation. The
$loop$ function is defined in terms of $cycle$, which also returns the
sequence of output $A$s, and $cycle$ is defined in terms of $gen$,
which is a function that given an argument $n$ returns a list of
outputs for the time steps from $0$ to $n$. Notice that
$gen\;a_0\;p\;bs\;n$ will always return $n+1$ elements (e.g., at
argument 0, it will return a 1 element list containing the output at
time step 0), which means that the call to $last$ in $cycle$ is
actually safe. In order to calculate $gen$, we need to recursively
calculate the outputs for all smaller time steps, and this is what
$\hat{p}$ is needed for --- it is what lets us know that $p$ has a good
finite approximation.

All of these definitions are familiar to functional programmers, and
there are many techniques to prove properties of these functions ---
coinductive proofs, the $take$-lemma of \citet{bird-wadler}, arguments based on
the isomorphism between streams and functions from natural
numbers. All of these serve to make proving properties about stream
transducers very pleasant. For example, one property we will need in
the next section is the following:

\begin{lemma}{(Loop Unrolling)} We have that 
  \begin{displaymath}
    cycle\;a_0\;p\;bs = f\;(zip\;(a_0\cdot(map\;\pi_1\;(cycle\;a_0\;p\;bs)))\;bs)
  \end{displaymath}
\end{lemma}

\begin{proof}
  This is easily proved using Bird and Wadler's $take$-lemma, which
  says that two streams are equal if all their finite prefixes are
  equal.
\end{proof}


\subsection{Realizing Stream Transducers with Notifications}

\begin{figure}
\begin{tabbing}
$\ST{A}{B} \equiv \celltype{A} \to \monad{\celltype{B}}$ \\[1em]

$\liftop : (A \to B) \to \ST{A}{B}$ \\
$\liftop\;f\;input = $ \\
\;\; $\newcell\; (\bind\;(\readcell input)\; (\fun{x}{A}{\return (f\;x)}))$ \\[1em]

$\composeop : \ST{A}{B} \to \ST{B}{C} \to \ST{A}{C}$ \\
$\composeop p\;q\;input = [$\=$\letv{middle}{p\;input}$ \\
                            \>$\letv{output}{q\;middle}$ \\ 
                            \>$\;output]$ \\[1em]

$\parop : \ST{A}{B} \to \ST{C}{D} \to \ST{A \times C}{B \times D}$ \\
$\parop p \; q \; input = $ \\
\;\;$[$\=$\ctext{letv}\;a = \newcell\; (\bind$\=$(\readcell\;input)$ \\
     \>                                   \>$(\fun{x}{A\times B}{\return (\fst{x})}))$ \\
     \>$\ctext{letv}\;b = {p\;a}{}$ \\
     \>$\ctext{letv}\;c = \newcell\; (\bind$\=$(\readcell\;input)\;$\\ 
     \>                                   \>$(\fun{x}{A\times B}{\return (\snd{x})}))$ \\
     \>$\ctext{letv}\;d = {q\;b} = $ \\
     \>$\ctext{letv}\;output = \newcell\; ($\=$\bind (\readcell b)\; (\lambda b:B.$ \\
     \>                                   \>$\bind (\readcell d)\; (\lambda d:D.$ \\
     \>                                   \>$\;\;\return \pair{b}{d})))] \;\ctext{in}$ \\
     \>$\;output]$ \\[1em]

$\switchop : \N \to \ST{A}{B} \to \ST{A}{B} \to \ST{A}{B}$ \\
$\switchop k\;p\;q\; input =  $ \\
\;\;$[$\=$\letv{r}{\newref{\N}{0}}{}$ \\
    \>$\letv{a}{p\;input}$ \\
    \>$\letv{b}{p\;input}$ \\
    \>$\ctext{letv}\; out = \newcell\; ($\=$\bind (\getref r) \;(\lambda i:\N.\;$ \\
    \>                                 \>$\bind (\setref r\;(i+1)) \; (\lambda q:\unittype.$ \\
    \>                                 \>$\;\;\ctext{if}(i < k, \readcell a, \readcell b)))) \;\ctext{in}$ \\
    \>$\;\;out]$\\[1em]

$\loopop : A \to \ST{A\times B}{A\times C} \to \ST{B}{C}$ \\
$\loopop a_0\; p \; input = $ \\
\;\;$[$\=$\letv{r}{\newref{A}{a_0}}{}$ \\
    \>$\ctext{letv}\; ab = \newcell\; ($\=$\bind (\readcell input)\; (\lambda b:B.$ \\
    \>                                \>$\bind (\getref r)\;       (\lambda a:A.$ \\
    \>                                \>$\;\; \return \pair{a}{b}))) \;\ctext{in}$ \\
    \>$\letv{ac}{p\;ab}{}$ \\
    \>$\ctext{letv}\;c = \newcell\; ($\=$\bind (\readcell ac) \;(\lambda v:A \times C.$ \\
    \>                              \>$\bind (\setref r\;(\fst{v})) \;(\lambda q:\unittype.$ \\
    \>                              \>$\;\;\return (\snd{v}))))\;\ctext{in}$ \\
    \>$\;\;c]$ 
\end{tabbing}
\caption{Imperative Stream Transducers}
\label{imperative-transducer-semantics}
\end{figure}

While the definitions in the previous subsection yield very clean
proofs, they are not suitable as implementations --- for example,
$loop$ recomputes an entire history at each time step! We can derive
better implementations by thinking about how imperative, event-driven
programming works.

The intuition underlying event-driven programming is that a stream
transducer is implemented with the combination of a notification
network, and an \emph{event loop}.  The event loop is a
(possibly-infinite) loop which updates an input cell at teach time
step, to reflect the events that occurred on that time step, and then
it reads the output cell of the network. When the input cell is
updated, invalidations are propagated throughout the dependency
network, and when the outputs are read, exactly the necessary
re-computations are performed.

We will shortly formalize exactly this idea, but we will first discuss
the implementation given in
Figure~\ref{imperative-transducer-semantics} in informal terms. Here,
we define the type of imperative stream transducers as a function type
$\celltype{A} \to \monad{(\celltype{B})}$. This type should be read as
saying that the implementation is a function that, given an input cell
of type $A$, will \emph{construct} a dataflow notification network
realizing the corresponding transducer, and whose return value is the
output cell of type $B$ that the event loop should read. 

The simplest example of this is $\liftop\;f$. It will take an input
cell $input$, and build a new cell which reads $input$, and return $f$
applied to that value. Likewise, given two imperative implementations $p$
and $q$, $\composeop\;p\;q$ will take an input cell, and feed the
input to $p$ to build a network whose output is named $middle$, and
will then give $middle$ to $q$ to get the final output cell. The
overall network will be network built by the calls to both $p$ and
$q$, which interact through $p$'s network putting a value in $middle$,
and $q$'s network reading it.

The operation $\switchop\;k\;p\;q$ is the first example that uses 
local state. Given an $input$ cell, we first build networks corresponding
to $p$ and to $q$ (with outputs $a$ and $b$, respectively). Then we
create a local reference $r$, initialized to $0$. Then we build a cell $out$,
whose code reads and increments $r$, and which will read $a$ or $b$ depending
on whether the reference's contents are less than or equal to $k$. Notice
that the demand-driven nature of evaluation means that we never redundantly
evaluate $p$ or $q$'s networks --- we only ever execute one of them. 

Finally, the operation $\loopop a_0\;p$ builds a feedback network by
explicitly creating a reference to hold an accumulator parameter. It
constructs a local reference initialized to $a_0$, and then constructs
a cell $ab$ which reads the input and the local reference to produce a
pair of type $A \times B$. This cell is given to $p$, to construct a
network with an output cell $ac$, yielding pairs of type $A \times
C$. Finally, we construct the overall output cell $c$, which reads
$ac$ and updates the local reference with a new value of type $A$, and
returns a value of type $C$. The use of a local reference (rather than
a cell) to store the current state of $A$ is essential, because we need
to maintain the acyclicity of the dataflow graph. 

Now, with these ideas in mind, we come to the definition of what it
means for a dataflow network to realize a stream transducer. This
property is (unavoidably) quite large, but despite its size is quite
pleasant to work with.
\begin{tabbing}
$Realize(i, \Phi, o, f) \triangleq$ \\
\;\;\= $\forall v:$\=$\,\stream{A}.\;$ \fbox{$\exists \phi:\stream{\formula}, u:\stream{\powersetfin{ecell}}.$} \\
\>\> $\{\forall n:\N.\;$\=$\closed{\phi_n}{\domain{\phi_n} \cup \setof{i}} \;\land$ \\
\>\>\>$\domain{\phi_n} = \domain{\phi_{n+1}}\; \land$ \\
\>\>\>$\forall \psi.\; \unready{\psi}{i} \implies \unready{\psi \otimes \phi_n}{o}\}$ \\
\>\> $\specand \; \{\Phi = \phi_0\}$ \\
\>\> $\specand \;  Transduce(i, \phi, o, u, f, v)$ \\[1em]


$Transduce(i, \phi, o, u, f, v) \triangleq$ \\
\> $\forall $\=$ n:\N, \phi_i, \phi'_i, u_i.$ \\
\>\> \fbox{$\astep{\phi_i}{\readcell i}{\phi'_i}{v_n}{\setof{i}}{u_i}$} $\; \land$ \\
\>\> $\{$\=$\domain{\phi_i} = \domain{\phi'_i} \;\land $ \\
\>\>     \>$\closed{\phi_i}{\domain{\phi_i}} \land 
            \closed{\phi_i}{\domain{\phi'_i}} \land $ \\
\>\>     \>$\unready{\phi_i}{i} \land
            i \in u_i\}$ \\
\>\> $\Longrightarrow$ \\
\>\> \fbox{$\astep{\phi_i \otimes \phi_n}{\readcell o}
              {\phi'_i \otimes \phi_{n+1}}{(f\;v)_n}
              {\setof{o}}{u_i \cup u_n}$} \\
\>\> $\specand\; \{u \subseteq \domain{\phi_n} \land o \in u\}$ \\
\end{tabbing}

We read $Realize(i, \Phi, o, f)$ as saying ``the dataflow network
$\Phi$ realizes the stream transducer $f$, when the event loop writes
inputs into $i$ and reads outputs from $o$''. We have highlighted the
key pieces of this definition with boxes. First, given some input
stream $v$, we existentially assert the existence of a \emph{stream}
of abstract heap formulas $\phi$. This stream represents the evolving
state of the network over time --- because our notification networks
contain local state, that state can potentially have a different value
at each time step. Then, in the unboxed formulas, we assert some
well-formedness properties, such as 1) the initial value of the stream
matching our $\Phi$, 2) the domain of the network (i.e., the cells whose
atomic formulas are in that formula) remaining constant
over time, 3) that the output is unready if the input is, and 4) that
the network reads only the specified input cell. (All of these
conditions could be relaxed, but for the basic FRP combinators we
consider in this paper there is no need. It \emph{would} become
necessary if we added combinators that dynamically created new
transducers as the program ran, since at that point we would need to
create new cells at each time step.)

We state the stream property with the $Transducer$ sub-predicate. It
asserts for all time $n$, that if we have input state $\phi_i$ which
yields $v_i$ when $i$ is read (the second boxed formula), then 
reading $o$ in a larger state containing both the input $phi_i$ and the
$n$-th transducer state $\phi_n$ will 1) return the $n$-th output
value $(f\;v)_n$, and 2) will update the state of the network to 
the next time step $\phi_{n+1}$ (the third boxed formula). 

With this property in hand, we can prove the following specifications: 

\begin{prop}{(FRP Correctness)}
We define the $Relate$ predicate:
\begin{tabbing}
$Relate_{A,B}(p, f) \triangleq$ \\
\;\; $\forall \psi:\formula, i:\celltype{A}.$ \\ 
\qquad $\setof{G(\psi)}\;p\;i\;\{o:\celltype{B}.\;
               \exists \Phi.\;$\=$G(\Phi \otimes \psi) \;\land$ \\
                               \>$ Realize(i, \Phi, o, f) \valid\}$
\end{tabbing}
\noindent Then, the following specifications are provable: 
\begin{tabbing}
$\forall f:A\to B.\; Relate(\liftop f, lift\;f)$ \\[1em]

$\forall p, f, q, g.\;$\=$Relate(p, f) \specand Relate(q, g)$ \\
                       \>$\specimp Relate(\composeop p\;q, compose\;f\;g)$ \\[1em]

$\forall p, f, q, g.\;$\=$Relate(p, f) \specand Relate(q, g)$ \\
                       \>$\specimp Relate(\parop p\;q, par\;f\;g)$ \\[1em]

$\forall k, p, f, q, g.\;$\=$Relate(p, f) \specand Relate(q, g)$ \\
                          \>$\specimp Relate(\switchop k\;p\;q, switch\;k\;f\;g)$ \\[1em]

$\forall a_0, p, f.\; 
  Relate(p, f) \specimp Relate(\loopop\;a_0\;p, loop\;a_0\;f)$
\end{tabbing}
\end{prop}

These lemmas permit us to reason about our transducer implementation
as if it were a pure implementation --- for each combinator in the
interface, we have a proof that shows the corresponding implementation
combinator lifts related arguments to related results. As a result, 
we can show that, for example, $\composeop (\liftop f) \;(\liftop g)$
and $\liftop (g \circ f)$ both realize the same function $lift\;(g\circ f)$. 

\section{Future Work}

Now that we have introduced the idea of ramifications, we see many
possible applications for them. 

First, there are numerous algorithms --- such as unification, the
union-find algorithm, and the chaotic iteration constraint propagation
algorithm used in dataflow analysis --- which rely on using mutation
and assignment as a way of globally broadcasting information to the
rest of the program state. These algorithms have all been naturally
resistant to modular proofs, because of the apparent need to know
``the rest of the world'' in the program invariant. It would be
interesting to see if ramifications can help.

Second, we would like to investigate the relationship between
ramification operators and methods based on
rely-guarantee~\cite{rely-guarantee-jones}. Rely-guarantee works by
imposing a mutual contract between a piece of code and the rest of the
world, which is at least conceptually similar to the idea of a
ramification, though we see no obvious direct relationship.

Third, in this work we have presented ramifications as a style of
specification useful for verifying a particular library. Might it be
sensible or useful to bake ramification operators into the basic
logical framework? If so, what are their logical properties --- $R(u,
\phi)$ looks like a family of modal operators on the formula $\phi$, but
even in our first example we needed a number of auxiliary interaction
lemmas to make them truly useful.

\section{Related Work}

Versions of separation logic~\cite{sep-logic} supporting higher-order
languages and quantification over predicates have been proposed by
\citet{htt} with Hoare Type Theory, and by
\citet{parkinson-bierman}. It would be interesting to study how the
proof techniques in this paper could be adapted to their setting.

Prior work on verifying callbacks using separation logic includes work
by \citet{tldi09,ftfjp07} work by
\citet{parkinson-iwaco-07}. The approach in both of these
papers was similar; each subject predicate was equipped with a list of
observers, rather than maintaining them implicitly in the
invariant. These approaches were not modular, because the observers
had to be explicitly named. Worse still, this approach broke down when
faced with chains of subjects and observers, because separation logic
is a resource-based logic, and the existence of multiple paths to an
object meant that ownership of observer state became ambiguous.

citet{shaner-leavens-naumann} studied using gray-box model
programs to model higher-order method calls (which can be understood
as a a variant of refinement calculus-based methods) in JML, and 
\citet{history-invariants} have applied Liskov and Wing's
idea of history invariants~\cite{liskov-wing} to model observers. As
before, both of these methods are also non-modular, since they require
knowing what all of the observers are, and furthermore both of these
methods sharply restrict the kinds of invariants that can be used,
making it very difficult to model the code-update-based protocol seen
in our FRP example.

\citet{self-adjusting} have proposed
\emph{self-adjusting computation} as a technique for using change
propagation to write programs that incrementally
recompute answers as the inputs are adjusted. It would be worth 
studying if our techniques could help verifying implementations of this
idea.

Functional reactive programming (FRP) was proposed by \citet{frp} as a
declarative formalism for interactive programming. The API in our
paper differs from theirs in two ways. First, our interface is a
variation of the \emph{arrowized FRP} interface proposed in
\cite{afrp}, and secondly, we use a discrete model of time, rather
than a continuous model of time --- though we found the core idea of
using a declarative semantics as a specification for the interface an
inspiring one.

Furthermore, our work could also serve as a bridge between the work on
purely functional reactive programming, and more imperative
implementations, such as the work done by \citet{superglue} on
SuperGlue and by \citet{frtime} on the FrTime system.


\section{Appendix: The $\cellset$ Interface}

In this section, we'll describe the interface to the $\cellset$ type,
which is intended to be used to represent pure collections of
existentially quantified cells. Specifying this interface is not
entirely trivial, because of the way equality works for this
type. Ordinarily, we would simply give a two-place predicate $set(v,
elts)$ relating a value $v$, and the mathematical set of elements
$elts$ it represents.

However, this approach is not sufficient in our case. In order to
manage dependencies, we need to be able to test cells of
\emph{different} concrete type for equality, and the natural equality
for references only permits testing references of the same type. As a
result, we can't unpack an existentially-quantified cell and compare
the elements in its tuple, because we don't know that the two cells
are of the same type (and indeed, they might not be). 

To get around this problem, we generate a unique integer identifier
for each cell we create, and then compare those identifiers to
establish equality. Since these identifiers are all generated
dynamically along with the cells, this means that the precise partial
equivalence relation we need to use is determined dynamically as
well. So we add an additional index to the set predicate $set(W, v,
elts)$. The extra parameter $W$ is the \emph{world}, the set of all
the cells allocated so far, which must collectively satisfy the
constraint that they are equal if and only if their identifier fields
match. 

All of the usual set operations exist. We have an $\ctext{emptyset}$,
representing an empty set of cells. We have $\ctext{addset}\;v\;x$,
which adds the element $x$ to the set $v$ represents, and
$\ctext{removeset}\;v\;x$, which removes $x$ from the set $v$
represents. We also have $\ctext{iterset}\;f\;v$, which iterates over
the elements of $v$'s set and applies $f$ to each element in some
sequential order. (Observe that the specification makes use of two 
auxiliary predicates: $matches$, which assert that a
set and a list have the same elements; and $iterseq$, which constructs
a command representing the sequential execution of those elements.) 

We have three axioms that our implementation must satisfy. First, if a
$\cellset$ value $v$ represents a set in a world $W$, it will also
represent a set in any larger world $W'$. Second, the values in a set
are always a subset of the world $W$. Finally, we require that
$set(W,v,elts)$ is a pure predicate (i.e., is not heap-dependent),
which implies that it have a purely functional implementation (for
example, as a binary tree). 


\begin{tabbing}
$World = $ \\
\;\;\;\;\=$\{D \in \powersetfin{ecell}\;|\;\forall c,d \in D.\;$\=$identity(c) = identity(d)$\\
        \>                                                      \>$\iff c = d\}$\\[1em]

$\exists \cellset : \star.$ \\
$\exists set : World \To \cellset \To \powersetfin{ecell} \To \assert.$ \\
$\exists \ctext{emptyset}    : cellset.$ \\
$\exists \ctext{addset}      : \cellset \to ecell \to \cellset.$ \\
$\exists \ctext{removeset}   : \cellset \to ecell \to \cellset.$ \\
$\exists \ctext{iterset}     : \cellset \to (ecell \to \monad{\unittype}) \to \unittype.$\\[1em]

$\forall W \in World.\; set(W, \ctext{emptyset}, \emptyset)$ \\[1em]


$\forall W \in World, v : \cellset, x : ecell, elts \in \powersetfin{ecell}.$ \\ 
\> $set(W, v, elts) \land x \in W \implies set(W, \ctext{addset}\;v\;x, elts \cup \setof{x})$ \\[1em]


$\forall W \in World, v : \cellset, x : ecell, elts \in \powersetfin{ecell}.$ \\ 
\> $set(W, v, elts) \land x \in W \implies set(W, \ctext{removeset}\;v\;x, elts - \setof{x})$ \\[1em]


$\forall W \in World, v : \cellset, elts \in \powersetfin{ecell}, 
         f : (ecell \to \monad{\unittype}).$ \\ 
\> $set(W, v, elts) \implies \exists L : \seqsort{ecell}.\;$\=$matches\;elts\;L\; \land$ \\
\>                                \>$\ctext{iterset}\;f\;v = iterseq\;f\;L$ \\[1em]


$\forall W, W' \in World, v, elts.$ \\
\>$set(W,v,elts) \land W \subseteq W' \implies set(W',v, elts)$\\[1em]

$\forall W, W' \in World, v, elts.$ \\
\>$set(W, v, elts) \implies elts \subseteq W$ \\[1em]

$\forall W, v, elts.\; pure(set(W,v,elts))$ \\[1em]
  

$matches\;elts\;[] \qquad\;\;\; = elts = \emptyset$ \\
$matches\;elts\;(v :: vs) = v \in elts \land matches\;(elts - \setof{v})\;vs$\\[1em]

$iterseq\; f\; [] \qquad\;\;\;\;$\=$= \comp{\unit}$ \\
$iterseq\; f\; (v :: vs)$\>$= \comp{\letv{\unit}{f\;v}{\ctext{run}\;iterseq\;f\;vs}}$ \\
\end{tabbing}


\paragraph{Acknowledgements}

We would like to thank Peter O'Hearn for pointing out the connection
of our work with the ramification problem of AI.

