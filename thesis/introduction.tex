\chapter{Introduction}

My thesis is that it is possible to give modular correctness proofs of
interesting higher-order imperative programs using higher-order
separation logic.

\section{Motivation}

It is more difficult to reason about programs that use aliasing than
ones that do not use mutable shared data. It is more difficult to
reason about programs that use higher order features than first order
programs. Put both together, and the problems become both challenging
and interesting for formal verification.

The challenge arises from the fact that combining aliased mutable
state with higher order programming yields a language that is more
difficult than the sum of its parts. Techniques to reason about purely
functional programs, which extensively use higher-order methods but
eschew mutable state, are an extremely well-developed branch of
programming language theory, with a long and successful history.

Historically, techniques to reason about mutable state have lagged
behind, but some years ago O'Hearn and Reynolds proposed separation
logic~\citep{sep-logic}, which has proven extremely successful at
enabling reasoning about even such intricate imperative programs as
garbage collectors~\citep{gc-proof}. However, separation logic has
historically focused on low-level programming languages that lack a
strong type discipline and allow the use of techniques such as
treating pointers as integers.

Most high level languages that allow the use of state are also prone
to aliasing errors, since (with a few exceptions) type systems do not
track interference properties. Even languages such as Haskell, which
isolate all side-effects within a particular family of monadic types,
still face the problem that reasoning about monadic code remains as
difficult as ever. Mutation and state have been imprisoned, but they
have not been rehabilitated.

If this combination only posed challenges, but had no applications,
then it would be only of theoretical interest. However, it is possible
to use it to simply and naturally express many common kinds of
programs. For example, graphical user interfaces are typically
structured as families of callbacks that operate over the shared data
structures representing the interface and program. These callbacks are
structured using the subject-observer pattern, which use collections of 
callbacks to implicitly synchronize mutable data structures across 
a program. 


\section{Proposed Thesis}

The thesis I want to demonstrate is: ``The modular formal verification
of interesting higher-order imperative programs is feasible using
higher-order separation logic.''

\section{Higher Order Separation Logic}

The main tool I will use to prove programs correct is a higher order
separation logic for a simple higher-order imperative programming
details. The details of the language and the specification logic are
given at the end of the proposal document, but I will say a few words
about them here to set the stage for what follows. 

The programming language I consider is a simply-typed monadic
functional language, in Pfenning-Davies~\cite{pfenning-davies}
style. It can be considered an instance of Moggi's ~\cite{moggi}
monadic metalanguage -- all heap effects are encapsulated within the
monadic type constructor.

The basic form of specification is the Hoare triple over an imperative
computation $\spec{P}{c}{a:\tau}{Q}$. Here, $P$ is the precondition,
$c$ is the computation that we are specifying, and $Q$ is the post-
condition. The binder $a:\tau$ names the return value of the computation,
so that we can mention it in the post-condition. 

The assertion logic for the pre- and post-conditions is higher order
separation logic~\cite{hosl}. This is a substructural logic that 
extends ordinary logic with three spatial connectives, which enable
reasoning about the aliasing behavior of data. 

The Hoare triples themselves form the atomic propositions of a
first-order intuitionistic logic of specifications. The quantifiers
range over the sorts of the assertion logic, so that we can
universally and existentially quantify over assertions and
expressions.


\section{Modular Verification}

In order for a verification methodology to scale up to even modestly-
sized programs, it must be modular. There are three informal senses in
which I use the word ``modular'', each of which is supported by
different features of our logic.

\begin{enumerate}
\item First, we should be able to verify programs library by
  library. That is, we should be able to give the specification of a
  library, and prove both that implementations satisfy the
  specification without knowing anything about the clients that
  use it, and likewise prove the correctness of a client program
  without knowing anything about the details of the implementation
  (beyond what is promised by the specification). We will handle this
  problem by making use of the fact that the presence of existential
  specifications in our specification logic lets  use the Mitchell-
  Plotkin encoding of modules as existentials. 

\item Related to this is a modular treatment of aliasing. Ordinary
  Hoare logic becomes non-modular when asked to treat mutable, shared
  state, because we must explicitly specify the aliasing or
  non-alsiasing of every variable and mutable data structure in our
  pre- and post-conditions. Besides the quadratic dependence on the
  size of the program, the relevant set of variables grows whenever 
  a subprogram is embedded in a larger. Separation logic resolves this
  problem by introducing via the frame rule, a feature which we carry
  forward in our logic. 

\item Finally, it is important to ensure that the abstractions we
  introduce compose. Benton\cite{benton-modularity}described a
  situation where he was able to prove that a memory allocator and the
  rest of the program interacted only through a particular interface,
  but when he tried to divide the rest of the program into further
  modules, he encountered difficulties, because it was unclear how to
  share the resources and responsibility for upholding the contract
  with the allocator. 


\end{enumerate}


The specification logic I give is a first order logic of Hoare
triples. This gives us a simple technique for separating
implementation and client -- namely, we can use existentially
quantified specifications to create the boundary between a module and
its client. We can verify a client program by proving it under the
\emph{hypothesis} that it has access to a program satisfying the
existentially quantified specification, and we can verify the module
implementation by proving that it \emph{witnesses} the truth of the
specification. Then, linking the two together is merely an application
of the cut rule of logic.

As described above, this is essentially an application of Mitchell and
Plotkin's identification of data abstraction and existential
types. It's worth noting at this point that we do not have existential
types in our programming language (yet?). Instead, we are abstracting
over the \emph{heap}. As a concrete example, consider the
specification of a counter module.

\begin{tabbing}
$\exists $\= $counter : \reftype{\N} \times \N \To prop$ \\
$\exists \mathtt{create} : \unittype \to \monad{\tau_c}$ \\
$\exists \mathtt{next} : \tau_c \to \monad{\N}$ \\
\> $\spec{\emp}{\mathtt{create}()}{a:\tau_c}{counter(a, 0)}$ \\
\> $\specand$ \\
\> $\forall c, n.\; \spec{counter(c, n)}{\mathtt{next}(c)}{a:\N}{a = n \land counter(c, n+1)}$
\\
\end{tabbing} 

Here are two possible implementations for the existential witnesses:

\begin{displaymath}
\begin{array}{lcl}
counter & \equiv & \lambda (r, n).\; (r \pointsto n) \\[1em]
\mathtt{create} & \equiv & \lambda ().\; [\newref{\N}{0}] \\[1em]
\mathtt{next}  & \equiv & \lambda r.\; [\letv{n}{[!r]}{
                                        \letv{()}{[r := n + 1]}{n}}] \\
\end{array}
\end{displaymath}

and also                                        
\begin{displaymath}
\begin{array}{lcl}
counter & \equiv & \lambda (r, n).\; (r \pointsto n + 7) \\[1em]
\mathtt{create} & \equiv & \lambda ().\; [\newref{\N}{7}] \\[1em]
\mathtt{next}  & \equiv & \lambda r.\; [\letv{n}{[!r]}{
                                        \letv{()}{[r := n + 1]}{n - 7}}] \\
\end{array}
\end{displaymath}

As an aside, note that a client program cannot distinguish these two
implementations, even though the reference type is a concrete (and not
abstract) type, because of the resource semantics of separation
logic. Merely knowing that the counter module's implementation type
does not let us access the pointer, because we must have a points-to
assertion $e \pointsto e'$ to dereference the pointer $e$. 

\subsection{Imperative Modularity}

While separating modules and clients in this way is obviously useful,
for imperative programs it is not entirely a sufficient mechanism on
its own.

In a purely functional program, all communication between an
implementation and a client of a module is mediated by the signature
of the module. The addition of state makes it possible to create an
extra channel of communication through the heap, and in our formalism
this is captured by adding pre- and post-conditions to the
specifications of imperative actions.

However, suppose we have a client program with specification $C$,
which makes use of a particular module implementation with a
specification $S$. Furthermore, suppose that \emph{both} the client
program implemeting $C$, and the module implementing the $S$
interface, rely upon yet another module with interface $F$. As an
example, imagine an $F$ that has a purely functional interface, but
which makes use of an imperative cache for efficiency reasons (for
example, to implement memoization). Now, even though there is aliased
state between $C$ and $I$, they cannot interfere with one one another
through it.

Therefore, a properly modular specification \emph{should not require}
$S$ to mention the use of the caching module. Otherwise we will be
forced to mention more and more extraneous details, as we layer
modules on top of one another, with the total size of the
specifications growing with the size of the reachable module graph.

The language I am proposing is able to properly describe this
situation, thanks to our ability to use the frame rule. Concretely,
suppose we have the following interface for our third module.

\begin{tabbing}
$\forall n:\N.\; \spec{cache}{\mathtt{fib}(n)}{a:\N}{cache \land a = fibonacci(n)}$ \\
\end{tabbing} 

Here, we can imagine \texttt{fib} produces fibonacci numbers using the
naive exponential-time recursive implementation, but adds caching to
``optimize'' it to linear time. The $cache$ predicate appears in the
pre- and post-conditions of $\mathtt{fib}$, so at first it would seem
that a $C$ and $S$ that used this function would have to mention that
they require the heap associated with $cache$.

However, note that $cache$ is the same in both the pre- and
post-condition. This means that we can prove $C$ and $I$ using a
specification 
$$F \equiv \spec{\emp}{\mathtt{fib}(n)}{a:\N}{\emp \land a =
  fibonacci(n)}$$
and then we can frame $cache$ on at the very end, after
completing the proofs of $C$ and $I$. Concretely, the proof tree 
looks like this:

\begin{mathpar}
\inferrule*[right=Cut]
           {\vdash F \otimes cache \\ 
            \inferrule*[right=Frame]
                       {\inferrule*{F \vdash S \\
                                  F, S \vdash C}
                                 {F \vdash C}}
                      {F \otimes cache \vdash C}
          }
          {\vdash C \otimes cache}
\end{mathpar}

Here, I write $R \otimes p$ to suggest that $p$ is framed onto the
triples in the specification $R$. Note that in the subderivation above
the use of the Frame rule, we do not need to carry around $cache$ in
our specification of $\mathtt{fib}$, which means that when we
introduce $S$, we will not need to mention $cache$ in its specification.

Simple caching schemes are very easy to hide -- all we need is to make
use of the frame rule~\cite{sep-inf}. A stronger test of our ability
to hide irrelevant detail will be if we can hide Liskov and Wing's
\emph{monotonic constraints}~\cite{liskov-wing}.

Monotonic invariants are program invariants that hold monotonically
throughout a program's execution. Suppose we have a partial order $(S,
\sqsubseteq)$ and a predicate $P$, such that if $s' \sqsubseteq s$
then $P(s') \implies P(s)$. Now, if we then ensure that any triples
that mention $P$ are of the form $\spec{P(s)}{c}{a:\tau}{\exists s'.\;
  P(s') \land s' \sqsubseteq s}$, then we know that no unknown uses of
this module can break our invariant, since they can only make $P$ ``more
true''. 

This property is used in several verification methodologies to prove
programs with potentially unknown calls~\cite{boogie-sub-obs}, but 
using it for information hiding purposes may be novel. 

Related to monotonic effects are \emph{commutative effects}~\cite{idioms}. 
These are side-effects in which the order they happen in does not affect
correctness. For example, a gensym operation is a commutative effect, since 
all we care about is the fact that two different invocations return distinct
symbols. 




\section{Interesting Higher-Order Imperative Programs}

Validating my thesis means that I have to demonstrate that it will
work on interesting imperative programs. So what do I mean with the
word ``interesting''? There are several senses in which I use it:

\begin{itemize}
\item One kind of interesting program are programs that pose
  particular difficulties for verification, such as ``Landin's knot'',
  which implements recursion by backpatching a function pointer. These
  kinds of procedures are typically small, and may not be found in
  actual software very often\footnote{Recently, I found a natural use
    of this technique in a real program, so I do not mean ``not very
    often'' as a polite way of saying ``never''.}, but help
  demonstrate the capabilities and limits of a verification
  methodology.

\item A second kind of interesting program are example programs that
  capture a particular stylized patterns of use that arise frequently
  in practice. Design patterns~\cite{gof} are a good example of this
  kind of program -- they represent what might be called engineering
  wisdom, and formalizing these patterns can help demonstrate that our
  verification methodology can accomodate typical good design, and
  moreover its formal version is not overly cumbersome to use. 

\item A third kind of interesting program are big programs. Even if a
  verification methodology works successfully on small examples, it
  remains to be seen whether it will work on larger examples, because
  in a larger example we will have to show that the different patterns
  of use can coexist peacefully, and that they compose gracefully. 

  Also, I only mean ``big'' relative to the size of typical examples
  (which are usually in the tens of lines of code). I will be happy
  with verifying a program that is on the order of high hundreds to
  low thousands of lines of code. This is, of course, very small by
  industrial standards, but it is large enough to learn whether or
  not the different pieces of this methodology fit together. 
\end{itemize}


\subsubsection{Design Patterns}

I will begin by analyzing individual design patterns, giving a
specification, an implementation, and a small client program for each
of the design patterns with a substantially stateful part.

The intuition here is that 1) object-oriented software is the biggest
repository of higher-order imperative code out there, and 2) the Gang
of Four design patterns are well-respected engineering wisdom about
how to manage such designs. So if I can show that I can formalize that
practice, I'll have taken a big step towards showing I can formally
capture the sorts of reasoning that software engineers use to manage
such programs.

Furthermore, there's also a \emph{negative} component here. I want to
figure out what patterns of reasoning are difficult to capture in my
framework, and systematically analyzing the design patterns should
help me search the design space.

One further thing that's worth noting is that a design pattern
\emph{isn't} a particular formula of specification logic. Any
specification we write represent particular balances between
generality (that is, more possible implementations) and specificity
(that is, more invariants for a client to use). For example, suppose
we have an iterator specification which supports deletion. For some
collection types (such as stretchable arrays), deleting an element
invalidates only those iterators that have are further along the array
than the one that did the deletion. If we put this fact into the
specification, then clients can use this fact to not invalidate some
iterators. But the price is that this specification rules out
collection implementations such as binary trees, where any deletion
will invalidate all other iterators.\footnote{This means that any spec
will be criticized by both those who think it is too liberal, and
those who think it is too restrictive. I think this is a feature,
not a bug -- making assumptions explicit in a way that permits
detailed argument is a big win!}


\begin{itemize}
\item[Iterator]
\item[Subject Observer]

These two I have written about extensively. I need to pull in the text
from those here.

\item[Flyweight]

The basic idea here is similar to memoization: cache the construction
of (immutable) objects, and on an object request only return a new one
if a satisfactory one is not in the cache. The cache, of course, is a
global mutable data structure that \emph{we don't want to see} in the
proof of any clients. This makes it a good use case for the frame
rule, and the issues that arise here will be a good test of whether we
can actually scale proofs.

\item[Memento]

This pattern is commonly used to support undo. It's basically an
opaque token representing an object's old state, and the object has a
method that the client can use (with the memento) to reset the state.
We can implement this with higher-order state -- the token can be a
command the user can invoke, and there are some nice sequencing issues
in when you can invoke the undo operation or not. We should also look
at multilevel undo, and compare this to an internal undo operation.

\item[Chain of Responsibility]

The Chain of Responsibility pattern is a pattern for composing
commands from other commands. Rather than building a monolithic
command, we break it up into processing units that take a message and
handle the parts of the update they can, delegating the rest to the
rest of the chain. This doesn't have to be a strict chain, so you can
have a ``tree of responsibility'' as well.

There's probably a very slick specification of this in terms of state
transformers, and I think there might be a nice ownership transfer
idiom lurking in here that's worth unearthing. Then again, maybe not
-- I've had trouble finding any really complex uses of aliasing yet.

\item[State]

This is a pattern in which an object is made from other objects which
provide abstract state -- it does not know the precise implementation
of its components. In fact, it's okay for the implementation of the
concrete state to \emph{change} as the program runs. Verifying this
example is primarily useful as a diagnostic test; of course we can
handle it, but even the slightest hint of ugliness in our development
indicates trouble in the logic.

\item[Composite]

At first, I thought that the Composite pattern didn't have any
interesting issues, but then I realized I was wrong, because we are
often composing \emph{stateful} objects. For example, consider an
iterator library (such as the \texttt{itertools} library in Python)
which constructs new iterators out of existing ones.

Concretely, suppose we have \texttt{PairIterator} that takes two
iterators and returns a new one that produces pairs of values,
iterating over the two arguments in lockstep. This seems
straightforward, until you think that this implies certain sharing
constraints -- in particular, the arguments have to be two
\emph{different} iterators.

So we need to be able to say something about sharing here. This may be
easiest if we use multiple specifications for the same function (for
example, if the two iterators come from the same underlying
collection, or not).

\item[Proxy] 

The Proxy pattern does not have very much complexity in its
implementation; it's essentially just a (possibly-imperative) wrapper
function -- its ML type could be \texttt{proxy : (a -> b) -> (a -> b)}.
However, analyzing it will still offer some interesting evidence for 
or against my thesis. That's because at heart a Proxy transforms 
the specification of its argument, and we can examine how hard it is
to model that. 

Note that state can play an important in implementing a real
Proxy, because it provides us with a ``back channel'' for
communication.  For example, a security Proxy that checks to see
whether the owner of a piece of data is authorized before invoking a
method might have a data structure it consults for that authorization
information.

The Decorator pattern is similar to this, only we can add methods
to our input in addition to transforming old ones. 

\item[Strategy]

In a language with first-class functions, there's not much to the
Strategy pattern -- we just need a pointer to a function. But the very
simplicity of this pattern means that we ought to be able to prove
some fun programs, like showing the correctness of a program that
updates its strategy as it runs.


\item[Facade]

The Facade pattern takes some existing modules and presents a nice
abstraction of them to the programmer. The way I conceptualize it, we
want to compose imperative modules in a way that turns their state
into a single new predicate. This is pretty straightforward in easy
cases, but I'm interesting in finding out if there's a way to make
this \emph{hard} I'd like to stress-test our ability to abstract over
state.

In particular, the test of whether this works will be seeing how big
the predicates grow. If I composing two modules with $n$-place
predicates describing their heap yields a Facade with a $2n$-place
predicate, it's pretty clear that information hiding isn't genuinely
happening. This contrasts with the Flyweight, where the test of
modularity is seeing how many starred subformulas show up in the
pre- and post-conditions. 

\end{itemize}

\subsection{Larger Programs}

Proving that I can prove the correctness of a wide variety of design
patterns is nice, but design patterns are a means, not an end. The
question remains whether these techniques can scale up to a larger
program. There are two main reasons that a larger program might prove
more difficult:

\begin{enumerate}
\item A \emph{formula} of moderate size is much smaller than a 
\emph{program} of moderate size; so we may see a larger gap between 
the structure of a program's invariants and the structure of its code
in larger examples. I suspect this is a result of the fact that an
interesting program will tend to use a sophisticated implementation to
realize its specification. For example, an optimization pass for a
compiler works because the code implements a fixed point iteration
justified from various facts in lattice theory, but its actual
implementation will typically be some kind of imperative worklist
algorithm that pushes elements on and off a queue.

\item Secondly, we might discover that we can specify the components
of a program (e.g., design patterns), but that composing them is
difficult, because of unforseen interactions between different
aspects. For example, suppose we have two modules which each depend on
a common (stateful) module. An important question is whether their
specifications \emph{necessarily} reveal that dependence in a way that
causes specification blow-up. Or alternately, suppose we have a data
structure that implements multiple design patterns. Under what
conditions can we avoid (or not avoid) interactions between those
different facets of an API?
\end{enumerate}

So to validate my thesis, I should look for a program that exhibits
both of these features -- we want a program with a high level
specification divorced from the structure, and which is systematically
constructed from smaller components. I am currently considering two
possibilities for such a program.

\subsubsection{A Graphical User Interface Library}

The first possibility is to implement and prove correct a graphical
user interface library. This is attractive for several reasons.

First, the structure of typical GUI code is quite awkward, in
\emph{precisely} a higher-order imperative way. A GUI program is
structured as a set of callback functions which are executed by an
event loop, which invokes each callback whenever an event of interest
to it happens. All communication between callbacks happens via the
heap. This is a bit like converting a program to continuation-passing
style and then representing the continuation as heap data the
programmer must manually modify. Such code is notoriously difficult to
write and debug, and many patterns such as the model-view-controller
pattern have been proposed to help tame the complexity.

The next question is what we might mean by proving the correctness of
a GUI. We can only prove a program correct with respect to a
specification, and so we need a specification of a GUI. Here, I
propose adapting a simplified version of the work of
Courtney~\cite{courtney-fruit}, on denotational semantics for GUI
libraries. He gave a semantics in terms of synchronous dataflow, where
each GUI component is a signal transformer that takes an input signal
of events and transforms it into an output signal of window layouts.

Below, I give partial ML-style datatypes describing events and
layouts, and then define two simple GUI widgets as signal
transformers, in a purely functional style.

\begin{tabbing}
$ST(A, B)\;\;$ \= $=(\mathrm{Time} \to A) \to (\mathrm{Time} \to B)$ \\[1em]

$GUI(A, B)$ \> $=ST(A \times Event, B \times Layout)$ \\[1em]

$Layout$ \> $=$\ \= \texttt{Label of String $|$ Button of Unit $|$} \\
         \> \> \texttt{Stack of Layout * Layout $|$ $\cdots$} \\[1em]

$Event$ \> $=$ \texttt{Click of Unit $|$ Key of Char $|$ $\cdots$}\\[1em]

$\mathtt{label}$ \= $ : GUI(String, Unit)$ \\
\> $ \equiv $ \= $\lambda $\=$input : \mathrm{Time} \to (String \times Event).\;$ \\
\> \> $\lambda t : \mathrm{Time}.\;$ \\
\> \> \> $\mathrm{let}\; (s, \_) = input(t)\; \mathrm{in}$ \\
\> \> \> $(\mathtt{Label}(s), ())$ \\[1em]

$\mathtt{button}$ \= $ : GUI(Unit, Bool)$ \\
\> $\equiv$ \= $\lambda$\=$ input: \mathrm{Time} \to (Unit \times Event).$\\
\> \> $\lambda t : \mathrm{Time}.$ \\
\> \> \> $\mathrm{let}\; (\_, e) = input(t) \;\mathrm{in}$\\
\> \> \> $\mathrm{case}\; e \; \mathrm{of}$ \\
\> \> \> $\bnfalt \mathtt{Click()} $\=$\To (true, \mathtt{Button}())$ \\
\> \> \> $\bnfalt \_$ \> $\To (false, \mathtt{Button}())$ \\[1em]

$\mathtt{stack}$ \= $ : GUI(A, B) \times GUI(X, Y) \to 
                        GUI(A \times X, B \times Y) $ \\
\> $\equiv$\= $\lambda$\=$(w_1, w_2) : GUI(A, B) \times GUI(X,Y).$\\
\> \> $\lambda input : \mathrm{Time} \to A \times X.$ \\
\> \> $\lambda t : \mathrm{Time}.$\\
\> \> \> $\mathrm{let} (b, l_1) = w_1\;input\;t \;\mathrm{in}$ \\
\> \> \> $\mathrm{let} (y, l_2) = w_2\;input\;t \;\mathrm{in}$ \\
\> \> \> $((b,y), \mathtt{Stack}(l_1, l_2))$ \\
\end{tabbing}

The \texttt{label} widget takes a \texttt{String} input signal and at
each time step produces a \texttt{Label} value carrying that string.
Likewise, the \texttt{button} widget produces a boolean output signal,
registering whether the button has been clicked on this time step or
not. The \texttt{stack} widget takes two widgets, and combines them into
a new, larger widget which multiplexes the inputs and outputs. 

We can also specify various operations to combine widgets and
compositionally construct a GUI. For example, two signals $ST(A,B)$
and $ST(B, C)$ can be composed into a result $ST(A, C)$, similarly to
function composition. More exotically, we can take a signal
transformer of type $ST(A \times X, B \times X)$ along with an element
of $X$, and construct a feedback loop of type $ST(A,B)$.

This functional-dataflow specification style is quite different from
an implementation, which would have a single, mutable layout that gets
updated by a family of callbacks on each trip through the event loop.
So this will let us test the claim that we can prove programs with a
reasonable gap between the specification and the implementation.

Additionally, we would want to respect the modularity constraints
within the implementation -- the callbacks and the event loop should
of course be oblivious to one anothers' concrete implementations, and
the callback code should be structured in proper MVC style. This will
let us test the idea that we can prove a program modularly, out of 
modularly-proven sub-components. 

\subsubsection{A Dataflow Optimization Infrastructure}

Another possible program that would be interesting to verify would be
a dataflow optimization framework, such as might be found inside a
compiler.

\begin{itemize}
\item Worklist algorithm over mutable data

The core algorithm in these frameworks is a worklist algorithm which
uses a transfer function to update some data until it reaches a fixed
point.

\item Plugin architecture

Adding a new optimization pass is a matter of giving the worklist
framework callback functions that will update some new data in a way
that implements an appropriate transfer function.


\item Lattice theory for specification

The justification for why this optimization works is generally an
appeal to the Knaster-Tarski theorem and an observation that the
dataflow lattice is finite. Combined analyses can be justified from
the various functorial operations on lattices (e.g., the product of
two lattices is a lattice). 

These proofs would need to be ``parametric'' in the exact lattice
structure, because of the possibility of new plugins.


This is interesting because there are ways of combining analyses that
yield more power than either one run separately, and we'd want to
show how that fits together too. 

\end{itemize}

\subsection{``Feasible''}



\section{Related Work}

\subsection{Separation Logic}

General stuff here. 

Mention Hongseok and Lars's work here. 

Who invented the admissible closure trick? Nick? Lars? Hongseok?

\subsection{Specification Logic}

Talk about specification logic for Algol.
\subsection{Separation Logic for Java}

Matthew Parkinson's work, plus abstract predicates.

\subsection{Hoare Type Theory}

\subsection{JML}

\subsection{Spec\# and Boogie}

\subsection{Linear Types and Typestates}

Obviously, talk about Kevin's work, Bob's stuff on linearity, Boyland's
fractional permisssions.

\subsection{Regions? Step-indexed models? Syntactic control of interference?}

This is starting to get further afield.



\section{Contributions}

To demonstrate my thesis, I make a number of concrete contributions.
In Chapter 2, I define a higher-order imperative programming language
based on a predicate variant of $F_\omega$~\citep{fomega}, augmented
with reference types which uses a monadic language to encapsulate its
side-effects (including both modification of the heap, and
nontermination). I give this language a domain-theoretic denotational
semantics based on the techniques of \citet{smyth-plotkin}, which lets
me validate strong equational reasoning principles --- including both
the $\beta-$ and $\eta$-rules of the lambda calculus ---

Equational reasoning is less helpful for imperative programs, and to
support reasoning about this part of the programming language, I
define and prove the soundness of a program logic in Chapter 3. The
program logic combines ideas from specification logic and higher-order
separation logic to give an expressive program logic 
